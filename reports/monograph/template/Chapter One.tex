%%%%%%%%%%%%%%%%%%%%%%%%%%%%%%%%%%%%%%%%%%%%%%%%%%
\documentclass[b5paper,11pt, titlepage]{book}
%%%%%%%%%%%%%%%%%%%%%%%%%%%%%%%%%%%%%%%%%%%%%%%%%%
\usepackage[pdftex]{graphicx,color}
%\usepackage[T1,plmath]{polski}
\usepackage[cp1250]{inputenc}
\usepackage{indentfirst}
\usepackage[numbers,sort&compress]{natbib} % sort and compress citations
%\usepackage[none]{hyphenat} % brak podzia³u wyrazów
\usepackage{geometry}
\newgeometry{tmargin=3.6cm, bmargin=3.6cm, lmargin=3.2cm, rmargin=3.2cm}
\usepackage{multirow}
\graphicspath{ {E:/aidd_new/aidd/reports/monograph/template/Graphics/} }

\renewcommand{\figurename}{Fig.}
\renewcommand{\tablename}{Tab.}

%%%%%%%%%%%%%%%%%%%%%%%%%%%%%%%%%%%%%%%%%%%%%%%%%%
\begin{document}
%%%%%%%%%%%%%%%%%%%%%%%%%%%%%%%%%%%%%%%%%%%%%%%%%%
%\title{\textbf{Modelowanie struktur anizotropowych \\Wstêpne badania eksperymentalne\\
%	\vspace{10cm}}
%\normalsize{Abstract} }
%\normalsize{W ramach projektu pt.: \\ \textit{Wp³yw jednoczesnego oddzia³ywania temperatury i wilgotnoœci \\na struktury %anizotropowe: od teorii do badañ doœwiadczalnych\\}
%NCN OPUS 12}}
	
%{Micha³ Jurek}

%\date{ }
	
%\date{Gdañsk, Maj 2018\\ (Nr Arch. 221/2018)}


	
%\maketitle
%\newpage
%\tableofcontents
%\newpage
%\listoffigures
%\listoftables
%\newpage

\chapter{Literature Review}

\textbf{Abdalraheem Abdullah Yousef Ijjeh}

\tableofcontents
\newpage
%%%%%%%%%%%%%%%%%%%%%%%%%%%%%%%%%%%%%%%%%%%%%%%%%%
\section{Structural Health Monitoring and motivations}
%%%%%%%%%%%%%%%%%%%%%%%%%%%%%%%%%%%%%%%%%%%%%%%%%%
 Structural health monitoring (SHM) intends to describe a real-time evaluation 
 of the component materials, of the different parts, and the full construction 
 of these parts as a whole during the structure life-cycle~\cite{Balageas2010}. 
 Furthermore, SHM supports detecting and characterizing damages in structures 
 as a whole or in their parts which may impair the ability to fully and safely 
 deliver the demanded duty~\cite{Yuan2016}.
 \paragraph{}
 The purpose of SHM is to distinguish any potential change that occurs at 
 structure that could decay the performance of the whole system, at the 
 earliest possible time so that an action can be taken to reduce the downtime, 
 operational costs and maintenance costs, consequently reducing the risk of 
 catastrophic failure, injury, or even loss of life.
 Moreover, SHM improves the work organization of maintenance services within 
 trying to replace scheduled and periodic maintenance inspection with 
 performance-based maintenance (long term) or at least (short term) through 
 decreasing maintenance labour, in particular by avoiding dismounting parts 
 where there is no hidden defect, and through reducing the individual 
 involvement~\cite{Balageas2010}.
 \paragraph{}
We can look at SHM as an improved method to perform Non-Destructive Evaluation, 
Nonetheless, SHM involves sensors that are integrated into structures, data 
transmission, computational power, and processing ability within 
structures~\cite{Balageas2010}. The typical organization of a SHM system is 
depicted in Fig. \ref{fig:NationalSHM}, in which a typical SHM system is built 
from a diagnostic part (low level), which holds the levels responsible of 
detection, localization, and evaluation of any damage, and a prognosis part 
(high level), which includes the production of information concerning the 
outcomes from the diagnosed damage.
\begin{figure} [h!]
	\begin{center}
		\includegraphics[width=\textwidth]{figure1_1.jpg}
	\end{center}
	\caption{Figure National SHM system~\cite{Yuan2016}} 
	\label{fig:NationalSHM}
\end{figure}
\paragraph{}
Generally,  we can categorize SHM strategies into two main schemes, local and 
global schemes. Local schemes were discussed in 
~\cite{Grimberg2001,Maldague1993,Raghavan2007}
and global schemes were discussed in~\cite{Adams2002,Doebling1998,Uhl2004}. 
Local schemes monitor a small area of the structure enclosing the transducers 
that used for registering the data signals after the structure being exited. 
For this purpose, few phenomena are used like Ultrasonic 
waves~\cite{Raghavan2007}, eddy currents~\cite{Grimberg2001}, thermal 
field~\cite{Maldague1993} and acoustic emission~\cite{Pao1978}. On the other 
hand, Global schemes in SHM are performed if a global motion of the structure 
is induced during its operation~\cite{Balageas2010}. For this purpose, 
vibration techniques are utilized, in which two types of techniques are 
introduced, the signal based~\cite{Ihn2004} and the 
model-based~\cite{natke2000model}.
The signal-based approaches analyze measured responses of the structure after 
ambient excitation and possible damages~\cite{Stepinski2013}. 
The model-based approaches use various types of models of a monitored structure 
to detect and localize damage in the structure by utilizing relations 
between the model parameters and distinct damages~\cite{Stepinski2013}. 

\section{Structural Health Monitoring for Composite Materials}
A composite material can be described as a compound of two or more different 
materials to achieve new features that can�t be achieved by those of specific 
components functioning separately.
Distinct from metallic alloys, each material has its 
characteristics~\cite{Campbell2010}.
Composite materials have a reinforcing and 
matrix phases that are usually categorized into~\cite{Jones1999}:

\begin{itemize}
	\item Fibre-reinforced composite materials that consist of three parts: the 
	fibres as the dispersed phase, the matrix as the continuous phase, and the 
	fine inter-phase region, also known as the interface~\cite{Cantwell1991}.
	\item Laminated composite materials that are an assembly of layers of a  
	fibre-reinforced composite material that can be combined to implement 
	necessary design features~\cite{Ramirez1999}.
	\item Particulate composite materials that are characterized as being composed of particles suspended in a matrix.

\end{itemize}

When comparing composite materials to regular metallic materials, we can notice 
that  composites have more advantages over metallic, those advantages can be 
summarized into~\cite{Campbell2010}:

\begin{itemize}
	\item Low density with high strength and stiffness. 
	\item Greater vibration damping capacity, and more temperature resistant.
	\item Strong texture in micro-structures that makes it easy to design and 
	satisfy different application needs. 
\end{itemize}

Composite materials are not immune to some difficulties.
Due to the nature of multiphase materials, composites materials present 
different anisotropic characteristics. 
Their material capacities, mainly associating with manufacturing processes, are 
dispersive~\cite{Awad2012}. 
Furthermore, composites materials are sensitive to impact damages resulting 
from the lack of reinforcement in the out-of-plane direction~\cite{Cai2012}. 
Under a high energy impact, little penetration rises in composites materials. 
On the other hand, for low to medium energy impact, matrix crack will happen 
and interact, causing the delamination process~\cite{Cai2012}. 
Fibre breakage would also occur at the opposite side to the 
impact~\cite{Montalvao2006}, furthermore, damages can be produced in composites 
by mistaken procedures through production and assembling, aging or service 
condition~\cite{Cai2012}. 
Generally, composites materials damages can happen due to fibre breakage, 
matrix cracking, fibre-matrix debonding and delamination among layers, most of 
which happen below the top surfaces and are hardly visible~\cite{Cai2012}. 
These damages can seriously decrease the composites performance, therefore, 
they should be detected in time to avoid catastrophic structural collapses.  
Damage can only be discovered by analyzing the responses of the structure, 
obtained by sensors, before and after it happens,
hence, we cannot expect to have �damage sensors�, the only way to detect the 
damage is by processing and comparing and differentiating the signals received 
from the sensors before and after damage~\cite{s18041094},  then attempting to 
classify the parameters, that are sensitive to minor damage and that can be 
distinguished from the response to natural and environmental 
disturbances~\cite{s18041094}. 
Consequently, SHM  methods are very essential in damage detection, since SHM 
implies different types of sensors mixed with damage detection techniques. 

%%%%%%%%%%%%%%%%%%%%%%%%%%%%%%%%%%%%%%%%%%%%%%%%%%
\section{Guided waves based Structural Health Monitoring}

The approach behind adopting elastic waves propagation methods in Structural 
health monitoring includes generating elastic waves in the examined structure 
and recording their amplitude as a function of time~\cite{Ostachowicz2012}. The 
produced waves are travelling in packets, those packets keep propagating until 
they reflect from discontinuities, edges or damage in the structure, the 
reflected waves hold information about the location and the size of the damage. 
\paragraph{}
Designing a strong and robust SHM system demands the existence of qualified 
personalities in various scientific fields\cite{Willberg2015a}, e.g. 
mechanical and electrical engineering, as well as in computer science, 
mathematics, and physics~\cite{Willberg2013}.
Moreover, it requires a deep understanding of various material types, and the 
design of transducers and how it works in networks, also there is a need to be 
familiar with signal processing methods and damage evaluation 
techniques~\cite{Willberg2013}.
\paragraph{}
There have been various SHM techniques introduced and
performed in recent years for different types of structures. Vibration-based 
procedures which was mentioned earlier are one such regularly investigated in 
the field with various extensive 
articles and books~\cite{Doebling1998,Deraemaeker2010,Beskhyroun2012}.

In this literature will we focus on other SHM techniques that are the guided 
wave-based SHM techniques in composite materials, which has brought large 
attention in the past two decades~\cite{Mitra2016}.
Guided waves which are essentially elastic waves propagating within bounded 
structures~\cite{Mitra2016}, and due to the fact that there is no such 
infinite plane, accordingly, they are being guided by the structure's 
boundaries. 
\paragraph{}
There a few benefits from adopting guided wave-based schemes for SHM in
structures over Vibration based methods, the price of transducers are generally 
cheap and affordable, also 
usually, due to the lightweight of those transducers, it can be implemented 
easily in the structure, in addition, it is possible to scan a relatively large 
area compared to a little number of transducers~\cite{Mitra2016}. 
Moreover, an important advantage for guided waves over a vibration-based scheme 
is their high sensitivity for detecting small damages due to the ability to 
high-frequency excitation, and in case of low-frequency ambient vibration, it 
does not affect the guided waves propagation~\cite{Mitra2016,Croxford2007}.

\paragraph{} 
Various types of guided waves have been investigated for the purpose of SHM. 
A well-known approach is the use of Lamb waves, that propagate 
within thin-plates and shells bounded by stress-free surfaces~\cite{Mitra2016}.
Lamb waves were given their name after Horace Lamb, who discovered them and 
developed a theory to describe the phenomena of their propagation 
~\cite{Ostachowicz2012}. 
However, Lamb could not able to generate those waves physically, until 
Worlton~\cite{Worlton1961} who saw the opportunity to utilize lamb waves 
characteristics in damage detection~\cite{Ostachowicz2012}.
Lamb waves, in general, are generated and received by piezoelectric (PZT) 
transducers~\cite{Cai2012}.
Due to the multi-mode and dispersion properties, the propagation of Lamb waves 
is quite complex~\cite{Ostachowicz2012}. 
In practical applications, two forms of lamb waves arise depending on the 
distribution of the displacement on the top and bottom bounding surfaces, these 
forms are symmetric, denoted as \(S0,S1,S2,...., \)and asymmetric, denoted as 
\(A0,A1,A2,....,\) ~\cite{Ostachowicz2012}. Fig \ref{fig:LambModes} presents 
Lamb wave for \(S0\) and \(A0\) modes~\cite{Willberg2015}.


\begin{figure} [h!]
	\begin{center}
	 	\centering
		\includegraphics[width=\textwidth]{Figure1_2_Lamb_wave.jpg}
	\end{center}
	\caption{Lamb wave mode shapes ~\cite{Yuan2016}} 
	\label{fig:LambModes}
\end{figure} 
\paragraph{}

Regardless of Lamb waves promising characteristics, using them for SHM 
applications hold some essential challenges. 
The dispersive nature of Lamb waves propagating modes in which can convert into 
each other in the presence of damages and other changes in the mechanical 
impedance~\cite{Willberg2015}. 
Moreover, ascribed to some flaws in the bonding within actuators sensors and 
the structure, random noise will emerge in the relevant sensors due to the high 
sensitivity of Lamb waves toward structural perturbations. 
Also, noise arising from environmental sources, like temperature changing, or 
anisotropy of the material also summed up to the received signals making them 
very complicated and challenging to recognize and interpret~\cite{Willberg2015}.
Moreover, an essential point concerns the choice of a carrier frequency for the 
Lamb waves because the higher the frequency is, the damage detection of small 
size is more likely detected. however,  when the frequency increases, the 
number of propagating wave modes will increase accordingly, as a result of 
occurring multiple wave modes and because of the different velocities of each 
wave mode which causes a problem with reflection identification and 
misinterpretation of the damage location and shape size~\cite{Ostachowicz2012}. 
It was found that each wave mode shows a varying sensitivity to individual 
damage. Authors in~\cite{Kessler2002,Ihn2008,Ihn2004} found that \(A0\) mode is 
suitable for delaminations to be detected in composite materials, and for 
\(S0\) mode it was found suitable for cracks detection in metal 
elements~\cite{Ihn2004,Ihn2008}.It was also observed that the design of the 
transducer effects in a great manner the produced and registered wave 
mode~\cite{Ostachowicz2010}.

\section{Damage Detection and Localization by Guided Waves for SHM}
A damage can be defined as changes occurred in a system, either 
deliberately or accidentally, that adversely alter the current or future 
performance of the system~\cite{Farrar2012}. For damage detection and localization purpose traditionally, in SHM systems, a combination of piezoelectric transducers (PZTs) are utilised for exciting the structure and sensing the reflected signals. Based on the arrangement of PZTs, two main approaches are available: Pulse echo and Pitch-catch as presented in Fig. \ref{fig:Pulse_echo_Pitch_catch}.
In pulse-echo, the same PZT transducer is used to generate and receive the Lamb waves, while in the pitch-catch approach, two PZT transducers are used, the first PZT generates Lamb waves and the other PZT receives it.

\begin{figure} [h!]
	\begin{center}
	\centering
	\includegraphics[width=\textwidth]{Figure1_3_Pulse_ech0_Pitch_catch.png}
	\end{center}
	\caption{(a) Pulse echo	(b) Pitch catch} 
	\label{fig:Pulse_echo_Pitch_catch}
\end{figure}

PZT transducers configurations for damage detection and localization for SHM generally are classified into two main arrangements which are concentrated and distributed arrangement. Hence, a lot of work was performed in the literature utilizing PZT configurations for generating and sensing  Lamb waves.
\paragraph{}
The following research articles are examples in which they used the concentrated transducers arrangement within their methods.
The author in~\cite{Giurgiutiu2006}implanted  PZT wafer active sensor (PWAS) in phased array to investigate Lamb waves within plates, in which the results which he obtained were encouraging regarding the location of the damage and it is size.
Additionally, the author in~\cite{Wilcox2003}, investigated Omni-directional wave transducer arrays for the rapid inspection of large areas of plate structures. In this work, two arrangements of PZTs were examined. The first one consists of a densely circular area with PZTs in which it presented an excellent concentrated peak at the location of the reflector, though it requires plenty of transducers. The other arrangement consists of a single circular ring of PZTs which hardly efficient in any circumstance that involves various reflectors.
Moreover, in~\cite{Malinowski2009} authors performed a numerical analysis on an array of PZTs of a star shape for various damage scenarios. Their method confirmed a good damage localization.
\paragraph{}
Furthermore, the distributed arrangement was implemented in many research articles. In this arrangement, the examined area is spread entirely with PZT transducers. Authors in~\cite{Schubert2008}, introduced and tested Different types of the before-mentioned arrangements. 
Moreover,  authors in~\cite{Qiang2009} used a rectangular network of transducers
on a composite material, while a triangular network of transducers was examined in~\cite{Wandowski2009} for an isotropic specimen.
\paragraph{}
It can be concluded from previous works that using these approaches for damage detection and localization is only suitable for simple structures. 
Furthermore, the estimation of damage size is more challenging to be extracted by the registered signals at PZTs despite their locations. 
These challenges arise due to various limitations like the added mass and attached cables to the structure alter the propagating waves. 
Additionally, it is difficult to distinguish the registered signals among different objects e.g. bolts and rivets, the edges, and the actual damage. Another challenge is induced by the temperature which affects the propagating waves, accordingly, it becomes important to compensate for this issue\cite{Marzani1999}.
Consequently, to exceed these limitations, a full wavefield measurement approach was introduced. 
As a result of utilising a full wavefield, a damage influence map is produced, which makes it possible to estimate the size of the damage\cite{Ostachowicz2014}.
\paragraph{}
Scanning Laser Dopper Vibrometry (SLDV) was developed and presented in the experimental research in the earlies of 1980s. 
SLDV employs Doppler frequency shift principle to measure the velocity of a moving object in which the amount of the shifted frequency depends on the velocity of the moving object\cite{Stanbridge1999}. 
SLDV was a solution of applying a dense array of PZTs to localize damage in a structure, that produces images with a low resolution, besides it is impossible in some scenarios to apply such array of PZTs. 
Additionally, SLDV was employed for full wavefield measurements instead of array of PZTs. 
An SLDV links computer-controlled XY scanning mirror with a camera inside the optical head, which densely scans the vibrating surface of the structure and get a large number of high-resolution measurements\cite{Helfrick2011}. 
Consequently, the structure's vibration can be measured accurately and the propagation of guided waves also can be registered accurately\cite{Ostachowicz2014}.
\paragraph{}
However, in many situations, it is needed to examine an object in three dimensions. 
In such situations, a 3D vibrometer is used which holds three 1D scanning vibrometers in addition to the data acquisition system and a control system.
A 3D vibrometer measures a location with three independent beams that hit the target from three different directions, which yields a measurement of the complete in-plane and out-of-plane velocity of the target.
\paragraph{}
SLDV has been broadly used for sensing of Lamb wave. 
There are several works in the literature are concentrated on imaging method for damage detection from the grid point data sensed and recorded by SLDV.
For instance, authors in~\cite{Yu2013} applied a frequency wavenumber domain analysis utilising a 2D Fourier transform to detect a crack an aluminium plate. Wavenumber frequency filtering of SLDV sensed Lamb wave responses were applied to image damage in~\cite{Ruzzene2007}. 
Authors in~\cite{Kudela2015} introduced a new method of imaging crack growth in a structure.
In which they employed a full wavefield data captured by SLDV.
Also, authors in~\cite{Harb2015} utilized SLDV based measurement for inferring  the dispersion curves for \(A0\) Lamb wave mode. 
Moreover, SLDV has been used to scan and capture Lamb waves in honeycomb core sandwich structure to detect damage influence in~\cite{Lamboul2013}.
\paragraph{}
Despite all the advantages of utilizing a SLDV, there are some disadvantages. 
The first drawback concerns the specimen's surface which must be smooth and distinguished by a proper reflectivity, otherwise, the captured signal to noise ratio will be decreased\cite{Ostachowicz2014}. 
Furthermore, experimenting using  SLDV requires much time since the SLDV performs measurements at a single point in space at a time due to registering a full wavefield of lamb waves the process of measurements must be repeated by keeping the same excitation and pause until the wave completely attenuates\cite{Ostachowicz2014}.

\section{Introduction to Our work }
\subsection{Project motivations and objectives }
\subsection{tools used for measurement and dataset generation }
\subsection{Introduction to AI and Deep learning }

%%%%%%%%%%%%%%%%%%%%%%%%%%%%%%%%%%%%%%%%%%%%%%%%%%

%%%%%%%%%%%%%%%%%%%%%%%%%%%%%%%%%%%%%%%%%%%%%%%%%%

%%%%%%%%%%%%%%%%%%%%%%%%%%%%%%%%%%%%%%%%%%%%%%%%%%
\begin{figure} [h!]
	\begin{center}
		%\includegraphics[width=14cm]{Graphics/bc.jpg}
	\end{center}
	\caption{Figure caption.} 
	\label{fig:bc}
\end{figure}

%%%%%%%%%%%%%%%%%%%%%%%%%%%%%%%%%%%%%%%%%%%%%%%%%%
\begin{table}[h]
	\centering
	\caption{Table caption}
	\begin{tabular}{cccc}
		\hline
		\textbf{a}	& \textbf{x} & \textbf{y} & \textbf{z} \\
		\hline
		-50 & -0.289 & -0.289 & -0.598\\ 
		-40 & -0.248 & -0.248 & -0.512\\ 
		\hline 
	\end{tabular} 
	\label{tab:xyz}
\end{table}
%%%%%%%%%%%%%%%%%%%%%%%%%%%%%%%%%%%%%%%%%%%%%%%%%%

The scheme of experimental setup is shown in Fig.~\ref{fig:bc}.  
The values are collected in Tab.~\ref{tab:xyz}.


%The details are described in a book~\cite{udd2011fibre}. 

%Similar case was analyzed by Hill et al.~\cite{hill1978photosensitivity}


\bibliography{E:/aidd_new/aidd/reports/monograph/template/refrences/ref} % 
%bibliography data in report.bib
\bibliographystyle{abbrv}
 % makes bibtex use spiebib.bst



%%%%%%%%%%%%%%%%%%%%%%%%%%%%%%%%%%%%%%%%%%%%%%%%%%
\end{document}
%%%%%%%%%%%%%%%%%%%%%%%%%%%%%%%%%%%%%%%%%%%%%%%%%%