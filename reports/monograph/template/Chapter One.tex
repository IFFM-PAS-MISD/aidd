%%%%%%%%%%%%%%%%%%%%%%%%%%%%%%%%%%%%%%%%%%%%%%%%%%
\documentclass[b5paper,11pt, titlepage]{book}
%%%%%%%%%%%%%%%%%%%%%%%%%%%%%%%%%%%%%%%%%%%%%%%%%%
\usepackage[pdftex]{graphicx,color}
%\usepackage[T1,plmath]{polski}
\usepackage[cp1250]{inputenc}
\usepackage{indentfirst}
\usepackage[numbers,sort&compress]{natbib} % sort and compress citations
%\usepackage[none]{hyphenat} % brak podzia³u wyrazów
\usepackage{geometry}
\newgeometry{tmargin=3.6cm, bmargin=3.6cm, lmargin=3.2cm, rmargin=3.2cm}
\usepackage{multirow}
\graphicspath{ {E:/aidd_new/aidd/reports/monograph/template/Graphics/} }

\renewcommand{\figurename}{Fig.}
\renewcommand{\tablename}{Tab.}

%%%%%%%%%%%%%%%%%%%%%%%%%%%%%%%%%%%%%%%%%%%%%%%%%%
\begin{document}
%%%%%%%%%%%%%%%%%%%%%%%%%%%%%%%%%%%%%%%%%%%%%%%%%%
%\title{\textbf{Modelowanie struktur anizotropowych \\Wstêpne badania eksperymentalne\\
%	\vspace{10cm}}
%\normalsize{Abstract} }
%\normalsize{W ramach projektu pt.: \\ \textit{Wp³yw jednoczesnego oddzia³ywania temperatury i wilgotnoœci \\na struktury %anizotropowe: od teorii do badañ doœwiadczalnych\\}
%NCN OPUS 12}}
	
%{Micha³ Jurek}

%\date{ }
	
%\date{Gdañsk, Maj 2018\\ (Nr Arch. 221/2018)}


	
%\maketitle
%\newpage
%\tableofcontents
%\newpage
%\listoffigures
%\listoftables
%\newpage

\chapter{Literature Review}

\textbf{Abdalraheem Abullah Yousef Ijjeh}

\tableofcontents
\newpage
%%%%%%%%%%%%%%%%%%%%%%%%%%%%%%%%%%%%%%%%%%%%%%%%%%
\section{Structural Health Monitoring and motivations}
%%%%%%%%%%%%%%%%%%%%%%%%%%%%%%%%%%%%%%%%%%%%%%%%%%
 Structural health monitoring (SHM) intends to describe a real-time evaluation of the component materials, of the different parts, and the full construction of these parts as a whole during the structure life-cycle \cite{Balageas2010}. Furthermore, SHM supports detecting and characterizing damages in structures as a whole or in their parts which may impair the ability to fully and safely deliver the demanded duty \cite{Yuan2016}.
 \paragraph{}
 The purpose of SHM is to distinguish any potential change that occurs at 
 structure that could decay the performance of the whole system, at the 
 earliest possible time so that an action can be taken to reduce the downtime, 
 operational costs, and maintenance costs, consequently, reducing the risk of 
 catastrophic failure, injury, or even loss of life.
 Moreover, SHM improves the work organization of maintenance services within 
 trying to replace scheduled and periodic maintenance inspection with 
 performance-based maintenance (long term) or at least (short term) through 
 decreasing maintenance labor, in particular by avoiding dismounting parts 
 where there is no hidden defect, and through reducing the individual 
 involvement\cite{Balageas2010}.
 \paragraph{}
We can look at SHM as an improved method to perform Non-Destructive Evaluation, 
Nonetheless, SHM involves sensors that are integrated into structures, data 
transmission, computational power, and processing ability within 
structures\cite{Balageas2010}. The typical organization of a SHM system is 
depicted in Fig \ref{fig:NationalSHM}, in which a typical SHM system is built 
from a diagnostic part (low level), which holds the levels responsible of 
detection, localization, and evaluation of any damage, and a prognosis part 
(high level), which includes the production of information concerning the 
outcomes from the diagnosed damage.
\begin{figure} [h!]
	\begin{center}
		\includegraphics[width=\textwidth]{figure1_1.jpg}
	\end{center}
	\caption{Figure National SHM system \cite{Yuan2016}} 
	\label{fig:NationalSHM}
\end{figure}


\section{Structural Health Monitoring for Composite Materials}
A composite material can be described as a compound of two or more different 
materials to achieve new features that can�t be achieved by those of specific 
components functioning separately.
Distinct from metallic alloys, each material has its 
characteristics\cite{Campbell2010}.
Composite materials have a reinforcing and 
matrix phases that are usually categorized into\cite{Jones1999}:

\begin{itemize}
	\item Fiber-reinforced composite materials that consist of three parts: the fibers as the dispersed phase, the matrix as the continuous phase, and the fine inter-phase region, also known as the interface\cite{Cantwell1991}.
	\item Laminated composite materials that are an assembly of layers of fiber-reinforced composite material that can be combined to implement necessary design features\cite{Ramirez1999}.
	\item Particulate composite materials that are characterized as being composed of particles suspended in a matrix.

\end{itemize}

When comparing composite materials to regular metallic materials, we can notice that  composites have more advantages over metallic, those advantages can be summarized into\cite{Campbell2010}:

\begin{itemize}
	\item Low density with high strength and stiffness. 
	\item Greater vibration damping capacity, and more temperature resistant.
	\item Strong texture in micro-structures that makes it easy to design and 
	satisfy different application needs. 
\end{itemize}

Composite materials are not immune to some difficulties.
Due to the nature of multiphase materials, composites materials present 
different anisotropic characteristics. 
Their material capacities, mainly associating with manufacturing processes, are 
dispersive\cite{Awad2012}. 
Furthermore, composites materials are sensitive to impact damages resulting 
from the lack of reinforcement in the out-of-plane direction\cite{Cai2012}. 
Under a high energy impact, little penetration rises in composites materials. 
On the other hand, for low to medium energy impact, matrix crack will happen 
and interact, causing the delamination process\cite{Cai2012}. 
Fiber breakage would also occur at the opposite side to the 
impact\cite{Montalvao2006}, furthermore, damages can be produced in composites 
by mistaken procedures through production and assembling, aging or service 
condition\cite{Cai2012}. 
Generally, composites materials damages can happen due to fiber breakage, 
matrix cracking, fiber-matrix debonding and delamination among layers, most of 
which happen below the top surfaces and are hardly visible\cite{Cai2012}. 
These damages can seriously decrease the composites performance, therefore, 
they should be detected in time to avoid catastrophic structural collapses.  
Damage can only be discovered by analyzing the responses of the structure, 
obtained by sensors, before and after it happens,
hence, we cannot expect to have �damage sensors�, the only way to detect the 
damage is by processing and comparing and differentiating the signals received 
from the sensors before and after damage\cite{s18041094},  then attempting to 
classify the parameters, that are sensitive to minor damage and that can be 
distinguished from the response to natural and environmental 
disturbances\cite{s18041094}. 
Consequently, SHM  methods are very essential in damage detection, since SHM 
implies different types of sensors mixed with damage detection techniques. 

%%%%%%%%%%%%%%%%%%%%%%%%%%%%%%%%%%%%%%%%%%%%%%%%%%
\section{Guided waves based Structural Health Monitoring}


Designing a strong and robust SHM system demands the existence of qualified 
personalities in various scientific fields\cite{Willberg2015a}, e.g. 
mechanical and electrical engineering, as well as in computer science, 
mathematics, and physics\cite{Willberg2013}.
Moreover, it requires a deep understanding of various material types, and the 
design of transducers and how it works in networks, also there is a need to be 
familiar with signal processing methods and damage evaluation 
techniques\cite{Willberg2013}.
\paragraph{}
There have been various SHM techniques introduced and
performed in recent years for different types of structures. Vibration-based 
procedures are one such regularly investigated in the field with various 
extensive 
articles and books\cite{Doebling1998,Deraemaeker2010,Beskhyroun2012}.
In this literature will we focus on other SHM techniques that are the guided 
wave-based SHM techniques in composite materials, which has brought large 
attention in the past two decades\cite{Mitra2016}.
Guided waves which are essentially elastic waves propagating within bounded 
structures\cite{Mitra2016}, and due to the fact that there is no such 
infinite plane, accordingly, they are being guided by the structure's 
boundaries. 
\paragraph{}
There a few benefits from adopting guided wave-based schemes for SHM in
structures over Vibration based methods, the price of transducers are generally 
cheap and affordable, also 
usually, due to the lightweight of those transducers, it can be implemented 
easily in the structure, in addition, it is possible to scan a relatively large 
area compared to a little number of transducers\cite{Mitra2016}. 
Moreover, an important advantage for guided waves over a vibration-based scheme 
is their high sensitivity for detecting small damages due to the ability to 
high-frequency excitation, and in case of low-frequency ambient vibration, it 
does not affect the guided waves propagation\cite{Mitra2016,Croxford2007}.

\paragraph{} 
Various types of guided waves have been investigated for the purpose of SHM. 
A well-known approach is the use of Lamb waves, that propagate 
within thin-plates and shells bounded by stress-free surfaces\cite{Mitra2016}.
Lamb waves were given their name after Horace Lamb, who discovered them and 
developed a theory to describe the phenomena of their propagation 
\cite{Ostachowicz2012}. 
However, Lamb could not able to generate those waves physically, until 
Worlton\cite{Worlton1961} who saw the opportunity to utilize lamb waves 
characteristics in damage detection\cite{Ostachowicz2012}.
Lamb waves, in general, are generated and received by piezoelectric (PZT) 
transducers\cite{Cai2012}.
Due to the multi-mode and dispersion properties, the propagation of Lamb waves 
is quite complex\cite{Ostachowicz2012}. 
In practical applications, two forms of lamb waves arise depending on the 
distribution of the displacement on the top and bottom bounding surfaces, these 
forms are symmetric, denoted as \(S0,S1,S2,...., \)and asymmetric, denoted as 
\(A0,A1,A2,....,\) \cite{Ostachowicz2012}. Fig \ref{fig:LambModes} presents 
Lamb 
wave for \(S0\) 
and \(A0\) modes\cite{Willberg2015}.


\begin{figure} [h!]
	\begin{center}
	 	\centering
		\includegraphics[width=\textwidth]{Figure1_2_Lamb_wave.jpg}
	\end{center}
	\caption{Lamb wave mode shapes \cite{Yuan2016}} 
	\label{fig:LambModes}
\end{figure} 

\paragraph{}
Regardless of Lamb waves promising characteristics, using them for SHM 
applications holds some essential challenges. 
The dispersive nature of Lamb waves propagating modes in which can convert into 
each other in the presence of damages and other changes in the mechanical 
impedance\cite{Willberg2015}. 
Moreover, ascribed to some flaws in the bonding within actuators sensors and 
the structure, random noise will emerge in the relevant sensors due to the high 
sensitivity of Lamb waves toward structural perturbations. 
Also, noise arising from environmental sources, like temperature changing, or 
anisotropy of the material also summed up to the received signals making them 
very complicated and difficult to recognize and interpret\cite{Willberg2015}.



%%%%%%%%%%%%%%%%%%%%%%%%%%%%%%%%%%%%%%%%%%%%%%%%%%
\section{Damage Detection and Localization using guided waves for SHM  }




\subsection{Implemented methods using sensor network}
\subsection{Approach using Hybrid PZT - Laser Vibrometer }

\section{Introduction to Our work }
\subsection{Project motivations and objectives }

\subsection{tools used for measurement and dataset generation }
\subsection{Introduction to AI and Deep learning }

%%%%%%%%%%%%%%%%%%%%%%%%%%%%%%%%%%%%%%%%%%%%%%%%%%

%%%%%%%%%%%%%%%%%%%%%%%%%%%%%%%%%%%%%%%%%%%%%%%%%%
\paragraph{}


%%%%%%%%%%%%%%%%%%%%%%%%%%%%%%%%%%%%%%%%%%%%%%%%%%
\begin{figure} [h!]
	\begin{center}
		%\includegraphics[width=14cm]{Graphics/bc.jpg}
	\end{center}
	\caption{Figure caption.} 
	\label{fig:bc}
\end{figure}

%%%%%%%%%%%%%%%%%%%%%%%%%%%%%%%%%%%%%%%%%%%%%%%%%%
\begin{table}[h]
	\centering
	\caption{Table caption}
	\begin{tabular}{cccc}
		\hline
		\textbf{a}	& \textbf{x} & \textbf{y} & \textbf{z} \\
		\hline
		-50 & -0.289 & -0.289 & -0.598\\ 
		-40 & -0.248 & -0.248 & -0.512\\ 
		\hline 
	\end{tabular} 
	\label{tab:xyz}
\end{table}
%%%%%%%%%%%%%%%%%%%%%%%%%%%%%%%%%%%%%%%%%%%%%%%%%%

The scheme of experimental setup is shown in Fig.~\ref{fig:bc}.  
The values are collected in Tab.~\ref{tab:xyz}.


%The details are described in a book~\cite{udd2011fiber}. 

%Similar case was analyzed by Hill et al.~\cite{hill1978photosensitivity}


\bibliography{E:/aidd_new/aidd/reports/monograph/template/refrences/ref} % 
%bibliography data in report.bib
\bibliographystyle{abbrv}
 % makes bibtex use spiebib.bst



%%%%%%%%%%%%%%%%%%%%%%%%%%%%%%%%%%%%%%%%%%%%%%%%%%
\end{document}
%%%%%%%%%%%%%%%%%%%%%%%%%%%%%%%%%%%%%%%%%%%%%%%%%%