%%%%%%%%%%%%%%%%%%%%%%%%%%%%%%%%%%%%%%%%%%%%%%%%%%
\documentclass[b5paper,11pt, titlepage]{book}
%%%%%%%%%%%%%%%%%%%%%%%%%%%%%%%%%%%%%%%%%%%%%%%%%%
\usepackage[pdftex]{graphicx,color}
%\usepackage[T1,plmath]{polski}
\usepackage[cp1250]{inputenc}
\usepackage{indentfirst}
\usepackage[numbers,sort&compress]{natbib} % sort and compress citations
%\usepackage[none]{hyphenat} % brak podzia�?u wyraz??w
\usepackage{geometry}
\newgeometry{tmargin=3.6cm, bmargin=3.6cm, lmargin=3.2cm, rmargin=3.2cm}
\usepackage{multirow}
\usepackage{amsmath}

\renewcommand{\figurename}{Fig.}
\renewcommand{\tablename}{Tab.}

%%%%%%%%%%%%%%%%%%%%%%%%%%%%%%%%%%%%%%%%%%%%%%%%%%
\begin{document}
%%%%%%%%%%%%%%%%%%%%%%%%%%%%%%%%%%%%%%%%%%%%%%%%%%
%\title{\textbf{Modelowanie struktur anizotropowych \\Wst�pne badania eksperymentalne\\
%	\vspace{10cm}}
%\normalsize{Abstract} }
%\normalsize{W ramach projektu pt.: \\ \textit{Wp�yw jednoczesnego oddzia�ywania temperatury i wilgotno�ci \\na struktury anizotropowe: od teorii do bada� do�wiadczalnych\\}
%NCN OPUS 12}}
	
%{Micha� Jurek}

%\date{ }
	
%\date{Gda�sk, Maj 2018\\ (Nr Arch. 221/2018)}


	
%\maketitle
%\newpage
%\tableofcontents
%\newpage
%\listoffigures
%\listoftables
%\newpage

\chapter{Introduction to Structural Health Monitoring and Artificial Intelligence}
\textbf{Saeed Ullah}
%%%%%%%%%%%%%%%%%%%%%%%%%%%%%%%%%%%%%%%%%%%%%%%%%%
\section{Structural Health Monitoring (SHM)}
%%%%%%%%%%%%%%%%%%%%%%%%%%%%%%%%%%%%%%%%%%%%%%%%%%
Numerous civil engineering and aerospace structures are exceeding or approaching their design lives. Therefore, assessing the condition of these structures is essential in order to determine their serviceability, safety
and load-carry capacity. Recently, Damage detection methods have been widely studied for the purpose of locating and quantifying structural damages. A damage can be defined as any degradation in the structural parameters which changes the dynamic behavior of the structure\cite{Chang2000}. Damage adversely affect the current or future performance of the system. There are many ways and indicators for detecting damage in a structure such as variations in strain, natural frequencies, time signal, etc.\cite{DeLuca2020}.

\section{SHM and Nondestructive Evaluation/Testing (NDE/T)}
%%%%%%%%%%%%%%%%%%%%%%%%%%%%%%%%%%%%%%%%%%%%%%%%%%
text

\section{The Importance of SHM}
%%%%%%%%%%%%%%%%%%%%%%%%%%%%%%%%%%%%%%%%%%%%%%%%%%
text

\section{Different Types of SHM}
%%%%%%%%%%%%%%%%%%%%%%%%%%%%%%%%%%%%%%%%%%%%%%%%%%
text

\section{Different Tools used in SHM}
%%%%%%%%%%%%%%%%%%%%%%%%%%%%%%%%%%%%%%%%%%%%%%%%%%
text

\section{Types of Sensors used in SHM}
%%%%%%%%%%%%%%%%%%%%%%%%%%%%%%%%%%%%%%%%%%%%%%%%%%
text

\section{SHM in Composite Structures}
%%%%%%%%%%%%%%%%%%%%%%%%%%%%%%%%%%%%%%%%%%%%%%%%%%
text

\section{Major Challenges in SHM}
%%%%%%%%%%%%%%%%%%%%%%%%%%%%%%%%%%%%%%%%%%%%%%%%%%
text

\section{Artificial Intelligence, Machine Learning and Deep Learning}
%%%%%%%%%%%%%%%%%%%%%%%%%%%%%%%%%%%%%%%%%%%%%%%%%%
text

\section{AI, ML and DL in SHM}
%%%%%%%%%%%%%%%%%%%%%%%%%%%%%%%%%%%%%%%%%%%%%%%%%%
text


%%%%%%%%%%%%%%%%%%%%%%%%%%%%%%%%%%%%%%%%%%%%%%%%%%
%\subsection{Description}
%%%%%%%%%%%%%%%%%%%%%%%%%%%%%%%%%%%%%%%%%%%%%%%%%%
%text
%%%%%%%%%%%%%%%%%%%%%%%%%%%%%%%%%%%%%%%%%%%%%%%%%%

%\begin{figure} [h!]
%	\begin{center}
		%\includegraphics[width=14cm]{Graphics/bc.jpg}
%	\end{center}
%	\caption{Figure caption.} 
%	\label{fig:bc}
%\end{figure}

%%%%%%%%%%%%%%%%%%%%%%%%%%%%%%%%%%%%%%%%%%%%%%%%%%

%\begin{table}[h]
%\centering
%	\caption{Table caption}
%	\begin{tabular}{cccc}
%		\hline
%	\textbf{a}	& \textbf{x} & \textbf{y} & \textbf{z} \\
%		\hline
%		-50 & -0.289 & -0.289 & -0.598\\ 
%		-40 & -0.248 & -0.248 & -0.512\\ 
%		\hline 
%	\end{tabular} 
%	\label{tab:xyz}
%\end{table}
%%%%%%%%%%%%%%%%%%%%%%%%%%%%%%%%%%%%%%%%%%%%%%%%%%

%The scheme of experimental setup is shown in Fig.~\ref{fig:bc}.  
%The values are collected in Tab.~\ref{tab:xyz}.


%The details are described in a book~\cite{udd2011fiber}. 

%Similar case was analysed\cite{Badrinarayanan2017a} by Hill et al.
%Additional information:
%\begin{itemize}
%\item fonts Times New Roman, 11pt
%\item keep figures separately in greyscale with resolution 600 dpi (publisher requirement)
%\item 20-30 pages
%\item Bibliography\cite references in the order of citations within the text
%\end{itemize}

\bibliography{report} % 
%bibliography data in report.bib
\bibliographystyle{abbrv}
% makes bibtex use spiebib.bst


%%%%%%%%%%%%%%%%%%%%%%%%%%%%%%%%%%%%%%%%%%%%%%%%%%
\end{document}
%%%%%%%%%%%%%%%%%%%%%%%%%%%%%%%%%%%%%%%%%%%%%%%%%%
