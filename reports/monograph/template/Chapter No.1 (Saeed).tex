%%%%%%%%%%%%%%%%%%%%%%%%%%%%%%%%%%%%%%%%%%%%%%%%%%
\documentclass[b5paper,11pt, titlepage]{book}
%%%%%%%%%%%%%%%%%%%%%%%%%%%%%%%%%%%%%%%%%%%%%%%%%%
\usepackage[pdftex]{graphicx,color}
%\usepackage[T1,plmath]{polski}
\usepackage[cp1250]{inputenc}
\usepackage{indentfirst}
\usepackage[numbers,sort&compress]{natbib} % sort and compress citations
%\usepackage[none]{hyphenat} % brak podzia�?u wyraz??w
\usepackage{geometry}
\newgeometry{tmargin=3.6cm, bmargin=3.6cm, lmargin=3.2cm, rmargin=3.2cm}
\usepackage{multirow}
\usepackage{amsmath}

\renewcommand{\figurename}{Fig.}
\renewcommand{\tablename}{Tab.}

%%%%%%%%%%%%%%%%%%%%%%%%%%%%%%%%%%%%%%%%%%%%%%%%%%
\begin{document}
%%%%%%%%%%%%%%%%%%%%%%%%%%%%%%%%%%%%%%%%%%%%%%%%%%
%\title{\textbf{Modelowanie struktur anizotropowych \\Wst�pne badania eksperymentalne\\
%	\vspace{10cm}}
%\normalsize{Abstract} }
%\normalsize{W ramach projektu pt.: \\ \textit{Wp�yw jednoczesnego oddzia�ywania temperatury i wilgotno�ci \\na struktury anizotropowe: od teorii do bada� do�wiadczalnych\\}
%NCN OPUS 12}}
	
%{Micha� Jurek}

%\date{ }
	
%\date{Gda�sk, Maj 2018\\ (Nr Arch. 221/2018)}


	
%\maketitle
%\newpage
%\tableofcontents
%\newpage
%\listoffigures
%\listoftables
%\newpage

\chapter{Introduction to Structural Health Monitoring and Artificial Intelligence}
\textbf{Saeed Ullah}
%%%%%%%%%%%%%%%%%%%%%%%%%%%%%%%%%%%%%%%%%%%%%%%%%%
\section{Structural Health Monitoring (SHM)}
%%%%%%%%%%%%%%%%%%%%%%%%%%%%%%%%%%%%%%%%%%%%%%%%%%
Numerous civil engineering and aerospace structures are exceeding or approaching their design lives. Therefore, assessing the condition of these structures is essential in order to determine their serviceability, safety
and load-carry capacity\cite{Chang2000}. It is very crucial to monitor the health of structural elements in mechanical, civil and aerospace industries where the essence of small defects may result in a very catastrophic failure. A defect or damage can be defined as any degradation in the structural parameters which changes the dynamic behavior of the structure\cite{Chang2000}. These changes can be in a micro-scale level such as material matrix anomalies or in the macro-scale level such as cracks. Damage adversely affect the current or future performance of the system. Recently, Damage detection methods have been widely studied for the purpose of locating and quantifying structural damages\cite{Chang2000}. There are many ways and indicators for detecting damage in a structure such as variations in strain, natural frequencies, time signal, etc.\cite{DeLuca2020}. Damage detection is usually accomplished in the context of one or more closely related disciplines which include: structural health monitoring (SHM), nondestructive evaluation (NDE) also known as nondestructive testing (NDT), condition monitoring (CM), health and usage monitoring system (HUMS), damage prognosis (DP) and statistical process control (SPC)\cite{Farrar2007, Farrar2012}. SHM can be defined as the process of implementing a damage detection and health assessment strategy for civil, mechanical engineering or aerospace  infrastructure\cite{Farrar2007, Farrar2012}. This process includes continuous monitoring of a mechanical system or structure using dynamic response measurements. For determining the current state of system health, the damage-sensitive features extracted from these measurements are used\cite{Farrar2007, Farrar2012}. The output of these measurements can be periodically updated for long-term SHM. These measurements are very helpful in the case of an extreme event. SHM could be used for providing, in near real time, reliable information and rapid condition screening about the performance of the system\cite{Farrar2012}. SHM aims to detect, identify and characterize the damage and degradation in engineering structures. Sensors are used in SHM system for monitoring physical quantities such as temperature, acceleration, humidity, tensile and compressive stress, and so on.\cite{Lamonaca2018}. SHM systems' for damage detection needs few special characteristics such as (i) low possibility of missing the damage (ii) rapid calculation and suitable for continual on-line monitoring (iii) handling of huge information applicable for large engineering structures\cite{lee2008overview}.

SHM systems� tasks can be categorized as a process composed of five activities which forms five important levels or elements, as shown in Figure 1.1. These are: (i) damage detection, (ii) damage localization, (iii), assessment of damage size, (iv) remaining life prediction, and (v) smart structures with self-evaluating, or control capabilities\cite{stepinski2013advanced}. 

In this context detection gives a qualitative indication that damage might be present, localization gives information about the probable position of damage, assessment estimates its severity by providing information about damage type and size and finally, prognosis estimates the residual structural life and predicts possible breakdown or failure. The first three levels (i.e. detection, localization and assessment) are mostly related to system identification, modelling and signal processing aspects. The level of prognosis falls into the field of fatigue analysis, fracture mechanics, design assessment, reliability and statistical analysis. This level is very intensively investigated in many laboratories but there are currently no commercially available solutions. All these levels require various elements of data, signal and/or information processing\cite{stepinski2013advanced}. 


\section{SHM and Nondestructive Evaluation/Testing (NDE/T)}
%%%%%%%%%%%%%%%%%%%%%%%%%%%%%%%%%%%%%%%%%%%%%%%%%%
text

\section{The Importance of SHM}
%%%%%%%%%%%%%%%%%%%%%%%%%%%%%%%%%%%%%%%%%%%%%%%%%%
text

\section{Different Types of SHM}
%%%%%%%%%%%%%%%%%%%%%%%%%%%%%%%%%%%%%%%%%%%%%%%%%%
text

\section{Different Tools used in SHM}
%%%%%%%%%%%%%%%%%%%%%%%%%%%%%%%%%%%%%%%%%%%%%%%%%%
text

\section{Types of Sensors used in SHM}
%%%%%%%%%%%%%%%%%%%%%%%%%%%%%%%%%%%%%%%%%%%%%%%%%%
text

\section{SHM in Composite Structures}
%%%%%%%%%%%%%%%%%%%%%%%%%%%%%%%%%%%%%%%%%%%%%%%%%%
text

\section{Major Challenges in SHM}
%%%%%%%%%%%%%%%%%%%%%%%%%%%%%%%%%%%%%%%%%%%%%%%%%%
text

\section{Artificial Intelligence, Machine Learning and Deep Learning}
%%%%%%%%%%%%%%%%%%%%%%%%%%%%%%%%%%%%%%%%%%%%%%%%%%
text

\section{AI, ML and DL in SHM}
%%%%%%%%%%%%%%%%%%%%%%%%%%%%%%%%%%%%%%%%%%%%%%%%%%
text


%%%%%%%%%%%%%%%%%%%%%%%%%%%%%%%%%%%%%%%%%%%%%%%%%%
%\subsection{Description}
%%%%%%%%%%%%%%%%%%%%%%%%%%%%%%%%%%%%%%%%%%%%%%%%%%
%text
%%%%%%%%%%%%%%%%%%%%%%%%%%%%%%%%%%%%%%%%%%%%%%%%%%

%\begin{figure} [h!]
%	\begin{center}
		%\includegraphics[width=14cm]{Graphics/bc.jpg}
%	\end{center}
%	\caption{Figure caption.} 
%	\label{fig:bc}
%\end{figure}

%%%%%%%%%%%%%%%%%%%%%%%%%%%%%%%%%%%%%%%%%%%%%%%%%%

%\begin{table}[h]
%\centering
%	\caption{Table caption}
%	\begin{tabular}{cccc}
%		\hline
%	\textbf{a}	& \textbf{x} & \textbf{y} & \textbf{z} \\
%		\hline
%		-50 & -0.289 & -0.289 & -0.598\\ 
%		-40 & -0.248 & -0.248 & -0.512\\ 
%		\hline 
%	\end{tabular} 
%	\label{tab:xyz}
%\end{table}
%%%%%%%%%%%%%%%%%%%%%%%%%%%%%%%%%%%%%%%%%%%%%%%%%%

%The scheme of experimental setup is shown in Fig.~\ref{fig:bc}.  
%The values are collected in Tab.~\ref{tab:xyz}.


%The details are described in a book~\cite{udd2011fiber}. 

%Similar case was analysed\cite{Badrinarayanan2017a} by Hill et al.
%Additional information:
%\begin{itemize}
%\item fonts Times New Roman, 11pt
%\item keep figures separately in greyscale with resolution 600 dpi (publisher requirement)
%\item 20-30 pages
%\item Bibliography\cite references in the order of citations within the text
%\end{itemize}

\bibliography{report} % 
%bibliography data in report.bib
\bibliographystyle{abbrv}
% makes bibtex use spiebib.bst


%%%%%%%%%%%%%%%%%%%%%%%%%%%%%%%%%%%%%%%%%%%%%%%%%%
\end{document}
%%%%%%%%%%%%%%%%%%%%%%%%%%%%%%%%%%%%%%%%%%%%%%%%%%
