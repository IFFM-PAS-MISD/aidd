%%%%%%%%%%%%%%%%%%%%%%%%%%%%%%%%%%%%%%%%%%%%%%%%%%
\section*{Summary}
\label{conclusion}
%%%%%%%%%%%%%%%%%%%%%%%%%%%%%%%%%%%%%%%%%%%%%%%%%%
In this chapter, five different DL models: Res-UNet, VGG16 encoder-decoder, PSPNet, FCN-DenseNet, and GCN have been introduced and trained to perform image semantic segmentation in order to identify delamination in CFRP materials.
The models were trained on a numerically generated dataset simulating a full wavefield of propagating guided waves.
The results of DL models in identifying various forms of delaminations in terms of their locations, shapes, sizes, and angles are encouraging.
Furthermore, the models demonstrate good generalization when it comes to predicting delamination in numerically acquired data which has not been observed before.
Furthermore, the models demonstrate their generalisation ability by detecting delamination in experimentally measured data.

Regarding the numerical test data and the experimental scenario, it can be stated that the GCN model has the highest identification accuracy among all implemented models.
Furthermore, the PSPNet model had the lowest identification accuracy of all the models.
As a result, among all applied DL models, the GCN model has the best performance.
Furthermore, training DL models on experimental data will allow them to learn new complex patterns, hence, their performance will increase.
