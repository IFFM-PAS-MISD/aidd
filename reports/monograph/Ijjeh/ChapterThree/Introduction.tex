\section{Introduction}
Composite materials have gained popularity in recent years in a variety of industries, including aerospace, automobiles, construction, marine, and others, due to their light-weight, excellent fatigue, and corrosion resistance.
However, composite materials may suffer from matrix cracks, fibre breakage, debonding, and delamination~\cite{ip2004delamination, smith2009composite}.
Delamination (separation of layers in a laminate composite) is one of the most dangerous of these flaws since it occurs largely below top surfaces and is scarcely noticeable~\cite{Cai2012a}.

Delamination in composite materials can be caused by a variety of factors, including manufacturing flaws, notches, and impact events.
As a result, delamination can diminish the engineering structure's strength and performance.
To avoid such complications, real-time delamination detection is necessary.
In order to monitor the integrity of engineering structures, numerous physics-based approaches for damage detection and localization have been developed in the disciplines of structural health monitoring (SHM) and non-destructive testing (NDT).

Guided waves, namely Lamb waves, are used in a well-known physics-based technique in the field of SHM for damage diagnosis.
Lamb waves are elastic waves that propagate through thin plates and shells that are bounded by stress-free surfaces~\cite{mitra2016guided}.
Lamb waves are notable for their strong sensitivity to discontinuities (cracks, delaminations) and low amplitude loss, particularly in metallic elements of structures~\cite{Keulen2014}.

An array of PZT transducers can be used to excite the examined structure and generate Lamb waves, which can subsequently be registered as reflected waves from damage.
After that, a damage influence map is generated.
The number of sensing points determines the accuracy of the damage influence map in displaying damage location.
As a result, damage localization resolution may be limited.
Alternatively, a Scanning Laser Doppler Vibrometer (SLDV) is used to measure Lamb waves in a dense grid of spots over the structure under investigation.
The collected data represents a complete wavefield propagation, resulting in high-resolution damage influence maps.
Damage detection systems that use whole wavefield signals can accurately estimate the extent and location of damage~\cite{Girolamo2018a, kudela2018impact}.
\\
Support vector machine (SVM) \cite{noori2010application, Khoa2014, Ghiasi2016}, K-Nearest Neighbor (KNN) \cite{Vitola2017}, decision tree~\cite{Mariniello2020}, particle swarm optimization (PSO)~\cite{Khatir2018, NouriShirazi2014}, principle component analysis (PCA)~\cite{wang2014principal, nguyen2010fault, liu2014research} are considered a conventional machine learning methods that are utilised for damage detection with SHM.
Such methods among others have proven to be capable of detecting damage in the structures under investigation.
However, when dealing with large amounts of data, such methods have drawbacks since they demand a sophisticated feature engineering calculation~\cite{Gulgec2019}, which also necessitates a high level of expertise and abilities to extract damage-sensitive features for specific SHM applications.
As a result, in recent years, a data-driven solutions for SHM applications have emerged in the form of deep learning (DL) end-to-end approaches, which automate the feature engineering and classification processes.

High- and abstract-level features can be translated into a hierarchical order of simple- and low-level obtained features using deep learning techniques~\cite{goodfellow2016deep}.
As a result, DL approaches may handle big issues by breaking them down into a huge number of basic blocks.
Another important benefit of applying deep learning techniques is so-called transfer learning, which refers to the ability to reuse a pre-trained model created for one task in another.

Due to rapid advancements in computer hardware and software, big data, and cloud-based computations, DL techniques in many SHM domains have gotten increased attention in recent years~\cite{Azimi}.
For SHM of civil engineering structures, several DL-based approaches were used for damage detection and localization~\cite{Cha2018, Kong2018}, corrosion detection~\cite{Atha2018}, and concrete crack detection~\cite{Dung2019}.
%%%%%%%%%%%%%%%%%%%%%%%%%%%%%%%%%%%%%%%%%%%%%%%%%%%%%%%%%%%%%%%%%%%%%%%%%%%%%%%%

De Assis et al.~\cite{DeAssis2021} presented a comparative study based on modal responses for crack identification in laminated composites.
Furthermore, the authors of this paper solved an inverse crack identification problem using the metaheuristic sunflower optimization (SFO) algorithm, artificial neural networks (ANNs), and the response surface approach.
The results showed that the SFO and ANN approaches may both be utilized to estimate a crack's location, size, and orientation.
Furthermore, Oliver et al.~\cite{Oliver2021} developed ANNs for delamination detection in composite laminates that were merely trained on frequency fluctuation values as inputs.
The proposed model had a damage quantification success rate up to \(95\% \).
%%%%%%%%%%%%%%%%%%%%%%%%%%%%%%%%%%%%%%%%%%%%%%%%%%%%%%%%%%%%%%%%%%%%%%%%%%%%%%%%

In comparison to vibration-based SHM, DL applications for guided wave-based SHM have received less attention in the literature.

Several DL approaches for guided wave-based damage identification and localization are discussed in the following sections.
For damage detection in curved composite panels, Chetwynd et al.~\cite{Chetwynd2008damage} suggested a multi-layer perceptron MLP network.
A force applicator with a circular tip loaded by a mass was used to imitate the damage.
Lamb waves propagating through the panel were also generated and registered using an array of PZT transducers.
In addition, for each Lamb wave response, a novelty index was calculated and compared to specific threshold values.
As a result, if the index exceeds the threshold, it shows that the structure is damaged.
The generated novelty indices were fed into the MLP network, which performed two operations: classification and regression.
The classification network was created to assess whether the panel was damaged or not by defining three convex sections of the panel.
The regression network, on the other hand, can pinpoint the specific area of the damage.

In addition, the authors~\cite{DeFenza2015} presented an ANN model for detecting damage in plates comprised of aluminum alloys and composites.
The ANN was trained using synthetic data generated with the finite element approach.
Furthermore, the researchers calculated damage indexes using the data collected from propagating Lamb waves.

Authors in~\cite{Melville2018} developed a CNN model for damage state prediction based on full wavefield scans of thin aluminum plates.
When compared to SVM, the model achieved superior accuracy in terms of damage $(99.98\%)$ compared to SVM ($62\%$).

Ewald et al.~\cite{Ewald2019b} presented a CNN model for signal categorization using Lamb waves (DeepSHM).
The model uses reflected signals acquired by sensors to perform an end-to-end strategy for SHM.
The authors also used the wavelet transform to preprocess response data in order to generate the wavelet coefficient matrix (WCM), which was then input into the CNN model.

Liu and Zhang~\cite{Liu2020a} developed a CNN model for detecting damage in thin aluminum plates.
Lamb waves were generated using analytical procedures that were used for training and validation.
Furthermore, the authors validated their model by putting it to the test on real-world data using a notch crack to simulate damage.

In addition, Esfandabadi et al.~\cite{esfandabadideep} studied the use of compressive sensing technique~\cite{Candes2006} in conjunction with super-resolution techniques to obtain high-resolution wavefield scans using neural networks trained on various aluminum and Composite Fibre Reinforced Polymer (CFRP) plates.
To recover high spatial frequency information from low-resolution wavefield scans, the authors implemented two types of CNN architecture: Super-Resolution Convolutional Neural Networks (SRCNNs) and Very-Deep Super Resolution (VDSR) using compressive sensing.
Enhancing the resolution, on the other hand, has a negative impact on the affected region, changing the damage characteristics.
%%%%%%%%%%%%%%%%%%%%%%%%%%%%%%%%%%%%%%%%%%%%%%%%%%%%%%%%%%%%%%%%%%%%%%%%%%%%%%%%

Current SHM approaches can effectively localize impact events~\cite{Ciampa2012}, or damage~\cite{Nokhbatolfoghahai2020}.
However, with a sparse array of sensors, determining the magnitude and shape of damage is challenging.
To solve this issue, full wavefield ultrasonic methods can be used locally to assess damage severity and then to estimate damage prognosis.
%%%%%%%%%%%%%%%%%%%%%%%%%%%%%%%%%%%%%%%%%%%%%%%%%%%%%%%%%%%%%%%%%%%%%%%%%%%%%%%%

In this chapter, we compare five DL models for semantic image segmentation that have been used in CFRP for delamination detection, localization, and size estimate.
To demonstrate their generalization ability, the models were tested using numerical and experimental data.
%%%%%%%%%%%%%%%%%%%%%%%%%%%%%%%%%%%%%%%%%%%%%%%%%%%%%%%%%%%%%%%%%%%%%%%%%%%%%%%%