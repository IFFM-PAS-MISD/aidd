%%%%%%%%%%%%%%%%%%%%%%%%%%%%%%%%%%%%%%%%%%%%%%%%%%
\documentclass[b5paper, 11pt, openany, titlepage]{book}
%%%%%%%%%%%%%%%%%%%%%%%%%%%%%%%%%%%%%%%%%%%%%%%%%%


\usepackage[pdftex]{graphicx,color}
%\usepackage[T1,plmath]{polski}
\usepackage[cp1250]{inputenc}
\usepackage{indentfirst}
\usepackage[numbers,sort&compress]{natbib}
\usepackage{geometry}
\newgeometry{tmargin=3.6cm, bmargin=3.6cm, lmargin=3.2cm, rmargin=3.2cm}
\usepackage{multirow}
\usepackage{amsmath}
\usepackage{amssymb}
\usepackage{pdflscape}
\usepackage{adjustbox} 
\usepackage{etoolbox}
\usepackage{chapterbib}
% fix long section titles in toc
\usepackage{booktabs}
\usepackage{suffix}
\usepackage{textcomp}
\usepackage{hyperref}
\usepackage{xspace} 
\usepackage{textcomp} 
\usepackage{csquotes}
\usepackage{subcaption}
\renewcommand{\figurename}{Fig.}
\renewcommand{\tablename}{Tab.}

%%%%%%%%%%%%%%%%%%%%%%%%%%%%%%%%%%%%%%%%%%%%%%%%%%
\begin{document}
%%%%%%%%%%%%%%%%%%%%%%%%%%%%%%%%%%%%%%%%%%%%%%%%%%


\title{Data-driven based approach for damage detection}
\author{Abdalraheem A. Ijjeh}
\maketitle
\tableofcontents
\chapter[Data-driven approach]{Data-driven based approach for damage detection}
%\newpage
%%%%%%%%%%%%%%%%%%%%%%%%%%%%%%%%%%%%%%%%%%%%%%%%%%
\section{Introduction}
Engineering structures such as buildings, roads, tunnels, power generation systems, rotating machinery, and aircrafts are considered very important in our modern life.
However, such structures are prone to various types of damage, therefore it is essential to maintain them and keep them safe during their operational lifetime.
Health monitoring presents an essential tool in management activities as it allows identifying early and progressive structural damage~\cite{farrar2007introduction}. 
Obtained data from monitoring structures are large and need to be transformed into valuable information to assist the development and design of maintenance activities, improve safety, reduce uncertainties and extend our knowledge regarding the monitored structure.
Structural health monitoring (SHM) is one of the most robust
tools for managing infrastructure.
Traditionally, the procedure of performing an autonomous damage identification for engineering structures whether civil, mechanical or aerospace is referred to as SHM~\cite{farrar2001vibration}.
SHM aims to describe a real-time evaluation of a structure during its life-cycle~\cite{Balageas2010}. 
Moreover, SHM assists in detecting and characterizing damage in a structure as a whole or its parts. 
Damage detection in a structure is crucial since it may reduce safety and performance during its operational lifetime~\cite{Yuan2016}.
Furthermore, the SHM approach involves monitoring a structure continuously through an array of sensors that periodically measure the response of the structure then extracting the sensitive damage features from these measurements to perform statistical analysis on these features to examine the condition of the structure.
Generally, there are two approaches to SHM: physics-based and
data-based.
In the physics-based approach, the inverse problem method is applied in which numerical models such as finite element models are implemented. 
Furthermore, damage identification results by comparing the registered readings from the structure and the estimated data from the models.
On the other hand,  the data-based approach is related to the artificial intelligence domain (machine learning and deep learning), in which artificial models are developed to learn the behavior of the structure based on earlier registered data that leads to performing pattern recognition for the damage identification.
%%%%%%%%%%%%%%%%%%%%%%%%%%%%%%%%%%%%%%%%%%%%%%%%%%
The data-based approach can be applied in both supervised and unsupervised learning~\cite{worden2007application}.
Supervised learning can be utilised in the field of SHM where data of the damaged and undamaged conditions are available in which the detection models can train~\cite{figueiredo2018machine}.
On the other hand, unsupervised learning is applied when data of undamaged cases are only available, therefore the detection models train only on such data~\cite{figueiredo2018machine}. 
%%%%%%%%%%%%%%%%%%%%%%%%%%%%%%%%%%%%%%%%%%%%%%%%%%
In this chapter, several machine learning (ML) and deep learning (DL) techniques for damage detection are presented.
Furthermore, ML techniques such as principal component analysis (PCA), Gaussian mixture models (GMMS), Mahalanobis squared distance (MSD) and Bioinspired algorithms will be illustrated.
For the DL approach, techniques such as artificial neural networks ANN, Convolutional neural networks (ConvNet/CNN), and recurrent neural networks (RNN) will be illustrated.
Furthermore, data acquired based on the guided waves approach and vibration-based approach will be presented. 
%%%%%%%%%%%%%%%% Section 2 %%%%%%%%%%%%%%%%%%%%%%%
\section{Machine learning approach}
ML techniques in SHM were heavily utilised by researchers for damage detection~\cite{Doebling1998, alvandi2006assessment, fan2011vibration, raghavan2008effects, su2009identification, Mitra2016}.
Moreover, machine learning techniques attempt to map the patterns of the input data acquired by sensors to output targets for a damage estimation at different levels ~\cite{rytter1993vibrational}.
Accordingly, ML techniques demands high domain knowledge of the examiner to perform hand-crafted damage-sensitive feature extraction on the raw data acquired by sensors before being fed into a suitable ML model.
Generally, the process of damage-sensitive features extraction (hand-crafted) in the field of SHM emerged due to the enormous development in the physics-based SHM techniques such as modal strain energy MSE~\cite{Kim}, modal curvature MC~\cite{Wahab}, modal assurance criterion (MAC), and Coordinate MAC~\cite{Allemang2003}, modal flexibility (MF)~\cite{Jaishi}, damage locating vector (DLV)~\cite{Bernal2002}, wavelet transform~\cite{Staszewski,Kima} and probabilistic reconstruction algorithm (PRA)~\cite{Hay2006} among others.

In this section, we are going to describe several feature extraction techniques and classification models used with machine learning utilised for structural damage detection.
These algorithms are suitable for scenarios where the sensitive damage features obtained from the structural responses are affected by the changes that occur due to the operational and environmental variability and the changes made by the damage.
%%%%%%%%%%%%%%%%%%%%%%%%%%%%%%%%%%%%%%%%%%%%%%%%%%
\subsection{Feature extraction techniques}
\subsubsection{Principal component analysis}
PCA is a popular method used for damage identification in SHM.
Further, PCA shows a solid and efficient performance in feature extraction, and structural damage detection~\cite{liu2014research, wang2014principal, nguyen2010fault}. 
Besides, PCA proves to be an effective tool to improve the training efficiency and enhance the classification accuracy for other ML algorithms, such as unsupervised learning methods~\cite{liu2019rapid, datteo2017statistical, torres2014data}. 

PCA is dimensionality reduction technique utilised to reduce the dimensionality of large data (input space) into a lower dimension (feature space) through transforming a large set of variables into a smaller one with minimal loss information~\cite{Jolliffe2002}.
Moreover, PCA can be utilised for damage detection by eliminating noise and obtaining sensitive features of damage as eigenvectors.
The PCA technique is illustrated below.
In the beginning, a matrix \(U(t)\) is constructed as shown in Eqn. \ref{U(t)}, which contains all registered data with time histories.
\begin{equation}
	U(t)=
	\begin{bmatrix}
		u_1{(t1)}       & u_2{(t1)} & \dots & u_M{(t1)} \\
		u_1{(t2)}       & u_2{(t2)} & \dots & u_M{(t2)} \\
		\vdots 			& \vdots 	& \ddots & \vdots \\
		u_1{(t_N)}      & u_2{(t_N)} & \dots & u_M{(t_N)}
	\end{bmatrix}
	\label{U(t)}
\end{equation}
Where \(t\) corresponds to the time, \(u_i\ (i = 1, 2, ..., M)\) represents to the response from the \(i-th\) sensor installed in the monitored structure, \(M\) represents the total number of sensors, \(t_j\ (j = 1, 2, ..., N)\) represents the \(j-th\) time step of the data registering and \(N\) is the total time observations during monitoring.
Additionally, each column represents data registration of one sensor.
The next step is to normalise the time series of each sensor data registrations by subtracting the mean value shown in Eqn.~\ref{mean value}
\begin{equation}
	\bar{u_i} = \frac{1}{N}\sum_{j=1}^{N}u_i(t_j)
	\label{mean value}
\end{equation}
Equation~\ref{normalised matrix} represents the normalised matrix.
\begin{equation}
	U'(t)=
	\begin{bmatrix}
		u_1{(t1)}-\bar{u_1}       & u_2{(t1)}-\bar{u_2} & \dots  & u_M{(t1)}-\bar{u_M} \\
		u_1{(t2)}-\bar{u_1}       & u_2{(t2)}-\bar{u_2} & \dots  & u_M{(t2)}-\bar{u_M} \\
		\vdots 					  & \vdots 	  			& \ddots & \vdots \\
		u_1{(t_N)}-\bar{u_1}      & u_2{(t_N)}-\bar{u_2}& \dots  & u_M{(t_N)}-\bar{u_M}
	\end{bmatrix}
	\label{normalised matrix}
\end{equation}
After computing the normalised matrix, the covariance matrix is computed as shown in Eqn.~\ref{covariance}
\begin{equation}
	C = \frac{1}{M}U'^TU'
	\label{covariance}
\end{equation}
Next, the eigenvalue and the corresponding eigenvector of the covariance matrix are computed through solving the following equation~\ref{eigvalue}.
\begin{equation}
	(C-\lambda_iI)\psi_i =0
	\label{eigvalue}
\end{equation}
Where \(I\) represents the \(M\times M\) identity matrix, \(\psi_i = [\psi_{i,1},\psi_{i,2}, \hdots, \psi_{i,j}]^T\) in which \(\psi_{i,j}(j=1, 2, \hdots, M)\) is the element related to the \(j-th\) sensor.
Usually, eigenvalues are sorted into decreasing order, particularly \(\lambda_1>\lambda_2>\hdots>\lambda_M\). 
Then, the first eigenvector \(\psi_1\) corresponding to \(\lambda_1\) holds the greatest variance and consequently holds the most important information for the original matrix U. 
The first few principal components hold most of the variance, whereas the remaining less important components involve the measurement of noise.
Accordingly, the first few eigenvectors are utilised as sensitive features for damage detection and localisation.
%%%%%%%%%%%%%%%%%%%%%%%%%%%%%%%%%%%%%%%%%%%%%%%%%%
\subsubsection{Mahalanobis squared distance}
MSD is an effective multivariate distance measuring technique in which it measures the distance between a point and a distribution.
Therefore, MSD is utilised with multivariate statistics outlier detection~\cite{Worden2000}.
Assuming \(X\) to be a training set with data acquired when the undamaged structure is under environmental and/or operational variations (EOVs) with multivariate mean vector \(\mu\) and covariance matrix \(\Sigma\)~\cite{Farrar2013}.
Accordingly, the damage index \((DI_i)\) between feature vectors from training set \(X\) and any new feature vector from the test matrix \(Z\) is calculated using Eqn.~\ref{msd}.
\begin{equation}
	DI_i = (z_i-\mu)\Sigma^{-1}(z_i-\mu)^T
	\label{msd}
\end{equation}
where \(z_i\) is a tested feature vector.
The performance of this technique mainly relies on acquiring all likely EOVs in the training set~
\cite{Farrar2013}.
%%%%%%%%%%%%%%%%%%%%%%%%%%%%%%%%%%%%%%%%%%%%%%%%%%
\subsubsection{Gaussian mixture models}
GMM is a clustering method commonly used with unsupervised learning, in which it aims to find main clusters of points in a dataset that share some common characteristics or features.
Additionally, GMM has also been referred to as Expectation-Maximization (EM) clustering that is based on the optimization strategy.
%%%%%%%%%%
The damage detection is performed based on multiple MSD-based algorithms, in which the covariance matrices and mean vectors are functions of the main components.
%%%%%%%%%%
A GMM is defined as a superposition of K Gaussian distributions as shown in Eqn. \ref{gmm}.

\begin{equation}
	p(x) = \sum_{k=1}^K P(k) \mathcal{N}(x|\mu_k,\Sigma_k) 
	\label{gmm}
\end{equation}
where \(x\) represents the training samples in the dataset, and \(P(k)\) corresponds to the mixture proportion (contribution weight) of the \(k-\)th distribution, in which the mixture proportion must satisfy \(0\leq P(x)\leq 1\).
The sum of all mixture proportion satisfies the following Eqn.~\ref{mixture}
\begin{equation}
	\sum_{k=1}^{K}P(x) =1 
	\label{mixture}
\end{equation}  
\(\mathcal{N}(x|\mu_k,\Sigma_k)\) refers to the conditional probability of the instance \(x\) for the \(k-\)th Gaussian distribution \(\mathcal{N}(\mu_k,\Sigma_k)\) presented in Eqn.~\ref{conditional}, where \(\mu_k\) and \(\Sigma_k\) are the mean and the covariance of that Gaussian distribution respectively.
\begin{equation}
	\mathcal{N}(x|\mu_k,\Sigma_k) = \frac{\exp(-\frac{1}{2}(x-\mu_k)^T\Sigma_k^{-1}(x-\mu_k))}{(2\pi)^{\frac{d}{2}\sqrt{\det(\Sigma_k)}}}
	\label{conditional}		
\end{equation}
The complete GMM is parameterized by the mean vectors, covariance matrices and the mixture weights from all component densities \(\{\mu_k,\Sigma_k, P(x)\}_{k=1,\hdots,K}\).

The parameters can be carried out from the training data using the classical maximum likelihood estimator (CMLE) based on the EM algorithm~\cite{Dempster1977}.
Damage can be detected through estimating \(k\) \(DIs\) for each data sample \(x\) as shown in Eqn. \ref{DIs}
\begin{equation}
	DI_q(x) = (x-\mu_k)\Sigma_k^{-1}(x-\mu_k)^T
	\label{DIs}
\end{equation}
where \(\mu_k\) and \(\Sigma_k\) refers to all observations from the \(k\) data component.
For each observation the DI is given by the smallest DI estimated on each component as in Eqn. \ref{DI}
\begin{equation}
	DI(x) = \min[DI_k(x)]
	\label{DI}
\end{equation}
%%%%%%%%%%%%%%%%%%%%%%%%%%%%%%%%%%%%%%%%%%%%%%%%%%
\subsection{Classification models}
\subsubsection{Support vector machine}
Support vector machine (SVM) is a supervised ML model that is utilised as a classification and regression tool.  
The idea behind SVM is to find an optimal hyperplane (e.g separate line) in N-dimensional space (N is the number of features) that separates the classes, furthermore, the aim of the hyperplane is to maximize the margin between the points on either side hence so called \enquote{decision line/boundary}.
Furthermore, when we try to separate two classes of data points, we could have many possible hyperplanes, however, our goal is to find the hyperplane that has the maximum margin (maximum distance between data points of both classes). 
Figure~\ref{fig:SVM} shows SVM hyperplanes in 2D feature space and 3D feature space.
\begin{figure}[!h]
	\begin{subfigure}[b]{0.49\textwidth}		
		\centering
		\includegraphics[width=.8\linewidth]{figures/2d_svm.png}
		\caption{Hyperplane 2D feature space } 
		\label{fig:2dsvm}
	\end{subfigure}
	\hfill
	\begin{subfigure}[b]{0.49\textwidth}
		\centering
		\includegraphics[width=1.0\linewidth]{figures/3d_svm.png}
		\caption{Hyperplane 3D feature space} 
		\label{fig:3dsvm}
	\end{subfigure}	
	\caption{SVM for 2D and 3D feature space.}
	\label{fig:SVM}
\end{figure}

\subsubsection{K-Nearest Neighbor}
K-Nearest Neighbor (KNN) is a supervised ML technique utilized to perform classification tasks.
KNN does not have a specialized training phase.
It saves all the training data and uses the entire training set for classifying a new data point, which adds time complexity at the testing time.
Moreover, KNN is a non-parametric learning algorithm, which means it does not have any assumptions regarding the input data, which is useful considering the real-world data does not obey the typical theoretical assumptions such as linear separability, uniform distribution among others.

In the KNN technique, at the first, the distance between the new data point and the whole other data points is calculated.
Furthermore, any distance method can be applied e.g. Euclidean, Manhattan, etc.
Accordingly, it picks the K-nearest points, where K is an integer number (number of neighbors) that can be chosen in such a way the model will be able to predict new unseen data accurately. 
Then, it assigns the new data point to the class to which the majority of the K data points belong.
In Fig.~\ref{fig:datapoints} shows initial data points (training set) before classification, and Fig.~\ref{fig:KNN_K_5} shows the result of applying KNN techniques on the data points (3 classes) assuming \(K=6\).
\begin{figure}[!h]
	\begin{subfigure}[b]{0.49\textwidth}		
		\centering
		\includegraphics[width=1\linewidth]{figures/KNN_datapoints.png}
		\caption{Data points } 
		\label{fig:datapoints}
	\end{subfigure}
	\hfill
	\begin{subfigure}[b]{0.49\textwidth}
		\centering
		\includegraphics[width=1.0\linewidth]{figures/KNN_K_6.png}
		\caption{3-Classes with \(K=6\)} 
		\label{fig:KNN_K_5}
	\end{subfigure}	
	\caption{KNN algorithm: data classification with \(K=6\) .}
	\label{fig:KNN}
\end{figure}

\subsubsection{Decision tree}
Decision trees are supervised ML that is used in applications for classification and regression. 
Additionally, decision trees are considered the bases for many other ML techniques such as random forests, bagging and boosted decision trees.
The idea of a decision tree is to represent the whole data as a tree where each internal node represents a test on an attribute (a decision rule) and each branch represents an outcome of the test, and finally each leaf node (terminal node) holds the label of the class.

Decision tree can be divided into two categories:
\begin{enumerate}
	\item Categorical variable decision trees: which includes categorical target variables that are divided into categories. A category means that the decision falls into one of the categories and there is no in-between such as (Yes/No category).
	\item Continuous variable decision trees: which has a continuous target variable that can be predicted based on available information (e.g. crack length).
\end{enumerate}
Figure~\ref{fig:Decision_tree} presents a typical decision tree.
Any decision tree has a root node where data input is carried through.
Furthermore, the root node is split into sets of decision rules that result either in a leaf node which is a non-splitting node, or into another decision rule, creating what so-called a branch or sub-tree.
In case there are decision rules that can be eliminated from the tree, a process called \enquote{pruning} is applied to minimize the complexity of the algorithm.
\begin{figure}[!h]
	\begin{center}
		\includegraphics[width=1.0\linewidth]{figures/decision_tree.png}
	\end{center}
	\caption{Decision tree.}
	\label{fig:Decision_tree}
\end{figure} 

%%%%%%%%%%%%%%%% Section 3 %%%%%%%%%%%%%%%%%%%%%%%
\section{Deep learning approach}
Conventional ML techniques are incapable of processing large registered data in their raw form.
In conventional ML, the process of features engineering requires high expertise and skills to extract damage-sensitive features for specific SHM applications.
Accordingly, there is no guarantee that such features can be reused for other structures due to the nonlinear behaviour. 

It can be said that the huge development that occurred in the computational powers (e.g. central processing units (CPU), graphical processing units (GPU), etc.), in addition to the availability of big data, and the development of new learning algorithms~\cite{Yuan2020},  allowed DL techniques to develop rapidly.
Consequently, DL-based SHM methods have been utilized to overcome issues related to ML-based SHM.  
DL approach makes it possible to use registered data in their raw form without any need to perform feature engineering, hence, such an approach has an end-to-end structure that will automatically learn and discover the hidden features in high dimensional input data~\cite{LeCun, Networks}. 
Figure ~\ref{fig:DL_ML} illustrates the main differences between the conventional ML-based SHM and DL-based SHM approaches.

\begin{figure}[!h]
	\begin{center}
		\includegraphics[width=1.0\linewidth]{figures/DL_vs_ML.png}
	\end{center}
	\caption{(a) Conventional ML based SHM vs. (b) DL based SHM.}
	\label{fig:DL_ML}
\end{figure} 

In current section, Convolutional neural networks (CNN) and Recurrent neural networks (RNN) will be illustrated.

\subsection{Convolutional neural networks}
ConvNet or CNN is a feed-forward artificial neural network (ANN) inspired by visual cortex in the human brain.
Mainly, CNN architectures are utilised for computer vision applications in which images are processed for classification and segmentation purposes.
The main components of any ANN are neurons or perceptrons.
Figure~\ref{fig:neuron} depicts the general neuron architecture, further, a neuron is capable of performing some sort of non-linear computation through an activation function that acts as a gate to forward the processed signal to the next layer.
\begin{figure}
	\begin{center}
		\includegraphics[scale=1]{figures/fig_neron.png}
	\end{center}
	\captionof{figure}{Neuron architechture.}
	\label{fig:neuron}
\end{figure}
An example of an activation function is Relu (that changes all negative values of the feature map to zero and keeps all positive values unchanged).
Furthermore, a neuron has several connections of weighted inputs and outputs that are updated through a learning process referred as back-propagation. 
Back-propagation is a procedure in which the weights are updated in a way the model learns how to predict the desired output.
For this purpose, a cost function (objective function) is utilised to estimate the difference between the predicted output and the targeted output.
Accordingly, the back-propagation (e.g Gradient descent, Adam and RMSprop) procedure is applied to minimize the cost function.

A typical structure of a CNN is presented in Fig. ~\ref{CNN}. 
It consists of three parts: convolutional layer, downsampling layer and dense layer.
\begin{figure}
	\begin{center}
		\includegraphics[width=1.0\linewidth]{figures/cnn.png}
	\end{center}
	\captionof{figure}{CNN architecture.}
	\label{CNN}
\end{figure}
Convolutional layers are used to extract features from the input image.
Further, a convolution (dot product) is carried out through sliding a window (filter or kernel) of size \((w_f,h_f,d_f)\) all over the input image of a size \((w,h,d)\)  to produce feature maps that are locally correlated.
Usually, a convolutional operation is followed by a non-linear activation function such as Relu that changes all negative values of the feature map to zero.
The next layer is the downsampling (pooling) that joins the related features into one feature for reducing the computation complexity~\cite{LeCun}. 
Dense layers can be fully or partially connected followed by the output layer which produces the predicted outputs. 
%%%%%%%%%%%%%%%%%%%%%%%%%%%%%%%%%%%%%%%%%%%%%%%%%%
\subsection{Recurrent neural networks}
A recurrent neural network (RNN) is a DL model that handles time-series data (sequential data).
Moreover, the RNN technique can remember its data input, because of its internal memory which makes it a powerful and promising technique in the field of DL.
Since there are temporal problems such as natural language processing, language translation, image captioning and so on, they require to be handled sequentially.
In the traditional deep neural networks (feed-forward) it is assumed that there is no correlation between the inputs and the outputs, while this assumption is not true for the RNN technique, that means the output of the RNN depends on the prior input sequence.
The future events can also be used in predicting the output of a given sequence.
Figure~\ref{rnn_vs_nn} depicts the difference between RNN and feed-forward deep neural networks.
\begin{figure}
	\begin{center}
		\includegraphics[width=1.0\linewidth]{figures/rnn_vs_nn.png}
	\end{center}
	\captionof{figure}{(a) RRN vs (b) feed-forward neural network.}
	\label{rnn_vs_nn}
\end{figure}
As shown in the Fig.~\ref{rnn_vs_nn} (a), for the RNN, the output of a certain layer is looped back to its input which helps in making the prediction.
However, in the feed-forward networks, the inputs and outputs are independent, and there is no connection between them.
Figure~\ref{unrolled_rnn} shows the visualisation of an unrolled RNN, where \(x_{t}\) corresponds to the sequential timestamped input at time \(t\), \(h_{t}\) corresponds to internal state  and \(Y_{t}\) corresponds to the predicted timestamped output at time \(t\).
Additionally, the figure shows that an unrolled RNN can be seen as a cascaded sequence of feed-forward networks.
\begin{figure}
	\begin{center}
		\includegraphics[width=1.0\linewidth]{figures/unrolled_rnn.png}
	\end{center}
	\captionof{figure}{Unrolled RNN.}
	\label{unrolled_rnn}
\end{figure}
In the feed-forward neural networks, as mentioned earlier, the learnable parameters (adjustable weights) are available only for the forward path of data propagation that are updated through back-propagation algorithm.
However, for RNN, since there are two paths of data propagation (forward and backward) there are learnable weights for both directions.
In RNN technique, weights are updated using back-propagation through time (BBTT)~\cite{Werbos1990}.
Essentially, BBTT performs back-propagation algorithm on unrolled RNN and since BBTT depends on the number of timestamps this could be computationally expensive when there are a high number of timestamps.

A gradient measures the change in all weights regarding the change in error (the difference between the actual predicted output and the ground truth).
As a result, when implementing RNNs, two issues may arise during updating the learnable weights using BBTT:
\begin{enumerate}
	\item \textbf{Exploding gradients}: that can occur when the assigned values to the weights become so large that leads to overflow and results in not a number (NaN) values~\cite{Brownlee2017a}.
	\item \textbf{Vanishing gradients}: that can occur when the assigned values to the weights become too small which affects the learning process to be very slow or to stop~\cite{Brownlee2017a}.
\end{enumerate}
To solve such issues, a long short-term memory (LSTM)  was introduced~\cite{Hochreiter1997} which is an memory extension for regular RNN.
LSTM addresses the problem of long-term dependencies.

LSTM is composed of four units: an input gate, a cell state, a forget gate, and a output gate.
\begin{figure}
	\begin{center}
		\includegraphics[width=1.0\linewidth]{figures/lstm.png}
	\end{center}
	\captionof{figure}{LSTM.}
	\label{lstm}
\end{figure}
The purpose of the forget gate is to figure out which information needs to be considered and which needs to be neglected.
The current input \(x_t\) and the previous hidden state \(h_{t-1}\) are passed through a sigmoid function which will produce values between \(0\) and \(1\) then the outputs of the sigmoid are multiplied with the previous cell state.
The input gate takes the current input \(x_t\) and the previous hidden state \(h_{t-1}\) and apply a sigmoid function over them in order to transform them to values in a range between \(0\) (not important) and \(1\) (important), then the same current input and the hidden state are passed through a \(tanh\) function which will regulate the network by transferring the values into a range between \(-1\) and \(1\).
Then, the outputs from the sigmoid and \(tanh\) functions are multiplied point-by-point in order to eliminate \(0\) values.
At this point, the network has sufficient information obtained from the input and forget gates.
Therefore, the current cell state \(c_t\) can be calculated through multiplying the previous cell state \(c_{t-1}\) with the output of the forget gate (all 0 values will be dropped) and the result is added to the calculated input values.
Afterward, the output gate computes the next hidden state \(h_t\) that holds information belongs to the current inputs.
Initially, the current input \(x_t\) and the previous hidden state \(h_{t-1}\) are passed through a third sigmoid function which will produce values between 
\(0\) and \(1\), and the current cell state \(c_t\) is passed though a \(tanh\) function.
Then the calculated values from the third sigmoid function and the \(tanh\) function are multiplied point-by-point.
Finally, the computed hidden state \(h_t\) is used for the prediction and it is transferred to the next timestamp.
%%%%%%%%%%%%%%%% Section 4 %%%%%%%%%%%%%%%%%%%%%%%
\section{Guided waves Based SHM through DL}
The guided waves approach is widely utilised in SHM/NDT, due to the fact it is able to detect very small damage sizes ~\cite{Guemes2020}. 
Damage detection and identification approaches using guided waves are based on the measurements of the PZT sensors whether bonded or embedded into the investigated structure. 
In which, PZT sensor(s) responsible for the excitation of the structure by a short ultrasonic pulse (usually, the used frequency is in the range of a hundreds of kHz) that propagates through an investigated structure such as plates or pipes as an elastic wave.
The registered signals (baseline) are stored and compared with other registered signals acquired through the lifetime of the investigated structure.
Damage detection using the baseline subtraction approach for guided waves is based on subtracting damage-free registered measurements from the newly registered measurements to obtain the new changes that occurred to the structure.
These changes are considered as damage information.
The baseline approach is effective in controlled environments where the variations of the operational/environments (i.e. considerations of multiple sensing modalities, uncertainty in material properties, bounding conditions, etc ) are negligible ~\cite{Yuan2020}.  
Such variations can alter registered data leading to false alarms.
The effect of such variations can be reduced through physics-based modeling, which can simulate an undamaged scenario (baseline) for the wave propagation through the investigated structure.
Then, the simulated baseline can be used in the subtraction for damage detection.
However, for real-world structures, it is difficult to tune the model parameter to match the experimental registered data.
Accordingly, data-driven techniques based on ML and DL approaches can be the solution and deliver a robust models for many real-life variations.

In the following, several guided wave for SHM/NDT based on data-driven techniques for damage detection and localisation are presented.
Authors in~\cite{Melville1949} proposed a CNN model for the prediction of damage state in thin metal plates to overcome the issue of inaccurate representation of guided wave propagation when applying conventional approaches. 
The model utilizes the full wavefield scans of thin plates (aluminum).
Moreover, the acquired raw data used for training the model was divided into undamaged and damaged states equally.
The model achieved higher accuracy regarding damage  \(99.98\%\) when compared to SVM that achieved \(62\%\).
Authors in~\cite{Sammons2016} proposed a CNN model based on X-ray computed tomography for delamination estimation in a composite structure.
Furthermore, image segmentation was applied to the input images to identify the damage.
However, the model was only able to identify small delaminations.
Moreover, Chetwynd et al. ~\cite{Chetwynd2008} presented a multi-layer perceptron (MLP) network for damage detection in curved composite panels, in which, stiffeners were added to represent the damage.
The Authors in this work investigated the propagation of Lamb waves through the panel in which they were generated and registered by a PZT array.
Furthermore, for each Lamb wave response, a novelty index was obtained.
The index value is compared to some threshold value, in which if the index value exceeds the threshold it implies that there is damage in the structure.
Accordingly, the MLP network was fed by obtained novelty indexes, and performed two operations: classification and regression.
The classification network was designed to define three convex regions of the panel then to determine whether the panel is damaged or not.
On the other hand, the regression network is capable of estimating the exact location of the damage.
Furthermore, authors in~\cite{DeFenza2015} proposed an artificial neural network (ANN) model for damage detection in plates made of aluminum alloys and composite utilising Lamb waves.
Response data of wave propagation were used to calculate damage indexes which were fed into the model as an input.
Accordingly, the model performs automatic feature extraction in conjunction with the probability ellipse-based method. 
The ANN model and probability ellipse (PE) method were applied to identify damage location.
The results from the ANN model and the PE presents how it is useful to apply damage indexes as a baseline for such methods in order to evaluate damage in aluminum and composite structures. 
Ewald et al.~\cite{Ewald2019} present a CNN model called (DeepSHM) for signal classification using Lamb waves.
Furthermore, the model provides an end-to-end approach for SHM by utilising response signals captured by sensors.
Moreover, response signals were preprocessed by wavelet transform to get the wavelet coefficient matrix (WCM).
Further, the CNN model was trained with the WCM to obtain neural weights.
Authors in~\cite{Ijjeh2021} presented a fully convolutional network (FCN)  for damage identification in composite plates base on a supervised learning approach.
Furthermore, the authors utilised a full wavefield of Lamb waves propagation, which was numerically generated resembling measurements acquired by scanning laser Dopler vibrometer (SLDV).
The model performs a pixel-wise segmentation that is able to identify the delamination which results in damaged and undamaged classes.
Moreover, the model results were validated through a comparison with a conventional wavefield signal processing method i.e. adaptive wavenumber filtering~\cite{Radzienski2019,Kudela2018}.
The proposed model achieved an accuracy of \(93.3\%\) in damage detection on numerical data compared to  \(64.8\%\) with the conventional method.
Furthermore, the proposed model was verified on experimental data and it proved its ability for generalisation.


%%%%%%%%%%%%%%%% Section 5 %%%%%%%%%%%%%%%%%%%%%%%
\section{Vibration based SHM though DL}
The vibration-based approach for damage assessment using ML techniques has been investigated thoroughly  for several SHM applications.
Furthermore, introducing DL techniques for data-driven SHM applications has presented new scopes for investigating large scale structures and enhanced the process of data acquisition and processing of large datasets acquired by sensors of different types~\cite{Carden2004,Sohn1996}.
Generally, the conventional approach for damage localisation requires prior knowledge of the approximate damage locations~\cite{Xu2018,Dorafshan2016}. 
Therefore, the identification process regarding candidates for the damaged locations is complex and can consume plenty of time.
Damage locations identification under the vibrational approach is based on the fact that the damage cause changes in the vibration characteristics such as modal shapes, frequencies, and damping~\cite{Doebling},
which can be utilised in the identification of damaged locations from the registered data response of a structure.
A vibration-based approach can be categorised into two classes:
model-based (parametric) and non-model-based or (non-parametric).
Parametric methods require computational models and associated assumptions about the investigated structure.
In general parametric methods can achieve good accuracy, however, there is no guarantee regarding the availability of accurate information about the structural system in the real-world~\cite{Azimi2020}. 
As a result, the non-parametric methods arise due to the challenges in developing robust computational models. 
With non-parametric methods, there are no prior assumptions about the structural system.
%ML techniques contributed to both of these categories. 
%They are usually used to extract the modal parameters in the scope of non-model-based methods [43,44]. 
%The traditional ML methods involve two phases in non-model-based methods. 
%The first phase is feature extraction, in which sensor data (e.g., acceleration) are used to extract effective features, thereby eliminating the cumbersome manual feature extraction process. 
%The second phase is a classification procedure that identifies the location
%and/or level of damage [45].  
%Support Vector Machines (SVM), Probabilistic ANNs (PNN) [46–48], Fuzzy ANNs (FNN) [49], and Extreme Learning Machine Networks (e.g., online sequential) [50] are some of the popular methods that are used for vibration-based SHM.

In the following, several vibration-based for SHM using DL techniques are presented.
Authors in~\cite{Abdeljaber2017} introduced a damage identification approach based on output-only response data.
In which, various damage cases (loose bolt) were investigated, accordingly training data were generated based on the acceleration response.
Authors in this approach have trained several CNNs separately regarding each damage case, and accordingly, the probability of damage (PoD) was determined.
By investigating scenarios of undamaged, single damage and multiple damage cases, they obtained \(0.54\%\) average error for specifically identified cases.

Authors in~\cite{Lin2017} introduced a new approach to structural damage detection using CNN.
Moreover, the authors have simulated a numerical model of simply supported Euler Bernoulli beam.
The detection model was designed to learn features and to identify damaged locations, moreover, it led to excellent results regarding the accuracy of damaged locations on the noise-free and noisy dataset.
Wang and Cha in~\cite{Cha2018} proposed an unsupervised CNN model, that is able to extract the feature representations from the unlabelled data.
The authors in their model used raw acceleration signals (sensitive to the damage presence) that were acquired from an intact lab-scale steel bridge.
Then, the acquired response vector was normalised followed by applying the continuous wavelet transform (CWT) and fast Fourier transform (FFT).
The output is then fed into a CNN auto-encoder,
Accordingly, the extracted damage features were fed into one-class (OC) SVMs as novelty detectors corresponding to the sensors.
Consequently, the approximation of damage location (loose-bolt) is estimated based on the locations of the sensors with the highest novelty rates.

Motivated by human vision and thinking, authors in~\cite{Cha2018} presented a computer vision and deep-learning framework for anomaly detection.
The proposed approach consists of two steps.
In the first step, data conversion by data visualisation is carried out, in which it mimics human vision and thinking.
In data visualisation,  the registered data response of acceleration is transformed into images plotted in gray-scale. 
In the second step, the training dataset is labeled manually, then fed into deep convolutional neural networks (DCNNs).
The proposed technique was tested on one-year data and achieved a global accuracy of \(87,0\%\) and it could be used for real-time SHM.
Moreover, Tang et al. in~\cite{Tang2019} presented a DL technique for data anomaly detection which can be considered as an improved technique to the previous work in~\cite{Cha2018}.
Initially, the raw time series measured data are split into segments, and data in the time and frequency domain are visualised. 
Images related to each section are stacked as a single dual-channel (red and green).
Then, the training dataset is fed into a CNN that learns how to perform data anomaly classification.
The main difference between the previous approach and this approach was in using imbalanced data in which the number of samples of different classes was unequal, however, in this approach the used data were balanced.
Finally, the comparison shows that this approach outperformed the previous one and achieved higher accuracy for all data anomaly patterns.

Authors in~\cite{Wu2019} presented a study of the deep CNN method in estimating the dynamic response of a linear single-degree-of-freedom (SDOF) system, a nonlinear SDOF, and a multidegree of freedom (MDOF) streel frame.
In some cases, the convolutional kernel can approximate the numerical integration operator, and the convolutional layer can be interpreted as a dominant frequency extraction operator.
Moreover, different cases of noise-contaminated signals were investigated. 
Additionally, the multilayer perceptron (MLP) method was used as a reference to the proposed CNN approach.
A comparison between the results obtained by the MLP and CNN shows that the CNN approach is more accurate and robust against noisy input data.

Authors in ~\cite{Oh2019}  presented a study of the CNN technique for SHM application for response estimation of tall buildings under wind excitation.
The proposed CNN model was trained on measured structural response data which take wind data measured as inputs in order to predict strains in future wind loads.
In order to measure the performance of the proposed technique, it was verified with unseen data never used at the training phase and it was able to accurately estimate the maximum and minimum strains.
Authors in~\cite{Li2020} proposed a CNN model for damage detection of a bridge structure.
Moreover, the authors compared the performance of the CNN model with other techniques such as random forest, SVM, KNN, and decision tree, and the results showed that the accuracy was enhanced by at least \(15\%\).
Since the acceleration response signal is highly prone to noise~\cite{Azimi2020}, researchers begin utilising other data sensor types or use alternative features.  

Li et al. in ~\cite{Li2020a} investigated damage in bridge structure accordingly, proposed a supervised learning technique based on the CNN model.
Dataset was acquired by deflection of a scaled-down model bridge by a fibre-optic gyroscope.
Then, the dataset was fed into a 1D-CNN model to classify three states of damage and an intact class (benchmark/damage-free).
To investigate the performance of the proposed model, a cross-validation technique was applied. 
It showed that the accuracy of the CNN model increased by at least \(15.3\%\) over other conventional methods such as SVM, KNN, decision trees, and random forests.
Authors in \cite{Lopez-Pacheco2020} introduced a novel frequency-domain convolutional neural network (FDCNN) for damage detection based on Bouc-Wen hysteric model~\cite{Ismail2009}.  
In the FDCNN method, only acceleration measurements for damage diagnosis, that are sensitive to environmental noise.
Moreover, FDCNN reduces the computational time during the learning process, which adds to increase robustness against noise.
The FDCNN introduced the spectral pooling operator responsible for attenuating the noise in measurements.
The proposed method was validated through comparing it with different CNN model. 
The performance of the proposed method was higher regarding damage identification in building structures.

Finally, with smart monitoring as a target, authors in~\cite{Hung2020}  proposed a hybrid deep learning model for damage detection for SHM.
The proposed model can deal with different damage levels and accurately detect damage by combining 1D-CNN and Long-Short Term Memory (LSTM) into a single end-to-end model fed by the raw time-series, and as a result, avoiding signal preprocessing step.
Moreover, the proposed model verified that with low noise levels,  accurate damage detection can be achieved.



%%%%%%%%%%%%%%%%%%%%%%%%%%%%%%%%%%%%%%%%%%%%%%%%%%
\section{Summary}
In this chapter, the author discussed problems with conventional damage detection techniques for SHM and the importance of the artificial intelligence approach.
Furthermore, in the second section of the chapter, the author introduced the ML approach in the SHM field.
Moreover, several techniques for feature extraction such as PCA, MSD, and GMMs were described. 
Further, several classification models such as SVM, KNN, and decision trees were introduced.
In the third section, the author presents the deep learning approach, in which techniques such as CNN  and RNN were presented.
Finally, the author presents several deep learning techniques for damage detection used regarding the SHM field based on guided waves and vibration approaches.
\section*{Acknowledgments}
This work was funded by the Polish National Science Center under grant agreement no 2018/31/B/ST8/00454.
Also, I would like to thank my supervisor Professor Pawel Kudela for his consistent support and guidance during the running of this project.
%%%%%%%%%%%%%%%%%%%%%%%%%%%%%%%%%%%%%%%%%%%%%%%%%%
\bibliography{biblography.bib} 
\bibliographystyle{unsrt} 
%%%%%%%%%%%%%%%%%%%%%%%%%%%%%%%%%%%%%%%%%%%%%%%%%%
\end{document}
%%%%%%%%%%%%%%%%%%%%%%%%%%%%%%%%%%%%%%%%%%%%%%%%%%
