\section{Guided waves Based SHM through DL}
The guided waves approach is widely utilised in SHM/NDT, due to the fact it is able to detect very small damage sizes ~\cite{Guemes2020}. 
Damage detection and identification approaches using guided waves are based on the measurements of the PZT sensors whether bonded or embedded into the investigated structure. 
In which, PZT sensor(s) responsible for the excitation of the structure by a short ultrasonic pulse (usually, the used frequency is in the range of a hundreds of kHz) that propagates through an investigated structure such as plates or pipes as an elastic wave.
The registered signals (baseline) are stored and compared with other registered signals acquired through the lifetime of the investigated structure.
Damage detection using the baseline subtraction approach for guided waves is based on subtracting damage-free registered measurements from the newly registered measurements to obtain the new changes that occurred to the structure.
These changes are considered as damage information.
The baseline approach is effective in controlled environments where the variations of the operational/environments (i.e. considerations of multiple sensing modalities, uncertainty in material properties, bounding conditions, etc ) are negligible ~\cite{Yuan2020}.  
Such variations can alter registered data leading to false alarms.
The effect of such variations can be reduced through physics-based modeling, which can simulate an undamaged scenario (baseline) for the wave propagation through the investigated structure.
Then, the simulated baseline can be used in the subtraction for damage detection.
However, for real-world structures, it is difficult to tune the model parameter to match the experimental registered data.
Accordingly, data-driven techniques based on ML and DL approaches can be the solution and deliver a robust models for many real-life variations.

In the following, several guided wave for SHM/NDT based on data-driven techniques for damage detection and localisation are presented.
Authors in~\cite{Melville1949} proposed a CNN model for the prediction of damage state in thin metal plates to overcome the issue of inaccurate representation of guided wave propagation when applying conventional approaches. 
The model utilizes the full wavefield scans of thin plates (aluminum).
Moreover, the acquired raw data used for training the model was divided into undamaged and damaged states equally.
The model achieved higher accuracy regarding damage  \(99.98\%\) when compared to SVM that achieved \(62\%\).
Authors in~\cite{Sammons2016} proposed a CNN model based on X-ray computed tomography for delamination estimation in a composite structure.
Furthermore, image segmentation was applied to the input images to identify the damage.
However, the model was only able to identify small delaminations.
Moreover, Chetwynd et al. ~\cite{Chetwynd2008} presented a multi-layer perceptron (MLP) network for damage detection in curved composite panels, in which, stiffeners were added to represent the damage.
The Authors in this work investigated the propagation of Lamb waves through the panel in which they were generated and registered by a PZT array.
Furthermore, for each Lamb wave response, a novelty index was obtained.
The index value is compared to some threshold value, in which if the index value exceeds the threshold it implies that there is damage in the structure.
Accordingly, the MLP network was fed by obtained novelty indexes, and performed two operations: classification and regression.
The classification network was designed to define three convex regions of the panel then to determine whether the panel is damaged or not.
On the other hand, the regression network is capable of estimating the exact location of the damage.
Furthermore, authors in~\cite{DeFenza2015} proposed an artificial neural network (ANN) model for damage detection in plates made of aluminum alloys and composite utilising Lamb waves.
Response data of wave propagation were used to calculate damage indexes which were fed into the model as an input.
Accordingly, the model performs automatic feature extraction in conjunction with the probability ellipse-based method. 
The ANN model and probability ellipse (PE) method were applied to identify damage location.
The results from the ANN model and the PE presents how it is useful to apply damage indexes as a baseline for such methods in order to evaluate damage in aluminum and composite structures. 
Ewald et al.~\cite{Ewald2019} present a CNN model called (DeepSHM) for signal classification using Lamb waves.
Furthermore, the model provides an end-to-end approach for SHM by utilising response signals captured by sensors.
Moreover, response signals were preprocessed by wavelet transform to get the wavelet coefficient matrix (WCM).
Further, the CNN model was trained with the WCM to obtain neural weights.
Authors in~\cite{Ijjeh2021} presented a fully convolutional network (FCN)  for damage identification in composite plates base on a supervised learning approach.
Furthermore, the authors utilised a full wavefield of Lamb waves propagation, which was numerically generated resembling measurements acquired by scanning laser Dopler vibrometer (SLDV).
The model performs a pixel-wise segmentation that is able to identify the delamination which results in damaged and undamaged classes.
Moreover, the model results were validated through a comparison with a conventional wavefield signal processing method i.e. adaptive wavenumber filtering~\cite{Radzienski2019,Kudela2018}.
The proposed model achieved an accuracy of \(93.3\%\) in damage detection on numerical data compared to  \(64.8\%\) with the conventional method.
Furthermore, the proposed model was verified on experimental data and it proved its ability for generalisation.

