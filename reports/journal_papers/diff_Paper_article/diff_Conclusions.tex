%%%%%%%%%%%%%%%%%%%%%%%%%%%%%%%%%%%%%%%%%%%%%%%%%%
\section{Conclusions}
%%%%%%%%%%%%%%%%%%%%%%%%%%%%%%%%%%%%%%%%%%%%%%%%%%
In this paper, we addressed delamination detection in composite materials using a deep learning technique. 
For this purpose, we have trained an FCN-DenseNet for semantic segmentation on a numerically generated data -- simulated full wavefield of propagating guided waves.
To see the feasibility of such a study, we have compared the deep learning model with adaptive wavenumber filtering technique.
The results were promising, and the deep learning model surpasses the conventional technique in detecting the delaminations of different shapes, sizes and angles. 
Further, the model can be improved by training it on new experimental data.
It means that new patterns will be learned, hence it will enhance its ability to differentiate among various complex patterns.
Currently, our work focuses on delamination identification, however, it can be extended to the identification of different types of damage in composite materials.

\DIFaddbegin \DIFadd{It should be underlined that the proposed approach is more suitable for NDT than SHM because measurements by using SLDV are conducted rather stationary and are time consuming.
However, as laser technology progresses, it is expected that acquisition will be possible in an array of points instead of single point, significantly shortening the measurement process.
}

\DIFaddend In future, we are planning to implement several deep learning architectures to perform a comparative study of various deep learning models regarding delamination identification in composite materials\DIFaddbegin \DIFadd{.
}

\DIFadd{Since the proposed approach has proven to be feasible, the next step would be gathering large experimental data set consisting of full wavefields not only from simple plate-like structures but also from structures with stiffeners and rivets.
We believe that such data would greatly enhance the performance of the proposed models in comparison to naive numerical data set.
Nevertheless, it requires tremendous effort which is over the capabilities of one team.
We are planning to work in this direction in the future and hope that other scientists in the field of SHM/NDT will contribute and share data sets so that stimulate progress in the field}\DIFaddend .