\section{Introduction}
%%%%%%%%%%%%%%%%%%%%%%%%%%%%%%%%%%%%%%%%%%%%%%%%%%%%%%%%%%%%%%%%%%%%%%%%%%%%%%%%%%%%%%%%

Detecting delamination in composite materials poses a significant challenge for conventional visual inspection techniques as it occurs between plies of the composite laminate and remains invisible from external surfaces~\cite{staszewski2009health, tuo2019damage}. 
As a result, various nondestructive testing (NDT) and structural health 
monitoring (SHM) techniques have been proposed for delamination identification 
in composite structures. 
Among these techniques, ultrasonic guided waves are widely recognized as one of 
the most promising approaches for quantitatively identifying defects in 
composites~\cite{tian2015delamination, munian2018lamb}. 
Their extensive application is attributed to their high sensitivity to small defects, low propagation attenuation, and ability to monitor large areas using only a small number of sparsely distributed transducers~\cite{Barthorpe2020, Ihn2008, Cantero-Chinchilla2020}.

However, using a smaller number of transducers does not allow to obtain 
high-quality resolution damage maps and accurately asses the size of 
experienced damage. 
Conversely, employing a very dense array of transducers is impractical in most 
situations. To address these issues, a scanning laser Doppler vibrometer (SLDV) 
can be employed. 
SLDV can measure the propagation of guided waves in a highly dense grid of points over the surface of a large specimen, collectively known as a full wavefield~\cite{Radzienski2019a}. 
In recent years, full wavefield signals have been utilized for detecting and localizing defects in composite structures~\cite{Radzienski2019a, Girolamo2018a, kudela2018impact, rogge2013characterization}. 
These damage identification techniques using full wavefield signals can 
effectively estimate not only the location but also the size of the 
damage~\cite{Girolamo2018a, kudela2018impact}. 
Full wavefield scans offer valuable insights into the interaction of guided waves with defects. However, acquiring the full wavefield of guided waves is a time-consuming process.

One possible solution to address this problem involves obtaining Lamb waves in 
a low-resolution format and subsequently applying compressive sensing (CS) 
eventually enhanced by deep learning (DL) methods~\cite{esfandabadideep}, such 
as end-to-end DL-based super-resolution technique~\cite{ijjeh2023deep}. 

Nevertheless, the damage identification methods operating on the full wavefield cannot be directly extended to the cases when only spatially sparse data is available, e.g. an array of sensors is installed on the structure.
In such a case, usually, an inverse procedure is employed in which efficient methods for solving wave equations are required (forward solver).  
However, numerical modelling of ultrasonic guided wave propagation in solid media exhibiting discontinuities, such as damage, is complex, requires fine discretization and is computationally intensive.
Even methods such as p-version of the finite element method (p-FEM)~\cite{Duczek2013}, the iso-geometric analysis (IGA)~\cite{Anitescu2019}, the spectral cell method (SCM)~\cite{Mossaiby2019} or the time-domain spectral element method (SEM)~\cite{Ostachowicz2012} are not efficient enough.
In the end, calling the objective function for each damage case scenario in which the forward solver is used, is unfeasible.

An alternative approach is to utilize a DL-based surrogate model for generating full wavefield data or time series resembling signals registered at an array of sensors. 
A surrogate model imitates the behavior of the simulation model while replacing time-consuming forward simulations with approximate solutions.

DL has seen extensive research and successful implementation in the fields of nondestructive testing (NDT) and structural health monitoring (SHM). 
Convolutional neural network (CNN) is among the most popular DL architectures. 
Initially introduced for image processing, CNN has now been extended to various research domains, including NDT and SHM applications such as damage detection, localization, and characterization~\cite{rautela2019deep, pandey2022explainable, ijjeh2021full, ijjeh2022deep}.

In modern DL, the focus is on constructing more efficient and streamlined architectures, wherein key components are accentuated and essential attributes from the original data are preserved. 
This process of crafting architectures involves the meticulous extraction of feature representations from data, which is commonly referred to as feature engineering. 
However, this practice demands specialized knowledge and a significant investment of time. 
Moreover, the intricacies of feature engineering differ across various data types, posing a challenge in establishing universally applicable procedures.

An established option for feature extraction, alleviating feature engineering, 
is delivered in the form of an autoencoder, a type of neural network that 
possesses the capability to autonomously learn features from unlabeled data 
using unsupervised learning techniques. 
This unique ability obviates the need for extensive feature 
engineering~\cite{pinaya2020autoencoders, ardelean2023study, 
simpson2021machine}. 
Consisting of two integral components, an autoencoder comprises an encoder responsible for mapping inputs to a desired latent space, and a decoder that skilfully reverts the latent space back to the original input domain. 
By harnessing appropriately curated training data, autoencoders have the capacity to generate a latent representation, thereby negating the necessity for labour-intensive feature engineering endeavours.

However, general autoencoders may not capture spatial features, such as images, or sequential information when dealing with dynamics, like time-series forecasting. 
To address the limitation of capturing spatial features, the use of a CNN-based autoencoder is recommended, whereas, an RNN-based autoencoder is usually employed for learning features from time-series data.
Deep CNN-based autoencoders (DCAEs) excel at extracting spatial features from images-based input data, they may not be sufficient for extracting features from sequences of images, particularly in cases involving full wavefield data, which contains numerous sequential frames/images for each delamination scenario. 
For such situations, ConvLSTM~\cite{shi2015convolutional} is employed. 
ConvLSTM combines CNN and LSTM, enabling it to effectively learn features from sequences of images. 
In ConvLSTM architecture, CNN is responsible for learning features from images, while LSTM retains sequential information.

DCAE based surrogate modelling has been implemented in~\cite{jo2021adaptive, nikolopoulos2022non, sharma2022wave}. 
Jo et al.~\cite{jo2021adaptive} developed a DCAE framework for the purpose of extracting latent features from spatial properties and investigating adaptive surrogate estimation to sequester $CO_2$ into heterogeneous deep saline aquifers. 
They used DCAE and a fully-convolutional network for reducing the computational costs and extracting dimensionality-reduced features for conserving spatial characteristics. 
Nikolopoulos et al.~\cite{nikolopoulos2022non} presented a non-intrusive DL-based surrogate modelling scheme for predictive modelling of complex systems, which they described by parametrized time-dependent partial differential equations. 
Sharma et al.~\cite{sharma2022wave} proposed a DCAE based surrogate predictive model for wave propagation. 
Their model is able to generate data for a given crack location and depth of the one-dimensional rod of isotropic material.

Recently, Peng et al.~\cite{peng2021structural} proposed an encoding convolution long short-term memory (encoding ConvLSTM) framework for building a surrogate structural model with spatiotemporal evolution, estimating structural spatio-temporal states, and predicting dynamic responses under future dynamic load conditions. 
Zargar and Yuan~\cite{zargar2021impact} presented a hybrid CNN-recurrent neural network (RNN) for handling spatiotemporal information extraction challenges in impact damage detection problems.

In this work, a novel approach is employed to investigate the propagation of guided waves in composite structures with varying instances of delamination. 
The method involves utilising a deep ConvLSTM autoencoder-based surrogate model. 
The main function of the developed model is to capture and learn the full wavefield representation present within frames containing delamination scenarios. 
Subsequently, it transforms this information into a compressed domain known as the latent space.

During training, the encoder is presented with inputs containing both reference frames (without delaminations) and explicit data detailing the shape and location of the delaminations. 
This process eliminates the need to repeatedly solve the system's governing equations, resulting in significant time and computational cost savings in comparison to forward modelling by using conventional techniques.

The novelty of this research lies in the implementation of ConvLSTM-based autoencoders for the generation of full wavefield data of propagating guided waves in composite structures.
The DL model for full wavefield prediction was applied for the first time in the inverse problem of damage identification.
For this purpose, particle swarm optimisation (PSO) was applied~\cite{Keneddy1995}.
%%%%%%%%%%%%%%%%%%%%%%%%%%%%%%%%%%%%%%%%%%%%%%%%%%%%%%%%%%%%%%%%%%%%%%%%%%%%%%%%%%%%%%%%
\section{General concept}
%%%%%%%%%%%%%%%%%%%%%%%%%%%%%%%%%%%%%%%%%%%%%%%%%%%%%%%%%%%%%%%%%%%%%%%%%%%%%%%%%%%%%%%%
A framework of the proposed inverse method for damage identification is shown in Fig~\ref{fig:complete_flowchart}.
It is composed of three building blocks: (i) dataset computation, (ii) supervised learning and (iii) inverse method.

%%%%%%%%%%%%%%%%%%%%%%%%%%%%%%%%%%%%%%%%%%%%%%%%%%%%%%%%%%%%%%%%%%%%%%%%%%%%%%%%
\begin{figure} [h!]
	\begin{center}
		\includegraphics[width=12cm]{figure1.png}
	\end{center}
	\caption{Flowchart of the proposed inverse method for damage identification.} 
	\label{fig:complete_flowchart}
\end{figure}
%%%%%%%%%%%%%%%%%%%%%%%%%%%%%%%%%%%%%%%%%%%%%%%%%%%%%%%%%%%%%%%%%%%%%%%%%%%%%%%%

For the problem of guided wave-based damage identification, it is infeasible to 
collect a large experimental dataset which would cover various damage case 
scenarios because it would require manufacturing and damaging multiple samples.
Therefore, a possible alternative is a dataset computed numerically by using 
e.g. a finite element solver.
In particular, the dataset which is employed in the current research was generated by using the time domain spectral element method~\cite{Kudela2020}.
The dataset focuses on delamination defects as it these represent the most 
dangerous types of damage occurring in composite laminates such as carbon 
fibre-reinforced polymers (CFRP).
The process of generating this dataset involved modelling a square plate with delaminations of varying shapes and positions into the parametric unstructured mesh. 
Following this, a forward solver was utilised to capture the interactions between the propagating guided waves with the delamination and the boundaries of the plate.
It resulted in 512 frames for each delamination scenario.
The dataset is made available online~\cite{kudela_pawel_2021_5414555} so that 
the results can be easily replicated.

The dataset was used for the supervised training of the DL model.
The idea was to input to the DL model a binary image in which ones (white pixels) represent  the delamination region and zeros (black pixels) represent the healthy region for the respective delamination cases.
Once the model is trained, it can be used in the inverse procedure as a surrogate model instead of a computationally expensive p-FEM or SEM forward solver.

It should be noted that the PSO algorithm was used in the inverse method due to 
its efficiency in handling more general formulations of objective functions (as 
opposed to stricter algebraic constructs), but other algorithms can be used as 
well. 
On one hand, inputs are initial particles represented by binary images in Fig~\ref{fig:complete_flowchart}.
These are fed to the trained DL model which in turn is predicting the full wavefield of propagating waves for respective delamination scenarios.
On the other hand, full wavefield measurements are acquired. 
An objective function is built upon the computation of the mean square error (MSE) between the predicted wavefield and the measured wavefield.
Next, particle positions and their velocities are updated accordingly until the termination criterion is met.
Finally, the identified delamination is visualized for the best match.
It should be added that the proposed method was validated on synthetic data only. 

The next sections describe each building block of the proposed method in detail.

