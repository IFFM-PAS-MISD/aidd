\section{Introduction:}
%%%%%%%%%%%%%%%%%%%%%%%%%%%%%%%%%%%%%%%%%%%%%%%%%%%%%%%%%%%%%%%%%%%%%%%%%%%%%%%%%%%%%%%%

Detecting delamination in composite materials poses a significant challenge for 
conventional visual inspection techniques as it occurs between plies of the 
composite laminate and remains invisible from external 
surfaces~\cite{staszewski2009health, tuo2019damage}. As a result, various 
nondestructive testing (NDT) and structural health monitoring (SHM) techniques 
have been proposed for delamination identification in composite structures. 
Among these techniques, ultrasonic guided waves are widely recognized as one of 
the most promising approaches for quantitatively identifying defects in 
composites. Their extensive application is attributed to their high sensitivity 
to small defects, low propagation attenuation, and ability to monitor large 
areas using only a small number of sparsely distributed 
transducers~\cite{Barthorpe2020, Ihn2008, Cantero-Chinchilla2020}.

However, using a smaller number of transducers is inadequate for obtaining 
high-quality resolution damage maps and assessing the size of the damage 
accurately. Conversely, employing a very dense array of transducers is 
impractical in most situations. To address these issues, a scanning laser 
Doppler vibrometer (SLDV) is employed. SLDV can measure the propagation of 
guided waves in a highly dense grid of points over the surface of a large 
specimen, collectively known as full wavefield~\cite{Radzienski2019a}. In 
recent years, full wavefield signals have been utilized for detecting and 
localizing defects in composite structures~\cite{Radzienski2019a, 
	Girolamo2018a, kudela2018impact, rogge2013characterization}. These damage 
identification techniques using full wavefield signals can effectively estimate 
not only the location but also the size of the damage~\cite{Girolamo2018a, 
	kudela2018impact}. Full wavefields offer valuable insights into the 
	interaction 
of guided Lamb waves with potential defects. However, acquiring the full 
wavefield of guided waves is a time-consuming process.


One possible solution to address this problem involves obtaining Lamb waves in 
a low-resolution format and subsequently applying a compressive sensing (CS) or 
deep learning-based super-resolution technique to enhance the low-resolution 
full wavefield data~\cite{ijjeh2023deep}. Another approach is to utilize a deep 
learning-based surrogate model for generating full wavefield data. A surrogate 
model imitates the behavior of the simulation model while replacing 
time-consuming forward simulations with approximate solutions.

Deep learning has seen extensive research and successful implementation in the 
fields of nondestructive testing (NDT) and structural health monitoring (SHM). 
Convolutional neural network (CNN) is among the most popular deep learning 
architectures. Initially introduced for image processing, CNN has now been 
extended to various research domains, including NDT and SHM applications such 
as damage detection, localization, and characterization~\cite{rautela2019deep, 
	pandey2022explainable, ijjeh2021full, ijjeh2022deep}.

In modern deep learning, the focus is on creating more efficient and simpler 
architectures by emphasizing the most significant elements and preserving 
relevant features of the original data. Feature engineering, the process of 
extracting feature representations from data, requires specific knowledge and 
is time-consuming. Additionally, it varies for different types of data, making 
it challenging to have generic procedures. An autoencoder (AE) is a neural 
network capable of automatically learning features from unlabeled data in an 
unsupervised way, eliminating the need for extensive feature 
engineering~\cite{pinaya2020autoencoders}. 
An AE consists of two parts, the encoder, which maps inputs to a desired latent 
space, and the decoder, which decodes the latent space back into the original 
input space. By using adequate training data, AEs can create a latent 
representation, thus avoiding the need for substantial feature engineering.

However, general AEs may not capture spatial features, such as images, or 
sequential information when dealing with dynamics, like time-series 
forecasting. To address the limitation of capturing spatial feature, the use of 
a CNN-based autoencoder is recommended, whereas, an RNN-based autoencoder is 
usually employed for learning features from time-series data.
Deep CNN-based autoencoders (DCAEs) excel at extracting spatial features from 
images-based input data, they may not be sufficient for extracting features 
from sequences of images, particularly in cases involving full wavefield data, 
which contains numerous sequential frames/images for each delamination 
scenario. 
For such situations, ConvLSTM~\cite{shi2015convolutional} is 
employed. ConvLSTM combines CNN and LSTM, enabling it to effectively learn 
features from sequences of images. In ConvLSTM architecture, CNN is responsible 
for learning features from images, while LSTM retains sequential information.

DCAE based surrogate modeling has been implemented in~\cite{jo2021adaptive, 
	nikolopoulos2022non, sharma2022wave}. 
Jo et al.~\cite{jo2021adaptive} developed a DCAE framework for the purpose of  
extracting latent features from spatial properties and investigating adaptive 
surrogate estimation to sequester $CO_2$ into heterogeneous deep saline 
aquifers. 
They used DCAE and a fully-convolutional network for reducing the
computational costs and extracting dimensionality-reduced features for  
conserving spatial characteristics. 
Nikolopoulos et al.~\cite{nikolopoulos2022non} presented a non-intrusive 
deep learning based surrogate modeling scheme for predictive modeling of 
complex systems, which they described by parametrized time-dependent partial 
differential equations. 
Sharma et al.~\cite{sharma2022wave} proposed a DCAE based surrogate predictive 
model for wave propagation. 
Their model is able to generate data for a given crack location and depth
of one-dimensional rod of isotropic material.

Recently, Peng et al.~\cite{peng2021structural} proposed an encoding 
convolution long short-term memory (encoding ConvLSTM) framework for building a 
surrogate structural model with spatiotemporal evolution, estimating structural 
spatiotemporal states, and predicting dynamic responses under future dynamic 
load conditions. Zargar and Yuan~\cite{zargar2021impact} presented a hybrid 
CNN-recurrent neural network (RNN) for handling spatiotemporal information 
extraction challenges in impact damage detection problems.

In this research work, a deep ConvLSTM autoencoder-based surrogate model is 
applied to composite structures with various delamination cases. The model 
first learns the full wavefield representation from frames containing 
delaminations and project it to a compressed domain known as the latent space. 
This latent space is used as labels for the encoder part of the training, where 
input frames without delaminations and ground truth (delamination information) 
data for the respective delamination cases are provided.
This process eliminates the need to repeatedly solve the system's governing 
equations, resulting in significant time and computational cost savings, making 
it particularly suitable for issues requiring repeated model computations. 
The novelty of this research lies in the implementation of ConvLSTM-based 
autoencoders for the generation of full wavefield data in composite 
structures by providing the delamination size and location.
A complete working procedure of the proposed system is shown in 
Fig~\ref{fig:complete_flowchart}.
%%%%%%%%%%%%%%%%%%%%%%%%%%%%%%%%%%%%%%%%%%%%%%%%%%%%%%%%%%%%%%%%%%%%%%%%%%%%%%%%
\begin{figure} [h!]
	\begin{center}
		\includegraphics[width=12cm]{Graphics/figure1.png}
	\end{center}
	\caption{Flowchart of the complete working system of the proposed model.} 
	\label{fig:complete_flowchart}
\end{figure}
%%%%%%%%%%%%%%%%%%%%%%%%%%%%%%%%%%%%%%%%%%%%%%%%%%%%%%%%%%%%%%%%%%%%%%%%%%%%%%%%
