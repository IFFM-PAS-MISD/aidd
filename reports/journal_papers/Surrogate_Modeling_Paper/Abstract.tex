%%%%%%%%%%%%%%%%%%%%%%%%%%%%%%%%%%%%%%%%%%%%%%%%%%%%%%%%%%%%%%%%%%%%%%%%%%%%%%%%
In this work, a novel approach of guided wave-based damage identification in composite laminates is proposed. 
The novelty of this research lies in the implementation of ConvLSTM-based autoencoders for the generation of full wavefield data of propagating guided waves in composite structures.
The developed surrogate deep learning model takes as input full wavefield frames of propagating waves in a healthy plate along with a binary image representing delamination and predicts frames of propagating waves in a plate which contains single delamination.
The evaluation of the surrogate model is ultrafast.
Therefore, unlike traditional forward solvers, the surrogate model can be employed efficiently in the inverse framework of damage identification.
In this work, particle swarm optimisation is applied as a suitable tool to this 
end. 


The proposed method was tested on a synthetic dataset showing that it is 
capable to estimate delamination location and size with a good accuracy.
The test involved full wavefield data in the objective function of the inverse 
method but it should be underlined that also partial data with measurements can 
be implemented.
This is extremely important for practical applications in structural health 
monitoring where only signals at a finite number of locations are available.