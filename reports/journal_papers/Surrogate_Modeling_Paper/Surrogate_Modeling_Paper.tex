
%% 
%% Copyright 2007, 2008, 2009 Elsevier Ltd
%% 
%% This file is part of the 'Elsarticle Bundle'.
%% ---------------------------------------------
%% 
%% It may be distributed under the conditions of the LaTeX Project Public
%% License, either version 1.2 of this license or (at your option) any
%% later version.  The latest version of this license is in
%%    http://www.latex-project.org/lppl.txt
%% and version 1.2 or later is part of all distributions of LaTeX
%% version 1999/12/01 or later.
%% 
%% The list of all files belonging to the 'Elsarticle Bundle' is
%% given in the file `manifest.txt'.
%% 
%% Template article for Elsevier's document class `elsarticle'
%% with harvard style bibliographic references
%% SP 2008/03/01

\documentclass[preprint,9pt]{elsarticle}


%% Use the option review to obtain double line spacing
%\documentclass[authoryear,preprint,review,12pt]{elsarticle}
%\documentclass[preprint,review,12pt]{elsarticle}

%% Use the options 1p,twocolumn; 3p; 3p,twocolumn; 5p; or 5p,twocolumn
%% for a journal layout:
%% \documentclass[final,1p,times,authoryear]{elsarticle}
%% \documentclass[final,1p,times,twocolumn,authoryear]{elsarticle}
%% \documentclass[final,3p,times,authoryear]{elsarticle}
%%\documentclass[final,3p,times,twocolumn,authoryear]{elsarticle}
%% \documentclass[final,5p,times,authoryear]{elsarticle}
%% \documentclass[final,5p,times,twocolumn,authoryear]{elsarticle}

%% For including figures, graphicx.sty has been loaded in
%% elsarticle.cls. If you prefer to use the old commands
%% please give \usepackage{epsfig}

%% The amssymb package provides various useful mathematical symbols
\usepackage{amsmath,amssymb,bm}
%\usepackage[dvips,colorlinks=true,citecolor=green]{hyperref}
\usepackage[colorlinks=true,citecolor=green]{hyperref}
%% my added packages
\usepackage{float}
\usepackage{csquotes}
\usepackage{verbatim}
\usepackage{caption}
\usepackage{subcaption}
\usepackage{booktabs} % for nice tables
\usepackage{csvsimple} % for csv read
\usepackage{graphicx}
\usepackage{natbib}
\newcommand{\RNum}[1]{\uppercase\expandafter{\romannumeral #1\relax}}

%\usepackage{breqn}
\usepackage{multirow}
%\usepackage{cite}
%\usepackage[style=numeric-comp]{biblatex}

% matrix command 
\newcommand{\matr}[1]{\mathbf{#1}} % bold upright (Elsevier, Springer)
% vector command 
\newcommand{\vect}[1]{\mathbf{#1}} % bold upright (Elsevier, Springer)
\newcommand{\ud}{\mathrm{d}}
\renewcommand{\vec}[1]{\mathbf{#1}}
\newcommand{\veca}[2]{\mathbf{#1}{#2}}
\renewcommand{\bm}[1]{\mathbf{#1}}
\newcommand{\bs}[1]{\boldsymbol{#1}}
% limits underneath
\DeclareMathOperator*{\argmin}{arg\,min}
\DeclareMathOperator*{\argmax}{arg\,max}

\graphicspath{{Graphics/}{//pkudela_odroid_sensors/ALPHORN/surrogate_modelling_paper/Graphics/}}
%\graphicspath{{//pkudela_odroid_sensors/ALPHORN/surrogate_modelling_paper/Graphics/}}

\bibliographystyle{elsarticle-num}
%\bibliographystyle{unsrt}

%\graphicspath{ {Graphics/Figures/} }
%% The amsthm package provides extended theorem environments
%% \usepackage{amsthm}
%% The lineno packages adds line numbers. Start line numbering with
%% \begin{linenumbers}, end it with \end{linenumbers}. Or switch it on
%% for the whole article with \linenumbers.
%% \usepackage{lineno}
\journal{Engineering Applications of Artificial Intelligence}

\begin{document}
	\begin{frontmatter}
		\addcontentsline{toc}{section}{References}
		%% Title, authors and addresses
		%% use the tnoteref command within \title for footnotes;
		%% use the tnotetext command for theassociated footnote;
		%% use the fnref command within \author or \address for footnotes;
		%% use the fntext command for theassociated footnote;
		%% use the corref command within \author for corresponding author 
		%%footnotes;
		%% use the cortext command for theassociated footnote;
		%% use the ead command for the email address,
		%% and the form \ead[url] for the home page:
		%% \title{Title\tnoteref{label1}}
		%% \tnotetext[label1]{}
		%% \author{Name\corref{cor1}\fnref{label2}}
		%% \ead{email address}
		%% \ead[url]{home page}
		%% \fntext[label2]{}
		%% \cortext[cor1]{}
		%% \address{Address\fnref{label3}}
		%% \fntext[label3]{}
		\title{Simulation of full wavefield data with deep learning approach for delamination identification}
		
		%% use optional labels to link authors explicitly to addresses:
		%% \author[label1,label2]{}
		\address[IFFM]{Institute of Fluid Flow Machinery, Polish Academy of 
		Sciences, Poland}
		\address[ETH]{Department of Civil, Environmental, and Geomatic Engineering, ETH Z\"{u}rich, Switzerland}
		\author{Saeed Ullah\fnref{IFFM}}
		\author{Pawel Kudela\corref{cor1}\fnref{IFFM}}
		\ead{pk@imp.gda.pl}
		\author{Abdalraheem A. Ijjeh\fnref{IFFM}}
		\author{Eleni Chatzi \fnref{ETH}}
		\author{Wieslaw Ostachowicz \fnref{IFFM}}	
		\cortext[cor1]{Corresponding author}
		
		\begin{abstract}
			%%%%%%%%%%%%%%%%%%%%%%%%%%%%%%%%%%%%%%%%%%%%%%%%%%%%%%%%%%%%%%%%%%%%%%%%%%%%%%%%
In this work, a novel approach of guided wave-based damage identification in composite laminates is proposed. 
The novelty of this research lies in the implementation of Convolutional Long Short Term Memory (ConvLSTM)-based autoencoders for the generation of full wavefield data of propagating guided waves in composite structures.
The developed surrogate deep learning model takes as input full wavefield frames of propagating waves in a healthy plate along with a binary image representing delamination and predicts frames of propagating waves in a plate which contains single delamination.
The evaluation of the surrogate model is ultrafast.
Therefore, unlike traditional forward solvers, the surrogate model can be employed efficiently in the inverse framework of damage identification.
In this work, particle swarm optimisation is applied as a suitable tool to this 
end. 

The proposed method was tested on a synthetic dataset showing that it is capable to estimate delamination location and size with a good accuracy.
The test involved full wavefield data in the objective function of the inverse 
method but it should be underlined that also partial data with measurements can 
be implemented.
This is extremely important for practical applications in structural health 
monitoring where only signals at a finite number of locations are available.
		\end{abstract}
		
		\begin{keyword}
			%% keywords here, in the form: keyword \sep keyword
			Lamb waves \sep structural health monitoring \sep surrogate 
			modeling\sep delamination identification \sep deep learning 
			\sep  autoencoders \sep ConvLSTM 
			%% PACS codes here, in the form: \PACS code \sep code
			
			%% MSC codes here, in the form: \MSC code \sep code
			%% or \MSC[2008] code \sep code (2000 is the default)
			
		\end{keyword}
	\end{frontmatter}
	%%%%%%%%%%%%%%%%%%%%%%%%%%%%%%%%%%%%%%%%%%%%%%%%%%%%%
	%%%%%%%%%%%%%%%%%%%%%%%%%%%%%%%%%%%%%%%%%%%%%%%%%%%%%
\section{Introduction}
%%%%%%%%%%%%%%%%%%%%%%%%%%%%%%%%%%%%%%%%%%%%%%%%%%
Composite materials are very prone to various kinds of defects such as cracks, fibre breakage, debonding, and delamination~\cite{ip2004delamination, smith2009composite}. Among these defects, delamination is one of the most hazardous forms of the defects, which essentially leads to very catastrophic failures if not detected at early stages~\cite{valdes1999delamination}. 
Therefore, it is essential to effectively identify the delamination in composite structures for the purpose of safe and reliable implementation in various real-world applications. 
Accordingly, different types of Structural Health Monitoring (SHM) techniques have been developed for delamination detection in composite structures. 
Recently, guided Lamb waves based SHM gained high popularity for damage detection in composite structures due to their higher sensitivity to small defects, propagation with low attenuation, and potential to monitor large areas with low-voltage and only a small number of sparsely distributed transducers~\cite{alleyne1992interaction, giurgiutiu2003lamb, ihn2008pitch, mitra2016guided}. 
However, utilising a smaller number of transducers are not suitable for acquiring high-quality resolution damage maps. Whereas, the employment of a very dense array of transducers is also not feasible in most of the situations. 
For alleviating such problem Scanning Laser Doppler Vibrometer (SLDV) is employed. SLDV is capable to measure guided Lamb waves in a very dense grid of points over the surface of a large specimen. 
This collection of signals is known as full wavefield~\cite{radzienski2019damage}. 
Damage detection techniques employing full wavefield signals are capable of effectively estimating the size and location of damage~\cite{girolamo2018impact, kudela2018impact}. From the last few years, full wavefield signals are continually being assessed for the detection and localisation of defects in composite structures~\cite{sohn2011delamination, sohn2011automated, rogge2013characterization, kudela2018impact, radzienski2019damage}.

Currently, full wavefield signals based damage detection techniques are employing various physics and classical machine learning-based methods. 
These structural damage detection approaches are composed of two processes: feature extraction and feature classification. 
The feature extraction process usually needs a great deal of human labor and computational effort which prevents these techniques of being applicable in real-time SHM utilisation. 
Further, such systems also needs a notable amount of expertise from the practitioner, which is very difficult to be always available. Moreover, in many situations, the extracted handcrafted features by these techniques may fail to precisely characterise the acquired signal which leads to poor classification performance~\cite{zhao2019deep, yuan2020machine}. Additionally, these systems are also not suitable for modeling large-scale data.

Recently, deep learning which is originated from Artificial Neural Network (ANN) has shown very promising results in various domains such as computer vision, object detection, speech recognition, remote sensing, medical sciences and many more~\cite{deng2014deep, mohanty2016using, zhang2020well, pashaei2020review}. 
In recent years, deep learning has shown significant improvements in image segmentation due to the advancement in deep Convolutional Neural Networks (CNN). 

Image segmentation is a fundamental component in numerous visual recognition systems. In the last few years, image segmentation has widely been employed in autonomous driving~\cite{zhang2013understanding, cordts2016cityscapes, ros2016synthia, li2018real}, medical applications~\cite{taghanaki2020deep}, agriculture sciences~\cite{milioto2018real}, augmented reality~\cite{miksik2015semantic} and many more. 
Image segmentation techniques partition images or video frames into multiple objects or segments~\cite{szeliski2010computer}. 
It can be expressed as a pixel-level classification problem with semantic labels, which is known as semantic segmentation or can be partitioning the images into individual objects which are called instance segmentation~\cite{minaee2020image}. 
Semantic segmentation functions on pixel-wise labeling with a set of object categories of an image. 
Therefore, it is generally a more difficult task than image classification, which only predicts a single label for the entire image~\cite{minaee2020image}. 
Furthermore, semantic image segmentation not only depends on the semantics in the question but also on the problem that needs to be addressed~\cite{ghosh2019understanding}.

Deep learning-based systems intend to derive hierarchical representations from the input data via constructing deep neural networks by multiple layers of non-linear transformations. 
In deep learning architectures, the output of one layer act as the input to the other subsequent layer. 
The application of one layer in deep learning acquires a new representation of the input data and then, the stacking composition of many layers enables the model to learn complex patterns from the simple notions that can be formed from raw input. 
Therefore, these systems do not need extensive human labor and knowledge for hand-crafted feature design~\cite{zhao2019deep, yuan2020machine}.

Deep learning techniques have widely been utilised for the inspection and maintenance of civil infrastructures and has shown very promising results ~\cite{cha2017deep, lin2017structural, liu2019computer}. 
However, deep learning is still less explored for the purpose of delamination detection in composite materials.   

A few researchers have applied various deep learning techniques for damage detection with guided Lamb waves in composite structures. 
Fenza et al.~\cite{de2015application} presented the utilisation of ANN and probability ellipse techniques for the detection, location, and degree of defects in aluminum and fabric composite plates with the use of Lamb waves. 
Both the ANN and probability ellipse techniques were based on the damage index assessed by examining the variations in the Lamb waves acquired before and after the damage in each analysed path. 
The results from both methods proved that guided Lamb waves have prominent advantages in localisation and the detection of different kinds of defects in plate-like structures. 
Feng et al.~\cite{feng2019locating} proposed two time of flight (ToF) based algorithms of scattered guided Lamb waves in carbon fiber reinforced polymer (CFRP) plates. 
Their first algorithm is a probabilistic approach that constructs a probability matrix. The probability matrix is used for the localisation of delamination while the second algorithm is based on ANN which is then employed for improving the accuracy of defect localisation. 
The neural network receives the input from the ToF of scattered waves acquired from three sensor pairs.
Chetwynd et al.~\cite{chetwynd2008damage} used MLP neural network for the classification and regression tasks of damage detection in a stiffened curved CFRP investigated using Lamb waves with the use of eight surface bonded piezoelectric transducers. 
Many localised defects were fabricated through a force applicator, and Lamb wave responses were received for the damaged and healthy cases. 
For each case, the Lamb wave response was then transformed into a scalar novelty index with the help of outlier analysis. 
These novelty indices of 28 sensor paths were then provided as input to the MLP classification and regression architectures. 
For the classification of damaged and undamaged regions of the panel, the MLP classificier was employed, whereas the MLP regressor was used for evaluating the accurate location of damage on the panel. 
They achieved quite better results with both the classifier and regerssor. 
The classification accuracy of their MLP based classifier was 88.1\% on the test data while the Mean Square Error (MSE) value of the regerssor was 3.1\% on the unseen data. Su and Ye~\cite{su2004lamb} presented a Lamb wave based delamination identification technique in composite structures with the use of wavelet transform and multi-layer feedforward ANN architecture. 
The ANN was employed with the error-backpropagation (BP) algorithm. 
They also developed an Intelligent Signal Processing and Pattern Recognition (ISPPR) package for the extraction and digitision of spectrographic characterisitics of simulated Lamb waves in the time-frequency domain, which is known as Digital Damage Fingerprints (DOF) and is used for constructing a Damage Parameters Database (DPD). 
The DPD is then employed offline for training the neural network. 
They validated their approach with identifying actual delamination in different composites and also proved that their system has achieved excellent quantitative diagnosis results for different damage parameters such as the presence, location, orientation and geometry of defects. 
Hussain et al.~\cite{hussaintemporal} proposed a Temporal Convolutional Network (TCN) based transfer learning system for delamination prediction in CFRP cross-ply laminates. They employed a CFRP dataset from NASA which is composed of signals from Lamb waves sensors and X-ray images of specimens for capturing the propagation of defects in carbon fiber composite under fatigue loading. 
The TCN model was trained with various combinations of lengths of the sensors signals and different frequencies at which Lamb wave signals were sensed. 
They demonstrated that their approach needs very little time for the training and can also predict the delamination on a new composite coupon by utilising only a few samples of the test coupon. 
Melville et al.~\cite{melville2018structural} applied SVM and deep learning techniques for damage detection on full wavefield signals of ultrasonic guided wave images. The wavefield data was acquired via a laser Doppler vibrometer and piezoelectric actuators on thin metal plates. 
They showed that the deep learning methods achieved quite better damage prediction results as compared to the SVM based methods. 
Esfandabadi et al.~\cite{keshmiri2019deep} investigated the applications of super-resolution techniques to acquire high-resolution wavefields via the training of neural networks on different aluminum and CFRP plates. 
They applied two variants of CNN architecture: Super-Resolution Convolutional Neural Networks (SRCNNs) and Very-Deep Super Resolution (VDSR) with compressive sensing for the recovery of high spatial frequency information from low-resolution wavefield images. 
A dataset of 652 wavefield images (326 with defects and 326 without defects) were constructed, acquired with laser Doppler vibrometer of guided ultrasonic waves propagation.
Additionally, 273 images of wavefield were employed as a testing database for the validation purpose of the proposed methodology.           

Readers are advised to refer to our previous work entitled (Full Wavefield Processing by Using FCN for Delamination Detection) (under review) since for our knowledge it was the first work of using full wavefield images in delamination detection in composite materials using deep learning techniques. 
In this work, we have implemented four different deep learning based semantic segmentation models for delamination detection in composite materials.
The models were validated on numerical and experimental data in order to show their ability to generalise.
The models were compared based on their Intersection over Union (IoU) and the total number of parameters.

The paper is organised as follows, the acquisition and preprocessing of the required data are presented in section~\ref{section:Data_acquisition_and_preprocessing}.
In section~\ref{section:semantic_segmentation} the semantic segmentation models used for delamination detection were illustrated in details. 
Next, the detailed comparison of these models were elaborated in section~\ref{section:results_and_discussions}.
Finally, the conclusion and future work is presented in section~\ref{conclusion}.
	%%%%%%%%%%%%%%%%%%%%%%%%%%%%%%%%%%%%%%%%%%%%%%%%%%%%%
	\section{Dataset computation and preprocessing}
\subsection{Dataset computation}
%%%%%%%%%%%%%%%%%%%%%%%%%%%%%%%%%%%%%%%%%%%%%%%%%%%%%%%%%%%%%%%%%%%%%%%%%%%%%%%%%%%%%%%%
In this work, a synthetic dataset of propagating waves in carbon fibre reinforced composite plates was computed by using the parallel implementation of the time domain spectral element method~\cite{Kudela2020}. 
Essentially, the dataset resembles the particle velocity measurements at the bottom surface of the plate acquired by the SLDV in the transverse direction as a response to the piezoelectric transducer excitation placed at the centre of the plate's top surface. 
The input signal was a five-cycle Hann window-modulated sinusoidal tone burst. 
The carrier frequency was assumed to be 50 kHz. 
The total wave propagation time was set to 0.75 ms.
The number of time integration steps was 150000, which was selected for the 
stability of the central difference scheme.

The material was a typical cross-ply CFRP laminate. 
The stacking sequence [0/90]\(_4\) was used in the model. 
The properties of a single ply were as follows [GPa]:
\(C_{11} = 52.55, \, C_{12} = 6.51, \, C_{22} = 51.83, C_{44} = 2.93, C_{55} = 
2.92, C_{66} = 3.81\). 
The assumed mass density was 1522.4 kg/m\textsuperscript{3}.
These properties were selected so that wavefields simulated numerically are matching the wavefields measured by SLDV on real CFRP specimens.
The wavelength of the dominating A0 Lamb wave mode was 21.2 mm.

475 cases were simulated, representing Lamb waves propagation and interaction 
with single delamination for each case. 
The computation took about 3 hours and 20 minutes for each case.
The following random factors were used in simulated delamination scenarios:
\begin{itemize}
	\item coordinates of the centre of delamination,
	\item delamination geometrical size	determined by ellipse minor and major axis randomly selected from the range $10-40$ mm,
	\item delamination angle randomly selected from the range $ 0^{\circ}-180^{\circ}$.
	
\end{itemize}
The delamination modelling was realized by writing custom geometry files which were used to generate unstructured mesh consisting of quadrilateral elements by using gmsh software~\cite{Geuzaine2009}.
An exemplary mesh of quadrilateral elements is shown in Fig.~\ref{fig:random_delam_mesh} in which green elements highlight the delamination whereas red elements represent the location of the piezoelectric actuator.
Next, the mesh was modified by doubling elements and splitting nodes at delamination region.
Additionally, the quadrilateral elements were converted to the 36-node spectral elements by using a custom MATLAB script.
The wave propagation problem was solved by using in-house code of the time domain SEM which was run on GPU.
%%%%%%%%%%%%%%%%%%%%%%%%%%%%%%%%%%%%%%%%%%%%%%%%%%
\begin{figure} [h!]
	\begin{center}
		\includegraphics{figure2.png}
	\end{center}
	\caption{Exemplary mesh containing piezoelectric transducer (red) and random delamination (green) used for Lamb wave propagation modelling.} 
	\label{fig:random_delam_mesh}
\end{figure}
%%%%%%%%%%%%%%%%%%%%%%%%%%%%%%%%%%%%%%%%%%%%%%%%%%
The dataset contains 475 different delamination cases, with 512 frames per case, producing a total number of 243,\,200 frames with a frame size of \((500\times500)\)~pixels representing the geometry of the specimen surface of size \((500\times500)\)~mm\(^{2}\).
The frames in the dataset are 8-bit .png greyscale images.
The spatial size of the wavefield was further downsampled to \((256\times256)\) for the purpose of 
reducing the computational complexity.

It is important to note that input data to the DL model in the form of the binary image representing the respective delamination case as it is presented in Fig.~\ref{fig:complete_flowchart} was insufficient to train a reliable model.
It was necessary to provide additional inputs in the form of reference full wavefield frames (without delaminations).
This is explained along with the DL model in section~\ref{sec:proposed_approach}.
%%%%%%%%%%%%%%%%%%%%%%%%%%%%%%%%%%%%%%%%%%%%%%%%%%%%%%%%%%%%%%%%%%%%%%%%%%%%%%%%
\subsection{Data augmentation}
The best way for improving the generalisation capabilities of the neural network is to acquire more data. 
However, in practice, it can be difficult to acquire more data and the amount of available data for neural network training is limited.
For tackling this issue, one way is to create some fake data based on the original dataset and add it to the training set, which is termed data augmentation. 
Data augmentation is an efficient approach for various computer vision and DL tasks. 
Data augmentation includes randomly cropping a region from the original image, adjusting contrast, rotation for a small angle and flipping images, etc.~\cite{szegedy2015going}.
In this research work, the dataset is composed of 475 delamination cases which is not enough for a targeted DL model to perform well.
Therefore, all the images in the 475 delamination cases are flipped diagonally, horizontally, and vertically in order to enhance the performance of the proposed DL model. 
Therefore, the total dataset after data augmentation is now composed of 1900 \((475\times4 = 1900)\) delamination cases.
%%%%%%%%%%%%%%%%%%%%%%%%%%%%%%%%%%%%%%%%%%%%%%%%%%%%%%%%%%%%%%%%%%%%%%%%%%%%%%%%
\subsection{Dataset division and preprocessing}
For training and evaluation of the proposed DL model, the dataset was divided into two sets: training and testing, with a ratio of \(80\%\) and \(20\%\), respectively.
Moreover, \(20\%\) of the training set was preserved as a validation set to validate the model during the training process.

Moreover, the dataset was normalised to a range of \((0, 1)\) to improve the convergence of the gradient descent algorithm.
Due to memory limitations, \(32\) consecutive frames in each delamination case were selected for DL model training.
Additionally, frames displaying the propagation of guided waves before interaction with the delamination have no features to be extracted.
Hence, only a certain number of frames were selected from the initial occurrence of the interactions with the delamination.
%%%%%%%%%%%%%%%%%%%%%%%%%%%%%%%%%%%%%%%%%%%%%%%%%%%


	%%%%%%%%%%%%%%%%%%%%%%%%%%%%%%%%%%%%%%%%%%%%%%%%%%%%%
	\section{The proposed DL model for supervised learning}
\label{sec:proposed_approach}
%%%%%%%%%%%%%%%%%%%%%%%%%%%%%%%%%%%%%%%%%%%%%%%%%%%%%%%%%%%%%%%%%%%%%%%%%%%%%%%
In this research work, we developed a novel deep ConvLSTM autoencoder-based surrogate model utilising full wavefield frames of Lamb wave propagation for the purpose of data generation for delamination identification in thin plates made of composite materials.
The developed DL model takes as an input \(32\) frames without delamination (reference frames) representing the full wavefield and the delamination information of the respective delamination case in the form of binary image for the purpose of producing full wavefield propagation of Lamb waves through space and time (3D matrix).
The most important aspect of the DL model is the prediction of the interaction of Lamb waves with the delamination so that the delamination location, shape, and size can be estimated.
%%%%%%%%%%%%%%%%%%%%%%%%%%%%%%%%%%%%%%%%%%%%%%%%%%%%%%%%%%%%%%%%%%%%%%%%%%%%%%%%
%%%%%%%%%%%%%%%%%%%%%%%%%%%%%%%%%%%%%%%%%%%%%%%%%%
\begin{figure} [h!]
	\begin{center}
		\includegraphics[width=9cm]{figure4.png}
	\end{center}
	\caption{The flowchart of the proposed DL model.} 
	\label{fig:proposed_model}
\end{figure}
%%%%%%%%%%%%%%%%%%%%%%%%%%%%%%%%%%%%%%%%%%%%%%%%%%

The complete flowchart of the proposed DL model is presented in Fig.~\ref{fig:proposed_model}. 
The training and evaluation process of the proposed model can be summarized in 
the following three steps:  
\begin{enumerate}
	\item{\textbf{Feature extraction:} As we have no labels for the dataset, the dataset is composed of delamination cases. 
		So the first task was to extract features from all of the delamination cases, and then use these features as labels in the second step during model training.
		Therefore, in this step, the encoder and decoder parts of the proposed model are trained jointly, so the decoder part can be used separately for full wavefield predictions.
		During this step, the features are extracted with very minimal reconstruction error in a compressed form, which matches the dimensions of the latent space.}
	\item{\textbf{Model training:} In this step, the actual model training is being carried out. 
		The full wavefield frames in a plate without delamination along with the binary image of the respective delamination case are fed into the DL model for training. 
		The features extracted from encoder part of the first step are used as labels in this step, as shown in Fig.~\ref{fig:proposed_model}}.
	\item{\textbf{Evaluation of the proposed DL model on unseen data:} At this stage, both of the pre-trained models (pre-trained decoder from step 1, and pre-trained encoder from step 2) are utilised for the prediction of full wavefield frames on unseen data.
		During this step, the model just takes reference frames with the delamination information and produces the output as the full wavefield frames containing interaction of Lamb waves with delamination.}
\end{enumerate}

%%%%%%%%%%%%%%%%%%%%%%%%%%%%%%%%%%%%%%%%%%%%%%%%%%
\begin{figure} [h!]
	\begin{center}
		\includegraphics[width=11cm]{figure5.png}
	\end{center}
	\caption{The architecture of the proposed ConvLSTM autoencoder model.} 
	\label{fig:convlstm}
\end{figure}
%%%%%%%%%%%%%%%%%%%%%%%%%%%%%%%%%%%%%%%%%%%%%%%%%%

The proposed ConvLSTM autoencoder model takes \(32\) frames as input concatenated with a binary image which is replicated 32 times (see Fig.~\ref{fig:proposed_model}). 
The DL model consists of six ConvLSTM layers.
The first ConvLSTM layer has \(32\) filters, the second and third layer has \(192\) filters, the fourth layer has \(32\) filters, and the last two ConvLSTM layers has \(192\) filters.
The kernel size of the ConvLSTM layers was set to (\(3\times3\)) with a stride of \((1)\). 
Padding was set to "same", which makes the output the same as the input in the case of stride \(1\).
Furthermore, a \(\tanh\) (the hyperbolic tangent) activation function was used within the ConvLSTM layers that output values in a range between (\(-1\) and \(1\)).
Maxpooling and upsampling were applied at each ConvLSTM layer for reducing the size of features and reconstruction purposes, respectively. 
Moreover, a batch normalization technique~\cite{Santurkar2018} was applied at each of the ConvLSTM layers.
At the final output layer, a 2D convolutional layer followed by a sigmoid activation function is applied.

To alleviate the over-fitting, we used an early-stopping mechanism that monitors the validation loss during the training of the model and stops the training of the model after 30 epochs if there is no improvement. 
Adam optimizer was employed for back-propagation and MSE as a loss function for both training steps.

For evaluating the performance of the proposed model, two evaluation metrics, namely peak signal-to-noise ratio (PSNR), and Pearson correlation coefficient (Pearson CC) were utilized. 
The PSNR measures the maximum potential power of a signal and the power of the noise that affects the quality of its representation and is expressed mathematically in Eq.~(\ref{eqn:psnr}):

\begin{equation}
	\mathrm{PSNR}=20 \log _{10} \frac{L}{\sqrt{\mathrm{MSE}}}
	\label{eqn:psnr}
\end{equation}

Where \(L\) denotes the highest degree of variation present in the input image. 
Meanwhile, MSE stands for mean squared error, which represents the discrepancy between the predicted output and the relevant ground truth.

Pearson CC is a metric that estimates the linear connection between two sets of variables, \(\vect{x}\) (which represents the ground truth values) and \(\vect{y}\) (which represents the predicted values). 
The mathematical formula for computing Pearson CC is shown in Eq.~(\ref{eqn:pearsoncc}):

\begin{equation}
	r_{x 
		y}=\frac{\sum_{k=1}^n\left(x_k-\bar{x}\right)\left(y_k-\bar{y}\right)}{\sqrt{\sum_{k=1}^n\left(x_k-\bar{x}\right)^2}
		\sqrt{\sum_{k=1}^n\left(y_k-\bar{y}\right)^2}},
	\label{eqn:pearsoncc}
\end{equation}

where $r_{xy}$ represents the Pearson CC, \(n\) represents the number of data points in a sample, and $x_k$ and $y_k$ denote the values of the ground truth and predicted values, respectively, for each data point. 
Additionally, $\bar{x}$ denotes the mean value of the sample, $\bar{y}$ represents the mean value of the predicted values. 
The values of $r_{xy}$ ranges from ‘-1’ to ‘+1’. 
Value ‘0’ specifies that there is no relation between the samples and the predicted values. 
A value greater than ‘0’ indicates a positive relationship between the samples and the predicted data, whereas, a value less than ‘0’ represents a negative relationship between them.

The maximum PSNR and Pearson CC values on the validation data were noted as 23.7 dB and 0.99, respectively.

	%%%%%%%%%%%%%%%%%%%%%%%%%%%%%%%%%%%%%%%%%%%%%%%%%%%%%
	\section{Inverse method for damage identification}

A global-best PSO algorithm implemented in Python was used in the optimisation process~\cite{MirandaLesterJames}.
It takes a set of candidate solutions and tries to find the best solution using a position-velocity update method. 
It uses a star-topology where each particle is attracted to the best-performing particle.
The algorithm follows two basic steps:
\begin{itemize}
	\item the position update:
	\begin{equation}
		x_i(t+1) = x_i(t) + v_i(t+1),\label{eq:position_update}
	\end{equation}
	\item and the velocity update:
	\begin{equation}
		v_{ij}(t+1) = w\, v_{ij}(t) + c_1\, r_{1j}(t) \,[y_{ij}(t) - x_{ij}(t)] + c_2\, r_{2j}(t)\,[\hat{y}_j(t) - x_{ij}(t)],\label{eq:velocity_update}
	\end{equation}
\end{itemize}
where $r$ are random numbers, $y_{ij}$ is the particle's best-known position, $\hat{y}_j$ is the swarm's best known position, $c_1$ is the cognitive parameter, $c_2$ is the social parameter and $w$ is the inertia parameter which controls the swarm's movement.
Cognitive and social parameters control the particle's behaviour given two choices: (i) to follow its personal best or (ii) follow the swarm’s global best position.
Overall, this dictates if the swarm is explorative or exploitative in nature. 
In our tests, we used the following parameters: $c_1 = c_2 = 0.3$ and $w=0.8$.
Good convergence was achieved for these set of parameters, therefore further parameter tuning was unnecessary.

The following decision variables were used in the PSO:
\begin{itemize}
	\item delamination coordinates $(x_c, y_c)$ with bounds [0 mm, 500 m],
	\item delamination elliptic shape represented by semi-major and semi-minor axis $a, b$ with bounds [5 mm, 20 mm],
	\item delamination rotation angle $\alpha$ with bounds [$0^\circ$, $180^\circ$].
\end{itemize}

Based on decision variables, binary images of $(256\times256)$ pixels are generated (one image per particle - see Fig.~\ref{fig:complete_flowchart}).
In these images, ones (white pixels) represent delamination whereas zeros (black pixels) represent healthy area.

The most important component of the proposed inverse method is the surrogate DL model described in section~\ref{sec:proposed_approach}.
The trained DL model is used for ultrafast full wavefield prediction as illustrated in Fig.~\ref{fig:complete_flowchart}.
For a single particle and respective binary image, the DL model is evaluated 7 times for 32 consecutive frames giving as an output 224 frames. 
These predicted frames are compared to 'measured' frames by using the MSE metric which is utilised in the objective function.
However, for the sake of replicability of the results and compatibility with the available dataset~\cite{kudela_pawel_2021_5414555}, we used synthetic data instead of measured data (acquired by SLDV).

For each PSO iteration, particles are updated according to Eqs.~(\ref{eq:position_update})-(\ref{eq:velocity_update}).
The termination criterion was assumed as 100 iterations but it was observed that the objective function value converges much faster.
In the final step, the best matching wavefields indicate coordinates, semi-major, semi-minor axis and rotation angle of the elliptic-shaped delamination. 
These parameters are used for a visual representation of the best-matched delamination in the form of binary image compared against the ground truth (see also Fig.~\ref{fig:complete_flowchart}).

As an evaluation metric for assessing the accuracy of the identified 
delamination, we used the intersection over union (IoU), which measures the 
degree to which the predicted delamination overlaps the true delamination. 
It is defined as:
\begin{equation}
	IoU=\frac{Intersection}{Union}=\frac{\hat{Y} \cap Y}{\hat{Y} \cup Y}
	\label{eq:iou}
\end{equation}
where \(\hat{Y}\) is the predicted output, and \(Y\) is the ground truth (true delamination) in the form of binary images.

	%%%%%%%%%%%%%%%%%%%%%%%%%%%%%%%%%%%%%%%%%%%%%%%%%%%%%
	\section{Results and discussions}
%%%%%%%%%%%%%%%%%%%%%%%%%%%%%%%%%%%%%%%%%%%%%%%%%%
In this section, we present the evaluation of the proposed model based 
on numerical test data of \(95\) delamination cases representing the frames of 
the full wavefield propagation, which was not shown to the proposed model 
during training. 
The proposed model was evaluated using numerical test data to 
demonstrate the capability to predict delamination location, shape, and size 
from the reference frame (without delamination), and the delamination 
information in binary form.

Three different representative cases were selected from the numerical dataset 
to show the performance of the developed model.
Figures~\ref{fig:first_scenario},~\ref{fig:second_scenario}, 
and~\ref{fig:third_scenario} shows three different frames from three different 
numerical test cases.  
Figures~\ref{fig:first_scenario_ref_28},~\ref{fig:first_scenario_ref_30},
~\ref{fig:first_scenario_ref_32},~\ref{fig:second_scenario_ref_28},
~\ref{fig:second_scenario_ref_30},~\ref{fig:second_scenario_ref_32},
~\ref{fig:third_scenario_ref_28},~\ref{fig:third_scenario_ref_30}, 
and~\ref{fig:third_scenario_ref_32} represents the frame without delamination 
(reference) frame, which is given as input to the deep learning model for the 
prediction purpose. 
Figures~\ref{fig:first_scenario_pred_28},~\ref{fig:first_scenario_pred_30},
~\ref{fig:first_scenario_pred_32},~\ref{fig:second_scenario_pred_28},
~\ref{fig:second_scenario_pred_30},~\ref{fig:second_scenario_pred_32},
~\ref{fig:third_scenario_pred_28}, ~\ref{fig:third_scenario_pred_30}, 
and~\ref{fig:third_scenario_pred_32} represents the predicted frame by the deep 
learning model. 
Whereas, 
figures~\ref{fig:first_scenario_lab_28},~\ref{fig:first_scenario_lab_30},
~\ref{fig:first_scenario_lab_32},~\ref{fig:second_scenario_lab_28},
~\ref{fig:second_scenario_lab_30},~\ref{fig:second_scenario_lab_32},
~\ref{fig:third_scenario_lab_28}, ~\ref{fig:third_scenario_lab_30}, 
and~\ref{fig:third_scenario_lab_32} represents the the label frame, to which 
the prediction of the proposed model is compared.
Furthermore, figures~\ref{fig:first_scenario_ref_28}, 
~\ref{fig:first_scenario_pred_28},~\ref{fig:first_scenario_lab_28},
~\ref{fig:second_scenario_ref_28}, 
~\ref{fig:second_scenario_pred_28},~\ref{fig:second_scenario_lab_28},
~\ref{fig:third_scenario_ref_28}, ~\ref{fig:third_scenario_pred_28} 
and~\ref{fig:third_scenario_lab_28} represents the $28\textsuperscript{th}$ 
frame after the interaction with the delamination. 
Figures~\ref{fig:first_scenario_ref_30}, 
~\ref{fig:first_scenario_pred_30},~\ref{fig:first_scenario_lab_30},
~\ref{fig:second_scenario_ref_30}, 
~\ref{fig:second_scenario_pred_30},~\ref{fig:second_scenario_lab_30},
~\ref{fig:third_scenario_ref_30}, ~\ref{fig:third_scenario_pred_30} 
and~\ref{fig:third_scenario_lab_30} represents $30\textsuperscript{th}$ frame 
after the interaction with the delamination.
Whereas, figures~\ref{fig:first_scenario_ref_32}, 
~\ref{fig:first_scenario_pred_32},~\ref{fig:first_scenario_lab_32},
~\ref{fig:second_scenario_ref_32}, 
~\ref{fig:second_scenario_pred_32},~\ref{fig:second_scenario_lab_32},
~\ref{fig:third_scenario_ref_32}, ~\ref{fig:third_scenario_pred_32} 
and~\ref{fig:third_scenario_lab_32} represents $32\textsuperscript{th}$ frame 
after the interaction with the delamination.

%%%%%%%%%%%%%%%%%%%%%%%%%%%%%%%%%%%%%%%%%%%%%%%%%%%%%%%%%%%%%%%%%%%%%%%%%%%%%%%%
\begin{figure} [!ht]
	\centering
	\begin{subfigure}[b]{0.32\textwidth}
		\centering
		\includegraphics[scale=0.7]{Graphics/figure6a.png}
		\caption{Reference}
		\label{fig:first_scenario_ref_28}
	\end{subfigure}
	\hfill
	\begin{subfigure}[b]{0.32\textwidth}
		\centering
		\includegraphics[scale=0.7]{Graphics/figure6b.png}
		\caption{Prediction}
		\label{fig:first_scenario_pred_28}
	\end{subfigure}
	\hfill
	\begin{subfigure}[b]{0.32\textwidth}
		\centering
		\includegraphics[scale=0.7]{Graphics/figure6c.png}
		\caption{Label}
		\label{fig:first_scenario_lab_28}
	\end{subfigure}	
	\hfill
	\begin{subfigure}[b]{0.32\textwidth}
		\centering
		\includegraphics[scale=0.7]{Graphics/figure6d.png}
		\caption{Reference}
		\label{fig:first_scenario_ref_30}
	\end{subfigure}
	\hfill
	\begin{subfigure}[b]{0.32\textwidth}
		\centering
		\includegraphics[scale=0.7]{Graphics/figure6e.png}
		\caption{Prediction}
		\label{fig:first_scenario_pred_30}
	\end{subfigure}
	\hfill
	\begin{subfigure}[b]{0.32\textwidth}
		\centering
		\includegraphics[scale=0.7]{Graphics/figure6f.png}
		\caption{Label}
		\label{fig:first_scenario_lab_30}
	\end{subfigure}

	\hfill
	\begin{subfigure}[b]{0.32\textwidth}
		\centering
		\includegraphics[scale=0.7]{Graphics/figure6g.png}
		\caption{Reference}
		\label{fig:first_scenario_ref_32}
	\end{subfigure}
	\hfill
	\begin{subfigure}[b]{0.32\textwidth}
	\centering
	\includegraphics[scale=0.7]{Graphics/figure6h.png}
	\caption{Prediction}
	\label{fig:first_scenario_pred_32}
	\end{subfigure}
	\hfill
	\begin{subfigure}[b]{0.32\textwidth}
	\centering
	\includegraphics[scale=0.7]{Graphics/figure6i.png}
	\caption{Label}
	\label{fig:first_scenario_lab_32}
	\end{subfigure}
	
	\caption{First Scenario: Comparison of predicted frames with the label 
		frames at $28\textsuperscript{th}$, $30\textsuperscript{th}$, and 
		$32\textsuperscript{th}$ frame after the interaction with delamination.}
	\label{fig:first_scenario}
\end{figure}
%%%%%%%%%%%%%%%%%%%%%%%%%%%%%%%%%%%%%%%%%%%%%%%%%%%%%%%%%%%%%%%%%%%%%%%%%%%%%%%%
\begin{figure} [!ht]
	\centering
	\begin{subfigure}[b]{0.32\textwidth}
		\centering
		\includegraphics[scale=0.7]{Graphics/figure7a.png}
		\caption{Reference}
		\label{fig:second_scenario_ref_28}
	\end{subfigure}
	\hfill
	\begin{subfigure}[b]{0.32\textwidth}
		\centering
		\includegraphics[scale=0.7]{Graphics/figure7b.png}
		\caption{Prediction}
		\label{fig:second_scenario_pred_28}
	\end{subfigure}
	\hfill
	\begin{subfigure}[b]{0.32\textwidth}
		\centering
		\includegraphics[scale=0.7]{Graphics/figure7c.png}
		\caption{Label}
		\label{fig:second_scenario_lab_28}
	\end{subfigure}	
	\hfill
	\begin{subfigure}[b]{0.32\textwidth}
		\centering
		\includegraphics[scale=0.7]{Graphics/figure7d.png}
		\caption{Reference}
		\label{fig:second_scenario_ref_30}
	\end{subfigure}
	\hfill
	\begin{subfigure}[b]{0.32\textwidth}
		\centering
		\includegraphics[scale=0.7]{Graphics/figure7e.png}
		\caption{Prediction}
		\label{fig:second_scenario_pred_30}
	\end{subfigure}
	\hfill
	\begin{subfigure}[b]{0.32\textwidth}
		\centering
		\includegraphics[scale=0.7]{Graphics/figure7f.png}
		\caption{Label}
		\label{fig:second_scenario_lab_30}
	\end{subfigure}
    	\hfill
    \begin{subfigure}[b]{0.32\textwidth}
    	\centering
    	\includegraphics[scale=0.7]{Graphics/figure7g.png}
    	\caption{Reference}
    	\label{fig:second_scenario_ref_32}
    \end{subfigure}
    \hfill
    \begin{subfigure}[b]{0.32\textwidth}
    	\centering
    	\includegraphics[scale=0.7]{Graphics/figure7h.png}
    	\caption{Prediction}
    	\label{fig:second_scenario_pred_32}
    \end{subfigure}
    \hfill
    \begin{subfigure}[b]{0.32\textwidth}
    	\centering
    	\includegraphics[scale=0.7]{Graphics/figure7i.png}
    	\caption{Label}
    	\label{fig:second_scenario_lab_32}
    \end{subfigure}
	
	\caption{Second Scenario: Comparison of predicted frames with the label 
		frames at $28\textsuperscript{th}$, $30\textsuperscript{th}$, and 
		$32\textsuperscript{th}$ frame after the interaction with delamination.}
	\label{fig:second_scenario}
\end{figure}
%%%%%%%%%%%%%%%%%%%%%%%%%%%%%%%%%%%%%%%%%%%%%%%%%%%%%%%%%%%%%%%%%%%%%%%%%%%%%%%%
\begin{figure} [!ht]
	\centering
	\begin{subfigure}[b]{0.32\textwidth}
		\centering
		\includegraphics[scale=0.7]{Graphics/figure8a.png}
		\caption{Reference}
		\label{fig:third_scenario_ref_28}
	\end{subfigure}
	\hfill
	\begin{subfigure}[b]{0.32\textwidth}
		\centering
		\includegraphics[scale=0.7]{Graphics/figure8b.png}
		\caption{Prediction}
		\label{fig:third_scenario_pred_28}
	\end{subfigure}
	\hfill
	\begin{subfigure}[b]{0.32\textwidth}
		\centering
		\includegraphics[scale=0.7]{Graphics/figure8c.png}
		\caption{Label}
		\label{fig:third_scenario_lab_28}
	\end{subfigure}	
	\hfill
	\begin{subfigure}[b]{0.32\textwidth}
		\centering
		\includegraphics[scale=0.7]{Graphics/figure8d.png}
		\caption{Reference}
		\label{fig:third_scenario_ref_30}
	\end{subfigure}
	\hfill
	\begin{subfigure}[b]{0.32\textwidth}
		\centering
		\includegraphics[scale=0.7]{Graphics/figure8e.png}
		\caption{Prediction}
		\label{fig:third_scenario_pred_30}
	\end{subfigure}
	\hfill
	\begin{subfigure}[b]{0.32\textwidth}
		\centering
		\includegraphics[scale=0.7]{Graphics/figure8f.png}
		\caption{Label}
		\label{fig:third_scenario_lab_30}
	\end{subfigure}

		\hfill
	\begin{subfigure}[b]{0.32\textwidth}
		\centering
		\includegraphics[scale=0.7]{Graphics/figure8g.png}
		\caption{Reference}
		\label{fig:third_scenario_ref_32}
	\end{subfigure}
	\hfill
	\begin{subfigure}[b]{0.32\textwidth}
		\centering
		\includegraphics[scale=0.7]{Graphics/figure8h.png}
		\caption{Prediction}
		\label{fig:third_scenario_pred_32}
	\end{subfigure}
	\hfill
	\begin{subfigure}[b]{0.32\textwidth}
		\centering
		\includegraphics[scale=0.7]{Graphics/figure8i.png}
		\caption{Label}
		\label{fig:third_scenario_lab_32}
	\end{subfigure}
	
	\caption{Third Scenario: Comparison of predicted frames with the label 
		frames at $28\textsuperscript{th}$, $30\textsuperscript{th}$, and 
		$32\textsuperscript{th}$ frame after the interaction with delamination.}
	\label{fig:third_scenario}
\end{figure}
%%%%%%%%%%%%%%%%%%%%%%%%%%%%%%%%%%%%%%%%%%%%%%%%%%%%%%%%%%%%%%%%%%%%%%%%%%%%%%%%
As can be seen in the first and second scenario (Fig~\ref{fig:first_scenario}, 
and Fig~\ref{fig:second_scenario}),  the delamination is occurred at the 
top-left of the plate, whereas in the third scenario, 
Fig~\ref{fig:third_scenario} the 
delamination is occurred at the top-center of the plate. 
As shown in Fig~\ref{fig:first_scenario}, the delamination is very tiny in this 
case and not clearly seen by human eyes, but our deep learning algorithm is 
somehow able to reconstruct the full wavefield along with the delamination. 
From all of the 
Figs~\ref{fig:first_scenario_pred_28},~\ref{fig:first_scenario_pred_30},
~\ref{fig:first_scenario_pred_32},~\ref{fig:second_scenario_pred_28},
~\ref{fig:second_scenario_pred_30},~\ref{fig:second_scenario_pred_32},
~\ref{fig:third_scenario_pred_28},~\ref{fig:third_scenario_pred_30} 
and~\ref{fig:third_scenario_pred_32} it can be confirmed the that the proposed 
deep learning-based surrogate model has reconstructed the full wavefield 
containing delamination with minimal error. 
Furthermore, the PSNR and Pearson CC values of all these three scenarios are 
shown in Table~\ref{tab:psnr_pearson}. 
%\newpage%
%%%%%%%%%%%%%%%%%%%%%%%%%%%%%%%%%%%%%%%%%%%%%%%%%%%%%%%%%%%%%%%%%%%%%%%%%%%%%%%%
% Please add the following required packages to your document preamble:
% \usepackage{booktabs}
\begin{table}[ht]
	\centering
	\caption{Evaluation metric for three numerical cases}
	\begin{tabular}{@{}ccc@{}}
		\toprule
		case number       & PSRN    & Pearson CC \\ \midrule
		1 ( $28\textsuperscript{th}$ frame) & 22.3 dB & 0.96       \\ \midrule
		1 ( $30\textsuperscript{th}$ frame)    & 22.7 dB & 0.98       \\ 
		\midrule
		1 ( $32\textsuperscript{th}$ frame)    & 23.1 dB & 0.98       \\ 
		\midrule
		2 ( $28\textsuperscript{th}$ frame) & 22.0 dB & 0.96       \\ \midrule
		2 ( $30\textsuperscript{th}$ frame)    & 22.6 dB & 0.98       \\ 
		\midrule
		2 ( $32\textsuperscript{th}$ frame)    & 23.0 dB & 0.98       \\ 
		\midrule
		3 ( $28\textsuperscript{th}$ frame) & 21.8 dB & 0.97       \\ \midrule
		3 ( $30\textsuperscript{th}$ frame) & 22.3 dB & 0.98       \\ \midrule
		3 ( $32\textsuperscript{th}$ frame)    & 23.2 dB & 0.99       \\ 
		\bottomrule
	\end{tabular}
	\label{tab:psnr_pearson}
\end{table}
%%%%%%%%%%%%%%%%%%%%%%%%%%%%%%%%%%%%%%%%%%%%%%%%%%%%%%%%%%%%%%%%%%%%%%%%%%%%%%%%


The mean PSNR value was 21.8 dB, and the mean Pearson CC value was 0.98 
on all of the test data.
%%%%%%%%%%%%%%%%%%%%%%%%%%%%%%%%%%%%%%%%%%%%%%%%%%%%%%%%%%%%%%%%%%%%%%%%%%%%%%%%
	%%%%%%%%%%%%%%%%%%%%%%%%%%%%%%%%%%%%%%%%%%%%%%%%%%%%%
	\clearpage
	%%%%%%%%%%%%%%%%%%%%%%%%%%%%%%%%%%%%%%%%%%%%%%%%%%
\section{Conclusions}
%%%%%%%%%%%%%%%%%%%%%%%%%%%%%%%%%%%%%%%%%%%%%%%%%%
In this paper, we addressed delamination detection in composite materials using a deep learning technique. 
For this purpose, we have trained an FCN-DenseNet for semantic segmentation on a numerically generated data to simulate a full wavefield elastic wave propagation.
To see the feasibility of such a study, we have compared the deep learning model with adaptive wavenumber filtering technique.
The results were promising, and the deep learning model surpasses the conventional technique in detecting the delaminations of different shapes, sizes and angles. 
Further, the model can be improved by training it on new experimental data, that means new patterns will be learned, hence it will enhance its ability to differentiate among different complex patterns.
Currently, we are in the progress of implementing several deep learning architectures in order to perform a comparative study of different deep learning models regarding delamination identification in composite materials.
Current work focuses on delamination identification, however, our work can be extended to the identification of different types of damage in composite materials.
	%%%%%%%%%%%%%%%%%%%%%%%%%%%%%%%%%%%%%%%%%%%%%%%%%%%%%
	
	\clearpage	
	%\appendix
	\section*{Acknowledgments}
	The research was funded by the Polish National Science Center under grant agreements no 2019/01/Y/ST8/00060.
	
	\bibliography{biblography}

\end{document}


