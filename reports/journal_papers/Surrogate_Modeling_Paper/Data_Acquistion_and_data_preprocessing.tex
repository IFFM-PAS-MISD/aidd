\section{Data acquisition and data preprocessing:}
\subsection{Data acquisition:}
%%%%%%%%%%%%%%%%%%%%%%%%%%%%%%%%%%%%%%%%%%%%%%%%%%%%%%%%%%%%%%%%%%%%%%%%%%%%%%%%%%%%%%%%
In this work, a synthetic dataset of propagating waves in carbon fibre 
reinforced composite plates was computed by using the parallel implementation 
of the time domain spectral element method~\cite{Kudela2020}. 
Essentially, the dataset resembles the particle velocity measurements at the 
bottom surface of the plate acquired by the SLDV in the transverse direction as 
a response to the piezoelectric (PZT) excitation at the centre of the plate. 
The input signal was a five-cycle Hann window modulated sinusoidal tone burst. 
The carrier frequency was assumed to be 50 kHz. 
The total wave propagation time was set to 0.75 ms so that the guided wave 
could propagate to the plate edges and back to the actuator twice.
The number of time integration steps was 150000, which was selected for the 
stability of the central difference scheme.

The material was a typical cross-ply CFRP laminate. 
The stacking sequence [0/90]\(_4\) was used in the model. 
The properties of a single ply were as follows [GPa]:
\(C_{11} = 52.55, \, C_{12} = 6.51, \, C_{22} = 51.83, C_{44} = 2.93, C_{55} = 
2.92, C_{66} = 3.81\). 
The assumed mass density was 1522.4 kg/m\textsuperscript{3}.
These properties were selected so that they simulated numerically wave front 
patterns and wavelengths that are similar to the wavefields measured by SLDV on 
CFRP specimens used later on for testing the developed approach for 
delamination identification.
The shortest wavelength of the propagating A0 Lamb wave mode was 21.2 mm for 
numerical simulations and 19.5 mm for experimental measurements.

475 cases were simulated, representing Lamb waves propagation and interaction 
with single delamination for each case. 
The following random factors were used in simulated delamination scenarios:
\begin{itemize}
	\item delamination geometrical size	\(2b\) and \(2a\), namely ellipse minor 
	and major axis randomly selected from the interval \newline \(\left[10 \, 
	\textrm{mm}, 40\, \textrm{mm}\right]\),
	\item delamination angle \(\alpha\) randomly selected from the interval \( 
	\left[ 0^{\circ}, 180^{\circ} \right]\),
	\item coordinates of the centre of delamination \((x_c,y_c)\) randomly 
	selected from the interval \(\left[0\, \textrm{mm}, 250\, \textrm{mm} 
	-\delta \right]\) and \newline \( \left[250\, \textrm{mm}+\delta, 500\, 
	\textrm{mm} 
	\right] \), where \(\delta = 10\, \textrm{mm}\)).
\end{itemize}
These parameters are defined in Fig.~\ref{fig:random_delaminations} which 
illustrates exemplary possible locations, sizes, and shapes of random 
delaminations used for Lamb wave propagation modeling.
It should be noted that the numerical cases include delaminations located at 
the edge and corners of the plate.
%%%%%%%%%%%%%%%%%%%%%%%%%%%%%%%%%%%%%%%%%%%%%%%%%%
\begin{figure} [h!]
	\begin{center}
		\includegraphics[width=9cm]{figure2.png}
	\end{center}
	\caption{Exemplary locations, sizes and shapes of random delaminations used 
		for Lamb wave propagation modeling.} 
	\label{fig:random_delaminations}
\end{figure}
%%%%%%%%%%%%%%%%%%%%%%%%%%%%%%%%%%%%%%%%%%%%%%%%%%

The dataset contains frames of propagating waves (512  frames for each 
delamination scenario) and is available online~\cite{kudela_pawel_2021_5414555}.
The synthetic dataset is used for training the first step of the proposed 
neural network with the aim of delamination identification directly from SLDV 
measurements without the need for a baseline wavefield.

%%%%%%%%%%%%%%%%%%%%%%%%%%%%%%%%%%%%%%%%%%%%%%%%%%
The dataset contains 475 different cases, with 512 frames 
per case, producing a total number of 243,\,200 frames with a frame size of 
\((500\times500)\)~pixels representing the geometry of the specimen of size 
\((500\times500)\)~mm\(^{2}\).
The size of dataset is reduced into \((256\times256)\) for the purpose of 
reducing the computational complexity.
A full wavefield of frames without delaminations are used as input data to the 
proposed system in the seconds step alongwith the ground truth of the 
respective delamination case.
%%%%%%%%%%%%%%%%%%%%%%%%%%%%%%%%%%%%%%%%%%%%%%%%%%%%%%%%%%%%%%%%%%%%%%%%%%%%%%%%
\subsection{Data augmentation}
The best way of generalizing the neural network well is to acquire more data. 
In practice, the amount of available data are usually less for training a deep 
learning model. 
For tackling this issue, one way is to create some fake data based on the 
original dataset and adding it to the training set, which is termed as data 
augmentation. 
Data augmentation is an efficient approach for various computer vision and deep 
learning tasks. 
Data augmentation includes randomly cropping a region from the original image, 
adjusting contrast, rotation for a small angle and flipping of the images, 
etc.~\cite{szegedy2015going}.
In this research work, the dataset is composed of 475 delamination case which 
are not enough for a deep learning model to perform well.
Therefore, all the images in the 475 delamination cases are flipped diagonally, 
horizontally, and vertically in order to enhance the performance of the 
proposed deep learning model. 
Therefore, the total dataset after data augmentation is now composed of 1900 
\((475\times4 = 1900)\) delamination cases.
%%%%%%%%%%%%%%%%%%%%%%%%%%%%%%%%%%%%%%%%%%%%%%%%%%%%%%%%%%%%%%%%%%%%%%%%%%%%%%%%
\subsection{Dataset distribution}
For training and evaluation of the proposed deep learning model, the dataset 
was divided into two sets: training and testing, with 
a ratio of \(80\%\) and \(20\% \) respectively.
Moreover, \(10\%\) of the training set was preserved as a validation 
set to validate the model during the training process, the dataset distribution 
process is shown in Fig.~\ref{fig:data_distribution}.
\begin{figure} [h!]
	\begin{center}
		\includegraphics[width=8cm]{figure3.png}
	\end{center}
	\caption{Dataset distribution process for this research work.} 
	\label{fig:data_distribution}
\end{figure}
%%%%%%%%%%%%%%%%%%%%%%%%%%%%%%%%%%%%%%%%%%%%%%%%%%
Additionally, the dataset was normalised to a range of \((0, 1)\) to improve 
the convergence of the gradient descent algorithm.
The author selected \(32\) consecutive frames in each delamination case as 
using all frames in each case has high computational and memory 
costs.
Additionally, frames displaying the propagation of guided waves before 
interaction with the delamination has no features to be extracted.
Hence, only a certain number of frames were selected from the initial 
occurrence of the interactions with the delamination.
%%%%%%%%%%%%%%%%%%%%%%%%%%%%%%%%%%%%%%%%%%%%%%%%%%%
%%%%%%%%%%%%%%%%%%%%%%%%%%%%%%%%%%%%%%%%%%%%%%%%%%%
