\section{Dataset computation and preprocessing}
\subsection{Dataset computation}
%%%%%%%%%%%%%%%%%%%%%%%%%%%%%%%%%%%%%%%%%%%%%%%%%%%%%%%%%%%%%%%%%%%%%%%%%%%%%%%%%%%%%%%%
In this work, a synthetic dataset of propagating waves in carbon fibre reinforced composite plates was computed by using the parallel implementation of the time domain spectral element method~\cite{Kudela2020}. 
Essentially, the dataset resembles the particle velocity measurements at the bottom surface of the plate acquired by the SLDV in the transverse direction as a response to the piezoelectric transducer excitation placed at the centre of the plate's top surface. 
The input signal was a five-cycle Hann window-modulated sinusoidal tone burst. 
The carrier frequency was assumed to be 50 kHz. 
The total wave propagation time was set to 0.75 ms.
The number of time integration steps was 150000, which was selected for the 
stability of the central difference scheme.

The material was a typical cross-ply CFRP laminate. 
The stacking sequence [0/90]\(_4\) was used in the model. 
The properties of a single ply were as follows [GPa]:
\(C_{11} = 52.55, \, C_{12} = 6.51, \, C_{22} = 51.83, C_{44} = 2.93, C_{55} = 
2.92, C_{66} = 3.81\). 
The assumed mass density was 1522.4 kg/m\textsuperscript{3}.
These properties were selected so that wavefields simulated numerically are matching the wavefields measured by SLDV on real CFRP specimens.
The wavelength of the dominating A0 Lamb wave mode was 21.2 mm.

475 cases were simulated, representing Lamb waves propagation and interaction 
with single delamination for each case. 
The computation took about 3 hours and 20 minutes for each case.
The following random factors were used in simulated delamination scenarios:
\begin{itemize}
	\item coordinates of the centre of delamination,
	\item delamination geometrical size	determined by ellipse minor and major axis randomly selected from the range $10-40$ mm,
	\item delamination angle randomly selected from the range $ 0^{\circ}-180^{\circ}$.
	
\end{itemize}
The delamination modelling was realized by writing custom geometry files which were used to generate unstructured mesh consisting of quadrilateral elements by using gmsh software~\cite{Geuzaine2009}.
An exemplary mesh of quadrilateral elements is shown in Fig.~\ref{fig:random_delam_mesh} in which green elements highlight the delamination whereas red elements represent the location of the piezoelectric actuator.
Next, the mesh was modified by doubling elements and splitting nodes at delamination region.
Additionally, the quadrilateral elements were converted to the 36-node spectral elements by using a custom MATLAB script.
The wave propagation problem was solved by using in-house code of the time domain SEM which was run on GPU.
%%%%%%%%%%%%%%%%%%%%%%%%%%%%%%%%%%%%%%%%%%%%%%%%%%
\begin{figure} [h!]
	\begin{center}
		\includegraphics{figure2.png}
	\end{center}
	\caption{Exemplary mesh containing piezoelectric transducer (red) and random delamination (green) used for Lamb wave propagation modelling.} 
	\label{fig:random_delam_mesh}
\end{figure}
%%%%%%%%%%%%%%%%%%%%%%%%%%%%%%%%%%%%%%%%%%%%%%%%%%
The dataset contains 475 different delamination cases, with 512 frames per case, producing a total number of 243,\,200 frames with a frame size of \((500\times500)\)~pixels representing the geometry of the specimen surface of size \((500\times500)\)~mm\(^{2}\).
The frames in the dataset are 8-bit .png greyscale images.
The spatial size of the wavefield was further downsampled to \((256\times256)\) for the purpose of 
reducing the computational complexity.

It is important to note that input data to the DL model in the form of the binary image representing the respective delamination case as it is presented in Fig.~\ref{fig:complete_flowchart} was insufficient to train a reliable model.
It was necessary to provide additional inputs in the form of reference full wavefield frames (without delaminations).
This is explained along with the DL model in section~\ref{sec:proposed_approach}.
%%%%%%%%%%%%%%%%%%%%%%%%%%%%%%%%%%%%%%%%%%%%%%%%%%%%%%%%%%%%%%%%%%%%%%%%%%%%%%%%
\subsection{Data augmentation}
The best way for improving the generalisation capabilities of the neural network is to acquire more data. 
However, in practice, it can be difficult to acquire more data and the amount of available data for neural network training is limited.
For tackling this issue, one way is to create some fake data based on the original dataset and add it to the training set, which is termed data augmentation. 
Data augmentation is an efficient approach for various computer vision and DL tasks. 
Data augmentation includes randomly cropping a region from the original image, adjusting contrast, rotation for a small angle and flipping images, etc.~\cite{szegedy2015going}.
In this research work, the dataset is composed of 475 delamination cases which is not enough for a targeted DL model to perform well.
Therefore, all the images in the 475 delamination cases are flipped diagonally, horizontally, and vertically in order to enhance the performance of the proposed DL model. 
Therefore, the total dataset after data augmentation is now composed of 1900 \((475\times4 = 1900)\) delamination cases.
%%%%%%%%%%%%%%%%%%%%%%%%%%%%%%%%%%%%%%%%%%%%%%%%%%%%%%%%%%%%%%%%%%%%%%%%%%%%%%%%
\subsection{Dataset division and preprocessing}
For training and evaluation of the proposed DL model, the dataset was divided into two sets: training and testing, with a ratio of \(80\%\) and \(20\%\), respectively.
Moreover, \(20\%\) of the training set was preserved as a validation set to validate the model during the training process.

Moreover, the dataset was normalised to a range of \((0, 1)\) to improve the convergence of the gradient descent algorithm.
Due to memory limitations, \(32\) consecutive frames in each delamination case were selected for DL model training.
Additionally, frames displaying the propagation of guided waves before interaction with the delamination have no features to be extracted.
Hence, only a certain number of frames were selected from the initial occurrence of the interactions with the delamination.
%%%%%%%%%%%%%%%%%%%%%%%%%%%%%%%%%%%%%%%%%%%%%%%%%%%

