\section{Results and discussions}
%%%%%%%%%%%%%%%%%%%%%%%%%%%%%%%%%%%%%%%%%%%%%%%%%%
In this section, we present the evaluation of the proposed model based 
on numerical test data of \(95\) delamination cases representing the frames of 
the full wavefield propagation, which was not shown to the proposed model 
during training. 
The proposed model was evaluated using numerical test data to 
demonstrate the capability to predict delamination location, shape, and size 
from the reference frame (without delamination), and the delamination 
information in binary form.

Three different representative cases were selected from the numerical dataset 
to show the performance of the developed model.
Figures~\ref{fig:first_scenario},~\ref{fig:second_scenario}, 
and~\ref{fig:third_scenario} shows three different frames from three different 
numerical test cases.  
Figures~\ref{fig:first_scenario_ref_28},~\ref{fig:first_scenario_ref_30},
~\ref{fig:first_scenario_ref_32},~\ref{fig:second_scenario_ref_28},
~\ref{fig:second_scenario_ref_30},~\ref{fig:second_scenario_ref_32},
~\ref{fig:third_scenario_ref_28},~\ref{fig:third_scenario_ref_30}, 
and~\ref{fig:third_scenario_ref_32} represents the frame without delamination 
(reference) frame, which is given as input to the deep learning model for the 
prediction purpose. 
Figures~\ref{fig:first_scenario_pred_28},~\ref{fig:first_scenario_pred_30},
~\ref{fig:first_scenario_pred_32},~\ref{fig:second_scenario_pred_28},
~\ref{fig:second_scenario_pred_30},~\ref{fig:second_scenario_pred_32},
~\ref{fig:third_scenario_pred_28}, ~\ref{fig:third_scenario_pred_30}, 
and~\ref{fig:third_scenario_pred_32} represents the predicted frame by the deep 
learning model. 
Whereas, 
figures~\ref{fig:first_scenario_lab_28},~\ref{fig:first_scenario_lab_30},
~\ref{fig:first_scenario_lab_32},~\ref{fig:second_scenario_lab_28},
~\ref{fig:second_scenario_lab_30},~\ref{fig:second_scenario_lab_32},
~\ref{fig:third_scenario_lab_28}, ~\ref{fig:third_scenario_lab_30}, 
and~\ref{fig:third_scenario_lab_32} represents the the label frame, to which 
the prediction of the proposed model is compared.
Furthermore, figures~\ref{fig:first_scenario_ref_28}, 
~\ref{fig:first_scenario_pred_28},~\ref{fig:first_scenario_lab_28},
~\ref{fig:second_scenario_ref_28}, 
~\ref{fig:second_scenario_pred_28},~\ref{fig:second_scenario_lab_28},
~\ref{fig:third_scenario_ref_28}, ~\ref{fig:third_scenario_pred_28} 
and~\ref{fig:third_scenario_lab_28} represents the $28\textsuperscript{th}$ 
frame after the interaction with the delamination. 
Figures~\ref{fig:first_scenario_ref_30}, 
~\ref{fig:first_scenario_pred_30},~\ref{fig:first_scenario_lab_30},
~\ref{fig:second_scenario_ref_30}, 
~\ref{fig:second_scenario_pred_30},~\ref{fig:second_scenario_lab_30},
~\ref{fig:third_scenario_ref_30}, ~\ref{fig:third_scenario_pred_30} 
and~\ref{fig:third_scenario_lab_30} represents $30\textsuperscript{th}$ frame 
after the interaction with the delamination.
Whereas, figures~\ref{fig:first_scenario_ref_32}, 
~\ref{fig:first_scenario_pred_32},~\ref{fig:first_scenario_lab_32},
~\ref{fig:second_scenario_ref_32}, 
~\ref{fig:second_scenario_pred_32},~\ref{fig:second_scenario_lab_32},
~\ref{fig:third_scenario_ref_32}, ~\ref{fig:third_scenario_pred_32} 
and~\ref{fig:third_scenario_lab_32} represents $32\textsuperscript{th}$ frame 
after the interaction with the delamination.

%%%%%%%%%%%%%%%%%%%%%%%%%%%%%%%%%%%%%%%%%%%%%%%%%%%%%%%%%%%%%%%%%%%%%%%%%%%%%%%%
\begin{figure} [!ht]
	\centering
	\begin{subfigure}[b]{0.32\textwidth}
		\centering
		\includegraphics[scale=0.7]{figure6a.png}
		\caption{Reference}
		\label{fig:first_scenario_ref_28}
	\end{subfigure}
	\hfill
	\begin{subfigure}[b]{0.32\textwidth}
		\centering
		\includegraphics[scale=0.7]{figure6b.png}
		\caption{Prediction}
		\label{fig:first_scenario_pred_28}
	\end{subfigure}
	\hfill
	\begin{subfigure}[b]{0.32\textwidth}
		\centering
		\includegraphics[scale=0.7]{figure6c.png}
		\caption{Label}
		\label{fig:first_scenario_lab_28}
	\end{subfigure}	
	\hfill
	\begin{subfigure}[b]{0.32\textwidth}
		\centering
		\includegraphics[scale=0.7]{figure6d.png}
		\caption{Reference}
		\label{fig:first_scenario_ref_30}
	\end{subfigure}
	\hfill
	\begin{subfigure}[b]{0.32\textwidth}
		\centering
		\includegraphics[scale=0.7]{figure6e.png}
		\caption{Prediction}
		\label{fig:first_scenario_pred_30}
	\end{subfigure}
	\hfill
	\begin{subfigure}[b]{0.32\textwidth}
		\centering
		\includegraphics[scale=0.7]{figure6f.png}
		\caption{Label}
		\label{fig:first_scenario_lab_30}
	\end{subfigure}

	\hfill
	\begin{subfigure}[b]{0.32\textwidth}
		\centering
		\includegraphics[scale=0.7]{figure6g.png}
		\caption{Reference}
		\label{fig:first_scenario_ref_32}
	\end{subfigure}
	\hfill
	\begin{subfigure}[b]{0.32\textwidth}
	\centering
	\includegraphics[scale=0.7]{figure6h.png}
	\caption{Prediction}
	\label{fig:first_scenario_pred_32}
	\end{subfigure}
	\hfill
	\begin{subfigure}[b]{0.32\textwidth}
	\centering
	\includegraphics[scale=0.7]{figure6i.png}
	\caption{Label}
	\label{fig:first_scenario_lab_32}
	\end{subfigure}
	
	\caption{First Scenario: Comparison of predicted frames with the label 
		frames at $28\textsuperscript{th}$, $30\textsuperscript{th}$, and 
		$32\textsuperscript{th}$ frame after the interaction with delamination.}
	\label{fig:first_scenario}
\end{figure}
%%%%%%%%%%%%%%%%%%%%%%%%%%%%%%%%%%%%%%%%%%%%%%%%%%%%%%%%%%%%%%%%%%%%%%%%%%%%%%%%
\begin{figure} [!ht]
	\centering
	\begin{subfigure}[b]{0.32\textwidth}
		\centering
		\includegraphics[scale=0.7]{figure7a.png}
		\caption{Reference}
		\label{fig:second_scenario_ref_28}
	\end{subfigure}
	\hfill
	\begin{subfigure}[b]{0.32\textwidth}
		\centering
		\includegraphics[scale=0.7]{figure7b.png}
		\caption{Prediction}
		\label{fig:second_scenario_pred_28}
	\end{subfigure}
	\hfill
	\begin{subfigure}[b]{0.32\textwidth}
		\centering
		\includegraphics[scale=0.7]{figure7c.png}
		\caption{Label}
		\label{fig:second_scenario_lab_28}
	\end{subfigure}	
	\hfill
	\begin{subfigure}[b]{0.32\textwidth}
		\centering
		\includegraphics[scale=0.7]{figure7d.png}
		\caption{Reference}
		\label{fig:second_scenario_ref_30}
	\end{subfigure}
	\hfill
	\begin{subfigure}[b]{0.32\textwidth}
		\centering
		\includegraphics[scale=0.7]{figure7e.png}
		\caption{Prediction}
		\label{fig:second_scenario_pred_30}
	\end{subfigure}
	\hfill
	\begin{subfigure}[b]{0.32\textwidth}
		\centering
		\includegraphics[scale=0.7]{figure7f.png}
		\caption{Label}
		\label{fig:second_scenario_lab_30}
	\end{subfigure}
    	\hfill
    \begin{subfigure}[b]{0.32\textwidth}
    	\centering
    	\includegraphics[scale=0.7]{figure7g.png}
    	\caption{Reference}
    	\label{fig:second_scenario_ref_32}
    \end{subfigure}
    \hfill
    \begin{subfigure}[b]{0.32\textwidth}
    	\centering
    	\includegraphics[scale=0.7]{figure7h.png}
    	\caption{Prediction}
    	\label{fig:second_scenario_pred_32}
    \end{subfigure}
    \hfill
    \begin{subfigure}[b]{0.32\textwidth}
    	\centering
    	\includegraphics[scale=0.7]{figure7i.png}
    	\caption{Label}
    	\label{fig:second_scenario_lab_32}
    \end{subfigure}
	
	\caption{Second Scenario: Comparison of predicted frames with the label 
		frames at $28\textsuperscript{th}$, $30\textsuperscript{th}$, and 
		$32\textsuperscript{th}$ frame after the interaction with delamination.}
	\label{fig:second_scenario}
\end{figure}
%%%%%%%%%%%%%%%%%%%%%%%%%%%%%%%%%%%%%%%%%%%%%%%%%%%%%%%%%%%%%%%%%%%%%%%%%%%%%%%%
\begin{figure} [!ht]
	\centering
	\begin{subfigure}[b]{0.32\textwidth}
		\centering
		\includegraphics[scale=0.7]{figure8a.png}
		\caption{Reference}
		\label{fig:third_scenario_ref_28}
	\end{subfigure}
	\hfill
	\begin{subfigure}[b]{0.32\textwidth}
		\centering
		\includegraphics[scale=0.7]{figure8b.png}
		\caption{Prediction}
		\label{fig:third_scenario_pred_28}
	\end{subfigure}
	\hfill
	\begin{subfigure}[b]{0.32\textwidth}
		\centering
		\includegraphics[scale=0.7]{figure8c.png}
		\caption{Label}
		\label{fig:third_scenario_lab_28}
	\end{subfigure}	
	\hfill
	\begin{subfigure}[b]{0.32\textwidth}
		\centering
		\includegraphics[scale=0.7]{figure8d.png}
		\caption{Reference}
		\label{fig:third_scenario_ref_30}
	\end{subfigure}
	\hfill
	\begin{subfigure}[b]{0.32\textwidth}
		\centering
		\includegraphics[scale=0.7]{figure8e.png}
		\caption{Prediction}
		\label{fig:third_scenario_pred_30}
	\end{subfigure}
	\hfill
	\begin{subfigure}[b]{0.32\textwidth}
		\centering
		\includegraphics[scale=0.7]{figure8f.png}
		\caption{Label}
		\label{fig:third_scenario_lab_30}
	\end{subfigure}

		\hfill
	\begin{subfigure}[b]{0.32\textwidth}
		\centering
		\includegraphics[scale=0.7]{figure8g.png}
		\caption{Reference}
		\label{fig:third_scenario_ref_32}
	\end{subfigure}
	\hfill
	\begin{subfigure}[b]{0.32\textwidth}
		\centering
		\includegraphics[scale=0.7]{figure8h.png}
		\caption{Prediction}
		\label{fig:third_scenario_pred_32}
	\end{subfigure}
	\hfill
	\begin{subfigure}[b]{0.32\textwidth}
		\centering
		\includegraphics[scale=0.7]{figure8i.png}
		\caption{Label}
		\label{fig:third_scenario_lab_32}
	\end{subfigure}
	
	\caption{Third Scenario: Comparison of predicted frames with the label 
		frames at $28\textsuperscript{th}$, $30\textsuperscript{th}$, and 
		$32\textsuperscript{th}$ frame after the interaction with delamination.}
	\label{fig:third_scenario}
\end{figure}
%%%%%%%%%%%%%%%%%%%%%%%%%%%%%%%%%%%%%%%%%%%%%%%%%%%%%%%%%%%%%%%%%%%%%%%%%%%%%%%%
As can be seen in the first and second scenario (Fig~\ref{fig:first_scenario}, 
and Fig~\ref{fig:second_scenario}),  the delamination is occurred at the 
top-left of the plate, whereas in the third scenario, 
Fig~\ref{fig:third_scenario} the 
delamination is occurred at the top-center of the plate. 
As shown in Fig~\ref{fig:first_scenario}, the delamination is very tiny in this 
case and not clearly seen by human eyes, but our deep learning algorithm is 
somehow able to reconstruct the full wavefield along with the delamination. 
From all of the 
Figs~\ref{fig:first_scenario_pred_28},~\ref{fig:first_scenario_pred_30},
~\ref{fig:first_scenario_pred_32},~\ref{fig:second_scenario_pred_28},
~\ref{fig:second_scenario_pred_30},~\ref{fig:second_scenario_pred_32},
~\ref{fig:third_scenario_pred_28},~\ref{fig:third_scenario_pred_30} 
and~\ref{fig:third_scenario_pred_32} it can be confirmed the that the proposed 
deep learning-based surrogate model has reconstructed the full wavefield 
containing delamination with minimal error. 
Furthermore, the PSNR and Pearson CC values of all these three scenarios are 
shown in Table~\ref{tab:psnr_pearson}. 
%\newpage%
%%%%%%%%%%%%%%%%%%%%%%%%%%%%%%%%%%%%%%%%%%%%%%%%%%%%%%%%%%%%%%%%%%%%%%%%%%%%%%%%
% Please add the following required packages to your document preamble:
% \usepackage{booktabs}
\begin{table}[ht]
	\centering
	\caption{Evaluation metric for three numerical cases}
	\begin{tabular}{@{}ccc@{}}
		\toprule
		case number       & PSRN    & Pearson CC \\ \midrule
		1 ( $28\textsuperscript{th}$ frame) & 22.3 dB & 0.96       \\ \midrule
		1 ( $30\textsuperscript{th}$ frame)    & 22.7 dB & 0.98       \\ 
		\midrule
		1 ( $32\textsuperscript{th}$ frame)    & 23.1 dB & 0.98       \\ 
		\midrule
		2 ( $28\textsuperscript{th}$ frame) & 22.0 dB & 0.96       \\ \midrule
		2 ( $30\textsuperscript{th}$ frame)    & 22.6 dB & 0.98       \\ 
		\midrule
		2 ( $32\textsuperscript{th}$ frame)    & 23.0 dB & 0.98       \\ 
		\midrule
		3 ( $28\textsuperscript{th}$ frame) & 21.8 dB & 0.97       \\ \midrule
		3 ( $30\textsuperscript{th}$ frame) & 22.3 dB & 0.98       \\ \midrule
		3 ( $32\textsuperscript{th}$ frame)    & 23.2 dB & 0.99       \\ 
		\bottomrule
	\end{tabular}
	\label{tab:psnr_pearson}
\end{table}
%%%%%%%%%%%%%%%%%%%%%%%%%%%%%%%%%%%%%%%%%%%%%%%%%%%%%%%%%%%%%%%%%%%%%%%%%%%%%%%%


The mean PSNR value was 21.8 dB, and the mean Pearson CC value was 0.98 
on all of the test data.
%%%%%%%%%%%%%%%%%%%%%%%%%%%%%%%%%%%%%%%%%%%%%%%%%%%%%%%%%%%%%%%%%%%%%%%%%%%%%%%%