%Version 3 October 2023
% See section 11 of the User Manual for version history
%
%%%%%%%%%%%%%%%%%%%%%%%%%%%%%%%%%%%%%%%%%%%%%%%%%%%%%%%%%%%%%%%%%%%%%%
%%                                                                 %%
%% Please do not use \input{...} to include other tex files.       %%
%% Submit your LaTeX manuscript as one .tex document.              %%
%%                                                                 %%
%% All additional figures and files should be attached             %%
%% separately and not embedded in the \TeX\ document itself.       %%
%%                                                                 %%
%%%%%%%%%%%%%%%%%%%%%%%%%%%%%%%%%%%%%%%%%%%%%%%%%%%%%%%%%%%%%%%%%%%%%

%%\documentclass[referee,sn-basic]{sn-jnl}% referee option is meant for double 
%%%line spacing

%%=======================================================%%
%% to print line numbers in the margin use lineno option %%
%%=======================================================%%

%%\documentclass[lineno,sn-basic]{sn-jnl}% Basic Springer Nature Reference 
%%%Style/Chemistry Reference Style

%%======================================================%%
%% to compile with pdflatex/xelatex use pdflatex option %%
%%======================================================%%

%%\documentclass[pdflatex,sn-basic]{sn-jnl}% Basic Springer Nature Reference 
%%%Style/Chemistry Reference Style


%%Note: the following reference styles support Namedate and Numbered 
%%referencing. By default the style follows the most common style. To switch 
%%between the options you can add or remove “Numbered” in the optional 
%%parenthesis. 
%%The option is available for: sn-basic.bst, sn-vancouver.bst, sn-chicago.bst%  

%%\documentclass[sn-nature]{sn-jnl}% Style for submissions to Nature Portfolio 
%%%journals
%%\documentclass[sn-basic]{sn-jnl}% Basic Springer Nature Reference 
%%%Style/Chemistry Reference Style
\documentclass[sn-mathphys-num]{sn-jnl}% Math and Physical Sciences Numbered 
%Reference Style 
%%\documentclass[sn-mathphys-ay]{sn-jnl}% Math and Physical Sciences Author 
%%%Year Reference Style
%%\documentclass[sn-aps]{sn-jnl}% American Physical Society (APS) Reference 
%%%Style
%%\documentclass[sn-vancouver,Numbered]{sn-jnl}% Vancouver Reference Style
%%\documentclass[sn-apa]{sn-jnl}% APA Reference Style 
%%\documentclass[sn-chicago]{sn-jnl}% Chicago-based Humanities Reference Style

%%%% Standard Packages
%%<additional latex packages if required can be included here>

\usepackage{graphicx}%
\usepackage{multirow}%
\usepackage{amsmath,amssymb,amsfonts}%
\usepackage{amsthm}%
\usepackage{mathrsfs}%
\usepackage[title]{appendix}%
\usepackage{xcolor}%
\usepackage{textcomp}%
\usepackage{manyfoot}%
\usepackage{booktabs}%
\usepackage{algorithm}%
\usepackage{algorithmicx}%
\usepackage{algpseudocode}%
\usepackage{listings}%
%\usepackage{subfigure}

%%%%

%%%%%=============================================================================%%%%
%%%%  Remarks: This template is provided to aid authors with the preparation
%%%%  of original research articles intended for submission to journals 
%%%%published 
%%%%  by Springer Nature. The guidance has been prepared in partnership with 
%%%%  production teams to conform to Springer Nature technical requirements. 
%%%%  Editorial and presentation requirements differ among journal portfolios 
%%%%and 
%%%%  research disciplines. You may find sections in this template are 
%%%%irrelevant 
%%%%  to your work and are empowered to omit any such section if allowed by the 
%%%%  journal you intend to submit to. The submission guidelines and policies 
%%%%  of the journal take precedence. A detailed User Manual is available in 
%%%%the 
%%%%  template package for technical guidance.
%%%%%=============================================================================%%%%

%% as per the requirement new theorem styles can be included as shown below
\theoremstyle{thmstyleone}%
\newtheorem{theorem}{Theorem}%  meant for continuous numbers
%%\newtheorem{theorem}{Theorem}[section]% meant for sectionwise numbers
%% optional argument [theorem] produces theorem numbering sequence instead of 
%%independent numbers for Proposition
\newtheorem{proposition}[theorem]{Proposition}% 
%%\newtheorem{proposition}{Proposition}% to get separate numbers for theorem 
%%%and proposition etc.

\theoremstyle{thmstyletwo}%
\newtheorem{example}{Example}%
\newtheorem{remark}{Remark}%

\theoremstyle{thmstylethree}%
\newtheorem{definition}{Definition}%

\raggedbottom
%%\unnumbered% uncomment this for unnumbered level heads

\graphicspath{{Graphics/}{//pkudela_odroid_sensors/ALPHORN/surrogate_modelling_paper/Graphics/}}

\begin{document}
	%\begin{frontmatter}
		%\addcontentsline{toc}{section}{References}
		%% Title, authors and addresses
		%% use the tnoteref command within \title for footnotes;
		%% use the tnotetext command for theassociated footnote;
		%% use the fnref command within \author or \address for footnotes;
		%% use the fntext command for theassociated footnote;
		%% use the corref command within \author for corresponding author 
		%%footnotes;
		%% use the cortext command for theassociated footnote;
		%% use the ead command for the email address,
		%% and the form \ead[url] for the home page:
		%% \title{Title\tnoteref{label1}}
		%% \tnotetext[label1]{}
		%% \author{Name\corref{cor1}\fnref{label2}}
		%% \ead{email address}
		%% \ead[url]{home page}
		%% \fntext[label2]{}
		%% \cortext[cor1]{}
		%% \address{Address\fnref{label3}}
		%% \fntext[label3]{}
		\title{Simulation of full wavefield data with deep learning approach for delamination identification}
		
%%=============================================================%%
%% GivenName	-> \fnm{Joergen W.}
%% Particle	-> \spfx{van der} -> surname prefix
%% FamilyName	-> \sur{Ploeg}
%% Suffix	-> \sfx{IV}
%% \author*[1,2]{\fnm{Joergen W.} \spfx{van der} \sur{Ploeg} 
	%%  \sfx{IV}}\email{iauthor@gmail.com}
%%=============================================================%%

\author[1]{\fnm{Saeed} \sur{Ullah}}\email{sullah@imp.gda.pl}

\author*[2]{\fnm{Pawel} \sur{Kudela}}\email{pk@imp.gda.pl}

\author[3]{\fnm{Abdalraheem} \sur{A. Ijjeh}}\email{iiiauthor@gmail.com}

\author[4]{\fnm{Eleni} \sur{Chatzi}}\email{iiiauthor@gmail.com}

\author[5]{\fnm{Wieslaw} \sur{Ostachowicz}}\email{wieslaw@imp.gda.pl}

\affil*[1, 2, 3, 5]{\orgdiv{Institute of Fluid Flow Machinery}, \orgname{Polish 
Academy of Sciences, Poland}, 
\orgaddress{\street{Generała Józefa Fiszera 14}, \city{Gdańsk}, 
\postcode{80-231}, \country{Poland}}}

\affil[4]{\orgdiv{Department of Civil, Environmental, and Geomatic 
Engineering}, \orgname{ETH Z\"{u}rich, Switzerland}, 
\orgaddress{\street{Stefano-​Franscini-Platz 5}, \city{Z\"{u}rich}, 
\postcode{8093}, \country{Switzerland}}}
		
		\begin{abstract}
			%%%%%%%%%%%%%%%%%%%%%%%%%%%%%%%%%%%%%%%%%%%%%%%%%%%%%%%%%%%%%%%%%%%%%%%%%%%%%%%%
Guided Lamb waves-based systems mostly employ an array of transducers for 
point-wise measurements.
In such systems, scanning laser Doppler vibrometer (SLDV) along-with pulse 
laser or piezoelectric transducers are used for the excitation and measurement 
of guided waves. 
This process of acquiring a full wavefield of guided Lamb waves is very costly, 
troublesome, lengthy, and time-consuming.
One possible solution to tackle this problem is to acquire the Lamb waves in a 
low-resolution form and then apply a compressive sensing (CS) or deep 
learning-based super-resolution approach to that low-resolution form of full 
wavefield data.
In this research work, we applied two deep learning-based super-resolution 
approaches on a large low-resolution dataset of simulation of full wave fields 
of Lamb waves.
The developed deep learning approaches are elaborated and the results obtained 
from both models are compared with the conventional CS approach. 
Furthermore, the validation of the proposed deep learning models in a 
real-world scenario is performed by an experiment on a carbon fibre reinforced 
polymer plate with embedded Teflon inserts simulating delaminations. 
It is concluded that the performance of both models is good as compared to the 
conventional CS approach. 


		\end{abstract}
	
	\keywords{Lamb waves, structural health monitoring, surrogate modeling, 
	delamination identification, deep learning, autoencoders, ConvLSTM}
	%\end{frontmatter}
	%%%%%%%%%%%%%%%%%%%%%%%%%%%%%%%%%%%%%%%%%%%%%%%%%%%%%
	\section{Introduction}
%%%%%%%%%%%%%%%%%%%%%%%%%%%%%%%%%%%%%%%%%%%%%%%%%%%%%%%%%%%%%%%%%%%%%%%%%%%%%%%%
Guided waves, in particular Lamb waves, are often utilised for structural health monitoring (SHM) as well as non-destructive testing (NDT).
For structural health monitoring usually an array of transducers is used for point-wise measurements.
In the case of active guided wave-based SHM, these are usually piezoelectric transducers that can work as actuators and sensors.
It should be noted that round-robin actuator-sensor measurements can be conducted very fast, therefore nearly online monitoring of a structure is possible.

Recently, a lot of research on application of scanning laser Doppler vibrometer (SLDV) for NDT is reported~\cite{Flynn2013,Kudela2015,Kudela2018d,Segers2021,Segers2022}. 
In this method, either piezoelectric transducer or pulse laser is used for guided wave excitation while the measurements by SLDV are taken at one point on the surface of an inspected structure.
The process is repeated for other points automatically in scanning fashion until full wavefield of Lamb waves is acquired.

Full wavefield measurements are taken on a very dense grid of points opposite to sparsely measured signals by sensors.
Hence, deliver much more useful data from which information about damage can be extracted in comparison to signals measured by an array of transducers.
On the other hand, SLDV measurements take much more time than measurements conducted by an array of transducers.
It makes the SLDV approach unsuitable for SHM in which continuous monitoring is required.
But it is very capable for offline NDT applications.

One can imagine that in a future matrix of laser heads instead of a single laser head used nowadays will be developed to reduce SLDV measurement time.
Alternatively, compressive sensing (CS) and/or deep learning super-resolution (DLSR) can be applied.
It means that SLDV measurements can be taken on a low-resolution grid of points and then full wavefield can be reconstructed at high resolution.

CS was originally proposed in the field of statistics~\cite{Candes2006,Donoho2006} and used for efficient acquisition and reconstruction of signals and images.
It assumes that a signal or an image can be represented in a sparse form in another domain with appropriate bases (Fourier, cosine, wavelet).
In such a bases many coefficients are close or equal to zero.
The sparsity can be exploited to recover a signal or image from fewer samples than required by the Nyquist–Shannon sampling theorem.
However, there is no unique solution for estimation of unmeasured data.
Therefore, optimisation methods for solving under-determined system of linear equations that promote sparsity are applied~\cite{Chen1998,VanEwoutBerg2008,VandenBerg2019}.
Moreover, a suitable sampling strategy is required.

Since then, CS have found applications in medical imaging~\cite{Lustig2007}, communication systems~\cite{Gao2018}, and seismology~\cite{Herrmann2012}.
It is also considered in the field of guided waves and ultrasonic signal processing~\cite{Harley2013,Mesnil2016,Perelli2012,Perelli2015,DiIanni2015,KeshmiriEsfandabadi2018,Chang2020}

{\color{RubineRed}

compressive sensing in guided waves
based on dispersion curves and analytic solution
\cite{Harley2013,Mesnil2016}
others

Perelli warp frequency~\cite{Perelli2012}
Perelli best basis~\cite{Perelli2015}
Ianni curvelet transform, Fourier basis is better~\cite{DiIanni2015}
Esfandabadi wavefield images~\cite{KeshmiriEsfandabadi2018}
Chang Shenfang corrosion monitoring ~\cite{Chang2020}

super-resolution literature, mostly images and video games, none for waves

compressive sensing plus super-resolution enhancement (Esfandabadi, de Marchi)
two-step approach
~\cite{Park2017a,KeshmiriEsfandabadi2020}
}

We propose a framework for full wavefield reconstruction of propagating Lamb waves from spatially sparse SLDV measurements of resolution below the Nyquist wavelength $\lambda_N$. 
The Nyquist wavelength is the shortest spatial wavelength that can be accurately recovered from wavefield by sequential observations with spacing $\Delta x$ which is defined as $\lambda_N = 2 \Delta x$. 

For the first time, an end-to-end approach is used in which deep learning neural network is trained on a synthetic dataset and tested on experimental data acquired by using SLDV.
It means that the approach is solely based on DLSR.
It is different than methods presented in the literature which utilize CS theory~\cite{Harley2013,KeshmiriEsfandabadi2018} or CS theory in conjunction with super-resolution convolutional neural networks for wavefield image enhancement~\cite{Park2017a,KeshmiriEsfandabadi2020}.
The efficacy of the developed framework is presented and compared with conventional CS approach.  
The performance of the proposed technique is validated by an experiment performed on a plate made of carbon fibre reinforced polymer (CFRP) with embedded Teflon inserts simulating delaminations.

	%%%%%%%%%%%%%%%%%%%%%%%%%%%%%%%%%%%%%%%%%%%%%%%%%%%%%
	\section{Dataset computation and preprocessing}
\subsection{Dataset computation}
%%%%%%%%%%%%%%%%%%%%%%%%%%%%%%%%%%%%%%%%%%%%%%%%%%%%%%%%%%%%%%%%%%%%%%%%%%%%%%%%%%%%%%%%
In this work, a synthetic dataset of propagating waves in carbon fibre reinforced composite plates was computed by using the parallel implementation of the time domain spectral element method~\cite{Kudela2020}. 
Essentially, the dataset resembles the particle velocity measurements at the bottom surface of the plate acquired by the SLDV in the transverse direction as a response to the piezoelectric transducer excitation placed at the centre of the plate's top surface. 
The input signal was a five-cycle Hann window-modulated sinusoidal tone burst. 
The carrier frequency was assumed to be 50 kHz. 
The total wave propagation time was set to 0.75 ms.
The number of time integration steps was 150000, which was selected for the 
stability of the central difference scheme.

The material was a typical cross-ply CFRP laminate. 
The stacking sequence [0/90]\(_4\) was used in the model. 
The properties of a single ply were as follows [GPa]:
\(C_{11} = 52.55, \, C_{12} = 6.51, \, C_{22} = 51.83, C_{44} = 2.93, C_{55} = 
2.92, C_{66} = 3.81\). 
The assumed mass density was 1522.4 kg/m\textsuperscript{3}.
These properties were selected so that wavefields simulated numerically are matching the wavefields measured by SLDV on real CFRP specimens.
The wavelength of the dominating A0 Lamb wave mode was 21.2 mm.

475 cases were simulated, representing Lamb waves propagation and interaction 
with single delamination for each case. 
The following random factors were used in simulated delamination scenarios:
\begin{itemize}
	\item delamination geometrical size	determined by ellipse minor and major axis randomly selected from the range $10-40$ mm,
	\item delamination angle randomly selected from the range $ 0^{\circ}-180^{\circ}$,
	\item coordinates of the centre of delamination.
\end{itemize}
The delamination modelling was realized by writing custom geometry files which were used to generate unstructured mesh consisting of quadrilateral elements by using gmsh software~\cite{Geuzaine2009}.
An exemplary mesh of quadrilateral elements is shown in Fig.~\ref{fig:random_delam_mesh} in which green elements highlight the delamination whereas red elements represent the location of the piezoelectric actuator.
Next, the mesh was modified by doubling elements and splitting nodes at delamination region.
Additionally, the quadrilateral elements were converted to the 36-node spectral elements by using a custom MATLAB script.
The wave propagation problem was solved by using in-house code of the time domain SEM which was run on GPU.
%%%%%%%%%%%%%%%%%%%%%%%%%%%%%%%%%%%%%%%%%%%%%%%%%%
\begin{figure} [h!]
	\begin{center}
		\includegraphics{figure2.png}
	\end{center}
	\caption{Exemplary mesh containing piezoelectric transducer (red) and random delamination (green) used for Lamb wave propagation modelling.} 
	\label{fig:random_delam_mesh}
\end{figure}
%%%%%%%%%%%%%%%%%%%%%%%%%%%%%%%%%%%%%%%%%%%%%%%%%%
The dataset contains 475 different delamination cases, with 512 frames per case, producing a total number of 243,\,200 frames with a frame size of \((500\times500)\)~pixels representing the geometry of the specimen surface of size \((500\times500)\)~mm\(^{2}\).
The frames in the dataset are 8-bit .png greyscale images.
The spatial size of the wavefield was further downsampled to \((256\times256)\) for the purpose of 
reducing the computational complexity.

It is important to note that input data to the DL model in the form of the binary image representing the respective delamination case as it is presented in Fig.~\ref{fig:complete_flowchart} was insufficient to train a reliable model.
It was necessary to provide additional inputs in the form of reference full wavefield frames (without delaminations).
This is explained along with the DL model in section~\ref{sec:proposed_approach}.
%%%%%%%%%%%%%%%%%%%%%%%%%%%%%%%%%%%%%%%%%%%%%%%%%%%%%%%%%%%%%%%%%%%%%%%%%%%%%%%%
\subsection{Data augmentation}
The best way for improving the generalisation capabilities of the neural network is to acquire more data. 
However, in practice, it can be difficult to acquire more data and the amount of available data for neural network training is limited.
For tackling this issue, one way is to create some fake data based on the original dataset and add it to the training set, which is termed data augmentation. 
Data augmentation is an efficient approach for various computer vision and DL tasks. 
Data augmentation includes randomly cropping a region from the original image, adjusting contrast, rotation for a small angle and flipping images, etc.~\cite{szegedy2015going}.
In this research work, the dataset is composed of 475 delamination cases which is not enough for a targeted DL model to perform well.
Therefore, all the images in the 475 delamination cases are flipped diagonally, horizontally, and vertically in order to enhance the performance of the proposed DL model. 
Therefore, the total dataset after data augmentation is now composed of 1900 \((475\times4 = 1900)\) delamination cases.
%%%%%%%%%%%%%%%%%%%%%%%%%%%%%%%%%%%%%%%%%%%%%%%%%%%%%%%%%%%%%%%%%%%%%%%%%%%%%%%%
\subsection{Dataset division and preprocessing}
For training and evaluation of the proposed DL model, the dataset was divided into two sets: training and testing, with a ratio of \(80\%\) and \(20\%\), respectively.
Moreover, \(20\%\) of the training set was preserved as a validation set to validate the model during the training process.

Moreover, the dataset was normalised to a range of \((0, 1)\) to improve the convergence of the gradient descent algorithm.
Due to memory limitations, \(32\) consecutive frames in each delamination case were selected for DL model training.
Additionally, frames displaying the propagation of guided waves before interaction with the delamination have no features to be extracted.
Hence, only a certain number of frames were selected from the initial occurrence of the interactions with the delamination.
%%%%%%%%%%%%%%%%%%%%%%%%%%%%%%%%%%%%%%%%%%%%%%%%%%%


	%%%%%%%%%%%%%%%%%%%%%%%%%%%%%%%%%%%%%%%%%%%%%%%%%%%%%
	\section{The proposed DL model for supervised learning}
\label{sec:proposed_approach}
%%%%%%%%%%%%%%%%%%%%%%%%%%%%%%%%%%%%%%%%%%%%%%%%%%%%%%%%%%%%%%%%%%%%%%%%%%%%%%%
In this research work, we developed a novel deep ConvLSTM autoencoder-based surrogate model utilising full wavefield frames of Lamb wave propagation for the purpose of data generation for delamination identification in CFRP materials.
The developed DL model takes as an input \(32\) frames without delamination (reference frames) representing the full wavefield and the delamination information of the respective delamination case in the form of binary image for the purpose of producing full wavefield propagation of Lamb waves through space and time (3D matrix).
The most important aspect of the DL model is the prediction of the interaction of Lamb waves with the delamination so that the delamination location, shape, and size can be estimated.
%%%%%%%%%%%%%%%%%%%%%%%%%%%%%%%%%%%%%%%%%%%%%%%%%%%%%%%%%%%%%%%%%%%%%%%%%%%%%%%%
%%%%%%%%%%%%%%%%%%%%%%%%%%%%%%%%%%%%%%%%%%%%%%%%%%
\begin{figure} [h!]
	\begin{center}
		\includegraphics[width=9cm]{figure4.png}
	\end{center}
	\caption{The flowchart of the proposed DL model.} 
	\label{fig:proposed_model}
\end{figure}
%%%%%%%%%%%%%%%%%%%%%%%%%%%%%%%%%%%%%%%%%%%%%%%%%%

The complete flowchart of the proposed DL model is presented in Fig.~\ref{fig:proposed_model}. 
The training and evaluation process of the proposed model can be summarized in 
the following three steps:  
\begin{enumerate}
	\item{\textbf{Feature extraction:} As we have no labels for the dataset, the dataset is composed of delamination cases. 
		So the first task was to extract features from all of the delamination cases, and then use these features as labels in the second step during model training.
		Therefore, in this step, the encoder and decoder parts of the proposed model are trained jointly, so the decoder part can be used separately for full wavefield predictions.
		During this step, the features are extracted with very minimal reconstruction error in a compressed form, which matches the dimensions of the latent space.}
	\item{\textbf{Model training:} In this step, the actual model training is being carried out. 
		The full wavefield frames in a plate without delamination along with the binary image of the respective delamination case are fed into the DL model for training. 
		The features extracted from encoder part of the first step are used as labels in this step, as shown in Fig.~\ref{fig:proposed_model}}.
	\item{\textbf{Evaluation of the proposed DL model on unseen data:} At this stage, both of the pre-trained models (pre-trained decoder from step 1, and pre-trained encoder from step 2) are utilised for the prediction of full wavefield frames on unseen data.
		During this step, the model just takes reference frames with the delamination information and produces the output as the full wavefield frames containing interaction of Lamb waves with delamination.}
\end{enumerate}

%%%%%%%%%%%%%%%%%%%%%%%%%%%%%%%%%%%%%%%%%%%%%%%%%%
\begin{figure} [h!]
	\begin{center}
		\includegraphics[width=11cm]{figure5.png}
	\end{center}
	\caption{The architecture of the proposed ConvLSTM autoencoder model.} 
	\label{fig:convlstm}
\end{figure}
%%%%%%%%%%%%%%%%%%%%%%%%%%%%%%%%%%%%%%%%%%%%%%%%%%

The proposed ConvLSTM autoencoder model takes \(32\) frames as input concatenated with a binary image which is replicated 32 times (see Fig.~\ref{fig:proposed_model}). 
The DL model consists of six ConvLSTM layers.
The first ConvLSTM layer has \(32\) filters, the second and third layer has \(192\) filters, the fourth layer has \(32\) filters, and the last two ConvLSTM layers has \(192\) filters.
The kernel size of the ConvLSTM layers was set to (\(3\times3\)) with a stride of \((1)\). 
Padding was set to "same", which makes the output the same as the input in the case of stride \(1\).
Furthermore, a \(\tanh\) (the hyperbolic tangent) activation function was used within the ConvLSTM layers that output values in a range between (\(-1\) and \(1\)).
Maxpooling and upsampling were applied at each ConvLSTM layer for reducing the size of features and reconstruction purposes, respectively. 
Moreover, a batch normalization technique~\cite{Santurkar2018} was applied at each of the ConvLSTM layers.
At the final output layer, a 2D convolutional layer followed by a sigmoid activation function is applied.

To alleviate the over-fitting, we used an early-stopping mechanism that monitors the validation loss during the training of the model and stops the training of the model after 30 epochs if there is no improvement. 
Adam optimizer was employed for back-propagation and MSE as a loss function for both training steps.

For evaluating the performance of the proposed model, two evaluation metrics, namely peak signal-to-noise ratio (PSNR), and Pearson correlation coefficient (Pearson CC) were utilized. 
The PSNR measures the maximum potential power of a signal and the power of the noise that affects the quality of its representation and is expressed mathematically in Eq.~(\ref{eqn:psnr}):

\begin{equation}
	\mathrm{PSNR}=20 \log _{10} \frac{L}{\sqrt{\mathrm{MSE}}}
	\label{eqn:psnr}
\end{equation}

Where \(L\) denotes the highest degree of variation present in the input image. 
Meanwhile, MSE stands for mean squared error, which represents the discrepancy between the predicted output and the relevant ground truth.

Pearson CC is a metric that estimates the linear connection between two sets of variables, \(\vect{x}\) (which represents the ground truth values) and \(\vect{y}\) (which represents the predicted values). 
The mathematical formula for computing Pearson CC is shown in Eq.~(\ref{eqn:pearsoncc}):

\begin{equation}
	r_{x 
		y}=\frac{\sum_{k=1}^n\left(x_k-\bar{x}\right)\left(y_k-\bar{y}\right)}{\sqrt{\sum_{k=1}^n\left(x_k-\bar{x}\right)^2}
		\sqrt{\sum_{k=1}^n\left(y_k-\bar{y}\right)^2}},
	\label{eqn:pearsoncc}
\end{equation}

where $r_xy$ represents the Pearson CC, \(n\) represents the number of data points in a sample, and $x_k$ and $y_k$ denote the values of the ground truth and predicted values, respectively, for each data point. 
Additionally, $\bar{x}$ denotes the mean value of the sample, $\bar{y}$ represents the mean value of the predicted values. 
The values of $r_{xy}$ ranges from ‘-1’ to ‘+1’. 
Value ‘0’ specifies that there is no relation between the samples and the predicted values. 
A value greater than ‘0’ indicates a positive relationship between the samples and the predicted data, whereas, a value less than ‘0’ represents a negative relationship between them.

The maximum PSNR and Pearson CC values on the validation data were noted as 23.7 dB and 0.99, respectively.

	%%%%%%%%%%%%%%%%%%%%%%%%%%%%%%%%%%%%%%%%%%%%%%%%%%%%%
	\section{Inverse method for damage identification}

A global-best PSO algorithm implemented in Python was used in the optimisation process~\cite{MirandaLesterJames}.
It takes a set of candidate solutions and tries to find the best solution using a position-velocity update method. 
It uses a star-topology where each particle is attracted to the best-performing particle.
The algorithm follows two basic steps:
\begin{itemize}
	\item the position update:
	\begin{equation}
		x_i(t+1) = x_i(t) + v_i(t+1),\label{eq:position_update}
	\end{equation}
	\item and the velocity update:
	\begin{equation}
		v_{ij}(t+1) = w\, v_{ij}(t) + c_1\, r_{1j}(t) \,[y_{ij}(t) - x_{ij}(t)] + c_2\, r_{2j}(t)\,[\hat{y}_j(t) - x_{ij}(t)],\label{eq:velocity_update}
	\end{equation}
\end{itemize}
where $r$ are random numbers, $y_{ij}$ is the particle's best-known position, $\hat{y}_j$ is the swarm's best known position, $c_1$ is the cognitive parameter, $c_2$ is the social parameter and $w$ is the inertia parameter which controls the swarm's movement.
Cognitive and social parameters control the particle's behaviour given two choices: (i) to follow its personal best or (ii) follow the swarm’s global best position.
Overall, this dictates if the swarm is explorative or exploitative in nature. 
In our tests, we used the following parameters: $c_1 = c_2 = 0.3$ and $w=0.8$.
Good convergence was achieved for these set of parameters, therefore further parameter tuning was unnecessary.

The following decision variables were used in the PSO:
\begin{itemize}
	\item delamination coordinates $(x_c, y_c)$ with bounds [0 mm, 500 m],
	\item delamination elliptic shape represented by semi-major and semi-minor axis $a, b$ with bounds [5 mm, 20 mm],
	\item delamination rotation angle $\alpha$ with bounds [$0^\circ$, $180^\circ$].
\end{itemize}

Based on decision variables, binary images of $(256\times256)$ pixels are generated (one image per particle - see Fig.~\ref{fig:complete_flowchart}).
In these images, ones (white pixels) represent delamination whereas zeros (black pixels) represent healthy area.

The most important component of the proposed inverse method is the surrogate DL model described in section~\ref{sec:proposed_approach}.
The trained DL model is used for ultrafast full wavefield prediction as illustrated in Fig.~\ref{fig:complete_flowchart}.
For a single particle and respective binary image, the DL model is evaluated 7 times for 32 consecutive frames giving as an output 224 frames. 
These predicted frames are compared to 'measured' frames by using the MSE metric which is utilised in the objective function.
However, for the sake of replicability of the results and compatibility with the available dataset~\cite{kudela_pawel_2021_5414555}, we used synthetic data instead of measured data (acquired by SLDV).

For each PSO iteration, particles are updated according to Eqs.~(\ref{eq:position_update})-(\ref{eq:velocity_update}).
The termination criterion was assumed as 100 iterations but it was observed that the objective function value converges much faster.
In the final step, the best matching wavefields indicate coordinates, semi-major, semi-minor axis and rotation angle of the elliptic-shaped delamination. 
These parameters are used for a visual representation of the best-matched delamination in the form of binary image compared against the ground truth (see also Fig.~\ref{fig:complete_flowchart}).

As an evaluation metric for assessing the accuracy of the identified 
delamination, we used the intersection over union (IoU), which measures the 
degree to which the predicted delamination overlaps the true delamination. 
It is defined as:
\begin{equation}
	IoU=\frac{Intersection}{Union}=\frac{\hat{Y} \cap Y}{\hat{Y} \cup Y}
	\label{eq:iou}
\end{equation}
where \(\hat{Y}\) is the predicted output, and \(Y\) is the ground truth (true delamination) in the form of binary images.

	%%%%%%%%%%%%%%%%%%%%%%%%%%%%%%%%%%%%%%%%%%%%%%%%%%%%%
	\section{Results and discussions}
In this section, the evaluation of the developed DLSR models based on the numerical test cases representing the LR frames are presented.
Additionally, the developed models were evaluated on an experimental test case to demonstrate their capability of super-resolution image reconstruction.

%%%%%%%%%%%%%%%%%%%%%%%%%%%%%%%%%%%%%%%%%%%%%%%%%%%%%%%%%%%%%%%%%%%%%%%%%%%%%%%%
Two metrics were utilised to evaluate the performance of the developed DLSR models:
The first one is the peak signal-to-noise ratio (PSNR), which refers to the maximum possible power of a signal and the power of the distorting noise that affects the quality of its representation.
Equation~\ref{PSNR} depicts the mathematical representation of the PSNR:
\begin{equation}
	PSNR=20log_{10}\left(\frac{R}{\sqrt{MSE}}\right)
	\label{PSNR}
\end{equation}
where \(R\) represents the maximum fluctuation value that exists in the input image, and its equal to \(255\).
\(MSE\) refers to the mean square error between the predicted output and the corresponding ground truth.

The second metric is the Pearson correlation coefficient (Pearson CC) that measures the linear relationship between two variable sets \textbf{\(X\)} (represents the ground truth values) and \textbf{\(Y\)} (represents the predicted values).
Equation~\ref{Pearson} depicts the mathematical formula to calculate Pearson CC commonly denoted as \(r_{xy}\):
\begin{equation}
	r_{xy} = \frac{\sum_{i=1}^{n}(x_i - \bar{x})(y_i-\bar{y})}{\sqrt{\sum_{i=1}^{n}(x_i - \bar{x})^2}\sqrt{\sum_{i=1}^{n}(y_i - \bar{y})^2}}
	\label{Pearson}
\end{equation}
where \(n\) is the number of sample points, \(x_i\), \(y_i\) are the individual value points representing the ground truth and predicted values, respectively, and \(\bar{x}\) is the mean value of the sample, analogously to \(\bar{y}\).

The developed DLSR models were trained using MSE (L2 norm) loss function.
Further, we implemented the proposed networks with the Keras~\cite{chollet2015keras} API running on the top of TensorFlow, and trained them using NVIDIA RTX 2080 and Tesla V100 GPU. 
The source code is publicly available online.(we can add link here to GitHub)

%%%%%%%%%%%%%%%%%%%%%%%%%%%%%%%%%%%%%%%%%%%%%%%%%%%%%%%%%%%%%%%%%%%%%%%%%%%%%%%%
\subsection{Numerical cases}
%%%%%%%%%%%%%%%%%%%%%%%%%%%%%%%%%%%%%%%%%%%%%%%%%%%%%%%%%%%%%%%%%%%%%%%%%%%%%%%%

\begin{figure} [!h]
	\centering
	\begin{subfigure}[b]{.32\textwidth}
		\centering
		\includegraphics[width=1\textwidth]{HR_image_case_475_frame_1.png}
		\caption{HR frame \\ (PSNR/ Pearson CC)}
		\label{fig:HR_1}
	\end{subfigure}
	\hfill
	\begin{subfigure}[b]{.32\textwidth}
		\centering
		\includegraphics[width=1\textwidth]{SR_output_475_frame_1.png}
		\caption{Model~I \\($46.38\ \text{dB}/\ 0.9991$)}
		\label{fig:num_f1_ijjeh}
	\end{subfigure}
	\hfill
	\begin{subfigure}[b]{.32\textwidth}
		\centering
		\includegraphics[width=1\textwidth]{Saeed_SR_output_475_frame_1.png}
		\caption{Model~II \\ ($48.43\ \text{dB}/\ 0.9995$)}
		\label{fig:num_f1_saeed}	
	\end{subfigure}
	\hfill
	\begin{subfigure}[b]{.32\textwidth}
		\centering
		\includegraphics[width=1\textwidth]{HR_image_case_475_frame_64.png}
		\caption{HR frame \\ (PSNR/ Pearson CC)}
		\label{fig:HR_2}
	\end{subfigure}
	\hfill
	\begin{subfigure}[b]{.32\textwidth}
		\centering
		\includegraphics[width=1\textwidth]{SR_output_475_frame_64.png}
		\caption{Model~I \\ ($45.23\ \text{dB}/\ 0.9978$)}
		\label{fig:num_f64_ijjeh}
	\end{subfigure}
	\hfill
	\begin{subfigure}[b]{.32\textwidth}
		\centering
		\includegraphics[width=1\textwidth]{Saeed_SR_output_475_frame_64.png}
		\caption{Model~II \\ $(48.34\ \text{dB}/\ 0.9990)$}
		\label{fig:num_f64_saeed}	
	\end{subfigure}
	\hfill
	\begin{subfigure}[b]{.32\textwidth}
		\centering
		\includegraphics[width=1\textwidth]{HR_image_case_475_frame_128.png}
		\caption{HR frame \\ (PSNR/ Pearson CC)}
		\label{fig:HR_3}
	\end{subfigure}
	\hfill
	\begin{subfigure}[b]{.32\textwidth}
		\centering
		\includegraphics[width=1\textwidth]{SR_output_475_frame_128.png}
		\caption{Model~I \\ ($42.66\ \text{dB}/\ 0.9949$)}
		\label{fig:num_f128_ijjeh}
	\end{subfigure}
	\hfill
	\begin{subfigure}[b]{.32\textwidth}
		\centering
		\includegraphics[width=1\textwidth]{Saeed_SR_output_475_frame_128.png}
		\caption{Model~II \\ $(44.51\ \text{dB}/\ 0.9966)$}
		\label{fig:num_f128_saeed}	
	\end{subfigure}
	\caption{
		Comparison of reconstructed frames with DLSR Model~I and II for frames $N_f = 91, 154\ \text{and}\ 218 $ respectively. }
	\label{fig:num_results}
\end{figure}
\clearpage
%%%%%%%%%%%%%%%%%%%%%%%%%%%%%%%%%%%%%%%%%%%%%%%%%%%
\begin{figure} [h!]
	\centering
	\begin{subfigure}[b]{1\textwidth}
		\centering
		\includegraphics[scale=1]{frame_metrics_DLSR_model_1_num.png}
%		\caption{}
		\label{fig:num_model_I}
	\end{subfigure} \\
	\begin{subfigure}[b]{1\textwidth}
		\centering
		\includegraphics[scale=1]{frame_metrics_DLSR_model_2_num.png}
%		\caption{}
		\label{fig:num_model_II}
	\end{subfigure}
	\caption{Comparison of reconstruction accuracy for a numerical test case.}
	\label{fig:num_case_475_metrics}
\end{figure}

\subsection{Experimental cases}

%%%%%%%%%%%%%%%%%%%%%%%%%%%%%%%%%%%%%%%%%%%%%%%%%%%%%%%%%%%%%%%%%%%%%%%%%%%%%%%%

\colorbox{green}{Will be written by Maciej}
%%%%%%%%%%%%%%%%%%%%%%%%%%%%%%%%%%%%%%%%%%%%%%%%%%%%%%%%%%%%%%%%%%%%%%%%%%%%%%%%

\begin{figure} [h!]
	\centering
	\includegraphics[scale=.8]{figure8.png}
	\caption{Delamination arrangements in the specimen.}
	\label{fig:specimen}
\end{figure}

\begin{figure} [h!]
	\centering
	\includegraphics[scale=1]{figure9.png}
	\caption{Comparison of reconstruction accuracy depending on the number of measurement points $N_p$.}
	\label{fig:points_metrics}
\end{figure}

\begin{figure} [h!]
	\centering
	\begin{subfigure}[b]{0.32\textwidth}
		\centering
		\includegraphics[scale=0.8]{figure10a.png}
		\caption{Reference}
		\label{fig:frame110_ref}
	\end{subfigure}
	\hfill
	\begin{subfigure}[b]{0.32\textwidth}
		\centering
		\includegraphics[scale=0.8]{figure10b.png}
		\caption{CS: 1024 points}
		\label{fig:frame110_CS1024}
	\end{subfigure}
	\hfill
	\begin{subfigure}[b]{0.32\textwidth}
		\centering
		\includegraphics[scale=0.8]{figure10c.png}
		\caption{CS: 3000 points}
		\label{fig:frame110_CS3000}
	\end{subfigure}	
	\hfill
	\begin{subfigure}[b]{0.32\textwidth}
		\centering
		\includegraphics[scale=0.8]{figure10d.png}
		\caption{CS: 4000 points}
		\label{fig:frame110_CS4000}
	\end{subfigure}
	\hfill
	\begin{subfigure}[b]{0.32\textwidth}
		\centering
		\includegraphics[scale=0.8]{figure10e.png}
		\caption{DLSR: model I}
		\label{fig:frame110_Abdalraheem}
	\end{subfigure}
	\hfill
	\begin{subfigure}[b]{0.32\textwidth}
		\centering
		\includegraphics[scale=0.8]{figure10f.png}
		\caption{DLSR: model II}
		\label{fig:frame110_Saeed}
	\end{subfigure}
	
	\caption{Comparison of reference wavefield with reconstructed one by CS and DLSR for the frame $N_f = 110$. Rectangle box indicates the region of the strongest reflection from delamination.}
	\label{fig:frame110_comparison}
\end{figure} 

\begin{figure} [h!]
	\centering
	\begin{subfigure}[b]{0.32\textwidth}
		\centering
		\includegraphics[scale=0.8]{figure11a.png}
		\caption{Reference}
		\label{fig:frame110delam_ref}
	\end{subfigure}
	\hfill
	\begin{subfigure}[b]{0.32\textwidth}
		\centering
		\includegraphics[scale=0.8]{figure11b.png}
		\caption{CS: 1024 points}
		\label{fig:frame110delam_CS1024}
	\end{subfigure}
	\hfill
	\begin{subfigure}[b]{0.32\textwidth}
		\centering
		\includegraphics[scale=0.8]{figure11c.png}
		\caption{CS: 3000 points}
		\label{fig:frame110delam_CS3000}
	\end{subfigure}	
	\hfill
	\begin{subfigure}[b]{0.32\textwidth}
		\centering
		\includegraphics[scale=0.8]{figure11d.png}
		\caption{CS: 4000 points}
		\label{fig:frame110delam_CS4000}
	\end{subfigure}
	\hfill
	\begin{subfigure}[b]{0.32\textwidth}
		\centering
		\includegraphics[scale=0.8]{figure11e.png}
		\caption{DLSR: model I}
		\label{fig:frame110delam_Abdalraheem}
	\end{subfigure}
	\hfill
	\begin{subfigure}[b]{0.32\textwidth}
		\centering
		\includegraphics[scale=0.8]{figure11f.png}
		\caption{DLSR: model II}
		\label{fig:frame110delam_Saeed}
	\end{subfigure}
	
	\caption{Comparison of reference wavefield with reconstructed one by CS and DLSR for the region of delamination reflection (close up region of frame $N_f = 110$ as indicated in Fig.~\ref{fig:frame110_comparison}.}
	\label{fig:frame110del_comparison}
\end{figure} 
\clearpage
\begin{figure} [h!]
	\centering
 	\includegraphics[scale=1]{figure12.png}
	\caption{Comparison of reconstruction accuracy at frame number $N_f$.}
	\label{fig:frame_metrics}
\end{figure}

\begin{table}[h!]
	\renewcommand{\arraystretch}{1.3}
	\centering \footnotesize
	\caption{Quality metrics for tested methods for various number of points $N_p$ and corresponding compression ratios CR calculated for the frame no $N_f=110$.}	
	\begin{tabular}{lrrrcrc} 
		\toprule
		& & & \multicolumn{2}{c}{plate} & \multicolumn{2}{c}{delamination} \\
		\cmidrule(lr){4-5} \cmidrule(lr){6-7}
		Method & $N_p$ & CR [\%] & PSNR & PEARSON CC& PSNR & PEARSON CC \\
		\midrule
		\csvreader
		[table head=\toprule,
		late after line=\\ 
		]{table_metrics.csv}{
		1=\one, 2=\two, 3=\three, 4=\four, 5=\five, 6=\six, 7=\seven
		}%
		{\one & \two & \three & \four & \five & \six & \seven }%	
		\bottomrule
	\end{tabular}	
	\label{tab:csv_results}
\end{table}
	%%%%%%%%%%%%%%%%%%%%%%%%%%%%%%%%%%%%%%%%%%%%%%%%%%%%%
	\clearpage
	\section{Conclusions}
\label{conclusion}
In this paper, we demonstrate two super-resolution deep learning techniques to reconstruct the HR full wavefield of propagating guided waves.
The motivation for using two deep learning models was to check the feasibility of different deep learning architectures for our research task. 
Hence, Model-I is computationally complex as it is composed of a large number of trainable parameters that use a Residual Dense Network (RDN) architecture, in which it is composed of many residual dense blocks (RDB). 
On the other hand, Model-II is less complex than Model-I as it is composed of 16 cascaded layers of convolutional neural networks (CNNs).
Accordingly, a large synthetic dataset was generated, resembling the full wavefields acquired by an SLDV.
In DLSR models, the utilised compression rate was \(19.2\%\)\ of the Nyquist sampling rate.
To see the feasibility of such a study, we have compared the DLSR models with the conventional CS technique. 
The results were promising, and the deep learning models surpassed the conventional technique in reconstructing the full wavefield frames for heavily sub-sampled cases.
Additionally, DLSR models showed their ability to generalise by reconstructing the full wavefield frames acquired experimentally by SLDV.
It has been found that the accuracy of the developed DLSR models is similar.
Consequently, DLSR leads to a slightly better reconstruction of the wavefield than CS, and it outperforms it for the reconstruction of the wavefield in the area of delamination.
Furthermore, deep learning techniques for SR can highly enhance the speed of data acquisition by SLDV.


	%%%%%%%%%%%%%%%%%%%%%%%%%%%%%%%%%%%%%%%%%%%%%%%%%%%%%
	
	\clearpage	
	%\appendix
	\section*{Acknowledgments}
	The research was funded by the Polish National Science Center under grant agreements no 2019/01/Y/ST8/00060.
	
	\bibliography{sn-bibliography}% common bib file
	%% if required, the content of .bbl file can be included here once bbl is 
	%%generated
	%%\input sn-article.bbl

\end{document}


