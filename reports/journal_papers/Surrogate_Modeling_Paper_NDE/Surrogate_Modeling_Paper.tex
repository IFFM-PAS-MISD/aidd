%Version 3 October 2023
% See section 11 of the User Manual for version history
%
%%%%%%%%%%%%%%%%%%%%%%%%%%%%%%%%%%%%%%%%%%%%%%%%%%%%%%%%%%%%%%%%%%%%%%
%%                                                                 %%
%% Please do not use \input{...} to include other tex files.       %%
%% Submit your LaTeX manuscript as one .tex document.              %%
%%                                                                 %%
%% All additional figures and files should be attached             %%
%% separately and not embedded in the \TeX\ document itself.       %%
%%                                                                 %%
%%%%%%%%%%%%%%%%%%%%%%%%%%%%%%%%%%%%%%%%%%%%%%%%%%%%%%%%%%%%%%%%%%%%%

%%\documentclass[referee,sn-basic]{sn-jnl}% referee option is meant for double 
%%%line spacing

%%=======================================================%%
%% to print line numbers in the margin use lineno option %%
%%=======================================================%%

%%\documentclass[lineno,sn-basic]{sn-jnl}% Basic Springer Nature Reference 
%%%Style/Chemistry Reference Style

%%======================================================%%
%% to compile with pdflatex/xelatex use pdflatex option %%
%%======================================================%%

%%\documentclass[pdflatex,sn-basic]{sn-jnl}% Basic Springer Nature Reference 
%%%Style/Chemistry Reference Style


%%Note: the following reference styles support Namedate and Numbered 
%%referencing. By default the style follows the most common style. To switch 
%%between the options you can add or remove “Numbered” in the optional 
%%parenthesis. 
%%The option is available for: sn-basic.bst, sn-vancouver.bst, sn-chicago.bst%  

%%\documentclass[sn-nature]{sn-jnl}% Style for submissions to Nature Portfolio 
%%%journals
%%\documentclass[sn-basic]{sn-jnl}% Basic Springer Nature Reference 
%%%Style/Chemistry Reference Style
\documentclass[pdflatex,sn-mathphys-num]{sn-jnl}% Math and Physical Sciences Numbered 
%Reference Style 
%%\documentclass[sn-mathphys-ay]{sn-jnl}% Math and Physical Sciences Author 
%%%Year Reference Style
%%\documentclass[sn-aps]{sn-jnl}% American Physical Society (APS) Reference 
%%%Style
%%\documentclass[sn-vancouver,Numbered]{sn-jnl}% Vancouver Reference Style
%%\documentclass[sn-apa]{sn-jnl}% APA Reference Style 
%%\documentclass[sn-chicago]{sn-jnl}% Chicago-based Humanities Reference Style

%%%% Standard Packages
%%<additional latex packages if required can be included here>

\usepackage{graphicx}%
\usepackage{multirow}%
\usepackage{amsmath,amssymb,amsfonts}%
\usepackage{amsthm}%
\usepackage{mathrsfs}%
\usepackage[title]{appendix}%
\usepackage{xcolor}%
\usepackage{textcomp}%
\usepackage{manyfoot}%
\usepackage{booktabs}%
\usepackage{algorithm}%
\usepackage{algorithmicx}%
\usepackage{algpseudocode}%
\usepackage{listings}%
\usepackage{amsmath} % for \vec command
\usepackage{caption}


%%%%

%%%%%=============================================================================%%%%
%%%%  Remarks: This template is provided to aid authors with the preparation
%%%%  of original research articles intended for submission to journals 
%%%%published 
%%%%  by Springer Nature. The guidance has been prepared in partnership with 
%%%%  production teams to conform to Springer Nature technical requirements. 
%%%%  Editorial and presentation requirements differ among journal portfolios 
%%%%and 
%%%%  research disciplines. You may find sections in this template are 
%%%%irrelevant 
%%%%  to your work and are empowered to omit any such section if allowed by the 
%%%%  journal you intend to submit to. The submission guidelines and policies 
%%%%  of the journal take precedence. A detailed User Manual is available in 
%%%%the 
%%%%  template package for technical guidance.
%%%%%=============================================================================%%%%

%% as per the requirement new theorem styles can be included as shown below
%\theoremstyle{thmstyleone}%
%\newtheorem{theorem}{Theorem}%  meant for continuous numbers
%%\newtheorem{theorem}{Theorem}[section]% meant for sectionwise numbers
%% optional argument [theorem] produces theorem numbering sequence instead of 
%%independent numbers for Proposition
%\newtheorem{proposition}[theorem]{Proposition}% 
%%\newtheorem{proposition}{Proposition}% to get separate numbers for theorem 
%%%and proposition etc.

%\theoremstyle{thmstyletwo}%
%\newtheorem{example}{Example}%
%\newtheorem{remark}{Remark}%

%\theoremstyle{thmstylethree}%
%\newtheorem{definition}{Definition}%

\raggedbottom
%%\unnumbered% uncomment this for unnumbered level heads
\newcommand{\vect}[1]{\mathbf{#1}} % bold upright (Elsevier, Springer)
\graphicspath{{Graphics/}{//pkudela_odroid_sensors/ALPHORN/surrogate_modelling_paper/Graphics/}}

\begin{document}
	%\begin{frontmatter}
		%\addcontentsline{toc}{section}{References}
		%% Title, authors and addresses
		%% use the tnoteref command within \title for footnotes;
		%% use the tnotetext command for theassociated footnote;
		%% use the fnref command within \author or \address for footnotes;
		%% use the fntext command for theassociated footnote;
		%% use the corref command within \author for corresponding author 
		%%footnotes;
		%% use the cortext command for theassociated footnote;
		%% use the ead command for the email address,
		%% and the form \ead[url] for the home page:
		%% \title{Title\tnoteref{label1}}
		%% \tnotetext[label1]{}
		%% \author{Name\corref{cor1}\fnref{label2}}
		%% \ead{email address}
		%% \ead[url]{home page}
		%% \fntext[label2]{}
		%% \cortext[cor1]{}
		%% \address{Address\fnref{label3}}
		%% \fntext[label3]{}
		\title{Simulation of full wavefield data with deep learning approach for delamination identification}
		
%%=============================================================%%
%% GivenName	-> \fnm{Joergen W.}
%% Particle	-> \spfx{van der} -> surname prefix
%% FamilyName	-> \sur{Ploeg}
%% Suffix	-> \sfx{IV}
%% \author*[1,2]{\fnm{Joergen W.} \spfx{van der} \sur{Ploeg} 
	%%  \sfx{IV}}\email{iauthor@gmail.com}
%%=============================================================%%

\author[1]{\fnm{Saeed} \sur{Ullah}}\email{sullah@imp.gda.pl}

\author*[2]{\fnm{Pawel} \sur{Kudela}}\email{pk@imp.gda.pl}

\author[3]{\fnm{Abdalraheem} \sur{A. Ijjeh}}\email{aijjeh@us.es}

\author[4]{\fnm{Eleni} \sur{Chatzi}}\email{chatzi@ibk.baug.ethz.ch}

\author[5]{\fnm{Wieslaw} \sur{Ostachowicz}}\email{wieslaw@imp.gda.pl}

\affil[1,2,5]{\orgdiv{Institute of Fluid Flow Machinery}, \orgname{Polish Academy of Sciences, Poland}, 
\orgaddress{\street{Fiszera 14}, \city{Seville}, \postcode{80-231}, \country{Poland}}}

\affil[3]{\orgdiv{GRVC Robotics Laboratory}, \orgname{University of Seville}, 
	\orgaddress{\street{San Fernando 4}, \city{Seville}, \postcode{41004}, \country{Spain}}}

\affil[4]{\orgdiv{Department of Civil, Environmental, and Geomatic Engineering}, \orgname{ETH Zurich, Switzerland}, \orgaddress{\street{Stefano-Franscini-Platz 5}, \city{Zurich}, \postcode{8093}, \country{Switzerland}}}
		
\abstract{In this work, a novel approach of guided wave-based damage 		 
	identification in composite laminates is proposed. 
	The novelty of this research lies in the implementation of ConvLSTM-based 
	autoencoders for the generation of full wavefield data of propagating 
	guided waves in composite structures.
	The developed surrogate deep learning model takes as input full wavefield 
	frames of propagating waves in a healthy plate along with a binary image 
	representing delamination and predicts frames of propagating waves in a 
	plate which contains single delamination.
	The evaluation of the surrogate model is ultrafast.
	Therefore, unlike traditional forward solvers, the surrogate model can be 
	employed efficiently in the inverse framework of damage identification.
	In this work, particle swarm optimisation is applied as a suitable tool to 
	this end. 
	
	The proposed method was tested on a synthetic dataset showing that it is 
	capable to estimate delamination location and size with a good accuracy.
	The test involved full wavefield data in the objective function of the 
	inverse method but it should be underlined that also partial data with 
	measurements can be implemented.
	This is extremely important for practical applications in structural health 
	monitoring where only signals at a finite number of locations are 
	available.}
	
	\keywords{Lamb waves, structural health monitoring, surrogate modeling, 
	delamination identification, deep learning, autoencoders, ConvLSTM}
	
	\maketitle
	
	%%%%%%%%%%%%%%%%%%%%%%%%%%%%%%%%%%%%%%%%%%%%%%%%%%%%%
	\section{Introduction}\label{sec1}
	%%%%%%%%%%%%%%%%%%%%%%%%%%%%%%%%%%%%%%%%%%%%%%%%%%%%%%%%%%%%%%%%%%%%%%%%%%%%%%%%%%%%%%%%
	
	Detecting delamination in composite materials poses a significant challenge 
	for conventional visual inspection techniques as it occurs between plies of 
	the composite laminate and remains invisible from external 
	surfaces~\cite{staszewski2009health, tuo2019damage}. 
	As a result, various nondestructive testing (NDT) and structural health 
	monitoring (SHM) techniques have been proposed for delamination 
	identification in composite structures. 
	Among these techniques, ultrasonic guided waves are widely recognized as 
	one of the most promising approaches for quantitatively identifying defects 
	in composites~\cite{tian2015delamination, munian2018lamb}. 
	Their extensive application is attributed to their high sensitivity to 
	small defects, low propagation attenuation, and ability to monitor large 
	areas using only a small number of sparsely distributed 
	transducers~\cite{Barthorpe2020, Ihn2008, Cantero-Chinchilla2020}.
	
	However, using a smaller number of transducers does not allow to obtain 
	high-quality resolution damage maps and accurately asses the size of 
	experienced damage. 
	Conversely, employing a very dense array of transducers is impractical in 
	most situations. 
	To address these issues, a scanning laser Doppler vibrometer (SLDV) 
	can be employed. 
	SLDV can measure the propagation of guided waves in a highly dense grid of 
	points over the surface of a large specimen, collectively known as a full 
	wavefield~\cite{Radzienski2019a}. 
	In recent years, full wavefield signals have been utilized for detecting 
	and localizing defects in composite structures~\cite{Radzienski2019a, 
	Girolamo2018a, kudela2018impact, rogge2013characterization}. 
	These damage identification techniques using full wavefield signals can 
	effectively estimate not only the location but also the size of the 
	damage~\cite{Girolamo2018a, kudela2018impact}. 
	Full wavefield scans offer valuable insights into the interaction of guided 
	waves with defects. However, acquiring the full wavefield of guided waves 
	is a time-consuming process.
	
	One possible solution to address this problem involves obtaining Lamb waves 
	in a low-resolution format and subsequently applying compressive sensing 
	(CS) eventually enhanced by deep learning (DL) 
	methods~\cite{esfandabadideep}, such as end-to-end DL-based 
	super-resolution technique~\cite{ijjeh2023deep}. 
	
	Nevertheless, the damage identification methods operating on the full 
	wavefield cannot be directly extended to the cases when only spatially 
	sparse data is available, e.g. an array of sensors is installed on the 
	structure.
	In such a case, usually, an inverse procedure is employed in which 
	efficient methods for solving wave equations are required.  
	However, numerical modelling of ultrasonic guided wave propagation in solid 
	media exhibiting discontinuities, such as damage, is complex, requires fine 
	discretization and is computationally intensive.
	Even methods such as p-version of the finite element method 
	(p-FEM)~\cite{Duczek2013}, the iso-geometric analysis 
	(IGA)~\cite{Anitescu2019}, the spectral cell method 
	(SCM)~\cite{Mossaiby2019} or the time-domain spectral element method 
	(SEM)~\cite{Ostachowicz2012} are not efficient enough.
	In the end, calling the objective function for each damage case scenario in 
	which the forward solver is used, is unfeasible.
	
	An alternative approach is to utilize a DL-based surrogate model for 
	generating full wavefield data or time series resembling signals registered 
	at an array of sensors. 
	A surrogate model imitates the behavior of the simulation model while 
	replacing time-consuming forward simulations with approximate solutions.
	
	DL has seen extensive research and successful implementation in the fields 
	of NDT and SHM. 
	Convolutional neural network (CNN) is among the most popular DL 
	architectures. 
	Initially introduced for image processing, CNN has now been extended to 
	various research domains, including NDT and SHM applications such as damage 
	detection, localization, and characterization~\cite{rautela2019deep, 
	pandey2022explainable, ijjeh2021full, ijjeh2022deep}.
	
	In modern DL, the focus is on constructing more efficient and streamlined 
	architectures, wherein key components are accentuated and essential 
	attributes from the original data are preserved. 
	This process of crafting architectures involves the meticulous extraction 
	of feature representations from data, which is commonly referred to as 
	feature engineering. 
	However, this practice demands specialized knowledge and a significant 
	investment of time. 
	Moreover, the intricacies of feature engineering differ across various data 
	types, posing a challenge in establishing universally applicable procedures.
	
	An established option for feature extraction, alleviating feature 
	engineering, 
	is delivered in the form of an autoencoder, a type of neural network that 
	possesses the capability to autonomously learn features from unlabeled data 
	using unsupervised learning techniques. 
	This unique ability obviates the need for extensive feature 
	engineering~\cite{pinaya2020autoencoders, ardelean2023study, 
		simpson2021machine}. 
	Consisting of two integral components, an autoencoder comprises an encoder 
	responsible for mapping inputs to a desired latent space, and a decoder 
	that skilfully reverts the latent space back to the original input domain. 
	By harnessing appropriately curated training data, autoencoders have the 
	capacity to generate a latent representation, thereby negating the 
	necessity for labour-intensive feature engineering endeavours.
	
	However, general autoencoders may not capture spatial features, such as 
	images, or sequential information when dealing with dynamics, like 
	time-series forecasting. 
	To address the limitation of capturing spatial features, the use of a 
	CNN-based autoencoder is recommended, whereas, an RNN-based autoencoder is 
	usually employed for learning features from time-series data.
	Deep CNN-based autoencoders (DCAEs) excel at extracting spatial features 
	from images-based input data, they may not be sufficient for extracting 
	features from sequences of images, particularly in cases involving full 
	wavefield data, which contains numerous sequential frames/images for each 
	delamination scenario. 
	For such situations, ConvLSTM~\cite{shi2015convolutional} is employed. 
	ConvLSTM combines CNN and LSTM, enabling it to effectively learn features 
	from sequences of images. 
	In ConvLSTM architecture, CNN is responsible for learning features from 
	images, while LSTM retains sequential information.
	
	DCAE-based surrogate modelling has been implemented 
	in~\cite{jo2021adaptive, nikolopoulos2022non, sharma2022wave}. 
	Jo et al.~\cite{jo2021adaptive} developed a DCAE framework for the purpose 
	of extracting latent features from spatial properties and investigating 
	adaptive surrogate estimation to sequester $CO_2$ into heterogeneous deep 
	saline aquifers. 
	They used DCAE and a fully-convolutional network for reducing the 
	computational costs and extracting dimensionality-reduced features for 
	conserving spatial characteristics. 
	Nikolopoulos et al.~\cite{nikolopoulos2022non} presented a non-intrusive 
	DL-based surrogate modelling scheme for predictive modelling of complex 
	systems, which they described by parametrized time-dependent partial 
	differential equations. 
	Sharma et al.~\cite{sharma2022wave} proposed a DCAE-based surrogate 
	predictive model for wave propagation. 
	Their model is able to generate data for a given crack location and depth 
	of the one-dimensional rod of isotropic material.
	
	Recently, Peng et al.~\cite{peng2021structural} proposed an encoding 
	convolution long short-term memory (encoding ConvLSTM) framework for 
	building a surrogate structural model with spatiotemporal evolution, 
	estimating structural spatio-temporal states, and predicting dynamic 
	responses under future dynamic load conditions. 
	Zargar and Yuan~\cite{zargar2021impact} presented a hybrid CNN-recurrent 
	neural network (RNN) for handling spatiotemporal information extraction 
	challenges in impact damage detection problems.
	
	In this work, a novel approach is employed to investigate the propagation 
	of guided waves in composite structures with varying instances of 
	delamination. 
	The method involves utilising a deep ConvLSTM autoencoder-based surrogate 
	model. 
	The main function of the developed model is to capture and learn the full 
	wavefield representation present within frames containing delamination 
	scenarios. 
	Subsequently, it transforms this information into a compressed domain known 
	as the latent space.
	
	During training, the encoder is presented with inputs containing both 
	reference frames (without delaminations) and explicit data detailing the 
	shape and location of the delaminations. 
	This process eliminates the need to repeatedly solve the system's governing 
	equations, resulting in significant time and computational cost savings in 
	comparison to forward modelling by using conventional techniques.
	
	The novelty of this research lies in the implementation of ConvLSTM-based 
	autoencoders for the generation of full wavefield data of propagating 
	guided waves in composite structures.
	The DL model for full wavefield prediction was applied for the first time 
	in the inverse problem of damage identification.
	For this purpose, particle swarm optimisation (PSO) was 
	applied~\cite{Keneddy1995}.
	%%%%%%%%%%%%%%%%%%%%%%%%%%%%%%%%%%%%%%%%%%%%%%%%%%%%%%%%%%%%%%%%%%%%%%%%%%%%%%%%%%%%%%%%
	\section{General concept}\label{sec2}
	%%%%%%%%%%%%%%%%%%%%%%%%%%%%%%%%%%%%%%%%%%%%%%%%%%%%%%%%%%%%%%%%%%%%%%%%%%%%%%%%%%%%%%%%
	A framework of the proposed inverse method for damage identification is 
	shown in Fig~\ref{fig:complete_flowchart}.
	It is composed of three building blocks: (i) dataset computation, (ii) 
	supervised learning and (iii) inverse method.
	
	%%%%%%%%%%%%%%%%%%%%%%%%%%%%%%%%%%%%%%%%%%%%%%%%%%%%%%%%%%%%%%%%%%%%%%%%%%%%%%%%
	\begin{figure} [h!]
		\begin{center}
			\includegraphics[width=12cm]{figure1.png}
		\end{center}
		\caption{Flowchart of the proposed inverse method for damage 
		identification.} 
		\label{fig:complete_flowchart}
	\end{figure}
	%%%%%%%%%%%%%%%%%%%%%%%%%%%%%%%%%%%%%%%%%%%%%%%%%%%%%%%%%%%%%%%%%%%%%%%%%%%%%%%%
	
	For the problem of guided wave-based damage identification, it is 
	infeasible to collect a large experimental dataset which would cover 
	various damage case scenarios because it would require manufacturing and 
	damaging multiple samples.
	Therefore, a possible alternative is a dataset computed numerically by 
	using e.g. a finite element solver.
	In particular, the dataset which is employed in the current research was 
	generated by using the time domain spectral element 
	method~\cite{Kudela2020}.
	The dataset focuses on delamination defects as it these represent the most 
	dangerous types of damage occurring in composite laminates such as carbon 
	fibre-reinforced polymers (CFRP).
	The process of generating this dataset involved modelling a square plate 
	with delaminations of varying shapes and positions into the parametric 
	unstructured mesh. 
	Following this, a forward solver was utilised to capture the interactions 
	between the propagating guided waves with the delamination and the 
	boundaries of the plate.
	It resulted in 512 frames for each delamination scenario.
	The dataset is made available online~\cite{kudela_pawel_2021_5414555} so 
	that the results can be easily replicated.
	
	The dataset was used for the supervised training of the DL model.
	The idea was to input to the DL model a binary image in which ones (white 
	pixels) represent  the delamination region and zeros (black pixels) 
	represent the healthy region for the respective delamination cases.
	Once the model is trained, it can be used in the inverse procedure as a 
	surrogate model instead of a computationally expensive p-FEM or SEM forward 
	solver.
	
	It should be noted that the PSO algorithm was used in the inverse method 
	due to its efficiency in handling more general formulations of objective 
	functions (as opposed to stricter algebraic constructs), but other 
	algorithms can be used as well. 
	On one hand, inputs are initial particles represented by binary images in 
	Fig~\ref{fig:complete_flowchart}.
	These are fed to the trained DL model which in turn is predicting the full 
	wavefield of propagating waves for respective delamination scenarios.
	On the other hand, full wavefield measurements are acquired. 
	An objective function is built upon the computation of the mean square 
	error (MSE) between the predicted wavefield and the measured wavefield.
	Next, particle positions and their velocities are updated accordingly until 
	the termination criterion is met.
	Finally, the identified delamination is visualized for the best match.
	It should be added that the proposed method was validated on synthetic data 
	only. 
	
	The next sections describe each building block of the proposed method in 
	detail.
	%%%%%%%%%%%%%%%%%%%%%%%%%%%%%%%%%%%%%%%%%%%%%%%%%%%%%
	\section{Dataset computation and preprocessing}\label{sec3}
	\subsection{Dataset computation}\label{subsec1}
	%%%%%%%%%%%%%%%%%%%%%%%%%%%%%%%%%%%%%%%%%%%%%%%%%%%%%%%%%%%%%%%%%%%%%%%%%%%%%%%%%%%%%%%%
	In this work, a synthetic dataset of propagating waves in carbon fibre 
	reinforced composite plates was computed by using the parallel 
	implementation of the time domain spectral element 
	method~\cite{Kudela2020}. 
	Essentially, the dataset resembles the particle velocity measurements at 
	the bottom surface of the plate acquired by the SLDV in the transverse 
	direction as a response to the piezoelectric transducer excitation placed 
	at the centre of the plate's top surface. 
	The input signal was a five-cycle Hann window-modulated sinusoidal tone 
	burst. 
	The carrier frequency was assumed to be 50 kHz. 
	The total wave propagation time was set to 0.75 ms.
	The number of time integration steps was 150000, which was selected for the 
	stability of the central difference scheme.
	
	The material was a typical cross-ply CFRP laminate. 
	The stacking sequence [0/90]\(_4\) was used in the model. 
	The properties of a single ply were as follows [GPa]:
	\(C_{11} = 52.55, \, C_{12} = 6.51, \, C_{22} = 51.83, C_{44} = 2.93, 
	C_{55} = 
	2.92, C_{66} = 3.81\). 
	The assumed mass density was 1522.4 kg/m\textsuperscript{3}.
	These properties were selected so that wavefields simulated numerically are 
	matching the wavefields measured by SLDV on real CFRP specimens.
	The wavelength of the dominating A0 Lamb wave mode was 21.2 mm.
	
	475 cases were simulated, representing Lamb waves propagation and 
	interaction 
	with single delamination for each case. 
	The following random factors were used in simulated delamination scenarios:
	\begin{itemize}
		\item coordinates of the centre of delamination,
		\item delamination geometrical size	determined by ellipse minor and 
		major axis randomly selected from the range $10-40$ mm,
		\item delamination angle randomly selected from the range $ 
		0^{\circ}-180^{\circ}$.
		
	\end{itemize}
	The delamination modelling was realized by writing custom geometry files 
	which were used to generate unstructured mesh consisting of quadrilateral 
	elements by using gmsh software~\cite{Geuzaine2009}.
	An exemplary mesh of quadrilateral elements is shown in 
	Fig.~\ref{fig:random_delam_mesh} in which green elements highlight the 
	delamination whereas red elements represent the location of the 
	piezoelectric actuator.
	Next, the mesh was modified by doubling elements and splitting nodes at 
	delamination region.
	Additionally, the quadrilateral elements were converted to the 36-node 
	spectral elements by using a custom MATLAB script.
	The wave propagation problem was solved by using in-house code of the time 
	domain SEM which was run on GPU.
	%%%%%%%%%%%%%%%%%%%%%%%%%%%%%%%%%%%%%%%%%%%%%%%%%%
	\begin{figure} [h!]
		\begin{center}
			\includegraphics{figure2.png}
		\end{center}
		\caption{Exemplary mesh containing piezoelectric transducer (red) and 
		random delamination (green) used for Lamb wave propagation modelling.} 
		\label{fig:random_delam_mesh}
	\end{figure}
	%%%%%%%%%%%%%%%%%%%%%%%%%%%%%%%%%%%%%%%%%%%%%%%%%%
	The dataset contains 475 different delamination cases, with 512 frames per 
	case, producing a total number of 243,\,200 frames with a frame size of 
	\((500\times500)\)~pixels representing the geometry of the specimen surface 
	of size \((500\times500)\)~mm\(^{2}\).
	The frames in the dataset are 8-bit .png greyscale images.
	The spatial size of the wavefield was further downsampled to 
	\((256\times256)\) for the purpose of 
	reducing the computational complexity.
	
	It is important to note that input data to the DL model in the form of the 
	binary image representing the respective delamination case as it is 
	presented in Fig.~\ref{fig:complete_flowchart} was insufficient to train a 
	reliable model.
	It was necessary to provide additional inputs in the form of reference full 
	wavefield frames (without delaminations).
	This is explained along with the DL model in 
	section~\ref{sec:proposed_approach}.
	%%%%%%%%%%%%%%%%%%%%%%%%%%%%%%%%%%%%%%%%%%%%%%%%%%%%%%%%%%%%%%%%%%%%%%%%%%%%%%%%
	\subsection{Data augmentation}\label{subsec2}
	The best way for improving the generalisation capabilities of the neural 
	network is to acquire more data. 
	However, in practice, it can be difficult to acquire more data and the 
	amount of available data for neural network training is limited.
	For tackling this issue, one way is to create some fake data based on the 
	original dataset and add it to the training set, which is termed data 
	augmentation. 
	Data augmentation is an efficient approach for various computer vision and 
	DL tasks. 
	Data augmentation includes randomly cropping a region from the original 
	image, adjusting contrast, rotation for a small angle and flipping images, 
	etc.~\cite{szegedy2015going}.
	In this research work, the dataset is composed of 475 delamination cases 
	which is not enough for a targeted DL model to perform well.
	Therefore, all the images in the 475 delamination cases are flipped 
	diagonally, horizontally, and vertically in order to enhance the 
	performance of the proposed DL model. 
	Therefore, the total dataset after data augmentation is now composed of 
	1900 \((475\times4 = 1900)\) delamination cases.
	%%%%%%%%%%%%%%%%%%%%%%%%%%%%%%%%%%%%%%%%%%%%%%%%%%%%%%%%%%%%%%%%%%%%%%%%%%%%%%%%
	\subsection{Dataset division and preprocessing}\label{subsec3}
	For training and evaluation of the proposed DL model, the dataset was 
	divided into two sets: training and testing, with a ratio of \(80\%\) and 
	\(20\%\), respectively.
	Moreover, \(20\%\) of the training set was preserved as a validation set to 
	validate the model during the training process.
	
	Moreover, the dataset was normalised to a range of \((0, 1)\) to improve 
	the convergence of the gradient descent algorithm.
	Due to memory limitations, \(32\) consecutive frames in each delamination 
	case were selected for DL model training.
	Additionally, frames displaying the propagation of guided waves before 
	interaction with the delamination have no features to be extracted.
	Hence, only a certain number of frames were selected from the initial 
	occurrence of the interactions with the delamination.
	%%%%%%%%%%%%%%%%%%%%%%%%%%%%%%%%%%%%%%%%%%%%%%%%%%%%%
	\section{The proposed DL model for supervised learning}
	\label{sec:proposed_approach}
	%%%%%%%%%%%%%%%%%%%%%%%%%%%%%%%%%%%%%%%%%%%%%%%%%%%%%%%%%%%%%%%%%%%%%%%%%%%%%%%
	In this research work, we developed a novel deep ConvLSTM autoencoder-based 
	surrogate model utilising full wavefield frames of Lamb wave propagation 
	for the purpose of data generation for delamination identification in thin 
	plates made of composite materials.
	The developed DL model takes as an input \(32\) frames without delamination 
	(reference frames) representing the full wavefield and the delamination 
	information of the respective delamination case in the form of binary image 
	for the purpose of producing full wavefield propagation of Lamb waves 
	through space and time (3D matrix).
	The most important aspect of the DL model is the prediction of the 
	interaction of Lamb waves with the delamination so that the delamination 
	location, shape, and size can be estimated.
	%%%%%%%%%%%%%%%%%%%%%%%%%%%%%%%%%%%%%%%%%%%%%%%%%%%%%%%%%%%%%%%%%%%%%%%%%%%%%%%%
	%%%%%%%%%%%%%%%%%%%%%%%%%%%%%%%%%%%%%%%%%%%%%%%%%%
	\begin{figure} [h!]
		\begin{center}
			\includegraphics[width=9cm]{figure4.png}
		\end{center}
		\caption{The flowchart of the proposed DL model.} 
		\label{fig:proposed_model}
	\end{figure}
	%%%%%%%%%%%%%%%%%%%%%%%%%%%%%%%%%%%%%%%%%%%%%%%%%%
	
	The complete flowchart of the proposed DL model is presented in 
	Fig.~\ref{fig:proposed_model}. 
	The training and evaluation process of the proposed model can be summarized 
	in 
	the following three steps:  
	\begin{enumerate}
		\item{\textbf{Feature extraction:} As we have no labels for the 
		dataset, the dataset is composed of delamination cases. 
			So the first task was to extract features from all of the 
			delamination cases, and then use these features as labels in the 
			second step during model training.
			Therefore, in this step, the encoder and decoder parts of the 
			proposed model are trained jointly, so the decoder part can be used 
			separately for full wavefield predictions.
			During this step, the features are extracted with very minimal 
			reconstruction error in a compressed form, which matches the 
			dimensions of the latent space.}
		\item{\textbf{Model training:} In this step, the actual model training 
		is being carried out. 
			The full wavefield frames in a plate without delamination along 
			with the binary image of the respective delamination case are fed 
			into the DL model for training. 
			The features extracted from encoder part of the first step are used 
			as labels in this step, as shown in Fig.~\ref{fig:proposed_model}}.
		\item{\textbf{Evaluation of the proposed DL model on unseen data:} At 
		this stage, both of the pre-trained models (pre-trained decoder from 
		step 1, and pre-trained encoder from step 2) are utilised for the 
		prediction of full wavefield frames on unseen data.
			During this step, the model just takes reference frames with the 
			delamination information and produces the output as the full 
			wavefield frames containing interaction of Lamb waves with 
			delamination.}
	\end{enumerate}
	
	%%%%%%%%%%%%%%%%%%%%%%%%%%%%%%%%%%%%%%%%%%%%%%%%%%
	\begin{figure} [h!]
		\begin{center}
			\includegraphics[width=11cm]{figure5.png}
		\end{center}
		\caption{The architecture of the proposed ConvLSTM autoencoder model.} 
		\label{fig:convlstm}
	\end{figure}
	%%%%%%%%%%%%%%%%%%%%%%%%%%%%%%%%%%%%%%%%%%%%%%%%%%
	
	The proposed ConvLSTM autoencoder model takes \(32\) frames as input 
	concatenated with a binary image which is replicated 32 times (see 
	Fig.~\ref{fig:proposed_model}). 
	The DL model consists of six ConvLSTM layers.
	The first ConvLSTM layer has \(32\) filters, the second and third layer has 
	\(192\) filters, the fourth layer has \(32\) filters, and the last two 
	ConvLSTM layers has \(192\) filters.
	The kernel size of the ConvLSTM layers was set to (\(3\times3\)) with a 
	stride of \((1)\). 
	Padding was set to "same", which makes the output the same as the input in 
	the case of stride \(1\).
	Furthermore, a \(\tanh\) (the hyperbolic tangent) activation function was 
	used within the ConvLSTM layers that output values in a range between 
	(\(-1\) and \(1\)).
	Maxpooling and upsampling were applied at each ConvLSTM layer for reducing 
	the size of features and reconstruction purposes, respectively. 
	Moreover, a batch normalization technique~\cite{Santurkar2018} was applied 
	at each of the ConvLSTM layers.
	At the final output layer, a 2D convolutional layer followed by a sigmoid 
	activation function is applied.
	
	To alleviate the over-fitting, we used an early-stopping mechanism that 
	monitors the validation loss during the training of the model and stops the 
	training of the model after 30 epochs if there is no improvement. 
	Adam optimizer was employed for back-propagation and MSE as a loss function 
	for both training steps.
	
	For evaluating the performance of the proposed model, two evaluation 
	metrics, namely peak signal-to-noise ratio (PSNR), and Pearson correlation 
	coefficient (Pearson CC) were utilized. 
	The PSNR measures the maximum potential power of a signal and the power of 
	the noise that affects the quality of its representation and is expressed 
	mathematically in Eq.~(\ref{eqn:psnr}):
	
	\begin{equation}
		\mathrm{PSNR}=20 \log _{10} \frac{L}{\sqrt{\mathrm{MSE}}}
		\label{eqn:psnr}
	\end{equation}
	
	Where \(L\) denotes the highest degree of variation present in the input 
	image. 
	Meanwhile, MSE stands for mean squared error, which represents the 
	discrepancy 
	between the predicted output and the relevant ground truth. The calculation 
	of 
	the MSE is shown in Eq.~(\ref{eqn:mse}): 
	
	\begin{equation}
		M S E=\frac{1}{M * N} \sum_{M, N}\left(Y_{(m, n)}-\hat{Y}_{(m, 
		n)}\right)^2
		\label{eqn:mse}
	\end{equation}
	
	Where \(M\) and \(N\) represent the number of rows and columns in the input 
	images, $Y_{(m, n)}$ is the ground truth value, and $\hat{Y}_{(m, n)}$ is 
	the predicted value.
	
	Pearson CC is a metric that estimates the linear connection between two 
	sets of variables, \(\vect{x}\) (which represents the ground truth values) 
	and \(\vect{y}\) (which represents the predicted values). 
	The mathematical formula for computing Pearson CC is shown in 
	Eq.~(\ref{eqn:pearsoncc}):
	
	\begin{equation}
		r_{x 
			y}=\frac{\sum_{k=1}^n\left(x_k-\bar{x}\right)\left(y_k-\bar{y}\right)}{\sqrt{\sum_{k=1}^n\left(x_k-\bar{x}\right)^2}
			\sqrt{\sum_{k=1}^n\left(y_k-\bar{y}\right)^2}},
		\label{eqn:pearsoncc}
	\end{equation}
	
	where $r_{xy}$ represents the Pearson CC, \(n\) represents the number of 
	data points in a sample, and $x_k$ and $y_k$ denote the values of the 
	ground truth and predicted values, respectively, for each data point. 
	Additionally, $\bar{x}$ denotes the mean value of the sample, $\bar{y}$ 
	represents the mean value of the predicted values. 
	The values of $r_{xy}$ ranges from ‘-1’ to ‘+1’. 
	Value ‘0’ specifies that there is no relation between the samples and the 
	predicted values. 
	A value greater than ‘0’ indicates a positive relationship between the 
	samples and the predicted data, whereas, a value less than ‘0’ represents a 
	negative relationship between them.
	
	The maximum PSNR and Pearson CC values on the validation data were noted as 
	23.7 dB and 0.99, respectively.
	%%%%%%%%%%%%%%%%%%%%%%%%%%%%%%%%%%%%%%%%%%%%%%%%%%%%%
	\section{Inverse method for damage identification}
	
	A global-best PSO algorithm implemented in Python was used in the 
	optimisation process~\cite{MirandaLesterJames}.
	It takes a set of candidate solutions and tries to find the best solution 
	using a position-velocity update method. 
	It uses a star-topology where each particle is attracted to the 
	best-performing particle.
	The algorithm follows two basic steps:
	\begin{itemize}
		\item the position update:
		\begin{equation}
			x_i(t+1) = x_i(t) + v_i(t+1),\label{eq:position_update}
		\end{equation}
		\item and the velocity update:
		\begin{equation}
			v_{ij}(t+1) = w\, v_{ij}(t) + c_1\, r_{1j}(t) \,[y_{ij}(t) - 
			x_{ij}(t)] + c_2\, r_{2j}(t)\,[\hat{y}_j(t) - 
			x_{ij}(t)],\label{eq:velocity_update}
		\end{equation}
	\end{itemize}
	where $r$ are random numbers, $y_{ij}$ is the particle's best-known 
	position, $\hat{y}_j$ is the swarm's best known position, $c_1$ is the 
	cognitive parameter, $c_2$ is the social parameter and $w$ is the inertia 
	parameter which controls the swarm's movement.
	Cognitive and social parameters control the particle's behaviour given two 
	choices: (i) to follow its personal best or (ii) follow the swarm’s global 
	best position.
	Overall, this dictates if the swarm is explorative or exploitative in 
	nature. 
	In our tests, we used the following parameters: $c_1 = c_2 = 0.3$ and 
	$w=0.8$.
	Good convergence was achieved for these set of parameters, therefore 
	further parameter tuning was unnecessary.
	
	The following decision variables were used in the PSO:
	\begin{itemize}
		\item delamination coordinates $(x_c, y_c)$ with bounds [0 mm, 500 m],
		\item delamination elliptic shape represented by semi-major and 
		semi-minor axis $a, b$ with bounds [5 mm, 20 mm],
		\item delamination rotation angle $\alpha$ with bounds [$0^\circ$, 
		$180^\circ$].
	\end{itemize}
	
	Based on decision variables, binary images of $(256\times256)$ pixels are 
	generated (one image per particle - see Fig.~\ref{fig:complete_flowchart}).
	In these images, ones (white pixels) represent delamination whereas zeros 
	(black pixels) represent healthy area.
	
	The most important component of the proposed inverse method is the 
	surrogate DL model described in section~\ref{sec:proposed_approach}.
	The trained DL model is used for ultrafast full wavefield prediction as 
	illustrated in Fig.~\ref{fig:complete_flowchart}.
	For a single particle and respective binary image, the DL model is 
	evaluated 7 times for 32 consecutive frames giving as an output 224 frames. 
	These predicted frames are compared to 'measured' frames by using the MSE 
	metric which is utilised in the objective function.
	However, for the sake of replicability of the results and compatibility 
	with the available dataset~\cite{kudela_pawel_2021_5414555}, we used 
	synthetic data instead of measured data (acquired by SLDV).
	
	For each PSO iteration, particles are updated according to 
	Eqs.~(\ref{eq:position_update})-(\ref{eq:velocity_update}).
	The termination criterion was assumed as 100 iterations but it was observed 
	that the objective function value converges much faster.
	In the final step, the best matching wavefields indicate coordinates, 
	semi-major, semi-minor axis and rotation angle of the elliptic-shaped 
	delamination. 
	These parameters are used for a visual representation of the best-matched 
	delamination in the form of binary image compared against the ground truth 
	(see also Fig.~\ref{fig:complete_flowchart}).
	
	As an evaluation metric for assessing the accuracy of the identified 
	delamination, we used the intersection over union (IoU), which measures the 
	degree to which the predicted delamination overlaps the true delamination. 
	It is defined as:
	\begin{equation}
		IoU=\frac{Intersection}{Union}=\frac{\hat{Y} \cap Y}{\hat{Y} \cup Y}
		\label{eq:iou}
	\end{equation}
	where \(\hat{Y}\) is the predicted output, and \(Y\) is the ground truth 
	(true delamination) in the form of binary images.
	%%%%%%%%%%%%%%%%%%%%%%%%%%%%%%%%%%%%%%%%%%%%%%%%%%%%%
	\section{Results and discussions}
	%%%%%%%%%%%%%%%%%%%%%%%%%%%%%%%%%%%%%%%%%%%%%%%%%%
	\subsection{Evaluation of the surrogate DL model}
	%%%%%%%%%%%%%%%%%%%%%%%%%%%%%%%%%%%%%%%%%%%%%%%%%%
	In this section, we present the evaluation of the proposed DL model based 
	on numerical test data of \(95\) delamination cases representing the frames 
	of the full wavefield propagation, which was not shown in the proposed DL 
	model during training. 
	The proposed DL model was evaluated using numerical test data to 
	demonstrate the capability to predict the interaction of Lamb waves with 
	delamination of various locations, shapes and sizes.
	
	Three different representative cases were selected from the numerical 
	dataset to show the performance of the developed DL model.
	Figures~\ref{fig:num_415},~\ref{fig:num_453}, 
	and~\ref{fig:num_462} shows three different frames from three selected 
	numerical test cases.  
	Frames on the left column represent the labels to which the prediction of 
	the proposed DL model is compared.
	Frames on the right column represent prediction by the DL model.
	Particular frames were selected to show the interaction of propagating Lamb 
	waves with the delamination, namely $10\textsuperscript{th}$, 
	$20\textsuperscript{th}$ and $30\textsuperscript{th}$ frame after the 
	interaction with the delamination. These frame numbers were easily 
	extracted knowing the A0 mode velocity and modelled delamination location. 
	
	As can be seen in the first scenario (Fig~\ref{fig:num_415}), the 
	delamination occurred at the upper-left side of the plate, in the second 
	scenario (Fig~\ref{fig:num_453}), the delamination occurred at the top-left 
	of the plate whereas in the third scenario, (Fig~\ref{fig:num_462}) the 
	delamination occurred at the top-centre of the plate. 
	
	In all presented cases (Figs.~\ref{fig:num_415}--\ref{fig:num_462}), the 
	change of wave velocity due to delamination is well reproduced by the DL 
	model.
	The wave reflections from the delamination are very well predicted in 
	Fig.~\ref{fig:num_453} whereas in some cases and frames the reflection 
	pattern differs between the label and the DL prediction - compare 
	Fig.~\ref{fig:num_462_label3} to Fig.~\ref{fig:num_462_pred3}.
	It should be underlined that these reflections are of much smaller 
	amplitude than the main wavefront and the proposed DL model is not able to 
	reproduce correctly all detailed intricacies of Lamb wave reflections. 
	A more complex DL model could be required to further improve the results.
	Nevertheless, testing results are satisfactory and very promising.
	
	%%%%%%%%%%%%%%%%%%%%%%%%%%%%%%%%%%%%%%%%%%%%%%%%%%%%%%%%%%%%%%%%%%%%%%%%%%%%%%%%
\begin{figure}
	\centering
	\begin{minipage}[b]{0.44\textwidth}
		\centering
		\includegraphics[width=0.9\textwidth]{figure6a.png}
		\caption*{Label, $10\textsuperscript{th}$ frame}
		\label{fig:num_415_label1}
	\end{minipage}
	\hfill
	\begin{minipage}[b]{0.44\textwidth}
		\centering
		\includegraphics[width=0.9\textwidth]{figure6b.png}
		\caption*{Prediction, $10\textsuperscript{th}$ frame}
		\label{fig:num_415_pred1}
	\end{minipage}
	\\
	\begin{minipage}[b]{0.44\textwidth}
		\centering
		\includegraphics[width=0.9\textwidth]{figure6c.png}
		\caption*{Label, $20\textsuperscript{th}$ frame}
		\label{fig:num_415_label2}
	\end{minipage}
	\hfill
	\begin{minipage}[b]{0.44\textwidth}
		\centering
		\includegraphics[width=0.9\textwidth]{figure6d.png}
		\caption*{Prediction, $20\textsuperscript{th}$ frame}
		\label{fig:num_415_pred2}
	\end{minipage}
	\\
	\begin{minipage}[b]{0.44\textwidth}
		\centering
		\includegraphics[width=0.9\textwidth]{figure6e.png}
		\caption*{Label, $30\textsuperscript{th}$ frame}
		\label{fig:num_415_label3}
	\end{minipage}
	\hfill
	\begin{minipage}[b]{0.44\textwidth}
		\centering
		\includegraphics[width=0.9\textwidth]{figure6f.png}
		\caption*{Prediction, $30\textsuperscript{th}$ frame}
		\label{fig:num_415_pred3}
	\end{minipage}
	\caption{First scenario: comparison of predicted frames with the label 
	frames at $10\textsuperscript{th}$, $20\textsuperscript{th}$, and 
	$30\textsuperscript{th}$ frame after the interaction with delamination.}
	\label{fig:num_415}
\end{figure}
%%%%%%%%%%%%%%%%%%%%%%%%%%%%%%%%%%%%%%%%%%%%%%%%%%%%%%%%%%%%%%%%%%%%%%%%%%%%%%%%%%
%%%%%%%%%%%%%%%%%%%%%%%%%%%%%%%%%%%%%%%%%%%%%%%%%%%%%%%%%%%%%%%%%%%%%%%%%%%%%%%%%%
\begin{figure}
	\centering
	\begin{minipage}[b]{0.44\textwidth}
		\centering
		\includegraphics[width=0.9\textwidth]{figure7a.png}
		\caption*{Label, $10\textsuperscript{th}$ frame}
		\label{fig:num_453_label1}
	\end{minipage}
	\hfill
	\begin{minipage}[b]{0.44\textwidth}
		\centering
		\includegraphics[width=0.9\textwidth]{figure7b.png}
		\caption*{Prediction, $10\textsuperscript{th}$ frame}
		\label{fig:num_453_pred1}
	\end{minipage}
	\\
	\begin{minipage}[b]{0.44\textwidth}
		\centering
		\includegraphics[width=0.9\textwidth]{figure7c.png}
		\caption*{Label, $20\textsuperscript{th}$ frame}
		\label{fig:num_453_label2}
	\end{minipage}
	\hfill
	\begin{minipage}[b]{0.44\textwidth}
		\centering
		\includegraphics[width=0.9\textwidth]{figure7d.png}
		\caption*{Prediction, $20\textsuperscript{th}$ frame}
		\label{fig:num_453_pred2}
	\end{minipage}
	\\
	\begin{minipage}[b]{0.44\textwidth}
		\centering
		\includegraphics[width=0.9\textwidth]{figure7e.png}
		\caption*{Label, $30\textsuperscript{th}$ frame}
		\label{fig:num_453_label3}
	\end{minipage}
	\hfill
	\begin{minipage}[b]{0.44\textwidth}
		\centering
		\includegraphics[width=0.9\textwidth]{figure7f.png}
		\caption*{Prediction, $30\textsuperscript{th}$ frame}
		\label{fig:num_453_pred3}
	\end{minipage}
	\caption{Second scenario: comparison of predicted frames with the label 
	frames at $10\textsuperscript{th}$, $20\textsuperscript{th}$, and 
	$30\textsuperscript{th}$ frame after the interaction with 
	delamination.}
\label{fig:num_453}
\end{figure}
%%%%%%%%%%%%%%%%%%%%%%%%%%%%%%%%%%%%%%%%%%%%%%%%%%%%%%%%%%%%%%%%%%%%%%%%%%%%%%%%%%
%%%%%%%%%%%%%%%%%%%%%%%%%%%%%%%%%%%%%%%%%%%%%%%%%%%%%%%%%%%%%%%%%%%%%%%%%%%%%%%%%%
\begin{figure}
	\centering
	\begin{minipage}[b]{0.44\textwidth}
		\centering
		\includegraphics[width=0.9\textwidth]{figure8a.png}
		\caption*{Label, $10\textsuperscript{th}$ frame}
		\label{fig:num_462_label1}
	\end{minipage}
	\hfill
	\begin{minipage}[b]{0.44\textwidth}
		\centering
		\includegraphics[width=0.9\textwidth]{figure8b.png}
		\caption*{Prediction, $10\textsuperscript{th}$ frame}
		\label{fig:num_462_pred1}
	\end{minipage}
	\\
	\begin{minipage}[b]{0.44\textwidth}
		\centering
		\includegraphics[width=0.9\textwidth]{figure8c.png}
		\caption*{Label, $20\textsuperscript{th}$ frame}
		\label{fig:num_462_label2}
	\end{minipage}
	\hfill
	\begin{minipage}[b]{0.44\textwidth}
		\centering
		\includegraphics[width=0.9\textwidth]{figure8d.png}
		\caption*{Prediction, $20\textsuperscript{th}$ frame}
		\label{fig:num_462_pred2}
	\end{minipage}
	\\
	\begin{minipage}[b]{0.44\textwidth}
		\centering
		\includegraphics[width=0.9\textwidth]{figure8e.png}
		\caption*{Label, $30\textsuperscript{th}$ frame}
		\label{fig:num_462_label3}
	\end{minipage}
	\hfill
	\begin{minipage}[b]{0.44\textwidth}
		\centering
		\includegraphics[width=0.9\textwidth]{figure8f.png}
		\caption*{Prediction, $30\textsuperscript{th}$ frame }
		\label{fig:num_462_pred3}
	\end{minipage}
	\caption{Third scenario: comparison of predicted frames with the label 
	frames at $10\textsuperscript{th}$, $20\textsuperscript{th}$, and 
	$30\textsuperscript{th}$ frame after the interaction with 
	delamination.}
\label{fig:num_462}
\end{figure}
\clearpage
%%%%%%%%%%%%%%%%%%%%%%%%%%%%%%%%%%%%%%%%%%%%%%%%%%%%%%%%%%%%%%%%%%%%%%%%%%%%%%%%
%%%%%%%%%%%%%%%%%%%%%%%%%%%%%%%%%%%%%%%%%%%%%%%%%%%%%%%%%%%%%%%%%%%%%%%%%%%%%%%%
	From all three scenarios, it can be confirmed that the proposed DL-based 
	surrogate model has reconstructed the full wavefield containing 
	delamination with minimal error. 
	Furthermore, the PSNR and Pearson CC values of all these three scenarios 
	are shown in Table~\ref{tab:psnr_pearson}. 
	The mean PSNR value was 21.8 dB, and the mean Pearson CC value was 0.98 on 
	all of the test data.
	It confirms that the predictions by the proposed DL model are accurate.
	
%%%%%%%%%%%%%%%%%%%%%%%%%%%%%%%%%%%%%%%%%%%%%%%%%%%%%%%%%%%%%%%%%%%%%%%%%%%%%%%%
\begin{table}[h]
	\caption{DL surrogate model evaluation metrics for three numerical cases}
	\label{tab:psnr_pearson}
	\begin{tabular*}{\textwidth}{@{\extracolsep{\fill}}llcc}
		\toprule
		\multicolumn{2}{c}{Scenario} & PSNR    & Pearson CC \\ 
		\midrule
		\multirow{3}{*}{First} & $10\textsuperscript{th}$ frame & 23.3 dB & 
		0.96 \\ 
		& $20\textsuperscript{th}$ frame & 23.4 dB & 0.98 \\ 
		& $30\textsuperscript{th}$ frame & 23.7 dB & 0.98 \\ 
		\midrule
		\multirow{3}{*}{Second} & $10\textsuperscript{th}$ frame & 21.4 dB & 
		0.96 \\ 
		& $20\textsuperscript{th}$ frame & 22.1 dB & 0.98 \\ 
		& $30\textsuperscript{th}$ frame & 22.6 dB & 0.98 \\ 
		\midrule
		\multirow{3}{*}{Third} & $10\textsuperscript{th}$ frame & 21.8 dB & 
		0.97 \\ 
		& $20\textsuperscript{th}$ frame & 22.1 dB & 0.98 \\ 
		& $30\textsuperscript{th}$ frame & 22.3 dB & 0.99 \\ 
		\bottomrule
	\end{tabular*}
\end{table}

%%%%%%%%%%%%%%%%%%%%%%%%%%%%%%%%%%%%%%%%%%%%%%%%%%%%%%%%%%%%%%%%%%%%%%%%%%%%%%%
	\subsection{Delamination identification results}
	%%%%%%%%%%%%%%%%%%%%%%%%%%%%%%%%%%%%%%%%%%%%%%%%%%
	The delamination identification results obtained by using PSO aided by the 
	DL-based surrogate model are presented in Fig.~\ref{fig:pso_identification}.
	Several runs were performed due to the meta-heuristic nature of the PSO 
	algorithm and the results were selected for two runs for illustration 
	purposes.
	The following cases were selected, namely case 1 
	(Fig.~\ref{fig:pso_case391_run1}, Fig.~\ref{fig:pso_case391_run2}), case 2 
	(Fig.~\ref{fig:pso_case462_run1}, Fig.~\ref{fig:pso_case462_run2}) and case 
	3 (Fig.~\ref{fig:pso_case453_run1}, Fig.~\ref{fig:pso_case453_run2}),  
	where the damage identification difficulty can be ranked from highest to 
	lowest.
	The most difficult case is case 1 in which delamination is in the corner of 
	the plate.
	The delamination for case 2 is located very close to the top edge of the 
	plate where edge reflections can overshadow reflections from delamination.
	The delamination for case 3 is quite large and far away from the plate's 
	edges so it should be easily detected.
	
	Actually, the damage identification difficulty level is reflected in the 
	obtained IoU values which are shown in the zoomed-in regions around 
	delamination in Fig.~\ref{fig:pso_identification}.
	On average, the lowest IoU values were obtained for case 1.
	
	The visualisation of damage identification results is performed in such a 
	way that the delamination ground truth is shown in green colour, DL model 
	prediction in red colour and the intersection of the two is made by colour 
	mixing which gives yellow colour.
	Therefore, the more yellow pixels, the greater overlap of delaminations 
	and, in turn, better accuracy.
	
	It should be noted that despite low IoU values in certain cases, the 
	identification algorithm performed remarkably well because delamination was 
	localised accurately for each scenario.
	
%%%%%%%%%%%%%%%%%%%%%%%%%%%%%%%%%%%%%%%%%%%%%%%%%%%%%%%%%%%%%%%%%%%%%%%%%%%%%%%%
%%%%%%%%%%%%%%%%%%%%%%%%%%%%%%%%%%%%%%%%%%%%%%%%%%%%%%%%%%%%%%%%%%%%%%%%%%%%%%%%
\begin{figure}
	\centering
	\begin{minipage}[b]{0.44\textwidth}
		\centering
		\includegraphics[]{pso_case391_run1.png}
		\caption*{Case 1, run 1}
		\label{fig:pso_case391_run1}
	\end{minipage}
	\hfill
	\begin{minipage}[b]{0.44\textwidth}
		\centering
		\includegraphics[]{pso_case391_run2.png}
		\caption*{Case 1, run 2}
		\label{fig:pso_case391_run2}
	\end{minipage}
	\\
	\begin{minipage}[b]{0.44\textwidth}
		\centering
		\includegraphics[]{pso_case462_run1.png}
		\caption*{Case 2, run 1}
		\label{fig:pso_case462_run1}
	\end{minipage}
	\hfill
	\begin{minipage}[b]{0.44\textwidth}
		\centering
		\includegraphics[]{pso_case462_run2.png}
		\caption*{Case 2, run 2}
		\label{fig:pso_case462_run2}
	\end{minipage}
	\\
	\begin{minipage}[b]{0.44\textwidth}
		\centering
		\includegraphics[]{pso_case453_run1.png}
		\caption*{Case 3, run 1}
		\label{fig:pso_case453_run1}
	\end{minipage}
	\hfill
	\begin{minipage}[b]{0.44\textwidth}
		\centering
		\includegraphics[]{pso_case453_run2.png}
		\caption*{Case 3, run 2}
		\label{fig:pso_case453_run2}
	\end{minipage}
	\caption{Delamination identification results; green - ground truth, red 
	- prediction, yellow - intersection.}
	\label{fig:pso_identification}
	\end{figure}
%%%%%%%%%%%%%%%%%%%%%%%%%%%%%%%%%%%%%%%%%%%%%%%%%%%%%%%%%%%%%%%%%%%%%%%%%%%%%%%%
%%%%%%%%%%%%%%%%%%%%%%%%%%%%%%%%%%%%%%%%%%%%%%%%%%%%%%%%%%%%%%%%%%%%%%%%%%%%%%%%
	
	The delamination identification results in terms of IoU values are gathered 
	in Table~\ref{tab:iou} where also the results from~\cite{Ullah2023} are 
	added for comparison.
	The method presented in~\cite{Ullah2023} is completely different than the 
	one presented here but it relies on the same dataset.
	However, the frame size was larger, namely \((512\times512)\)~pixels versus 
	\((256\times256)\)~pixels here, giving better resolution of damage 
	identification.
	
	Although, according to Table~\ref{tab:iou}, the current results are not as 
	good as compared to our previous paper~\cite{Ullah2023}, the advantage of 
	the proposed method is that it can be easily extended to the cases in which 
	only a limited number of signals are available in comparison to full 
	wavefield data.
	This is extremely important for practical applications in structural health 
	monitoring where only signals at sensor locations are available.
	
	It should also be stressed, that the complexity of the proposed DL model 
	and dataset size used for training is limited by the memory of a single 
	Nvidia Tesla V100 GPU (32 GB memory) which was available to us.
	Certainly, the surrogate DL model can be improved by using a larger number 
	of frames in the sequence of ConvLSTM layers. 
	It is expected that damage identification would improve as well with a more 
	accurate surrogate DL model.
	
%%%%%%%%%%%%%%%%%%%%%%%%%%%%%%%%%%%%%%%%%%%%%%%%%%%%%%%%%%%%%%%%%%%%%%%%%%%%%%%%
\begin{table}[ht]
	\caption{Damage identification evaluation metrics for three numerical cases}
	\label{tab:iou}
	%\setlength{\tabcolsep}{5pt} % Adjust the column separation
	\begin{tabular*}{\textwidth}{@{\extracolsep{\fill}}cccc}
		\toprule
		\multicolumn{2}{c}{Case Number} & \multicolumn{2}{c}{IoU} \\ 
		\cmidrule(lr){3-4} 
		& & Current & \cite{Ullah2023}\\
		\midrule
		\multirow{3}{*}{1} & Run 1 & 0.25 & \multirow{3}{*}{0.74} \\ 
		& Run 2 & 0.41 & \\ 
		& Run 3 & 0.34 & \\ 
		\midrule
		\multirow{3}{*}{2} & Run 1 & 0.84 & \multirow{3}{*}{0.76} \\ 
		& Run 2 & 0.32 & \\ 
		& Run 3 & 0.34 & \\ 
		\midrule
		\multirow{3}{*}{3} & Run 1 & 0.78 & \multirow{3}{*}{0.88} \\ 
		& Run 2 & 0.60 & \\ 
		& Run 3 & 0.85 & \\ 
		\bottomrule
	\end{tabular*}
\end{table}
%%%%%%%%%%%%%%%%%%%%%%%%%%%%%%%%%%%%%%%%%%%%%%%%%%%%%%%%%%%%%%%%%%%%%%%%%%%%%%%%
	%\clearpage
	\section{Conclusions and future work}
	\label{conclusion}
	In this research work, we presented a novel DL-based surrogate model. 
	The developed model adopts the architecture of autoencoder-decoder in 
	conjunction with the ConvLSTM for the prediction of a full wavefield 
	containing interacting Lamb waves with delamination. 
	In the proposed model, the encoder and decoder parts are trained jointly on 
	a synthetic dataset consisting of frames which contain wave patterns of 
	delamination reflections and changes of wavefront due to delamination. 
	Then the encoder is trained separately on reference data without 
	delamination. 
	The delamination information in the form of binary images is also provided 
	along with the reference images to the encoder part for training the 
	encoder, and the final prediction of frames propagating in the plate with 
	delamination.
	In simple words, this DL-based surrogate model takes full wavefield frames 
	of propagating waves in a healthy plate as input and predicts full 
	wavefields in a plate which contains single delamination.
	
	The proposed DL model performed well as proved by the results.
	The proposed DL model is helpful for the prediction of the full wavefield 
	data from the time of excitation initiation to the desired simulation time. 
	The predicted wavefield from the proposed architecture can be used for 
	inverse problems in NDT (shown here) and SHM (possibly in future).
	The wavefield prediction by the proposed DL model is ultrafast, therefore 
	objective functions which require multiple evaluations can be applied in 
	the inverse methods.
	In contrast, such an approach is unfeasible with conventional forward 
	solvers, e.g. based on p-FEM or SEM. 
	
	It is planned to test the proposed damage identification framework on 
	experimental data in future.
	%%%%%%%%%%%%%%%%%%%%%%%%%%%%%%%%%%%%%%%%%%%%%%%%%%%%%
	
	\clearpage	
	%\appendix
	\section*{Acknowledgments}
	The research was funded by the Polish National Science Center under grant agreements no 2019/01/Y/ST8/00060.
	
	\bibliography{sn-bibliography}% common bib file
	%% if required, the content of .bbl file can be included here once bbl is 
	%%generated
	%%\input sn-article.bbl

\end{document}


