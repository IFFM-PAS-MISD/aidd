\section{Inverse method for damage identification}

A global-best PSO algorithm implemented in Python was used in the optimisation process~\cite{MirandaLesterJames}.
It takes a set of candidate solutions and tries to find the best solution using a position-velocity update method. 
It uses a star-topology where each particle is attracted to the best-performing particle.
The algorithm follows two basic steps:
\begin{itemize}
	\item the position update:
	\begin{equation}
		x_i(t+1) = x_i(t) + v_i(t+1),\label{eq:position_update}
	\end{equation}
	\item and the velocity update:
	\begin{equation}
		v_{ij}(t+1) = w\, v_{ij}(t) + c_1\, r_{1j}(t) \,[y_{ij}(t) - x_{ij}(t)] + c_2\, r_{2j}(t)\,[\hat{y}_j(t) - x_{ij}(t)],\label{eq:velocity_update}
	\end{equation}
\end{itemize}
where $r$ are random numbers, $y_{ij}$ is the particle's best-known position, $\hat{y}_j$ is the swarm's best known position, $c_1$ is the cognitive parameter, $c_2$ is the social parameter and $w$ is the inertia parameter which controls the swarm's movement.
Cognitive and social parameters control the particle's behaviour given two choices: (i) to follow its personal best or (ii) follow the swarm’s global best position.
Overall, this dictates if the swarm is explorative or exploitative in nature. 
In our tests, we used the following parameters: $c_1 = c_2 = 0.3$ and $w=0.8$.
Good convergence was achieved for these set of parameters, therefore further parameter tuning was unnecessary.

The following decision variables were used in the PSO:
\begin{itemize}
	\item delamination coordinates $(x_c, y_c)$ with bounds [0 mm, 500 m],
	\item delamination elliptic shape represented by semi-major and semi-minor axis $a, b$ with bounds [5 mm, 20 mm],
	\item delamination rotation angle $\alpha$ with bounds [$0^\circ$, $180^\circ$].
\end{itemize}

Based on decision variables, binary images of $(256\times256)$ pixels are generated (one image per particle - see Fig.~\ref{fig:complete_flowchart}).
In these images, ones (white pixels) represent delamination whereas zeros (black pixels) represent healthy area.

The most important component of the proposed inverse method is the surrogate DL model described in section~\ref{sec:proposed_approach}.
The trained DL model is used for ultrafast full wavefield prediction as illustrated in Fig.~\ref{fig:complete_flowchart}.
For a single particle and respective binary image, the DL model is evaluated 7 times for 32 consecutive frames giving as an output 224 frames. 
These predicted frames are compared to 'measured' frames by using the MSE metric which is utilised in the objective function.
However, for the sake of replicability of the results and compatibility with the available dataset~\cite{kudela_pawel_2021_5414555}, we used synthetic data instead of measured data (acquired by SLDV).

For each PSO iteration, particles are updated according to Eqs.~(\ref{eq:position_update})-(\ref{eq:velocity_update}).
The termination criterion was assumed as 100 iterations but it was observed that the objective function value converges much faster.
In the final step, the best matching wavefields indicate coordinates, semi-major, semi-minor axis and rotation angle of the elliptic-shaped delamination. 
These parameters are used for a visual representation of the best-matched delamination in the form of binary image compared against the ground truth (see also Fig.~\ref{fig:complete_flowchart}).

As an evaluation metric for assessing the accuracy of the identified 
delamination, we used the intersection over union (IoU), which measures the 
degree to which the predicted delamination overlaps the true delamination. 
It is defined as:
\begin{equation}
	IoU=\frac{Intersection}{Union}=\frac{\hat{Y} \cap Y}{\hat{Y} \cup Y}
	\label{eq:iou}
\end{equation}
where \(\hat{Y}\) is the predicted output, and \(Y\) is the ground truth (true delamination) in the form of binary images.
