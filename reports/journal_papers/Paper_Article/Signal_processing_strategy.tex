%%%%%%%%%%%%%%%%%%%%%%%%%%%%%%%%%%%%%%%%%%%%%%%%%%%%
\subsection{Signal processing strategy}
%%%%%%%%%%%%%%%%%%%%%%%%%%%%%%%%%%%%%%%%%%%%%%%%%%%%
Two signal processing strategies are applied: newly proposed image processing based on the FCN and conventional one which is the adaptive wavenumber filtering.
The former one is represented by the left branch on the scheme shown in~Fig.~\ref{fig:sig_proc_strategy} whereas the latter one is represented by the right branch.
The starting point for both approaches is the dataset consisted of frames of propagating guided waves.
However, the FCN takes as an input the RMS which combines all frames into one image. 
It essentially represents the spatial energy distribution of propagating waves.
In the adaptive wavenumber filtering method, all frames are utilised (right branch of Fig.~\ref{fig:sig_proc_strategy}).
The RMS is applied later on to output final damage map in the form of an image.

It should be noted that we tested two types of activation functions as a last layer of the FCN: softmax and sigmoid.
The former one gives the binary output which segments pixels in two categories: damaged and undamaged.
On the other hand, the threshold needs to be applied when the sigmoid activation function is used.
Also, damage maps resulted from the adaptive wavenumber filtering must be thresholded so that the area and shape of delamination can be estimated.

Finally, intersection over union (IoU) is applied as a measure of delamination size, shape and location so that all presented approaches can be compared quantitatively. 
	\begin{figure}
		\centering
		\includegraphics[scale=0.8]{FCN_adaptive_filtering_diagram_MSSP.png}
		\caption{Diagram of signal processing strategy by the proposed Fully Convolutional Network (left branch) in comparison to signal processing utilising adaptive wavenumber filtering method (right branch). }
		\label{fig:sig_proc_strategy}
	\end{figure}
The particular blocks presented in the Fig.~\ref{fig:sig_proc_strategy} will be explained in more details in the next sections.