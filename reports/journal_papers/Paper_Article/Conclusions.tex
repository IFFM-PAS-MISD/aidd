%%%%%%%%%%%%%%%%%%%%%%%%%%%%%%%%%%%%%%%%%%%%%%%%%%
\section{Conclusions}
%%%%%%%%%%%%%%%%%%%%%%%%%%%%%%%%%%%%%%%%%%%%%%%%%%
In this paper, we addressed delamination detection in composite materials using a deep learning technique. 
For this purpose, we have trained an FCN-DenseNet for semantic segmentation on a numerically generated data to simulate a full wavefield elastic wave propagation.
To see the feasibility of such a study, we have compared the deep learning model with adaptive wavenumber filtering technique.
The results were promising, and the deep learning model surpasses the conventional technique in detecting the delaminations of different shapes, sizes and angles. 
Further, the model can be improved by training it on new experimental data which mean new patterns are learned, hence it will enhance its ability to differentiate among different complex patterns.
Currently, we are in progress of implementing several deep learning architectures in order to perform a comparative study of different deep learning models regarding delamination detection in composite materials.
Current work focuses on delamination detection, however, our work can be extended for other types of damage in composite materials.