%%%%%%%%%%%%%%%%%%%%%%%%%%%%%%%%%%%%%%%%%%%%%%%%%%%%%
\section{Introduction}
%%%%%%%%%%%%%%%%%%%%%%%%%%%%%%%%%%%%%%%%%%%%%%%%%%
Composite materials are prone to various kinds of defects such as cracks, fibre breakage, debonding, and delamination~\cite{ip2004delamination, smith2009composite}. 
Among these defects, delamination is one of the most hazardous, which essentially leads to very catastrophic failures if not detected at early stages~\cite{valdes1999delamination}. 
Therefore, it is essential to effectively identify the delamination in composite structures for safe and reliable implementation in various real-world applications.  
Accordingly, different types of Structural Health Monitoring (SHM) techniques have been developed for delamination detection in composite structures. 
Recently, guided Lamb waves based SHM gained high popularity for damage detection in composite structures due to their higher sensitivity to small defects, propagation with low attenuation, and potential to monitor large areas with low-voltage and only a small number of sparsely distributed transducers~\cite{alleyne1992interaction, giurgiutiu2003lamb, Ihn2008, mitra2016guided}. 
However, utilising a smaller number of transducers are not suitable for acquiring high-quality resolution damage maps. 
On the other hand, the employment of a very dense array of transducers is not feasible in most situations. 
For alleviating such problem Scanning Laser Doppler Vibrometery (SLDV) is employed. 
SLDV is capable to measure guided Lamb waves in a very dense grid of points over the surface of a large specimen. 
This collection of signals is known as full wavefield~\cite{Radzienski2019a}. 
Damage detection techniques employing full wavefield signals are capable of effectively estimating the size and location of damage~\cite{Girolamo2018a, kudela2018impact}. 
From the last few years, full wavefield signals are continually being assessed for the detection and localisation of defects in composite structures~\cite{Radzienski2019a, kudela2018impact, sohn2011delamination, sohn2011automated, rogge2013characterization}.
Currently, guided waves based damage detection techniques are employing various physics and classical machine learning-based methods. 
These structural damage detection approaches are composed of two processes: feature extraction and feature classification. 
The feature extraction process usually needs a great deal of human labour and computational effort that prevents these techniques of being applicable in real-time SHM utilisation. 
Further, such systems also need a notable amount of expertise from the practitioner, which is very difficult to be always available.
Moreover, in many situations, the extracted handcrafted features by these techniques may fail to precisely characterise the acquired signal that leads to poor classification performance~\cite{Zhao2019b, Yuan2020}. 
Additionally, these systems are also not suitable for modelling large-scale data.

The reader is advised to refer to our previous work in~\cite{Ijjeh2021}. Since, for our knowledge, it was the first work of using full wavefield images in delamination detection in composite materials using deep learning techniques. 
In this work, we have implemented four different deep learning-based semantic segmentation models for delamination detection in composite materials.
The models were validated on numerical and experimental data to show their ability to generalise.
Further, the models were compared based on Intersection over Union (IoU) and the total number of parameters.

The paper is organised as follows, in the second section~\ref{related_works} the related works are presented.
In the third section~\ref{methodology} the dataset and the semantic segmentation models used for delamination detection are illustrated. 
Next, a detailed comparison of these models was made in section~\ref{section:results_and_discussions}.
Finally, the conclusion and future work is presented in section~\ref{conclusion}.