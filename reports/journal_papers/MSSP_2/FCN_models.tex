\subsubsection{U-Net based model}
%%%%%%%%%%%%%%%%%%%%%%%%%%%%%%%%%%%%%%%%%%%%%%%%%%%%
U-Net is a well-known architecture based on encoder-decoder scheme was introduced by Ronneberger et al.~\cite{Ronneberger2015} for biomedical image segmentation. 
The principle function of the encoder is to capture the context of an input image, while the decoder is responsible for enabling a precise localisation. 
%U-net is composed of three parts: The downsampling section, The bottleneck, and the upsampling section. 
%%%%%%%%%%%%%%%%%%%%%%%%%%%%%%%%%%%%%%%%%%%%%%%%%%%%%%%%%%%%%%%%%%%%%%%%%%%%%%%%%%%%%%%%
The encoder part holds several downsampling blocks. 
Each block takes an input applies two convolutional layers followed by a (\(2\times2\)) max pooling with a (\(2\times2\)) strides that picks the maximum value in a local pool filter in one feature map (or \(n\)-feature maps), resulting in a reduction in the dimension of feature maps~\cite{Lecun2015}, consequently, reducing computation complexity.
Each convolutional layer performs (\(3\times3\)) convolution operations, followed by Relu activation function.
Furthermore, to enhance the model training performance we applied batch normalization (BN)~\cite{Ioffe2015} after each convolutional layer.
Moreover, the number of convolutional filters is doubled after each downsampling block therefore the model can learn complex patterns effectively. 
%%%%%%%%%%%%%%%%%%%%%%%%%%%%%%%%%%%%%%%%%%%%%%%%%%%%%%%%%%%%%%%%%%%%%%%%%%%%%%%%%%%%%%%%
The bottleneck layer meddles between the encoder and the decoder as a joining point is the deepest layer in the model.
The bottleneck contains two convolutional layers, with 256 filters which helps the model to learn and recognize the complex patterns.
%It composed of two (\(3\times3\)) convolution layers followed by (\(2\times2\)) up convolution layer (Transpose convolution).
%%%%%%%%%%%%%%%%%%%%%%%%%%%%%%%%%%%%%%%%%%%%%%%%%%%%%%%%%%%%%%%%%%%%%%%%%%%%%%%%%%%%%%%%
The decoder consists of several upsampling blocks. 
Each upsampling block passes the input into two convolution layers as in the downsampling block followed by a transmission up layer consists of a transposed convolutional layer (upsampling). 
The purpose of upsampling is to retrieve the dimensions and increase the resolution.
Transposed convolutional layer differs from the regular upsampling function, by introducing learnable parameters regarding the transposed convolution filters that enhance the learning process of the model. 
Moreover, after each upsampling operation, the number of feature maps used by convolutional layer reduced by half to keep the model symmetrical. 
Further, skip connections were added by appending feature maps of the downsampling block with the corresponding upsampling block to retrieve lost spatial information during the downsampling due to the reduction of the input resolution.
Therefore, the model ensures that the feature maps which were learned during the downsampling will be utilized in the reconstruction. 
%%%%%%%%%%%%%%%%%%%%%%%%%%%%%%%%%%%%%%%%%%%%%%%%%%%%%%%%%%%%%%%%%%%%%%%%%%%%%%%%%%%%%%%%
\begin{figure} [h!]
	\begin{center}
		\includegraphics[scale= 0.8]{Unet_model.png}
	\end{center}
	\caption{U-Net architecture.} 
	\label{fig:Unet}
\end{figure}
Fig.~\ref{fig:Unet}  illustrates the network architecture. 
The encoder is on the left side in which the downsampling is performed and the decoder is on the right side where the upsampling is performed in addition to the bottleneck where is join both sides.
%%%%%%%%%%%%%%%%%%%%%%%%%%%%%%%%%%%%%%%%%%%%%%%%%%%%%
\subsubsection{VGG16 encoder-decoder}
%%%%%%%%%%%%%%%%%%%%%%%%%%%%%%%%%%%%%%%%%%%%%%%%%%%%%
In this model, we address the use of VGG16 architecture  ~\cite{simonyan2014very} as a backbone encoder to the U-Net architecture.
VGG16 is composed of 13 convolutional layers, pooling layers and dense layers, and it is used for classification purposes. 
We applied VGG16 encoder-decoder for pixel-wise image segmentation.
Figure~\ref{vgg16} presents the architecture of VGG16 encoder-decoder model. 
The model consists of two parts: downsampling and upsampling.
The downsampling path consists of \(5\) convolutional blocks,  with a total \(13\) convolutional layers  with \enquote{same} padding with a kernel size (\(3\times3\))and 32 filters for each layer, followed by BN and activation function Relu.
Each convolutional layer is responsible for extracting high level features from the input image such as edges.
A Maxpool operation with pool size of (\(2\times2\))  and (\(2\times2\)) strides followed by dropout is performed after each convolutional block. 
The upsampling path is introduced to recover spatial resolution, it also has \(5\) convolutional blocks with a total \(13\) convolutional layers  with same padding and kernel size (\(3\times3\))and 32 filters for each layer, followed by BN and activation function Relu.
For upsampling, bilinear interpolation with (\(2\times2\)) kernel size is applied.
Skip connections were added between downsampling blocks and the corresponding upsampling blocks in order to enhance recovering fine-grained details by enabling feature re-usability from earlier layers.
\begin{figure} [h!]
	\begin{center}
		\includegraphics[scale=1]{VGG16_encoder_decoder.png}
	\end{center}
	\caption{VGG16 encoder decoder architecture.} 
	\label{vgg16}
\end{figure}
%%%%%%%%%%%%%%%%%%%%%%%%%%%%%%%%%%%%%%%%%%%%%%%%%%%%
\subsubsection{FCN-DenseNet model}
%%%%%%%%%%%%%%%%%%%%%%%%%%%%%%%%%%%%%%%%%%%%%%%%%%%%%	
The one hundred layer tiramisu model (FCN-DenseNet) was introduced by Simon Jegou et al.~\cite{Jegou} for semantic segmentation.
FCN-DenseNet is similar to the U-Net architecture, FCN-DenseNet utilises the U-shape of the encoder-decoder scheme with skip connections between the downsampling and the upsampling paths to increase the resolution to the final feature map.
%Skip connections from the downsampling path to the corresponding upsampling path are essential for recovering spatially detailed information by reusing feature maps.

The main component in FCN-DenseNet is the dense block.
The purpose of the dense block is to concatenate layer input (feature maps) with its output (feature maps) to emphasize spatial details information.
The dense block is constructed from \(n\) varying number of layers, each layer is composed of a series of operations.
Figure~\ref{dense_block} illustrates the architecture of the dense block.
\begin{figure} [h!]
	\begin{center}
		\includegraphics[scale=1.0,angle=-90]{DenseBlock_layer.png}
	\end{center}
	\caption{Dense block architecture.} 
	\label{dense_block}
\end{figure}
%It has an input (\(x\)) (input image or output of transition layer) with \(k\) feature maps which is concatenated with the output of first layer and this process is recursively performed for all layers in the dense block ending up with output (\(y\)) with a (\(n\times k\)) feature maps. 
%%%%%%%%%%%%%%%%%%%%%%%%%%%%%%%%%%%%%%%%%%%%%%%%%%%%%%%%%%%%%%%%%%%%%%%%%%%%%%%%%%%%%%%%
Transition down layer was introduced to reduce the spatial dimensionality of the feature maps by performing a (\(1\times 1\)) convolution followed by (\(2\times2\)) Maxpool operation. 
%%%%%%%%%%%%%%%%%%%%%%%%%%%%%%%%%%%%%%%%%%%%%%%%%%%%%%%%%%%%%%%%%%%%%%%%%%%%%%%%%%%%%%%%
For the transition up layer, it was introduced in FCN-DenseNet to recover the input spatial resolution, to do that a transpose convolution operation is performed which upsamples the previous feature maps.
%%%%%%%%%%%%%%%%%%%%%%%%%%%%%%%%%%%%%%%%%%%%%%%%%%%%%%%%%%%%%%%%%%%%%%%%%%%%%%%%%%%%%%%%
Feature maps emerging from upsampling are concatenated with the ones resulting from the skip connection forming the input to a new dense block.
During the upsampling, the input to the dense block is not concatenated with its output to overcome the overhead of memory shortage since the upsampling path expands the spatial resolution of the feature maps.
Table~\ref{layers} presents the architecture of a single layer, the transition down  and transition up layers in details.
Figure~\ref{fcn} illustrates the FCN-DenseNet architecture for image segmentation used for delamination detection.
%%%%%%%%%%%%%%%%%%%%%%%%%%%%%%%%%%%%%%%%%%%%%%%%%%%%%%%%%%%%%%%%%%%%
\begin{table}[h!]
	\renewcommand{\arraystretch}{1.3}
	\centering
	\scriptsize
	\begin{tabular}{|c|l|c|l|c|}
		\cline{1-1} \cline{3-3} \cline{5-5}
		\textbf{Layer} &  & \textbf{Transition Down} &  & \textbf{Transition Up} \\ \cline{1-1} \cline{3-3} \cline{5-5} 
		Batch Normalization &  & Batch Normalization &  & \multirow{5}{*}{\begin{tabular}[c]{@{}c@{}}(\(3\times3\)) Transposed Convolution, \\ strides = (\(2\times2\))\end{tabular}} \\ \cline{1-1} \cline{3-3}
		Relu &  & Relu &  &  \\ \cline{1-1} \cline{3-3}
		(\(3\times3\)) Convolution &  & (\(1\times1\)) Convolution &  &  \\ \cline{1-1} \cline{3-3}
		\multirow{2}{*}{Dropout \(p = 0.2\)} &  & Dropout \(p = 0.2\) &  &  \\ \cline{3-3}
		&  & (\(2\times2\)) Maxpooling &  &  \\ \cline{1-1} \cline{3-3} \cline{5-5} 
	\end{tabular}
	\caption{Layer, Transition Down and Transition Up layers.} 
	\label{layers}
\end{table}\\

\begin{figure} [h!]
	\begin{center}
		\includegraphics[scale= 1.0]{FCN_dense_net.png}
	\end{center}
	\caption{FCN-DenseNet architecture.} 
	\label{fcn}
\end{figure}
%Our constructed model is composed of \(3\) dense blocks in the downsampling path, one dense block in bottleneck and 3 dense blocks for the upsampling path. 
%Each dense block in the downsampling and upsampling paths consists of \(2\) layers, the bottleneck dense block consists of \(4\) layers.
%The model input is the RMS image with size of (\(512\times 512\)).
%At the beginning, we perform a  convolution operation and concatenate the original input with the output, then the concatenated output is fed into the first dense block that consists of (\(2\)) layers.
%
%Each layer is composed of Ronneberger2015normalization (BN) followed by Relu, then (\(3\times3\)) convolution with same padding is applied followed by a dropout with probability \(p = 0.2\).
%Then, the output of the first dense layer is concatenated with its input and is fed into a transition down layer. 
%
%The transition down layer is composed of BN followed by Relu, then (\(1\times1\)) convolution followed by a dropout with probability \(p = 0.2\) and finally (\(2\times2\)) Maxpool with strides of (\(2\times2\)).
%
%This process is repeated until the bottleneck dense block.
%The bottleneck dense block consists of 4 layers.
%The output of the bottleneck is directed into the upsampling path starting with transition up layer.
%Accordingly, the output of the transition up layer is concatenated with the corresponding dense block output in the downsampling path.
%The final layer in the network is  (\(1\times1\)) convolution followed by either a sigmoid function or a softmax function to calculate the probability of damage for each pixel.
%Hence, we have two versions of the FCN-DenseNet model with different output layer function.


