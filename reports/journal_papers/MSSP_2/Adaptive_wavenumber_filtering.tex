
%%%%%%%%%%%%%%%%%%%%%%%%%%%%%%%%%%%%%%%%%%%%%%%%%%%%	
\subsection{Adaptive wavenumber filtering}
%%%%%%%%%%%%%%%%%%%%%%%%%%%%%%%%%%%%%%%%%%%%%%%%%%%%
Adaptive wavenumber filtering is a well-established method for processing of images of propagating elastic waves.
The method has proved to be useful for crack size estimation~\cite{Kudela2015}, impact induced damage assessment~\cite{Kudela2018},  delamination and disbonding detection and localisation~\cite{Radzienski2019}.  
It is used here as a reference point for comparison purposes against proposed strategies based on FCN.

The method involves steps such as 2D Fourier Transform, wavenumber filtering, inverse Fourier Transform and RMS.
It can be used as an automated tool for producing damage maps which are easy in interpretation.
However, it still requires setting a threshold for the filter mask and quantisation threshold useful for damage size estimation.
These thresholds can be estimated empirically and even certain rigid range of thresholds will lead to satisfactory results.
Nevertheless, it is not a fully automatic process.
FCN based approach seems to have an advantage in this regard.