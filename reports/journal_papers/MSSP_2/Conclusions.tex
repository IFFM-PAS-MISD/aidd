%%%%%%%%%%%%%%%%%%%%%%%%%%%%%%%%%%%%%%%%%%%%%%%%%%
\section{Conclusions}
\label{conclusion}
%%%%%%%%%%%%%%%%%%%%%%%%%%%%%%%%%%%%%%%%%%%%%%%%%%
In this paper, we extend our previous work on delamination detection in composite materials using deep learning techniques. 
For this purpose, we have trained four deep learning models: UNet, VGG16 encoder-decoder, PSPNet, and  FCN-DenseNet for semantic segmentation.
The models were trained on a numerically generated dataset simulating a  full wavefield of propagating guided waves.
Deep learning models show promising results in identifying various types of delaminations regarding their shapes, sizes and angles. 
The models show good generalisation behaviour on predicting the delamination in the unseen numerically generated data.
Moreover, the models are capable of detecting the delamination in the experimental data.
The performance can be further improved if the models are trained on experimental data, which allow them to learn new complex patterns.
In this work, we are focused on delamination identification in composite materials, however, we aim to extend our work on different types of defects in composite materials.
