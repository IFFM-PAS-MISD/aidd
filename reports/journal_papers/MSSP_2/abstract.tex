In this paper, several deep learning techniques for image segmentation are employed for delamination detection and localisation in composite materials.
All models were trained and validated on our previously generated dataset that resembles full wavefield measurements acquired by scanning laser Doppler vibrometer.
Additionally, a thorough comparison between all presented models is provided based on several evaluation metrics.
Furthermore, the models were additionally verified on experimentally acquired data with a Teflon insert representing delamination.
Therefore, the developed models can be used for delamination size estimation.
The obtained results from the current models are promising.

%A novel full wavefield processing method by using fully convolutional neural networks is presented.
%The full wavefield of propagating Lamb waves in the fibre-reinforced composite plate was simulated by the parallel spectral element method.
%It resembles a full wavefield measurements acquired on a surface of the plate by the scanning laser Doppler vibrometer.
%The aim of the proposed technique is an identification of delamination location, size and shape.
%It is achieved by pixel-wise image segmentation by using the end-to-end approach.
%It is possible because of the large dataset of Lamb wave propagation patterns resulting from interaction with delaminations of random location, size and shape.
%It is demonstrated that the proposed method, tested on numerical data, is performing better than conventional adaptive wavenumber filtering method which was developed in previous work.
%Moreover, it enables better automation of delamination identification so that the damage map can be created without user intervention.
%The method was also tested on experimental data acquired on the surface of the specimen in which delamination was artificially created by a Teflon insert.
%The obtained results are promising.