%%%%%%%%%%%%%%%%%%%%%%%%%%%%%%%%%%%%%%%%%%%%%%%%%%%%
\subsubsection{FCN-DenseNet model}
%%%%%%%%%%%%%%%%%%%%%%%%%%%%%%%%%%%%%%%%%%%%%%%%%%%%%	
FCN-DenseNet which was introduced in~\cite{Jegou} was applied in our previous work~\cite{Ijjeh2021} for image segmentation.
The results were promising, since it outperformed the conventional damage detection technique i.e (adaptive wavenumber filtering method). 
FCN-DenseNet has a U-shape of the encoder-decoder scheme with skip connections between the downsampling and the upsampling paths to increase the resolution to the final feature map.
The main component in FCN-DenseNet is the dense block.
The dense block is constructed from \(n\) varying number of layers, each layer consists of a series of operations as shown in Table~\ref{layers}.
The purpose of the dense block is to concatenate the input (\(x\)) (feature maps) of a layer  with its output (feature maps) to emphasize spatial details information.
In this work, we have updated the FCN-DenseNet model by increasing the number of dense blocks and the learnable parameters (filters).
The architecture of the dense block is presented in Fig.~\ref{dense_block}. 
\begin{figure} [h!]
	\begin{center}
		\includegraphics[scale=1.0,angle=-90]{DenseBlock_layer.png}
	\end{center}
	\caption{Dense block architecture.} 
	\label{dense_block}
\end{figure}
%%%%%%%%%%%%%%%%%%%%%%%%%%%%%%%%%%%%%%%%%%%%%%%%%%%%%%%%%%%%%%%%%%%%
To reduce the spatial dimensionality of the produced feature maps, a transition down layer was added to perform a (\(1\times 1\)) convolution followed by (\(2\times2\)) Maxpooling operation. 
%%%%%%%%%%%%%%%%%%%%%%%%%%%%%%%%%%%%%%%%%%%%%%%%%%%%%%%%%%%%%%%%%%%%
Consequently, to recover the spatial resolution, a transition-up layer was added. 
It applies a transpose convolution operation to upsample feature maps from the previous layer.
%%%%%%%%%%%%%%%%%%%%%%%%%%%%%%%%%%%%%%%%%%%%%%%%%%%%%%%%%%%%%%%%%%%%
Feature maps emerging from upsampling are concatenated with the ones resulting from the skip connection forming the input to a new dense block.
During the upsampling, the input to the dense block is not concatenated with its output to overcome the overhead of memory shortage since the upsampling path expands the spatial resolution of the feature maps. 
%(hint:- We can refer to previous paper and say that the same architecture was applied here. Than we can skip Fig.~\ref{fcn}).
%\begin{figure} [h!]
%	\begin{center}
%		\includegraphics[scale=1.0]{FCN_dense_net.png}
%	\end{center}
%	\caption{FCN-DenseNet architecture.} 
%	\label{fcn}
%\end{figure}
Table~\ref{layers} presents the architecture of a single layer, the transition down  and transition up layers in details.
%Figure~\ref{fcn} illustrates the FCN-DenseNet architecture for image segmentation used for delamination detection.
%%%%%%%%%%%%%%%%%%%%%%%%%%%%%%%%%%%%%%%%%%%%%%%%%%%%%%%%%%%%%%%%%%%%
\begin{table}[h!]
	\renewcommand{\arraystretch}{1.3}
	\centering
	\scriptsize
	\resizebox{\textwidth}{!}
	{
	\begin{tabular}{ccccc}
		\hline
		Layer &  &  Transition Down &  &  Transition Up \\ 
		\hline
		Batch Normalization &  & Batch Normalization &  &  \(3\times 3\) Transposed Convolution  \\ 
		Relu &  & Relu &  & strides = (\(2\times2\))  \\ 
		(\(3\times3\)) Convolution &  & (\(1\times1\)) Convolution &  &  \\ 
%		&  &   \\ 
		Dropout \(p=0.2\) &  &Dropout \(p=0.2\)  &  &  \\ 
		 &  & (\(2\times2\)) Maxpooling &  &  \\ 
	    \hline
	\end{tabular}
	}
	\caption{Layer, Transition Down and Transition Up layers.} 
	\label{layers}
	
\end{table}\\
