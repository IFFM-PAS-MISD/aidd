%%%%%%%%%%%%%%%%%%%%%%%%%%%%%%%%%%%%%%%%%%%%%%%%%%%%%
\subsection{Fully Convolutional Network approach}
%%%%%%%%%%%%%%%%%%%%%%%%%%%%%%%%%%%%%%%%%%%%%%%%%%%%%
In this section, a deep learning approach for delamination detection in composite materials is presented. 
%%%%%%%%%%%%%%%%%%%%%%%%%%%%%%%%%%%%%%%%%%%%%%%%%%%%%
\subsubsection{Data preprocessing}
%%%%%%%%%%%%%%%%%%%%%%%%%%%%%%%%%%%%%%%%%%%%%%%%%%%%%
It should be noted that the output from the wave propagation model is in the form of a 3D matrix which contains amplitudes of propagating waves at location \((x, y)\) and time \(t\). We can look at it as a set of frames of propagating waves at discrete time moments \(t_k\).

The data preprocessing as it is indicated in Fig.~\ref{fig:sig_proc_strategy} include a step of computation of root mean square value:
\begin{equation}
	\hat{s}(x,y) = \sqrt{\frac{1}{N}\sum_{k=1}^{N} s(x,y,t_k)^2}
	\label{eq:rms}
\end{equation}
where the number of sampling points \(N\) was 512.
In this way, the dataset was collapsed to 475 2D matrices in which amplitudes are stored as double-precision values.
The next step was the conversion of these matrices to greyscale images (colour image quantisation).
Colour scale values of obtained images vary between (\(0 - 255\)) hence normalization
to a range of (\(0-1\)) was applied to enhance the optimizer function during the learning process. 
	
Furthermore, data augmentation was achieved by flipping images horizontally, vertically and diagonally. 
It increased the dataset size four times -- \(1900\) images were produced.
Such data augmentation can enhance the learning process by enabling the model to learn and recognise new complex patterns.
	
The data set was split into two portions:  \(80\%\) for the training set and \(20\%\) for the testing set.
Additionally, the validation set was created as a \(20\%\) of the training set.
%%%%%%%%%%%%%%%%%%%%%%%%%%%%%%%%%%%%%%%%%%%%%%%%%%%%%
