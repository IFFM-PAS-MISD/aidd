\documentclass[11pt,a2paper]{report}
\usepackage[dvipsnames]{xcolor}
%\usepackage{dirtytalk}
\usepackage{graphicx}
\usepackage{multirow}
\usepackage{amsmath,amssymb,bm}
%\usepackage[dvips,colorlinks=true,citecolor=green]{hyperref}
\usepackage[colorlinks=true,citecolor=green]{hyperref}
%% my added packages
\usepackage{float}
\usepackage{csquotes}
\usepackage{verbatim}
\usepackage{caption}
\usepackage{subcaption}
\usepackage{booktabs} % for nice tables
\usepackage{csvsimple} % for csv read
\usepackage{graphicx}
%\usepackage[outdir=//odroid-sensors/sensors/aidd/reports/journal_papers/MSSP_Paper/Figures/]{epstopdf}
%\usepackage{breqn}
\usepackage{multirow}

\begin{document}
	
	\noindent We appreciate the time and effort that the reviewers have dedicated to provide valuable feedback on our manuscript. 
	We would like to thank the reviewers for constructive comments which helped us to improve the manuscript. 
	We have incorporated changes to reflect the suggestions provided by the reviewers. 
	We have highlighted the changes in a separate differential PDF document. 
	The additional text is in the blue print. 
	The removed text is in red. \\ \\
	Here is a point-by-point response to the reviewers’ comments and concerns.
	\\ \\
	
	\textbf{Reviewer: 1}: \\
	This study is well reported and it is dedicated to an important topic for the implementation of guided wavefield inspections: how to speed up the wavefield acquisition process.
	The applied method is clearly detailed and the validation is solid.
	I have just some minor comments: \\ \\
	\textcolor{Cyan}{
		\textbf{Response:}
	We would like to thank the reviewer for his positive feedback.
    }
	\begin{enumerate}
		\item  It should be better clarified how the Nyquist wavelength defined in the Introduction is related to the Nyquist sampling rate used in the compression rate definition (Eq. (1)), and, in particular, how the authors from the 512x512 pixels frames have generated the 73x73 pixels frames mentioned in Equation 1.
		\\ \\
		\textcolor{Cyan}
		{
			\textbf{Response:}
			Thank you for pointing this out.
			We have updated the manuscript as follows: \\
			\emph{
				The maximum permissible distance between grid points according to Nyquist theorem is calculated as in Eqn.~(\ref{eq:dx}):
				\begin{equation}
					d_{max}= \frac{1}{2*k_{max}} = \frac{1}{2*51.28\ [\textup{m}]} = \frac{\lambda}{2} = \frac{19.5}{2}\ \textup{[mm]}.
					\label{eq:dx}	
				\end{equation} 
				where $k_{max}$ is the maximum wavenumber, and $\lambda$ is the shortest wavelength. \\				
				On the other hand, the longest distance between grid points on uniform square grid in 2D space is along the diagonal as shown in Fig.~\ref{fig:Nyquist}.
				Therefore, the number of Nyquist sampling points along edges of the plate is defined as:
				\begin{align}
					\begin{split}
						N_x= \frac{L}{d_{max}/\sqrt{2}}, \\
						N_y=  \frac{W}{d_{max}/\sqrt{2}},
					\end{split}
					\label{eq:Nyq}
				\end{align}
				\\
				where $L$ is the plate length, and $W$ is the plate width.
				\\
				In our particular case, $L=W=500$~[mm], and number of Nyquist points $N_x= N_y= N_{Nyq} =73$.
				\begin{figure} [!h]
					\centering
					\includegraphics[scale=1]{Nyquist_wavelength.png}
					\caption{\textcolor{Cyan}{Longest distance between grid points.}}
					\label{fig:Nyquist}
				\end{figure} 
			}
		}
		\item A second comment concerning the Nyquist sampling rate is related to the discussion of the experimental results: the authors (p. 16, l. 20) wrote that "The conventional CS method is not efficient below the Nyquist sampling rate". 
		This comment requires some additional elaboration because it seems to imply that there is no utility in using the CS at all. 		
		\\	\\
		\textcolor{Cyan}
		{
			\textbf{Response:}
			Thank you for your constructive comment. \\
			Actually, in our case, we found that when we applied the conventional CS technique to data below the Shannon–Nyquist rate, it showed poor results compared to the deep learning method, and this is our statement. 
		} 
		\item Please clarify the number of pixels in the reference image of Figure 12(a). 
		\\ \\
		\textcolor{Cyan}
		{
			\textbf{Response:}
			Thank you for pointing this out. \\
			The reference frame has a size of \((512\times512)\) pixels, and it was updated in the manuscript.
		}	
		\item The proposed method seems to be very effective in recovering the wavefield on the delamination region but the Pearson CC has negative values when it is computed on the whole plate. 
		How do the authors explain these discrepancies?
		\\ \\
		\textcolor{Cyan}
		{
			\textbf{Response:}
			Thank you for your constructive comment.\\ 	
		}
		\item In figure 9, it can be observed a decreasing trend in the computed figure of merit (PSNR, Pearson CC). 
		Is it because the wavefield intensity tends to decrease due to the geometrical spreading?
		\\ \\
		\textcolor{Cyan}
		{
			\textbf{Response:}
			Thank you for pointing this out. \\
			We can say that as the propagated waves reflect from the boundaries of the specimen, complex patterns start to appear; therefore, it becomes hard to recover the registered data into the high-resolution frame.
			Furthermore, the developed deep learning models were trained to recover the high-resolution frames only on a certain number of frames (128 frames) starting from the time/frame of the initial interaction with delamination.
			However, the deep learning models can generalise in such a way as to recover those frames that we did not use during the training process.
			Consequently, this behaviour in the evaluation metrics is expected.
		}
	\end{enumerate}
	
	\noindent\textbf{Reviewer: 2} \\ \\
	This paper proposes a deep learning-based super resolution method for full wavefield reconstruction of Lamb waves from spatially sparse SLDV measurements of resolution below the Nyquist wavelength. 
	The proposed ideas are valuable for promoting the real-time application of SLDV. 
	But more efforts are needed to improve the quality of this paper. \\ \\
	\textcolor{Cyan}
	{
		\textbf{Response:}
		We would like to thank the reviewer for his efforts in reviewing our manuscript.
	} 
	\\ \\
	The detailed comments are given as follows:
	\begin{enumerate}
		\item The focus of this paper is on deep learning super-resolution based full wavefield reconstruction of guided waves, while compressive sensing should only be a method used for comparison, but the paper seems to spend an inordinate amount of space on its principles, current state of application, etc. 
		The paper should focus more on the newly proposed deep learning based approach, especially in the introduction section.
		\\ \\ 
		\textcolor{Cyan}
		{
			\textbf{Response:}\\
			Thank you for your constructive comment. \\
			In this work, we proposed a deep learning approach for full wavefield frame reconstruction.
			Accordingly, our aim in this work is to establish a framework for investigating the feasibility of utilising deep learning approach to recover the low-resolution acquired data by SLDV.			
			Therefore, we focused entirely on comparing the conventional compressive sensing theory with the deep learning approach.
			Our next planned work is to make a comparative study among various deep learning techniques.
			Hence, it will be entirely based on presenting various deep learning techniques for full wavefield reconstruction.
		}
		\item  In section 2 please add a flowchart to summarize the proposed approach for a more visual understanding.
		\\ \\ 
		\textcolor{Cyan}
		{
			\textbf{Response:}
			Thank you for your constructive comment. \\
			A flow chart illustrating the complete procedure of the proposed approach was added.
		}
		\item The two deep learning models in this paper achieve the same function and show similar reconstruction results, so what is the significance of applying two neural network architectures instead of one in this paper? A comparison of the two models is not seen in the conclusions. 
		This is not clearly explained in the paper. 
		\\ \\ 
		\textcolor{Cyan}
		{
			\textbf{Response:}
		}
		\item In section 3.2, the experimental setup is not clearly described. 
		The authors should add photos of the experimental setup and explain how the scanning points of the SLDV are distributed, etc.
		\\ \\ 
		\textcolor{Cyan}
		{
			\textbf{Response:}
		}
		\item  Observing Figure 12, the deep learning-based reconstruction result seems to be significantly better than the one based on compressive sensing when Nf=1024. 
		But why do their PSNR and Pearson correlation coefficients show almost opposite results in Table 2?
		\\ \\ 
		\textcolor{Cyan}
		{
			\textbf{Response:}
		}
		\item  Can the authors explain why the deep learning-based super resolution approach has similar reconstruction results with the compressive sensing-based approach for the whole plate, but outperforms the latter for the region of delamination?
		\\ \\ 
		\textcolor{Cyan}
		{
			\textbf{Response:}
			Thank you for pointing this out. \\
			
		}
		\item From Table 2, it appears that deep learning-based super resolution approach does not perfectly outperform the compressive sensing-based approach. 
		So the statements in the conclusions and introduction seem inappropriate.
		\\ \\ 
		\textcolor{Cyan}
		{
			\textbf{Response:}
			Thank you for your constructive comment. \\
		}
		\item Is it representative enough that most of the discussion in section 3.2 is devoted to frame 110? If not, please add the analysis of the other 3 to 5 frames.
		\\ \\ 
		\textcolor{Cyan}
		{
			\textbf{Response:}
			Thank you for your constructive comment. \\
			We have mentioned that the developed deep learning models were trained on a certain number of frames (128 frames), which starts from the initial interaction of the guided waves with the damage.
			Accordingly, in our case, we focused on frame $N_f =110$ since it represents the initial interaction of the guided waves with the delamination.
			Furthermore, in Fig. 16, we presented a comparison of reconstruction accuracy depending on the frame number $N_f$.
		}
		\item On page 16, line 38, should Figure 12b be changed to Figure 12a?
		\\ \\ 
		\textcolor{Cyan}
		{
			\textbf{Response:}
			Thank you for your constructive comment. 
			\\
			We have updated it accordingly.
		}
	\end{enumerate}
\end{document}