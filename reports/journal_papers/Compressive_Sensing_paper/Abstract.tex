%%%%%%%%%%%%%%%%%%%%%%%%%%%%%%%%%%%%%%%%%%%%%%%%%%%%%%%%%%%%%%%%%%%%%%%%%%%%%%%%
Guided Lamb waves-based systems mostly employ an array of transducers for 
point-wise measurements.
In such systems, scanning laser Doppler vibrometer (SLDV) along-with pulse 
laser or piezoelectric transducers are used for the excitation and measurement 
of guided waves. 
This process of acquiring a full wavefield of guided Lamb waves is very costly, 
troublesome, lengthy, and time-consuming.
One possible solution to tackle this problem is to acquire the Lamb waves in a 
low-resolution form and then apply a compressive sensing (CS) or deep 
learning-based super-resolution approach to that low-resolution form of full 
wavefield data.
In this research work, we applied two deep learning-based super-resolution 
approaches on a large low-resolution dataset of simulation of full wave fields 
of Lamb waves.
The developed deep learning approaches are elaborated and the results obtained 
from both models are compared with the conventional CS approach. 
Furthermore, the validation of the proposed deep learning models in a 
real-world scenario is performed by an experiment on a carbon fibre reinforced 
polymer plate with embedded Teflon inserts simulating delaminations. 
It is concluded that the performance of both models is good as compared to the 
conventional CS approach. 

