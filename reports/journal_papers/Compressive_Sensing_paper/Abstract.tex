%%%%%%%%%%%%%%%%%%%%%%%%%%%%%%%%%%%%%%%%%%%%%%%%%%%%%%%%%%%%%%%%%%%%%%%%%%%%%%%%
Scanning laser Doppler vibrometer is a popular tool for the acquisition of the full wavefield of propagating guided waves, in particular Lamb waves. Signal processing of such a wavefield enables us to reveal damage in the inspected structure.
However, the process of acquiring the full wavefield of guided waves is time-consuming.
One possible solution to tackle this problem is to acquire the Lamb waves in a low-resolution form and then apply a compressive sensing (CS) or deep learning-based super-resolution approach to that low-resolution form of full wavefield data. 
In this work, we applied two deep learning-based super-resolution approaches on a large synthetic low-resolution dataset of full wavefields of propagating Lamb waves in a plate of CFRP. 
The developed deep learning approaches are elaborated, and the results obtained from both models are compared with the conventional CS approach. 
Furthermore, the validation of the proposed deep learning models in a real-world scenario is performed by an experiment on a CFRP plate with embedded Teflon inserts simulating delaminations. 
It is concluded that the performance of both models outperforms the conventional CS approach.