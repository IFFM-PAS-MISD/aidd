\section{Introduction}
%%%%%%%%%%%%%%%%%%%%%%%%%%%%%%%%%%%%%%%%%%%%%%%%%%%%%%%%%%%%%%%%%%%%%%%%%%%%%%%%
Guided waves, in particular Lamb waves, are often utilised for structural health monitoring (SHM) as well as non-destructive testing (NDT).
For structural health monitoring usually array of transducers is used for point-wise measurements.
In active guided wave based SHM these are usually piezoelectric transducers which can work as actuators and sensors.
It should be noted that round-robin actuator-sensor measurements can be conducted very fast, therefore nearly online monitoring of a structure is possible.

Recently, a lot of research on application of scanning laser Doppler vibrometer (SLDV) for NDT is reported \textcolor{RubineRed}{[citation needed!]}.
In this method either piezoelectric transducer or pulse laser are used for guided wave excitation while the measurements by SLDV are taken at one point on the surface of inspected structure.
The process is repeated for other points automatically in scanning fashion until full wavefield of Lamb waves is acquired.

Full wavefield measurements are taken on very dense grid of points in opposite to sparsely measured signals by sensors.
Hence, deliver much more useful data from which information about damage can be extracted in comparison to signals measured by an array of transducers.
On the other hand, SLDV measurements take much more time than measurements conducted by an array of transducers.
It makes SLDV approach unsuitable for SHM in which continuous monitoring is required.
But it is very capable for offline NDT application.

One can imagine that in future matrix of laser heads instead of single laser head used nowadays will be developed to reduce SLDV measurement time.
Alternatively, compressive sensing and/or deep learning super-resolution can be applied.
It means that SLDV measurements can be taken on low resolution grid of points and than full wavield can be reconstructed at high resolution.

{\color{RubineRed}

Compressive sensing concept/literature

compressive sensing in guided waves

compressive sensing plus super-resolution enhancement (Esfandabadi, de Marchi)

super-resolution literature, mostly images and video games, none for waves

end-to-end approach of DLSS - our original contribution
}