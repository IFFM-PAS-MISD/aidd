\section{Results and discussions}
In this section, the evaluation of the developed DLSR models based on the numerical test cases representing the LR frames are presented.
Additionally, the developed models were evaluated on an experimental test case to demonstrate their capability of super-resolution image reconstruction.

%%%%%%%%%%%%%%%%%%%%%%%%%%%%%%%%%%%%%%%%%%%%%%%%%%%%%%%%%%%%%%%%%%%%%%%%%%%%%%%%
Two metrics were utilised to evaluate the performance of the developed DLSR models:
The first one is the peak signal-to-noise ratio (PSNR), which refers to the maximum possible power of a signal and the power of the distorting noise that affects the quality of its representation.
Equation~\ref{PSNR} depicts the mathematical representation of the PSNR:
\begin{equation}
	PSNR=20log_{10}\left(\frac{R}{\sqrt{MSE}}\right)
	\label{PSNR}
\end{equation}
where \(R\) represents the maximum fluctuation value that exists in the input image, and its equal to \(255\).
\(MSE\) refers to the mean square error between the predicted output and the corresponding ground truth.

The second metric is the Pearson correlation coefficient (Pearson CC) that measures the linear relationship between two variable sets \textbf{\(X\)} (represents the ground truth values) and \textbf{\(Y\)} (represents the predicted values).
Equation~\ref{Pearson} depicts the mathematical formula to calculate Pearson CC:
\begin{equation}
	r_{xy} = \frac{\sum_{i=1}^{n}(x_i - \bar{x})(y_i-\bar{y})}{\sqrt{\sum_{i=1}^{n}(x_i - \bar{x})^2}\sqrt{\sum_{i=1}^{n}(y_i - \bar{y})^2}}
	\label{Pearson}
\end{equation}
where \(n\) is the number of sample points, \(x_i\), \(y_i\) are the individual value points representing the ground truth and predicted values, respectively, and \(\bar{x}\) is the mean value of the sample, analogously to \(\bar{y}\).

The developed DLSR models were trained using MSE (L2 norm) loss function.
Further, we implemented the proposed networks with the Keras~\cite{chollet2015keras} API running on the top of TensorFlow, and trained them using NVIDIA RTX 2080 and Tesla V100 GPU. 
The source code is publicly available online.(we can add link here to GitHub)
\clearpage
%%%%%%%%%%%%%%%%%%%%%%%%%%%%%%%%%%%%%%%%%%%%%%%%%%%%%%%%%%%%%%%%%%%%%%%%%%%%%%%%
\subsection{Numerical cases}
%%%%%%%%%%%%%%%%%%%%%%%%%%%%%%%%%%%%%%%%%%%%%%%%%%%%%%%%%%%%%%%%%%%%%%%%%%%%%%%%
The results of the reconstruction of HR frames for a numerical test case by the developed DLSR models are presented in Fig.~\ref{fig:num_results}.
It must be noted that the LR frames used to recover the HR frames were not used in the training set.

Figure~\ref{fig:num_results} presents the recovery of three HR frames (\ref{fig:HR_1}, \ref{fig:HR_2}, and \ref{fig:HR_3}) at different time steps with frame numbers \(N_f=91, 154,\ \text{and}\ 218\) respectively.
Figures~\ref{fig:num_f1_ijjeh} and~\ref{fig:num_f1_saeed} show the recovery of the HR frame presented in~\ref{fig:HR_1}, which depicts the initial interaction of propagating Lamb waves with the damage.
Figures~\ref{fig:num_f64_ijjeh} and~\ref{fig:num_f64_saeed} show the recovery of the HR frame presented in~\ref{fig:HR_2}, which depicts the propagating Lamb and their reflections from damage and boundaries of the plate.
Figures~\ref{fig:num_f128_ijjeh} and~\ref{fig:num_f128_saeed} show the recovery of complex phenomena of wave propagation in the plate presented in Fig.~\ref{fig:HR_3}.
Table~\ref{tab:num_results} presents the comparison results for the reconstructed frames with the developed DLSR models at three $N_f$.
For the tested frames, both DLSR models recovered the HR frames with high PSNR and Pearson CC values.
\begin{figure} [!ht]
	\centering
	\begin{subfigure}[b]{.32\textwidth}
		\centering
		\includegraphics[scale=0.8]{HR_image_case_475_frame_1.png}
		\caption{HR frame \\ (PSNR/ Pearson CC)}
		\label{fig:HR_1}
	\end{subfigure}
	\hfill
	\begin{subfigure}[b]{.32\textwidth}
		\centering
		\includegraphics[scale=0.8]{SR_output_475_frame_1.png}
		\caption{Model~I \\($46.38\ \text{dB}/\ 0.9991$)}
		\label{fig:num_f1_ijjeh}
	\end{subfigure}
	\hfill
	\begin{subfigure}[b]{.32\textwidth}
		\centering
		\includegraphics[scale=0.8]{Saeed_SR_output_475_frame_1.png}
		\caption{Model~II \\ ($48.43\ \text{dB}/\ 0.9995$)}
		\label{fig:num_f1_saeed}	
	\end{subfigure}
	\hfill
	\begin{subfigure}[b]{.32\textwidth}
		\centering
		\includegraphics[scale=0.8]{HR_image_case_475_frame_64.png}
		\caption{HR frame \\ (PSNR/ Pearson CC)}
		\label{fig:HR_2}
	\end{subfigure}
	\hfill
	\begin{subfigure}[b]{.32\textwidth}
		\centering
		\includegraphics[scale=0.8]{SR_output_475_frame_64.png}
		\caption{Model~I \\ ($45.23\ \text{dB}/\ 0.9978$)}
		\label{fig:num_f64_ijjeh}
	\end{subfigure}
	\hfill
	\begin{subfigure}[b]{.32\textwidth}
		\centering
		\includegraphics[scale=0.8]{Saeed_SR_output_475_frame_64.png}
		\caption{Model~II \\ $(48.34\ \text{dB}/\ 0.9990)$}
		\label{fig:num_f64_saeed}	
	\end{subfigure}
	\hfill
	\begin{subfigure}[b]{.32\textwidth}
		\centering
		\includegraphics[scale=0.8]{HR_image_case_475_frame_128.png}
		\caption{HR frame \\ (PSNR/ Pearson CC)}
		\label{fig:HR_3}
	\end{subfigure}
	\hfill
	\begin{subfigure}[b]{.32\textwidth}
		\centering
		\includegraphics[scale=0.8]{SR_output_475_frame_128.png}
		\caption{Model~I \\ ($42.66\ \text{dB}/\ 0.9949$)}
		\label{fig:num_f128_ijjeh}
	\end{subfigure}
	\hfill
	\begin{subfigure}[b]{.32\textwidth}
		\centering
		\includegraphics[scale=0.8]{Saeed_SR_output_475_frame_128.png}
		\caption{Model~II \\ $(44.51\ \text{dB}/\ 0.9966)$}
		\label{fig:num_f128_saeed}	
	\end{subfigure}
	\caption{
		Comparison of reconstructed frames with DLSR Model~I and II for frames $N_f = 91, 154\ \text{and}\ 218 $ respectively. }
	\label{fig:num_results}
\end{figure}


\begin{table}[!ht]
	\renewcommand{\arraystretch}{1.3}
	\centering \footnotesize
	\caption{Quality metrics for numerical case reconstructed with DLSR models at three time steps.}	
	\begin{tabular}{llcc} 
		\toprule		
		$N_f$ & DLSR & PSNR & PEARSON CC\\
		\midrule
		\multirow{2}{*}{91} & Model-I  & \(46.38\) dB & \(0.9991\) \\
		 				   & Model-II & \(48.43\) dB & \(0.9995\) \\
		\midrule
		\multirow{2}{*}{154} & Model-I  & \(45.23\) dB & \(0.9978\) \\
							& Model-II & \(48.34\) dB & \(0.9990\) \\
		\midrule
		\multirow{2}{*}{154} & Model-I  & \(42.66\) dB & \(0.9949\) \\
							& Model-II & \(44.51\) dB & \(0.9966\) \\
		\bottomrule
	\end{tabular}	
	\label{tab:num_results}
\end{table}

The achieved PSNR and Pearson CC metrics regarding the full wavefield of 512 frames for the previously evaluated test case are presented in Fig.\ref{fig:num_case_475_metrics}, where Figs~\ref{fig:num_model_I} and~\ref{fig:num_model_II} relate to models I and II, respectively.
For both DLSR models, the reconstructed HR frames have high PSNR and Pearson CC values.
However, as we move forward in frames, the metrics values start to decrease, which is expected as the wave patterns become more complex and sophisticated.
It is noteworthy to restate that the DLSR models were trained only on a set of $128$ frames per case representing the propagation of Lamb waves starting from the time point (frame) of interaction with damage. 
However, the developed DLSR models are capable of reconstructing the full wavefield of $512$ frames, indicating the capability to generalise.
%%%%%%%%%%%%%%%%%%%%%%%%%%%%%%%%%%%%%%%%%%%%%%%%%%%
\begin{figure} [!ht]
	\centering
	\begin{subfigure}[b]{1\textwidth}
		\centering
		\includegraphics[scale=1]{frame_metrics_DLSR_model_1_num.png}
		\caption{}
		\label{fig:num_model_I}
	\end{subfigure} \\
	\begin{subfigure}[b]{1\textwidth}
		\centering
		\includegraphics[scale=1]{frame_metrics_DLSR_model_2_num.png}
		\caption{}
		\label{fig:num_model_II}
	\end{subfigure}
	\caption{Comparison of reconstruction accuracy for a numerical test case of full wavefield.}
	\label{fig:num_case_475_metrics}
\end{figure}
\clearpage

\subsection{Experimental cases}

%%%%%%%%%%%%%%%%%%%%%%%%%%%%%%%%%%%%%%%%%%%%%%%%%%%%%%%%%%%%%%%%%%%%%%%%%%%%%%%%
\colorbox{green}{Will be written by Maciej}
%%%%%%%%%%%%%%%%%%%%%%%%%%%%%%%%%%%%%%%%%%%%%%%%%%%%%%%%%%%%%%%%%%%%%%%%%%%%%%%%

\begin{figure} [!ht]
	\centering
	\includegraphics[scale=.8]{figure8.png}
	\caption{Delamination arrangements in the specimen.}
	\label{fig:specimen}
\end{figure}

\begin{figure} [!ht]
	\centering
	\includegraphics[scale=1]{figure9.png}
	\caption{Comparison of reconstruction accuracy depending on the number of measurement points $N_p$.}
	\label{fig:points_metrics}
\end{figure}

\begin{figure} [!ht]
	\centering
	\begin{subfigure}[b]{0.32\textwidth}
		\centering
		\includegraphics[scale=0.8]{figure10a.png}
		\caption{Reference}
		\label{fig:frame110_ref}
	\end{subfigure}
	\hfill
	\begin{subfigure}[b]{0.32\textwidth}
		\centering
		\includegraphics[scale=0.8]{figure10b.png}
		\caption{CS: 1024 points}
		\label{fig:frame110_CS1024}
	\end{subfigure}
	\hfill
	\begin{subfigure}[b]{0.32\textwidth}
		\centering
		\includegraphics[scale=0.8]{figure10c.png}
		\caption{CS: 3000 points}
		\label{fig:frame110_CS3000}
	\end{subfigure}	
	\hfill
	\begin{subfigure}[b]{0.32\textwidth}
		\centering
		\includegraphics[scale=0.8]{figure10d.png}
		\caption{CS: 4000 points}
		\label{fig:frame110_CS4000}
	\end{subfigure}
	\hfill
	\begin{subfigure}[b]{0.32\textwidth}
		\centering
		\includegraphics[scale=0.8]{figure10e.png}
		\caption{DLSR: model I}
		\label{fig:frame110_Abdalraheem}
	\end{subfigure}
	\hfill
	\begin{subfigure}[b]{0.32\textwidth}
		\centering
		\includegraphics[scale=0.8]{figure10f.png}
		\caption{DLSR: model II}
		\label{fig:frame110_Saeed}
	\end{subfigure}
	
	\caption{Comparison of reference wavefield with reconstructed one by CS and DLSR for the frame $N_f = 110$. Rectangle box indicates the region of the strongest reflection from delamination.}
	\label{fig:frame110_comparison}
\end{figure} 

\begin{figure} [!ht]
	\centering
	\begin{subfigure}[b]{0.32\textwidth}
		\centering
		\includegraphics[scale=0.8]{figure11a.png}
		\caption{Reference}
		\label{fig:frame110delam_ref}
	\end{subfigure}
	\hfill
	\begin{subfigure}[b]{0.32\textwidth}
		\centering
		\includegraphics[scale=0.8]{figure11b.png}
		\caption{CS: 1024 points}
		\label{fig:frame110delam_CS1024}
	\end{subfigure}
	\hfill
	\begin{subfigure}[b]{0.32\textwidth}
		\centering
		\includegraphics[scale=0.8]{figure11c.png}
		\caption{CS: 3000 points}
		\label{fig:frame110delam_CS3000}
	\end{subfigure}	
	\hfill
	\begin{subfigure}[b]{0.32\textwidth}
		\centering
		\includegraphics[scale=0.8]{figure11d.png}
		\caption{CS: 4000 points}
		\label{fig:frame110delam_CS4000}
	\end{subfigure}
	\hfill
	\begin{subfigure}[b]{0.32\textwidth}
		\centering
		\includegraphics[scale=0.8]{figure11e.png}
		\caption{DLSR: model I}
		\label{fig:frame110delam_Abdalraheem}
	\end{subfigure}
	\hfill
	\begin{subfigure}[b]{0.32\textwidth}
		\centering
		\includegraphics[scale=0.8]{figure11f.png}
		\caption{DLSR: model II}
		\label{fig:frame110delam_Saeed}
	\end{subfigure}
	
	\caption{Comparison of reference wavefield with reconstructed one by CS and DLSR for the region of delamination reflection (close up region of frame $N_f = 110$ as indicated in Fig.~\ref{fig:frame110_comparison}.}
	\label{fig:frame110del_comparison}
\end{figure} 
\clearpage
\begin{figure} [!ht]
	\centering
 	\includegraphics[scale=1]{figure12.png}
	\caption{Comparison of reconstruction accuracy at frame number $N_f$.}
	\label{fig:frame_metrics}
\end{figure}

\begin{table}[!ht]
	\renewcommand{\arraystretch}{1.3}
	\centering \footnotesize
	\caption{Quality metrics for tested methods for various number of points $N_p$ and corresponding compression ratios CR calculated for the frame no $N_f=110$.}	
	\begin{tabular}{lrrrcrc} 
		\toprule
		& & & \multicolumn{2}{c}{plate} & \multicolumn{2}{c}{delamination} \\
		\cmidrule(lr){4-5} \cmidrule(lr){6-7}
		Method & $N_p$ & CR [\%] & PSNR & PEARSON CC& PSNR & PEARSON CC \\
		\midrule
		\csvreader
		[table head=\toprule,
		late after line=\\ 
		]{table_metrics.csv}{
		1=\one, 2=\two, 3=\three, 4=\four, 5=\five, 6=\six, 7=\seven
		}%
		{\one & \two & \three & \four & \five & \six & \seven }%	
		\bottomrule
	\end{tabular}	
	\label{tab:csv_results}
\end{table}