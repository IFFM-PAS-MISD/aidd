\section{Results and discussions}

\subsection{Numerical cases}
%%%%%%%%%%%%%%%%%%%%%%%%%%%%%%%%%%%%%%%%%%%%%%%%%%%
\begin{figure} [!h]
	\centering
	\begin{subfigure}[b]{.48\textwidth}
		\centering
		\includegraphics[scale=1]{SR_output_475_frame_1.png}
		\caption{HR frame / GT}
		\label{fig:num_f1_ijjeh}
	\end{subfigure}
	\hfill
	\begin{subfigure}[b]{.48\textwidth}
		\centering
		\includegraphics[scale=1]{Saeed_SR_output_475_frame_1.png}
		\caption{LR frame}
		\label{fig:num_f1_saeed}	
	\end{subfigure}
	\hfill
	\begin{subfigure}[b]{.48\textwidth}
		\centering
		\includegraphics[scale=1]{SR_output_475_frame_64.png}
		\caption{HR frame / GT}
		\label{fig:num_f64_ijjeh}
	\end{subfigure}
	\hfill
	\begin{subfigure}[b]{.48\textwidth}
		\centering
		\includegraphics[scale=1]{Saeed_SR_output_475_frame_64.png}
		\caption{LR frame}
		\label{fig:num_f64_saeed}	
	\end{subfigure}
	\hfill
	\begin{subfigure}[b]{.48\textwidth}
		\centering
		\includegraphics[scale=1]{SR_output_475_frame_128.png}
		\caption{HR frame / GT}
		\label{fig:num_f128_ijjeh}
	\end{subfigure}
	\hfill
	\begin{subfigure}[b]{.48\textwidth}
		\centering
		\includegraphics[scale=1]{Saeed_SR_output_475_frame_128.png}
		\caption{LR frame}
		\label{fig:num_f128_saeed}	
	\end{subfigure}
	\caption{Model~I:~(a), (c) and (e), Model~II:~(b), (d) and (f)}
	\label{fig:num_results}
\end{figure}
\clearpage
%%%%%%%%%%%%%%%%%%%%%%%%%%%%%%%%%%%%%%%%%%%%%%%%%%%


\subsection{Experimental cases}

Experiment description

\begin{figure} [h!]
	\centering
	\includegraphics[scale=.8]{figure8.png}
	\caption{Delamination arrangements in the specimen.}
	\label{fig:specime}
\end{figure}

\begin{figure} [h!]
	\centering
	\includegraphics[scale=1]{figure9.png}
	\caption{Comparison of reconstruction accuracy depending on the number of measurement points $N_p$.}
	\label{fig:points_metrics}
\end{figure}

\begin{figure} [h!]
	\centering
	\begin{subfigure}[b]{0.32\textwidth}
		\centering
		\includegraphics[scale=0.8]{figure10a.png}
		\caption{Reference}
		\label{fig:frame110_ref}
	\end{subfigure}
	\hfill
	\begin{subfigure}[b]{0.32\textwidth}
		\centering
		\includegraphics[scale=0.8]{figure10b.png}
		\caption{CS: 1024 points}
		\label{fig:frame110_CS1024}
	\end{subfigure}
	\hfill
	\begin{subfigure}[b]{0.32\textwidth}
		\centering
		\includegraphics[scale=0.8]{figure10c.png}
		\caption{CS: 3000 points}
		\label{fig:frame110_CS3000}
	\end{subfigure}	
	\hfill
	\begin{subfigure}[b]{0.32\textwidth}
		\centering
		\includegraphics[scale=0.8]{figure10d.png}
		\caption{CS: 4000 points}
		\label{fig:frame110_CS4000}
	\end{subfigure}
	\hfill
	\begin{subfigure}[b]{0.32\textwidth}
		\centering
		\includegraphics[scale=0.8]{figure10e.png}
		\caption{DLSR: model I}
		\label{fig:frame110_Abdalraheem}
	\end{subfigure}
	\hfill
	\begin{subfigure}[b]{0.32\textwidth}
		\centering
		\includegraphics[scale=0.8]{}
		\caption{DLSR: model II}
		\label{fig:frame110_Saeed}
	\end{subfigure}
	
	\caption{Comparison of reference wavefield with reconstructed one by CS and DLSR for the frame $N_f = 110$. Rectangle box indicates the region of the strongest reflection from delamination.}
	\label{fig:frame110_comparison}
\end{figure} 

\begin{figure} [h!]
	\centering
	\begin{subfigure}[b]{0.32\textwidth}
		\centering
		\includegraphics[scale=0.8]{figure11a.png}
		\caption{Reference}
		\label{fig:frame110delam_ref}
	\end{subfigure}
	\hfill
	\begin{subfigure}[b]{0.32\textwidth}
		\centering
		\includegraphics[scale=0.8]{figure11b.png}
		\caption{CS: 1024 points}
		\label{fig:frame110delam_CS1024}
	\end{subfigure}
	\hfill
	\begin{subfigure}[b]{0.32\textwidth}
		\centering
		\includegraphics[scale=0.8]{figure11c.png}
		\caption{CS: 3000 points}
		\label{fig:frame110delam_CS3000}
	\end{subfigure}	
	\hfill
	\begin{subfigure}[b]{0.32\textwidth}
		\centering
		\includegraphics[scale=0.8]{figure11d.png}
		\caption{CS: 4000 points}
		\label{fig:frame110delam_CS4000}
	\end{subfigure}
	\hfill
	\begin{subfigure}[b]{0.32\textwidth}
		\centering
		\includegraphics[scale=0.8]{figure11e.png}
		\caption{DLSR: model I}
		\label{fig:frame110delam_Abdalraheem}
	\end{subfigure}
	\hfill
	\begin{subfigure}[b]{0.32\textwidth}
		\centering
		\includegraphics[scale=0.8]{}
		\caption{DLSR: model II}
		\label{fig:frame110delam_Saeed}
	\end{subfigure}
	
	\caption{Comparison of reference wavefield with reconstructed one by CS and DLSR for the region of delamination reflection (close up region of frame $N_f = 110$ as indicated in Fig.~\ref{fig:frame110_comparison}.}
	\label{fig:frame110del_comparison}
\end{figure} 
\clearpage
\begin{figure} [h!]
	\centering
 	\includegraphics[scale=1]{figure12.png}
	\caption{Comparison of reconstruction accuracy at frame number $N_f$.}
	\label{fig:frame_metrics}
\end{figure}

\begin{table}[h!]
	\renewcommand{\arraystretch}{1.3}
	\centering \footnotesize
	\caption{Quality metrics for tested methods for various number of points $N_p$ and corresponding compression ratios CR calculated for the frame no $N_f=110$.}	
	\begin{tabular}{lrrrcrc} 
		\toprule
		& & & \multicolumn{2}{c}{plate} & \multicolumn{2}{c}{delamination} \\
		\cmidrule(lr){4-5} \cmidrule(lr){6-7}
		Method & $N_p$ & CR [\%] & PSNR & PEARSON CC& PSNR & PEARSON CC \\
		\midrule
		\csvreader
		[table head=\toprule,
		late after line=\\ 
		]{table_metrics.csv}{
		1=\one, 2=\two, 3=\three, 4=\four, 5=\five, 6=\six, 7=\seven
		}%
		{\one & \two & \three & \four & \five & \six & \seven }%	
		\bottomrule
	\end{tabular}	
	\label{tab:csv_results}
\end{table}