\section{Methodology}
\subsection{Dataset Acquisition}
In this work, we have utilised our synthetic dataset~\cite{kudela_pawel_2021_5414555} for training the deep learning models for super-resolution image reconstruction.
The dataset was generated by using in-house code of the time domain spectral element method.
The dataset resembles the particle velocity measurements at the bottom surface of a plate acquired by the SLDV in the transverse direction as a response to the piezoelectric excitation at the centre of the plate.
The plate has a size of \( (500\times500)\) mm\(^2\) and is made of eight layers of cross-ply CFRP with a total thickness of \(3.9\) mm.
The input signal was a five-cycle Hann window modulated sinusoidal tone burst. The carrier frequency was \(50\) kHz.

The dataset contains \(475\) cases of full wavefield representing the propagation of Lamb waves and their interactions with random delaminations of different locations, shapes, sizes, and orientations in a CFRP plate.
Figure~\ref{fig:All_cases} shows the plate with 475 delaminations overlayed.
%%%%%%%%%%%%%%%%%%%%%%%%%%%%%%%%%%%%%%%%%%%%%%%%%%%%%%%%%%%%%%%%%%%%%%%%%%%%%%%%
\begin{figure} [!ht]
	\begin{center}
		\includegraphics[scale=1.0]{figure1.png}
	\end{center}
	\caption{The plate with 475 cases of random delaminations.} 
	\label{fig:All_cases}
\end{figure}
%%%%%%%%%%%%%%%%%%%%%%%%%%%%%%%%%%%%%%%%%%%%%%%%%%%%%%%%%%%%%%%%%%%%%%%%%%%%%%%%

Each simulated case in the dataset is a 3D matrix, which holds the out-of-plane displacement or velocity of propagating waves in time \((t)\) at a location \((x,y)\). 
Accordingly, these matrices can be seen as a sequence of frames at the discrete time of propagating waves, as shown in Fig.~\ref{fig:wavefield_propagation}. 
Further, for each simulated case, we have generated a \((512)\) frame.
Accordingly, the dataset has a shape of \((475\times500\times500\times512)\) in which \(475\) represents the total number of simulated cases, \((500\times500)\) represents the height and width of the frame, and \(512\) is the total number of frames per each case.
Hence, the dataset resembles high-resolution frames of propagating waves.
%%%%%%%%%%%%%%%%%%%%%%%%%%%%%%%%%%%%%%%%%%%%%%%%%%%%%%%%%%%%%%%%%%%%%%%%%%%%%%%%
\begin{figure}[!ht]
	\begin{center}
		\includegraphics[scale=1.0]{wavefield_propagation.png}
	\end{center}
	\caption{Full wavefield frames of Lamb waves propagation.} 
	\label{fig:wavefield_propagation}
\end{figure}
%%%%%%%%%%%%%%%%%%%%%%%%%%%%%%%%%%%%%%%%%%%%%%%%%%%%%%%%%%%%%%%%%%%%%%%%%%%%%%%%

%%%%%%%%%%%%%%%%%%%%%%%%%%%%%%%%%%%%%%%%%%%%%%%%%%%%%%%%%%%%%%%%%%%%%%%%%%%%%%%%
\subsection{Dataset preprocessing}
%%%%%%%%%%%%%%%%%%%%%%%%%%%%%%%%%%%%%%%%%%%%%%%%%%%%%%%%%%%%%%%%%%%%%%%%%%%%%%%%
To train deep learning models to perform super-resolution image reconstruction, we have to reproduce a low-resolution training set from the original high-resolution dataset. 
Initially, we resized the frames in the original high-resolution dataset to \((512\times512)\) pixels to obtain the desired output frame shape while performing image reconstruction from the low-to high-resolution.

In this work, we have generated a low-resolution training set with a frame size \((32\times32)\) pixels, which is below the Nyquist sampling rate of a 2D frame.
Hence, we have performed image subsampling with bi-cubic interpolation and a uniform mesh of size \((32\times32)\) pixels with a compression rate (CR) of \(19.2\%\) from the Nyquist sampling rate as depicted in Eqn.~\ref{CR}:
\begin{equation}
	CR = \frac{(Low-resolution\ dimension)^2}{(Nyquist\ sampling\ rate)^2} = \frac{(32\times32)}{(73\times73)}=19.2\%
	\label{CR}
\end{equation}

Figure~\ref{fig:SR_LR} shows three SR frames with their corresponding LR frames at different time steps.

To reduce the computation complexity during the training process of the deep learning models, we selected \((128)\) consecutive frames per each delamination case.
Frames displaying the propagation of guided waves before interacting with the delamination have no features to be extracted. 
Hence, only a certain number of frames were selected from the initial occurrence of the interactions with the delamination.

\begin{figure} [!ht]
	\centering
	\begin{subfigure}[b]{.48\textwidth}
		\centering
	\includegraphics[scale=1]{HR_image_case_475_frame_1.png}
		\caption{HR frame}
		\label{fig:SR_1}
	\end{subfigure}
	\hfill
	\begin{subfigure}[b]{.48\textwidth}
		\centering
		\includegraphics[scale=1]{LR_DLSS_case_475_frame_1.png}
		\caption{LR frame}
		\label{fig:LR_1}	
	\end{subfigure}
	\hfill
		\begin{subfigure}[b]{.48\textwidth}
		\centering
		\includegraphics[scale=1]{HR_image_case_475_frame_64.png}
		\caption{HR frame}
		\label{fig:SR_2}
	\end{subfigure}
	\hfill
	\begin{subfigure}[b]{.48\textwidth}
		\centering
		\includegraphics[scale=1]{LR_DLSS_case_475_frame_64.png}
		\caption{LR frame}
		\label{fig:LR_2}	
	\end{subfigure}
	\hfill
		\begin{subfigure}[b]{.48\textwidth}
		\centering
		\includegraphics[scale=1]{HR_image_case_475_frame_128.png}
		\caption{HR frame}
		\label{fig:SR_3}
	\end{subfigure}
	\hfill
	\begin{subfigure}[b]{.48\textwidth}
		\centering
		\includegraphics[scale=1]{LR_DLSS_case_475_frame_128.png}
		\caption{LR frame}
		\label{fig:LR_3}	
	\end{subfigure}
	\caption{High-resolution and Low-resolution frames at different time steps: (a) and (b) at $N_f=91$, (c) and (d) at $N_f=154$, and (e) and (f) at $N_f=218$.}
	\label{fig:SR_LR}
\end{figure}
%%%%%%%%%%%%%%%%%%%%%%%%%%%%%%%%%%%%%%%%%%%%%%%%%%%
\newpage