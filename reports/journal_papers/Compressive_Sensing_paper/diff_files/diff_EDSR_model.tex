\subsubsection{\DIFdelbegin \DIFdel{Model II}\DIFdelend \DIFaddbegin \DIFadd{Model-II}\DIFaddend :}
In the second developed DLSR model, we applied a simple approach of convolutional neural networks (CNNs). 
The model is composed of a total of 16 cascaded convolutional layers, in which the first layer has a kernel of size \((5\times 5)\).
The remaining layers have a kernel size of \((3\times 3)\). 
Accordingly, a ReLU (Rectified Linear Unit) activation function is applied 
after each convolution operation.
For the final output of the model, the UPNet as of Model-I with sub-pixel convolution operation is applied. 
As the input is low-resolution, the data here is also \((32\times 32)\), therefore an up-scaling factor equal to 16 is applied to acquire the HR output image of size \((512\times 512)\). 
The architecture of this model is shown in Figure~\ref{fig:Model_II}\DIFaddbegin \DIFadd{.
}

\DIFadd{It should be noted that Model-II is much simpler than Model-I (the total number of parameters in Model-II is 1 million). 
Therefore, Model-II can be trained faster than Model-I on a GPU with less memory}\DIFaddend .
%%%%%%%%%%%%%%%%%%%%%%%%%%%%%%%%%%%%%%%%%%%%%%%%%%%%%%%%%%%%%%%%%%%%%%%%%%%%%%%%
\begin{figure} [ht!]
	\begin{center}
		\includegraphics[scale=1.0]{Model_II.png}
	\end{center}
	\caption{Model-II architecture.} 
	\label{fig:Model_II}
\end{figure}
%%%%%%%%%%%%%%%%%%%%%%%%%%%%%%%%%%%%%%%%%%%%%%%%%%%%%%%%%%%%%%%%%%%%%%%%%%%%%%%%