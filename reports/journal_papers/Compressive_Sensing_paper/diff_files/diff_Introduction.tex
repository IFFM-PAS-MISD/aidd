\section{Introduction}
%%%%%%%%%%%%%%%%%%%%%%%%%%%%%%%%%%%%%%%%%%%%%%%%%%%%%%%%%%%%%%%%%%%%%%%%%%%%%%%%
Guided waves, in particular Lamb waves, are often utilised for structural health monitoring (SHM) as well as non-destructive testing (NDT).
In the former case, usually, an array of transducers is used for point-wise measurements.
These are usually piezoelectric transducers that can work as actuators and sensors, i.e., in active guided wave-based SHM.
It should be noted that round-robin actuator-sensor measurements can be conducted very fast, so nearly online monitoring of a structure is possible.

Recently, a lot of research on the application of scanning laser Doppler vibrometer (SLDV) for NDT has been reported~\cite{Flynn2013,Kudela2015, Kudela2018d, Segers2021, Segers2022}. 
In this method, either a piezoelectric transducer or pulse laser is used for guided wave excitation while the measurements are taken by SLDV at one point on the surface of an inspected structure.
The process is repeated for other points automatically in a scanning fashion until the full wavefield of Lamb waves is acquired.

Full wavefield measurements are taken on a very dense grid of points in opposite to the sparsely measured signals by sensors.
Hence, deliver much more useful data from which information about damage can be extracted in comparison to signals measured by an array of transducers.
On the other hand, SLDV measurements take much more time than measurements conducted by an array of transducers.
It makes the SLDV approach unsuitable for SHM, in which continuous monitoring is required.
But it is very capable for offline NDT applications.

One can imagine that in a future matrix of laser heads instead of a single laser head used nowadays will be developed to reduce SLDV measurement time.
Alternatively, compressive sensing (CS) and/or deep learning super-resolution (DLSR) can be applied.
It means that SLDV measurements can be taken on a low-resolution grid of points, and then the full wavefield can be reconstructed at high-resolution.

CS was originally proposed in the field of statistics~\cite{Candes2006,Donoho2006} and used for efficient acquisition and reconstruction of signals and images.
It assumes that a signal or an image can be represented in a sparse form in another domain with appropriate bases (Fourier, cosine, wavelet).
On such bases, many coefficients are close or equal to zero.
The sparsity can be exploited to recover a signal or image from fewer samples than required by the Nyquist–Shannon sampling theorem.
However, there is no unique solution for the estimation of unmeasured data.
Therefore, optimisation methods for solving under-determined systems of linear equations that promote sparsity are applied~\cite{Chen1998,VanEwoutBerg2008,VandenBerg2019}.
Moreover, a suitable sampling strategy is required.

Since then, CS has found applications in medical imaging~\cite{Lustig2007}, communication systems~\cite{Gao2018}, and seismology~\cite{Herrmann2012}.
It is also considered in the field of guided waves and ultrasonic signal processing~\cite{Harley2013,Mesnil2016,Perelli2012,Perelli2015,DiIanni2015,KeshmiriEsfandabadi2018,Chang2020}

Harley and Mura~\cite{Harley2013} utilised a general model for Lamb waves propagating in a plate structure (without defects) and $L_1$ optimisation strategies to recover their frequency-wavenumber representation. 
They applied sparse recovery by basis pursuit and sparse wavenumber synthesis.
They used a limited number of transducers and achieved a good correlation between the true and estimated responses across a wide range of frequencies.
Mesnil and Ruzzene~\cite{Mesnil2016} were focused on the reconstruction of a wavefield that includes the interaction of Lamb waves with delamination.
Similar to previous studies, analytic solutions were utilised to create a compressive sensing matrix.
However, the limitation of these methods is that dispersion curves of Lamb waves propagating in the analysed plate have to be known a priori.

Perelli et al.~\cite{Perelli2012} incorporated the warped frequency transform into a compressive sensing framework for improved damage localisation.
The wavelet packet transform and frequency warping was used in~\cite{Perelli2015} to generate a sparse decomposition of the acquired dispersive signal.

Di Ianni et al.~\cite{DiIanni2015} investigated various bases in compressive sensing to reduce the acquisition time of SLDV measurements.
Similarly, a damage detection and localisation technique based on a compressive sensing algorithm was presented in~\cite{KeshmiriEsfandabadi2018}.
The authors have shown that the acquisition time can be reduced significantly without losing detection accuracy.

Another application of compressive sensing was reported in~\cite{Chang2020}. 
The authors used signals registered by an array of sensors for tomography of corrosion.
They investigated the reconstruction success rate depending on the number of actuator-sensor paths.

The group of DLSR methods is applied mostly to images~\cite{Dahl2017,Zhang2018,Wang2019} and videos~\cite{Zhang2017,Yan2019}.
Image super-resolution (SR) is the process of recovering high-resolution images from low-resolution images.
A similar approach can be used in videos, where data is treated as a sequence of images.
Notable applications are medical imaging, satellite imaging, surveillance and security, and astronomical imaging, amongst others.
Also, deep learning super sampling developed by Nvidia and FidelityFX super-resolution developed by AMD were adopted for video games~\cite{Claypool2006}.
Mostly supervised techniques are employed, which benefit from recent advancements in deep learning methods ranging from enhanced convolutional neural networks (CNN)~\cite{Zhang2017}, through an extension of PixelCNN~\cite{Dahl2017} to generative adversarial networks (GANs)~\cite{Wang2019}, to name a few.
Nevertheless, so far neither of these methods has been applied to the wavefields of propagating Lamb waves.
The exception is an enhancement of wavefields as the second step of SR \DIFdelbegin \DIFdel{followed }\DIFdelend \DIFaddbegin \DIFadd{preceded }\DIFaddend by classic CS~\cite{Park2017a,KeshmiriEsfandabadi2020}.

We propose a framework for full wavefield reconstruction of propagating Lamb waves from spatially sparse SLDV measurements of resolution below the Nyquist wavelength $\lambda_N$. 
The Nyquist wavelength is the shortest spatial wavelength that can be accurately recovered from a wavefield by sequential observations with spacing $\Delta x$ which is defined as $\lambda_N = 2 \Delta x$. 

For the first time, an end-to-end approach for the SR problem is used in which a deep learning neural network is trained on a synthetic dataset and tested on experimental data acquired by SLDV.
It means that the approach is solely based on DLSR.
It is different from methods presented in the literature which utilize CS theory~\cite{Harley2013, KeshmiriEsfandabadi2018} or CS theory in conjunction with super-resolution convolutional neural networks for wavefield image enhancement~\cite{Park2017a, KeshmiriEsfandabadi2020}.
The efficacy of the developed framework is presented and compared with the conventional CS approach.
The performance of the proposed technique is validated by an experiment performed on a plate made of carbon fibre reinforced polymer (CFRP) with embedded Teflon inserts simulating delaminations.
