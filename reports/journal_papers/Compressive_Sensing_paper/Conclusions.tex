\section{Conclusions}
\label{conclusion}
In this paper, we demonstrate two super-resolution deep learning techniques to reconstruct the HR full wavefield of propagating guided waves.
The motivation for using two deep learning models was to check the feasibility of different deep learning architectures for our research task. 
Hence, Model-I is computationally complex as it is composed of a large number of trainable parameters that use a Residual Dense Network (RDN) architecture, in which it is composed of many residual dense blocks (RDB). 
On the other hand, Model-II is less complex than Model-I as it is composed of 16 cascaded layers of convolutional neural networks (CNNs).
Accordingly, a large synthetic dataset was generated, resembling the full wavefields acquired by an SLDV.
In DLSR models, the utilised compression rate was \(19.2\%\)\ of the Nyquist sampling rate.
To see the feasibility of such a study, we have compared the DLSR models with the conventional CS technique. 
The results were promising, and the deep learning models surpassed the conventional technique in reconstructing the full wavefield frames for heavily sub-sampled cases.
Additionally, DLSR models showed their ability to generalise by reconstructing the full wavefield frames acquired experimentally by SLDV.
It has been found that the accuracy of the developed DLSR models is similar.
Consequently, DLSR leads to a slightly better reconstruction of the wavefield than CS, and it outperforms it for the reconstruction of the wavefield in the area of delamination.
Furthermore, deep learning techniques for SR can highly enhance the speed of data acquisition by SLDV.

