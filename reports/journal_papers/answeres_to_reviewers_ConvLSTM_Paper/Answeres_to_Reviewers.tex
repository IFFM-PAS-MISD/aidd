\documentclass[11pt,a2paper]{report}
\usepackage[dvipsnames]{xcolor}
%\usepackage{dirtytalk}
\usepackage{graphicx}
\usepackage{multirow}
\usepackage{amsmath,amssymb,bm}
%\usepackage[dvips,colorlinks=true,citecolor=green]{hyperref}
\usepackage[colorlinks=true,citecolor=green]{hyperref}
%% my added packages
\usepackage{float}
\usepackage{csquotes}
\usepackage{verbatim}
\usepackage{caption}
\usepackage{subcaption}
\usepackage{booktabs} % for nice tables
\usepackage{csvsimple} % for csv read
\usepackage{graphicx}
%\usepackage[outdir=//odroid-sensors/sensors/aidd/reports/journal_papers/MSSP_Paper/Figures/]{epstopdf}
%\usepackage{breqn}
\usepackage{multirow}
\newcommand{\RNum}[1]{\uppercase\expandafter{\romannumeral #1\relax}}

\begin{document}
	
	\noindent We appreciate the time and effort that the reviewers have dedicated to provide valuable feedback on our manuscript. 
	We would like to thank the reviewers for constructive comments which helped us to improve the manuscript. 
	We have incorporated changes to reflect the suggestions provided by the reviewers. 
	We have highlighted the changes in a separate differential PDF document. The additional text is in the blue print. 
	The removed text is in red. \\ \\
	Here is a point-by-point response to the reviewers’ comments and concerns.
	\\ \\
	\textbf{Reviewer \# 1}: \\
	In this paper, authors proposed two end-to-end deep learning-based models 
	of many-to-one sequence prediction to perform pixel-wise image 
	segmentation. Extensive experiments verified the effectiveness and 
	efficiency of the proposed method. This paper investigates an interesting 
	problem, and the structure is relative good. However, minor revisions are 
	needed before the acceptance.
	
	\textcolor{Cyan}{
	\newline\textbf{Response:}
	Thank you for your positive feedback.
    }
	\begin{enumerate}
		\item It seems in the proposed framework, most components (e.g., 
		ConvLSTM) are well-studied by others. Therefore, what is the main novel 
		contribution of this work when adapting the well-studied techniques to 
		this work.
		
		\textcolor{Cyan}{
			\textbf{Response:}
		---------------.
		}
		
		\item Can this work be applied to other detection waves besides the 
		mentioned Lamb waves?
		
		\textcolor{Cyan}{
			\textbf{Response:}
			---------------.
		}
		\item For the evaluation, the dataset/algorithm choosing reasons, the 
		detailed platform configurations and the discussion on other untested 
		datasets should be introduced in the revised paper.
		
		\textcolor{Cyan}{
			\textbf{Response:\\}
			Thank you for your constructive comment. \\
		}
	
			\item Please add a notation table to explain all used math symbols 
			for easy understanding.
	
	\textcolor{Cyan}{
		\textbf{Response:}
		---------------.
	}	

			\item Please go through the paper carefully and double check 
			whether the right template are used. Correct some typos and 
			formatting issues (e.g., in Figure 7"Model I" $\to$ "Model-I"?).

    \textcolor{Cyan}{
	\textbf{Response:}
	---------------.
}

			\item Some references lack the necessary information (e.g., [10]), 
			please provide all information according to the right template.

\textcolor{Cyan}{
	\textbf{Response:}
	---------------.
}

			\item Make the References more comprehensive, besides delamination 
			identification, some other promising scenarios (e.g., Big data, 
			other IoT systems) can be covered in this work. If the above 
			related work can be discussed, it can strongly improve the research 
			significance. For the improvement, the following papers can be 
			considered to make the references more comprehensive.
			\newline
			\newline
			\newline
			J. Wang, Y. Zou, P. Lei, et al. Research on Recurrent Neural 
			Network Based Crack Opening Prediction of Concrete Dam. Journal of 
			Internet Technology, 2020, 21(4):1151-1160
			\newline
			\newline
			Bin Pu, Kenli Li, Shengli Li, Ningbo Zhu: Automatic Fetal 
			Ultrasound Standard Plane Recognition Based on Deep Learning and 
			IIoT. IEEE Trans. Ind. Informatics 17(11): 7771-7780 (2021)
			\newline
			\newline
			Dun Cao, Kai Zeng, Jin Wang, Pradip Kumar Sharma, Xiaomin Ma, and 
			Yonghe Liu, BERT-based Deep Spatial-Temporal Network for Taxi 
			Demand 
			Prediction. IEEE Transactions on Intelligent Transportation 
			Systems, 2021, DOI:10.1109/TITS.2021.3122114
			\newline
			\newline
			Cen Chen, Kenli Li, Sin G. Teo, Xiaofeng Zou, Keqin Li, Zeng Zeng: 
			Citywide Traffic Flow Prediction Based on Multiple Gated 
			Spatio-temporal Convolutional Neural Networks. ACM Trans. Knowl. 
			Discov. Data 14(4): 42:1-42:23 (2020)
			\newline
			\newline
			J. Wang, Y. Yang, T. Wang, R. Sherratt, J. Zhang. Big Data Service 
			Architecture: A Survey. Journal of Internet Technology, 2020, 
			21(2): 393-405
			\newline
			\newline
			Jianguo Chen, Kenli Li, Kashif Bilal, Xu Zhou, Keqin Li, Philip S. 
			Yu: A Bi-layered Parallel Training Architecture for Large-Scale 
			Convolutional Neural Networks. IEEE Trans. Parallel Distributed 
			Syst. 30(5): 965-976 (2019)
			\newline
			\newline
			J. Zhang, S. Zhong, T. Wang, H.-C. Chao, J. Wang. Blockchain-Based 
			Systems and Applications: A Survey. Journal of Internet Technology, 
			2020, 21(1): 1-14
			\newline
			\newline
			Om, K., Boukoros, S., Nugaliyadde, A., McGill, T., Dixon, M., 
			Koutsakis, P., \& Wong, K. W., "Modelling email traffic workloads 
			with RNN and LSTM models." Human-centric Computing and Information 
			Sciences, vol. 10, article no. 39, 2020. 
			\url{https://doi.org/10.1186/s13673-020-00242-w}
			

\textcolor{Cyan}{
	\textbf{Response:}
	---------------.
}	
\end{enumerate}	
	
\newpage 
\textbf{Reviewer \# 2}:
\newline The authors proposed a ConvLSTM-based DL method to detect delamination 
using Lamb waves, which extends its application fields. The animation derived 
from SLDV coincides with the target objects of ConvLSTM network. Two models are 
presented and are adequately verified by numerical and experimental cases. 
Moreover, the whole essay is well-organized and well-written. In a word, it is 
an interesting and valuable work. Therefore, I would recommend acceptance of 
this submission after a minor revision. Below are some suggestions that can be 
considered for the authors to further improve the manuscript.

\textcolor{Cyan}{
	\newline\textbf{Response:}
	---------------.
}
\begin{enumerate}
	\item Without the support of SLDV, can this method work well? As SLDV is 
	not suitable for in-situ case and it is expensive for most guys.
	
	\textcolor{Cyan}{
		\textbf{Response:}
		---------------.
	}
	
	\item The ConvLSTM was firstly presented by Shi in 2015. The relevant study 
	about ConvLSTM and its application in other areas are not mentioned in 
	Introduction section. Please make a more comprehensive literature research.
	
	\textcolor{Cyan}{
		\textbf{Response:}
		---------------.
	}
	\item Page 6, Why the coordinates of the center of delamination is 
	restricted to [0mm,240mm] and [260mm,500mm], rather than [0mm,500mm] and 
	[0mm,500mm], which is the same as the size of specimen plate?
	
	\textcolor{Cyan}{
		\textbf{Response:}
		Thank you for pointing this out.\\
		The coordinates of centres of delamination \(C_{xy}\) $\epsilon$ [0 \dots 500] mm, but $\delta$ was added to prevent a delamination from being synthetically modelled at the center of the specimen as there is a PZT actuator.
	}
	
	\item Page 8, "k represents the total number of windows", while in Eq (1), 
	N the total number of windows. Please keep the notation consistent.
	
	\textcolor{Cyan}{
		\textbf{Response:}
		---------------.
	}	
	
	\item Fig.5b shows the architecture of Model-2, but not detailed enough. It 
	is suggested to mark every level of encoder, bottleneck and decoder parts.
	
	\textcolor{Cyan}{
		\textbf{Response:} \\
		Thank you for your constructive comment.\\
		A detailed architecture of Model-II presented in Figure 6 was added to the revised paper.
	}
	
	\item Page 18, Line 6, small mistake, 500 × 500 mm2.
	
	\textcolor{Cyan}{
		\textbf{Response:}
		---------------.
	}	

	\item In experiment case with single delamination there are 256 frames in 
	total, while the others are 512 frames. Is there any difference or 
	requirements regarding the number of test frames?

\textcolor{Cyan}{
	\textbf{Response:}
	---------------.
}

	\item Why m=64 for Model-\RNum{1}, which is larger than Model-\RNum{2}, 
	m=24?

\textcolor{Cyan}{
	\textbf{Response:}
	---------------.
}

	\item According to the manuscript, my understanding about the procedure is 
	as follows: In the training stage, only a certain number of frames that are 
	connected with delamination were selected as input, while in test stage all 
	the frames are included in the calculation by shifting the window. And 
	every intermediate prediction map corresponds with a window. Am I right? If 
	so, it might be more understandable if some intermediate prediction maps 
	are displayed in the results section.

\textcolor{Cyan}{
	\textbf{Response:} \\
	Thank you for pointing this out. \\
	Indeed, this is the general procedure of the developed work in this paper.\\
	For training the deep learning models, a certain number of consecutive frames were selected, which present the initial interaction of guided waves with the delamination.
	However, in real-life situations, the damage location is unknown.
	Hence the starting interaction of guided waves with the damage is unknown.
	Accordingly, we calculate the RMS for all intermediate predictions to get one damage map. \\
	Additionally, we added Figure 10c to present the intermediate predictions at a different group of frames.
}

	\item Table 2 compares the IoU among 5 models. It is also advised to show 
	the predicted damage maps of FCN-DenseNet and GCN with the RMS images as 
	input.

\textcolor{Cyan}{
	\textbf{Response:}
	---------------.
}

\end{enumerate}	

\newpage 
\textbf{Reviewer \# 3}:
\newline This paper proposed an end-to-end scheme deep learning-based approach 
for delamination identification in composite laminates, and authors tested the 
method on several experimentally measured cases of single and multiple 
delaminations simulated by Teflon inserts. The reviewer has the following 
suggestions for the authors to consider:

\textcolor{Cyan}{
	\newline\textbf{Response:}
	Thank you for your positive feedback.
}
\begin{enumerate}
	\item The reviewer believes that the limitations of this paper include the 
	following two points, which limit the scope of application of this paper: 
	1. Lamb wave detection, 2. The tested material. The first is that Lamb wave 
	detection is fundamentally different from other forms of ultrasound (such 
	as SH), and the complex dispersion characteristics of Lamb waves are not 
	discussed in this paper; secondly, changes in the tested material will 
	cause the entire model to become unusable, that is, The scope of 
	application of the proposed method is relatively narrow.
	
	\textcolor{Cyan}{
		\textbf{Response:}
		---------------.
	}
	
	\item What are the advantages and disadvantages of the proposed model 
	compared to RMS-based models? Since SLDV is stationary and time-consuming, 
	this means that it can only be used offline and cannot be used for online 
	monitoring.
	
	\textcolor{Cyan}{
		\textbf{Response:}
		---------------.
	}
	\item Figure 8 needs more explanation, besides, "when a binary threshold is 
	applied, most identifed delaminations vanish as the pixel values are below 
	the threshold." Would an adjustable threshold perform better?
	
	\textcolor{Cyan}{
		\textbf{Response:}
		---------------.
	}
	
	\item Taking Fig. 7 as an example, the meaning of the colors needs to be 
	supplemented.
	
	\textcolor{Cyan}{
		\textbf{Response:}
		---------------.
	}	
	
\end{enumerate}	

\end{document}