\documentclass[11pt,a2paper]{report}
\usepackage[dvipsnames]{xcolor}
%\usepackage{dirtytalk}
\usepackage{graphicx}
\usepackage{multirow}
\usepackage{amsmath,amssymb,bm}
%\usepackage[dvips,colorlinks=true,citecolor=green]{hyperref}
\usepackage[colorlinks=true,citecolor=green]{hyperref}
%% my added packages
\usepackage{float}
\usepackage{csquotes}
\usepackage{verbatim}
\usepackage{caption}
\usepackage{subcaption}
\usepackage{booktabs} % for nice tables
\usepackage{csvsimple} % for csv read
\usepackage{graphicx}
\usepackage{geometry}
%\usepackage{showframe} %This line can be used to clearly show the new margins

\newgeometry{vmargin={25mm}, hmargin={30mm,30mm}}
%\usepackage[outdir=//odroid-sensors/sensors/aidd/reports/journal_papers/MSSP_Paper/Figures/]{epstopdf}
%\usepackage{breqn}
\usepackage{multirow}
\newcommand{\RNum}[1]{\uppercase\expandafter{\romannumeral #1\relax}}
\graphicspath{{figures/}}

\begin{document}
	
	\noindent We appreciate the time and effort that the reviewers have dedicated to provide valuable feedback on our manuscript. 
	We would like to thank the reviewers for constructive comments which helped us to improve the manuscript. 
	We have incorporated changes to reflect the suggestions provided by the reviewers. 
	We have highlighted the changes in a separate differential PDF document. The additional text is in the blue print. 
	The removed text is in red. \\ \\
	Here is a point-by-point response to the reviewers’ comments and concerns.
	\\ \\
	\textbf{Reviewer \# 1}: \\
	In this paper, authors proposed two end-to-end deep learning-based models 
	of many-to-one sequence prediction to perform pixel-wise image segmentation.
	Extensive experiments verified the effectiveness and efficiency of the 
	proposed method.
	This paper investigates an interesting problem, and the structure is 
	relative good. However, minor revisions are needed before the acceptance.
	
	\textcolor{Cyan}{
	\newline\textbf{Response:}
	We would like to thank the reviewer for his positive feedback.
    }
	\begin{enumerate}
		\item It seems in the proposed framework, most components (e.g., 
		ConvLSTM) are well-studied by others.
		Therefore, what is the main novel contribution of this work when 
		adapting the well-studied techniques to this work.
		
		\textcolor{Cyan}{
			\textbf{Response:}
		Thank you for this important question.
		The main novel contribution of this work is the generation of a large 
		synthetic dataset of full wavefields, which resembles the data acquired 
		from SLDV in a real-world scenario. 
		As acquiring the data from SLDV is very expensive, time-consuming, and 
		unfeasible, therefore a large dataset of full wavefields is simulated 
		in this work and the dataset is made publicly available. 
		\newline The other novelty in this work is the employment of 
		deep-learning models directly on sequences of full wavefields 
		(animations), without the need for a baseline wavefield.
		Previously, the researchers have to apply some intermediary steps on 
		full wavefields such as (RMS) before feeding the data into a 
		deep-learning model for producing one-to-one sequences for each 
		delamination case, which is usually time-consuming, and cumbersome.
		In this work, we have proposed a technique in which applying such 
		intermediary step of applying RMS is omitted.
		Furthermore, we have also compared the results obtained from our 
		approach (without using RMS) with the results obtained from the 
		deep-learning models which use RMS.}
		
		\item Can this work be applied to other detection waves besides the 
		mentioned Lamb waves?
		
		\textcolor{Cyan}{
			\textbf{Response:}
			Thank you for this constructive comment.
			Yes, this work can be enhanced to other detection waves besides the 
			Lamb waves by providing such data to our proposed model and 
			training the proposed model with the required dataset on which the 
			detection needs to be performed.
		}
		\item For the evaluation, the dataset/algorithm choosing reasons, the 
		detailed platform configurations and the discussion on other untested 
		datasets should be introduced in the revised paper.
		
		\textcolor{Cyan}{
			\textbf{Response:\\}
			Thank you for your constructive comment. \\
			We have evaluated the developed models on a synthetic test set and experimentally acquired data.
			We selected the test set randomly for the evaluation process from the complete dataset (475 different scenarios).
			Furthermore, we split the complete dataset into two sets: 
			1. Training set $80\%$ (380 scenarios). 
			2. Test set $20\%$ (95 scenarios).
			It is important to note that we trained the developed models only on the training set portion and not on the test set, which is only utilised for evaluating the models.
			Hence, the test set portion is considered unseen before the models during the evaluation phase.
			Also, the experimentally acquired data is considered unseen by the developed models because, during the training phase, only the training set was utilised.
		}
	
			\item Please add a notation table to explain all used math symbols 
			for easy understanding.
	
	\textcolor{Cyan}{
		\textbf{Response:}
	Thank you for your suggestion.
	We are sorry but we cannot fully agree with the reviewer because the RNNs, 
	LSTMs, ConvLSTM has a wide range of literature, which can be checked from 
	there. 
	We have briefly introduced all these topics in our paper and all 
	the symbols are described in the text.
    And other math symbols used in the paper such as in the dataset section and 
    results and discussion section are very simple and described well in the 
    paper and we believe that it is not needed to add an extra table for the 
    explanation of these math symbols.}	

			\item Please go through the paper carefully and double check 
			whether the right template are used. Correct some typos and 
			formatting issues (e.g., in Figure 7"Model I" $\to$ "Model-I"?).

    \textcolor{Cyan}{
	\textbf{Response:}
	Thank you for your constructive suggestion.
	We have reviewed the paper carefully, and double checked the template of 
	the paper.
	We have also corrected some typos and formatting issues in the updated 
	version of the paper.
}

			\item Some references lack the necessary information (e.g., [10]), 
			please provide all information according to the right template.

\textcolor{Cyan}{
	\textbf{Response:}
	Thank you for pointing this out.
	We have double checked all the references, and references [3, 10] have been 
	removed from the updated version of the paper.
}

			\item Make the References more comprehensive, besides delamination 
			identification, some other promising scenarios (e.g., Big data, 
			other IoT systems) can be covered in this work.
			If the above related work can be discussed, it can strongly improve 
			the research significance. For the improvement, the following 
			papers can be considered to make the references more comprehensive.
			\newline
			\newline
			\newline
			J. Wang, Y. Zou, P. Lei, et al. Research on Recurrent Neural 
			Network Based Crack Opening Prediction of Concrete Dam. Journal of 
			Internet Technology, 2020, 21(4):1151-1160
			\newline
			\newline
			Bin Pu, Kenli Li, Shengli Li, Ningbo Zhu: Automatic Fetal 
			Ultrasound Standard Plane Recognition Based on Deep Learning and 
			IIoT. IEEE Trans. Ind. Informatics 17(11): 7771-7780 (2021)
			\newline
			\newline
			Dun Cao, Kai Zeng, Jin Wang, Pradip Kumar Sharma, Xiaomin Ma, and 
			Yonghe Liu, BERT-based Deep Spatial-Temporal Network for Taxi 
			Demand 
			Prediction. IEEE Transactions on Intelligent Transportation 
			Systems, 2021, DOI:10.1109/TITS.2021.3122114
			\newline
			\newline
			Cen Chen, Kenli Li, Sin G. Teo, Xiaofeng Zou, Keqin Li, Zeng Zeng: 
			Citywide Traffic Flow Prediction Based on Multiple Gated 
			Spatio-temporal Convolutional Neural Networks. ACM Trans. Knowl. 
			Discov. Data 14(4): 42:1-42:23 (2020)
			\newline
			\newline
			J. Wang, Y. Yang, T. Wang, R. Sherratt, J. Zhang. Big Data Service 
			Architecture: A Survey. Journal of Internet Technology, 2020, 
			21(2): 393-405
			\newline
			\newline
			Jianguo Chen, Kenli Li, Kashif Bilal, Xu Zhou, Keqin Li, Philip S. 
			Yu: A Bi-layered Parallel Training Architecture for Large-Scale 
			Convolutional Neural Networks. IEEE Trans. Parallel Distributed 
			Syst. 30(5): 965-976 (2019)
			\newline
			\newline
			J. Zhang, S. Zhong, T. Wang, H.-C. Chao, J. Wang. Blockchain-Based 
			Systems and Applications: A Survey. Journal of Internet Technology, 
			2020, 21(1): 1-14
			\newline
			\newline
			Om, K., Boukoros, S., Nugaliyadde, A., McGill, T., Dixon, M., 
			Koutsakis, P., \& Wong, K. W., "Modelling email traffic workloads 
			with RNN and LSTM models." Human-centric Computing and Information 
			Sciences, vol. 10, article no. 39, 2020. 
			\url{https://doi.org/10.1186/s13673-020-00242-w}
			

\textcolor{Cyan}{
	\textbf{Response:}
	Thank you for your constructive comment and for recommending us a useful 
	list of references.
	We have made our references more comprehensive and added literature from 
	the recommended papers in the "Introduction" and in the 
	"Introduction to RNNs, LSTM, and ConvLSTM" sections in the updated version 
	of the paper.  
}	
\end{enumerate}	
	
\newpage 
\textbf{Reviewer \# 2}:
\newline The authors proposed a ConvLSTM-based DL method to detect delamination 
using Lamb waves, which extends its application fields.
The animation derived from SLDV coincides with the target objects of ConvLSTM 
network. 
Two models are presented and are adequately verified by numerical and 
experimental cases. 
Moreover, the whole essay is well-organized and well-written.
In a word, it is an interesting and valuable work. Therefore, I would recommend 
acceptance of this submission after a minor revision. Below are some 
suggestions that can be considered for the authors to further improve the 
manuscript.

\textcolor{Cyan}{
	\newline\textbf{Response:}
	We would like to thank the reviewer for his efforts in reviewing our 
	manuscript.
}
\begin{enumerate}
	\item Without the support of SLDV, can this method work well? As SLDV is 
	not suitable for in-situ case and it is expensive for most guys.
	
	\textcolor{Cyan}{
		\textbf{Response:}
		Thank you for this important question.
		Yes, this method can work well without the support of SLDV, as 
		mentioned in the paper that these full wavefields of Lamb waves can be 
		acquired via the employment of a very dense array of transducers.
		Therefore, if SLDV is not available such techniques can be applied for 
		acquiring the Lamb waves-based data.  
	}
	
	\item The ConvLSTM was firstly presented by Shi in 2015.
	The relevant study about ConvLSTM and its application in other areas are 
	not mentioned in Introduction section.
	Please make a more comprehensive literature research.
	
	\textcolor{Cyan}{
		\textbf{Response:}
		Thank you for pointing this out.
		We did not introduce ConvLSTM well in the Introduction section of the 
		paper. 
		Therefore, we are not adding any literature/applications of ConvLSTM in 
		the Introduction section of the paper.
		However, the relevant study about ConvLSTM and its applications in 
		other areas have been added as the following text to section 2.2 
		Introduction to RNNs, LSTM, and ConvLSTM:
		\newline"Recently, ConvLSTM has become very popular and is increasingly 
		being used in various domains for capturing spatio-temporal features 
		such as traffic flow prediction in intelligent transportation 
		system [\textcolor{red}{61}], surveillance-based 
		systems [\textcolor{red}{62}], remaining useful life prediction of 
		rolling bearings [\textcolor{red}{63}], crop yield 
		forecasting [\textcolor{red}{64}], action recognition and anomaly 
		detection in videos [\textcolor{red}{65, 66}], and many more image 
		processing and computer vision-related applications."
	}
	\item Page 6, Why the coordinates of the center of delamination is 
	restricted to [0mm,240mm] and [260mm,500mm], rather than [0mm,500mm] and 
	[0mm,500mm], which is the same as the size of specimen plate?
	
	\textcolor{Cyan}{
		\textbf{Response:}
		Thank you for pointing this out.\\
		The coordinates of centres of delamination \(C_{xy}\) $\epsilon$ [0 \dots 500] mm, but $\delta$ was added to prevent a delamination from being synthetically modelled at the center of the specimen as there is a PZT actuator.
	}
	
	\item Page 8, "k represents the total number of windows", while in Eq (1), 
	N the total number of windows.
	Please keep the notation consistent.
	
	\textcolor{Cyan}{
		\textbf{Response:}
		Thank your for your constructive comment. 
		We double checked the paper and all the notations.
		All the notations are consistent now in the updated version of the 
		paper. 
		Furthermore, \(N\) is the number of all samples/frames in a single 
		delamination scenario not the total number of windows.
		This information has also been added to the modified version of the 
		paper as:
		"where \(N\) represents the total number of samples in a single 
		delamination case:"
	}	
	
	\item Fig.5b shows the architecture of Model-2, but not detailed enough.
	It is suggested to mark every level of encoder, bottleneck and decoder 
	parts.
	
	\textcolor{Cyan}{
		\textbf{Response:} \\
		Thank you for your constructive comment.\\
		A detailed architecture of Model-II presented in Figure 6 was added to the revised paper.
	}
	
	\item Page 18, Line 6, small mistake, 500 × 500 mm2.
	
	\textcolor{Cyan}{
		\textbf{Response:}
		Thank you for pointing this out.
		We have made the required correction in the updated version of the 
		paper.
	}	

	\item In experiment case with single delamination there are 256 frames in 
	total, while the others are 512 frames.
	Is there any difference or requirements regarding the number of test frames?

\textcolor{Cyan}{
	\textbf{Response:} \\
	Thank you for pointing this out. \\
	}

	\item Why \(m\)=64 for Model-\RNum{1}, which is larger than Model-\RNum{2}, 
	\(m\)=24?

\textcolor{Cyan}{
	\textbf{Response:}
	Thank you for your constructive question.
	The selection of the value of \(m\) for these two models was based on how 
	much data can each model handle at a time and how the model's complexity 
	affects its capability of predicting the damage size, location, and shape, 
	and how the models would generalise with experimentally acquired data. 
	Therefore, model-I is considered simple and requires less parameters as it 
	is not deep, however, it requires much more data in order to identify 
	delaminations. 
	On the contrary, model-II is much more complex and it is considered a very 
	deep model as it has many layers at the encoder, consequently, this implies 
	that this model requires much more parameters due to its complexity and 
	requires less data for the identification of delaminations.
	Therefore, we assigned \(m\) = 64 for model-I in order to acquire more data 
	and, \(m\) = 24 for model-II in order to provide less data to model-II.
}

	\item According to the manuscript, my understanding about the procedure is 
	as follows: In the training stage, only a certain number of frames that are 
	connected with delamination were selected as input, while in test stage all 
	the frames are included in the calculation by shifting the window.
	And every intermediate prediction map corresponds with a window. 
	Am I right? If so, it might be more understandable if some intermediate 
	prediction maps are displayed in the results section.

\textcolor{Cyan}{
	\textbf{Response:} \\
	Thank you for pointing this out. \\
	Indeed, this is the general procedure of the developed work in this paper.\\
	For training the deep learning models, a certain number of consecutive frames were selected, which present the initial interaction of guided waves with the delamination.
	However, in real-life situations, the damage location is unknown.
	Hence the starting interaction of guided waves with the damage is unknown.
	Accordingly, we calculate the RMS for all intermediate predictions to get one damage map. \\
	Additionally, we added Figure 10c to present the intermediate predictions at a different group of frames.
}

	\item Table 2 compares the IoU among 5 models. 
	It is also advised to show 
	the predicted damage maps of FCN-DenseNet and GCN with the RMS images as 
	input.

\textcolor{Cyan}
{
	\textbf{Response:}
	Thank you for your constructive comment.\\
	\begin{table}[ht!]
		\centering
		\begin{tabular}{lccc}
			\toprule
			& \multicolumn{3}{l}{Numerical case number} \\
			& \(1^{st}\) & \(2^{nd}\) & \(3^{rd}\) \\ 
			\midrule
			FCN-DenseNet[46] & 0.56 & 0.58 & 0.75 \\
			GCN[46]          & 0.64 & 0.71 & 0.94 \\ 
			\bottomrule
		\end{tabular}
	\caption{IoU metric of the three numerical cases with respect to model FCN-DenseNet[46] and GCN[46]}
	\label{num_cases}
	\end{table}
Table~\ref{num_cases} presents the IoU values for the three numerical test cases presented in the manuscript.
Compared to Table 2 presented in the manuscript, the damage maps predicted by previously developed models: FCN-DenseNet[46] and GCN[46], show lower IoU values except for the third numerical test case (the simplest case).
Figure~\ref{fig:RMS_num_cases} presents damage maps predicted by FCN-DenseNet[46] and GCN[46] models.
The RMS input images in Figs.~\ref{fig:num_GT_391},~\ref{fig:num_GT_462} and~\ref{fig:Convlstm_binary_RMS_453}, respectively.
Figures~\ref{fig:Convlstm_binary_RMS_391},~\ref{fig:Convlstm_binary_RMS_462} and~\ref{fig:Convlstm_binary_RMS_453} show the damage maps predicted by FCN-DenseNet[46] model.
Figures~\ref{fig:num_GT_391},~\ref{fig:num_GT_462} and~\ref{fig:num_GT_453} show the damage maps predicted by GCN[46] model.
In our opinion, we would like not to add Fig.~\ref{fig:RMS_num_cases} in the revised paper, as we believe it may not add extra info to readers, as readers may refer to [45, 46].
}
	\begin{figure}[ht!]
		\centering
		%%%%%%%%%%%%%%%%%%%%%%%%%%%%%%%%%%%%%%%%%%%%%%%%%%%%%%%%%%%%%%%%%%%%%%%%
		\begin{subfigure}[b]{0.32\textwidth}
			\centering
			\includegraphics[width=1\textwidth]{RMS_flat_shell_Vz_391_500x500bottom.png}
			\caption{RMS (\(1^{st}\) case)}
			\label{fig:num_GT_391}
		\end{subfigure}
		\hfill
		\begin{subfigure}[b]{0.32\textwidth}
			\centering
			\includegraphics[width=1\textwidth]{fcn_densenet_num_41.png}
			\caption{FCN-DenseNet}
			\label{fig:Convlstm_binary_RMS_391}
		\end{subfigure}
		\hfill
		\begin{subfigure}[b]{0.32\textwidth}
			\centering
			\includegraphics[width=1\textwidth]{GCN_model41.png}
			\caption{GCN}
			\label{fig:AE_binary_RMS_391}
		\end{subfigure}
		%%%%%%%%%%%%%%%%%%%%%%%%%%%%%%%%%%%%%%%%%%%%%%%%%%%%%%%%%%%%%%%%%%%%%%%%
		\par\medskip
		%%%%%%%%%%%%%%%%%%%%%%%%%%%%%%%%%%%%%%%%%%%%%%%%%%%%%%%%%%%%%%%%%%%%%%%%
		\begin{subfigure}[b]{0.32\textwidth}
			\centering
			\includegraphics[width=1\textwidth]{RMS_flat_shell_Vz_462_500x500bottom.png}
			\caption{RMS (\(2^{nd}\) case)}
			\label{fig:num_GT_462}
		\end{subfigure}
		\hfill
		\begin{subfigure}[b]{0.32\textwidth}
			\centering
			\includegraphics[width=1\textwidth]{fcn_densenet_num_325.png}
			\caption{FCN-DenseNet}
			\label{fig:Convlstm_binary_RMS_462}
		\end{subfigure}
		\hfill
		\begin{subfigure}[b]{0.32\textwidth}
			\centering
			\includegraphics[width=1\textwidth]{GCN_model325.png}
			\caption{GCN}
			\label{fig:AE_binary_RMS_462}
		\end{subfigure}
		%%%%%%%%%%%%%%%%%%%%%%%%%%%%%%%%%%%%%%%%%%%%%%%%%%%%%%%%%%%%%%%%%%%%%%%%
		\par\medskip
		%%%%%%%%%%%%%%%%%%%%%%%%%%%%%%%%%%%%%%%%%%%%%%%%%%%%%%%%%%%%%%%%%%%%%%%%
		\begin{subfigure}[b]{0.32\textwidth}
			\centering
			\includegraphics[width=1\textwidth]{RMS_flat_shell_Vz_453_500x500bottom.png}
			\caption{RMS (\(3^{rd}\) case)}
			\label{fig:num_GT_453}
		\end{subfigure}
		\hfill	
		\begin{subfigure}[b]{0.32\textwidth}
			\centering
			\includegraphics[width=1\textwidth]{fcn_densenet_num_289.png}
			\caption{FCN-DenseNet}
			\label{fig:Convlstm_binary_RMS_453}
		\end{subfigure}
		\hfill
		\begin{subfigure}[b]{0.32\textwidth}
			\centering
			\includegraphics[width=1\textwidth]{GCN_model289.png}
			\caption{GCN}
			\label{fig:AE_binary_RMS_453}
		\end{subfigure}
		\caption{Predictions of FCN-DenseNet[46] and GCN[46] with respect to the three numerical test cases.}
		\label{fig:RMS_num_cases}
	\end{figure}


\end{enumerate}	

\newpage 
\textbf{Reviewer \# 3}:
\newline This paper proposed an end-to-end scheme deep learning-based approach 
for delamination identification in composite laminates, and authors tested the 
method on several experimentally measured cases of single and multiple 
delaminations simulated by Teflon inserts.
The reviewer has the following suggestions for the authors to consider:

\textcolor{Cyan}{
	\newline\textbf{Response:}
	Thank you for your positive feedback.
}
\begin{enumerate}
	\item The reviewer believes that the limitations of this paper include the 
	following two points, which limit the scope of application of this paper: 
	1. Lamb wave detection, 2. The tested material. The first is that Lamb wave 
	detection is fundamentally different from other forms of ultrasound (such 
	as SH), and the complex dispersion characteristics of Lamb waves are not 
	discussed in this paper; secondly, changes in the tested material will 
	cause the entire model to become unusable, that is, The scope of 
	application of the proposed method is relatively narrow.
	
	\textcolor{Cyan}{
		\textbf{Response:}
		---------------.
	}
	
	\item What are the advantages and disadvantages of the proposed model 
	compared to RMS-based models? Since SLDV is stationary and time-consuming, 
	this means that it can only be used offline and cannot be used for online 
	monitoring.
	
	\textcolor{Cyan}{
		\textbf{Response:}
		Thank you for your constructive comment.
		The advantages of the proposed model compared to RMS-based models is 
		that proposed model can be employed on the sequences (animations) of 
		full wavefields, which contains very rich information regarding the 
		damage, the damage can be easily identified with the use of the 
		proposed method. 
		However, in the case of RMS-based models only one image can be 
		generated for a single damage case.
		Therefore, it cannot identify the damage very well as compared to the 
		proposed model.  
		\newline Regarding the second comment, we agree with the reviewer that 
		SLDV is 
		still stationary, time-consuming and can only be used offline and 
		cannot be used for online monitoring.
		We have also added the following text in the conclusion section of our 
		paper:
		\newline "However, there are several limitations to the SLDV measurement
		technique, which is used for full wavefield acquisition. The 
		measurements constructed by using SLDV are stationary and 
		time-consuming. 
		Therefore, the proposed technique is more appropriate for NDT than SHM. 
		However, it is probable that in the future, as laser technology 
		progresses, the process of data acquisition will be possible at an 
		array of points instead of a single point, which will considerably 
		decrease the measurement time. It should be added, that the measurement
		time can also be reduced by using fewer points in the spatial grid 
		along with the compressive sensing approach."
	}
	\item Figure 8 needs more explanation, besides, "when a binary threshold is 
	applied, most identifed delaminations vanish as the pixel values are below 
	the threshold." Would an adjustable threshold perform better?
	
	\textcolor{Cyan}{
		\textbf{Response:}
		---------------.
	}
	
	\item Taking Fig. 7 as an example, the meaning of the colors needs to be 
	supplemented.
	
	\textcolor{Cyan}{
		\textbf{Response:}
		---------------.
	}	
	
\end{enumerate}	

\end{document}