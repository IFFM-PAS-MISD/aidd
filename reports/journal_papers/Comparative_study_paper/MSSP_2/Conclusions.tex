%%%%%%%%%%%%%%%%%%%%%%%%%%%%%%%%%%%%%%%%%%%%%%%%%%
\section{Conclusions}
\label{conclusion}
%%%%%%%%%%%%%%%%%%%%%%%%%%%%%%%%%%%%%%%%%%%%%%%%%%
In this paper, we extend our previous work~\cite{Ijjeh2021} regarding delamination identification in composite materials through implementing well-known DL techniques. 
Accordingly, we have trained five different DL models: Res-UNet, VGG16 encoder-decoder, PSPNet, FCN-DenseNet, and GCN for image semantic segmentation.
The models were trained on a numerically generated dataset simulating a full wavefield of propagating guided waves.
DL models show promising results in identifying various types of delaminations regarding their locations, shapes, sizes, and angles. 
Moreover, the models show good generalisation behaviour on predicting the delamination in the unseen numerically generated data.
It has been shown that achieved accuracy of the current models surpasses the accuracy of previous models in~\cite{Ijjeh2021} with an improvement up to \(22.47\%\).
Additionally, the models show their ability to generalise by detecting the delamination in the experimentally acquired data.

We can conclude that GCN model has the highest identification accuracy among all implemented models with respect to numerical data and to the experimental scenario.
Moreover, PSPNet model showed the lowest identification accuracy among all implemented models.
Accordingly, GCN model has the best performance among all implemented DL models.
 
Moreover, the performance of the models can be further improved if the they are trained on experimental data which allows them to learn new complex patterns.
However, the presented studies are limited to only one type of signal at a carrier frequency of 50 kHz. 
We are planning to create a new dataset for the case of a higher excitation frequency. 
Further, we intend to increase the number of experimental scenarios in the future.