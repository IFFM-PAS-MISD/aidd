%%%%%%%%%%%%%%%%%%%%%%%%%%%%%%%%%%%%%%%%%%%%%%%%%%%%%
\subsection{Semantic segmentation models}
\label{section:semantic_segmentation}
%%%%%%%%%%%%%%%%%%%%%%%%%%%%%%%%%%%%%%%%%%%%%%%%%%%%%%%%%%%%%%%%%%%%%%%%%%%%%%%%
In the last few years, DL techniques were rapidly developed in different life applications such as computer vision.  
Image segmentation is a well-known technique employed for computer vision. 
It aims to label each pixel in the input image to its matching class and it is applied in many real-life practical applications such as self-driving cars, medical imaging, traffic control systems, video surveillance, and many others.
In this work, we present a comparative study of five DL models based on Fully Convolutional Networks (FCN)~\cite{Long} to detect and localise delamination in composite plates.
Further, these models aim to perform image semantic segmentation by assigning every pixel of the input image as damaged or not damaged. 
FCN is created by stacking convolutional layers in an encoder-decoder scheme and skipping dense layers. 
The encoder part is responsible for extracting condensed feature maps from the input image at different scale levels by applying cascaded convolutions with strides followed by pooling operations.
The decoder part is responsible for upsampling the condensed feature maps to the same size as the original input image using transposed convolution with strides or upsampling with interpolation.

In this work, the softmax activation function was applied at the output layer for all implemented models in this comparative study.
The softmax calculates the probability for each predicted output of being damaged or undamaged for every single pixel, which implies that the sum of the two probabilities must be one. 
Eq.~(\ref{softmax}) depicts the softmax activation function, where \(P(x)_{i}\) is the probability of each target class \(x_{j}\) over all possible target classes \(x_{j}\), C in our case are two classes  (damaged and undamaged).
To predict the label of the output (\(y_{pred}\)) an \(\argmax\) function is applied to select the maximum probability between both of them.
\begin{equation}
	P(x)_{i} = \frac{e^{x_{i}}}{\sum_{j}^{C} e^{x_{j}}}
	\label{softmax}
\end{equation} 
\begin{equation}
	y_{pred} = \argmax_{i}\left( P(x)_{i} \right)
	\label{argmax}
\end{equation}
Selecting a suitable loss function is an important issue because it measures how well the model learns and performs.
Therefore, in all models, we have applied the categorical cross-entropy (CCE) loss function~\cite{Bonaccorso2020}, which is also called \enquote{softmax loss function}.
CCE is used as the objective function to estimate the difference between the actual damage (ground truth) and the predicted damage.
Further, since we have only two classes to be predicted, it is worth mentioning that a Sigmoid activation function at the output layer can be used with a binary cross-entropy (BCE), with no impacts on the predicted outputs.
Eq.~(\ref{CCE}) illustrates the CCE, where \( P(x)_{i}\) is the softmax value of the target class. 
\begin{equation}	
	CCE = -\log\left( P(x)_{i} \right)
	\label{CCE}
\end{equation}

Additionally, it is also important to select a proper accuracy metric of the model, therefore, we have applied intersection over union (\(IoU\)) (Jaccard index)~\cite{Bertels2019} as our accuracy metric. 
\(IoU\) is estimated by determining the intersection area between the ground truth and the predicted output.
In this work, we have two classes (damaged and undamaged), the \(IoU\) is computed by taking the \(IoU\) for the damaged class only.
The \(IoU\) metric is defined as in Eq.~(\ref{IoU}):
\begin{equation}
IoU = \frac{Intersection}{Union} = \frac{\hat{Y} \cap Y}{\hat{Y} \cup Y} 
\label{IoU}
\end{equation}
where \(\hat{Y}\) represents the predicted vector of damaged and undamaged values, and \(Y\) represents the vector of ground truth values.
The \(IoU\) can be calculated by multiplying the predicted output (matrix of \(zeros\) and \(ones\)) with its ground truth (matrix of \(zeros\) and \(ones\)) to find the intersection, then it is divided over the union which can be calculated by counting all pixels with non-zero values of the predicted output and its ground truth.

Furthermore, Adam optimizer was applied as our optimization method in order to increase the \(IoU\) and to reduce the loss during the training.
Figure~\ref{fig:flowchart} presents a diagram of the implemented DL models for pixel-wise semantic segmentation for delaminations identification. 
The details of the implemented models will be explained in detail in the next sections.
%%%%%%%%%%%%%%%%%%%%%%%%%%%%%%%%%%%%%%%%%%%%%%%%%%%%%%%%%%%%%%%%%%%%%%%%%%%%%%%%
\begin{figure} [h!]
	\begin{center}
		\includegraphics[scale=1.0]{figure3.png}
	\end{center}
	\caption{Schematic diagram of the approach used for comparison of semantic segmentation methods accuracy.} 
	\label{fig:flowchart}
\end{figure}
%%%%%%%%%%%%%%%%%%%%%%%%%%%%%%%%%%%%%%%%%%%%%%%%%%%%%%%%%%%%%%%%%%%%%%%%%%%%%%%%