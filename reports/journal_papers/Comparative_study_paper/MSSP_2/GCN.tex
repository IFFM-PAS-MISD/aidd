\subsubsection{Global Convolutional Network}
Global Convolutional Network (GCN) proposed by Peng et al. 
~\cite{Peng2017} addressed the importance to have large kernels for both 
localisation and classification for semantic segmentation to enlarge 
respective fields.
However, a contradiction arises when performing classification and localisation 
tasks. 
For instance, classification tasks require the models to be invariant for 
different transformations such as rotation and translation.
On the other hand, localisation tasks require the models to be sensitive for 
any transformation, to accurately assign each pixel for its semantic category.
Accordingly, to solve this contradiction, two design principles were suggested: 
\begin{enumerate}
	\item For the classification task, in order to improve the capability of 
	the model to handle different transformations, a large kernel size must be 
	used to enable dense connections between feature maps and per-pixel 
	classifiers.
	\item For localisation task, the model must be fully convolutional. 
	Additionally, fully connected or global pooling layers are not applied as 
	these layers will discard the localisation information. 
\end{enumerate}

Figure~\ref{fig:gcn} presents the proposed GCN module for semantic segmentation 
utilised for delamination identification.
\begin{figure} [h!]
	\begin{center}
		\includegraphics[scale=1.0]{figure8.png}
	\end{center}
	\caption{Global Convolution Network whole architecture.} 
	\label{fig:gcn}
\end{figure}
As shown in the Fig.~\ref{fig:gcn}, a residual network was utilised as a backbone for 
feature maps extraction, the residual block is shown in 
Fig.~\ref{fig:res_gcn_br}a.
After each residual block, a GCN block is inserted  
(Fig.~\ref{fig:res_gcn_br}b), which employs a combination of \((1\times 
k)\)+\((k\times 1)\) and \((k\times 1)\)+\((1\times k)\) convolutions which 
enables a dense connections within a large \((k\times k)\) region in the 
feature map.
In this work, we implemented the model with \(k=7\).
This is followed by a boundary refinement (BR) block shown in Fig.~\ref{fig:res_gcn_br}c, which can be considered as an additional residual block to refine the predictions near the object boundaries ended up generating a lower resolution score map. 
Furthermore, the upsampling operation is done recursively, it upsamples the low 
resolution score maps then concatenate it with a higher one to produce a new 
score maps.
The deconvolution operation is repeated until the original image size is 
obtained.
\begin{figure} [h!]
	\begin{center}
		\includegraphics[scale=1.0]{figure9.png}
	\end{center}
	\caption{(a) Residual block, (b) Global Convolution Network block, (c) 
		Boundary Refinement} 
	\label{fig:res_gcn_br}
\end{figure}


