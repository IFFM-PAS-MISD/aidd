
%% 
%% Copyright 2007, 2008, 2009 Elsevier Ltd
%% 
%% This file is part of the 'Elsarticle Bundle'.
%% ---------------------------------------------
%% 
%% It may be distributed under the conditions of the LaTeX Project Public
%% License, either version 1.2 of this license or (at your option) any
%% later version.  The latest version of this license is in
%%    http://www.latex-project.org/lppl.txt
%% and version 1.2 or later is part of all distributions of LaTeX
%% version 1999/12/01 or later.
%% 
%% The list of all files belonging to the 'Elsarticle Bundle' is
%% given in the file `manifest.txt'.
%% 
%% Template article for Elsevier's document class `elsarticle'
%% with harvard style bibliographic references
%% SP 2008/03/01

\documentclass[preprint,9pt]{elsarticle}


%% Use the option review to obtain double line spacing
%documentclass[authoryear,preprint,review,12pt]{elsarticle}

%% Use the options 1p,twocolumn; 3p; 3p,twocolumn; 5p; or 5p,twocolumn
%% for a journal layout:
%% \documentclass[final,1p,times,authoryear]{elsarticle}
%% \documentclass[final,1p,times,twocolumn,authoryear]{elsarticle}
%% \documentclass[final,3p,times,authoryear]{elsarticle}
%%\documentclass[final,3p,times,twocolumn,authoryear]{elsarticle}
%% \documentclass[final,5p,times,authoryear]{elsarticle}
%% \documentclass[final,5p,times,twocolumn,authoryear]{elsarticle}

%% For including figures, graphicx.sty has been loaded in
%% elsarticle.cls. If you prefer to use the old commands
%% please give \usepackage{epsfig}

%% The amssymb package provides various useful mathematical symbols
\usepackage{amsmath,amssymb,bm}
%\usepackage[dvips,colorlinks=true,citecolor=green]{hyperref}
\usepackage[colorlinks=true,citecolor=green]{hyperref}
%% my added packages
\usepackage{float}
\usepackage{csquotes}
\usepackage{verbatim}
\usepackage{caption}
\usepackage{subcaption}
\usepackage{booktabs} % for nice tables
\usepackage{csvsimple} % for csv read
\usepackage{graphicx}
\usepackage{natbib}

\newcommand{\RNum}[1]{\uppercase\expandafter{\romannumeral #1\relax}}


%\usepackage[outdir=//odroid-sensors/sensors/aidd/reports/journal_papers/MSSP_Paper/Figures/]{epstopdf}
%\usepackage{breqn}
\usepackage{multirow}
%\usepackage{cite}
%\usepackage[style=numeric-comp]{biblatex}
\usepackage[dvipsnames]{xcolor}
\usepackage[document]{ragged2e}

% matrix command 
\newcommand{\matr}[1]{\mathbf{#1}} % bold upright (Elsevier, Springer)
% vector command 
\newcommand{\vect}[1]{\mathbf{#1}} % bold upright (Elsevier, Springer)
\newcommand{\ud}{\mathrm{d}}
\renewcommand{\vec}[1]{\mathbf{#1}}
\newcommand{\veca}[2]{\mathbf{#1}{#2}}
\renewcommand{\bm}[1]{\mathbf{#1}}
\newcommand{\bs}[1]{\boldsymbol{#1}}
% limits underneath
\DeclareMathOperator*{\argmin}{arg\,min}
\DeclareMathOperator*{\argmax}{arg\,max}

\graphicspath{{figures/}}

%
%\graphicspath{ {Graphics/Figures/} }
%% The amsthm package provides extended theorem environments
%% \usepackage{amsthm}
%% The lineno packages adds line numbers. Start line numbering with
%% \begin{linenumbers}, end it with \end{linenumbers}. Or switch it on
%% for the whole article with \linenumbers.
%% \usepackage{lineno}
\journal{Mechanical Systems and Signal Processing}


\begin{document}
	\justifying	
	\noindent
	%%%%%%%%%%%%%%%%%%%%%%%%%%%%%%%%%%%%%%%%%%%%%%%%%%%%%
	\section{Deep learning model}
\subsection{Dataset preprocessing}
\noindent 

The synthetically generated dataset contains 9000 samples of unit cells that are classified into three equal subsets of 3000 samples each.
The shape of a unit cell is \((\textup{512}\times \textup{512})\) pixel points.
As mentioned earlier, the shape of the unit cell is diagonally symmetrical.
To reduce the complexity of the computation, we used only the upper left quarter of the unit cell shape with a size of (256 x 256) pixel points for developing the deep learning model.
Consequently, the total shape of the dataset is (3, 3000, 256, 256, 1).
For training purposes, 2900 samples of unit cells were randomly selected from each subset, resulting in a training set with a shape of (8700, 256, 256, 1).
The labels (ground truths) of the dataset are the dispersion diagrams (frequency versus wavenumber) values with a range exceeding \(500\) kHz.
It is important to mention that we have found that normalizing the frequency values to a range between (0, 1) gives better predicted outputs.


\subsection{Deep learning model}
\noindent
The calculation of the dispersion diagram for a single unit cell using COMSOL software is a time-consuming process.		
Therefore, we developed a surrogate deep learning model that is able to calculate such dispersion diagram in a very fast time.
In general, deep learning techniques such as artificial neural networks (ANNs) and convolution neural networks (CNNs) are considered as a universal approximation functions.

Basically, the developed model is an approximation function, which can map the input (shape of the unit cell) to an output of \(1464\) (frequency versus wavenumber) 
All dispersion diagrams in the synthetically generated dataset have fixed wavenumber values.
Thus, we only need to predict the frequencies.		
Accordingly, our developed deep learning model is a multi-output regressor.	

Figure~\ref{DL_model} presents the general architecture of the developed deep learning model, which is composed of two main parts:		
\begin{enumerate}
	\item The encoder
	\item The dense layers (ANN)
\end{enumerate}
The encoder is a CNN model that performs convolution operations followed by subsampling (pooling). 
Consequentially, the encoder is responsible for extracting the features from the shape of the unit cell.
Such features contain all the information required regarding the dispersion diagram (the predicted output).
The ANN part consists of several layers with different numbers of neurons and is responsible for mapping the extracted features by the encoder into the \(1464\) predicted frequency values.

\begin{figure}
	\centering
	\includegraphics[width=0.9\textwidth]{PC_unit_DL_model.png}
	\caption{DL Model architecture.}
	\label{DL_model}
\end{figure}

The task of hyperparameter optimization is crucial, especially, when developing a deep learning model to solve a complex task.
Hence, the quality of a predictive model depends largely on its hyperparameter configuration.
For this purpose we employed the Hyperband optimisation technique~\cite{Li2018}, which is an extension to the successive halving algorithm~\cite{Jamieson2016}.
The hyperband technique initially runs a random configuration of hyperparameters on a specific schedule of iterations (a small number of iterations) per configuration.
Then it takes the best performers and again runs them through a larger number of iterations.
This process is repeated until we get the best configuration of hyperparameters.

Table~\ref{tab:hyperparemeter_tuning} presents the hyperparameters that were tuned with the hyperband technique, the initial range of values of the hyperparameters, and the optimised value.
Regarding the pooling operation hyperparameter, convolution blocks [1, 2, 4, 5, 6, 7] have average pooling, and convolution block [3] has max pooling.
The output dense layer has 1464 units representing the number of frequencies we want to predict.
Furthermore, the MSE loss function was used as our objective loss function with the Adam optimizer.
Additionally, the early-stopping technique was used to reduce the model overfitting. 
\begin{table}[] 
	\caption{Hyperparameters tuning with hyperband technique.}
	\begin{tabular}{lll}
		\toprule[1.5pt]
		Hyperparameter   & initial range of values & optimised value \\
		\midrule
		Batch size   & {[}16: 64{]}    & 32  \\		
		Kernel size  & {[}3: 7{]}  & 5   \\
		Number of convolution  blocks & {[}3: 8{]}  &  7\\
		Number hidden layers & {[}1:5{]}   & 4\\
		Learning rate    & {[}5e-5 : 5e-4{]}   & 1.433e-4\\
		Pooling operation    & {[}average, max{]}  & Depends on the convolution block \\
		Dropout  & {[}0.15 - 0.3{]}    & 0.2\\
		Dense units 1    & {[}1024-8192{]} & 2048\\
		Dense units 2    & {[}1024-8192{]} & 6656\\
		Dense units 3    & {[}1464-2042{]} & 1944\\
		\bottomrule[1.5pt]  
	\end{tabular}
	\label{tab:hyperparemeter_tuning}
\end{table}



%\clearpage	
%\appendix
\section*{Acknowledgments}


\bibliography{biblography}
\bibliographystyle{unsrt}
\end{document}
