\section{Conclusions}
\label{conclusion}
In this work, we present a novel deep learning-based approach for delamination identification in composite laminates.
The developed approach introduces an end-to-end scheme that performs a many-to-one sequence prediction to identify delamination location, size, and shape.
Accordingly, we trained the models on a consecutive number of frames depicting the full wavefield of Lamb waves propagation in a plate of CFRP, and their interactions with the delamination, and the edges.
Hence, the models learn how to extract the valuable features regarding the damage from such frames in order to have a prediction.

To evaluate the performance of the developed models, we examined them on a numerical test-set that was unseen before.
The results verified their ability to identify the delaminations with high accuracy. 
Furthermore, to evaluate the generalisation capability of the models, we tested them on several experimentally measured cases of single and multiple delaminations of Teflon-insert.
The predicted results are promising, considering the experimental case of multiple delaminations is difficult as the models were trained only on cases of single delamination.
Consequently, the models showed their capability of identifying multiple delaminations at once in real-life cases.

It should be added that by using proposed models much better accuracy was obtained on experimental data in comparison to previously developed FCN-DenseNet models.
Furthermore, the proposed technique is more appropriate for NDT than SHM.
Due to measurements by utilising SLDV are performed stationary and are time-consuming. 
However, it is probable in the future, as laser technology progresses, the process of data acquisition will be achievable in an array of points instead of a single point, which will considerably decrease the measurement process time.


