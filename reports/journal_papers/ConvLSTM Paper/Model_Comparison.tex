\section{Model Comparison:}
\subsection{ConvLSTM Model One:}
%%%%%%%%%%%%%%%%%%%%%%%%%%%%%%%%%%%%%%%%%%%%%%%%%%%%%%%%%%%%%%%%%%%%%%%%%%%%%%%%%%%%%%%%

For developing this model, 64 images in sequential form (containing delamination) among 512 images from each observation were chosen.
Therefore, it makes 64 x 475 = 30,400 total images.
Only 64 images were chosen as input into the ConvLSTM model because it is unfeasible to input all 512 images from each observation to the model.
Further, it was highly computationally expensive to use all 512 images from each observation, and also not each image contains delamination.
The dataset was split into two main portions: 80\% of the data was chosen as a train dataset, and 20\% data were chosen as a test dataset. 
Furthermore, 10\% data from the train dataset was used as a validation dataset.
The distribution of the dataset was performed randomly. 

We developed a 3-layer ConvLSTM model with the filter sizes of 10, 5, and 10 respectively.
The kernel size of the ConvLSTM model in all the layers was used as a 3 x 3, and the 'same' padding was used with a stride of 1, the same padding makes the output the same as the input in the case of stride 1.
Furthermore, max-pooling of filter size 2 with stride size 1, batch normalization~\cite{santurkar2018does} and dropout~\cite{srivastava2014dropout} were also applied at the end of each ConvLSTM layer.  
The tanh (the hyperbolic tangent) activation function is used at each ConvLSTM layer, whereas, sigmoid activation function with binary cross-entropy is applied at the output layer of the network for the final prediction.
The final layer is used a simple convolutional layer.
The threshold value for the sigmoid function was chosen as 0.5, the reason for choosing this value for the sigmoid activation function is explained in our previous research work~\cite{ijjeh2021full}.
It should also be noted that the softmax function along with categorical cross-entropy was also tried at the output layer instead of the sigmoid. 
However, the results were almost the same as that of the sigmoid.
Therefore, the use of softmax and the results obtained through the softmax activation function are not discussed in this research work.
Figure~\ref{fig:model_one} shows the complete architecture of the applied ConvLSTM model.
%%%%%%%%%%%%%%%%%%%%%%%%%%%%%%%%%%%%%%%%%%%%%%%%%%
\begin{figure} [h!]
	\begin{center}
		\includegraphics[width=12cm,height=2.5cm]{Graphics/model-architecture.png}
	\end{center}
	\caption{Graphical representation of the applied ConvLSTM model.} 
	\label{fig:model_one}
\end{figure}
%%%%%%%%%%%%%%%%%%%%%%%%%%%%%%%%%%%%%%%%%%%%%%%%%%

For training the model, the mean IoU values were calculated at the end of each Epoch on the train and validation datasets.
A regularization technique, Early stopping was employed for stopping the training once the value of mean IoU on the validation dataset was not increasing anymore and the model with the best mean IoU value on validation data was saved for further use on the test and experimental data.
The model was saved at Epoch = 196, at which the mean IoU values were noted as 0.91, 0.89 on train and validation datasets respectively.
The model was trained with the use of Adadelta~\cite{zeiler2012adadelta} optimization technique by employing back-propagation through time (BPTT)~\cite{goodfellow2016deep}. 
The learning rate was set to 1.0, and batch size was used as 2.
Two GPU's each of 32 GB were utilized for training this model, and the model took about 13 hours to train.


The model was evaluated on the test data (95 observations) and achieved a mean IoU value of 0.90 on the test data.
The below figures present the prediction of our model in different scenarios from the test data.
Figures~\ref{fig:Figure2_b},~\ref{fig:Figure2_d}, and~\ref{fig:Figure2_f} shows prediction of our model on three different scenarios.
Whereas, Figures~\ref{fig:Figure2_a},~\ref{fig:Figure2_c}, and~\ref{fig:Figure2_e} represents the corresponding ground truths of the predictions.
Figure~\ref{fig:Figure2_b} shows the maximum IoU value 0.98 achieved throughout the whole test dataset, while Figure~\ref{fig:Figure2_d} depicts the minimum IoU value (0.65) which the model obtained from the delamination at the top of the plate.
Furthermore, ~\ref{fig:Figure2_f} elaborates the prediction of the delamination at the bottom of the plate, and the IoU value is 0.84 in this situation.

Figures~\ref{fig:Figure3_b},~\ref{fig:Figure3_d}, and~\ref{fig:Figure3_f} shows the prediction of delamination on three different plates in real world scenarios, where the model predicted the delamination with the IoU values of 0.44, 0.12, and 0.10 respectively.
Whereas, Figures~\ref{fig:Figure3_a},~\ref{fig:Figure3_c}, and~\ref{fig:Figure3_e} represents the corresponding ground truths of the predictions.
As the model was only trained with the simulation-based data, and the simulated data was composed of only one delamination at each plate and had no much noise.
While the observations from experimental data were too noisy and also composed of many delaminations per plate.
Therefore, the performance of the model on the experimental data is not quite well.
However, the model is able to predict and localize the delamination at the experimental data too as shown in Figures~\ref{fig:Figure3_b},~\ref{fig:Figure3_d}, and~\ref{fig:Figure3_f}. 
%%%%%%%%%%%%%%%%%%%%%%%%%%%%%%%%%%%%%%%%%%%%%%%%%%%
%First figure
%%%%%%%%%%%%%%%%%%%%%%%%%%%%%%%%%%%%%%%%%%%%%%%%%%%
\begin{figure} [!h]
	\centering
	\begin{subfigure}[b]{0.49\textwidth}
		\centering
		\includegraphics[scale=.35]{IoU_0.98_ground_truth.png}
		\caption{Ground truth}
		\label{fig:Figure2_a}
	\end{subfigure}
	\hfill
	\begin{subfigure}[b]{0.49\textwidth}
		\centering
		\includegraphics[scale=.35]{IoU_0.98.png}
		\caption{Maximum IoU value = 0.98}
		\label{fig:Figure2_b}
	\end{subfigure}
	\hfill
	\begin{subfigure}[b]{0.49\textwidth}
		\centering
		\includegraphics[scale=.35]{IoU_0.65_ground_truth.png}
		\caption{Ground truth}
		\label{fig:Figure2_c}
	\end{subfigure}
	\hfill
	\begin{subfigure}[b]{0.49\textwidth}
		\centering
		\includegraphics[scale=.35]{IoU_0.65.png}
		\caption{Minimum IoU value = 0.65}
		\label{fig:Figure2_d}
	\end{subfigure}
	\hfill
	\begin{subfigure}[b]{0.49\textwidth}
		\centering
		\includegraphics[scale=.35]{IoU_0.84_ground_truth.png}
		\caption{Ground truth}
		\label{fig:Figure2_e}
	\end{subfigure}
	\hfill	
	\begin{subfigure}[b]{0.49\textwidth}
		\centering
		\includegraphics[scale=.35]{IoU_0.84.png}
		\caption{IoU value = 0.84}
		\label{fig:Figure2_f}
	\end{subfigure}
	\caption{Three different delamination scenarios from test data.}
	\label{fig:Figure2}
\end{figure} 
%%%%%%%%%%%%%%%%%%%%%%%%%%%%%%%%%%%%%%%%%%%%%%%%%%%
%%%%%%%%%%%%%%%%%%%%%%%%%%%%%%%%%%%%%%%%%%%%%%%%%%%
%%%%%%%%%%%%%%%%%%%%%%%%%%%%%%%%%%%%%%%%%%%%%%%%%%%
%Second figure
%%%%%%%%%%%%%%%%%%%%%%%%%%%%%%%%%%%%%%%%%%%%%%%%%%%
\begin{figure} [!h]
	\centering
	\begin{subfigure}[b]{0.49\textwidth}
		\centering
		\includegraphics[scale=.35]{exp_data_CFRP_teflon_3o_375_375p_50kHz_5HC_x12_15Vpp_IoU_0.44_ground_truth.png}
		\caption{Ground truth}
		\label{fig:Figure3_a}
	\end{subfigure}
	\hfill
	\begin{subfigure}[b]{0.49\textwidth}
		\centering
		\includegraphics[scale=.35]{exp_data_CFRP_teflon_3o_375_375p_50kHz_5HC_x12_15Vpp_IoU_0.44.png}
		\caption{IoU value = 0.44}
		\label{fig:Figure3_b}
	\end{subfigure}
	\hfill
	\begin{subfigure}[b]{0.49\textwidth}
		\centering
		\includegraphics[scale=.35]{exp_data_333x333p_50kHz_10HC_18Vpp_x10_pzt_IoU_0.12_ground_truth.png}
		\caption{Ground truth}
		\label{fig:Figure3_c}
	\end{subfigure}
	\hfill
	\begin{subfigure}[b]{0.49\textwidth}
		\centering
		\includegraphics[scale=.35]{exp_data_333x333p_50kHz_10HC_18Vpp_x10_pzt_IoU_0.12.png}
		\caption{IoU value = 0.12}
		\label{fig:Figure3_d}
	\end{subfigure}
	\hfill
	\begin{subfigure}[b]{0.49\textwidth}
		\centering
		\includegraphics[scale=.35]{exp_data_333x333p_50kHz_5HC_18Vpp_x10_pzt_IoU_0.1_ground_truth.png}
		\caption{Ground truth}
		\label{fig:Figure3_e}
	\end{subfigure}
	\hfill	
	\begin{subfigure}[b]{0.49\textwidth}
		\centering
		\includegraphics[scale=.35]{exp_data_333x333p_50kHz_5HC_18Vpp_x10_pzt_IoU_0.1.png}
		\caption{IoU value = 0.1}
		\label{fig:Figure3_f}
	\end{subfigure}
	\caption{Three different delamination scenarios on experimental data.}
	\label{fig:Figure3}
\end{figure} 
%%%%%%%%%%%%%%%%%%%%%%%%%%%%%%%%%%%%%%%%%%%%%%%%%%%
%%%%%%%%%%%%%%%%%%%%%%%%%%%%%%%%%%%%%%%%%%%%%%%%%%%
