\section{Introduction:}
%%%%%%%%%%%%%%%%%%%%%%%%%%%%%%%%%%%%%%%%%%%%%%%%%%%%%%%%%%%%%%%%%%%%%%%%%%%%%%%%%%%%%%%%

Nowadays, composite materials are extensively used for achieving the desired performance in a wide range of industries such as wind turbine, aerospace, marine, automotive, and many more.
This vast usage of composite materials is due to their numerous advantages such as high strength, lightweight, low cost, non-conductivity, higher stiffness-to-mass ratio compared to metals and effective corrosion resistance~\cite{baker2004composite, giurgiutiu2015structural, stoik2010nondestructive, poudel2015comparison}. 
However, these materials are very prone to various kinds of defects such as cracks, fiber breakage, debonding, and delamination~\cite{poudel2015comparison, heslehurst2014defects, talreja2012damage}.
Among these defects, delamination is one of the most hazardous forms of defects in composite materials. 
Delamination depreciates the life of these structures which essentially leads to very catastrophic failures if not detected at an early stage~\cite{talreja2012damage, wisnom2012role}.
Therefore, it is essential to effectively identify the delamination
in composite structures for the purpose of safe and reliable implementation in
various real-world applications.

The detection of delamination in composite materials is very challenging for traditional visual inspection techniques because it mostly occurs between plies of composite laminate and is invisible from external surfaces~\cite{guinard20023d, staszewski2009health, tuo2019damage}.
Therefore, different types of Nondestructive Testing (NDT) and Structural Health Monitoring (SHM) techniques have been proposed for damage detection in composite structures.
Among these various damage detection techniques, ultrasonic guided Lamb waves are widely known as one of the most promising technique for the quantitative identification of delamination in composite structures.
This widespread applications of these waves is due to to their higher sensitivity to small defects, propagation with low attenuation, and potential to monitor large areas with low-voltage and only a small number of sparsely distributed transducers~\cite{alleyne1992interaction, mitra2016guided, giurgiutiu2003lamb, ihn2008pitch}.

However, utilising a smaller number of transducers are not suitable for acquiring high-quality resolution damage maps.
On the other hand, the employment of a very dense array of transducers is also not feasible in most of the situations. 
For alleviating such problems, Scanning Laser Doppler Vibrometer (SLDV) is employed. 
SLDV is capable to measure guided Lamb waves in a highly dense grid of points over the surface of a large specimen.
This collection of signals is known as full wavefield~\cite{radzienski2019damage}. 
From the last few years, full wavefield signals are continually being assessed for the detection and localisation of defects in composite
structures~\cite{radzienski2019damage, girolamo2018impact, kudela2018impact, sohn2011delamination, sohn2011automated, rogge2013characterization}.
Damage detection techniques employing full wavefield signals are capable of effectively estimating the size and location of damage~\cite{girolamo2018impact, kudela2018impact}.
These wavefields provide valuable information regarding the interaction of guided Lamb waves with potential defects.  
However, these full wavefields are very complex. 
Analysing such wavefields are very difficult for conventional physics or classical machine learning-based models. 

Conversely, deep learning which is originated from Artificial Neural Network (ANN), are capable of handling such complex and nonlinear data and has shown very promising results in various domains such as computer vision, object detection, speech recognition, remote sensing, medical sciences
and many more~\cite{deng2014deep, mohanty2016using, zhang2020well, pashaei2020review}.
In recent years, deep learning has shown significant improvements in image segmentation due to the advancement in deep Convolutional Neural Networks (CNN).

Image segmentation is a fundamental component in numerous visual recognition
systems. In the last few years, image segmentation has widely been
employed in autonomous driving~\cite{zhang2013understanding, cordts2016cityscapes, ros2016synthia, li2018real}, medical applications~\cite{taghanaki2021deep}, agriculture sciences~\cite{milioto2018real}, augmented reality~\cite{miksik2015semantic} and many more. 
The purpose of image segmentation is to partition images or video frames into multiple objects or segments~\cite{szeliski2010computer}.
It can be expressed as a pixel-level classification problem with semantic labels, which is known as semantic segmentation, or partitioning the images into individual objects which are called instance segmentation~\cite{szeliski2010computer, minaee2021image}. 
Semantic segmentation functions on pixel-wise labeling with a set of object categories of an image. 
Therefore, it is generally a more difficult task than image classification which only predicts a single label for the entire image~\cite{szeliski2010computer, minaee2021image}.
Furthermore, semantic image segmentation not only depends on the semantics in
the question but also on the problem that needs to be addressed~\cite{ghosh2019understanding}.

Deep learning-based systems intend to derive hierarchical representations
from the input data via constructing deep neural networks by multiple layers
of non-linear transformations.
In deep learning architectures, the output of one layer acts as the input to the other subsequent layer.
The application of one layer in deep learning acquires a new representation of the input data and then, the stacking composition of many layers enables the model to learn complex patterns from the simple notions that can be formed from raw input.
Therefore, these systems do not need extensive human labour and knowledge for hand-crafted feature design~\cite{zhao2019deep, yuan2020machine}.

Deep learning techniques have extensively been utilised for the inspection and
maintenance of civil infrastructures~\cite{cha2017deep, lin2017structural, liu2019computer} and rotating machinery~\cite{janssens2016convolutional, jia2016deep} and has shown very promising results.
Further, different researchers have used basic ANN and deep learning architectures for damage detection in composite structures by employing vibration, thermography and frequency-based approaches~\cite{islam1994damage, okafor1996delamination, chakraborty2005artificial, khan2019structural, luo2019temporal, bang2020defect}. 
However, only few researchers have applied ANN and deep learning-based approaches for damage detection in composite structures by utilising guided Lamb waves~\cite{su2004lamb, chetwynd2008damage, de2015application, tabian2019convolutional, feng2019locating, mardanshahi2020detection}.


Tabian et al.~\cite{tabian2019convolutional} developed a CNN-based approach for the detection and characterization of impacts in complex composite structures.
They first transferred the acquired guided wave-based data to 2D images and then they applied two different CNNs on those 2D images.
One CNN in their approach was applied for impact localisation and a separate CNN was implemented for impact categorization (energy level).
Mardanshahi et al.~\cite{mardanshahi2020detection} applied support vector machines (SVM), linear vector quantization (LVQ), and multilayer perceptron (MLP) neural networks for the classification of crack density in composite structures by using guided Lamb wave-based data.
They reported that the SVM outperformed the other basic neural networks in
their research work.



%%%%%%%%%%%%%%%%%%%%%%%%%%%%%%%%%%%%%%%%%%%%%%%%%%%
%%%%%%%%%%%%%%%%%%%%%%%%%%%%%%%%%%%%%%%%%%%%%%%%%%%
