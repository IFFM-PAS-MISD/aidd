%% Copyright 2007, 2008, 2009 Elsevier Ltd
%% This file is part of the 'Elsarticle Bundle'.
%% ---------------------------------------------
%% 
%% It may be distributed under the conditions of the LaTeX Project Public
%% License, either version 1.2 of this license or (at your option) any
%% later version.  The latest version of this license is in
%%    http://www.latex-project.org/lppl.txt
%% and version 1.2 or later is part of all distributions of LaTeX
%% version 1999/12/01 or later.
%% 
%% The list of all files belonging to the 'Elsarticle Bundle' is
%% given in the file `manifest.txt'.
%% 
%% Template article for Elsevier's document class `elsarticle'
%% with harvard style bibliographic references
%% SP 2008/03/01

\documentclass[preprint,9pt]{elsarticle}


%% Use the option review to obtain double line spacing
%documentclass[authoryear,preprint,review,12pt]{elsarticle}

%% Use the options 1p,twocolumn; 3p; 3p,twocolumn; 5p; or 5p,twocolumn
%% for a journal layout:
%% \documentclass[final,1p,times,authoryear]{elsarticle}
%% \documentclass[final,1p,times,twocolumn,authoryear]{elsarticle}
%% \documentclass[final,3p,times,authoryear]{elsarticle}
%%\documentclass[final,3p,times,twocolumn,authoryear]{elsarticle}
%% \documentclass[final,5p,times,authoryear]{elsarticle}
%% \documentclass[final,5p,times,twocolumn,authoryear]{elsarticle}

%% For including figures, graphicx.sty has been loaded in
%% elsarticle.cls. If you prefer to use the old commands
%% please give \usepackage{epsfig}

%% The amssymb package provides various useful mathematical symbols
\usepackage{amsmath,amssymb,bm}
%\usepackage[dvips,colorlinks=true,citecolor=green]{hyperref}
\usepackage[breaklinks,colorlinks=true,citecolor=green]{hyperref}
%% my added packages
\usepackage{float}
\usepackage{csquotes}
\usepackage{verbatim}
\usepackage{caption}
\usepackage{subcaption}
\usepackage{booktabs} % for nice tables
\usepackage{csvsimple} % for csv read
\usepackage{graphicx}
\usepackage{natbib}

\newcommand{\RNum}[1]{\uppercase\expandafter{\romannumeral #1\relax}}


%\usepackage[outdir=//odroid-sensors/sensors/aidd/reports/journal_papers/MSSP_Paper/Figures/]{epstopdf}
%\usepackage{breqn}
\usepackage{multirow}
%\usepackage{cite}
%\usepackage[style=numeric-comp]{biblatex}

% matrix command 
\newcommand{\matr}[1]{\mathbf{#1}} % bold upright (Elsevier, Springer)
% vector command 
\newcommand{\vect}[1]{\mathbf{#1}} % bold upright (Elsevier, Springer)
\newcommand{\ud}{\mathrm{d}}
\renewcommand{\vec}[1]{\mathbf{#1}}
\newcommand{\veca}[2]{\mathbf{#1}{#2}}
\renewcommand{\bm}[1]{\mathbf{#1}}
\newcommand{\bs}[1]{\boldsymbol{#1}}
% limits underneath
\DeclareMathOperator*{\argmin}{arg\,min}
\DeclareMathOperator*{\argmax}{arg\,max}

\graphicspath{{Graphics/}{//Graphics/}{/aidd/reports/journal_papers/ConvLSTM 
Paper/figs}}

%\graphicspath{ {Graphics/Figures/} }
%% The amsthm package provides extended theorem environments
%% \usepackage{amsthm}
%% The lineno packages adds line numbers. Start line numbering with
%% \begin{linenumbers}, end it with \end{linenumbers}. Or switch it on
%% for the whole article with \linenumbers.
%% \usepackage{lineno}
\journal{Composites Part B: Engineering}

\begin{document}
	\begin{frontmatter}
		\addcontentsline{toc}{section}{References}
		%% Title, authors and addresses
		%% use the tnoteref command within \title for footnotes;
		%% use the tnotetext command for theassociated footnote;
		%% use the fnref command within \author or \address for footnotes;
		%% use the fntext command for theassociated footnote;
		%% use the corref command within \author for corresponding author footnotes;
		%% use the cortext command for theassociated footnote;
		%% use the ead command for the email address,
		%% and the form \ead[url] for the home page:
		%% \title{Title\tnoteref{label1}}
		%% \tnotetext[label1]{}
		%% \author{Name\corref{cor1}\fnref{label2}}
		%% \ead{email address}
		%% \ead[url]{home page}
		%% \fntext[label2]{}
		%% \cortext[cor1]{}
		%% \address{Address\fnref{label3}}
		%% \fntext[label3]{}
		\title{Deep learning approach for delamination identification using animation of Lamb waves}

		%% use optional labels to link authors explicitly to addresses:
		%% \author[label1,label2]{}
		\address[IFFM]{Institute of Fluid Flow Machinery, Polish Academy of Sciences, Poland}
		\author{Saeed Ullah\fnref{IFFM}}
		\author{Abdalraheem A. Ijjeh\fnref{IFFM}}
		\author{Pawel Kudela\corref{cor1}\fnref{IFFM}}
		\ead{pk@imp.gda.pl}
		%\ead{pfiborek@imp.gda.pl}
		%\author{Tomasz Wandowski \fnref{IFFM}}	
		\cortext[cor1]{Corresponding author}

\begin{abstract}
	\begin{sloppypar}
		%%%%%%%%%%%%%%%%%%%%%%%%%%%%%%%%%%%%%%%%%%%%%%%%%%%%%%%%%%%%%%%%%%%%%%%%%%%%%%%%
In this work, a novel approach of guided wave-based damage identification in composite laminates is proposed. 
The novelty of this research lies in the implementation of Convolutional Long Short Term Memory (ConvLSTM)-based autoencoders for the generation of full wavefield data of propagating guided waves in composite structures.
The developed surrogate deep learning model takes as input full wavefield frames of propagating waves in a healthy plate along with a binary image representing delamination and predicts frames of propagating waves in a plate which contains single delamination.
The evaluation of the surrogate model is ultrafast.
Therefore, unlike traditional forward solvers, the surrogate model can be employed efficiently in the inverse framework of damage identification.
In this work, particle swarm optimisation is applied as a suitable tool to this 
end. 

The proposed method was tested on a synthetic dataset showing that it is capable to estimate delamination location and size with a good accuracy.
The test involved full wavefield data in the objective function of the inverse 
method but it should be underlined that also partial data with measurements can 
be implemented.
This is extremely important for practical applications in structural health 
monitoring where only signals at a finite number of locations are available.
	\end{sloppypar}	
\end{abstract}

\begin{keyword}
	%% keywords here, in the form: keyword \sep keyword
	Lamb waves \sep structural health monitoring \sep semantic segmentation\sep delamination identification \sep deep learning \sep  autoencoder \sep ConvLSTM 
	%% PACS codes here, in the form: \PACS code \sep code
	
	%% MSC codes here, in the form: \MSC code \sep code
	%% or \MSC[2008] code \sep code (2000 is the default)
	
\end{keyword}
	\end{frontmatter}
	%%%%%%%%%%%%%%%%%%%%%%%%%%%%%%%%%%%%%%%%%%%%%%%%%%%%%
	%%%%%%%%%%%%%%%%%%%%%%%%%%%%%%%%%%%%%%%%%%%%%%%%%%%%%
\section{Introduction}
%%%%%%%%%%%%%%%%%%%%%%%%%%%%%%%%%%%%%%%%%%%%%%%%%%
Composite materials are very prone to various kinds of defects such as cracks, fibre breakage, debonding, and delamination~\cite{ip2004delamination, smith2009composite}. Among these defects, delamination is one of the most hazardous forms of the defects, which essentially leads to very catastrophic failures if not detected at early stages~\cite{valdes1999delamination}. 
Therefore, it is essential to effectively identify the delamination in composite structures for the purpose of safe and reliable implementation in various real-world applications. 
Accordingly, different types of Structural Health Monitoring (SHM) techniques have been developed for delamination detection in composite structures. 
Recently, guided Lamb waves based SHM gained high popularity for damage detection in composite structures due to their higher sensitivity to small defects, propagation with low attenuation, and potential to monitor large areas with low-voltage and only a small number of sparsely distributed transducers~\cite{alleyne1992interaction, giurgiutiu2003lamb, ihn2008pitch, mitra2016guided}. 
However, utilising a smaller number of transducers are not suitable for acquiring high-quality resolution damage maps. Whereas, the employment of a very dense array of transducers is also not feasible in most of the situations. 
For alleviating such problem Scanning Laser Doppler Vibrometer (SLDV) is employed. SLDV is capable to measure guided Lamb waves in a very dense grid of points over the surface of a large specimen. 
This collection of signals is known as full wavefield~\cite{radzienski2019damage}. 
Damage detection techniques employing full wavefield signals are capable of effectively estimating the size and location of damage~\cite{girolamo2018impact, kudela2018impact}. From the last few years, full wavefield signals are continually being assessed for the detection and localisation of defects in composite structures~\cite{sohn2011delamination, sohn2011automated, rogge2013characterization, kudela2018impact, radzienski2019damage}.

Currently, full wavefield signals based damage detection techniques are employing various physics and classical machine learning-based methods. 
These structural damage detection approaches are composed of two processes: feature extraction and feature classification. 
The feature extraction process usually needs a great deal of human labor and computational effort which prevents these techniques of being applicable in real-time SHM utilisation. 
Further, such systems also needs a notable amount of expertise from the practitioner, which is very difficult to be always available. Moreover, in many situations, the extracted handcrafted features by these techniques may fail to precisely characterise the acquired signal which leads to poor classification performance~\cite{zhao2019deep, yuan2020machine}. Additionally, these systems are also not suitable for modeling large-scale data.

Recently, deep learning which is originated from Artificial Neural Network (ANN) has shown very promising results in various domains such as computer vision, object detection, speech recognition, remote sensing, medical sciences and many more~\cite{deng2014deep, mohanty2016using, zhang2020well, pashaei2020review}. 
In recent years, deep learning has shown significant improvements in image segmentation due to the advancement in deep Convolutional Neural Networks (CNN). 

Image segmentation is a fundamental component in numerous visual recognition systems. In the last few years, image segmentation has widely been employed in autonomous driving~\cite{zhang2013understanding, cordts2016cityscapes, ros2016synthia, li2018real}, medical applications~\cite{taghanaki2020deep}, agriculture sciences~\cite{milioto2018real}, augmented reality~\cite{miksik2015semantic} and many more. 
Image segmentation techniques partition images or video frames into multiple objects or segments~\cite{szeliski2010computer}. 
It can be expressed as a pixel-level classification problem with semantic labels, which is known as semantic segmentation or can be partitioning the images into individual objects which are called instance segmentation~\cite{minaee2020image}. 
Semantic segmentation functions on pixel-wise labeling with a set of object categories of an image. 
Therefore, it is generally a more difficult task than image classification, which only predicts a single label for the entire image~\cite{minaee2020image}. 
Furthermore, semantic image segmentation not only depends on the semantics in the question but also on the problem that needs to be addressed~\cite{ghosh2019understanding}.

Deep learning-based systems intend to derive hierarchical representations from the input data via constructing deep neural networks by multiple layers of non-linear transformations. 
In deep learning architectures, the output of one layer act as the input to the other subsequent layer. 
The application of one layer in deep learning acquires a new representation of the input data and then, the stacking composition of many layers enables the model to learn complex patterns from the simple notions that can be formed from raw input. 
Therefore, these systems do not need extensive human labor and knowledge for hand-crafted feature design~\cite{zhao2019deep, yuan2020machine}.

Deep learning techniques have widely been utilised for the inspection and maintenance of civil infrastructures and has shown very promising results ~\cite{cha2017deep, lin2017structural, liu2019computer}. 
However, deep learning is still less explored for the purpose of delamination detection in composite materials.   

A few researchers have applied various deep learning techniques for damage detection with guided Lamb waves in composite structures. 
Fenza et al.~\cite{de2015application} presented the utilisation of ANN and probability ellipse techniques for the detection, location, and degree of defects in aluminum and fabric composite plates with the use of Lamb waves. 
Both the ANN and probability ellipse techniques were based on the damage index assessed by examining the variations in the Lamb waves acquired before and after the damage in each analysed path. 
The results from both methods proved that guided Lamb waves have prominent advantages in localisation and the detection of different kinds of defects in plate-like structures. 
Feng et al.~\cite{feng2019locating} proposed two time of flight (ToF) based algorithms of scattered guided Lamb waves in carbon fiber reinforced polymer (CFRP) plates. 
Their first algorithm is a probabilistic approach that constructs a probability matrix. The probability matrix is used for the localisation of delamination while the second algorithm is based on ANN which is then employed for improving the accuracy of defect localisation. 
The neural network receives the input from the ToF of scattered waves acquired from three sensor pairs.
Chetwynd et al.~\cite{chetwynd2008damage} used MLP neural network for the classification and regression tasks of damage detection in a stiffened curved CFRP investigated using Lamb waves with the use of eight surface bonded piezoelectric transducers. 
Many localised defects were fabricated through a force applicator, and Lamb wave responses were received for the damaged and healthy cases. 
For each case, the Lamb wave response was then transformed into a scalar novelty index with the help of outlier analysis. 
These novelty indices of 28 sensor paths were then provided as input to the MLP classification and regression architectures. 
For the classification of damaged and undamaged regions of the panel, the MLP classificier was employed, whereas the MLP regressor was used for evaluating the accurate location of damage on the panel. 
They achieved quite better results with both the classifier and regerssor. 
The classification accuracy of their MLP based classifier was 88.1\% on the test data while the Mean Square Error (MSE) value of the regerssor was 3.1\% on the unseen data. Su and Ye~\cite{su2004lamb} presented a Lamb wave based delamination identification technique in composite structures with the use of wavelet transform and multi-layer feedforward ANN architecture. 
The ANN was employed with the error-backpropagation (BP) algorithm. 
They also developed an Intelligent Signal Processing and Pattern Recognition (ISPPR) package for the extraction and digitision of spectrographic characterisitics of simulated Lamb waves in the time-frequency domain, which is known as Digital Damage Fingerprints (DOF) and is used for constructing a Damage Parameters Database (DPD). 
The DPD is then employed offline for training the neural network. 
They validated their approach with identifying actual delamination in different composites and also proved that their system has achieved excellent quantitative diagnosis results for different damage parameters such as the presence, location, orientation and geometry of defects. 
Hussain et al.~\cite{hussaintemporal} proposed a Temporal Convolutional Network (TCN) based transfer learning system for delamination prediction in CFRP cross-ply laminates. They employed a CFRP dataset from NASA which is composed of signals from Lamb waves sensors and X-ray images of specimens for capturing the propagation of defects in carbon fiber composite under fatigue loading. 
The TCN model was trained with various combinations of lengths of the sensors signals and different frequencies at which Lamb wave signals were sensed. 
They demonstrated that their approach needs very little time for the training and can also predict the delamination on a new composite coupon by utilising only a few samples of the test coupon. 
Melville et al.~\cite{melville2018structural} applied SVM and deep learning techniques for damage detection on full wavefield signals of ultrasonic guided wave images. The wavefield data was acquired via a laser Doppler vibrometer and piezoelectric actuators on thin metal plates. 
They showed that the deep learning methods achieved quite better damage prediction results as compared to the SVM based methods. 
Esfandabadi et al.~\cite{keshmiri2019deep} investigated the applications of super-resolution techniques to acquire high-resolution wavefields via the training of neural networks on different aluminum and CFRP plates. 
They applied two variants of CNN architecture: Super-Resolution Convolutional Neural Networks (SRCNNs) and Very-Deep Super Resolution (VDSR) with compressive sensing for the recovery of high spatial frequency information from low-resolution wavefield images. 
A dataset of 652 wavefield images (326 with defects and 326 without defects) were constructed, acquired with laser Doppler vibrometer of guided ultrasonic waves propagation.
Additionally, 273 images of wavefield were employed as a testing database for the validation purpose of the proposed methodology.           

Readers are advised to refer to our previous work entitled (Full Wavefield Processing by Using FCN for Delamination Detection) (under review) since for our knowledge it was the first work of using full wavefield images in delamination detection in composite materials using deep learning techniques. 
In this work, we have implemented four different deep learning based semantic segmentation models for delamination detection in composite materials.
The models were validated on numerical and experimental data in order to show their ability to generalise.
The models were compared based on their Intersection over Union (IoU) and the total number of parameters.

The paper is organised as follows, the acquisition and preprocessing of the required data are presented in section~\ref{section:Data_acquisition_and_preprocessing}.
In section~\ref{section:semantic_segmentation} the semantic segmentation models used for delamination detection were illustrated in details. 
Next, the detailed comparison of these models were elaborated in section~\ref{section:results_and_discussions}.
Finally, the conclusion and future work is presented in section~\ref{conclusion}.
	%%%%%%%%%%%%%%%%%%%%%%%%%%%%%%%%%%%%%%%%%%%%%%%%%%%%%
	\section{Methodology}
\begin{sloppypar}
	Two deep learning models were developed and trained on dataset in the form of animations of Lamb waves calculated numerically. Than the models were evaluated on unseen numerical and experimental animations of Lamb waves to assess their accuracy for delamination identification.
	
	The synthetic dataset is used for training the proposed neural network architectures with the aim of delamination identification directly from SLDV measurements without the need for a baseline wavefield.
	\subsection{Dataset}
	It is infeasible to gather a large dataset which includes interactions of guided waves with various defects by using the SLDV on real structures. 
	Therefore, in this work, a synthetic dataset of propagating waves in carbon fibre reinforced composite plates was computed by using the parallel implementation of the time domain spectral element method~\cite{Kudela2020}. 
	Essentially, the dataset resembles the particle velocity measurements at the bottom surface of the plate acquired by the SLDV in the transverse direction as a response to the piezoelectric (PZT) excitation at the centre of the plate. 
	The input signal was a five-cycle Hann window modulated sinusoidal tone burst. The carrier frequency was assumed to be 50 kHz. 
	The total wave propagation time was set to 0.75 ms so that the guided wave could propagate to the plate edges and back to the actuator twice.
	The number of time integration steps was 150000, which was selected for the stability of the central difference scheme.
	
	The material was a typical cross-ply CFRP laminate. 
	The stacking sequence [0/90]\(_4\) was used in the model. 
	The properties of a single ply were as follows [GPa]:
	\(C_{11} = 52.55, \, C_{12} = 6.51, \, C_{22} = 51.83, C_{44} = 2.93, C_{55} = 2.92, C_{66} = 3.81\). 
	The assumed mass density was 1522.4 kg/m\textsuperscript{3}.
	These properties were selected so that wave front patterns and wavelengths simulated numerically are similar to the wavefields measured by the SLDV on CFRP specimens used later on for testing the developed methods for delamination identification.
	The shortest wavelength of the propagating A0 Lamb wave mode was 21.2 mm for numerical simulations and 19.5 mm for experimental measurements, respectively.
	
	Similar to our previous work~\cite{Ijjeh2021, Ijjeh2022}, 475 cases were simulated, representing Lamb wave propagation and interaction with single delamination for each case. 
	The following random factors were used in simulated delamination scenarios:
	\begin{itemize}
		\item delamination geometrical size	\(2b\) and \(2a\), namely ellipse minor and major axis randomly selected from the interval \(\left[10 \, \textrm{mm}, 40\, \textrm{mm}\right]\),
		\item delamination angle \(\alpha\) randomly selected from the interval \( \left[ 0^{\circ}, 180^{\circ} \right]\),
		\item coordinates of the centre of delamination \((x_c,y_c)\) randomly selected from the interval \(\left[0\, \textrm{mm}, 250\, \textrm{mm} -\delta \right]\) and \( \left[250\, \textrm{mm}+\delta, 500\, \textrm{mm} \right] \), where \(\delta = 10\, \textrm{mm}\)).
	\end{itemize}
	These parameters are defined in Fig.~\ref{fig:random_delaminations} which illustrates exemplary possible locations, sizes, and shapes of random delaminations used for Lamb wave propagation modeling.
	It should be noted that the numerical cases include delaminations located at the edge and corners of the plate.
	\begin{figure}[!ht]
		\centering
		\includegraphics[scale=0.8]{figure1.png}
		\caption{Exemplary locations, sizes and shapes of random delaminations used for Lamb wave propagation modeling.}
		\label{fig:random_delaminations}
	\end{figure}
	
	It should be underlined that the previous dataset~\cite{Kudela2020d} contains the RMS of the full wavefield, representing wave energy spatial distribution in the form of images for each delamination case.
	However, the currently utilised dataset contains frames of the full wavefield of propagating waves (512 frames for each delamination scenario).
	The new dataset is available online at~\cite{kudela_pawel_2021_5414555}.
	
	As mentioned earlier, the dataset contains 475 different cases of delaminations, with 512 frames per case, producing a total number of 243,\,200 frames with a frame size of \((500\times500)\)~pixels representing the geometry of the specimen of size \((500\times500)\)~mm\(^{2}\).
	Thus, using all frames in each case has high computational and memory costs.
	Frames displaying the propagation of guided waves before interaction with the delamination have no features to be extracted (see Fig.~\ref{fig:Full_wave}).
	Hence, for training, only a certain number of frames were selected starting from the initial occurrence of the interactions with the delamination.
	
	Figure~\ref{fig:Full_wave} shows selected frames at different time-steps of the propagating Lamb waves before and after the interaction with the damage.
	The number of frames utilised to train the developed models was reasonably selected to prevent GPU memory overflow during training. 
	Frame \(f_{1}\) represents the initial interactions with the delamination, which was calculated using the delamination location and the velocity of the \(A0\) Lamb wave mode.
	While frame \(f_{m}\) represents the last frame in the training sequence window, accordingly, \(m=64\) for Model-\RNum{1}, and \(m=24\) for Model-\RNum{2} which will be discussed in the next subsection.
	\begin{figure}[!ht]
		\centering
		\includegraphics[width=1\textwidth]{figure2.png}
		\caption{Sample frames of full wave propagation.}
		\label{fig:Full_wave}
	\end{figure}
	
	Furthermore, the dataset was divided into two sets: training and testing, with a ratio of \(80\%\) and \(20\% \) respectively.
	Moreover, a certain portion of the training set was preserved as a validation set to validate the model during the training process.
	Additionally, the dataset was normalised to a range of \((0, 1)\) to improve the convergence of the gradient descent algorithm.
	
	Additionally, for the training set for Model-\RNum{2}, we have upsampled the frames (by using cubic interpolation) to \(512\times512\)~pixels to maintain the symmetrical shape during the encoding and decoding process.
	Further, the validation sets have portions of \(10\%\) and \(20\%\) regarding the training sets for Model-\RNum{1} and Model-\RNum{2}, respectively.
	%%%%%%%%%%%%%%%%%%%%%%%%%%%%%%%%%%%%%%%%%%%%%%%%%%%%%%%%%%%%%%%%%%%%%%%%%%%%
	
	Figure~\ref{fig:Diagram_exp_predictions} illustrates the complete procedure of obtaining intermediate predictions for the testing cases and finally calculating the RMS image, where \(f_{1}\) refers to the starting frame and \(f_{n}\) is the last frame, (\(n=512\)) in our dataset.
	Further, \(m\) refers to the number of frames in the window, hence, \(m=64\) frames for Model-\RNum{1} and \(m=24\) frames for Model~\RNum{2}, and \(k\) represents the total number of windows.
	Accordingly, we slide the window over all input frames.
	The shift of the window is one frame at a time.
	Deep learning model predictions \(\hat{Y_k}\) are obtained for each window and combined to final damage map by using the RMS:
	
	\begin{equation}
		RMS = \sqrt{\frac{1}{N}\sum_{k=1}^{N}\hat{Y_k}^2}.	
		\label{RMS}
	\end{equation}
	%%%%%%%%%%%%%%%%%%%%%%%%%%%%%%%%%%%%%%%%%%%%%%%%%%%%%%%%%%%%%%%%%%%%%%%%%%%%%%%%
	\begin{figure}[!ht]
		\centering
		\includegraphics[width=1\textwidth]{figure3.png}
		\caption{The procedure of calculating the RMS prediction image (damage map).}
		\label{fig:Diagram_exp_predictions}
	\end{figure}
	%%%%%%%%%%%%%%%%%%%%%%%%%%%%%%%%%%%%%%%%%%%%%%%%%%%%%%%%%%%%%%%%%%%%%%%%%%%%
\end{sloppypar}
\newpage
	%%%%%%%%%%%%%%%%%%%%%%%%%%%%%%%%%%%%%%%%%%%%%%%%%%%%%
	\subsection{Introduction to RNNs, LSTM, and ConvLSTM}
%%%%%%%%%%%%%%%%%%%%%%%%%%%%%%%%%%%%%%%%%%%%%%%%%%%%%%%%%%%%%%%%%%%%%%%%%%%%%%%%
Feed-forward neural networks such as traditional ANNs and CNNs cannot learn temporal features from the data and hence are not the best choice for sequential data processing.
To handle such problems, a recurrent neural network (RNN) was introduced, which was specifically developed to address sequential data~\cite{aggarwal2018neural, Lecun2015, goodfellow2016deep}. 
RNNs contain loops among the different nodes in their architecture to retain information in the model for long periods.
RNNs employ the current input with the previous memory state.
This ability to memory-keep enables RNNs to predict what comes next. 
Furthermore, RNNs were designed to handle sequential data, which implies that updating the learnable weights must consider the extent of the time dimension. 
Accordingly, the backpropagation algorithm~\cite{Rumelhart1986} responsible for updating the learnable weights needs some modification to work along with the time dimension.
To alleviate this problem, backpropagation through time (BPTT)~\cite{aggarwal2018neural, goodfellow2016deep} was introduced. 
Therefore, in basic RNNs, short-term memories are only preserved, so it becomes unfeasible in the case of dealing with long sequences of data. Hence, such RNNs may suffer from issues like vanishing or exploding gradients~\cite{bengio1994learning}.
Therefore, the fine-tuning of the model parameters and training of RNNs becomes very hard.
To overcome such issues, Hochreiter and Schmidhuber developed the Long-Short Term Memory networks (LSTMs)~\cite{Hochreiter1997}.
%%%%%%%%%%%%%%%%%%%%%%%%%%%%%%%%%%%%%%%%%%%%%%%%%%%
\begin{figure} [!h]
	\centering
	\begin{subfigure}[b]{1\textwidth}
		\centering
		\includegraphics[scale=1]{figure4a.png}
		\caption{LSTM}
		\label{fig:LSTM}
	\end{subfigure}
	\\ 
	\hfill
	\begin{subfigure}[b]{1\textwidth}
		\centering
		\includegraphics[scale=1]{figure4b.png}
		\caption{ConvLSTM}
		\label{fig:ConvLSTM}	
	\end{subfigure}
	\caption{LSTM and ConvLSTM architectures.}
	\label{fig:lstm_convlstm}
\end{figure}
%%%%%%%%%%%%%%%%%%%%%%%%%%%%%%%%%%%%%%%%%%%%%%%%%%%

LSTMs were developed to keep information related to long-term dependencies and to solve the problem of vanishing/exploding gradients.
Further, LSTMs handle inputs or outputs of any length, which makes LSTMs powerful for solving very complex sequential problems. 
Basic LSTM architecture shown in Fig.~\ref{fig:LSTM}  consists of four units: an input gate, a cell state, a forget gate, and an output gate.
These gates help regulate the flow of information that is added to or removed from the cell state. 
The hidden states in LSTM hold the short-term memory, while the cells state holds the long-term memory.

The purpose of the forget gate is to decide what information to keep and what to neglect. 
The current input \(x_{t}\) and the previous hidden state  \(h_{t-1}\) are passed through a sigmoid function (\(\sigma\)) which will produce values between \(0\) and \(1\).
Then the outputs of the sigmoid are multiplied with the previous cell state \(c_t-1\). 
Consequently, (\(0\)) outputs are discarded.
The mathematical calculation at the forget gate ($f_t$) is depicted in Eq.~(\ref{eq:eq1}):

\begin{align}
	&f_{t}=\sigma\left( W_{f}  
	\left[
	\begin{array}{c}
		h_{t-1} \\ x_{t}
	\end{array} 
	\right]
	+ b_{f} \right), \\
	&W_{f} = \left[ W_{h_{t-1}}  W_{x_{t}} \right],
	\label{eq:eq1}
\end{align}
where \(W_{f}\) represent the learnable weights at the hidden and input states, \(h_{t-1}\) and \(x_{t}\), respectively, and \(b_{f}\) represents the bias term. 

The input gate \(i_{t}\) takes the current input \(x_t\) with the previous hidden state \(h_{t-1}\), then applies the sigmoid function to get values in a range between 0 (not important) and 1 (important).
Then, the same current input \(x_t\), and the hidden state \(h_{t-1}\) are passed through a \(\tanh\) function at a candidate cell state (\(\tilde{c}_{t}\)) that will regulate the network by transferring the values into a range between \(-1\) and \(1\).
Then, the outputs from the sigmoid and \(\tanh\) functions are multiplied point-by-point to eliminate \(0\) values.  
Equation~(\ref{eq:eq2}) depicts the calculation at the input gate:
\begin{equation}
	\begin{aligned}
		i_{t} &=\sigma\left(W_{i} 
		\left[
		\begin{array}{c}
			h_{t-1} \\ x_{t}
		\end{array} 
		\right]+b_{i}\right), 
		\\
		\tilde{c}_{t} &=\tanh \left(W_{c} 
		\left[
		\begin{array}{c}
			h_{t-1} \\ x_{t}
		\end{array} 
		\right]+b_{c}\right). 
	\end{aligned} \label{eq:eq2}
\end{equation}
At this point, the network has sufficient information obtained from the input and forget gates. 
Hence, the current cell state \(c_{t}\) can be calculated by multiplying the previous cell state \(c_{t-1}\) with the output of the forget gate. 
Then, the result is added to the calculated input values as depicted in Eq.~(\ref{eq:eq3}):
\begin{equation}
	c_{t}=f_{t} \cdot c_{t-1}+i_{t} \cdot \tilde{c}_{t}.
	\label{eq:eq3}
\end{equation}
The output gate \(o_{t}\) computes the next hidden state \(h_{t}\) which
holds information related to the current inputs. 
Accordingly, the current input \(x_{t}\) and the previous hidden state \(h_{t-1}\) are passed through a third sigmoid function to produce values between \(0\) and \(1\).
The current cell state \(c_{t}\) is passed through a \(\tanh\) function and multiplied point-by-point with \(o_{t}\) to produce the new hidden state \(h_{t}\) which is transferred to the next timestamp.
Equation~(\ref{eq:eq4}) illustrates the calculations at the output gate:
%%%%%%%%%%%%%%%%%%%%%%%%%%%%%%%%%%%%%%%%%%%%%%%%%%%%%%%%%%%%%%%%%%%%%%%%%%%%%%%%
\begin{equation}
	\begin{aligned}
		o_{t} &=\sigma\left(W_{o} 
		\left[
		\begin{array}{c}
			h_{t-1} \\ x_{t}
		\end{array} 
		\right]
		+b_{o}\right), \\
		h_{t} &=o_{t} \cdot \tanh \left(c_{t}\right),
	\end{aligned}
	\label{eq:eq4}
\end{equation} 
%%%%%%%%%%%%%%%%%%%%%%%%%%%%%%%%%%%%%%%%%%%%%%%%%%%%%%%%%%%%%%%%%%%%%%%%%%%%%%%%
where \(W_{f}, W_{i}, W_{c}\) and \(W_{o}\) have shared learnable weights across time.

Recently, LSTMs have been widely used for large-scale learning of language translation models, speech recognition systems, chatbots, forecasting stock markets, text data analysis, and many more~\cite{graves2014towards, cho2014properties}. 
However, LSTMs are inefficient at capturing spatial information by themselves when the time series inputs are consecutive images.
Hence, the ConvLSTM unit was introduced by Shi et al.~\cite{xingjian2015convolutional} to solve such a problem.
For ConvLSTM, the convolution operations are applied both at the input-to-state transition and at the state-to-state transitions.  
The ConvLSTM unit, shown in Fig.~\ref{fig:ConvLSTM} is a variation of the LSTM cell as it performs a convolution operation within the LSTM cell.
ConvLSTM is a combination of a convolution operation and an LSTM cell.
Thus, ConvLSTM can capture the time-correlated and spatial features in a series of consecutive images.
Equation~(\ref{eq:eq5}) depicts the ConvLSTM operations as the inputs \(x_1, \dots, x_t\), hidden states \(h_1, \dots, h_t\), cell states \(c_1, \dots, c_t\) and input, forget, and output gates are represented as \(i_t, f_t\), and \(o_t\), respectively:
\begin{equation}
	\begin{aligned}
		i_{t} &=\sigma\left(W_{x_t} * x_{t}+W_{h_{t-1}} * h_{t-1}+W_{c i} \cdot c_{t-1}+b_{i}\right),
		\\
		f_{t} &=\sigma\left(W_{x f} * x_{t}+W_{h f} * h_{t-1}+W_{c f} \cdot c_{t-1}+b_{f}\right), \\
		c_{t} &=f_{t} \cdot c_{t-1}+i_{t} \cdot \tanh \left(W_{x c} * x_{t}+W_{h c} * h_{t-1}+b_{c}\right), 
		\\
		o_{t} &=\sigma\left(W_{x o} * x_{t}+W_{h o} * h_{t-1}+W_{c o} \cdot c_{t}+b_{o}\right), \\
		h_{t} &=o_{t} \cdot \tanh \left(c_{t}\right),
	\end{aligned}
	\label{eq:eq5}
\end{equation}
where (\(*\)) indicates the convolution operation, which is an element-wise multiplication operation.

Recently, ConvLSTM has become very popular and is increasingly being used in 
more and more image processing applications.
	%%%%%%%%%%%%%%%%%%%%%%%%%%%%%%%%%%%%%%%%%%%%%%%%%%%%%
	\subsection{Deep learning models}
\label{proposed_approach}
%%%%%%%%%%%%%%%%%%%%%%%%%%%%%%%%%%%%%%%%%%%%%%%%%%%%%%%%%%%%%%%%%%%%%%%%%%%%%%%
In this work, we developed two end-to-end deep learning models utilising full 
wavefield frames of Lamb wave propagation for delamination identification in 
CFRP materials, as presented in Figure~\ref{fig:proposed_models}.
The developed models have a scheme of many-to-one sequence prediction, which takes \(m\) number of frames representing the full wavefield propagation through time and their interaction with the delamination to extract the damage features and finally predict the delamination location, shape, and size in a single output image.
The proposed deep learning models were implemented on Keras API~\cite{chollet2015keras} running on top of TensorFlow on a Tesla V100 GPU from NVIDIA.
%%%%%%%%%%%%%%%%%%%%%%%%%%%%%%%%%%%%%%%%%%%%%%%%%%%%%%%%%%%%%%%%%%%%%%%%%%%%%%%%
\begin{figure} [!h]
	\centering
	\begin{subfigure}[b]{0.49\textwidth}
		\centering
		\includegraphics[width=.2\textheight]{figure5a.png}
		\caption{\centering Model-\RNum{1}} % : Convolutional LSTM model.
		\label{fig:convlstm_model}
	\end{subfigure}
	\hfill
	\begin{subfigure}[b]{0.49\textwidth}
		\centering
		\includegraphics[width=.2\textheight]{figure5b.png}
		\caption{\centering Model-\RNum{2}} % : Time distributed AE model.
		\label{fig:AE_convlstm}
	\end{subfigure}
	\caption{The architecture of the proposed deep learning models.}
	\label{fig:proposed_models}
\end{figure} 

%%%%%%%%%%%%%%%%%%%%%%%%%%%%%%%%%%%%%%%%%%%%%%%%%%
The first proposed model, presented in Figure~\ref{fig:convlstm_model} takes 
\(64\) frames as input, and it consists of three ConvLSTM layers that can 
process time series and computer vision tasks.
The first ConvLSTM layer has \(12\) filters, the second layer has \(6\) filters, and the third layer has \(12\) filters.
The kernel size of the ConvLSTM layers was set to (\(3\times3\)) with a stride of \((1)\). 
Padding was set to "same", which makes the output the same as the input in the case of stride \(1\).
Furthermore, a \(\tanh\) (the hyperbolic tangent) activation function was used within the ConvLSTM layers that output values in the range between (\(-1\) and \(1\)).
Moreover, we applied a batch normalization technique~\cite{Santurkar2018} after the first two ConvLSTM layers.

In the second implemented model presented in Figure~\ref{fig:AE_convlstm}, we 
applied an autoencoder technique (AE) which is well-known for extracting 
spatial features.
The idea of AE is to compress the input data within the encoding process and then learn how to reconstruct it back from the reduced encoded representation (latent space) to a representation that is as close to the original input as possible. 
In this model, we have investigated the use of AE to process a sequence of \(24\) frames to perform image segmentation.
Therefore, a time-dispersed layer, presented in Figure~\ref{fig:TD} was 
introduced to the model, in which it distributes the input frames into the AE 
layers in order to process them independently.
%%%%%%%%%%%%%%%%%%%%%%%%%%%%%%%%%%%%%%%%%%%%%%%%%%%%%%%%%%%%%%%%%%%%%%%%%%%%%%%%
\begin{figure}[!h]
	\centering
	\includegraphics[width=0.45\textwidth]{figure6.png}
	\caption{Flow of input frames using Time-distributed layer.}
	\label{fig:TD}
\end{figure}
%%%%%%%%%%%%%%%%%%%%%%%%%%%%%%%%%%%%%%%%%%%%%%%%%%%%%%%%%%%%%%%%%%%%%%%%%%%%%%%%

In general, an AE consists of three parts: the encoder, the bottleneck, and the decoder.
The encoder is responsible for learning how to reduce the input dimensions and compress the input data into an encoded representation.
In Figure~\ref{fig:AE_convlstm}, the encoder part consists of four levels of 
downsampling. 
The purpose of having different scale levels is to extract feature maps from the input image at different scales.
Every level at the encoder consists of two 2D convolution operations followed by a Batch Normalization, and then a Dropout is applied.
Furthermore, at the end of each level, a Maxpooling operation is applied to reduce the dimensionality of the inputs. 
The bottleneck presented in Figure~\ref{fig:AE_convlstm} has the lowest level 
of dimensions of the input data.
It consists of two 2D convolution operations followed by batch normalization.
The decoder part presented in Figure~\ref{fig:AE_convlstm} is responsible for 
learning how to restore the original dimensions of the input.
The decoder part consists of two 2D convolutional operations followed by batch normalization and dropout, and an upsampling is applied at the end of each decoder level to retrieve the dimensions of its inputs.
Skip connections linking the encoder with the corresponding decoder levels were added to enhance the feature extraction process.
The outputs of the decoder were forwarded into the ConvLSTM2D layer to learn long-term spatio-temporal features.

In both models, we applied a 2D convolutional layer as the final output layer followed by a sigmoid activation function that outputs values in a range from \((0,1)\) to indicate the delamination probability.
Consequently, a threshold value must be chosen to classify the output into damaged (represented by \(1\)) or undamaged (represented by \(0\)).
Hence, we set the threshold value to (\(0.5\)) to exclude all values below the threshold by considering them as undamaged and taking only those values greater than the threshold to be considered as damaged.

For evaluating the performance of the proposed models, the mean 
intersection over union \(IoU\) (Jaccard index) was applied as the accuracy metric. 
\(IoU\) is estimated by determining the intersection
area between the ground truth and the predicted output. 
Further, we have two output classes (damaged and undamaged), the \(IoU\) was calculated for the damaged class only. 
Equation~(\ref{eqn:iou}) defines the \(IoU\) metric: 
\begin{equation}
	IoU=\frac{Intersection}{Union}=\frac{\hat{Y} \cap Y}{\hat{Y} \cup Y},
	\label{eqn:iou}
\end{equation}
where \(\hat{Y}\) is the predicted output, and \(Y\) is the ground truth.
Additionally, the percentage area error $\epsilon$ depicted in 
equation~(\ref{eqn:mean_size_error}) was utilised to evaluate the performance 
of the models:
\begin{equation}
	\epsilon=\frac{|A-\hat{A}|}{A} \times 100\%,
	\label{eqn:mean_size_error}
\end{equation}
where \(A\) and \(\hat{A}\) refer to the area in mm\textsuperscript{2} of the damage class in the ground truth and the predicted output, respectively.
This metric can indicate how close the area of the predicted delamination is to the ground truth.
Accordingly, the lower the value of $\epsilon$, the higher the accuracy of the identified damage. 
Furthermore, for all predicted outputs, the delamination localisation error (the distance between the delamination centres of the GT and the predicted output) was less than \(0.001\%\), hence, it is not considered in the discussion section.
	%%%%%%%%%%%%%%%%%%%%%%%%%%%%%%%%%%%%%%%%%%%%%%%%%%%%%
	\section{Results and discussions}
%%%%%%%%%%%%%%%%%%%%%%%%%%%%%%%%%%%%%%%%%%%%%%%%%%
In this section, we present the evaluation of the proposed model based 
on numerical test data of \(95\) delamination cases representing the frames of 
the full wavefield propagation, which was not shown to the proposed model 
during training. 
The proposed model was evaluated using numerical test data to 
demonstrate the capability to predict delamination location, shape, and size 
from the reference frame (without delamination), and the delamination 
information in binary form.

Three different representative cases were selected from the numerical dataset 
to show the performance of the developed model.
Figures~\ref{fig:first_scenario},~\ref{fig:second_scenario}, 
and~\ref{fig:third_scenario} shows three different frames from three different 
numerical test cases.  
Figures~\ref{fig:first_scenario_ref_28},~\ref{fig:first_scenario_ref_30},
~\ref{fig:first_scenario_ref_32},~\ref{fig:second_scenario_ref_28},
~\ref{fig:second_scenario_ref_30},~\ref{fig:second_scenario_ref_32},
~\ref{fig:third_scenario_ref_28},~\ref{fig:third_scenario_ref_30}, 
and~\ref{fig:third_scenario_ref_32} represents the frame without delamination 
(reference) frame, which is given as input to the deep learning model for the 
prediction purpose. 
Figures~\ref{fig:first_scenario_pred_28},~\ref{fig:first_scenario_pred_30},
~\ref{fig:first_scenario_pred_32},~\ref{fig:second_scenario_pred_28},
~\ref{fig:second_scenario_pred_30},~\ref{fig:second_scenario_pred_32},
~\ref{fig:third_scenario_pred_28}, ~\ref{fig:third_scenario_pred_30}, 
and~\ref{fig:third_scenario_pred_32} represents the predicted frame by the deep 
learning model. 
Whereas, 
figures~\ref{fig:first_scenario_lab_28},~\ref{fig:first_scenario_lab_30},
~\ref{fig:first_scenario_lab_32},~\ref{fig:second_scenario_lab_28},
~\ref{fig:second_scenario_lab_30},~\ref{fig:second_scenario_lab_32},
~\ref{fig:third_scenario_lab_28}, ~\ref{fig:third_scenario_lab_30}, 
and~\ref{fig:third_scenario_lab_32} represents the the label frame, to which 
the prediction of the proposed model is compared.
Furthermore, figures~\ref{fig:first_scenario_ref_28}, 
~\ref{fig:first_scenario_pred_28},~\ref{fig:first_scenario_lab_28},
~\ref{fig:second_scenario_ref_28}, 
~\ref{fig:second_scenario_pred_28},~\ref{fig:second_scenario_lab_28},
~\ref{fig:third_scenario_ref_28}, ~\ref{fig:third_scenario_pred_28} 
and~\ref{fig:third_scenario_lab_28} represents the $28\textsuperscript{th}$ 
frame after the interaction with the delamination. 
Figures~\ref{fig:first_scenario_ref_30}, 
~\ref{fig:first_scenario_pred_30},~\ref{fig:first_scenario_lab_30},
~\ref{fig:second_scenario_ref_30}, 
~\ref{fig:second_scenario_pred_30},~\ref{fig:second_scenario_lab_30},
~\ref{fig:third_scenario_ref_30}, ~\ref{fig:third_scenario_pred_30} 
and~\ref{fig:third_scenario_lab_30} represents $30\textsuperscript{th}$ frame 
after the interaction with the delamination.
Whereas, figures~\ref{fig:first_scenario_ref_32}, 
~\ref{fig:first_scenario_pred_32},~\ref{fig:first_scenario_lab_32},
~\ref{fig:second_scenario_ref_32}, 
~\ref{fig:second_scenario_pred_32},~\ref{fig:second_scenario_lab_32},
~\ref{fig:third_scenario_ref_32}, ~\ref{fig:third_scenario_pred_32} 
and~\ref{fig:third_scenario_lab_32} represents $32\textsuperscript{th}$ frame 
after the interaction with the delamination.

%%%%%%%%%%%%%%%%%%%%%%%%%%%%%%%%%%%%%%%%%%%%%%%%%%%%%%%%%%%%%%%%%%%%%%%%%%%%%%%%
\begin{figure} [!ht]
	\centering
	\begin{subfigure}[b]{0.32\textwidth}
		\centering
		\includegraphics[scale=0.7]{Graphics/figure6a.png}
		\caption{Reference}
		\label{fig:first_scenario_ref_28}
	\end{subfigure}
	\hfill
	\begin{subfigure}[b]{0.32\textwidth}
		\centering
		\includegraphics[scale=0.7]{Graphics/figure6b.png}
		\caption{Prediction}
		\label{fig:first_scenario_pred_28}
	\end{subfigure}
	\hfill
	\begin{subfigure}[b]{0.32\textwidth}
		\centering
		\includegraphics[scale=0.7]{Graphics/figure6c.png}
		\caption{Label}
		\label{fig:first_scenario_lab_28}
	\end{subfigure}	
	\hfill
	\begin{subfigure}[b]{0.32\textwidth}
		\centering
		\includegraphics[scale=0.7]{Graphics/figure6d.png}
		\caption{Reference}
		\label{fig:first_scenario_ref_30}
	\end{subfigure}
	\hfill
	\begin{subfigure}[b]{0.32\textwidth}
		\centering
		\includegraphics[scale=0.7]{Graphics/figure6e.png}
		\caption{Prediction}
		\label{fig:first_scenario_pred_30}
	\end{subfigure}
	\hfill
	\begin{subfigure}[b]{0.32\textwidth}
		\centering
		\includegraphics[scale=0.7]{Graphics/figure6f.png}
		\caption{Label}
		\label{fig:first_scenario_lab_30}
	\end{subfigure}

	\hfill
	\begin{subfigure}[b]{0.32\textwidth}
		\centering
		\includegraphics[scale=0.7]{Graphics/figure6g.png}
		\caption{Reference}
		\label{fig:first_scenario_ref_32}
	\end{subfigure}
	\hfill
	\begin{subfigure}[b]{0.32\textwidth}
	\centering
	\includegraphics[scale=0.7]{Graphics/figure6h.png}
	\caption{Prediction}
	\label{fig:first_scenario_pred_32}
	\end{subfigure}
	\hfill
	\begin{subfigure}[b]{0.32\textwidth}
	\centering
	\includegraphics[scale=0.7]{Graphics/figure6i.png}
	\caption{Label}
	\label{fig:first_scenario_lab_32}
	\end{subfigure}
	
	\caption{First Scenario: Comparison of predicted frames with the label 
		frames at $28\textsuperscript{th}$, $30\textsuperscript{th}$, and 
		$32\textsuperscript{th}$ frame after the interaction with delamination.}
	\label{fig:first_scenario}
\end{figure}
%%%%%%%%%%%%%%%%%%%%%%%%%%%%%%%%%%%%%%%%%%%%%%%%%%%%%%%%%%%%%%%%%%%%%%%%%%%%%%%%
\begin{figure} [!ht]
	\centering
	\begin{subfigure}[b]{0.32\textwidth}
		\centering
		\includegraphics[scale=0.7]{Graphics/figure7a.png}
		\caption{Reference}
		\label{fig:second_scenario_ref_28}
	\end{subfigure}
	\hfill
	\begin{subfigure}[b]{0.32\textwidth}
		\centering
		\includegraphics[scale=0.7]{Graphics/figure7b.png}
		\caption{Prediction}
		\label{fig:second_scenario_pred_28}
	\end{subfigure}
	\hfill
	\begin{subfigure}[b]{0.32\textwidth}
		\centering
		\includegraphics[scale=0.7]{Graphics/figure7c.png}
		\caption{Label}
		\label{fig:second_scenario_lab_28}
	\end{subfigure}	
	\hfill
	\begin{subfigure}[b]{0.32\textwidth}
		\centering
		\includegraphics[scale=0.7]{Graphics/figure7d.png}
		\caption{Reference}
		\label{fig:second_scenario_ref_30}
	\end{subfigure}
	\hfill
	\begin{subfigure}[b]{0.32\textwidth}
		\centering
		\includegraphics[scale=0.7]{Graphics/figure7e.png}
		\caption{Prediction}
		\label{fig:second_scenario_pred_30}
	\end{subfigure}
	\hfill
	\begin{subfigure}[b]{0.32\textwidth}
		\centering
		\includegraphics[scale=0.7]{Graphics/figure7f.png}
		\caption{Label}
		\label{fig:second_scenario_lab_30}
	\end{subfigure}
    	\hfill
    \begin{subfigure}[b]{0.32\textwidth}
    	\centering
    	\includegraphics[scale=0.7]{Graphics/figure7g.png}
    	\caption{Reference}
    	\label{fig:second_scenario_ref_32}
    \end{subfigure}
    \hfill
    \begin{subfigure}[b]{0.32\textwidth}
    	\centering
    	\includegraphics[scale=0.7]{Graphics/figure7h.png}
    	\caption{Prediction}
    	\label{fig:second_scenario_pred_32}
    \end{subfigure}
    \hfill
    \begin{subfigure}[b]{0.32\textwidth}
    	\centering
    	\includegraphics[scale=0.7]{Graphics/figure7i.png}
    	\caption{Label}
    	\label{fig:second_scenario_lab_32}
    \end{subfigure}
	
	\caption{Second Scenario: Comparison of predicted frames with the label 
		frames at $28\textsuperscript{th}$, $30\textsuperscript{th}$, and 
		$32\textsuperscript{th}$ frame after the interaction with delamination.}
	\label{fig:second_scenario}
\end{figure}
%%%%%%%%%%%%%%%%%%%%%%%%%%%%%%%%%%%%%%%%%%%%%%%%%%%%%%%%%%%%%%%%%%%%%%%%%%%%%%%%
\begin{figure} [!ht]
	\centering
	\begin{subfigure}[b]{0.32\textwidth}
		\centering
		\includegraphics[scale=0.7]{Graphics/figure8a.png}
		\caption{Reference}
		\label{fig:third_scenario_ref_28}
	\end{subfigure}
	\hfill
	\begin{subfigure}[b]{0.32\textwidth}
		\centering
		\includegraphics[scale=0.7]{Graphics/figure8b.png}
		\caption{Prediction}
		\label{fig:third_scenario_pred_28}
	\end{subfigure}
	\hfill
	\begin{subfigure}[b]{0.32\textwidth}
		\centering
		\includegraphics[scale=0.7]{Graphics/figure8c.png}
		\caption{Label}
		\label{fig:third_scenario_lab_28}
	\end{subfigure}	
	\hfill
	\begin{subfigure}[b]{0.32\textwidth}
		\centering
		\includegraphics[scale=0.7]{Graphics/figure8d.png}
		\caption{Reference}
		\label{fig:third_scenario_ref_30}
	\end{subfigure}
	\hfill
	\begin{subfigure}[b]{0.32\textwidth}
		\centering
		\includegraphics[scale=0.7]{Graphics/figure8e.png}
		\caption{Prediction}
		\label{fig:third_scenario_pred_30}
	\end{subfigure}
	\hfill
	\begin{subfigure}[b]{0.32\textwidth}
		\centering
		\includegraphics[scale=0.7]{Graphics/figure8f.png}
		\caption{Label}
		\label{fig:third_scenario_lab_30}
	\end{subfigure}

		\hfill
	\begin{subfigure}[b]{0.32\textwidth}
		\centering
		\includegraphics[scale=0.7]{Graphics/figure8g.png}
		\caption{Reference}
		\label{fig:third_scenario_ref_32}
	\end{subfigure}
	\hfill
	\begin{subfigure}[b]{0.32\textwidth}
		\centering
		\includegraphics[scale=0.7]{Graphics/figure8h.png}
		\caption{Prediction}
		\label{fig:third_scenario_pred_32}
	\end{subfigure}
	\hfill
	\begin{subfigure}[b]{0.32\textwidth}
		\centering
		\includegraphics[scale=0.7]{Graphics/figure8i.png}
		\caption{Label}
		\label{fig:third_scenario_lab_32}
	\end{subfigure}
	
	\caption{Third Scenario: Comparison of predicted frames with the label 
		frames at $28\textsuperscript{th}$, $30\textsuperscript{th}$, and 
		$32\textsuperscript{th}$ frame after the interaction with delamination.}
	\label{fig:third_scenario}
\end{figure}
%%%%%%%%%%%%%%%%%%%%%%%%%%%%%%%%%%%%%%%%%%%%%%%%%%%%%%%%%%%%%%%%%%%%%%%%%%%%%%%%
As can be seen in the first and second scenario (Fig~\ref{fig:first_scenario}, 
and Fig~\ref{fig:second_scenario}),  the delamination is occurred at the 
top-left of the plate, whereas in the third scenario, 
Fig~\ref{fig:third_scenario} the 
delamination is occurred at the top-center of the plate. 
As shown in Fig~\ref{fig:first_scenario}, the delamination is very tiny in this 
case and not clearly seen by human eyes, but our deep learning algorithm is 
somehow able to reconstruct the full wavefield along with the delamination. 
From all of the 
Figs~\ref{fig:first_scenario_pred_28},~\ref{fig:first_scenario_pred_30},
~\ref{fig:first_scenario_pred_32},~\ref{fig:second_scenario_pred_28},
~\ref{fig:second_scenario_pred_30},~\ref{fig:second_scenario_pred_32},
~\ref{fig:third_scenario_pred_28},~\ref{fig:third_scenario_pred_30} 
and~\ref{fig:third_scenario_pred_32} it can be confirmed the that the proposed 
deep learning-based surrogate model has reconstructed the full wavefield 
containing delamination with minimal error. 
Furthermore, the PSNR and Pearson CC values of all these three scenarios are 
shown in Table~\ref{tab:psnr_pearson}. 
%\newpage%
%%%%%%%%%%%%%%%%%%%%%%%%%%%%%%%%%%%%%%%%%%%%%%%%%%%%%%%%%%%%%%%%%%%%%%%%%%%%%%%%
% Please add the following required packages to your document preamble:
% \usepackage{booktabs}
\begin{table}[ht]
	\centering
	\caption{Evaluation metric for three numerical cases}
	\begin{tabular}{@{}ccc@{}}
		\toprule
		case number       & PSRN    & Pearson CC \\ \midrule
		1 ( $28\textsuperscript{th}$ frame) & 22.3 dB & 0.96       \\ \midrule
		1 ( $30\textsuperscript{th}$ frame)    & 22.7 dB & 0.98       \\ 
		\midrule
		1 ( $32\textsuperscript{th}$ frame)    & 23.1 dB & 0.98       \\ 
		\midrule
		2 ( $28\textsuperscript{th}$ frame) & 22.0 dB & 0.96       \\ \midrule
		2 ( $30\textsuperscript{th}$ frame)    & 22.6 dB & 0.98       \\ 
		\midrule
		2 ( $32\textsuperscript{th}$ frame)    & 23.0 dB & 0.98       \\ 
		\midrule
		3 ( $28\textsuperscript{th}$ frame) & 21.8 dB & 0.97       \\ \midrule
		3 ( $30\textsuperscript{th}$ frame) & 22.3 dB & 0.98       \\ \midrule
		3 ( $32\textsuperscript{th}$ frame)    & 23.2 dB & 0.99       \\ 
		\bottomrule
	\end{tabular}
	\label{tab:psnr_pearson}
\end{table}
%%%%%%%%%%%%%%%%%%%%%%%%%%%%%%%%%%%%%%%%%%%%%%%%%%%%%%%%%%%%%%%%%%%%%%%%%%%%%%%%


The mean PSNR value was 21.8 dB, and the mean Pearson CC value was 0.98 
on all of the test data.
%%%%%%%%%%%%%%%%%%%%%%%%%%%%%%%%%%%%%%%%%%%%%%%%%%%%%%%%%%%%%%%%%%%%%%%%%%%%%%%%
	%%%%%%%%%%%%%%%%%%%%%%%%%%%%%%%%%%%%%%%%%%%%%%%%%%%%%
	\clearpage
	%%%%%%%%%%%%%%%%%%%%%%%%%%%%%%%%%%%%%%%%%%%%%%%%%%
\section{Conclusions}
%%%%%%%%%%%%%%%%%%%%%%%%%%%%%%%%%%%%%%%%%%%%%%%%%%
In this paper, we addressed delamination detection in composite materials using a deep learning technique. 
For this purpose, we have trained an FCN-DenseNet for semantic segmentation on a numerically generated data to simulate a full wavefield elastic wave propagation.
To see the feasibility of such a study, we have compared the deep learning model with adaptive wavenumber filtering technique.
The results were promising, and the deep learning model surpasses the conventional technique in detecting the delaminations of different shapes, sizes and angles. 
Further, the model can be improved by training it on new experimental data, that means new patterns will be learned, hence it will enhance its ability to differentiate among different complex patterns.
Currently, we are in the progress of implementing several deep learning architectures in order to perform a comparative study of different deep learning models regarding delamination identification in composite materials.
Current work focuses on delamination identification, however, our work can be extended to the identification of different types of damage in composite materials.
	%%%%%%%%%%%%%%%%%%%%%%%%%%%%%%%%%%%%%%%%%%%%%%%%%%%%%

\clearpage	
%\appendix
\section*{Acknowledgments}
The research was funded by the Polish National Science Center under grant agreement no 2018/31/B/ST8/00454.
We would like to acknowledge dr Maciej Radzienski for providing the experimental data of full wavefield measured by SLDV.

\bibliography{biblography}
\bibliographystyle{unsrt}
\end{document}


