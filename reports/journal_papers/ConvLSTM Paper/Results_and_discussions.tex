\section{Results and discussions}
\begin{sloppypar}
	In this section, we present the evaluation of the proposed models based on numerical data of \(95\) different cases representing the frames of the full wavefield propagation. 
	The developed models were evaluated using numerical and experimental data to demonstrate their capability to predict delamination location, shape, and size.	
	Hence, three representative cases were selected from the numerical dataset to show the performance of the developed models.
	Furthermore, to evaluate the generalisation capability of the developed models, experimental data of single and multiple delaminations were considered.
	
	During the testing stage of the models on numerical and experimental data, we fed the models with a consecutive windows to produce intermediate predictions and final RMS image (damage map) as previously explained in Fig.~\ref{fig:Diagram_exp_predictions}.
	It is important to mention, the developed models predict damage maps with pixel values range from \((0-1)\).
	To eliminate the excess noise from the predicted damage maps and make the delamination more visible, a binary threshold was applied.
	For all the following, the threshold $t_r=0.5$ was set, therefore any value less than $0.5$ will be excluded from the damage map.
	
	\subsection{Numerical cases}
	In the first numerical case, the delamination is located at the upper left corner, as shown in Fig.~\ref{fig:num_GT_391}, representing its ground truth (GT).
	This case is considered difficult due to edge wave reflections that have similar patterns as delamination reflection.
	Figures~\ref{fig:Convlstm_num_391}, and~\ref{fig:AE_num_391} show the predicted RMS images of all intermediate predictions (damage maps) with respect to Model-I, and Model-II, respectively.
	The colour scale range from 0 to 1 corresponds to sigmoid output and can be interpreted as a probability of delamination occurrence.
	Figures~\ref{fig:Convlstm_binary_RMS_391} and~\ref{fig:AE_binary_RMS_391} show the binary RMS of damage maps with respect to Model-I, and Model-II, respectively.
	
	For the second numerical case, the delamination is located at the upper centre of the plate, as shown in Fig.~\ref{fig:num_GT_462}, representing the GT.
	This case is also considered difficult due to the waves reflected from the edge have similar patterns to those reflected from the delamination.
	Figures~\ref{fig:Convlstm_num_462}, and~\ref{fig:AE_num_462} show the damage maps predicted with respect to Model-I, and Model-II, respectively.
	Figures~\ref{fig:Convlstm_binary_RMS_462} and~\ref{fig:AE_binary_RMS_462} show the binary RMS of damage maps with respect to Model-I, and Model-II, respectively.
	
	
	In the third case, the delamination is located in the upper left corner but a little farther from the edges, as shown in Fig.~\ref{fig:num_GT_453}, representing the GT. 
	Figures~\ref{fig:Convlstm_num_453}, and \ref{fig:AE_num_453} show the predicted damage maps with respect to Model-I, and Model-II, respectively.
	Figures~\ref{fig:Convlstm_binary_RMS_453} and~\ref{fig:AE_binary_RMS_453} show the binary RMS of damage maps with respect to Model-I, and Model-II, respectively.
	
	% Numerical cases
	%%%%%%%%%%%%%%%%%%%%%%%%%%%%%%%%%%%%%%%%%%%%%%%%%%%%%%%%%%%%%%%%%%%%%%%%%%%%%%%%
	\begin{figure} [!ht]
		\centering
		%%%%%%%%%%%%%%%%%%%%%%%%%%%%%%%%%%%%%%%%%%%%%%%%%%%%%%%%%%%%%%%%%%%%%%%%%%%%%%%%
		\begin{subfigure}[b]{0.48\textwidth}
			\centering
			\includegraphics[width=1\textwidth]{figure8a.png} 
			\caption{Model-I}
			\label{fig:Convlstm_num_391}
		\end{subfigure}
		\hfill
		\begin{subfigure}[b]{0.48\textwidth}
			\centering
			\includegraphics[width=1\textwidth]{figure8b.png}
			\caption{Model-II}
			\label{fig:AE_num_391}
		\end{subfigure}
		%%%%%%%%%%%%%%%%%%%%%%%%%%%%%%%%%%%%%%%%%%%%%%%%%%%%%%%%%%%%%%%%%%%%%%%%%%%%%%%%
		\par\medskip
		%%%%%%%%%%%%%%%%%%%%%%%%%%%%%%%%%%%%%%%%%%%%%%%%%%%%%%%%%%%%%%%%%%%%%%%%%%%%%%%%
		\begin{subfigure}[b]{0.48\textwidth}
			\centering
			\includegraphics[width=1\textwidth]{figure8c.png}
			\caption{Model-I}
			\label{fig:Convlstm_num_462}
		\end{subfigure}
		\hfill
		\begin{subfigure}[b]{0.48\textwidth}
			\centering
			\includegraphics[width=1\textwidth]{figure8d.png}
			\caption{Model-II}
			\label{fig:AE_num_462}
		\end{subfigure}
		%%%%%%%%%%%%%%%%%%%%%%%%%%%%%%%%%%%%%%%%%%%%%%%%%%%%%%%%%%%%%%%%%%%%%%%%%%%%%%%%
		\par\medskip
		%%%%%%%%%%%%%%%%%%%%%%%%%%%%%%%%%%%%%%%%%%%%%%%%%%%%%%%%%%%%%%%%%%%%%%%%%%%%%%%%
		\begin{subfigure}[b]{0.48\textwidth}
			\centering
			\includegraphics[width=1\textwidth]{figure8e.png}
			\caption{Model-I}
			\label{fig:Convlstm_num_453}
		\end{subfigure}
		\hfill	
		\begin{subfigure}[b]{0.48\textwidth}
			\centering
			\includegraphics[width=1\textwidth]{figure8f.png}
			\caption{Model-II}
			\label{fig:AE_num_453}
		\end{subfigure}
		%%%%%%%%%%%%%%%%%%%%%%%%%%%%%%%%%%%%%%%%%%%%%%%%%%%%%%%%%%%%%%%%%%%%%%%%%%%%%%%%
		\caption{Predicted damage maps for numerical data (Figures: (a), (c), and (e) correspond to model-I. 
			Figures: (b), (d) and (f) correspond to Model-II).}
		\label{fig:num_case}
	\end{figure} 
	
	
	\begin{figure}[ht!]
		\centering
		%%%%%%%%%%%%%%%%%%%%%%%%%%%%%%%%%%%%%%%%%%%%%%%%%%%%%%%%%%%%%%%%%%%%%%%%
		\begin{subfigure}[b]{0.32\textwidth}
			\centering
			\includegraphics[width=1\textwidth]{figure9a.png}
			\caption{GT image of 1st case}
			\label{fig:num_GT_391}
		\end{subfigure}
		\hfill
		\begin{subfigure}[b]{0.32\textwidth}
			\centering
			\includegraphics[width=1\textwidth]{figure9b.png}
			\caption{Model-I, IoU\(=0.64\)}
			\label{fig:Convlstm_binary_RMS_391}
		\end{subfigure}
		\hfill
		\begin{subfigure}[b]{0.32\textwidth}
			\centering
			\includegraphics[width=1\textwidth]{figure9c.png}
			\caption{Model-II, IoU\(=0.74\)}
			\label{fig:AE_binary_RMS_391}
		\end{subfigure}
		%%%%%%%%%%%%%%%%%%%%%%%%%%%%%%%%%%%%%%%%%%%%%%%%%%%%%%%%%%%%%%%%%%%%%%%%
		\par\medskip
		%%%%%%%%%%%%%%%%%%%%%%%%%%%%%%%%%%%%%%%%%%%%%%%%%%%%%%%%%%%%%%%%%%%%%%%%
		\begin{subfigure}[b]{0.32\textwidth}
			\centering
			\includegraphics[width=1\textwidth]{figure9d.png}
			\caption{GT image of 2nd case}
			\label{fig:num_GT_462}
		\end{subfigure}
		\hfill
		\begin{subfigure}[b]{0.32\textwidth}
			\centering
			\includegraphics[width=1\textwidth]{figure9e.png}
			\caption{Model-I, IoU\(=0.63\)}
			\label{fig:Convlstm_binary_RMS_462}
		\end{subfigure}
		\hfill
		\begin{subfigure}[b]{0.32\textwidth}
			\centering
			\includegraphics[width=1\textwidth]{figure9f.png}
			\caption{Model-II, IoU\(=0.76\)}
			\label{fig:AE_binary_RMS_462}
		\end{subfigure}
		%%%%%%%%%%%%%%%%%%%%%%%%%%%%%%%%%%%%%%%%%%%%%%%%%%%%%%%%%%%%%%%%%%%%%%%%
		\par\medskip
		%%%%%%%%%%%%%%%%%%%%%%%%%%%%%%%%%%%%%%%%%%%%%%%%%%%%%%%%%%%%%%%%%%%%%%%%
		\begin{subfigure}[b]{0.32\textwidth}
			\centering
			\includegraphics[width=1\textwidth]{figure9g.png}
			\caption{GT image of 3rd case}
			\label{fig:num_GT_453}
		\end{subfigure}
		\hfill	
		\begin{subfigure}[b]{0.32\textwidth}
			\centering
			\includegraphics[width=1\textwidth]{figure9h.png}
			\caption{Model-I, IoU\(=0.85\)}
			\label{fig:Convlstm_binary_RMS_453}
		\end{subfigure}
		\hfill
		\begin{subfigure}[b]{0.32\textwidth}
			\centering
			\includegraphics[width=1\textwidth]{figure9i.png}
			\caption{Model-II, IoU\(=0.88\)}
			\label{fig:AE_binary_RMS_453}
		\end{subfigure}
		\caption{Binary RMS predictions regarding Model-I and Model-II.}
		\label{fig:RMS_num_cases}
	\end{figure}
	
	The developed DL models are able to identify the delamination with high accuracy and almost no noise, as seen in all predicted damage maps.
	Table~\ref{tab:num_cases} presents the evaluation metrics for Model-I and Model-II, respectively, regarding the numerical cases shown in Fig.~\ref{fig:RMS_num_cases}.
	Table~\ref{tab:num_cases} gathers the actual delamination area \(A\), predicted delamination area \(\hat{A}\), intersection over union IoU and percentage area error \(\epsilon\) with respect to each case. 
	The performance of the Model-II is slightly better than the Model-I for the selected delamination scenarios.
	Furthermore, all delamination cases should be considered for evaluation so the mean values of the proposed metrics were calculated next.
	Table~\ref{tab:meanIoU_vs_input} presents a comparison of the mean intersection over union \(\overline{\textup{IoU}}\) for \(95\) numerical test cases with respect to the DL models.
	As shown in Table~\ref{tab:meanIoU_vs_input}, Model-I and Model-II are from the current work, and take as input animations of the full wavefields, whereas the rest of the models are from our previous works~\cite{Ijjeh2021, Ijjeh2022} and take as input RMS images.
%	It can be concluded that the models that take animations as an input surpass the models that take only the RMS images as input. 
	Moreover, Model-II has a higher \(\overline{\textup{IoU}}\) compared to Model-I (\(0.80\) versus \(0.74\)).
	Furthermore, the mean percentage area error \(\overline{\epsilon}\) was calculated for 95 numerical cases and it was equal to \(4.95 \%\) and \(5.6\%\) for Model-I and~Model-II, respectively.
	In summary, on average, Model-II has better accuracy of delamination identification than Model-I and previously developed models.
	%%%%%%%%%%%%%%%%%%%%%%%%%%%%%%%%%%%%%%%%%%%%%%%%%%%%%%%%%%%%%%%%%%%%%%%%%%%%
	\begin{table}[ht!]
		\caption{Evaluation metrics of the three numerical cases.}
		\begin{tabular}{cccccccc}
			\toprule
			\multirow{2}{*}{case number} & \multicolumn{1}{c}{\multirow{2}{*}{A [mm\textsuperscript{2}]}} & \multicolumn{3}{c}{Model-I} & \multicolumn{3}{c}{Model-II} \\ \cmidrule(lr){3-5} \cmidrule(lr){6-8} 
			& \multicolumn{1}{c}{}  & \multicolumn{1}{c}{IoU}  & \multicolumn{1}{c}{\(\hat{A}\) [mm\textsuperscript{2}]} & \(\epsilon\) & \multicolumn{1}{c}{IoU}  & \multicolumn{1}{c}{\(\hat{A}\) [mm\textsuperscript{2}]} & \(\epsilon\) \\ 
			\midrule
			1 & 272 & \multicolumn{1}{c}{0.64} & \multicolumn{1}{c}{267} & 1.84\% & \multicolumn{1}{c}{0.74} & \multicolumn{1}{c}{254} & 6.62\% \\ 
			2 &  186  & \multicolumn{1}{c}{0.63} & \multicolumn{1}{c}{192} & 3.23\% & \multicolumn{1}{c}{0.76} & \multicolumn{1}{c}{178} & 4.3\% \\ 
			3 & 842 & \multicolumn{1}{c}{0.85} &\multicolumn{1}{c}{843} & 0.12\%   & \multicolumn{1}{c}{0.88} & \multicolumn{1}{c}{831} & 1.31\% \\ 
			\bottomrule
		\end{tabular}	
		\label{tab:num_cases}
	\end{table}

	\begin{table}[ht!]
		\centering
		\caption{Mean intersection over union $\overline{\textup{IoU}}$ for numerical cases with respect to the input of the model.}
		\begin{tabular}{llc}
			\toprule
			Input & Model & $\overline{\textup{IoU}}$ \\ 
			\midrule
			\multirow{2}{*}{Animations} & Model-I & 0.74 \\ & Model-II                    & \(0.80\)    \\ \midrule
			\multirow{3}{*}{RMS images}  & FCN-DenseNet~\cite{Ijjeh2021} & 0.62     \\
			& FCN-DenseNet~\cite{Ijjeh2022} & 0.68     \\
			& GCN~\cite{Ijjeh2022}          & 0.76     \\ 
			\bottomrule
		\end{tabular}
		\label{tab:meanIoU_vs_input}
	\end{table}

	
	\clearpage
	\subsection{Experimental cases}
	In this section, we investigated our models using experimentally acquired data.
	Similarly to the synthetic dataset, we applied a frequency of \(50\)~kHz to excite a signal in a transducer placed at the centre of the plate. 
	\(A0\) mode wavelength for this particular CFRP material at such frequency is about \(19.5\)~mm. 
	The measurements were performed by using Polytec PSV-\(400\) SLDV on the 
	bottom surface of the plate with dimensions of 
	\((500\times500)\)~mm\(^{2}\). 
	The measurements were conducted on a regular grid of \(333\times333\) points. 
	The measurement area was aligned with the plate edges.
	The sampling frequency was \(512\)~kHz.
	To improve the signal-to-noise ratio, 10 averages were used.
	The total scanning time for one specimen was about 9h 40'.
	Next, a median filter using a window size of three was applied to each frame. 
%	Additionally, all frames were upsampled by using cubic interpolation to \(500 \times 500\) points for Model-I and \(512\times512\) points for Model-II, respectively.

	\subsection{Single delamination}
	The first experimental case is for a CFRP specimen with a single delamination. 
	A plain weave fabric reinforcement was used for manufacturing the composite specimen.
	The delamination between layers of the fabric was created artificially by a Teflon insert of a thickness \(250\ \mu\)m. 
	The Teflon of a square shape was inserted during specimen manufacturing, so its shape and location are known.
	Based on that, the ground truth was prepared manually and it is shown in Fig.~\ref{fig:exp_CFRP_teflon_3o_GT}. 
	The number of full wavefield frames for this case is \(f_n=256\).
%	Figure~\ref{fig:model_1_CFRP_teflon_3o} shows the delamination prediction for Model-I for a window of frames \((35-99)\) for which the highest \(IoU=0.53\) was achieved.
%	Figure~\ref{fig:model_2_CFRP_teflon_3o} shows the predicted output of Model-II for a window of frames \((72-96)\) for which the highest \(IoU=0.47\) was achieved.
%	Furthermore, the percentage area error metric \(\epsilon\) for Model-I and Model-II were equal to \(41.78 \%\) and \(86.67\%\), respectively.
%
%	%%%%%%%%%%%%%%%%%%%%%%%%%%%%%%%%%%%%%%%%%%%%%%%%%%%%%%%%%%%%%%%%%%%%%%%%%%%%%%%%
%	% Single delaminatio of Teflon inserted
%	%%%%%%%%%%%%%%%%%%%%%%%%%%%%%%%%%%%%%%%%%%%%%%%%%%%%%%%%%%%%%%%%%%%%%%%%%%%%%%%%
%	\begin{figure} [!ht]
%		%%%%%%%%%%%%%%%%%%%%%%%%%%%%%%%%%%%%%%%%%%%%%%%%%%%%%%%%%%%%%%%%%%%%%%%%%%%%
%		\centering
%		%%%%%%%%%%%%%%%%%%%%%%%%%%%%%%%%%%%%%%%%%%%%%%%%%%%%%%%%%%%%%%%%%%%%%%%%%%%%
%		\begin{subfigure}[b]{0.32\textwidth}
%			\centering
%			\includegraphics[width=1\textwidth]{figure8a.png}
%			\caption{GT of Teflon insert}
%			\label{fig:exp_CFRP_teflon_3o_GT}
%		\end{subfigure}
%		%%%%%%%%%%%%%%%%%%%%%%%%%%%%%%%%%%%%%%%%%%%%%%%%%%%%%%%%%%%%%%%%%%%%%%%%%%%%
%		\hfill
%		%%%%%%%%%%%%%%%%%%%%%%%%%%%%%%%%%%%%%%%%%%%%%%%%%%%%%%%%%%%%%%%%%%%%%%%%%%%%
%		\begin{subfigure}[b]{0.32\textwidth}
%			\centering
%			\includegraphics[width=1\textwidth]{figure8b.png}
%			\caption{IoU = 0.53 } 
%			\label{fig:model_1_CFRP_teflon_3o}
%		\end{subfigure}
%		%%%%%%%%%%%%%%%%%%%%%%%%%%%%%%%%%%%%%%%%%%%%%%%%%%%%%%%%%%%%%%%%%%%%%%%%%%%%
%		\hfill
%		%%%%%%%%%%%%%%%%%%%%%%%%%%%%%%%%%%%%%%%%%%%%%%%%%%%%%%%%%%%%%%%%%%%%%%%%%%%%
%		\begin{subfigure}[b]{0.32\textwidth}
%			\centering
%			\includegraphics[width=1\textwidth]{figure8c.png}
%			\caption{IoU = 0.47}
%			\label{fig:model_2_CFRP_teflon_3o}
%		\end{subfigure}
%		%%%%%%%%%%%%%%%%%%%%%%%%%%%%%%%%%%%%%%%%%%%%%%%%%%%%%%%%%%%%%%%%%%%%%%%%%%%%
%		\caption{Experimental case: single delamination of Teflon insert.}
%		\label{fig:exp_Teflon_insert}
%	\end{figure} 
%	%%%%%%%%%%%%%%%%%%%%%%%%%%%%%%%%%%%%%%%%%%%%%%%%%%%%%%%%%%%%%%%%%%%%%%%%%%%%%%%%
	
	In both models, the predictions were highest for the window of frames corresponding to the first interaction of the guided waves with the delamination.
	Accordingly, such frames contain the most valuable feature patterns regarding delamination. 
	This behaviour is depicted in Fig.~\ref{fig:CFRP_Teflon_3o_IoU_centre_window}, which shows the IoU values with respect to the predicted intermediate outputs as we slide the window over all input frames from the starting frame till the end.
	Since there are \(256\) frames of full wavefield in this damage case, there are \(192\) windows (group of consecutive frames) for Model-I, and \(232\) windows for Model-II.
	Consequently, Model-I has \(192\) consecutive predictions, and Model-II has \(232\) consecutive predictions.
	Furthermore, in Fig.~\ref{fig:CFRP_Teflon_3o_IoU_} we selected three places for the sliding window. 
	The first place depicted in a dark blue star shown in Fig.~\ref{fig:CFRP_teflon_3o_shapes_} represents a window centered at frame 84, which correspond to the initial interaction of guided waves with the delamination.
	The second place is depicted in the pink pentagon shape shown in Fig.~\ref{fig:CFRP_teflon_3o_shapes_} which represents a window centered at frame 141 corresponding to guided waves reflected from the edges.
	We can notice the drop in the IoU values as these frames have fewer damage features.
	The third place, depicted in the green circle shown in Fig.~\ref{fig:CFRP_teflon_3o_shapes_} represents a window centered at frame 218 corresponding to the interaction of guided waves reflected from the edges with the delamination.
	As we can see, the value of IoU increases again as the valuable feature patterns regarding delamination start to appear again.
	
	The intermediate predicted outputs of Model-I and Model-II for windows centered at frame 84 (the dark blue star), frame 141 (pink pentagon), and frame 218 (the green circle) are shown in Fig.~\ref{fig:CFRP_teflon_3o_intermediate}.
	Apart from correctly identified delamination, some noise is obtained near edges of the specimen.
	%%%%%%%%%%%%%%%%%%%%%%%%%%%%%%%%%%%%%%%%%%%%%%%%%%%%%%%%%%%%%%%%%%%%%%%%%%%%%%%%
	%% IoU ouput values with a sliding window
	%%%%%%%%%%%%%%%%%%%%%%%%%%%%%%%%%%%%%%%%%%%%%%%%%%%%%%%%%%%%%%%%%%%%%%%%%%%%%%%%
	\begin{figure} [!ht]
		%%%%%%%%%%%%%%%%%%%%%%%%%%%%%%%%%%%%%%%%%%%%%%%%%%%%%%%%%%%%%%%%%%%%%%%%%%%%
		\begin{subfigure}[b]{1\textwidth}
			\centering
			\includegraphics[scale=1]{figure10a.png}
			\caption{IoU for the sliding window centered at consecutive frames.}
			\label{fig:CFRP_Teflon_3o_IoU_}
		\end{subfigure}
		%%%%%%%%%%%%%%%%%%%%%%%%%%%%%%%%%%%%%%%%%%%%%%%%%%%%%%%%%%%%%%%%%%%%%%%%%%%%
		\par\medskip
		%%%%%%%%%%%%%%%%%%%%%%%%%%%%%%%%%%%%%%%%%%%%%%%%%%%%%%%%%%%%%%%%%%%%%%%%%%%%
		\begin{subfigure}[b]{1\textwidth}
			\centering
			\includegraphics[scale=1]{figure10b.png}
			\caption{Corresponding frames of guided waves.} 
			\label{fig:CFRP_teflon_3o_shapes_}
		\end{subfigure}
		%%%%%%%%%%%%%%%%%%%%%%%%%%%%%%%%%%%%%%%%%%%%%%%%%%%%%%%%%%%%%%%%%%%%%%%%%%%%
		%%%%%%%%%%%%%%%%%%%%%%%%%%%%%%%%%%%%%%%%%%%%%%%%%%%%%%%%%%%%%%%%%%%%%%%%%%%%
		\par\medskip
		%%%%%%%%%%%%%%%%%%%%%%%%%%%%%%%%%%%%%%%%%%%%%%%%%%%%%%%%%%%%%%%%%%%%%%%%%%%%
		\begin{subfigure}[b]{1\textwidth}
			\centering
			\includegraphics[scale=.9]{figure10c.png}
			\caption{Predictions of models I and II at different window places.} 
			\label{fig:CFRP_teflon_3o_intermediate}
		\end{subfigure}
		%%%%%%%%%%%%%%%%%%%%%%%%%%%%%%%%%%%%%%%%%%%%%%%%%%%%%%%%%%%%%%%%%%%%%%%%%%%%
		\caption{IoU for the sliding window of frames (Teflon insert-single delamination), and the intermediate predictions.}
		\label{fig:CFRP_Teflon_3o_IoU_centre_window}
	\end{figure} 
	%%%%%%%%%%%%%%%%%%%%%%%%%%%%%%%%%%%%%%%%%%%%%%%%%%%%%%%%%%%%%%%%%%%%%%%%%%%%%%%%
	%%%%%%%%%%%%%%%%%%%%%%%%%%%%%%%%%%%%%%%%%%%%%%%%%%%%%%%%%%%%%%%%%%%%%%%%%%%%%%%%
	%% Predicted outuputs at diffirent window places
	%%%%%%%%%%%%%%%%%%%%%%%%%%%%%%%%%%%%%%%%%%%%%%%%%%%%%%%%%%%%%%%%%%%%%%%%%%%%%%%%
%	\begin{figure}[!ht]
%		\centering
%		\includegraphics[scale=1]{figure10.png}
%		\caption{Predictions of Model~I and Model-II for window centered at selected frames (Teflon insert - single delamination).}
%		\label{fig:CFRP_Teflon_3o_predictions}
%	\end{figure}
%	%%%%%%%%%%%%%%%%%%%%%%%%%%%%%%%%%%%%%%%%%%%%%%%%%%%%%%%%%%%%%%%%%%%%%%%%%%%%%%%%
	
	Figures~\ref{fig:RMS_CFRP_Teflon_3o_saeed} and~\ref{fig:RMS_CFRP_Teflon_3o_ijjeh} show the RMS images (damage maps) for the experimental case of single delamination predicted by Model-I and Model-II, respectively.
	Additionally, to separate undamaged and damaged classes from the RMS images, a binary threshold was applied as shown in Figs.~\ref{fig:RMS_threshold_CFRP_Teflon_3o_saeed} and~\ref{fig:RMS_threshold_CFRP_Teflon_3o_ijjeh} for Model-I and Model-II, respectively. 
	The calculated IoU values for the case of single delamination are \(\textup{IoU}=0.43\) and \(\textup{IoU}=0.41\) for Model-I and Model-II, respectively.
	
	% RMS predictions
	%%%%%%%%%%%%%%%%%%%%%%%%%%%%%%%%%%%%%%%%%%%%%%%%%%%%%%%%%%%%%%%%%%%%%%%%%%%%%%%%
	\begin{figure} [!ht]
		\begin{subfigure}[b]{.48\textwidth}
			\centering
			\includegraphics[width=1\textwidth]{figure11a.png}
			\caption{Model-I}
			\label{fig:RMS_CFRP_Teflon_3o_saeed}
		\end{subfigure}
		\hfill
		\begin{subfigure}[b]{.48\textwidth}
			\centering
			\includegraphics[width=1\textwidth]{figure11b.png}
			\caption{Model-II} 
			\label{fig:RMS_CFRP_Teflon_3o_ijjeh}
		\end{subfigure}
		\caption{RMS images (damage maps); Teflon insert - single delamination.}
		\label{fig:RMS_CFRP_Teflon_3o_images}
	\end{figure} 

	\begin{figure} [!ht]
		\begin{subfigure}[b]{0.32\textwidth}
			\centering
			\includegraphics[width=1\textwidth]{figure12a.png}
			\caption{GT of Teflon insert}
			\label{fig:exp_CFRP_teflon_3o_GT}
		\end{subfigure}
		\hfill
		\begin{subfigure}[b]{.32\textwidth}
			\centering
			\includegraphics[width=1\textwidth]{figure12b.png}
			\caption{Model-I, IoU = \(0.43\)}
			\label{fig:RMS_threshold_CFRP_Teflon_3o_saeed}
		\end{subfigure}
		\hfill
		\begin{subfigure}[b]{.32\textwidth}
			\centering
			\includegraphics[width=1\textwidth]{figure12c.png}
			\caption{Model-II, IoU = \(0.41\)} 
			\label{fig:RMS_threshold_CFRP_Teflon_3o_ijjeh}
		\end{subfigure}
		\caption{Thresholded RMS images; Teflon insert - single delamination.}
		\label{fig:RMS_threshold_CFRP_Teflon_3o_images}
	\end{figure} 
	
	Table~\ref{tab:single_case} presents the evaluation metrics for Model-I and Model-II, receptively, regarding the experimental case of single delamination shown in Fig.~\ref{fig:RMS_threshold_CFRP_Teflon_3o_images}.
	As shown in Table~\ref{tab:single_case}, the actual areas \(A\) and predicted areas \(\hat{A}\) of delaminations were computed in [mm\textsuperscript{2}] with respect to each case. 
	The percentage area error \(\epsilon\) was calculated for both models accordingly.
	It is evident that better accuracy was obtained for Model-I because of higher IoU value and lower percentage area error than for Model-II. 
	
	It should be noted that for the same damage scenario, the IoU value for the models developed previously in~\cite{Ijjeh2021} was very low \((\textup{IoU}=0.081)\).
	%%%%%%%%%%%%%%%%%%%%%%%%%%%%%%%%%%%%%%%%%%%%%%%%%%%%%%%%%%%%%%%%%%%%%%%%%%%%
	\begin{table}[ht]
		\setlength{\tabcolsep}{3pt} %% default is 6pt
		
		\caption{Evaluation metrics for experimental case of single delamination.}
		\begin{tabular}{cccccccc}
			\toprule
			\multirow{2}{*}{\begin{tabular}[c]{@{}c@{}}Experimental \\ case\end{tabular}} & \multirow{2}{*}{\(A\) [mm\textsuperscript{2}]} & \multicolumn{3}{c}{Model-I} & \multicolumn{3}{c}{Model-II}  \\ 
			\cmidrule(lr){3-5} \cmidrule(lr){6-8}
			&  & \multicolumn{1}{c}{IoU} & \multicolumn{1}{c}{\(\hat{A}\) [mm\textsuperscript{2}] } & \(\epsilon\) & \multicolumn{1}{c}{IoU}  &\multicolumn{1}{c}{\(\hat{A}\) [mm\textsuperscript{2}]} & \(\epsilon\) \\ 
			\midrule
			Single delamination & 225 & \multicolumn{1}{c}{0.43} &  \multicolumn{1}{c}{364} & 61.78\%    & \multicolumn{1}{c}{0.41} & \multicolumn{1}{c}{386} & 71.56\%    \\
			\bottomrule
		\end{tabular}
		\label{tab:single_case}
	\end{table}
	%%%%%%%%%%%%%%%%%%%%%%%%%%%%%%%%%%%%%%%%%%%%%%%%%%%%%%%%%%%%%%%%%%%%%%%%%%%%
	
	\clearpage

	\subsection{Multiple delaminations}
	%%%%%%%%%%%%%%%%%%%%%%%%%%%%%%%%%%%%%%%%%%%%%%%%%%%%%%%%%%%%%%%%%%%%%%%%%%%%%%%%
	In the second experimental case, we investigated three specimens of carbon/epoxy laminate reinforced by 16 layers of plain weave fabric as shown in Fig.~\ref{fig:plate_delam_arrangment}. 
	Teflon inserts with a thickness of \(250\ \mu\)m were used to simulate the delaminations.
	The prepregs GG 205 P (fibres Toray FT 300–3K 200 tex) by G.~Angeloni and epoxy resin IMP503Z‐HT by Impregnatex Compositi were used for the fabrication of the specimen in the autoclave. 
	The average thickness of the specimen was \(3.9 \pm 0.1\) mm.
	%%%%%%%%%%%%%%%%%%%%%%%%%%%%%%%%%%%%%%%%%%%%%%%%%%%%%%%%%%%%%%%%%%%%%%%%%%%%
	
	In Specimen~II, three large artificial delaminations of elliptic shape were inserted in the upper thickness quarter of the plate between the \(4^{th}\) and the \(5^{th}\) layer.
	The delaminations were located at the same distance, equal to \(150\) mm from the centre of the plate.
	For Specimen~\RNum{3} delaminations were inserted in the middle of the cross-section of the plate between \(8^{th}\) layer and \(9^{th}\) layer.
	For Specimen~\RNum{4}, three small delaminations were inserted in the upper quarter of the cross-section of the plate, and three large delaminations were inserted at the lower quarter of the cross-section of the plate between the \(12^{th}\) layer and \(13^{th}\) layer.
	The details of Specimen~II,~III, and~IV are presented in Fig.~\ref{fig:plate_delam_arrangment}.
	
	Furthermore, the SLDV measurements were conducted from the bottom surface of the plate. 
	Consequently, Specimen II is the most difficult case.
	It is because the delaminations in the cross-section are farther away from the bottom surface than in other specimens (III and IV).
	As a consequence, the reflections from delaminations are barely visible in the measured wavefield.
	For Specimens~(II,~III, and~IV),  we have generated \(f_n=512\) consecutive frames representing the full wavefield measurements in the plate.
	The measurement parameters were the same as in the experiment with the single delamination.
	%%%%%%%%%%%%%%%%%%%%%%%%%%%%%%%%%%%%%%%%%%%%%%%%%%%%%%%%%%%%%%%%%%%%%%%%%%%%
	\begin{figure}[!ht]
		\centering
		\includegraphics[width=1\textwidth]{figure13.png}
		\caption{Experimental case of delamination arrangement.}
		\label{fig:plate_delam_arrangment}
	\end{figure}
	%%%%%%%%%%%%%%%%%%%%%%%%%%%%%%%%%%%%%%%%%%%%%%%%%%%%%%%%%%%%%%%%%%%%%%%%%%%%
	
	Since SLDV measurements were conducted from the bottom surface of the plate, the GT images and the predicted outputs of the proposed models are flipped horizontally (mirrored).
	Figures~\ref{fig:gt_specimen_2},~\ref{fig:gt_specimen_3},~\ref{fig:gt_specimen_4} show the GT image of Specimen~II,~III,~IV, respectively.

	The largest delaminations in the cross-sections were assumed to be GT because the full wavefield was acquired from the bottom surface of the specimen (see Specimen~\RNum{4} in Fig.~\ref{fig:plate_delam_arrangment}).
	We need to mention that such a case with stacked delaminations in the cross-section was not modeled numerically.
	Although the models were not trained on such a scenario, the predictions were satisfactory.
%	The predicted output of Model-I is shown in Fig.~\ref{fig:L3_S4_B_saeed} in which the highest IoU\(=0.18\) was achieved for window of frames \((235-299)\).
%	Figure~\ref{fig:L3_S4_B_ijjeh} shows the predicted output of Model-II, in which the highest IoU\(=0.27\) was achieved for window of frames \((68-92)\).
	%  Specimen~II
%	\begin{figure} [!ht]
%		\centering
%		\begin{subfigure}[b]{0.32\textwidth}
%			\centering
%			\includegraphics[width=1\textwidth]{figure14a.png}
%			\caption{GT of Specimen~II}
%			\label{fig:gt_specimen_2}
%		\end{subfigure}
%		\hfill
%		\begin{subfigure}[b]{0.32\textwidth}
%			\centering
%			\includegraphics[width=1\textwidth]{figure14b.png}
%			\caption{IoU = \(0.15\)} 
%			\label{fig:L3_S2_B_saeed}
%		\end{subfigure}
%		\hfill
%		\begin{subfigure}[b]{0.32\textwidth}
%			\centering
%			\includegraphics[width=1\textwidth]{figure14c.png}
%			\caption{IoU = \(0.35\)} 
%			\label{fig:L3_S2_B_ijjeh}
%		\end{subfigure}
%		\par\medskip
%		%  Specimen~\RNum{3}
%		\begin{subfigure}[b]{0.32\textwidth}
%			\centering
%			\includegraphics[width=1\textwidth]{figure14a.png}
%			\caption{GT of Specimen~\RNum{3}}
%			\label{fig:gt_specimen_3}
%		\end{subfigure}
%		%%%%%%%%%%%%%%%%%%%%%%%%%%%%%%%%%%%%%%%%%%%%%%%%%%%%%%%%%%%%%%%%%%%%%%%%%%%%%%%%
%		\hfill
%		\begin{subfigure}[b]{0.32\textwidth}
%			\centering
%			\includegraphics[width=1\textwidth]{figure14e.png}
%			\caption{IoU = \(0.18\)} 
%			\label{fig:L3_S3_B_saeed}
%		\end{subfigure}
%		%%%%%%%%%%%%%%%%%%%%%%%%%%%%%%%%%%%%%%%%%%%%%%%%%%%%%%%%%%%%%%%%%%%%%%%%%%%%%%%%
%		\hfill
%		\begin{subfigure}[b]{0.32\textwidth}
%			\centering
%			\includegraphics[width=1\textwidth]{figure14f.png}
%			\caption{IoU = \(0.32\)} 
%			\label{fig:L3_S3_B_ijjeh}
%		\end{subfigure}
%		%%%%%%%%%%%%%%%%%%%%%%%%%%%%%%%%%%%%%%%%%%%%%%%%%%%%%%%%%%%%%%%%%%%%%%%%%%%%
%		\par\medskip
%		%  Specimen~\RNum{4}
%		\begin{subfigure}[b]{0.32\textwidth}
%			\centering
%			\includegraphics[width=1\textwidth]{figure14a.png}
%			\caption{GT of Specimen~\RNum{4}}
%			\label{fig:gt_specimen_4}
%		\end{subfigure}
%		%%%%%%%%%%%%%%%%%%%%%%%%%%%%%%%%%%%%%%%%%%%%%%%%%%%%%%%%%%%%%%%%%%%%%%%%%%%%%%%%
%		\hfill
%		\begin{subfigure}[b]{0.32\textwidth}
%			\centering
%			\includegraphics[width=1\textwidth]{figure14h.png}
%			\caption{IoU = \(0.18\)}  
%			\label{fig:L3_S4_B_saeed}
%		\end{subfigure}
%		%%%%%%%%%%%%%%%%%%%%%%%%%%%%%%%%%%%%%%%%%%%%%%%%%%%%%%%%%%%%%%%%%%%%%%%%%%%%%%%%
%		\hfill
%		\begin{subfigure}[b]{0.32\textwidth}
%			\centering
%			\includegraphics[width=1\textwidth]{figure14i.png}
%			\caption{IoU = \(0.27\)} 
%			\label{fig:L3_S4_B_ijjeh}
%		\end{subfigure}
%		%%%%%%%%%%%%%%%%%%%%%%%%%%%%%%%%%%%%%%%%%%%%%%%%%%%%%%%%%%%%%%%%%%%%%%%%%%%%%%%%
%		\caption{Experimental cases of Specimens II, \RNum{3}, and \RNum{4}.}
%		\label{fig:exp_case}
%	\end{figure} 
	%%%%%%%%%%%%%%%%%%%%%%%%%%%%%%%%%%%%%%%%%%%%%%%%%%%%%%%%%%%%%%%%%%%%%%%%%%%%%%%%
	
	It is worth mentioning that we tested FCN-DenseNet, which we developed previously~\cite{Ijjeh2021} for data related to specimens~II-IV. 
	However, poor results were obtained. 
	This is attributed to the fact that RMS images are fed to FCN-DenseNet, which carries a limited amount of damage-related information. 
	On the other hand, currently proposed methods utilise full wavefield frames, which carry more damage-reach features. 
		
%	The red square depicted in Fig.~\ref{fig:L3_S4_B_333x333p_50kHz_5HC_shapes_} corresponds to IoU value calculated for the window of frames before the interactions with delaminations (the frame for the centre of the window is shown).
%	This behaviour is expected since the models were trained on those frames depicting the beginning of the interactions of guided wave with delamination. 
%	As a result, valuable feature patterns start to appear later on.
%	
%	The light blue star depicted in Fig.~\ref{fig:L3_S4_B_333x333p_50kHz_5HC_IoU_centre_window} refers to a window of frames regarding the initial interactions of propagating guided waves with the delaminations and before reflecting from the edges.
%	Hence, valuable feature patterns regarding the damage are starting to appear.
%	
%	The pink pentagon shape depicted in Fig.~\ref{fig:L3_S4_B_333x333p_50kHz_5HC_IoU_centre_window} refers to a window of frames that pass the initial interaction with the delaminations. 
%	Furthermore, it shows the reflections of the guided waves from the edges just before interacting with the delaminations again.
%	As can be seen, the calculated IoU values drop drastically, as expected, as there are no valuable damage features to be extracted.  
%	
%	The blue rectangle depicted in Fig.~\ref{fig:L3_S4_B_333x333p_50kHz_5HC_IoU_centre_window} represents high IoU value regarding Model-I and corresponding frame at the centre of the window.
%	The frame depicts the interactions of the reflected guided waves from the edges and their interactions with the delaminations. 
%	Hence, Model-I can identify the delaminations as it has a larger window.
%	
%	The green circle depicted in Fig.~\ref{fig:L3_S4_B_333x333p_50kHz_5HC_IoU_centre_window} represents high IoU value regarding Model-II and corresponding frame at the centre of the window.
%	Although this frame shows complex patterns of wave reflections, the model can extract the valuable damage features and identify the delaminations accordingly.
%	%%%%%%%%%%%%%%%%%%%%%%%%%%%%%%%%%%%%%%%%%%%%%%%%%%%%%%%%%%%%%%%%%%%%%%%%%%%%
	Figure~\ref{fig:L3_S234_B_333x333p_50kHz_5HC_IoU_centre_window} shows the calculated IoU values corresponding to predicted intermediate outputs of Model-I and Model-II regarding Specimen~II,~III, and ~IV, respectively, as we slide the window of size~\(64\) and \(24\) frames over the \(512\) full wavefield frames.
	As mentioned earlier, Specimen~II is the most difficult case.
	As a consequence, the maximum and average values of IoU curves are lowest in Fig.~\ref{fig:L3_S2_B_333x333p_50kHz_5HC_IoU}. 
	It can be seen in Fig.~\ref{fig:L3_S2_B_333x333p_50kHz_5HC_IoU} that the Model-II starts identifying the delaminations before Model-I.
	The RMS images depicting the damage maps of Specimen~II are presented in Figs.~\ref{fig:RMS_L3_S2_B_saeed} and~\ref{fig:RMS_L3_S2_B_ijjeh} regarding Model-I and Model-II, respectively.
	Figure~\ref{fig:RMS_threshold_L3_S2_B_saeed} shows the thresholded RMS image for Model-I, and the calculated IoU\(=0.00\).
	In the damage map (Fig.~\ref{fig:RMS_L3_S2_B_saeed}) produced by Model-I, the three delaminations are visible.
	However, when applying a binary threshold, the three identified delaminations vanish as their pixel values are below the threshold value. 
	Figure~\ref{fig:RMS_threshold_L3_S2_B_ijjeh} shows the thresholded RMS image for Model-II, and the calculated IoU\(=0.53\).
	
	Figure~\ref{fig:L3_S3_B_333x333p_50kHz_5HC_shapes_} shows the calculated IoU of the intermediate outputs of Model-I and Model-II regarding Specimen-III.
	It is clear for Specimen-III that Model-I starts to identify the delaminations earlier than in Specimen-II, as the damage features are more evident in this case.
	Figure~\ref{fig:RMS_threshold_L3_S3_B_saeed} shows the thresholded RMS image for Model-I, and the calculated IoU\(=0.01\).
	Moreover, for Specimen-III, the three delaminations are visible in the damage map (Fig.~\ref{fig:RMS_L3_S3_B_saeed}) produced by Model-I.
	However, when applying a binary threshold, almost all identified delaminations vanish as their pixel values are below the threshold value.
	Figure~\ref{fig:RMS_threshold_L3_S2_B_ijjeh} shows the thresholded RMS image for Model-II, and the calculated IoU\(=0.64\).
	
	Figure~\ref{fig:L3_S4_B_333x333p_50kHz_5HC_shapes_} shows the calculated IoU of the intermediate outputs of Model-I and Model-II regarding Specimen-IV.
	For Specimen-IV, the delaminations are closer to the bottom surface of the plate where the SLDV measurements were conducted.
	Hence, it is expected that both models will be able to identify the delaminations.
	Figure~\ref{fig:RMS_threshold_L3_S4_B_saeed} shows the thresholded RMS image for Model-I, and the calculated IoU\(=0.56\).
	Figure~\ref{fig:RMS_threshold_L3_S4_B_ijjeh} shows the thresholded RMS image for Model-II, and the calculated IoU\(=0.52\).
	
	For all three specimens (II,~III, and~IV), it can be concluded that Model-II showed a better performance regarding delamination identification compared to Model-I, which was only able to identify the delaminations of Specimen~IV.

	\begin{figure}[!ht]
		\centering
		%%%%%%%%%%%%%%%%%%%%%%%%%%%%%%%%%%%%%%%%%%%%%%%%%%%%%%%%%%%%%%%%%%%%%%%%
		\begin{subfigure}[b]{0.8\textwidth}
			\centering
			\includegraphics[width=1\textwidth]{figure14a.png}
			\caption{Specimen~II}
			\label{fig:L3_S2_B_333x333p_50kHz_5HC_IoU}
		\end{subfigure}
		%%%%%%%%%%%%%%%%%%%%%%%%%%%%%%%%%%%%%%%%%%%%%%%%%%%%%%%%%%%%%%%%%%%%%%%%
		\par\medskip
		%%%%%%%%%%%%%%%%%%%%%%%%%%%%%%%%%%%%%%%%%%%%%%%%%%%%%%%%%%%%%%%%%%%%%%%%
		\begin{subfigure}[b]{0.8\textwidth}
			\centering
			\includegraphics[width=1\textwidth]{figure14b.png}
			\caption{Specimen~III} 
			\label{fig:L3_S3_B_333x333p_50kHz_5HC_shapes_}
		\end{subfigure}
		%%%%%%%%%%%%%%%%%%%%%%%%%%%%%%%%%%%%%%%%%%%%%%%%%%%%%%%%%%%%%%%%%%%%%%%%
		\par\medskip
		%%%%%%%%%%%%%%%%%%%%%%%%%%%%%%%%%%%%%%%%%%%%%%%%%%%%%%%%%%%%%%%%%%%%%%%%
		\begin{subfigure}[b]{0.8\textwidth}
			\centering
			\includegraphics[width=1\textwidth]{figure14c.png}
			\caption{Specimen~IV} 
			\label{fig:L3_S4_B_333x333p_50kHz_5HC_shapes_}
		\end{subfigure}
		%%%%%%%%%%%%%%%%%%%%%%%%%%%%%%%%%%%%%%%%%%%%%%%%%%%%%%%%%%%%%%%%%%%%%%%%
		\caption{IoU for a sliding window of frames for Specimen II, III and IV for Model-I and Model-II.}
		\label{fig:L3_S234_B_333x333p_50kHz_5HC_IoU_centre_window}
	\end{figure} 

	%%%%%%%%%%%%%%%%%%%%%%%%%%%%%%%%%%%%%%%%%%%%%%%%%%%%%%%%%%%%%%%%%%%%%%%%%%%%
	% RMS predictions
	%%%%%%%%%%%%%%%%%%%%%%%%%%%%%%%%%%%%%%%%%%%%%%%%%%%%%%%%%%%%%%%%%%%%%%%%%%%%
	
	\begin{figure} [!ht]
		\begin{subfigure}[b]{.48\textwidth}
			\centering
			\includegraphics[width=1\textwidth]{figure15a.png}
			\caption{Model-I}
			\label{fig:RMS_L3_S4_B_saeed}
		\end{subfigure}
		\hfill
		\begin{subfigure}[b]{.48\textwidth}
			\centering
			\includegraphics[width=1\textwidth]{figure15b.png}
			\caption{Model-II} 
			\label{fig:RMS_L3_S2_B_ijjeh}
		\end{subfigure}
		\hfill
		\begin{subfigure}[b]{.48\textwidth}
			\centering
			\includegraphics[width=1\textwidth]{figure15c.png}
			\caption{Model-I}
			\label{fig:RMS_L3_S2_B_saeed}
		\end{subfigure}
		\hfill
		\begin{subfigure}[b]{.48\textwidth}
			\centering
			\includegraphics[width=1\textwidth]{figure15d.png}
			\caption{Model-II} 
			\label{fig:RMS_L3_S3_B_ijjeh}
		\end{subfigure}
		\hfill
		\begin{subfigure}[b]{.48\textwidth}
			\centering
			\includegraphics[width=1\textwidth]{figure15e.png}
			\caption{Model-I}
			\label{fig:RMS_L3_S3_B_saeed}
		\end{subfigure}
		\hfill
		\begin{subfigure}[b]{.48\textwidth}
			\centering
			\includegraphics[width=1\textwidth]{figure15f.png}
			\caption{Model-II} 
			\label{fig:RMS_L3_S4_B_ijjeh}
		\end{subfigure}
		\caption{RMS images (damage maps).}
		\label{fig:RMS_L3_S_B__images}
	\end{figure} 
	
	% RMS THERSHOLDED predictions
	\begin{figure} [!ht]
		\centering
		\begin{subfigure}[b]{0.32\textwidth}
			\centering
			\includegraphics[width=1\textwidth]{figure16a.png}
			\caption{GT of Specimen~II}
			\label{fig:gt_specimen_2}
		\end{subfigure}
		\hfill
		\begin{subfigure}[b]{.32\textwidth}
			\centering
			\includegraphics[width=1\textwidth]{figure16b.png}
			\caption{Model-I, IoU =$0.0$}
			\label{fig:RMS_threshold_L3_S2_B_saeed}
		\end{subfigure}
		\hfill
		\begin{subfigure}[b]{.32\textwidth}
			\centering
			\includegraphics[width=1\textwidth]{figure16c.png}
			\caption{Model-II, IoU=$0.53$} 
			\label{fig:RMS_threshold_L3_S2_B_ijjeh}
		\end{subfigure}
		\hfill
		\begin{subfigure}[b]{0.32\textwidth}
			\centering
			\includegraphics[width=1\textwidth]{figure16a.png}
			\caption{GT of Specimen~\RNum{3}}
			\label{fig:gt_specimen_3}
		\end{subfigure}
		\hfill
		\begin{subfigure}[b]{.32\textwidth}
			\centering
			\includegraphics[width=1\textwidth]{figure16e.png}
			\caption{Model-I, IoU=$0.01$}
			\label{fig:RMS_threshold_L3_S3_B_saeed}
		\end{subfigure}
		\hfill
		\begin{subfigure}[b]{.32\textwidth}
			\centering
			\includegraphics[width=1\textwidth]{figure16f.png}
			\caption{Model-II, IoU=$0.64$} 
			\label{fig:RMS_threshold_L3_S3_B_ijjeh}
		\end{subfigure}
		\hfill
		\begin{subfigure}[b]{0.32\textwidth}
			\centering
			\includegraphics[width=1\textwidth]{figure16a.png}
			\caption{GT of Specimen~\RNum{4}}
			\label{fig:gt_specimen_4}
		\end{subfigure}
		\hfill
		\begin{subfigure}[b]{.32\textwidth}
			\centering
			\includegraphics[width=1\textwidth]{figure16h.png}
			\caption{Model-I, IoU=$0.56$}
			\label{fig:RMS_threshold_L3_S4_B_saeed}
		\end{subfigure}
		\hfill
		\begin{subfigure}[b]{.32\textwidth}
			\centering
			\includegraphics[width=1\textwidth]{figure16i.png}
			\caption{Model-II, IoU=$0.52$} 
			\label{fig:RMS_threshold_L3_S4_B_ijjeh}
		\end{subfigure}
		\caption{Thresholded RMS images.}
		\label{fig:RMS_threshold_L3_S4_B__images}
	\end{figure} 
	%%%%%%%%%%%%%%%%%%%%%%%%%%%%%%%%%%%%%%%%%%%%%%%%%%%%%%%%%%%%%%%%%%%%%%%%%%%	
	
	Table~\ref{tab:multiple_cases} presents the evaluation metrics for Model-I and Model-II, receptively, regarding the experimental case of multiple delamination shown in Fig.~\ref{fig:plate_delam_arrangment}.
	As shown in Table~\ref{tab:multiple_cases}, the actual areas \(A\) and predicted areas \(\hat{A}\) of delaminations were computed in [mm\textsuperscript{2}] with respect to each case. 
	The percentage area error \(\epsilon\) was calculated for both models accordingly.
	
	\begin{table}[ht]
		\centering
		\setlength{\tabcolsep}{4pt} %% default is 6pt
		\caption{Evaluation metrics for experimental case of single delamination.}
		\begin{tabular}{cccccccc}
			\toprule
			\multirow{2}{*}{\begin{tabular}[c]{@{}c@{}}Specimen \\ \end{tabular}} & \multirow{2}{*}{\(A\) [mm\textsuperscript{2}]} & \multicolumn{3}{c}{Model-I} & \multicolumn{3}{c}{Model-II}  \\ 
			\cmidrule(lr){3-5} \cmidrule(lr){6-8} &  & \multicolumn{1}{c}{IoU} & \multicolumn{1}{c}{\(\hat{A}\) [mm\textsuperscript{2}] } & \(\epsilon\) & \multicolumn{1}{c}{IoU}  &\multicolumn{1}{c}{\(\hat{A}\) [mm\textsuperscript{2}]} & \(\epsilon\) \\ 
			\midrule
			II & \multirow{3}{*}{\begin{tabular}[c]{@{}c@{}}472 \\ \end{tabular}} & \multicolumn{1}{c}{0.0} &  \multicolumn{1}{c}{0} & 100\%    & \multicolumn{1}{c}{0.53} & \multicolumn{1}{c}{344} & 27.12\%    \\
			III &  & \multicolumn{1}{c}{0.01} &  \multicolumn{1}{c}{6} & 98.73\%    & \multicolumn{1}{c}{0.64} & \multicolumn{1}{c}{464} & 1.69\%    \\
			IV &  & \multicolumn{1}{c}{0.56} &  \multicolumn{1}{c}{270} & 42.8\%    & \multicolumn{1}{c}{0.52} & \multicolumn{1}{c}{706} & 49.58\%    \\
			\bottomrule
		\end{tabular}
		\label{tab:multiple_cases}
	\end{table}

	Additionally, in Fig.~\ref{fig:comparison_all_models}, we present the predicted outputs for the experimental case of multiple delaminations (Specimen~III) regarding the adaptive wavenumber filtering presented in~\cite{Kudela2015,Radzienski2019a}, the Res-UNet model presented in~\cite{Ijjeh2022}, which is RMS-based (one-to-one), and the Model-II developed in this work, which uses animations of Lamb wave propagation (many-to-one).
	It should be mentioned that, in the case of multiple delaminations, we have tested all the developed models presented in~\cite{Ijjeh2022}.
	However, only the Res-UNet model could identify some damage, and the results were poor.
	
	Figure~\ref{fig:ERMSF} shows the damage map that represents the energy-based root mean square index of frames filtered in the wavenumber domain.
	To eliminate low values in the damage map shown in~\ref{fig:ERMSF}, a binary threshold is applied as shown in Fig.~\ref{fig:binary_ERMSF} and the IoU\(=0.04\).
	Figure~\ref{fig:Res_UNet} shows the predicted output of the previously developed model Res-UNet in~\cite{Ijjeh2022} with IoU=$0.03$.
	Consequently, utilising animations of Lamb waves propagation has better outcomes for delamination identification than processing of a single image representing signal energy.
	
	%%%%%%%%%%%%%%%%%%%%%%%%%%%%%%%%%%%%%%%%%%%%%%%%%%%%%%%%%%%%%%%%%%%%%%%%%%%%
	% Adaptive wavenumber filtering vs DL method
	%%%%%%%%%%%%%%%%%%%%%%%%%%%%%%%%%%%%%%%%%%%%%%%%%%%%%%%%%%%%%%%%%%%%%%%%%%%%
	\begin{figure} [!ht]
		%%%%%%%%%%%%%%%%%%%%%%%%%%%%%%%%%%%%%%%%%%%%%%%%%%%%%%%%%%%%%%%%%%%%%%%%
		\begin{subfigure}[b]{.48\textwidth}
			\centering
			\includegraphics[width=1\textwidth]{figure17a.png}
			\caption{Adaptive wavenumber filtering}
			\label{fig:ERMSF}
		\end{subfigure}
		%%%%%%%%%%%%%%%%%%%%%%%%%%%%%%%%%%%%%%%%%%%%%%%%%%%%%%%%%%%%%%%%%%%%%%%%
		\hfill
		%%%%%%%%%%%%%%%%%%%%%%%%%%%%%%%%%%%%%%%%%%%%%%%%%%%%%%%%%%%%%%%%%%%%%%%%
		\begin{subfigure}[b]{.48\textwidth}
			\centering
			\includegraphics[width=1\textwidth]{figure17b.png}
			\caption{Binary output, IoU=$0.04$} 
			\label{fig:binary_ERMSF}
		\end{subfigure}
		%%%%%%%%%%%%%%%%%%%%%%%%%%%%%%%%%%%%%%%%%%%%%%%%%%%%%%%%%%%%%%%%%%%%%%%%
		\hfill
		%%%%%%%%%%%%%%%%%%%%%%%%%%%%%%%%%%%%%%%%%%%%%%%%%%%%%%%%%%%%%%%%%%%%%%%%
		\begin{subfigure}[b]{.48\textwidth}
			\centering
			\includegraphics[width=1\textwidth]{figure17c.png}
			\caption{RMS based Res-UNet, IoU=$0.03$} 
			\label{fig:Res_UNet}
		\end{subfigure}
		%%%%%%%%%%%%%%%%%%%%%%%%%%%%%%%%%%%%%%%%%%%%%%%%%%%%%%%%%%%%%%%%%%%%%%%%
		%%%%%%%%%%%%%%%%%%%%%%%%%%%%%%%%%%%%%%%%%%%%%%%%%%%%%%%%%%%%%%%%%%%%%%%%
		\hfill
		%%%%%%%%%%%%%%%%%%%%%%%%%%%%%%%%%%%%%%%%%%%%%%%%%%%%%%%%%%%%%%%%%%%%%%%%
		\begin{subfigure}[b]{.48\textwidth}
			\centering
			\includegraphics[width=1\textwidth]{figure17d.png}
			\caption{Model-II, IoU=\(0.64\)} 
			\label{fig:RMS_threshold_L3_S4_B_ijjeh_compare}
		\end{subfigure}
		%%%%%%%%%%%%%%%%%%%%%%%%%%%%%%%%%%%%%%%%%%%%%%%%%%%%%%%%%%%%%%%%%%%%%%%%
		\caption{Damage scenario of multiple delaminations regarding adaptive wavenumber filtering method, Res-UNet model (RMS-based), and currently developed Model-II.}
		\label{fig:comparison_all_models}
	\end{figure} 
	%%%%%%%%%%%%%%%%%%%%%%%%%%%%%%%%%%%%%%%%%%%%%%%%%%%%%%%%%%%%%%%%%%%%%%%%%%%%
\end{sloppypar}
