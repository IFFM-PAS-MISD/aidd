%%%%%%%%%%%%%%%%%%%%%%%%%%%%%%%%%%%%%%%%%%%%%%%%%%%%%%%%%%%%%%%%%%%%%%%%%%%%%%%%
The complex design of composite structures makes it very difficult for conventional visual inspection techniques to detect defects in these materials. 
Therefore, numerous types of nondestructive testing and structural health monitoring techniques have been developed for damage identification in composite structures, among which ultrasonic guided waves, particularly Lamb waves, have gained much popularity.
In recent years, apart from piezoelectric pointwise measurements, laser Doppler vibrometry has been used for full wavefield measurements of propagating Lamb waves. 
Damage imaging can be done with animations of Lamb waves interacting with defects.
In this work, two end-to-end deep learning-based models of many-to-one sequence prediction were developed to perform pixel-wise image segmentation.
A large dataset of full wavefield images resulting from the interaction with delamination of random size, shape, and location was utilised to train the proposed deep learning models.
The main goal of this work is to investigate the feasibility of implementing deep learning methods for delamination identification in composite laminates by only utilising the animations of Lamb waves.
The elaborated models worked well on the numerically generated test data and have also shown good performance on the real-world experimental data.
These methods can enable the automation of delamination identification and to create damage maps without the intervention of the user.