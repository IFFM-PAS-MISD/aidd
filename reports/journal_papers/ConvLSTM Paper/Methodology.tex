\section{Methodology}
\subsection{Dataset}
In this work, wave frames accompanied our previously numerically generated dataset in~\cite{Ijjeh2021} was utilised to train the proposed deep learning models.  
Essentially, the dataset resembles the velocity measurements acquired by SLDV in the transverse direction.
The dataset contains 475 simulated cases representing delaminations with different locations, shapes, and sizes. 
Initially, the guided waves were excited at the centre of the plate by applying equivalent piezoelectric forces.
Further, the carrier frequency is assumed \(50\) kHz, and the modulation frequency is \(10\) kHz.
Moreover, every single simulated case contains \(512\) frames depicting the full wavefield propagation of Lamb waves at the bottom surface of a CFRP plate representing its interaction and reflection from the delamination and boundaries.
Figure~\ref{fig:Full_wave} shows a sample number of frames at different time-steps of the propagated Lamb waves before and after the interaction with the delamination.
The frame \(f_{0}\) depicts the initial interactions with the delamination which was calculated using the delamination location and the velocity of \(A0\) Lamb wave mode.
\begin{figure}[!h]
	\centering
	\includegraphics[width=1\textwidth]{full_wavefield_updated.png}
	\caption{Sample frames of full wave propagation.}
	\label{fig:Full_wave}
\end{figure}

As mentioned earlier, the dataset contains \(475\) different cases, with a \(512\) frames per each case producing a total number of 243,\,200~frames, with a frame size of \((500\times500)\)~pixels.
Thus, utilising all frames in each case has high computational and memory costs.

Furthermore, the dataset was divided into two sets training and testing of a ratio \(80\%\) and \(20\% \) respectively.
Moreover, a certain portion of the training set was preserved as a validation set to validate the model during the training process.
Additionally, the dataset was normalised to a range of \((0, 1)\).
To train our proposed models, two training sets were prepared and tailored to two models described in section~\ref{proposed_approach}.
We selected \(64\) and \(24\) consecutive frames in each delamination case regarding the first and the second training sets, respectively.
Frames displaying the propagation of guided waves before interacting with the delamination have no features to be extracted.  
Hence, only a certain number of frames was selected from the initial occurrence of the interactions with the delamination (see Fig.~\ref{fig:Full_wave} for details).

Additionally, for the second training set, we have resized the frames to \(512\times512\)~pixels to maintain the symmetrical shape during the encoding and decoding process.
Further, the validation sets have portions of \(10\%\) and \(20\%\) regarding the first and second training sets, respectively.