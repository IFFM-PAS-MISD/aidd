\section{Dataset computation and preprocessing}
\subsection{Dataset computation}
%%%%%%%%%%%%%%%%%%%%%%%%%%%%%%%%%%%%%%%%%%%%%%%%%%%%%%%%%%%%%%%%%%%%%%%%%%%%%%%%%%%%%%%%
In this work, a synthetic dataset of propagating waves in carbon fibre reinforced composite plates was computed by using the parallel implementation of the time domain spectral element method~\cite{Kudela2020}. 
The dataset is a 3D matrix representing the particle velocities at uniform grid on the bottom surface of the plate over time as a response to piezoelectric actuation located at the centre.
The out-of-plane particle velocities are considered only.
In other words, the dataset is in the form of animation similar to the one which can be obtained by using one head of SLDV.
The excitation signal applied to piezoelectric transducer was a five-cycle Hann window-modulated sinusoidal tone burst of carrier frequency 50 kHz. 
The total time of the wave propagation was 0.75 ms.
The time integration step was selected considering the stability of the central difference scheme and was equal to 5 ns.

The CFRP stacking sequence assumed in the model was [0/90]\(_4\). 
The properties of a single layer were as follows [GPa]:
\(C_{11} = 52.55, \, C_{12} = 6.51, \, C_{22} = 51.83, C_{44} = 2.93, C_{55} = 
2.92, C_{66} = 3.81\),
whereas the mass density was 1522.4 kg/m\textsuperscript{3}.
The mechanical properties of the modelled CFRP were chosen to resemble properties of CFRP specimens used for testing.
Nevertheless, the wave propagation behaviour simulated numerically was slightly different than the one measured experimentally which exhibits in the wavelength of the A0 mode, namely 21.2 mm for numerical simulations and 19.5 mm for SLDV measurements.

Numerical simulations covered 475 cases of delamination with random parameters. 
Each simulation was performed for a mesh with a single delamination of the following random 
The following random factors were used in simulated delamination scenarios:
\begin{itemize}
	\item coordinates of the centre of delamination,
	\item delamination geometrical size	determined by ellipse minor and major axis randomly selected from the range $10-40$ mm,
	\item delamination angle randomly selected from the range $ 0^{\circ}-180^{\circ}$.
	
\end{itemize}
The delamination modelling was realized by writing custom geometry files which were used to generate unstructured mesh consisting of quadrilateral elements by using gmsh software~\cite{Geuzaine2009}.
An exemplary mesh of quadrilateral elements is shown in Fig.~\ref{fig:random_delam_mesh} in which green elements highlight the delamination whereas red elements represent the location of the piezoelectric actuator.
Next, the mesh was modified by doubling elements and splitting nodes at delamination region.
Additionally, the quadrilateral elements were converted to the 36-node spectral elements by using a custom MATLAB script.
The wave propagation problem was solved by using in-house code of the time domain SEM which was run on GPU.
The computation took about 3 hours and 20 minutes for each case.
%%%%%%%%%%%%%%%%%%%%%%%%%%%%%%%%%%%%%%%%%%%%%%%%%%
\begin{figure} [h!]
	\begin{center}
		\includegraphics{figure2.png}
	\end{center}
	\caption{Exemplary mesh containing piezoelectric transducer (red) and random delamination (green) used for Lamb wave propagation modelling.} 
	\label{fig:random_delam_mesh}
\end{figure}
%%%%%%%%%%%%%%%%%%%%%%%%%%%%%%%%%%%%%%%%%%%%%%%%%%
The dataset contains 475 different delamination cases, with 512 frames per case, giving a total number of 243,\,200 frames. 
The single frame representing the wave pattern on the surface of the plate of dimensions \((500\times500)\)~mm\(^{2}\) consisted of \((500\times500)\)~pixels.
The frames in the dataset are 8-bit .png greyscale images.
The spatial size of the wavefield was further downsampled to \((256\times256)\) for the purpose of 
reducing the computational complexity.

It is important to note that input data to the DL model in the form of the binary image representing the respective delamination case as it is presented in Fig.~\ref{fig:complete_flowchart} was insufficient to train a reliable model.
It was necessary to provide additional inputs in the form of reference full wavefield frames (without delaminations).
This is explained along with the DL model in section~\ref{sec:proposed_approach}.
%%%%%%%%%%%%%%%%%%%%%%%%%%%%%%%%%%%%%%%%%%%%%%%%%%%%%%%%%%%%%%%%%%%%%%%%%%%%%%%%
\subsection{Data augmentation}
The best way for improving the generalisation capabilities of the neural network is to acquire more data. 
However, in practice, it can be difficult to acquire more data and the amount of available data for neural network training is limited.
For tackling this issue, one way is to create some fake data based on the original dataset and add it to the training set, which is termed data augmentation. 
Data augmentation is an efficient approach for various computer vision and DL tasks. 
Data augmentation includes randomly cropping a region from the original image, adjusting contrast, rotation for a small angle and flipping images, etc.~\cite{szegedy2015going}.
In this research work, the dataset is composed of 475 delamination cases which is not enough for a targeted DL model to perform well.
Therefore, all the images in the 475 delamination cases are flipped diagonally, horizontally, and vertically in order to enhance the performance of the proposed DL model. 
Therefore, the total dataset after data augmentation is now composed of 1900 \((475\times4 = 1900)\) delamination cases.
%%%%%%%%%%%%%%%%%%%%%%%%%%%%%%%%%%%%%%%%%%%%%%%%%%%%%%%%%%%%%%%%%%%%%%%%%%%%%%%%
\subsection{Dataset division and preprocessing}
For training and evaluation of the proposed DL model, the dataset was divided into two sets: training and testing, with a ratio of \(80\%\) and \(20\%\), respectively.
Moreover, \(20\%\) of the training set was preserved as a validation set to validate the model during the training process.

Moreover, the dataset was normalised to a range of \((0, 1)\) to improve the convergence of the gradient descent algorithm.
Due to memory limitations, \(32\) consecutive frames in each delamination case were selected for DL model training.
Additionally, frames displaying the propagation of guided waves before interaction with the delamination have no features to be extracted.
Hence, only a certain number of frames were selected from the initial occurrence of the interactions with the delamination.
%%%%%%%%%%%%%%%%%%%%%%%%%%%%%%%%%%%%%%%%%%%%%%%%%%%

