\documentclass[11pt,a4paper]{article}
	\title{Delamination detection by using Pyramid Scene Parsing Network on Guided waves}
	%\address[IFFM]{Institute of Fluid Flow Machinery, Polish Academy of Sciences, Poland}
	\author{Saeed Ullah}
%	\author{Pawel Kudela\corref{cor1}\fnref{IFFM}}
%	\ead{pk@imp.gda.pl}

\begin{document}
	\maketitle
In the last few decades, composite materials are increasingly being used in multiple industries such as aerospace, wind turbine, marine, automobile, etc. This extensive usage of composite materials is due to its remarkable benefits such as higher stiffness-to-mass ratio compared to metals, high strength, lightweight, better corrosion resistance. However, these materials are more susceptible to impact damage in the form of cracks, debondings, delaminations, and degradation of material properties due to environmental conditions, etc. These problems could reduce load-bearing capability and potentially lead to structural failure. Even a low-intensity and a low-velocity impact can induct cracks which ultimately may lead to delamination. Delamination can progress and affect the mechanical properties and structural integrity which then may commence catastrophic failure of the whole structure. Delamination diminishes the life of these structures. Delamination is one of the most common types and vulnerable defects in composite structures. Delamination is usually invisible which is why it is severely dangerous. Therefore, it is essential to monitor the performance of the composite structures continuously for safety and damage detection at an early stage. Reliable damage detection is essential for the use of composites in different applications.
Due to complexity in the design of composites, the delamination is challenging to detect because it usually occurs between plies of composite laminate and is invisible from outside surfaces. This causes that it is difficult to detect it by traditional visual inspection techniques.
Numerous NDT/SHM procedures have been proposed for the detection of delamination in composite structures. Among the various SHM systems, ultrasonic guided wave propagation-based SHM is broadly recognized as one of the most promising mechanisms for quantitative identification of delamination in composite materials. Ultrasonic guided waves emerged as a reliable way to locate delamination in these structures. These techniques produce useful information about the presence, type, location, size, and extent of delamination in composite structures.
Recently, Scanning Laser Doppler Vibrometry (SLDV) is widely being used for measuring guided waves on a very dense grid of points over the surface of a large specimen (full wavefield). Such wavefield contains rich information about the interaction of guided waves with potential defects. However, these full wavefields are complex. Analyzing such wavefields are very difficult for conventional physics or classical machine learning-based models. Deep Learning techniques have shown quite better performance with handling such nonlinear and complex data in various domains. To the best of the authors' knowledge, no such study exists which emphasized deep learning-based segmentation techniques for damage detection on full wavefields analysis in composite structures.
In this research work, we are using a deep learning-based segmentation technique named Pyramid Scene Parsing Network (PSPNet) for detection, localization, and estimation of the size of delamination in composite structures. PSPNet is a state-of-the-art deep learning model for segmentation. PSPNet was proposed basically for understanding the objects according to the real-world scenarios. Recently, PSPNet based model has won various awards in different object and scene detection/recognition competitions. PSPNet utilizes Convolutional Neural Networks (CNN) in their hidden layers in the architecture for the final predictions. CNN is employed for the feature mapping and pixel-wise convolution in PSPNet.
In this work, the PSPNet model is applied to high resolution based images acquired from numerical simulations that resemble SLDV. The proposed segmentation model is capable of providing good results for the delamination detection and localization as well as estimation of the size of the delamination in a composite structure.
\end{document}