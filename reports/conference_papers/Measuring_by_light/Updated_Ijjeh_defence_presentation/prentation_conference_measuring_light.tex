%\PassOptionsToPackage{draft}{graphicx}
\documentclass[10pt,aspectratio=169,dvipsnames]{beamer} % , notes=only aspect ratio 16:9
%\graphicspath{{../../figures/}}

%\includeonlyframes{frame1,frame2,frame3}

%%%%%%%%%%%%%%%%%%%%%%%%%%%%%%%%%%%%%%%%%%%%%%%%%%
% Packages
%%%%%%%%%%%%%%%%%%%%%%%%%%%%%%%%%%%%%%%%%%%%%%%%%%
\usepackage{appendixnumberbeamer}
\usepackage{booktabs}
\usepackage{csvsimple} % for csv read
\usepackage[scale=2]{ccicons}
\usepackage{pgfplots}
\usepackage{xspace}
%\usepackage{amsmath}
\usepackage{totcount}
\usepackage{tikz}
\usepackage{bm}
\usepackage{float}
\usepackage{eso-pic} 
\usepackage{wrapfig}
\usepackage{animate,media9}
\usepackage{subfig}
\usepackage{fancybox}
%\usepackage{multimedia}
\usepackage{dashbox}
\usepackage{tcolorbox}
\usepackage{multicol}
\usepackage{multirow}
\usepackage{xcolor}
\usepackage[document]{ragged2e}
\usepackage[labelformat=empty]{caption}
\usepackage{comment}
\usepackage{mathtools}% Loads amsmath
\usepackage{efbox,graphicx}
\usepackage{pgfpages}
%\setbeameroption{show notes}
\captionsetup[figure]{labelformat=empty}%

\efboxsetup{linecolor=Lightskyblue,linewidth=2pt}
%%%%%%%%%%%%%%%%%%%%%%%%%%%%%%%%%%%%%%%%%%%%%%%%%%%%%%%%%%%%%%%%%%%%%%%%%%%%%%%%
\makeatletter
\renewcommand\tcbtitle{\ifx\tcbtitletext\@empty\else%
	{\kvtcb@fonttitle\kvtcb@haligntitle\kvtcb@before@title
		\leavevmode\color{tcbcol@title}\tcbtitletext\kvtcb@after@title}\fi}
\makeatother
%%%%%%%%%%%%%%%%%%%%%%%%%%%%%%%%%%%%%%%%%%%%%%%%%%%%%%%%%%%%%%%%%%%%%%%%%%%%%%%%
%\usepackage[font=footnotesize,labelfont=bf]{caption}
%\captionsetup[figure]{font=small}

%\usepackage[export]{adjustbox}
%\usepackage{background}
%\backgroundsetup{contents=preliminary,placement=bottom,color=blue}
%\usepackage{FiraSans}

%\usepackage{comment}
%\usetikzlibrary{external} % speedup compilation
%\tikzexternalize % activate!
%\usetikzlibrary{shapes,arrows} 

%\usepackage{bibentry}
%\nobibliography*
\usepackage{ifthen}
\newcounter{angle}
\setcounter{angle}{0}
%\usepackage{bibentry}
%\nobibliography*
\usepackage{caption}%


%\usepackage{etoolbox}
%\apptocmd{\frame}{}{\justifying}{} 

\graphicspath{{figures/}}

%%%%%%%%%%%%%%%%%%%%%%%%%%%%%%%%%%%%%%%%%%%%%%%%%%
% Metropolis theme custom modification file
%%%%%%%%%%%%%%%%%%%%%%%%%%%%%%%%%%%%%%%%%%%%%%%%%%
% Metropolis theme custom modification file
%%%%%%%%%%%%%%%%%%%%%%%%%%%%%%%%%%%%%%%%%%%%%%%%%%
% Metropolis theme custom colors
%%%%%%%%%%%%%%%%%%%%%%%%%%%%%%%%%%%%%%%%%%%%%%%%%%
\usetheme[progressbar=foot]{metropolis}
\useoutertheme{metropolis}
\useinnertheme{metropolis}
\usefonttheme{metropolis}
\setbeamercolor{background canvas}{bg=white}

%\usecolortheme{spruce}

\definecolor{myblue}{rgb}{0.19,0.55,0.91}
\definecolor{mediumblue}{rgb}{0,0,205}
\definecolor{darkblue}{rgb}{0,0,139}
\definecolor{Dodgerblue}{HTML}{1E90FF}
\definecolor{Navy}{HTML}{000080} % {rgb}{0,0,128}
\definecolor{Aliceblue}{HTML}{F0F8FF}
\definecolor{Lightskyblue}{HTML}{87CEFA}
\definecolor{logoblue}{RGB}{1,67,140}
\definecolor{Purple}{HTML}{911146}
\definecolor{Orange}{HTML}{CF4A30}

\setbeamercolor{progress bar}{bg=Lightskyblue}
\setbeamercolor{progress bar}{ fg=logoblue} 
\setbeamercolor{frametitle}{bg=logoblue}
\setbeamercolor{title separator}{fg=logoblue}
\setbeamercolor{block title}{bg=Lightskyblue!30,fg=black}
\setbeamercolor{block body}{bg=Lightskyblue!15,fg=black}
\setbeamercolor{alerted text}{fg=Purple}
% notes colors
\setbeamercolor{note page}{bg=white}
\setbeamercolor{note title}{bg=Lightskyblue}
%%%%%%%%%%%%%%%%%%%%%%%%%%%%%%%%%%%%%%%%%%%%%%%%%%
%  Theme modifications
%%%%%%%%%%%%%%%%%%%%%%%%%%%%%%%%%%%%%%%%%%%%%%%%%%
% modify progress bar linewidth
\makeatletter
\setlength{\metropolis@progressinheadfoot@linewidth}{2pt} 
\setlength{\metropolis@titleseparator@linewidth}{1pt}
\setlength{\metropolis@progressonsectionpage@linewidth}{1pt}

\setbeamertemplate{progress bar in section page}{
	\setlength{\metropolis@progressonsectionpage}{%
		\textwidth * \ratio{\thesection pt}{\totvalue{totalsection} pt}%
	}%
	\begin{tikzpicture}
		\fill[bg] (0,0) rectangle (\textwidth, 
		\metropolis@progressonsectionpage@linewidth);
		\fill[fg] (0,0) rectangle (\metropolis@progressonsectionpage, 
		\metropolis@progressonsectionpage@linewidth);
	\end{tikzpicture}%
}
\makeatother
\newcounter{totalsection}
\regtotcounter{totalsection}

\AtBeginDocument{%
	\pretocmd{\section}{\refstepcounter{totalsection}}{\typeout{Yes, prepending 
	was successful}}{\typeout{No, prepending was not successful}}%
}%
%%%%%%%%%%%%%%%%%%%%%%%%%%%%%%%%%%%%%%%%%%%%%%%%%%
%  Bibliography mods
%%%%%%%%%%%%%%%%%%%%%%%%%%%%%%%%%%%%%%%%%%%%%%%%%%
\setbeamertemplate{bibliography item}{\insertbiblabel} %% Remove book symbol 
%%from references and add number in square brackets
% kill the abominable icon (without number)
%\setbeamertemplate{bibliography item}{}
%\makeatletter
%\renewcommand\@biblabel[1]{#1.} % number only
%\makeatother
% remove line breaks in bibliography
\setbeamertemplate{bibliography entry title}{}
\setbeamertemplate{bibliography entry location}{}
%%%%%%%%%%%%%%%%%%%%%%%%%%%%%%%%%%%%%%%%%%%%%%%%%%
%  Bibliography custom commands
%%%%%%%%%%%%%%%%%%%%%%%%%%%%%%%%%%%%%%%%%%%%%%%%%%
\newcommand{\bibliotitlestyle}[1]{\textbf{#1}\par}

\newif\ifinbiblio
\newcounter{bibkey}
\newenvironment{biblio}[2][long]{%
	%\setbeamertemplate{bibliography item}{\insertbiblabel}
	\setbeamertemplate{bibliography item}{}% without numbers
	\setbeamerfont{bibliography item}{size=\footnotesize}
	\setbeamerfont{bibliography entry author}{size=\footnotesize}
	\setbeamerfont{bibliography entry title}{size=\footnotesize}
	\setbeamerfont{bibliography entry location}{size=\footnotesize}
	\setbeamerfont{bibliography entry note}{size=\footnotesize}
	\ifx!#2!\else%
	\bibliotitlestyle{#2}%
	\fi%
	\begin{thebibliography}{}%
		\inbibliotrue%
		\setbeamertemplate{bibliography entry title}[#1]%
	}{%
		\inbibliofalse%
		\setbeamertemplate{bibliography item}{}%
	\end{thebibliography}%
}

\newcommand{\biblioref}[5][short]{
	\setbeamertemplate{bibliography entry title}[#1]
	\stepcounter{bibkey}%
	\ifinbiblio%
	\bibitem{\thebibkey}%
	#2
	\newblock #4
	\ifx!#5!\else\newblock {\em #5}, #3 \fi%
	\else%
	\begin{biblio}{}
		\bibitem{\thebibkey}
		#2
		\newblock #4
		\ifx!#5!\else\newblock {\em #5}, #3\fi
	\end{biblio}
	\fi
}
%
%\newbibmacro*{hypercite}{%
%	\renewcommand{\@makefntext}[1]{\noindent\normalfont##1}%
%	\footnotetext{%
%		\blxmkbibnote{foot}{%
%			\printtext[labelnumberwidth]{%
%				\printfield{prefixnumber}%
%				\printfield{labelnumber}}%
%			\addspace
%			\fullcite{\thefield{entrykey}}}}}
%
%\DeclareCiteCommand{\hypercite}%
%{\usebibmacro{cite:init}}
%{\usebibmacro{hypercite}}
%{}
%{\usebibmacro{cite:dump}}
%
%% Redefine the \footfullcite command to use the reference number
%\renewcommand{\footfullcite}[1]{\cite{#1}\hypercite{#1}}
%\usefonttheme[onlymath]{Serif} % It should be uncommented if Fira fonts in 
%%math does not work

%%%%%%%%%%%%%%%%%%%%%%%%%%%%%%%%%%%%%%%%%%%%%%%%%%
% Custom commands
%%%%%%%%%%%%%%%%%%%%%%%%%%%%%%%%%%%%%%%%%%%%%%%%%%
% matrix command 
\newcommand{\matr}[1]{\mathbf{#1}} % bold upright (Elsevier, Springer)
%\newcommand{\matr}[1]{#1} % pure math version
%\newcommand{\matr}[1]{\bm{#1}} % ISO complying version
% vector command 
\newcommand{\vect}[1]{\mathbf{#1}} % bold upright (Elsevier, Springer)
% bold symbol
\newcommand{\bs}[1]{\boldsymbol{#1}}
% derivative upright command
\DeclareRobustCommand*{\drv}{\mathop{}\!\mathrm{d}}
\newcommand{\ud}{\mathrm{d}}
% 
\newcommand{\themename}{\textbf{\textsc{metropolis}}\xspace}

\def\checkmark{\tikz\fill[scale=0.4](0,.35) -- (.25,0) -- (1,.7) -- (.25,.15) -- cycle;} 

%
%\setbeameroption{show notes on second screen}
%\setbeamertemplate{note page}{\insertnote}
%%%%%%%%%%%%%%%%%%%%%%%%%%%%%%%%%%%%%%%%%%%%%%%%%%
% Title page options
%%%%%%%%%%%%%%%%%%%%%%%%%%%%%%%%%%%%%%%%%%%%%%%%%%
\date{28 - 30 MARCH 2023}
%\date{}
%%%%%%%%%%%%%%%%%%%%%%%%%%%%%%%%%%%%%%%%%%%%%%%%%%
% option 1
%%%%%%%%%%%%%%%%%%%%%%%%%%%%%%%%%%%%%%%%%%%%%%%%%%%
\justifying\title{{Deep learning aided laser Doppler vibrometry for delamination identification in composite laminates}}
\subtitle{{Measuring by light \\ Delft, Netherlands}}
\author{\textbf{D.Sc. Ph.D. Eng. Paweł Kudela} 
	\and \\ \textbf{Ph.D. candidate, Eng. Abdalraheem A. Ijjeh}} 
% logo align to Institute 

\institute{
	Institute of Fluid Flow Machinery \\ 
	Polish Academy of Sciences \\ 
	Gdansk, Poland \\
	\vspace{-1.5cm}
	\flushright 
	\includegraphics[width=6cm]{imp_logo.png}}

%%%%%%%%%%%%%%%%%%%%%%%%%%%%%%%%%%%%%%%%%%%%%%%%%%
%\tikzexternalize % activate!
%%%%%%%%%%%%%%%%%%%%%%%%%%%%%%%%%%%%%%%%%%%%%%%%%%
\setbeamertemplate{section in toc}[sections numbered]
\setbeamertemplate{subsection in toc}[subsections numbered]

\begin{document}
	%%%%%%%%%%%%%%%%%%%%%%%%%%%%%%%%%%%%%%%%%%%%%%%%%%
	\maketitle
	%%%%%%%%%%%%%%%%%%%%%%%%%%%%%%%%%%%%%%%%%%%%%%%%%%%%%%%%%%%%%%%%%%%%
	% SLIDES
	%%%%%%%%%%%%%%%%%%%%%%%%%%%%%%%%%%%%%%%%%%%%%%%%%%%%%%%%%%%%%%%%%%%%
	\begin{frame}[label=frame1]{Outlines}
		\setbeamertemplate{section in toc}[sections numbered]
		\setbeamertemplate{subsection in toc}[subsections numbered]
		\tableofcontents
	\end{frame}	
	%%%%%%%%%%%%%%%%%%%%%%%%%%%%%%%%%%%%%%%%%%%%%%%%%%%%%%%%%%%%%%%%%%%%
	\section{Introduction}
	\begin{frame}{Motivations}
		\begin{columns}[T]
			\begin{column}[t]{.55\textwidth}
				\begin{figure}[t]
					%					\centering
					%					\subfloat{\includegraphics[width=.7\textwidth]{Composite_advantages.png}}					
					%					\\
					\subfloat{\includegraphics[width=.85\textwidth]{delaminated_plate1.jpg}}
				\end{figure}
				\begin{tcolorbox}
					\justifying\noindent\alert{Delamination detection} in its early stages can significantly help avoiding catastrophic events.
				\end{tcolorbox}				
			\end{column}
			\begin{column}[t]{0.45\textwidth}
				\begin{figure}[t]
					\includegraphics[width=1\textwidth]{Crashes.png}
				\end{figure}
				\tiny {source: https://www.structuresinsider.com/post/the-difference-between-buckling-compression-shear}
			\end{column}
		\end{columns}
	\end{frame}
	%%%%%%%%%%%%%%%%%%%%%%%%%%%%%%%%%%%%%%%%%%%%%%%%%%%%%%%%%%%%%%%%%%%%
	\note
	{
		Composite laminates have a wide range of applications in various industries due to their good characteristics, such as:
		high strength, low density, light weight, and resistance to fatigue and corrosion, among others.
		
		However, composite laminates may encounter some defects, such as cracks, fibre breakage, and debonding.
		
		In particular, composite laminates are more sensitive to damage in the form of delamination due to weak transverse tensile and interlaminar shear strengths.
		
		Therefore, delaminations can seriously decrease the performance of composite structures.
		Accordingly, delamination detection in its early stages can significantly help to avoid catastrophic structural collapses.
	}
	%%%%%%%%%%%%%%%%%%%%%%%%%%%%%%%%%%%%%%%%%%%%%%%%%%%%%%%%%%%%%%%%%%%%
	\begin{frame}{Objectives}
		\justifying
		\begin{itemize}
			\item {Development of an artificial intelligence (AI) driven diagnostic system for delamination identification in composite laminates such as carbon fibre reinforced polymers (CFRP).}

			\item {Addressing the issue of slow data acquisition by scanning laser Doppler vibrometer (SLDV) of high-resolution full wavefields of Lamb wave propagation.}
		\end{itemize}		
	\end{frame}
	%%%%%%%%%%%%%%%%%%%%%%%%%%%%%%%%%%%%%%%%%%%%%%%%%%%%%%%%%%%%%%%%%%%%
	\begin{frame}{General workflow}
		\begin{columns}[T]
			\begin{column}[c]{0.9\textwidth}
				\begin{figure}
					\centering
					\includegraphics[height=.85\textheight]{full_procedure.png}	
				\end{figure}		
			\end{column}
		\end{columns}		
	\end{frame}
	%%%%%%%%%%%%%%%%%%%%%%%%%%%%%%%%%%%%%%%%%%%%%%%%%%%%%%%%%%%%%%%%%%%%
	\section*{Experimental measurements}
	%%%%%%%%%%%%%%%%%%%%%%%%%%%%%%%%%%%%%%%%%%%%%%%%%%%%%%%%%%%%%%%%%%%%
	\begin{frame}[t]{SLDV measurements: Setup}
		\begin{columns}[T]
			\column{0.5\textwidth}
			\begin{figure}
				\includegraphics[width=0.8\textwidth]{wibrometr-laserowy-1d_small-description.png}
			\end{figure}
			\column{0.5\textwidth}
			\begin{enumerate}
				\item Signal generator: TTI 1241 
				\item Amplifier: Piezo Systems EPA-104-230 $\pm$200 Vp
				\item Specimen
				\item Scanning head: Polytec PSV-400
				\item DAQ system: Polytec
			\end{enumerate}
		\end{columns}
		{\small
			Measurements were taken on a uniform grid of \textbf{333$\times$333 points}.\\
			Excitation in the form of Hann windowed sine signal of carrier frequency \textbf{50 kHz} was applied to piezoelectric transducer.}
		a	\end{frame}
	%%%%%%%%%%%%%%%%%%%%%%%%%%%%%%%%%%%%%%%%%%%%%%%%%%%%%%%%%%%%%%%%%%%%
	\note
	{
		In this slide, I present the experimental setup for data acquisition.
		
		The specimen was excited using a Piezoelectric transducer placed at the center of the plate.
		
		The scanning head is used to acquire the full wavefields of guided wave propagation from the bottom surface of the plate.	
	}
	%%%%%%%%%%%%%%%%%%%%%%%%%%%%%%%%%%%%%%%%%%%%%%%%%%%%%%%%%%%%%%%%%%%%
	\begin{frame}[t]{Composite specimen}
		\begin{columns}[T]
			\begin{column}[c]{0.6\textwidth}
				\begin{itemize}
					\item 16 layers set at the same angle 
					\item carbon: Prepreg GG 205 P (fibres Toray FT 300 - 3K 200 tex), $E=230$ GPa
					\item epoxy resin: IMP503Z-HT by Impregnatex Compositi 						
					\item density: 1522.4~kg/m\textsuperscript{3}
					\item dimensions: 500$\times$500$\times$3.9 mm
				\end{itemize}			
			\end{column}
			\begin{column}[c]{0.4\textwidth}
				\begin{figure}
					\centering
					\includegraphics[width=.6\textwidth]{weave-1.jpg}
					\caption{Plain weave fabric}
				\end{figure}
			\end{column}			
		\end{columns}
	\end{frame}
	%%%%%%%%%%%%%%%%%%%%%%%%%%%%%%%%%%%%%%%%%%%%%%%%%%%%%%%%%%%%%%%%%%%%
	\note{
		The specimen used for experimental evaluation are carbon fibre reinforcement polymer (CFRP) composed of 16 layers of plain weave fabric with the following characteristics:
		
		The specimen dimension is \(500 \times 500\)~mm with a thickness of 3.9~mm.
	}
	%%%%%%%%%%%%%%%%%%%%%%%%%%%%%%%%%%%%%%%%%%%%%%%%%%%%%%%%%%%%%%%%%%%%
	\begin{frame}[t]{Specimens with defects}
		\vspace{-0.5cm}
		\begin{columns}[T]
			\column{0.5\textwidth}
			\begin{figure}
				\includegraphics[width=0.9\textwidth]{plate_multi_delam_arrangement_large_fonts.png}
			\end{figure}
			\column{0.5\textwidth}
			\begin{figure}
				\includegraphics[width=0.75\textwidth]{plate_single_delam_arrangement_large_fonts.png}
			\end{figure}
		\end{columns}
	\end{frame}
	%%%%%%%%%%%%%%%%%%%%%%%%%%%%%%%%%%%%%%%%%%%%%%%%%%%%%%%%%%%%%%%%%%%%
	\note
	{	
		\footnotesize
		The specifications of the CFRP specimens used for evaluating the developed deep-learning models are shown in this slide.
		
		Specimens with multiple delaminations are presented on the left.
		In which there are three specimens with multiple delaminations located at the same distance from the centre of the plate, and placed at different thicknesses. 
		
		The figure on the right shows the specimen with a single delamination placed at the half thickness, and the total thickness of the specimen is 3.5 mm.
		
		Furthermore, all measurements with SLDV were conducted from the bottom surface of the plate.		
	}	
	%%%%%%%%%%%%%%%%%%%%%%%%%%%%%%%%%%%%%%%%%%%%%%%%%%%%%%%%%%%%%%%%%%%%
	\section*{Synthetic dataset generation}
	\setcounter{subfigure}{0}	%%%%%%%%%%%%%%%%%%%%%%%%%%%%%%%%%%%%%%%%%%%%%%%%%%%%%%%%%%%%%%%%%%%%
	\begin{frame}{Synthetic dataset generation}
		\begin{columns}[T]			
			\begin{column}{0.55\textwidth}
				\justifying
				\begin{itemize}
					\item Mindlin-Reissner plate theory
					\item Spectral element method (SEM)
					\item Splitting elements and nodes at delamination
					\item GMSH software was used for meshing quads then converted to spectral elements
				\end{itemize}	
				\begin{figure}
					\subfloat{\includegraphics[width=0.8\textwidth]{shell.png}}	
				\end{figure}
			\end{column}
			\begin{column}{0.45\textwidth}	
				\begin{figure}
					\animategraphics[controls,autoplay,loop,width=0.9\textwidth]{1}{/gif_figs/mesh/m1_rand_single_delam_}{1}{20}
				\end{figure}	
			\end{column}
		\end{columns}	
	\end{frame}
	%%%%%%%%%%%%%%%%%%%%%%%%%%%%%%%%%%%%%%%%%%%%%%%%%%%%%%%%%%%%%%%%%%%%
	\note{
		\footnotesize
		Now, to train my deep-learning models, I need a large data set of samples with different delamination sizes, locations, and shapes interacting with guided waves.
		And, I need to obtain the full wavefields of the guided wave propagation using the scanning laser Doppler vibrometer. 
		
		However, creating a real dataset with artificially modeled defects is very challenging, expensive, and time-consuming.
		
		Consequently, a large synthetic dataset was generated, resembling the measurements registered by the scanning laser Doppler vibrometer at the bottom surface of the plate as a response to the piezoelectric transducer excitation at the center of the plate..
		
		The synthetic dataset was created using the Mindlin-Reissner theory for thin plate analysis and the spectral element method to approximate the solution of wave propagation and their interaction with delamination.
		
		The animation on the left shows the generated meshes in which the  delamination is modelled in green and the piezoelectric transducer in red for each scenario.
	}
	%%%%%%%%%%%%%%%%%%%%%%%%%%%%%%%%%%%%%%%%%%%%%%%%%%%%%%%%%%%%%%%%%%%%
	\setcounter{subfigure}{0}
	\begin{frame}{Dataset description}
		\begin{columns}[T]
			\begin{column}[t]{0.35\textwidth}
				\justifying
				\begin{itemize}
					\item 475 delamination scenarios
					\item CFRP is made of 8-layers
					\item Delamination modelled between the \(3^{rd}\) and \(4^{th}\) layers
					\item Delamination size min 10~mm, max 40~mm
					\item \textbf{3-months of computing}
				\end{itemize}
			\end{column}
			\begin{column}[t]{0.2\textwidth}
				\begin{figure}[t]
					\centering
					\captionsetup{justification=centering}				
					\subfloat[Delamination placement]{\includegraphics[width=0.95\textwidth]{delamination_placement.png}}
				\end{figure}
			\end{column}
			\begin{column}[t]{0.45\textwidth}
				\begin{figure}[t]					
					\centering	
					\captionsetup{justification=centering}				
					\subfloat[Delamination orientation]{\includegraphics[width=0.95\textwidth]{figure1.png}}					
				\end{figure}
			\end{column}
		\end{columns}
	\end{frame}
	%%%%%%%%%%%%%%%%%%%%%%%%%%%%%%%%%%%%%%%%%%%%%%%%%%%%%%%%%%%%%%%%%%%%
	\note{
		The synthetically generated dataset consists of 475 delamination cases. 
		It was assumed that the composite laminate is made of eight layers with a total thickness of 3.9 mm. 
		The delamination was modeled between the third and fourth layers.
		The delamination's major and minor axes were randomly selected from interval \([10mm,\ 40mm]\). 
		The angle $\aleph$ ranges form \([0^{\circ}-180^{\circ}]\).
		The computation of the dataset took about three months.
	}
	%%%%%%%%%%%%%%%%%%%%%%%%%%%%%%%%%%%%%%%%%%%%%%%%%%%%%%%%%%%%%%%%%%%%
	\setcounter{subfigure}{0}
	\begin{frame}{Training Sample case}
		\begin{columns}[T]
			\begin{column}[c]{.32\textwidth}
				\begin{figure}
					\centering
					\captionsetup{justification=centering}					
					\animategraphics[autoplay, loop,width=0.85 \textwidth]{12}{figures/gif_figs/7_output/tk_flat_shell_Vz_7_500x500bottom-}{1}{512}
					\caption{Full wavefield $s(x,y,t_k)$}
				\end{figure}
			\end{column}
			\begin{column}[c]{.32\textwidth}
				\begin{figure}
					\centering
					\captionsetup{justification=centering}					
					\includegraphics[width=0.85 \textwidth]{RMS_flat_shell_Vz_7_500x500bottom.png}
					\caption{RMS image $\hat{s}(x,y)$}
				\end{figure}
			\end{column}
			\begin{column}[c]{.32\textwidth}
				\begin{figure}
					\centering
					\captionsetup{justification=centering}			
					\includegraphics[width=0.85 \textwidth]{m1_rand_single_delam_7.png}
					\caption{Ground truth (label)}
				\end{figure}
			\end{column}
		\end{columns}
		\begin{equation*}
			\hat{s}(x,y) = \sqrt{\frac{1}{N}\sum_{k=1}^{N}s(x,y,t_k)^2} 
			\label{eqn:rms} 
		\end{equation*}
		\footnotesize
		\(N\) is the number of sampling points equal to 512, \((x,y)\) are the point coordinates on the plate, and \(t_k\) is the time.
	\end{frame}
	%%%%%%%%%%%%%%%%%%%%%%%%%%%%%%%%%%%%%%%%%%%%%%%%%%%%%%%%%%%%%%%%%%%%
	\note{
		Here, I present a single training scenario.
		The animation on the left represents the full wavefield frames of guided wave propagation, where x and y are the point coordinates, and tk is the time moment.
		The output of applying the root mean square formula to the full wavefield is shown in the middle figure.
		The figure on the right is the ground truth label in which the delamination represented in white showing its location, size and shape.
	}
	%%%%%%%%%%%%%%%%%%%%%%%%%%%%%%%%%%%%%%%%%%%%%%%%%%%%%%%%%%%%%%%%%%%%
	\section{Damage imaging approach: An~Overview}	
	%%%%%%%%%%%%%%%%%%%%%%%%%%%%%%%%%%%%%%%%%%%%%%%%%%%%%%%%%%%%%%%%%%%%
	%%%%%%%%%%%%%%%%%%%%%%%%%%%%%%%%%%%%%%%%%%%%%%%%%%%%%%%%%%%%%%%%%%%%
	\setcounter{subfigure}{0}
	%%%%%%%%%%%%%%%%%%%%%%%%%%%%%%%%%%%%%%%%%%%%%%%%%%%%%%%%%%%%%%%%%%%%
	\begin{frame}[t]{Conventional signal processing and wavefield imaging}
		\begin{figure}
			\includegraphics[width=0.7\textwidth]{figs2/sensors_fig1_algorithm.png}
		\end{figure}
		\biblioref{M. Radzienski, P. Kudela, A. Marzani, L. de Marchi, W. Ostachowicz}{2019}{ Damage Identification in Various Types of Composite Plates Using Guided Waves Excited by a Piezoelectric Transducer and Measured by a Laser Vibrometer}{Sensors, 19, 1958}
	\end{frame}
	\setcounter{subfigure}{0}
%	\begin{frame}{ML vs. DL approaches for anomaly detection}
%		Conventional methods involve two processes:
%		\alert{\textbf{Feature extraction and classification.}}
%		\begin{figure}
%			\centering
%			\includegraphics[width=.95\textwidth]{conventional_ML.png}
%		\end{figure}	
%		Deep learning offers an \alert{\textbf{end-to-end}} approach: \alert{\textbf{Automatic}} feature extraction and classification.
%		\begin{figure}
%			\includegraphics[width=.95\textwidth]{DL_approach.png}
%		\end{figure}
%	\end{frame}
	\setcounter{subfigure}{0}	
	%%%%%%%%%%%%%%%%%%%%%%%%%%%%%%%%%%%%%%%%%%%%%%%%%%%%%%%%%%%%%%%%%%%%
	\begin{frame}{Computer vision}
		\begin{columns}[T]
			\begin{column}[c]{0.3\textwidth}
				\justifying
				\alert {\textbf{Computer vision}} is a~field of AI that enables computers and systems to derive meaningful information from digital images, videos and other visual inputs. 
			\end{column}

			\begin{column}[c]{0.7\textwidth}
				\begin{figure}
						\centering
						\efbox{\includegraphics[width=1\textwidth]{computer_vision_tasks_1.png}}					
				\end{figure}										
			\end{column}
		\end{columns}	
	\end{frame}
	%%%%%%%%%%%%%%%%%%%%%%%%%%%%%%%%%%%%%%%%%%%%%%%%%%%%%%%%%%%%%%%%%%%%
	\note{
		We now reach an essential concept that is computer vision. 
		which is a subfield of artificial intelligence that enables computer systems to obtain meaningful information from the visual inputs.
		
		Computer vision has three main hierarchical levels of tasks: 
		
		The first level performs a classification or detection for the whole input and predicts one output.
		
		The second level is the localisation of the object that we are looking for. 
		
		The ultimate level of computer vision is to perform Pixel-wise image segmentation, in which each pixel in the input image is classified into its proper class.	
		
		I am using pixel wise image segmentation in my PhD work for delamination identification.
		
	}	
	%%%%%%%%%%%%%%%%%%%%%%%%%%%%%%%%%%%%%%%%%%%%%%%%%%%%%%%%%%%%%%%%%%%%
	\setcounter{subfigure}{0}
	%%%%%%%%%%%%%%%%%%%%%%%%%%%%%%%%%%%%%%%%%%%%%%%%%%%%%%%%%%%%%%%%%%%%
	\begin{frame}{Convolution Neural Networks (CNNs)}
		\begin{itemize}
			\item \textbf{\alert{Convolutional Neural Network} (CNN)} is a powerful feature extraction tool.
			\item Features are extracted by applying convolution operations \textbf\alert{(sliding dot product)}.
		\end{itemize}
		\begin{columns}[T]
			\begin{column}[t]{0.58\textwidth}
				\begin{figure}[ht!]
					\centering		
					\caption{\textbf{Convolution operation}}			
					\efbox{\animategraphics[autoplay,loop,width =1\textwidth]{2}{figures/gif_figs/files/plot_convolution_process_}{0}{32}}					
				\end{figure}
			\end{column}		
			\begin{column}[t]{0.38\textwidth}
				\begin{figure}[ht!]
					\centering
					\caption{\textbf{CNN architecture}}
					\efbox{\includegraphics[width=1\textwidth]{cnn.png}}
				\end{figure}
			\end{column}			
		\end{columns}
	\end{frame}
	%%%%%%%%%%%%%%%%%%%%%%%%%%%%%%%%%%%%%%%%%%%%%%%%%%%%%%%%%%%%%%%%%%%%
	\note{
		Convolutional Neural Network (CNN) is a powerful deep learning architecture for mainly used with visual inputs as it can extract complex feature patterns using the convolution operation.
		
		The convolution operation in image processing is essentially a cross-correlation operation, also known as a sliding dot product.
		
		The kernel also called a filter or the feature detector slides over an input image to extract features.
		The output of the convolution operation is a feature map.
		
		Consequently, these kernels learn to detect different types of edges
		(vertical, horizontal, and diagonal edges), colour intensities, etc.
	}	
	%%%%%%%%%%%%%%%%%%%%%%%%%%%%%%%%%%%%%%%%%%%%%%%%%%%%%%%%%%%%%%%%%%%%
	\setcounter{subfigure}{0}
	%%%%%%%%%%%%%%%%%%%%%%%%%%%%%%%%%%%%%%%%%%%%%%%%%%%%%%%%%%%%%%%%%%%%
	\begin{frame}{Encoder-Decoder Architecture}
		\begin{columns}[T]
			\begin{column}[c]{.40\textwidth}
				\justifying
				\begin{itemize}
					\item \alert{Encoder}: extracts features
					\item \alert{Latent space}: condensed feature maps
					\item \alert{Decoder}: locates the features
				\end{itemize}	
			\end{column}
			\begin{column}[c]{.6\textwidth}
				\begin{figure}
					\centering
					\caption{\textbf{Encoder-decoder architecture}}
					\efbox{\includegraphics[width=1\textwidth]{nn_encoder_decoder.png}}
				\end{figure}	
			\end{column}
		\end{columns}			
	\end{frame}
	%%%%%%%%%%%%%%%%%%%%%%%%%%%%%%%%%%%%%%%%%%%%%%%%%%%%%%%%%%%%%%%%%%%%
	\note{
		The encoder-decoder in deep learning is a well-known architecture used for tasks such as computer vision.
		The encoder aims to produce compressed feature maps from the input image at various scale levels using cascaded convolutions and downsampling operations. 
		The decoder is responsible for upsampling the condensed feature maps in the latent space to the original input shape.		
	}
	%	%%%%%%%%%%%%%%%%%%%%%%%%%%%%%%%%%%%%%%%%%%%%%%%%%%%%%%%%%%%%%%%%%%%%
%	\section{Development environment}
%	\setcounter{subfigure}{0}
%	%%%%%%%%%%%%%%%%%%%%%%%%%%%%%%%%%%%%%%%%%%%%%%%%%%%%%%%%%%%%%%%%%%%	
%	\begin{frame}{Utilised Tools for DL models}
%		All developed DL models were \alert{coded in-house} \\
%		\alert{(Tailored specifically to meet tasks requirements)} 
%		\begin{itemize}	
%			\item \alert{Pycharm} (a dedicated Python Integrated Development Environment (IDE))		
%			\item \alert{Python}: versions: 3.7 - 3.9
%			\item \alert{TensorFlow} platform: versions: 2.0 - 2.6
%			\item \alert{Keras} (deep learning API over TensorFlow): versions: 2.0 - 2.6
%			\item \alert{GPUs}: NVIDIA RTX2080 /8GB, NVIDIA Tesla V100 /32 GB
%		\end{itemize}
%	\end{frame}
%	%%%%%%%%%%%%%%%%%%%%%%%%%%%%%%%%%%%%%%%%%%%%%%%%%%%%%%%%%%%%%%%%%%%%
%	\note
%	{
%		First of all, it is important to note that I have coded all deep-learning models in-house.
%		
%		To do so, I have utilised following tools:		
%		(Python, TensorFlow platform, Keras API) under Pycharm IDE, which is open-source.
%		Training a deep learning model has computational complexity.
%		Therefore, I utilised these GPUs: 
%	}
%	%%%%%%%%%%%%%%%%%%%%%%%%%%%%%%%%%%%%%%%%%%%%%%%%%%%%%%%%%%%%%%%%%%%%
%	\begin{frame}{Data preprocessing \& Hyperparameters tuning}	
%		\begin{table}[t]
%			\centering
%			\resizebox{1\textwidth}{!}{%
%				\begin{tabular}{lccc}
%					\toprule[1.5pt]
%					\multirow{2}{*}{\textbf{Applied techniques}}& \multicolumn{2}{c}{\textbf{Delamination identification}} & \multirow{2}{*}{\textbf{Super-resolution image reconstruction}} \\ \cmidrule(lr){2-3}
%					& RMS image based & Full wavefield based & \\ \midrule 
%					Dataset normalisation & \checkmark & \checkmark & \checkmark \\
%					Dataset augmentation & \checkmark & & \\
%					Cross validation & \checkmark & & \\ 
%					Early stoppling & \checkmark & \checkmark & \checkmark \\ 
%					Hyperparameter tuning & \checkmark & \checkmark & \checkmark \\ 
%					Regularization & \checkmark & \checkmark & \checkmark \\ 
%					\bottomrule[1.5pt]
%				\end{tabular}
%			}
%			\begin{table}[t]
%				\centering
%				\resizebox{1\textwidth}{!}{%
%					\begin{tabular}{l|c|c|c|c|c|c|c}
%						\toprule[1.5pt]
%						\textbf{Hyperparameters} & Learning rate & Momentum &
%						Dropout rate & Batch size & Epochs & 			Optimisation technique & Objective loss function \\						
%						\bottomrule[1.5pt]
%				\end{tabular}}
%			\end{table}
%		\end{table}
%		\begin{table}
%			\centering
%			\resizebox{.4\textwidth}{!}{%
%				\begin{tabular}{ccc}
%					\toprule[1.5pt]
%					Training set & Testing set & Validation set \\
%					\midrule
%					$80\%$ & $20\%$ & $20\%$ out of training set \\
%					\bottomrule[1.5pt]
%			\end{tabular}}
%		\end{table}					
%	\end{frame}
	%%%%%%%%%%%%%%%%%%%%%%%%%%%%%%%%%%%%%%%%%%%%%%%%%%%%%%%%%%%%%%%%%%%%
	\section{Part I: Delamination identification}
	%%%%%%%%%%%%%%%%%%%%%%%%%%%%%%%%%%%%%%%%%%%%%%%%%%%%%%%%%%%%%%%%%%%%%%%%%%%%%%%%
	\begin{frame}{Full wavefield animation-based approach}
		\begin{columns}[T]
			\centering
			\begin{column}[c]{0.47\textwidth}
				\centering
				\textbf{Multiple frames (animation)}
				\begin{figure}
					\centering
					\captionsetup{justification=centering}					
					\subfloat{\animategraphics[autoplay,loop,width=.75\textwidth]{16}{figures/gif_figs/381_output/output_381-}{1}{512}}
				\end{figure}
			\end{column}	
			\begin{column}[c]{0.47\textwidth}
				\centering
				\textbf{Single output}
				\begin{figure}
					\centering
					\captionsetup{justification=centering}					
					\subfloat{\includegraphics[width=.75\textwidth]{GCN_381.png}}
				\end{figure}
			\end{column}	
		\end{columns}		
	\end{frame}	
	%%%%%%%%%%%%%%%%%%%%%%%%%%%%%%%%%%%%%%%%%%%%%%%%%%%%%%%%%%%%%%%%%%%%
	\note{
		In part, I will present deep learning approaches for delamination identification in composite laminates.
		
		It is important to note how these models perform in an end-to-end fashion.
		
		Accordingly, two approaches for delamination identification based on their inputs were adopted:
		
		\begin{itemize}
			\item the first one is the One-to-one approach (takes one input that is the root mean squared of the full wavefield) and produces one output of damage map)
			\item the second one is the Many-to-one approach (takes animation of Lamb waves propagation as a sequence of frames and produces one output as damage map).
		\end{itemize}	
	}
	%%%%%%%%%%%%%%%%%%%%%%%%%%%%%%%%%%%%%%%%%%%%%%%%%%%%%%%%%%%%%%%%%%%%
	%	\subsection{Developed DL models}
	%\begin{frame}{Common deep learning architectures}
	%	
	%	\begin{column}[t]{0.45\textwidth}
	%		\textbf{RMS based}\\
	%		\begin{itemize}
	%			\item Convolutional neural networks (CNN)
	%			\item Fully convolutional network (FCN)
	%		\end{itemize}
	%	\end{column}
	%	\hfill
	%	\begin{column}[t]{0.45\textwidth}
	%		\textbf{Full wavefield frames}\\
	%		\begin{itemize}
	%			\item Recurrent neural network (RNN)
	%			\item Long short-term memory (LSTM)
	%			\item ConvLSTM
	%		\end{itemize}
	%	\end{column}
	%\end{frame}
	
%	%%%%%%%%%%%%%%%%%%%%%%%%%%%%%%%%%%%%%%%%%%%%%%%%%%%%%%%%%%%%%%%%%%%%
%	\begin{frame}{Developed model for delamination identification}
%		\begin{columns}[T]
%			\begin{column}[t]{0.45\textwidth}
%				\begin{block}{RMS based models}
%					\begin{itemize}
%						\item VGG 16 encoder-decoder
%						\item Res-UNet					
%						\item FCN-DenseNet
%						\item PSPNet
%						\item GCN
%					\end{itemize}				
%				\end{block}
%			\end{column}
%			\hfill
%			\begin{column}[t]{.50\textwidth}
%				\begin{block}{Full wavefield frames based model}					
%					\begin{itemize}
%						\item Autoencoder ConvLSTM (Convolutional~Long~Short-Term Memory)
%					\end{itemize}									
%				\end{block}
%			\end{column}
%		\end{columns}
%	\end{frame}	
%	%%%%%%%%%%%%%%%%%%%%%%%%%%%%%%%%%%%%%%%%%%%%%%%%%%%%%%%%%%%%%%%%%%%%
	%%%%%%%%%%%%%%%%%%%%%%%%%%%%%%%%%%%%%%%%%%%%%%%%%%%%%%%%%%%%%%%%%%%%
%	\section*{RMS based models}	
%	%%%%%%%%%%%%%%%%%%%%%%%%%%%%%%%%%%%%%%%%%%%%%%%%%%%%%%%%%%%%%%%%%%%%
%	\setcounter{subfigure}{0}
%	\begin{frame}{Developed RMS based models}
%		\begin{figure}
%			\includegraphics[width=0.95\textwidth]{Developed_rms_models.png}
%		\end{figure}
%	\end{frame}
%	%%%%%%%%%%%%%%%%%%%%%%%%%%%%%%%%%%%%%%%%%%%%%%%%%%%%%%%%%%%%%%%%%%%%
%	\setcounter{subfigure}{0}	
%	%%%%%%%%%%%%%%%%%%%%%%%%%%%%%%%%%%%%%%%%%%%%%%%%%%%%%%%%%%%%%%%%%%%%
%	\note{
%		Here, I present the architectures of all the deep learning models I've created that are based on the RMS image of the full wavefield.
%		
%		For further details about these models, kindly, I advise you to return to my thesis and my published papers. 
%		%		Both the Res-UNet and VGG16 encoder-decoders are autoencoders.
%		%		The main difference between them is the additional skip connections that were added to Res-Unet at the encoder and decoder levels. \\
%		%		FCN-DenseNet applies an encoder-decoder scheme with skip connections between the encoder and the decoder paths.
%		%		The main component in FCN-DenseNet is the dense block. 
%		%		The dense block is constructed from a varying number of convolutional layers. 
%		%		The purpose of the dense block is to concatenate feature maps of a layer with its output to emphasize spatial details information. \\		
%		%		The idea of PSPNet is to provide adequate global contextual information for pixel-level scene parsing by concatenating the local and global features together. 
%		%		Hence, a spatial pyramid pooling module was introduced to perform four different pooling levels with four different pool sizes. 
%		%		In this way, the pyramid pooling module can capture contextual features at different scales.\\		
%		%		GCN addresses the importance of having large kernels at the convolution operations for both localization and classification tasks for semantic segmentation.
%	}
%	%%%%%%%%%%%%%%%%%%%%%%%%%%%%%%%%%%%%%%%%%%%%%%%%%%%%%%%%%%%%%%%%%%%%
%	\setcounter{subfigure}{0}
%	%%%%%%%%%%%%%%%%%%%%%%%%%%%%%%%%%%%%%%%%%%%%%%%%%%%%%%%%%%%%%%%%%%%%
%	\section{Full wavefield frames based model}
	\begin{frame}{Autoencoder ConvLSTM}
		\begin{figure}[ht!]
			\centering
			\includegraphics[width=1\textwidth]{figure3.png}
		\end{figure}
	\end{frame}
	%%%%%%%%%%%%%%%%%%%%%%%%%%%%%%%%%%%%%%%%%%%%%%%%%%%%%%%%%%%%%%%%%%%%
	\note{
		Here, I present the developed autoencoder-ConvLSTM model, which takes a sequence of consecutive full wavefield frames of guided wave propagation as an input.
		
		To reduce the training complexity, I used a certain number of frames after the initial interaction of guided waves with the damage, as the delamination location is known.
		
		These selected frames, which contain the required features regarding the delamination shape and location, are fed into an encoder-decoder model at once using a time-distributed layer.
		
		Then, the output of the decoder is forwarded into the ConvLSTM layer, which handles time series data by learning long-term spatiotemporal features.
		
		For real-life situations where the damage location is unknown,  the full wavefield is tested through a sliding window that produces an intermediate output at a time.
		Finally, the root-mean-square formula is applied to all intermediate predictions to obtain the RMS damage map.		
	}
	%%%%%%%%%%%%%%%%%%%%%%%%%%%%%%%%%%%%%%%%%%%%%%%%%%%%%%%%%%%%%%%%%%%%
	\begin{frame}{Evaluation metrics for delamination identification}
		\begin{columns}[T]
			\begin{column}[c]{0.45\textwidth}
				For evaluating delamination identification
				\begin{itemize}
					\item Intersection over Union (IoU): 
					\begin{equation*}
						\textup{IoU}=\frac{Intersection}{Union}=\frac{\hat{Y} \cap Y}{\hat{Y} \cup Y}
						\label{eqn:iou}
					\end{equation*}
					\item Percentage area error $\epsilon$:
					\begin{equation*}
						\epsilon=\frac{|A-\hat{A}|}{A} \times 100\%
						\label{eqn:mean_size_error}
					\end{equation*}
				\end{itemize}
			\end{column}
			\begin{column}[c]{0.45\textwidth}
				\begin{figure}
					\centering
					\includegraphics[width=1.0\textwidth]{IoU_figure.png}		
				\end{figure}
			\end{column}
		\end{columns}
	\end{frame}
	%%%%%%%%%%%%%%%%%%%%%%%%%%%%%%%%%%%%%%%%%%%%%%%%%%%%%%%%%%%%%%%%%%%%
	\note{
		Now, to evaluate the developed models for delamination identification, 
		I used two metrics:
		The first metric is the mean intersection over union (also known as the Jaccard index), which calculates the area of intersection between actual and predicted values divided by the union of them.
		The second metric is the percentage area error \(\epsilon\), which calculates the percentage of the difference between actual and predicted areas. 
	}
	%%%%%%%%%%%%%%%%%%%%%%%%%%%%%%%%%%%%%%%%%%%%%%%%%%%%%%%%%%%%%%%%%%%%
	\section*{Evaluation: Numerical cases}
%	%%%%%%%%%%%%%%%%%%%%%%%%%%%%%%%%%%%%%%%%%%%%%%%%%%%%%%%%%%%%%%%%
%	\begin{frame}{Numerical test cases RMS based models (GCN model)}
%		\begin{columns}[T]
%			\begin{column}[c]{0.32\textwidth}
%				\begin{figure}[c]
%					\centering
%					\captionsetup{justification=centering}					
%					\animategraphics[controls,width=.9\textwidth]{8}{figures/gif_figs/456/intermediate_output-}{0}{82}
%					\caption{\(1^{st}\) numerical case, IoU=0.71}
%				\end{figure}
%			\end{column}
%			\hfill
%			\begin{column}[c]{0.32\textwidth}
%				\begin{figure}[c]
%					\centering
%					\captionsetup{justification=centering}					
%					\animategraphics[controls,width=.9\textwidth]{8}{figures/gif_figs/438/intermediate_output-}{0}{82}
%					\caption{\(2^{nd}\) numerical case, IoU=0.72}
%				\end{figure}
%			\end{column}
%			\hfill
%			\begin{column}[c]{0.32\textwidth}
%				\begin{figure}[c]
%					\centering
%					\captionsetup{justification=centering}
%					
%					\animategraphics[controls,width=.9\textwidth]{8}{figures/gif_figs/397/intermediate_output-}{0}{82}
%					\caption{\(3^{rd}\) numerical case, IoU=0.86}
%				\end{figure}					
%			\end{column}
%		\end{columns}
%	\end{frame}
	%%%%%%%%%%%%%%%%%%%%%%%%%%%%%%%%%%%%%%%%%%%%%%%%%%%%%%%%%%%%%%%%%%%%
	\note{
		This slide shows three numerical samples tested with the GCN model.
		For the first case, the delamination is barely visible by the naked eye, yet the model could identify the delamination with IoU = 0.71.
		For the second case, the IOU = 0.72, and for the third case, the IOU = .86
		Additionally, each animation shows all extracted feature maps from the RMS image input until we get the final prediction of the damage map.
		It's important to notice that GCN can identify the delamination with high accuracy and noise free.			
	}
	%%%%%%%%%%%%%%%%%%%%%%%%%%%%%%%%%%%%%%%%%%%%%%%%%%%%%%%%%%%%%%%%%%%%
	%	\begin{frame}{RMS based: Analysis of numerical cases}
	%		\begin{columns}[T]
	%%				\tiny
	%%				\begin{column}[c]{0.48\textwidth}
	%%					%%%%%%%%%%%%%%%%%%%%%%%%%%%%%%%%%%%%%%%%%%%%%%%%%%%%%%%%%%%%
	%%					\begin{table}[ht!]
	%%						\centering
	%%						\caption{Evaluation metrics of the three numerical cases.}
	%%						\label{tab:RMS_num_cases}
	%%						\begin{tabular}{cccccc}
	%%							\toprule[1.5pt]
	%%							\multirow{2}{*}{Model} & \multirow{2}{*}{case number} & \multicolumn{1}{c}{\multirow{2}{*}{A [mm\textsuperscript{2}]}} & \multicolumn{3}{c}{Predicted output} \\ 
	%%							\cmidrule(lr){4-6} & & & \multicolumn{1}{c}{IoU} & \multicolumn{1}{c}{\(\hat{A}\) [mm\textsuperscript{2}]} & \(\epsilon\) \\
	%%							\midrule
	%%							\multirow{3}{*}{Res-UNet} 							
	%%							& 1 & 257 & \multicolumn{1}{c}{0.45} & \multicolumn{1}{c}{143} & \(44.36\%\) \\ 
	%%							& 2 & 105 & \multicolumn{1}{c}{0.67} & \multicolumn{1}{c}{88} & \(16.19\%\) \\ 
	%%							& 3 & 537 & \multicolumn{1}{c}{0.80} & \multicolumn{1}{c}{478} & \(10.99\%\) \\ 
	%%							\midrule
	%%							\multirow{3}{*}{VGG16 encoder-decoder} 
	%%							& 1 & 257 & \multicolumn{1}{c}{0.69} & \multicolumn{1}{c}{203} & \(21.01\%\) \\ 
	%%							& 2 & 105 & \multicolumn{1}{c}{0.75} & \multicolumn{1}{c}{117} & \(11.43\%\) \\ 
	%%							& 3 & 537 & \multicolumn{1}{c}{0.65} & \multicolumn{1}{c}{385} & \(28.31\%\) \\ 
	%%							\midrule
	%%							\multirow{3}{*}{FCN-DenseNet} 
	%%							& 1 & 257 & \multicolumn{1}{c}{0.52} & \multicolumn{1}{c}{505} & \(96.50\%\) \\ 
	%%							& 2 & 105 & \multicolumn{1}{c}{0.66} & \multicolumn{1}{c}{118} & \(12.38\%\) \\ 
	%%							& 3 & 537 & \multicolumn{1}{c}{0.72} & \multicolumn{1}{c}{815} & \(51.77\%\) \\ 
	%%							\midrule
	%%							\multirow{3}{*}{PSPNet} 
	%%							& 1 & 257 & \multicolumn{1}{c}{0.00} & \multicolumn{1}{c}{0} & \(-\%\) \\ 
	%%							& 2 & 105 & \multicolumn{1}{c}{0.44} & \multicolumn{1}{c}{156} & \(48.57\%\) \\ 
	%%							& 3 & 537 & \multicolumn{1}{c}{0.77} & \multicolumn{1}{c}{610} & \(13.59\%\) \\ 
	%%							\midrule
	%%							\multirow{3}{*}{GCN} 
	%%							& 1 & 257 & \multicolumn{1}{c}{0.71} & \multicolumn{1}{c}{215} & \(16.34\%\) \\ 
	%%							& 2 & 105 & \multicolumn{1}{c}{0.72} & \multicolumn{1}{c}{177} & \(68.57\%\) \\ 
	%%							& 3 & 537 & \multicolumn{1}{c}{0.86} & \multicolumn{1}{c}{523} & \(2.61\%\) \\ 
	%%							\bottomrule[1.5pt]
	%%						\end{tabular}	
	%%					\end{table}
	%	 %%%%%%%%%%%%%%%%%%%%%%%%%%%%%%%%%%%%%%%%%%%%%%%%%%%%%%%%%%%%%%%
	%% \end{column}
	%%		\hfill
	%		\begin{column}[c]{0.9\textwidth}
	%			\begin{table}[ht!]
	%				\centering
	%				\caption{Analysis of numerical cases.}
	%				\label{tab:table_all_numerical_cases}	
	%				\begin{tabular}{lcc}
	%					\toprule[1.5pt]
	%					Model & mean IoU & max IoU \\ 
	%					\midrule 
	%					Res-UNet & \(0.66\) & \(0.89\) \\ 
	%					VGG16 encoder-decoder & \(0.57\) & \(0.84\) \\ 
	%					FCN-DenseNet & \(0.68\) & \(0.92\) \\ 
	%					PSPNet & \(0.55\) & \(0.91\) \\ 
	%					GCN & \textbf{\(0.76\)} & \textbf{\(0.93\)} \\ 
	%					\bottomrule[1.5pt]
	%				\end{tabular}
	%			\end{table}
	%		\end{column}
	%		\end{columns}
	%	\end{frame}
	%%%%%%%%%%%%%%%%%%%%%%%%%%%%%%%%%%%%%%%%%%%%%%%%%%%%%%%%%%%%%%%%%%%%
	%	\note{
	%		The table presents the mean and maximum values calculated for the previously unseen numerical test set for all RMS-based models. 
	%		It also shows that all models have a relatively high value, indicating their ability to detect and localize the delamination.
	%		However, the best performance was achieved by the GCN model.
	%	}
	%%%%%%%%%%%%%%%%%%%%%%%%%%%%%%%%%%%%%%%%%%%%%%%%%%%%%%%%%%%%%%%%%%%%
	\begin{frame}{Numerical test cases animation of Lamb waves}
		\setcounter{subfigure}{0}
		\only<1>{
			\begin{alertblock}{First test case}
				\begin{figure}
					\centering
					\captionsetup{justification=centering}
					\subfloat[Full wavefield (512 frames)]{\animategraphics[autoplay,loop,height=3cm,keepaspectratio]{32}{figures/gif_figs/381_output/output_381-}{1}{512}}\quad
					\subfloat[Intermediate outputs]{\animategraphics[autoplay,loop,height=3cm,keepaspectratio]{31}{figures/gif_figs/Numerical_case_381/num_case_381_frame_num-}{0}{487}}\quad
					\subfloat[RMS (damage map)]{\includegraphics[height=3.05cm,keepaspectratio]{figures/RMS_Ijjeh_num_case_381.png}}\quad
					\subfloat[Binary RMS, IoU= 0.88]{\includegraphics[height=3cm,keepaspectratio]{figures/Binary_RMS_Ijjeh_num_case381_.png}}\quad
				\end{figure}
		\end{alertblock}}
		\setcounter{subfigure}{0}
		\only<2>{
			\begin{alertblock}{Second test case}
				\begin{figure}
					\centering
					\captionsetup{justification=centering}
					\subfloat[Full wavefield (512 frames)]{\animategraphics[autoplay,loop,height=3cm,keepaspectratio]{32}{figures/gif_figs/385_output/output_385-}{1}{512}}\quad		
					\subfloat[Intermediate outputs]{\animategraphics[autoplay,loop,height=3cm,keepaspectratio]{31}{figures/gif_figs/Numerical_case_385/num_case_385_frame_num-}{0}{487}}\quad			
					\subfloat[RMS (damage map)]{\includegraphics[height=3.05cm,keepaspectratio]{figures/RMS_Ijjeh_num_case_385.png}}\quad
					\subfloat[Binary RMS, IoU= 0.58]{\includegraphics[height=3cm,keepaspectratio]{figures/Binary_RMS_Ijjeh_num_case385_.png}}
				\end{figure}
		\end{alertblock}}
		\setcounter{subfigure}{0}
%		\only<3>{
%			\begin{alertblock}{Third test case}
%				\begin{figure}
%					\centering
%					\captionsetup{justification=centering}
%					\subfloat[Full wavefield (512 frames)]{\animategraphics[autoplay,loop,height=3cm,keepaspectratio]{32}{figures/gif_figs/394_output/output_394-}{1}{512}}\quad
%					\subfloat[Intermediate outputs]{\animategraphics[autoplay,loop,height=3cm,keepaspectratio]{31}{figures/gif_figs/Numerical_case_394/num_case_394_frame_num-}{0}{487}}\quad
%					\subfloat[RMS (damage map)]{\includegraphics[height=3.05cm,keepaspectratio]{figures/RMS_Ijjeh_num_case_394.png}}\quad
%					\subfloat[Binary RMS, IoU= 0.8]{\includegraphics[height=3cm,keepaspectratio]{figures/Binary_RMS_Ijjeh_num_case394_.png}}
%				\end{figure}
%		\end{alertblock}}
	\end{frame}
	%%%%%%%%%%%%%%%%%%%%%%%%%%%%%%%%%%%%%%%%%%%%%%%%%%%%%%%%%%%%%%%%%%%%
	\note{
		In the following three slides, I will present three numerical samples evaluated with the autoencoder ConvLSTM model.
		
		Animation (a) shows the full wavefield of 512 frames as an input to the model.
		Figure (b) shows the RMS damage map for all intermediate predictions,
		and figure (c) shows the binary RMS with IoU = 0.88.
		The second case is more complex, as the delamination is near the corner of the plate.
		In this case, the IoU is 0.58.
		The third case is also complex, as the delamination is located near the plate edge.
		In this case, the IoU is 0.8.		
	}
	%%%%%%%%%%%%%%%%%%%%%%%%%%%%%%%%%%%%%%%%%%%%%%%%%%%%%%%%%%%%%%%%%%%%
	%	\begin{frame}{Animation based: Analysis of numerical cases}
	%		%%%%%%%%%%%%%%%%%%%%%%%%%%%%%%%%%%%%%%%%%%%%%%%%%%%%%%%%%%%%%%%%%%%%
	%		\begin{table}[!h]
	%			\centering
	%			\caption{Evaluation metrics of the three numerical cases.}
	%			\begin{tabular}{ccccc}
	%				\toprule[1.5pt]
	%				\multirow{2}{*}{case number} & \multicolumn{1}{c}{\multirow{2}{*}{A [mm\textsuperscript{2}]}} & \multicolumn{3}{c}{Predicted output} \\ 
	%				\cmidrule(lr){3-5} & & \multicolumn{1}{c}{IoU} & \multicolumn{1}{c}{\(\hat{A}\) [mm\textsuperscript{2}]} & \(\epsilon\) \\
	%				\midrule
	%				1 & 763 & \multicolumn{1}{c}{0.88} & \multicolumn{1}{c}{735} & \(3.67\%\) \\ 
	%				2 & 388 & \multicolumn{1}{c}{0.58} & \multicolumn{1}{c}{248} & \(36.08\%\) \\ 
	%				3 & 297 & \multicolumn{1}{c}{0.80} & \multicolumn{1}{c}{280} & \(5.72\%\) \\			 					
	%				\bottomrule[1.5pt]
	%			\end{tabular}	
	%			\label{tab:num_cases}
	%		\end{table}			
	%	\end{frame}
	%	%%%%%%%%%%%%%%%%%%%%%%%%%%%%%%%%%%%%%%%%%%%%%%%%%%%%%%%%%%%%%%%%%%%%%%%%%%%%
	%	\note{
	%		 The shown table presents the evaluation metrics for the autoencoder ConvLSTM model regarding the three numerical cases shown in the previous slide.
	%		 
	%		 The table gathers the actual delamination area, predicted delamination area, intersection over union IoU, and percentage area error \(\epsilon\) to each case.
	%	}
	%%%%%%%%%%%%%%%%%%%%%%%%%%%%%%%%%%%%%%%%%%%%%%%%%%%%%%%%%%%%%%%%%%%%
	\section*{Evaluation: Experimental cases}	
	\setcounter{subfigure}{0}		%%%%%%%%%%%%%%%%%%%%%%%%%%%%%%%%%%%%%%%%%%%%%%%%%%%%%%%%%%%%%%%%%%%%
%	\begin{frame}{Experimental results: RMS image-based (Single delamination)}
%		\begin{columns}[T]
%			\begin{column}[t]{.25\textwidth}
%				\begin{figure}[ht!]
%					\centering
%					\captionsetup{justification=centering}
%					\includegraphics[height=.35\textheight]{ERMS_with_label.png}
%					\caption{ERMS \& label}
%				\end{figure}
%				\justifying
%				\tiny
%				Kudela, P., Radzienski, M. and Ostachowicz, W., 2018. \textbf{Impact induced damage assessment by means of Lamb wave image processing}. \textit{Mechanical Systems and Signal Processing}, 102, pp.23-36.
%			\end{column}
%			\begin{column}[t]{0.5\textwidth}
%				\begin{block}{Adaptive wavenumber filtering}
%					\centering
%					\footnotesize
%					IoU=$0.401$
%					\begin{figure}[ht!]
%						\centering
%						\captionsetup{justification=centering}
%						\subfloat{\includegraphics[height=.35\textheight]{ERMSF_CFRP_teflon_3o_375_375p_50kHz_5HC_x12_15Vpp.png}}
%						\quad
%						\subfloat{\includegraphics[height=.35\textheight]{Binary_ERMSF_CFRP_teflon_3o_375_375p_50kHz_5HC_x12_15Vpp.png}}
%					\end{figure}
%				\end{block}					
%			\end{column}		
%			\begin{column}[t]{0.25\textwidth}
%				\begin{alertblock}{DL approach: GCN}
%					\centering
%					\footnotesize
%					IoU\(=0.723\)
%					\begin{figure}[ht!]	
%						\centering				
%						\subfloat{\includegraphics[height=.35\textheight]{Fig_GCN_7.png}}
%					\end{figure} 	
%				\end{alertblock}				
%			\end{column}
%		\end{columns}	
%	\end{frame}
%	%%%%%%%%%%%%%%%%%%%%%%%%%%%%%%%%%%%%%%%%%%%%%%%%%%%%%%%%%%%%%%%%%%%%
%	\note{
%		In this slide, I present the experimental results of a specimen with a single delamination using the RMS-based model, the GCN.
%		Additionally, I compared my developed model with the adaptive wavenumber filtering technique for damage imaging, which is a conventional signal processing technique.
%		Figure (a) shows the ERMS with a label depicting the delamination. 
%		Figure (b) shows the result of applying the adaptive wavenumber filtering method, and Figure (c) shows its binary output with IoU=0.401.
%		Figure (d) shows the predicted damage map by the GCN model with IoU = 0.723.
%		As shown, the deep learning model surpasses the conventional signal processing approach.
%	}
	%%%%%%%%%%%%%%%%%%%%%%%%%%%%%%%%%%%%%%%%%%%%%%%%%%%%%%%%%%%%%%%%%%%%
	%	\setcounter{subfigure}{0}
	%	\begin{frame}{RMS based: Analysis of experimental case}
	%		\begin{table}[!ht]
	%			\centering
	%			\caption{Evaluation metrics of the experimental case.}
	%			\label{tab:rms_exp_case}
	%			\begin{tabular}{lc}
	%				\toprule[1.5pt]
	%				Model & IoU	\\			
	%				\midrule
	%				Res-UNet & 0.58 \\ 
	%				VGG16 encoder-decoder & 0.62 \\ 
	%				FCN-DenseNet & 0.54 \\ 
	%				PSPNet & 0.49 \\ 
	%				GCN & 0.72\\ 
	%				\bottomrule[1.5pt]
	%			\end{tabular}		
	%		\end{table}
	%%			\begin{table}[!ht]
	%%				\centering
	%%				\caption{Evaluation metrics of the experimental case.}
	%%				\label{tab:rms_exp_case}
	%%				\begin{tabular}{l@{\ }cccc}
	%%					\toprule
	%%					\multicolumn{1}{l}{Model} & \multicolumn{1}{c}{A [mm\textsuperscript{2}]} & \multicolumn{3}{c}{Predicted output} \\ 
	%%					\cmidrule(lr){3-5} & & \multicolumn{1}{c}{IoU} & \multicolumn{1}{c}{\(\hat{A}\) [mm\textsuperscript{2}]} & \(\epsilon\) \\ \midrule
	%%					Res-UNet & \multicolumn{1}{c}{\multirow{5}{*}{210}} & \multicolumn{1}{c}{0.58} & \multicolumn{1}{c}{323} & \(53.8\%\) \\ 
	%%					VGG16 encoder-decoder & & \multicolumn{1}{c}{0.62} & \multicolumn{1}{c}{320} & \(52.4\%\) 
	%%					\\ 
	%%					FCN-DenseNet & & \multicolumn{1}{c}{0.54} & \multicolumn{1}{c}{386} & \(83.8\%\) \\ 
	%%					PSPNet & & \multicolumn{1}{c}{0.49} & \multicolumn{1}{c}{580} & \(176.2\%\) 
	%%					\\ 
	%%					GCN & & \multicolumn{1}{c}{0.72} & \multicolumn{1}{c}{309} & \(47.1\%\) 
	%%					\\ 
	%%					\bottomrule
	%%				\end{tabular}		
	%%			\end{table}
	%	\end{frame}
	%	%%%%%%%%%%%%%%%%%%%%%%%%%%%%%%%%%%%%%%%%%%%%%%%%%%%%%%%%%%%%%%%%%%%%
	%	\note{
	%		The table shows the IoU values for all developed RMS based models for the single delamination specimen.
	%		
	%		Similarly to the numerical dataset, the best accuracy was achieved by using GCN.
	%		
	%%		As shown, the models are capable of precise detection and localisation of the delamination. 
	%%		We can see that the models can identify the delamination with almost free noise indicating the models are capable of generalising and detecting the delamination on previously unseen data. 
	%	}
	%%%%%%%%%%%%%%%%%%%%%%%%%%%%%%%%%%%%%%%%%%%%%%%%%%%%%%%%%%%%%%%%%%%%
	\setcounter{subfigure}{0}
	\begin{frame}{Experimental results: Specimen with single delamination}		
		\begin{alertblock}{DL approach}
			IoU= $0.41$% and $\epsilon=71.56\%$ 
			\begin{figure}[ht!]
				\centering
				\subfloat[Input]{\animategraphics[autoplay,loop,height=3cm]{16}{figures/gif_figs/CFRP_teflon_3o_375_375p_50kHz_5HC_x12_15Vpp/CFRP_teflon_30-}{1}{256}}\quad
				\subfloat[Intermidate ouputs]{\animategraphics[autoplay,loop,height=3cm]{15}{figures/gif_figs/CFRP_ijjeh_single_delamination/intermediate_output-}{0}{231}}\quad
				\subfloat[RMS]{\includegraphics[height=3cm,keepaspectratio]{figures/RMS_CFRP_teflon_3o_375_375p_50kHz_5HC_x12_15Vpp_Ijjeh_updated_results_.png}}\quad
				\subfloat[Binary RMS]{\includegraphics[height=3cm,keepaspectratio]{figures/Binary_RMS_CFRP_teflon_3o__375_375p_50kHz_5HC_x12_15Vpp_Ijjeh_.png}}
			\end{figure}			
		\end{alertblock}	
	\end{frame}
	%%%%%%%%%%%%%%%%%%%%%%%%%%%%%%%%%%%%%%%%%%%%%%%%%%%%%%%%%%%%%%%%%%%%
	\note{
		In this slide, I present the predicted results using the autoencoder ConvLSTM model regarding the single delamination case.
		
		Animation (a) shows the full wavefield measured by SLDV with 256 frames.
		
		Animation (b) shows the intermediate predictions of the model. 
		
		Figure (c) shows the RMS of all intermediate predictions.
		
		And finally, Figure (d) shows the binary RMS with IoU= 0.41		
	}
	%%%%%%%%%%%%%%%%%%%%%%%%%%%%%%%%%%%%%%%%%%%%%%%%%%%%%%%%%%%%%%%%%%%%
	\setcounter{subfigure}{0}	
	\begin{frame}{Experimental results: Specimen with multiple delaminations}
		\begin{columns}[T]
			\begin{column}[t]{0.20\textwidth}
				\begin{block}{Input}
					\footnotesize Full wavefield					
					\begin{figure}[ht!]	
						\centering						
						\subfloat{\animategraphics[autoplay,loop,height=0.32\textheight]{32}{figures/gif_figs/input_specimen_3/specimen_3-}{1}{512}}
					\end{figure}
				\end{block}				
			\end{column}
			\begin{column}[t]{0.40\textwidth}				
				\begin{block}{Adaptive wavenumber filtering}
					\footnotesize IoU$=0.04$					
					\begin{figure}[ht!]	
						\centering
						\subfloat{\includegraphics[height=0.32\textheight]{figures/mul/figure17a.png}}
						\quad
						\centering
						\subfloat{\includegraphics[height=0.32\textheight]{figures/mul/figure17b.png}}								
					\end{figure}
				\end{block}	
			\end{column}
			\begin{column}[t]{0.40\textwidth}				
				\begin{alertblock}{DL approach}	
					\footnotesize IoU= $0.64$						
					\begin{figure}[ht!]	
						\centering
						\subfloat{\includegraphics[height=0.32\textheight]{figures/RMS_L3_S3_B_333x333p_50kHz_5HC_18Vpp_x10_pzt_Ijjeh_updated_results_.png}}
						\quad
						\subfloat{\includegraphics[height=0.32\textheight]{figures/Binary_RMS_L3_S3_B__333x333p_50kHz_5HC_18Vpp_x10_pzt_Ijjeh_.png}}						
					\end{figure}				
				\end{alertblock}				
			\end{column}				
		\end{columns}	
	\end{frame}
	%%%%%%%%%%%%%%%%%%%%%%%%%%%%%%%%%%%%%%%%%%%%%%%%%%%%%%%%%%%%%%%%%%%%
	\note{
		Here, I present the predicted results using the autoencoder ConvLSTM model compared to adaptive wavenumber filtering for the multiple delamination case.
		The animation on the left shows the full wavefield measured by the SLDV of 512 frames.
		The figure shows the damage map resulting from applying adaptive wavenumber filtering, and this figure shows its binary output with IoU = 0.04
		Figure shows the RMS of all intermediate predictions, and figure shows the binary RMS with IoU = 0.64
		It can be concluded that utilising animations of Lamb waves propagation has better outcomes for delamination identification than the processing of a single image representing signal energy or RMS.		
	}
	%%%%%%%%%%%%%%%%%%%%%%%%%%%%%%%%%%%%%%%%%%%%%%%%%%%%%%%%%%%%%%%%%%%%
	\setcounter{subfigure}{0}
	%%%%%%%%%%%%%%%%%%%%%%%%%%%%%%%%%%%%%%%%%%%%%%%%%%%%%%%%%%%%%%%%%%%%
	\section{Part II: Super-resolution image reconstruction}
	\begin{frame}{Super-resolution image reconstruction}
		\begin{columns}[T]
			\begin{column}[t]{0.6\textwidth}
				\begin{alertblock}{Deep learning super-resolution model (DLSR)}					
					\begin{footnotesize}
						\justifying
						\settowidth{\leftmargini}{\usebeamertemplate{itemize item}}
						\addtolength{\leftmargini}{\labelsep}
						\begin{itemize}
							\item Registering HR full wavefield with an SLDV is a~time-consuming process.
							\item DLSR model aims to recover HR full wavefield scans from a~LR measurements (below the Nyquist-Shannon sampling rate).
						\end{itemize} 
					\end{footnotesize}					
				\end{alertblock}						
				\begin{exampleblock}{Compressive sensing (CS) theory}
					\footnotesize
					\justifying
					Any natural signal (\(x\)), e.g. (sounds, images) can be recovered using considerably fewer measurements (\(y\)) than standard methods.
					\vfill
					\begin{figure}[ht!]
						\centering
						\includegraphics[width=.45\textwidth]{matrix_mask.png}
					\end{figure}
				\end{exampleblock}							
			\end{column}
			\begin{column}[t]{0.4\textwidth}
				\begin{figure}[ht!]
					\centering
					\includegraphics[width=1\textwidth]{superresolution_flowchart.png}
				\end{figure}
			\end{column}
		\end{columns}		
	\end{frame}
	%%%%%%%%%%%%%%%%%%%%%%%%%%%%%%%%%%%%%%%%%%%%%%%%%%%%%%%%%%%%%%%%%%%%
	\note{
		As mentioned earlier, the scanning laser Doppler vibrometer is a well-known non-contact tool for the acquisition of the full wavefield of propagating guided waves.
		However, the process of acquiring the full wavefield of guided waves is time-consuming. 
		One possible solution to tackle this problem is to acquire the Lamb waves at low-resolution grid points, and then the full wavefield can be reconstructed at high resolution.
		Furthermore, I have compared my developed model of deep learning for super-resolution with the compressive sensing technique, which states that any signal can be reconstructed from a linear combination of random measurements.
		To acquire the low-resolution measurements, I have used two types of masks, as shown in the figure: random and jitter.
		The low-resolution measurements are far fewer than required by the Nyquist-Shannon sampling theory.
	}
	
	%%%%%%%%%%%%%%%%%%%%%%%%%%%%%%%%%%%%%%%%%%%%%%%%%%%%%%%%%%%%%%%%%%%%
	\setcounter{subfigure}{0}
%	\begin{frame}{Evaluation metrics for DLSR model}
%		For evaluating the reconstructed HR full wavefield:
%		\begin{itemize}
%			\item Peak signal-to-noise ratio (PSNR):
%			\begin{equation*}
%				PSNR=20\log_{10}\left(\frac{R}{\sqrt{MSE}}\right)
%				\label{PSNR_}
%			\end{equation*}			
%			\item Pearson correlation coefficient (also known as Pearson's r):
%			\begin{equation*}
%				r_{xy} = \frac{\sum_{i=1}^{n}(x_i - \bar{x})(y_i-\bar{y})}{\sqrt{\sum_{i=1}^{n}(x_i - \bar{x})^2}\sqrt{\sum_{i=1}^{n}(y_i - \bar{y})^2}}
%				\label{Pearson_}
%			\end{equation*}
%		\end{itemize}
%	\end{frame}
%	%%%%%%%%%%%%%%%%%%%%%%%%%%%%%%%%%%%%%%%%%%%%%%%%%%%%%%%%%%%%%%%%%%%%
%	\note
%	{
%		To evaluate the developed model, I used two metrics to measure the quality of the reconstructed high-resolution full wavefield frames:
%		\begin{itemize}
%			\item The first one is the peak signal-to-noise ratio (PSNR), which refers to the maximum possible power of a signal and the power of the distorting noise that affects the quality of its representation.
%			%			This equation depicts the mathematical representation of PSNR, where R is the maximum fluctuation value in the input image, and MSE is the mean squared error between the actual and predicted output.
%			%			
%			\item The second metric is the Pearson correlation coefficient (Pearson CC), which measures the linear relationship between two
%			variable sets \(X\) (represents the ground truth values) and \(Y\) (represents the predicted values) as shown in the below equation.
%			
%			%			Where \(n\) is the number of sample points,\(x_i, y_i\) are the individual value points representing the ground truth and predicted values, respectively, and \(\bar{x}\) and \(\bar{y}\) are the mean values of the sample and the prediction.
%		\end{itemize}
%	}
	%%%%%%%%%%%%%%%%%%%%%%%%%%%%%%%%%%%%%%%%%%%%%%%%%%%%%%%%%%%%%%%%%%%%
%	\setcounter{subfigure}{0}
%	\begin{frame}{Numerical test cases}
%		\begin{columns}[T]				
%			\begin{column}[c]{0.5\textwidth}				
%				\begin{figure}
%					\centering
%					%%%%%%%%%%%%%%%%%%%%%%%%%%%%%%%%%%%%%%%%%%%%%%%%%%%%
%					%					\only<1>{
%					%						\begin{alertblock}{First test case}
%					%							\begin{figure}
%					%								\includegraphics[height=.35\textheight]{LR_456_frame_159_input.png}
%					%								\caption{LR input, $f_n=159$}
%					%							\end{figure}							
%					%							%%%%%%%%%%%%%%%%%%%%%%%%%%%%%%%%%%%%%%%%%%%%
%					%							\begin{table}[!h]
%					%								\centering 
%					%								\footnotesize
%					%								\begin{tabular}{cccc}
%					%									\toprule
%					%									\multicolumn{2}{c}{plate} & \multicolumn{2}{c}{delamination} \\
%					%									\cmidrule(lr){1-2} \cmidrule(lr){3-4}
%					%									PSNR & PEARSON CC & PSNR & PEARSON CC \\ 
%					%									\midrule
%					%									42.95 & 0.999 & 33.02 & 0.993 \\					
%					%									\bottomrule
%					%								\end{tabular}
%					%							\end{table}
%					%						\end{alertblock}}
%					%%%%%%%%%%%%%%%%%%%%%%%%%%%%%%%%%%%%%%%%%%%%%%%%
%					\only<1>{
%						\begin{alertblock}{First test case}
%							\begin{figure}
%								\includegraphics[height=.35\textheight]{LR_438_frame_154_input.png}
%								\caption{LR input, $f_n=154$}
%							\end{figure}							
%							%%%%%%%%%%%%%%%%%%%%%%%%%%%%%%%%%%%%%%%%%%%%
%							\begin{table}[!h]
%								\centering 
%								\footnotesize
%								\begin{tabular}{cccc}
%									\toprule
%									\multicolumn{2}{c}{plate} & \multicolumn{2}{c}{delamination} \\
%									\cmidrule(lr){1-2} \cmidrule(lr){3-4}
%									PSNR & PEARSON CC & PSNR & PEARSON CC \\ 
%									\midrule
%									47.00 & 0.998 & 38.52 & 0.995 \\					
%									\bottomrule
%								\end{tabular}
%								\label{tab:num_DLSR_results_2_}
%							\end{table}	
%							%%%%%%%%%%%%%%%%%%%%%%%%%%%%%%%%%%%%%%%%%%%%
%					\end{alertblock}}
%					%%%%%%%%%%%%%%%%%%%%%%%%%%%%%%%%%%%%%%%%%%%%%%%%%%%%
%					\only<2>{
%						\begin{alertblock}{Second test case}
%							\begin{figure}
%								\includegraphics[height=.35\textheight]{LR_397_frame_127_input.png}
%								\caption{LR input, $f_n=127$}
%							\end{figure}						
%							%%%%%%%%%%%%%%%%%%%%%%%%%%%%%%%%%%%%%%%%%%%%
%							\begin{table}[!h]
%								\centering 
%								\footnotesize	
%								\begin{tabular}{cccc}
%									\toprule
%									\multicolumn{2}{c}{plate} & \multicolumn{2}{c}{delamination} \\
%									\cmidrule(lr){1-2} \cmidrule(lr){3-4}
%									PSNR & PEARSON CC & PSNR & PEARSON CC \\ 
%									\midrule
%									48.60 & 0.998 & 46.67 & 0.998 \\					
%									\bottomrule
%								\end{tabular}
%							\end{table}
%							%%%%%%%%%%%%%%%%%%%%%%%%%%%%%%%%%%%%%%%%%%%%%%%%
%					\end{alertblock}}
%					%%%%%%%%%%%%%%%%%%%%%%%%%%%%%%%%%%%%%%%%%%%%%%%%%%%%
%				\end{figure}
%			\end{column}
%			%%%%%%%%%%%%%%%%%%%%%%%%%%%%%%%%%%%%%%%%%%%%%%%%%%%%%%%%%%%%
%			\begin{column}[c]{0.5\textwidth}
%				%				\only<1>{	
%				%					\setcounter{subfigure}{0}
%				%					\begin{figure}
%				%						\subfloat[listentry][HR ref]{\includegraphics[height=.35\textheight]{output_456_frame_159_full_frame_GT.png}}\quad
%				%						\subfloat[listentry][DLSR]{\includegraphics[height=.35\textheight]{output_456_frame_159_full_frame_pred.png}}
%				%						\\
%				%						\subfloat[listentry][Ref]{\includegraphics[height=.35\textheight]{output_456_frame_159_delamination_GT.png}}\quad
%				%						\subfloat[listentry][DLSR]{\includegraphics[height=.35\textheight]{output_456_frame_159_delamination_pred.png}}
%				%					\end{figure}}			
%				\only<1>{
%					\setcounter{subfigure}{0}					
%					%%%%%%%%%%%%%%%%%%%%%%%%%%%%%%%%%%%%%%%%%%%%%%%%%%%%
%					\begin{figure}
%						\subfloat[listentry][HR ref]{\includegraphics[height=.35\textheight]{output_438_frame_154_full_frame_GT.png}}\quad
%						\subfloat[listentry][DLSR]{\includegraphics[height=.35\textheight]{output_438_frame_154_full_frame_pred.png}}
%						\\
%						\subfloat[listentry][Ref]{\includegraphics[height=.35\textheight]{output_438_frame_154_delamination_GT.png}}\quad	
%						\subfloat[listentry][DLSR]{\includegraphics[height=.35\textheight]{output_438_frame_154_delamination_pred.png}}
%				\end{figure}}
%				\only<2>{	
%					\setcounter{subfigure}{0}
%					\begin{figure}
%						\centering
%						\subfloat[listentry][HR ref]{\includegraphics[height=.35\textheight]{output_397_frame_127_full_frame_GT.png}}\quad
%						\subfloat[listentry][DLSR]{\includegraphics[height=.35\textheight]{output_397_frame_127_full_frame_pred.png}}
%						\\
%						\subfloat[listentry][Ref]{\includegraphics[height=.35\textheight]{output_397_frame_127_delamination_GT.png}}\quad
%						\subfloat[listentry][DLSR]{\includegraphics[height=.35\textheight]{output_397_frame_127_delamination_pred.png}}
%				\end{figure}}
%			\end{column}
%		\end{columns}
%	\end{frame}
%	%%%%%%%%%%%%%%%%%%%%%%%%%%%%%%%%%%%%%%%%%%%%%%%%%%%%%%%%%%%%%%%%%%%%
%	\note
%	{
%		\footnotesize
%		In the following, the results of the reconstruction of HR frames for two numerical test cases will be presented.				
%		
%		In the first test case, the figure shows the low-resolution measurements at frame number 154.		
%		Figure a shows the actual HR frame, and figure b shows the predicted SR frame. 
%		The PSNR value is 47
%		
%		Figure c shows the HR sub-frame at the delamination region, figure d shows the SR prediction at the delamination region, and the PSNR is 38.52.
%		
%		In the second test case, the figure shows the low-resolution measurements at frame number 127.
%		Figure a shows the actual HR frame, and figure b shows the predicted SR frame. 
%		The PSNR value is 42.95
%		
%		Figure c shows the HR sub-frame at the delamination region, figure d shows the SR prediction at the delamination region, and the PSNR is 33.02.
%		
%		
%	}
	%%%%%%%%%%%%%%%%%%%%%%%%%%%%%%%%%%%%%%%%%%%%%%%%%%%%%%%%%%%%%%%%%%%%
	\setcounter{subfigure}{0}
	\begin{frame}{Experimental test case}		
		\begin{columns}[T]
			%%%%%%%%%%%%%%%%%%%%%%%%%%%%%%%%%%%%%%%%%%%%%%%%%%%%%%%%%%%%
			\begin{column}[t]{0.25\textwidth}				
				\begin{figure}	
					\centering					
					\includegraphics[width=1\textwidth]{frame110_32x32.png}
					%					\caption{LR input \((N_f = 110)\)}
				\end{figure}
				\footnotesize
				LR measurements (Input): \(32\times32=1024\)p. \\
				HR (Output): \(512\times512=262144\)p.
			\end{column}
			%%%%%%%%%%%%%%%%%%%%%%%%%%%%%%%%%%%%%%%%%%%%%%%%%%%%%%%%%%%%
			\begin{column}[t]{.25\textwidth}
				\begin{block}{HR label}
					\begin{figure}
						\centering
						\subfloat{\includegraphics[width=0.75\textwidth]{figure10a.png}}
						\vfill
						\subfloat{\includegraphics[width=0.75\textwidth]{figure11a.png}}
					\end{figure}
				\end{block}
			\end{column}
			%%%%%%%%%%%%%%%%%%%%%%%%%%%%%%%%%%%%%%%%%%%%%%%%%%%%%%%%%%%%
			\begin{column}[t]{.25\textwidth}
				\begin{block}{CS: 1024p}
					\begin{figure}
						\centering
						\subfloat{\includegraphics[width=0.75\textwidth]{figure10b.png}}
						\vfill						
						\subfloat{\includegraphics[width=0.75\textwidth]{figure11b.png}}
					\end{figure}
				\end{block}				
			\end{column}
			%%%%%%%%%%%%%%%%%%%%%%%%%%%%%%%%%%%%%%%%%%%%%%%%%%%%%%%%%%%%
			\begin{column}[t]{.25\textwidth}
				\begin{alertblock}{DLSR}
					\begin{figure}
						\centering
						\subfloat{\includegraphics[width=0.75\textwidth]{figure10e.png}}
						\vfill			
						\subfloat{\includegraphics[width=0.75\textwidth]{figure11e.png}}\quad
					\end{figure}
				\end{alertblock}				
				%%%%%%%%%%%%%%%%%%%%%%%%%%%%%%%%%%%%%%%%%%%%%%%%%%%%%%%%%%%%
			\end{column}				
		\end{columns}
	\end{frame}
	%%%%%%%%%%%%%%%%%%%%%%%%%%%%%%%%%%%%%%%%%%%%%%%%%%%%%%%%%%%%%%%%%%%%
	\note{
		\footnotesize
		In this slide, I present an experimental test case,
		
		The figure on the left represents an experimentally acquired full wavefield in it low resolution.	
		
		This Figure shows the actual HR reference frame for the whole plate and at the delamination area
		
		The figure in the middle represents the outputs of applying the compressive sensing technique in which it shows a poor quality of reconstruction. 
		
		
		The figure on the right presents the reconstructed HR frame with the DLSR model for the whole plate and at the delamination area as you can see the quality of reconstruction is very noticeable. 
	}
	%%%%%%%%%%%%%%%%%%%%%%%%%%%%%%%%%%%%%%%%%%%%%%%%%%%%%%%%%%%%%%%%%%%%
	\begin{frame}{Analysis of experimental case}
		\begin{table}[!ht]
			\renewcommand{\arraystretch}{1.3}
			\centering \footnotesize
			\caption{Quality metrics for tested methods.}	
			\begin{tabular}{lrrrcrc} 
				\toprule[1.5pt]
				& & & \multicolumn{2}{c}{plate} & \multicolumn{2}{c}{delamination} \\
				\cmidrule(lr){4-5} \cmidrule(lr){6-7}
				Method & $N_p$ & CR [\%] & PSNR & PEARSON CC& PSNR & PEARSON CC \\
				\midrule
				\csvreader
				[table head=\toprule,
				late after line=\\ 
				]{table_metrics.csv}{
					1=\one, 2=\two, 3=\three, 4=\four, 5=\five, 6=\six, 7=\seven
				}%
				{\one & \two & \three & \four & \five & \six & \seven }%	
				\bottomrule[1.5pt]
			\end{tabular}	
			\label{tab:csv_results_}
		\end{table}
	\end{frame}
	%%%%%%%%%%%%%%%%%%%%%%%%%%%%%%%%%%%%%%%%%%%%%%%%%%%%%%%%%%%%%%%%%%%%
	\note{
		The following table presents a detailed comparison of the quality metrics for CS methods with applied jitter and random masks and the DLSR model.
		As shown in the table the DLSR model achieved the highest PSNR and Pearson CC values.		
	}
	%%%%%%%%%%%%%%%%%%%%%%%%%%%%%%%%%%%%%%%%%%%%%%%%%%%%%%%%%%%%%%%%
	\section{Conclusions}
	%%%%%%%%%%%%%%%%%%%%%%%%%%%%%%%%%%%%%%%%%%%%%%%%%%%%%%%%%%%%%%%%
	\begin{frame}{Conclusions}		
%		\begin{footnotesize}
			\begin{justify}
				\settowidth{\leftmargini}{\usebeamertemplate{itemize item}}
				\addtolength{\leftmargini}{\labelsep}
				\begin{itemize}
					\item{Deep learning models trained on synthetic dataset generalise well and can be applied directly to experimental wavefields.}			
					\item{Deep learning approaches surpass conventional approaches for delamination identification.}	
					\item{Animation-based deep learning models perform better than image-based models but are more complex and require longer time for training.}					
					\item{Deep learning models can be generalised to other types of defects and materials.}
					\item{The DLSR model can recover the HR measurements from LR measurements with good accuracy.}
				\end{itemize}
			\end{justify}				
%			\begin{tcolorbox}
%				\begin{justify}
%					\textbf{According to my research study, investigations and results, I confirm my original thesis:}
%					\begin{alertblock}{Thesis}
%						It is possible to use an end-to-end approach in which DNN 
%						processes the animation of propagating waves (input) directly into a damage map (output).
%					\end{alertblock}
%				\end{justify}			
%			\end{tcolorbox}				
%		\end{footnotesize}			
	\end{frame}			
	\note{}	
%	%%%%%%%%%%%%%%%%%%%%%%%%%%%%%%%%%%%%%%%%%%%%%%%%%%%%%%%%%%%%%%%%
%	\begin{frame}{Future works}
%		\begin{columns}[T]
%			\begin{column}[t]{0.49\textwidth}
%				\centering
%				\textbf{\underline{Metamaterials: Dispersion diagrams}}
%				\begin{figure}
%					\centering
%					\efbox{\includegraphics[width=1\textwidth]{Surrogate_DL_model_for_PC_Abdalraheem.png}}
%				\end{figure}
%			\end{column}
%			\begin{column}[t]{0.49\textwidth}
%				\centering
%				\textbf{\underline{COMSOL vs. surrogate DL model predictions}}
%				\begin{figure}
%					\centering
%					\efbox{\includegraphics[width=1\textwidth]{plot_1029_4672_10425_KF_DL_FEM_BG_triple_tile.png}}							
%				\end{figure}			
%			\end{column}		
%		\end{columns}				
%	\end{frame}	
%	\note{}		
	%%%%%%%%%%%%%%%%%%%%%%%%%%%%%%%%%%%%%%%%%%%%%%%%%%%%%%%%%%%%%%%%
%	\begin{frame}{Publications}
%		\vspace{5pt}
%		\settowidth{\leftmargini}{\usebeamertemplate{itemize item}}
%		\addtolength{\leftmargini}{\labelsep}
%		\begin{tiny}					
%			\begin{columns}[T]
%				\begin{column}[t]{0.48\textwidth}
%					\underline{\textbf{Journal articles}}
%					\begin{enumerate}
%						\justifying
%						%%%%%%%%%%%%%%%%%%%%%%%%%%%%%%%%%%%%%%%%%%%%%%%%
%						\item Ijjeh, A., Ullah, S., Radzienski, M. and Kudela, P., 2023. Deep learning super-resolution for the reconstruction of full wavefield of Lamb waves. \textbf{\textit{Mechanical Systems and Signal Processing}}, 186, p.109878.						
%						\textbf{[200~points]/[IF:8.934]}
%						%%%%%%%%%%%%%%%%%%%%%%%%%%%%%%%%%%%%%%%%%%%%%%%%
%						\item Ullah, S., Ijjeh, A.A. and Kudela, P., 2023. Deep learning approach for delamination identification using animation of Lamb waves. 						
%						\textbf{\textit{Engineering Applications of Artificial Intelligence}}, 117, p.105520.		
%						\textbf{[140~points]/[IF:7.802]}
%						%%%%%%%%%%%%%%%%%%%%%%%%%%%%%%%%%%%%%%%%%%%%%%%%
%						\item Ijjeh, A.A. and Kudela, P., 2022. Deep learning based segmentation using full wavefield processing for delamination identification: A comparative study. \textbf{\textit{Mechanical Systems and Signal Processing}}, 168, p.108671. \textbf{[200~points]/[IF:8.934]}
%						%%%%%%%%%%%%%%%%%%%%%%%%%%%%%%%%%%%%%%%%%%%%%%%%
%						\item Ijjeh, A.A., Ullah, S. and Kudela, P., 2021. Full wavefield processing by using FCN for delamination detection. \textbf{\textit{Mechanical Systems and Signal Processing}}, 153, p.107537.		
%						\textbf{[200~points]/[IF:8.934]}	
%						%%%%%%%%%%%%%%%%%%%%%%%%%%%%%%%%%%%%%%%%%%%%%%%%
%					\end{enumerate}					
%				\end{column}
%				\begin{column}[t]{0.48\textwidth}
%					\underline{\textbf{Conference papers}}
%					\begin{enumerate}
%						\justifying
%						\item {Ijjeh, A.}, Kudela, P. Convolutional LSTM for delamination imaging in composite laminates. 
%						The 4th International Conference on Machine Learning and Intelligent Systems (MLIS 2022), November \(8^{th}\) - \(11^{th}\), 2022, Seoul, Republic of Korea.
%						\item Ijjeh, A. and Kudela, P., 2022, June. Delamination Identification Using Global Convolution Networks. 
%						In European Workshop on Structural Health Monitoring: EWSHM 2022-Volume 3 (pp. 521-529). Cham: Springer International Publishing.		
%						\item {Ijjeh, A.}, Kudela, P. Feasibility Study of Full Wavefield Processing by Using CNN for Delamination Detection. 
%						Proceedings of the International Conference on Structural Health Monitoring of Intelligent
%						Infrastructure, June \(30^{th}\) - July \(2^{nd}\), 2021, Porto, Portugal, ISSN 2564-3738, pages 709-713.
%					\end{enumerate}		
%					\underline{\textbf{Chapters}}					
%					\begin{enumerate}
%						\justifying
%						\item {Abdalraheem Ijjeh}, Deep Learning based Damage Imaging techniques, chapter in: Wybrane zagadnienia
%						inżynierii mechanicznej, Praca zbiorowa pod redakcja M.~Mieloszyk, T. Ochrymiuka, Wydawnictwo Instytutu
%						Maszyn Przepływowych PAN, Gdańsk, 2022, ISBN: 978-83-66928-09-1.
%						\item {Abdalraheem Ijjeh}, Data-driven based approach for damage detection, chapter in: Wybrane zagadnienia
%						inżynierii mechanicznej, Praca zbiorowa pod redakcja M.~Mieloszyk, T. Ochrymiuka, Wydawnictwo Instytutu
%						Maszyn Przepływowych PAN, Gdańsk, 2021, ISBN: 978-83-66928-00-8.				
%						\item {Abdalraheem Ijjeh}, Machine Learning for SHM: Literature Review, chapter in: Wybrane zagadnienia
%						inżynierii mechanicznej, Praca zbiorowa pod redakcja M.~Mieloszyk, T. Ochrymiuka, Wydawnictwo Instytutu
%						Maszyn Przepływowych PAN, Gdańsk, 2020, ISBN: 978-83-88237-97-3.
%					\end{enumerate}
%				\end{column}		
%			\end{columns}
%		\end{tiny}
%	\end{frame}	
%	\note{}	
	%%%%%%%%%%%%%%%%%%%%%%%%%%%%%%%%%%%%%%%%%%%%%%%%%%%%%%%%%%%%%%%%%%%%
	\setcounter{subfigure}{0}
	%%%%%%%%%%%%%%%%%%%%%%%%%%%%%%%%%%%%%%%%%%%%%%%%%%%%%%%%%%%%%%%%%%%%
	{
		\setbeamercolor{palette primary}{fg=blue, bg=white}
		\begin{frame}[standout]
			\centering
			Thank you for your listening!\\ \vspace{12pt}
			Questions?\\ \vspace{12pt}
			\url{pk@imp.gda.pl}
			\par\medskip
			\par\medskip
			\footnotesize
			The research work was funded by the Polish National Science Center under grant agreement no. 2018/31/B/ST8/00454.
		\end{frame}
	}
	\note{}	
	%	%%%%%%%%%%%%%%%%%%%%%%%%%%%%%%%%%%%%%%%%%%%%%%%%%
\end{document}