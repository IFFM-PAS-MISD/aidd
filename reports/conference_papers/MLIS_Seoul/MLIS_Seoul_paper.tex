
\documentclass{IOS-Book-Article}

\usepackage{mathptmx}
\usepackage{soul}\setuldepth{article}
\usepackage{graphicx}
\usepackage{subcaption}
%\usepackage{times}
%\normalfont
%\usepackage[T1]{fontenc}
%\usepackage[mtplusscr,mtbold]{mathtime}
%
\def\hb{\hbox to 10.7 cm{}}

\begin{document}

\pagestyle{headings}
\def\thepage{}

\begin{frontmatter}              % The preamble begins here.


%\pretitle{Pretitle}
\title{Convolutional LSTM for delamination imaging in composite laminates}

\markboth{}{July 2022\hb}
%\subtitle{Subtitle}

\author[A]{
	\fnms{Pawe{\l}} \snm{Kudela}%
\thanks{Institute of Fluid Flow Machinery, Polish Academy of Sciences, Poland; E-mail: pk@imp.gda.pl}
},
\author[A]{\fnms{Abdalraheem} \snm{Ijjeh}}
%and
%\author[B]{\fnms{Abdalraheem} \snm{Ijjeh}}

\runningauthor{P. Kudela et al.}
\address[A]{Institute of Fluid Flow Machinery, Polish Academy of Sciences, Poland}


\begin{abstract}
	Due to the complex design of composite structures, typical visual inspection techniques experience some difficulty with damage detection.
	As a result, several nondestructive testing and structural health monitoring approaches for detecting damage in composite structures have been developed, with ultrasonic guided waves, particularly Lamb waves, gaining popularity.
	Apart from piezoelectric pointwise measurements, laser Doppler vibrometry has been used in recent years for full wavefield measurements of propagating Lamb waves.
	Damage imaging can be performed through animations of Lamb waves interacting with damage.
	Accordingly, an end-to-end deep learning-based model of many-to-one sequence prediction was built in this work to perform pixel-wise image segmentation.
	The developed model performed well on numerically generated test data.
	This approach can automate delamination identification and generate damage maps without requiring user involvement.
\end{abstract}

\begin{keyword}
	Lamb waves \sep delamination identification \sep semantic 
	segmentation \sep deep learning \sep ConvLSTM.
\end{keyword}
\end{frontmatter}
\markboth{July 2022\hb}{July 2022\hb}
%\thispagestyle{empty}
%\pagestyle{empty}

\section{Introduction}
Composite materials are widely utilised in many distinct life applications such as aeronautics, wind turbines, among others, due to their lightweight, excellent fatigue and corrosion resistance.
However, composite materials could encounter different types of defects, such as matrix cracks, fibre breakage, debonding, and delamination~\cite{smith2009composite, ip2004delamination}. 
Among these defects, delamination (separation of layers from each other in a laminate composite) is one of the most hazardous since it mostly occurs below the top surfaces and is barely visible~\cite{Cai2012a}.
Delamination in composite materials can occur and develop from various sources, such as notches,  manufacturing defects, and impact events. Therefore, delamination can decrease the strength and performance of the structure. 
Accordingly, delamination identification is necessary to prevent such consequences. 
Accordingly, several physics-based methods for damage detection and localisation have been developed in the fields of Structural Health Monitoring (SHM) and Non-Destructive Testing (NDT) to monitor the integrity of structures.

Guided waves, particularly Lamb waves, are used in a well-known physics-based approach for damage detection in the field of SHM.
Lamb waves are elastic waves that propagate through thin plates and shells bounded by stress-free surfaces\cite{mitra2016guided}.
Lamb waves are distinguished by their strong sensitivity to discontinuities (cracks, delaminations) and low amplitude loss, especially in metallic structures~\cite{Keulen2014}.

Lamb waves can be generated by exciting the inspected structure with an array of PZT transducers, and then registering the reflected waves from damage.
Following that, a damage influence map is created.
The number of sensing points determines the accuracy of the damage influence map, which reveals the location of the damage.
As a result, the resolution of damage localisation can be low.
To measure Lamb waves in a dense grid of locations over the structure under inquiry, a Scanning Laser Doppler Vibrometer (SLDV) is employed.
Measurements of full wavefield propagation are collected, resulting in high-resolution damage influence maps.
Damage detection approaches that utilise full wavefield signals can estimate the size and location of damage accurately~\cite{Girolamo2018a, kudela2018impact}.

However, when dealing with large amounts of acquired data, traditional techniques have limitations as they require complex feature engineering computations.
They also require a high level of experience as well as the ability to extract damage-sensitive characteristics for particular SHM applications.
As a result, in recent years, a data-driven approach for SHM applications has attracted attention in the form of deep learning (DL) end-to-end methods, with the process of feature engineering and classification performed automatically.

Several DL techniques for guided wave-based damage identification and localisation are presented here.
Authors in~\cite{DeFenza2015a} applied a probability ellipse-based approach and 
basic ANNs with the use of guided Lamb waves in order to  determine the degree 
and location of damage in composite and metallic plates. Results of both of 
their approaches showed that guided Lamb waves have noticeable advantages in 
the detection and localization of micro-scale defects in plate-like 
structures. 

Furthermore, authors~\cite{okafor1996delamination} developed a back-propagation neural network to estimate the size of delamination in a smart composite beam.
As a result, delamination sizes for the first four modal frequencies were used to train their model, that can predict delamination of sizes ranging from (0.22) cm to (0.82) cm.

Their model was developed using a convolutional neural network (CNN) capable of image segmentation to detect delamination.

However, their model has difficulty detecting delaminations of large sizes.
Gopalakrishnan et al.~\cite{Gopalakrishnan2021} utilised guided Lamb waves to develop a deep learning-based technique for detecting structural defects in composite materials.
On two datasets, convolutional neural networks (CNNs) and regression-based CNNs were used to detect flaws, while long-short-term memory (LSTM) models were used to localize errors.
They demonstrated that deep learning-based approaches outperformed traditional machine-learning techniques.

In this work, we applied a deep learning-based semantic segmentation model to the full wavefield frames of propagating Lamb waves to identify delamination in CFRP plates.
The proposed deep learning model used a many-to-one prediction technique.
Accordingly, the model is fed with a sequence of full wavefield frames (animation).
The model is based on convolutional long-short-term memory (ConvLSTM) architectures and is modified to the specific objective of delamination identification.
The developed ConvLSTM model can provide accurate results for identifying delamination in composite structures.

\section{Methodology}

%%%%%%%%%%%%%%%%%%%%%%%%%%%%%%%%%%%%%%%%%%%%%%%%%%%%%%%%%%%%%%%%%%%%%%%%%%%%%%%%
\subsection{Data preprocessing}
%%%%%%%%%%%%%%%%%%%%%%%%%%%%%%%%%%%%%%%%%%%%%%%%%%%%%%%%%%%%%%%%%%%%%%%%%%%%%%%%
Similar to the previous work~\cite{Ijjeh2021}, 475 cases were simulated, representing Lamb wave propagation and interaction with single delamination for each case. 

It should be underlined that the previous dataset contained the RMS of the full wavefield, representing wave energy spatial distribution in the form of images for each delamination case~\cite{Kudela2020d}.
On the other hand, the currently utilised dataset contains frames of propagating waves (512 frames for each delamination scenario).
The new dataset is available online~\cite{Kudela2021}.

As mentioned earlier, the dataset contains 475 different cases of delaminations, with 512 frames per case, producing a total number of 243,\,200 frames with a frame size of \((500\times500)\)~pixels representing the geometry of the specimen of size \((500\times500)\)~mm\(^{2}\).
Thus, using all frames in each case has high computational and memory costs.
Frames displaying the propagation of guided waves before interaction with the delamination have no features to be extracted (see Fig.~\ref{fig:Full_wave}).
Hence, for training, only a certain number of frames were selected from the initial occurrence of the interactions with the delamination.

Figure~\ref{fig:Full_wave} shows selected frames at different time-steps of the propagating Lamb waves before and after the interaction with the damage.
Frame \(f_{1}\) represents the initial interactions with the delamination, which was calculated using the delamination location and the velocity of the \(A0\) Lamb wave mode.
While frame \(f_{m}\) represents the last frame in the training sequence window, \(m=24\) for the developed model which will be discussed in the next subsection.
\begin{figure}[!h]
	\centering
	\includegraphics[width=.8\textwidth]{Graphics/figure2.png}
	\caption{Sample frames of full wave propagation.}
	\label{fig:Full_wave}
\end{figure}

Furthermore, the dataset was divided into two sets: training and testing, with a ratio of \(80\%\) and \(20\% \) respectively.
Moreover, a certain portion of the training set was preserved as a validation set to validate the model during the training process.
Additionally, the dataset was normalised to a range of \((0, 1)\) to improve the convergence of the gradient descent algorithm.

Additionally, for the training purposes, I have upsampled the frames (by using cubic interpolation) to \(512\times512\)~pixels to maintain the symmetrical shape during the encoding and decoding process.
Further, the validation sets have portions of \(10\%\) and \(20\%\) regarding the training set.
%%%%%%%%%%%%%%%%%%%%%%%%%%%%%%%%%%%%%%%%%%%%%%%%%%%%%%%%%%%%%%%%%%%%%%%%%%%%%%%%

Figure~\ref{fig:Diagram_exp_predictions} illustrates the complete procedure of obtaining intermediate predictions for the testing cases and finally calculating the RMS image, where \(f_{1}\) refers to the starting frame and \(f_{n}\) is the last frame, (\(n=512\)) in our dataset.
Further, \(m\) refers to the number of frames in the window, hence, \(m=24\) frames for the developed model, and \(k\) represents the total number of windows.
Accordingly, I slide the window over all input frames.
The shift of the window is one frame at a time.
Deep learning model predictions \(\hat{Y_k}\) are obtained for each window and combined to the final damage map by using the $RMS$:

\begin{equation}
	RMS = \sqrt{\frac{1}{N}\sum_{k=1}^{N}\hat{Y_k}^2}.	
	\label{RMS}
\end{equation}
%%%%%%%%%%%%%%%%%%%%%%%%%%%%%%%%%%%%%%%%%%%%%%%%%%%%%%%%%%%%%%%%%%%%%%%%%%%%%%%%
\begin{figure}[!h]
	\centering
	\includegraphics[width=.65\textwidth]{Graphics/figure3_diagram.png}
	\caption{The procedure of calculating the RMS prediction image (damage map).}
	\label{fig:Diagram_exp_predictions}
\end{figure}
%%%%%%%%%%%%%%%%%%%%%%%%%%%%%%%%%%%%%%%%%%%%%%%%%%%%%%%%%%%%%%%%%%%%%%%%%%%%%%%%
%%%%%%%%%%%%%%%%%%%%%%%%%%%%%%%%%%%%%%%%%%%%%%%%%%
\section{PROPOSED MODEL}\label{sec:section4}
In this work, we applied an end-to-end deep learning-based model by utilising 
full wavefield frames of guided Lamb waves propagation for delamination 
identification in CFRP composite plates.
%%%%%%%%%%%%%%%%%%%%%%%%%%%%%%%%%%%%%%%%%%%%%%%%%%
\begin{figure} [h!]
	\begin{center}
		\includegraphics[width=0.25\textwidth]{Graphics/figure5b.png}
	\end{center}
	\caption{The architecture of the proposed deep learning model.} 
	\label{fig:AE_convlstm}
\end{figure}
%%%%%%%%%%%%%%%%%%%%%%%%%%%%%%%%%%%%%%%%%%%%%%%%%%

In the developed model presented in Fig.~\ref{fig:AE_convlstm}, we applied an autoencoder technique (AE) which is well-known for extracting spatial features. 
The idea of AE is to compress the input data within the encoding process and then learn how to reconstruct it back from the reduced encoded representation (latent space) to a representation that is as close to the original input as possible. 
In general, an AE consists of three parts: the encoder, the bottleneck, and the decoder.
The encoder is responsible for learning how to reduce the input dimensions and compress the input data into an encoded representation.
The bottleneck presented in Fig.~\ref{fig:AE_convlstm} has the lowest level of dimensions of the input data.
The decoder part presented in Fig.~\ref{fig:AE_convlstm} is responsible for learning how to restore the original dimensions of the input.
A time-dispersed layer was introduced to the model to distribute the input frames into the AE layers in order to process them independently.
In this model, we have investigated the use of AE to process a sequence of \(24\) frames to perform delamination identification.

To classify the predicted output into undamaged (represented by \(0\)) or 
damaged (represented by \(1\)), a threshold value of \(0.5\) was used. 
The outputs values less than the threshold value were considered as undamaged 
and the values greater than the threshold were considered as damaged.

For the evaluation of the performance of the proposed model, the mean 
intersection over union \(IoU\), also known as Jaccard index was applied as the 
accuracy metric. 
\(IoU\) is calculated by determining the intersection area between the 
predicted output and the ground truth. 
Further, we have two output classes (damaged and undamaged), the \(IoU\) was 
calculated for the damaged class only. 
Equation~(\ref{eqn:iou}) illustrates the \(IoU\) metric: 
\begin{equation}
	IoU=\frac{\hat{Y} \cap Y}{\hat{Y} \cup Y}
	\label{eqn:iou}
\end{equation}
where \(Y\) represents the ground truth, and \(\hat{Y}\) is the predicted 
output.
%%%%%%%%%%%%%%%%%%%%%%%%%%%%%%%%%%%%%%%%%%%%%%%%%%%%%%%%%%%%%%%%%%%%%%%%%%%%%%%%
\section{Results and discussion}
In this section, we present the evaluation of the proposed model based on numerical data of \(95\) different cases representing the frames of the full wavefield propagation.
The developed model was evaluated using numerical data to demonstrate its capability to predict delamination location, shape, and size.
To show the performance of the developed model, three representative cases were selected as shown in Fig.~\ref{fig:GT}.
Figure~\ref{fig:num_GT_381} presents the ground truth (label) of the first test case in which the delamination is located in the upper left quarter.
In the second test case the delamination is located near the upper left corner of the plate as shown in Fig.~\ref{fig:num_GT_385}.
The third test case has a delamination located at the upper left quarter near the edge of the plate as shown in Fig.~\ref{fig:num_GT_394}.
\begin{figure}[ht!]
	\begin{subfigure}[b]{0.32\textwidth}
		\centering
		\includegraphics[width=.8\textwidth]{Graphics/m1_rand_single_delam_381.png}
		\caption{1\textsuperscript{st} case: Label}
		\label{fig:num_GT_381}
	\end{subfigure}
	\hfill
	\begin{subfigure}[b]{0.32\textwidth}
		\centering
		\includegraphics[width=.8\textwidth]{Graphics/m1_rand_single_delam_385.png}
		\caption{2\textsuperscript{nd} case: Label}
		\label{fig:num_GT_385}
	\end{subfigure}
	\hfill
	\begin{subfigure}[b]{0.32\textwidth}
		\centering
		\includegraphics[width=.8\textwidth]{Graphics/m1_rand_single_delam_394.png}
		\caption{3\textsuperscript{rd} case: Label}
		\label{fig:num_GT_394}
	\end{subfigure}
	\caption{Ground truths (Labels)}
	\label{fig:GT}
\end{figure}

Figure~\ref{fig:num_cases} presents the predicted RMS outputs (damage maps) for the three test cases presented in Fig.~\ref{fig:GT}.
All predicted RMS images contain less noise, and delaminations are visible with the naked eye, as shown in Figs.~\ref{fig:num_1st_case}, \ref{fig:num_2nd_case} and \ref{fig:num_3rd_case}.
\begin{figure}[ht!]
	\centering
	\begin{subfigure}[b]{0.32\textwidth}
		\centering
		\includegraphics[width=1\textwidth]{Graphics/RMS_Ijjeh_num_case_381.png}
		\caption{1\textsuperscript{st} case, RMS}
		\label{fig:num_1st_case}
	\end{subfigure}
	\hfill
	\begin{subfigure}[b]{0.32\textwidth}
		\centering
		\includegraphics[width=1\textwidth]{Graphics/RMS_Ijjeh_num_case_385.png}
		\caption{2\textsuperscript{nd} case, RMS}
		\label{fig:num_2nd_case}
	\end{subfigure}
	\hfill
	\begin{subfigure}[b]{0.32\textwidth}
		\centering
		\includegraphics[width=1\textwidth]{Graphics/RMS_Ijjeh_num_case_394.png}
		\caption{3\textsuperscript{rd} case, RMS}
		\label{fig:num_3rd_case}
	\end{subfigure}
	\caption{RMS images (damage maps)}
	\label{fig:num_cases}
\end{figure}

Furthermore, a binary threshold was applied to eliminate excess noise from the predicted damage maps to make the delaminations more visible as shown in Fig.~\ref{fig:RMS_num_cases}.
The IoU values for the 1\textsuperscript{st}, 2\textsuperscript{nd}, and the 3\textsuperscript{rd} test cases are $0.88$, $0.58$ and $0.8$, as shown in Fig.~\ref{fig:Binary_RMS_381},~\ref{fig:Binary_RMS_385} and~\ref{fig:Binary_RMS_394}, respectively.
\begin{figure}
	\begin{subfigure}[b]{0.32\textwidth}
		\centering
		\includegraphics[width=.8\textwidth]{Graphics/Binary_RMS_Ijjeh_num_case381_.png}
		\caption{1\textsuperscript{st} case, IoU\(=0.88\)}
		\label{fig:Binary_RMS_381}
	\end{subfigure}
	\hfill	
	\begin{subfigure}[b]{0.32\textwidth}
		\centering
		\includegraphics[width=.8\textwidth]{Graphics/Binary_RMS_Ijjeh_num_case385_.png}
		\caption{2\textsuperscript{nd} case, IoU\(=0.58\)}
		\label{fig:Binary_RMS_385}
	\end{subfigure}
	\hfill
	\begin{subfigure}[b]{0.32\textwidth}
		\centering
		\includegraphics[width=.8\textwidth]{Graphics/Binary_RMS_Ijjeh_num_case394_.png}
		\caption{3\textsuperscript{rd} case, IoU\(=0.8\)}
		\label{fig:Binary_RMS_394}
	\end{subfigure}
	\caption{Binary RMS predictions}
	\label{fig:RMS_num_cases}
\end{figure}

\section{Conclusions}
	In this work, we present a novel deep learning-based approach for delamination identification in composite laminates.
	The developed approach introduces an end-to-end scheme that performs a many-to-one sequence prediction to identify delamination location, size, and shape.
	Accordingly, we trained our models on a consecutive number of frames depicting the full wavefield of Lamb waves propagation in a CFRP plate, and their interactions with the delamination, and the edges.
	Hence, the models learn how to extract the valuable features regarding the damage from such frames in order to have a prediction.

%\begin{thebibliography}{8}
%	\bibitem{mitra2016guided}
%	Mitra, Mira and Gopalakrishnan, S: Guided wave based structural health monitoring: A review. Smart Materials and Structures. (2016)
%	\bibitem{Giurgiutiu2015}
%	Giurgiutiu, V.: Structural health monitoring of aerospace composites (2015).
%	
%	\bibitem{Sridharan2008}
%	Sridharan, S.: Delamination behaviour of composites. Elsevier (2008).
%	
%	\bibitem{Mei2019}
%	Mei, H., Migot, A., Haider, M. F., Joseph, R., Bhuiyan, M. Y., Giurgiutiu, 
%	V.: Vibration-based in-situ detection and quantification of delamination in 
%	composite plates. Sensors, \textbf{19}(7), 1734 (2019).
%	
%	\bibitem{Jih1993}
%	Jih, C. J., Sun, C. T.: Prediction of delamination in composite laminates 
%	subjected to low velocity impact. Journal of composite materials, 
%	\textbf{27}(7), 684--701 (1993).
%	
%	\bibitem{Sohn2011}
%	Sohn, H., Dutta, D., Yang, J. Y., DeSimio, M., Olson, S., Swenson, E.: 
%	Automated detection of delamination and disbond from wavefield images obtained 
%	using a scanning laser vibrometer. Smart Materials and Structures, 
%	\textbf{20}(4), 
%	045017 (2011).
%	
%	\bibitem{Khan2018}
%	Khan, A., Kim, H. S.: Assessment of delaminated smart composite laminates via 
%	system identification and supervised learning. Composite Structures, 206 
%	354--362 (2018).
%	
%	\bibitem{Harb2016}
%	Harb MS, Yuan FG.: Non-contact ultrasonic technique for Lamb wave 
%	characterization in composite plates. Ultrasonics, \textbf{64}(162--9), (2016).
%	
%	\bibitem{Cawley2003}
%	Cawley, P., Lowe, M. J. S., Alleyne, D. N., Pavlakovic, B., Wilcox, P.: 
%	Practical long range guided wave inspection-applications to pipes and rail. 
%	Mater. Eval, \textbf{61}(1), 66--74 (2003).
%	
%	\bibitem{Kudela2019}
%	Radzieński, M., Kudela, P., Marzani, A., De Marchi, L., Ostachowicz, W.: Damage 
%	identification in various types of composite plates using guided waves excited 
%	by a piezoelectric transducer and measured by a laser vibrometer. Sensors,  
%	\textbf{19}(9), 1958 (2019).
%	
%	\bibitem{Kudela2018}
%	Kudela, P., Radzienski, M., Ostachowicz, W.: Impact induced damage assessment 
%	by means of Lamb wave image processing. Mechanical Systems and Signal 
%	Processing, (102), 23--36 (2018).
%	
%	\bibitem{Girolamo2018}
%	Girolamo, D., Chang, H. Y., Yuan, F. G.: Impact damage visualization in a 
%	honeycomb composite panel through laser inspection using zero-lag 
%	cross-correlation imaging condition. Ultrasonics, (87), 152--165 (2018).
%	
%	\bibitem{Deng2014}
%	Deng, L., Yu, D.: Deep learning: methods and applications. Foundations and 
%	trends in signal processing, \textbf{7}(3–4), 197--387 (2014).
%	
%	\bibitem{Islam1994}
%	Islam, A. S., Craig, K. C.: Damage detection in composite structures using 
%	piezoelectric materials (and neural net). Smart Materials and Structures, 
%	\textbf{3}(3), 318 (1994).
%	
%	\bibitem{Khan2019}
%	Khan, A., Ko, D. K., Lim, S. C., Kim, H. S.: Structural vibration-based 
%	classification and prediction of delamination in smart composite laminates 
%	using deep learning neural network. Composites Part B: Engineering, 161, 
%	586--594 (2019).
%	
%	\bibitem{Okafor1996}
%	Okafor, A. C., Chandrashekhara, K., Jiang, Y. P.: Delamination prediction in 
%	composite beams with built-in piezoelectric devices using modal analysis and 
%	neural network. Smart materials and structures, \textbf{5}(3), 338  (1996).
%	
%	\bibitem{Chetwynd2008}
%	Chetwynd, D., Mustapha, F., Worden, K., Rongong, J. A., Pierce, S. G., 
%	Dulieu‐Barton, J. M.: Damage localisation in a stiffened composite panel. 
%	Strain, 44(4), 298--307 (2008).
%	
%	\bibitem{Fenza2015}
%	De Fenza, A., Sorrentino, A., Vitiello, P. Application of Artificial Neural 
%	Networks and Probability Ellipse methods for damage detection using Lamb waves. 
%	Composite Structures, 133, 390--403 (2015).
%	
%	\bibitem{Feng2019}
%	Feng, B., Pasadas, D. J., Ribeiro, A. L., Ramos, H. G.: Locating defects in 
%	anisotropic CFRP plates using ToF-based probability matrix and neural networks. 
%	IEEE Transactions on Instrumentation and Measurement, \textbf{68}(5), 
%	1252--1260 (2019).
%	
%	\bibitem{Rautela2021}
%	Rautela, M., Gopalakrishnan, S.: Ultrasonic guided wave based structural damage 
%	detection and localization using model assisted convolutional and recurrent 
%	neural networks. Expert Systems with Applications, 167, 114189 (2021).
%	
%	\bibitem{Shi2018}
%	Shi, X., Chen, Z., Wang, H., Yeung, D. Y., Wong, W. K., Woo, W. C.: 
%	Convolutional LSTM network: A machine learning approach for precipitation 
%	nowcasting. Advances in neural information processing systems, 28 (2015).
%	
%	\bibitem{Kudela2020}
%	Kudela, P., Moll, J., Fiborek, P.: Parallel spectral element method for guided 
%	wave based structural health monitoring. Smart Materials and Structures, 
%	\textbf{29}(9), 095010 (2020).
%	
%	\bibitem{Kudela2021}
%	Pawel, K., Abdalraheem I.: Synthetic dataset of a full wavefield
%	representing the propagation of Lamb waves and their interactions with
%	delaminations, (2021).
%	
%	\bibitem{Bengio1994}
%	Bengio, Y., Simard, P., Frasconi, P.: Learning long-term dependencies with 
%	gradient descent is difficult. IEEE transactions on neural networks, 
%	\textbf{5}(2), 157--166 (1994).
%	
%	\bibitem{Hochreiter1997}
%	Hochreiter, S., Schmidhuber, J.: Long short-term memory. Neural computation, 
%	\textbf{9}(8), 1735--1780 (1997).
%	
%	\bibitem{Graves2014}
%	Graves, A., Jaitly, N.: Towards end-to-end speech recognition with recurrent 
%	neural networks. In International conference on machine learning. pp. 
%	1764--1772. PMLR (2014, June).
%	
%	\bibitem{Cho2014}
%	Cho, K., Van Merriënboer, B., Bahdanau, D., Bengio, Y.: On the properties of 
%	neural machine translation: Encoder-decoder approaches. arXiv preprint 
%	arXiv:1409.1259  (2014).
%	
%
%	
%\end{thebibliography}
%%%%%%%%%%%%%%%%%%%%%%%%%%%%%%%%%%%%%%%%%%%%%%%%%%%%%%%%%%%%%%%%%%%%%%%%%%%%%%
\section*{Acknowledgments}
The research was funded by the Polish National Science Center under grant agreement no 2018/31/B/ST8/00454.

\bibliography{ref.bib}
\bibliographystyle{unsrt}
\end{document}
