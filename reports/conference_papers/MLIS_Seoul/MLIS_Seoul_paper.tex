
\documentclass{IOS-Book-Article}

\usepackage{mathptmx}
\usepackage{soul}\setuldepth{article}
\usepackage{graphicx}
\usepackage{subcaption}
%\usepackage{times}
%\normalfont
%\usepackage[T1]{fontenc}
%\usepackage[mtplusscr,mtbold]{mathtime}
%
\def\hb{\hbox to 10.7 cm{}}

\begin{document}

\pagestyle{headings}
\def\thepage{}

\begin{frontmatter}              % The preamble begins here.


%\pretitle{Pretitle}
\title{Convolutional LSTM for delamination imaging in composite laminates}

\markboth{}{July 2022\hb}
%\subtitle{Subtitle}

\author[A]{
	\fnms{Pawe{\l}} \snm{Kudela}%
\thanks{Institute of Fluid Flow Machinery, Polish Academy of Sciences, Poland; E-mail: pk@imp.gda.pl}
},
\author[B]{\fnms{Saeed} \snm{Ullah}}
and
\author[B]{\fnms{Abdalraheem} \snm{Ijjeh}}

\runningauthor{P. Kudela et al.}
\address[A]{Institute of Fluid Flow Machinery, Polish Academy of Sciences, Poland}
\address[B]{Institute of Fluid Flow Machinery, Polish Academy of Sciences, Poland}

\begin{abstract}

\end{abstract}

\begin{keyword}
	Lamb waves \sep delamination identification \sep semantic 
	segmentation \sep deep learning \sep ConvLSTM.
\end{keyword}
\end{frontmatter}
\markboth{July 2022\hb}{July 2022\hb}
%\thispagestyle{empty}
%\pagestyle{empty}

\section{Introduction}
The applications of composite materials are becoming more prevalent in many 
industries such as automotive, maritime, wind energy structures, aerospace and 
many more.
The interest in these materials is rapidly growing due to their strength, 
lightweight, stiffness, energy-efficient fabrication, good corrosion 
resistance, and other various desired characteristics~\cite{Giurgiutiu2015}. 
However, these materials undergo different kinds of defects in their life-cycle.
The common types of defects that occur in these materials are matrix cracking, 
fiber breakage, debonding, and delamination.
Among these defects, delamination is the most dangerous kind of defect because 
it can grow inside plies of composite laminate and affect the mechanical 
properties and structural integrity of composite 
materials~\cite{Giurgiutiu2015}\cite{Sridharan2008}.
Therefore, it is very important to detect delamination at a very early stage 
for the safe functionality of these materials. 
Detection of delamination is very difficult for conventional visual inspection 
techniques because it occurs inside the structures which is invisible from the 
outer surfaces~\cite{Mei2019}.

Numerous analytical and experimental approaches have been implemented in order 
to better understand the mechanics and mechanisms of delamination in composite 
materials~\cite{Jih1993}\cite{Sohn2011}\cite{Khan2018}.
Among different delamination identification techniques, guided Lamb waves have 
recently received considerable attention and has proven as an effective and 
reliable tool for the identification of delamination in composite structures.
Guided Lamb waves can travel long distance without much attenuation in plate 
like structures and are highly sensitive to defects along with the propagation 
path. 
Therefore, by employing guided Lamb waves, even micro-scale defects can be 
identified and monitored with a relatively small number of sparsely distributed 
and low-voltage 
transducers~\cite{Harb2016}\cite{Cawley2003}.  


The main problem with the use of guided Lamb waves is the requirement of a 
dense array of sensors for the identification and localisation of small 
defects and inspection of a larger surface, which is not feasible in most of 
the situations.
In order to overcome this problem scanning laser Doppler vibrometry (SLDV) is 
used.
SLDV offers the measurements of guided Lamb waves in a dense array of points 
over the surface of a large specimen.
Such collection of signals is often known as full wavefield~\cite{Kudela2019}.
Recently, full wavefield signals have been increasingly used for the 
identification of delamination in composite materials. 
These techniques allow to estimate the location and size of the delamination 
effectively~\cite{Kudela2018}\cite{Girolamo2018}.   

However, these full wavefield signals are quite complex and are very difficult 
for conventional signal processing methods and classical machine learning-based 
approaches to analyse.
On the other hand, deep learning-based approaches have shown significant 
performance in handling very complex and nonlinear data in various domains such 
as computer vision, speech recognition, medical sciences, object detection and 
many more~\cite{Deng2014}.

Different researchers have implemented basic artificial neural networks (ANNs) 
and deep learning techniques for the identification of defects in composite 
structures by using vibration and frequency-based 
approaches~\cite{Islam1994}\cite{Khan2019}\cite{Okafor1996}.
However, only a few researchers have used ANNs and deep learning-based 
approaches for damage identification in composite structures with the use of 
guided Lamb 
waves~\cite{Chetwynd2008}\cite{Fenza2015}\cite{Feng2019}\cite{Rautela2021}. 

Chetwynd et al.~\cite{Chetwynd2008} implemented two multi-layer perceptrons 
neural networks on stiffened curved carbon fiber reinforced polymer 
(CFRP) for the regression and classification tasks with the use of guided Lamb 
waves.
Fenza et al.~\cite{Fenza2015} applied a probability ellipse-based approach and 
basic ANNs with the use of guided Lamb waves in order to  determine the degree 
and location of damage in composite and metallic plates. Results of both of 
their approaches showed that guided Lamb waves have noticeable advantages in 
the detection and localization of micro-scale defects in plate-like 
structures. 

Feng et al.~\cite{Feng2019} applied two algorithms based on the time of flight 
with the use of guided Lamb waves in CFRP composite plates.  
In their approach, a probability matrix is first created by employing a 
probabilistic technique and is used for the localisation of defects in CFRP 
composite plates and after that ANN is applied for improving the accuracy 
of the localisation of defects.
Rautela et al.~\cite{Rautela2021} implemented deep learning-based approach for 
the identification of structural defects in composite structures with the use 
of guided Lamb waves. 
The detection of defects is performed on two datasets with the use of 
convolutional neural networks (CNNs) and regression-based CNNs and 
long-short-term memory (LSTM) models were applied for the localisation of 
defects. 
They have shown that the deep learning-based approaches surpassed the classical 
machine-learning techniques.

In this research work, we implemented a deep learning-based semantic 
segmentation model on the full wavefield frames of propagating Lamb waves for 
delamination identification.
A many-to-one prediction approach was employed in the proposed deep learning 
model.
A sequence of full wavefield frames (animation) is inputted into the proposed 
deep learning model.
The proposed model is inspired by the convolutional
long short-term memory (ConvLSTM) cells~\cite{Shi2018}.
Further, two classes (damaged and undamaged) were identified in the pixel-wise 
segmentation problem.
The proposed segmentation-based ConvLSTM model is capable of providing good 
results for the identification of delamination in composite structures.

In the next section, the procedure for the acquisition of the full wavefield 
images and the dataset on which the proposed model is applied, are 
elaborated.
In section~\ref{sec:section3}, recurrent neural networks (RNNs), LSTMs, and 
ConvLSTMs are briefly introduced.
Whereas, the proposed model is explained in section~\ref{sec:section4} and the 
results are elaborated in section~\ref{sec:section5}, followed by conclusions 
in section~\ref{sec:section6}.

\section{Methodology}

%%%%%%%%%%%%%%%%%%%%%%%%%%%%%%%%%%%%%%%%%%%%%%%%%%%%%%%%%%%%%%%%%%%%%%%%%%%%%%%%
\subsection{Data preprocessing}
%%%%%%%%%%%%%%%%%%%%%%%%%%%%%%%%%%%%%%%%%%%%%%%%%%%%%%%%%%%%%%%%%%%%%%%%%%%%%%%%
Similar to the previous work~\cite{Ijjeh2021}, 475 cases were simulated, representing Lamb wave propagation and interaction with single delamination for each case. 

It should be underlined that the previous dataset contained the RMS of the full wavefield, representing wave energy spatial distribution in the form of images for each delamination case~\cite{Kudela2020d}.
On the other hand, the currently utilised dataset contains frames of propagating waves (512 frames for each delamination scenario).
The new dataset is available online~\cite{Kudela2021}.

As mentioned earlier, the dataset contains 475 different cases of delaminations, with 512 frames per case, producing a total number of 243,\,200 frames with a frame size of \((500\times500)\)~pixels representing the geometry of the specimen of size \((500\times500)\)~mm\(^{2}\).
Thus, using all frames in each case has high computational and memory costs.
Frames displaying the propagation of guided waves before interaction with the delamination have no features to be extracted (see Fig.~\ref{fig:Full_wave}).
Hence, for training, only a certain number of frames were selected from the initial occurrence of the interactions with the delamination.

Figure~\ref{fig:Full_wave} shows selected frames at different time-steps of the propagating Lamb waves before and after the interaction with the damage.
Frame \(f_{1}\) represents the initial interactions with the delamination, which was calculated using the delamination location and the velocity of the \(A0\) Lamb wave mode.
While frame \(f_{m}\) represents the last frame in the training sequence window, \(m=24\) for the developed model which will be discussed in the next subsection.
\begin{figure}[!h]
	\centering
	\includegraphics[width=.9\textwidth]{Graphics/figure2.png}
	\caption{Sample frames of full wave propagation.}
	\label{fig:Full_wave}
\end{figure}

Furthermore, the dataset was divided into two sets: training and testing, with a ratio of \(80\%\) and \(20\% \) respectively.
Moreover, a certain portion of the training set was preserved as a validation set to validate the model during the training process.
Additionally, the dataset was normalised to a range of \((0, 1)\) to improve the convergence of the gradient descent algorithm.

Additionally, for the training purposes, I have upsampled the frames (by using cubic interpolation) to \(512\times512\)~pixels to maintain the symmetrical shape during the encoding and decoding process.
Further, the validation sets have portions of \(10\%\) and \(20\%\) regarding the training set.
%%%%%%%%%%%%%%%%%%%%%%%%%%%%%%%%%%%%%%%%%%%%%%%%%%%%%%%%%%%%%%%%%%%%%%%%%%%%%%%%

Figure~\ref{fig:Diagram_exp_predictions} illustrates the complete procedure of obtaining intermediate predictions for the testing cases and finally calculating the RMS image, where \(f_{1}\) refers to the starting frame and \(f_{n}\) is the last frame, (\(n=512\)) in our dataset.
Further, \(m\) refers to the number of frames in the window, hence, \(m=24\) frames for the developed model, and \(k\) represents the total number of windows.
Accordingly, I slide the window over all input frames.
The shift of the window is one frame at a time.
Deep learning model predictions \(\hat{Y_k}\) are obtained for each window and combined to the final damage map by using the $RMS$:

\begin{equation}
	RMS = \sqrt{\frac{1}{N}\sum_{k=1}^{N}\hat{Y_k}^2}.	
	\label{RMS}
\end{equation}
%%%%%%%%%%%%%%%%%%%%%%%%%%%%%%%%%%%%%%%%%%%%%%%%%%%%%%%%%%%%%%%%%%%%%%%%%%%%%%%%
\begin{figure}[!h]
	\centering
	\includegraphics[width=.75\textwidth]{Graphics/figure3_diagram.png}
	\caption{The procedure of calculating the RMS prediction image (damage map).}
	\label{fig:Diagram_exp_predictions}
\end{figure}
%%%%%%%%%%%%%%%%%%%%%%%%%%%%%%%%%%%%%%%%%%%%%%%%%%%%%%%%%%%%%%%%%%%%%%%%%%%%%%%%

%%%%%%%%%%%%%%%%%%%%%%%%%%%%%%%%%%%%%%%%%%%%%%%%%%
\section{THE PROPOSED MODEL}\label{sec:section4}
In this work, we applied an end-to-end deep learning-based model by utilising 
full wavefield frames of guided Lamb waves propagation for delamination 
identification in CFRP composite plates.
%%%%%%%%%%%%%%%%%%%%%%%%%%%%%%%%%%%%%%%%%%%%%%%%%%
\begin{figure} [h!]
	\begin{center}
		\includegraphics[width=0.5\textwidth]{Graphics/ConvLSTM_model.png}
	\end{center}
	\caption{The architecture of the proposed deep learning model.} 
	\label{fig:proposed_models}
\end{figure}
%%%%%%%%%%%%%%%%%%%%%%%%%%%%%%%%%%%%%%%%%%%%%%%%%%
The proposed model, presented in Fig.~\ref{fig:proposed_models} takes 
\(64\) frames as input, and it is composed of three ConvLSTM layers.
The first ConvLSTM layer has \(12\) filters, the second layer has \(6\) 
filters and the third layer has \(12\) filters.
The kernel size of the ConvLSTM layers was set to (\(3\times3\)) with a stride 
of \(1\), and the padding was set to "same", which makes the output the same 
as the input in the case of stride equal to \(1\).
Furthermore, a \(\tanh\) (the hyperbolic tangent) activation function was used 
within the ConvLSTM layers that output values in a range between (\(-1\) and 
\(1\)). The batch normalization technique was also applied after the first two 
ConvLSTM layers.
A 2D convolutional layer followed by a sigmoid activation function was applied 
at the final output layer, which gives the output values in a range between 
\((0,1)\), and indicates the probability of the delamination.

Furthermore, to classify the output into undamaged (represented by \(0\)) or 
damaged (represented by \(1\)), a threshold value of \(0.5\) was used. 
The outputs values less than the threshold value were considered as undamaged 
and the values greater than the threshold were considered as damaged.

Moreover, for the evaluation of the performance of the proposed model, the mean 
intersection over union \(IoU\), also known as Jaccard index was applied as the 
accuracy metric. 
\(IoU\) is calculated by determining the intersection area between the 
predicted output and the ground truth. 
Further, we have two output classes (damaged and undamaged), the \(IoU\) was 
calculated for the damaged class only. 
Equation~(\ref{eqn:iou}) illustrates the \(IoU\) metric: 
\begin{equation}
	IoU=\frac{\hat{Y} \cap Y}{\hat{Y} \cup Y}
	\label{eqn:iou}
\end{equation}
where \(Y\) represents the ground truth, and \(\hat{Y}\) is the predicted 
output.
%%%%%%%%%%%%%%%%%%%%%%%%%%%%%%%%%%%%%%%%%%%%%%%%%%%%%%%%%%%%%%%%%%%%%%%%%%%%%%%%

\subsection{Font}

The font type for running text (body text) is 10~point Times New Roman.
There is no need to code normal type (roman text). For literal text, please use
\texttt{type\-writer} (\verb|\texttt{}|)
or \textsf{sans serif} (\verb|\textsf{}|). \emph{Italic} (\verb|\emph{}|)
or \textbf{boldface} (\verb|\textbf{}|) should be used for emphasis.

\subsection{General Layout}
Use single (1.0) line spacing throughout the document. For the main
body of the paper use the commands of the standard \LaTeX{}
``article'' class. You can add packages or declare new \LaTeX{}
functions if and only if there is no conflict between your packages
and the \texttt{IOS-Book-Article}.

Always give a \verb|\label| where possible and use \verb|\ref| for cross-referencing.


\subsection{(Sub-)Section Headings}
Use the standard \LaTeX{} commands for headings: {\small \verb|\section|, \verb|\subsection|, \verb|\subsubsection|, \verb|\paragraph|}.
Headings will be automatically numbered.

Use initial capitals in the headings, except for articles (a, an, the), coordinate
conjunctions (and, or, nor), and prepositions, unless they appear at the beginning
of the heading.

\subsection{Footnotes and Endnotes}
Please keep footnotes to a minimum. If they take up more space than roughly 10\% of
the type area, list them as endnotes, before the References. Footnotes and endnotes
should both be numbered in arabic numerals and, in the case of endnotes, preceded by
the heading ``Endnotes''.

\subsection{References}

Please use the Vancouver citing \& reference system, and the National Library of 
Medicine (NLM) style.

Place citations as numbers in square brackets in the text. All publications cited in 
the text should be presented in a list of references at the end of the manuscript. 
List the references in the order in which they appear in the text. Some examples of 
the NLM style:

\medskip
\noindent\ul{Journal article:}\par\noindent
Petitti DB, Crooks VC, Buckwalter JG, Chiu V. Blood pressure levels before dementia. 
Arch Neurol. 2005 Jan;62(1):112-6.

\medskip
\noindent\ul{Paper from a proceedings:}\par\noindent
Rice AS, Farquhar-Smith WP, Bridges D, Brooks JW. Canabinoids and pain. In: Dostorovsky 
JO, Carr DB, Koltzenburg M, editors. Proceedings of the 10th World Congress on Pain; 
2002 Aug 17-22; San Diego, CA. Seattle (WA): IASP Press; c2003. p. 437-68.

\medskip
\noindent\ul{Contributed chapter in a book:}\par\noindent
Whiteside TL, Heberman RB. Effectors of immunity and rationale for immunotherapy. In: 
Kufe DW, Pollock RE, Weichselbaum RR, Bast RC Jr, Gansler TS, Holland JF, Frei~E~3rd, 
editors. Cancer medicine 6. Hamilton (ON): BC Decker Inc; 2003. p. 221-8.

\medskip
\noindent\ul{Book by author(s):}\par\noindent
Jenkins PF. Making sense of the chest x-ray: a hands-on guide. New York: Oxford 
University Press; 2005. 194 p.

\medskip
\noindent\ul{Edited book:}\par\noindent
Izzo JL Jr, Black HR, editors. Hypertension primer: the essentials of high blood pressure. 
3rd ed. Philadelphia: Lippincott Williams \& Wilkins; c2003. 532 p.

\medskip
\noindent\ul{Proceedings:}\par\noindent
Ferreira de Oliveira MJ, editor. Accessibility and quality of health services. Proceedings of 
the 28th Meeting of the European Working Group on Operational Research Applied to Health 
Services (ORAHS); 2002 Jul 28-Aug 2; Rio de Janeiro, Brazil. Frankfurt (Germany): Peter Lang; 
c2004. 287 p.

\section{Illustrations}

\subsection{General Remarks on Illustrations}
The text should include references to all illustrations. Refer to illustrations in the
text as Table~1, Table~2, Figure~1, Figure~2, etc., not with the section or chapter number
included, e.g. Table 3.2, Figure 4.3, etc. Do not use the words ``below'' or ``above''
referring to the tables, figures, etc.

Do not collect illustrations at the back of your article, but incorporate them in the
text. Position tables and figures with at least 2 lines
extra space between them and the running text.

Illustrations should be centered on the page, except for small figures that can fit
side by side inside the type area. Tables and figures should not have text wrapped
alongside.

Place figure captions \textit{below} the figure, table captions \textit{above} the table.
Use bold for table/figure labels and numbers, e.g.: \textbf{Table 1.}, \textbf{Figure 2.},
and roman for the text of the caption. Keep table and figure captions justified. Center
short figure captions only.

The minimum \textit{font size} for characters in tables is 8 points, and for lettering in other
illustrations, 6 points.

On maps and other figures where a \textit{scale} is needed, use bar scales rather than
numerical ones of the type 1:10,000.

\subsection{Quality of Illustrations}
%Use only Type I fonts for lettering in illustrations.
Embed the fonts used if the application provides that option.
Ensure consistency by using similar sizes and fonts for a group of small figures.
To add lettering to figures, it is best to use Helvetica or Arial (sans serif fonts)
to avoid effects such as shading, outline letters, etc.

 Avoid using illustrations
taken from the Web. The resolution of images intended for viewing on a screen is
not sufficient for the printed version of the book.

If you are incorporating screen captures, keep in mind that the text may not be
legible after reproduction.

\subsection{Color Illustrations}
Please note, that illustrations will only be printed in color if the volume editor agrees to
pay the production costs for color printing. Color in illustrations will be retained
in the online (ebook) edition.


\section{Equations}
Number equations consecutively, not section-wise. Place the numbers in parentheses at
the right-hand margin, level with the last line of the equation. Refer to equations in the
text as Eq. (1), Eqs. (3) and (5).

\section{Fine Tuning}

\subsection{Type Area}
\textbf{Check once more that all the text and illustrations are inside the type area and
that the type area is used to the maximum.} You may of course end a page with one
or more blank lines to avoid `widow' headings, or at the end of a chapter.

\subsection{Capitalization}
Use initial capitals in the title and headings, except for articles (a, an, the), coordinate
conjunctions (and, or, nor), and prepositions, unless they appear at the beginning of the
title or heading.

\subsection{Page Numbers and Running Headlines}
You do not need to include page numbers or running headlines. These elements will be
added by the publisher.

\section{Submitting the Manuscript}
Submit the following to the volume editor:

\begin{enumerate}
\item The main source file, and any other required files. Do not submit more than
one version of any item.

\item The source files should compile without errors with pdflatex or latex.

\item Figures should be submitted in EPS, PDF, PNG or JPG format.

\item A high resolution PDF file generated from the source files you submit.
\end{enumerate}

\begin{thebibliography}{8}
	
	\bibitem{Giurgiutiu2015}
	Giurgiutiu, V.: Structural health monitoring of aerospace composites (2015).
	
	\bibitem{Sridharan2008}
	Sridharan, S.: Delamination behaviour of composites. Elsevier (2008).
	
	\bibitem{Mei2019}
	Mei, H., Migot, A., Haider, M. F., Joseph, R., Bhuiyan, M. Y., Giurgiutiu, 
	V.: Vibration-based in-situ detection and quantification of delamination in 
	composite plates. Sensors, \textbf{19}(7), 1734 (2019).
	
	\bibitem{Jih1993}
	Jih, C. J., Sun, C. T.: Prediction of delamination in composite laminates 
	subjected to low velocity impact. Journal of composite materials, 
	\textbf{27}(7), 684--701 (1993).
	
	\bibitem{Sohn2011}
	Sohn, H., Dutta, D., Yang, J. Y., DeSimio, M., Olson, S., Swenson, E.: 
	Automated detection of delamination and disbond from wavefield images obtained 
	using a scanning laser vibrometer. Smart Materials and Structures, 
	\textbf{20}(4), 
	045017 (2011).
	
	\bibitem{Khan2018}
	Khan, A., Kim, H. S.: Assessment of delaminated smart composite laminates via 
	system identification and supervised learning. Composite Structures, 206 
	354--362 (2018).
	
	\bibitem{Harb2016}
	Harb MS, Yuan FG.: Non-contact ultrasonic technique for Lamb wave 
	characterization in composite plates. Ultrasonics, \textbf{64}(162--9), (2016).
	
	\bibitem{Cawley2003}
	Cawley, P., Lowe, M. J. S., Alleyne, D. N., Pavlakovic, B., Wilcox, P.: 
	Practical long range guided wave inspection-applications to pipes and rail. 
	Mater. Eval, \textbf{61}(1), 66--74 (2003).
	
	\bibitem{Kudela2019}
	Radzieński, M., Kudela, P., Marzani, A., De Marchi, L., Ostachowicz, W.: Damage 
	identification in various types of composite plates using guided waves excited 
	by a piezoelectric transducer and measured by a laser vibrometer. Sensors,  
	\textbf{19}(9), 1958 (2019).
	
	\bibitem{Kudela2018}
	Kudela, P., Radzienski, M., Ostachowicz, W.: Impact induced damage assessment 
	by means of Lamb wave image processing. Mechanical Systems and Signal 
	Processing, (102), 23--36 (2018).
	
	\bibitem{Girolamo2018}
	Girolamo, D., Chang, H. Y., Yuan, F. G.: Impact damage visualization in a 
	honeycomb composite panel through laser inspection using zero-lag 
	cross-correlation imaging condition. Ultrasonics, (87), 152--165 (2018).
	
	\bibitem{Deng2014}
	Deng, L., Yu, D.: Deep learning: methods and applications. Foundations and 
	trends in signal processing, \textbf{7}(3–4), 197--387 (2014).
	
	\bibitem{Islam1994}
	Islam, A. S., Craig, K. C.: Damage detection in composite structures using 
	piezoelectric materials (and neural net). Smart Materials and Structures, 
	\textbf{3}(3), 318 (1994).
	
	\bibitem{Khan2019}
	Khan, A., Ko, D. K., Lim, S. C., Kim, H. S.: Structural vibration-based 
	classification and prediction of delamination in smart composite laminates 
	using deep learning neural network. Composites Part B: Engineering, 161, 
	586--594 (2019).
	
	\bibitem{Okafor1996}
	Okafor, A. C., Chandrashekhara, K., Jiang, Y. P.: Delamination prediction in 
	composite beams with built-in piezoelectric devices using modal analysis and 
	neural network. Smart materials and structures, \textbf{5}(3), 338  (1996).
	
	\bibitem{Chetwynd2008}
	Chetwynd, D., Mustapha, F., Worden, K., Rongong, J. A., Pierce, S. G., 
	Dulieu‐Barton, J. M.: Damage localisation in a stiffened composite panel. 
	Strain, 44(4), 298--307 (2008).
	
	\bibitem{Fenza2015}
	De Fenza, A., Sorrentino, A., Vitiello, P. Application of Artificial Neural 
	Networks and Probability Ellipse methods for damage detection using Lamb waves. 
	Composite Structures, 133, 390--403 (2015).
	
	\bibitem{Feng2019}
	Feng, B., Pasadas, D. J., Ribeiro, A. L., Ramos, H. G.: Locating defects in 
	anisotropic CFRP plates using ToF-based probability matrix and neural networks. 
	IEEE Transactions on Instrumentation and Measurement, \textbf{68}(5), 
	1252--1260 (2019).
	
	\bibitem{Rautela2021}
	Rautela, M., Gopalakrishnan, S.: Ultrasonic guided wave based structural damage 
	detection and localization using model assisted convolutional and recurrent 
	neural networks. Expert Systems with Applications, 167, 114189 (2021).
	
	\bibitem{Shi2018}
	Shi, X., Chen, Z., Wang, H., Yeung, D. Y., Wong, W. K., Woo, W. C.: 
	Convolutional LSTM network: A machine learning approach for precipitation 
	nowcasting. Advances in neural information processing systems, 28 (2015).
	
	\bibitem{Kudela2020}
	Kudela, P., Moll, J., Fiborek, P.: Parallel spectral element method for guided 
	wave based structural health monitoring. Smart Materials and Structures, 
	\textbf{29}(9), 095010 (2020).
	
	\bibitem{Kudela2021}
	Pawel, K., Abdalraheem I.: Synthetic dataset of a full wavefield
	representing the propagation of Lamb waves and their interactions with
	delaminations, (2021).
	
	\bibitem{Bengio1994}
	Bengio, Y., Simard, P., Frasconi, P.: Learning long-term dependencies with 
	gradient descent is difficult. IEEE transactions on neural networks, 
	\textbf{5}(2), 157--166 (1994).
	
	\bibitem{Hochreiter1997}
	Hochreiter, S., Schmidhuber, J.: Long short-term memory. Neural computation, 
	\textbf{9}(8), 1735--1780 (1997).
	
	\bibitem{Graves2014}
	Graves, A., Jaitly, N.: Towards end-to-end speech recognition with recurrent 
	neural networks. In International conference on machine learning. pp. 
	1764--1772. PMLR (2014, June).
	
	\bibitem{Cho2014}
	Cho, K., Van Merriënboer, B., Bahdanau, D., Bengio, Y.: On the properties of 
	neural machine translation: Encoder-decoder approaches. arXiv preprint 
	arXiv:1409.1259  (2014).
	
\end{thebibliography}
\end{document}
