%% SECTION HEADER /////////////////////////////////////////////////////////////////////////////////////
\section{Dataset preparation}
\label{sec62}
In order to train deep learning models to perform super-resolution image reconstruction, we have to reproduce a low-resolution training set from the original high-resolution dataset. 
Initially, we have resized the frames in the original high-resolution dataset to \((512\times512)\) pixels to obtain the desired output frame shape while preforming image reconstruction from the low- to high-resolutions.

In this work, we have generated a low-resolution training set with a frame size \((32\times32)\) pixels, which is below the Nyquist sampling rate of a 2D frame.
Hence, we have performed image subsampling with bi-cubic interpolation and a uniform mesh of size \((32\times32)\) pixels with a compression rate (CR) of \(21.5\%\) from the Nyquist sampling rate as depicted in Eqn.~\ref{CR}:

Figure~\ref{fig:SR_LR} shows a three SR Frames with their corresponding LR frames at different time steps.

To reduce the computation complexity during the training process of the deep learning models, we selected \((128)\) consecutive frames per each delamination case.
Frames displaying the propagation of guided waves before interacting with the delamination have no features to be extracted. 
Hence, only a certain number of frames was selected from the initial occurrence of the interactions with the delamination.
%%%%%%%%%%%%%%%%%%%%%%%%%%%%%%%%%%%%%%%%%%%%%%%%%%%%%%%%%%%%%%%%%%%%%%%%%%%%%%%%
\begin{equation}
	CR = \frac{(Low-resolution\ dimension)^2}{(Nyquist\ sampling\ rate)^2} = \frac{(32\times32)}{(69\times69)}=21.5\%
	\label{CR}
\end{equation}
%%%%%%%%%%%%%%%%%%%%%%%%%%%%%%%%%%%%%%%%%%%%%%%%%%%
\begin{figure} [!h]
	\centering
	\begin{subfigure}[b]{.48\textwidth}
		\centering
		\includegraphics[scale=1]{Figures/Chapter_6/SR_case_1_frame_1.png}
		\caption{SR Frame}
		\label{fig:SR_1}
	\end{subfigure}
	\hfill
	\begin{subfigure}[b]{.48\textwidth}
		\centering
		\includegraphics[scale=1]{Figures/Chapter_6/LR_case_1_frame_1.png}
		\caption{LR frame}
		\label{fig:LR_1}	
	\end{subfigure}
	\hfill
	\begin{subfigure}[b]{.48\textwidth}
		\centering
		\includegraphics[scale=1]{Figures/Chapter_6/SR_case_1_frame_63.png}
		\caption{SR frame}
		\label{fig:SR_2}
	\end{subfigure}
	\hfill
	\begin{subfigure}[b]{.48\textwidth}
		\centering
		\includegraphics[scale=1]{Figures/Chapter_6/LR_case_1_frame_63.png}
		\caption{LR frame}
		\label{fig:LR_2}	
	\end{subfigure}
	\hfill
	\begin{subfigure}[b]{.48\textwidth}
		\centering
		\includegraphics[scale=1]{Figures/Chapter_6/SR_case_1_frame_128.png}
		\caption{SR frame}
		\label{fig:SR_3}
	\end{subfigure}
	\hfill
	\begin{subfigure}[b]{.48\textwidth}
		\centering
		\includegraphics[scale=1]{Figures/Chapter_6/LR_case_1_frame_128.png}
		\caption{LR frame}
		\label{fig:LR_3}	
	\end{subfigure}
	\caption{High-resolution and Low-resolution frames at different time steps.}
	\label{fig:SR_LR}
\end{figure}
%%%%%%%%%%%%%%%%%%%%%%%%%%%%%%%%%%%%%%%%%%%%%%%%%%%
\newpage
