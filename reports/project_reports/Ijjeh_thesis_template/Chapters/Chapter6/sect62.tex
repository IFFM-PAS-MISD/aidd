\section{Future work}
\label{sec62}

In the conducted research work, the developed deep learning models for damage detection and localisation were trained on a single type of defect, "delamination", in CFRP structures.
Therefore, this work can be expanded to identify other types of defects in CFRP structures.
It is noteworthy to mention that generating the utilised dataset in the developed models in this work took roughly three months of computation.
However, with the advances in computational power, such a dataset can be generated in a reasonable time.
Accordingly, such a dataset will be utilised to train deep learning models that can detect and localise different types of defects at once in an end-to-end approach.
The performance of the developed models can be further improved if they are trained on experimental data, allowing them to learn new complex patterns.
However, the presented studies are limited to only one type of signal at a carrier frequency of 50 kHz.
Accordingly, a new dataset with a higher excitation frequency or broadband frequency (chirp signal) can be generated.
Furthermore, the proposed approaches have proven to be feasible for delamination identification.
As a result, the next step would be to acquire a large experimental data set consisting of full wavefields from structures with stiffeners and rivets as well as basic plate-like structures.
In comparison to a naive numerical data set, such data would significantly improve the performance of the proposed models.

Furthermore, the developed model for super-resolution image reconstruction in this work, trained only on a low-resolution dataset with a compression ratio of $19.2\%$ of the Nyquist sampling rate, which can be further extended to various compression ratios.
Another issue that can be further explored and investigated is when recovering an HR frame from an LR frame acquired with a very compressed number of scanning points (below Nyquist sampling rate). 
%The reflected waves from damage are weak because they represent a relatively small fraction of the wave energy acquired by the measuring points, and they rapidly fade away far from the damaged area.





