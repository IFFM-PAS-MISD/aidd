%% SECTION HEADER ////////////////////////////////////////////////////////////////////////////////
\section[SHM and motivations]{Structural Health Monitoring and motivations}
\label{sec11}

Structural health monitoring (SHM) intends to describe a real-time evaluation 
of the materials of a structural component or the full construction during the structure life-cycle~\cite{Balageas2010}. 
Furthermore, SHM supports detecting and characterising defects in structures 
as a whole or in their parts.
Detection of structural defect is critical because they may impair the safety of the structure during its operation~\cite{Yuan2016}. 

The purpose of SHM is to distinguish any potential change that occurs at 
a structure that could decay the performance of the whole system, at the 
earliest possible time so that an action can be taken to reduce the downtime, 
operational costs and maintenance costs, consequently reducing the risk of 
catastrophic failure, injury, or even loss of life.
Moreover, SHM improves the work organization of maintenance services replacing scheduled and periodic maintenance inspection with performance-based maintenance.
It decreases maintenance labour, in particular by avoiding dismounting undamaged parts and through reducing the individual involvement~\cite{Balageas2010}.

We can look at SHM as an improved method to perform Non-Destructive Evaluation (NDE). 
Nonetheless, SHM involves sensors that are integrated into structures, data 
transmission, computational power, and processing ability within 
structures~\cite{Balageas2010}. 
The typical organization of a SHM system is depicted in Fig.~\ref{fig:SHMsystem}. 
Such a system is built from a diagnostic part (low level) and a prognosis part (high level).
The diagnostic part is responsible for detection, localization, and evaluation of any damage.
The prognosis part includes the production of information concerning the outcomes of the diagnosed damage.
\begin{figure} [h!]
	\begin{center}
		\includegraphics[height=7cm]{Figures/Chapter_1/SHM_system.png}
	\end{center}
	\caption{Organization of SHM systems.} 
	\label{fig:SHMsystem}
\end{figure}

In general, we can categorize SHM strategies into two main schemes, local and 
global schemes. Local schemes were discussed in Refs.~\cite{Grimberg2001,Raghavan2007}
and global schemes were discussed in Refs.~\cite{Adams2002,Doebling1998,Uhl2004}. 
Local schemes aim at monitoring a small area of the structure enclosing the transducers that are used for registering the data signals after the structure being exited. 
For this purpose, few phenomena are used like ultrasonic waves~\cite{Raghavan2007}, eddy currents~\cite{Grimberg2001}, and acoustic emission~\cite{Grosse2008}. 
On the other hand, global schemes are related to the global behaviour of the structure~\cite{Balageas2010}. 
For this purpose, vibration techniques are utilized which can be classified as the signal-based and the model-based.
Signal-based approaches analyse measured responses of the structure after 
ambient excitation in order to identify possible defects~\cite{Stepinski2013}. 
The model-based approaches use various types of models of a monitored structure 
to detect and localize damage in the structure by utilising relations 
between the model parameters and distinct damage features~\cite{Stepinski2013}. 
