\section{Objectives and motivations}
\label{sec13}

The main objective of the work is to develop an artificial intelligence (AI) driven diagnostic system for delamination identification in composite laminates.
Furthermore, to explore the potential of utilising artificial intelligence-based approaches to investigate damage detection and identification based on the propagation of Lamb waves.
Data corresponding to elastic wave propagation has very complex patterns of wave reflections. 
It is difficult to explicitly program instructions that will output a damage intensity map of an element of a structure based on anomalies in propagating elastic waves (e.g., reflections from discontinuities).
Hence, this research aims to explore possible solutions that employ deep neural networks (DNN), as they are promising approaches.
The progress in the machine learning field in the last decade, along with increasing computation power capabilities, makes it a perfect time to investigate potential applications of DNN. 
DNN is an emerging tool that has successful applications in computer vision and speech recognition, among other applications. 
Nowadays, certain neural network (NN) architectures surpass human-level accuracy in image classification. 
The main advantage of DNN in comparison to other machine learning approaches is scalability. 
It means that the performance of DNN increases with a NN size as well as the size of data used for supervised learning. 

Therefore, the employment of an end-to-end approach that has the DNN process the animation of waves propagating and automatically producing a damage intensity map is up for debate.

Another objective of this research is to address the issue of slow data acquisition of high-resolution full wavefield of Lamb wave propagation.
Hence, to overcome such an issue, I aim to develop a deep learning system capable of recovering the high-resolution frames of Lamb wave propagation and their interaction with delamination and boundaries from low-resolution measurements with high accuracy.
Consequently, such a system will speed up the data acquisition process.


This research will help answer legitimate questions regarding the utilisation of deep learning techniques by processing the full wavefield of propagating elastic waves for delamination identification in composite laminates:
\begin{itemize}
	\item Can the proposed AI-driven diagnostic model be more accurate than the conventional signal processing technique?
	\item Knowing that experimental signals contain noise, is it adequate to use a numerical model for generating a dataset?
	\item Can a technique such as data augmentation enhance the training of deep learning models?
	\item How well deep learning models can generalise to new unseen data? Further, to experimental data acquired by SLDV?
	\item Is it computationally feasible to utilise all frames of propagating waves, or utilising certain frames can be efficient?
	\item Does the implementation of different deep learning architectures results in different accuracies on damage identification? Is the comparison metric among these architectures sufficient for determining the efficient one?
	\item Do deep learning techniques for delamination identification utilised in this thesis have any potential in practical applications in the long term?
	\item Can deep learning techniques developed for super-resolution image reconstruction be used to recover the high-resolution full wavefield frame from the low-resolution measurements with adequate accuracy to detect the damage?
\end{itemize}
