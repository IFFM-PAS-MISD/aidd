%% SECTION HEADER ////////////////////////////////////////////////////////////////////////////////
\section{Thesis organisation}
\label{sec15}
The thesis is organised as follows:

Chapter~\ref{ch2} introduces structural health monitoring (SHM) and its applications, in particular with guided waves.
Furthermore, an introduction to damage detection and localisation using guided waves is presented.

Chapter~\ref{ch3} introduces artificial intelligence approaches utilised for SHM applications.
Furthermore, in chapter~\ref{ch3} the data preprocessing, feature extraction, and classification techniques are presented.
Moreover, a detailed introduction to the deep learning approach is presented.
Finally, a related data-driven based SHM for damage detection and localisation is presented.

Chapter~\ref{ch4} introduces the acquisition of the synthetic dataset that resembles the full wavefield of the propagating Lamb waves in plate made of CFRP and their interactions with the damage and the boundaries of the plate.
Furthermore, in chapter~\ref{ch4} various damage detection and localisation techniques are presented.
Finally, I present an end-to-end deep learning model utilised for super-resolution image reconstruction that can be utilised for delamination identification.

Chapter~\ref{ch5} presents the numerical results for the various developed models.
Further, the developed models were evaluated on experimentally acquired data to investigate their ability to generalise.

Chapter~\ref{ch6} presents the conclusions of this research, and further, it presents the potential future works that could be carried out from the present work.
