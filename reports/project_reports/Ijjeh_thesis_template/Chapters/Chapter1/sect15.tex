%% SECTION HEADER ////////////////////////////////////////////////////////////////////////////////
\section{Thesis organisation}
\label{sec15}
The thesis is organised as follows:

Chapter~\ref{ch2} introduces structural health monitoring (SHM) and its applications, in particular guided waves based SHM for composite materials.
Furthermore, an introduction to damage detection and localisation using guided waves is presented.

Chapter~\ref{ch3} introduces artificial intelligence approaches utilised for SHM applications.
Furthermore, in chapter~\ref{ch3}  conventional machine learning approach consists of data preprocessing, feature extraction, and classification techniques are presented.
Moreover, a detailed illustration of the deep learning approach is presented.
Finally, a related data-driven-based SHM for damage detection and localisation is presented.

Chapter~\ref{ch4} introduces the acquisition of the synthetic dataset that resembles the full wavefield of the propagating Lamb waves in a plate made of CFRP and their interactions with discontinuities, i.e., the damage and the boundaries of the plate.
Furthermore, in chapter~\ref{ch4} various damage detection and localisation techniques are presented.
Finally, I present an end-to-end deep learning model utilised for super-resolution image reconstruc\-tion that can be utilised for delamination identification.

Chapter~\ref{ch5} presents the evaluation of the developed deep learning models on numerical test cases to demonstrate their capability to predict previously unseen data.

Furthermore, the developed models were also evaluated on experimentally acquired data with single and multiple Teflon inserts representing delamination to investigate their ability to generalise.

Chapter~\ref{ch6} presents the conclusions of this dissertation, and further, it presents the potential future works that could be carried out from the present work.
