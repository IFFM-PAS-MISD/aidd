%% SECTION HEADER ////////////////////////////////////////////////////////////////////////////////
\section{Objectives and Motivations}
\label{sec14}

The main objective of the work is to develop an artificial intelligence (AI) driven diagnostic system for delamination identification in composite laminates such as carbon fibre reinforced polymers (CFRP). 
The project will be focused on feasibility studies of machine learning approaches for elastic wave propagation analysis. 
Data corresponding to elastic wave propagation patterns are very complex, and it is difficult to explicitly program instructions that will output damage intensity map of an element of a structure based on anomalies in propagating elastic waves (e.g. reflections from delamination). 
The aim of the project is the exploration of other possible solutions which employ deep neural networks (DNN), one of the most promising machine learning approaches. 
The progress in the machine learning field in the last decade along with increasing computer power causes that it is a perfect time to investigate potential applications of DNN. 
It is an emerging tool that has found some successful applications in computer vision and speech recognition. 
It should be noted, that nowadays certain neural network (NN) architectures surpass human-level accuracy in image classification [1]. The main advantage of DNN in comparison to other machine learning approaches such as support vector machines is scalability. 
It means that the performance of DNN increases with a NN size as well as the size of data used for supervised learning. 
Therefore, we assume that it is possible to use the end-to-end approach in which DNN processes animation of propagating waves (input) directly into the damage intensity map (output).

This research will help answering legitimate questions regarding utilising deep learning techniques by processing full wavefield of propagating elastic waves for delamination identification in composite laminates:
\begin{itemize}
	\item Can the proposed AI-driven diagnostic model by more accurate than the conventional signal processing technique?
	\item Knowing that experimental signals contain noise, is it adequate to use numerical model for generating dataset?
	\item Can a technique such as data augmentation enhance training of deep learning models?
	\item How well deep learning models can generalise to new unseen data? Further, to experimental data acquired by SLDV?
	\item Is it computationally feasible to utilise all frames of propagating waves or utilising certain frames can be efficient?
	\item Does the implementation of different deep learning architectures results in different accuracies on damage identification? Is the comparison metric among these architectures sufficient for determining the efficient one?
	\item Do deep learning techniques for delamination identification utilised in this project have any potentials in practical applications in the long-term?
\end{itemize}
