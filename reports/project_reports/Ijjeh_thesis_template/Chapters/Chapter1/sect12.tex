\section{Purpose of the study}
\label{sec12}
In recent years, artificial intelligence (AI) based approaches have been utilized by numerous scientific fields. 
Accordingly, my aim is to utilise AI in the field of NDT/SHM to develop an AI-driven diagnostic system for delamination identification in composite laminates. 
Accordingly, embracing the AI approach will contribute to various aspects of NDT/SHM, elastic wave behaviour, image and animation processing.
Therefore, we propose an end-to-end deep learning approach that is capable of performing automatic feature extraction of the delamination characteristics, implying that the supervised deep learning model will learn by itself to extract the damage features and accordingly detect and localise the damage in
the investigated composite laminates.
For this purpose, a supervised learning-based approach was adopted.
Hence, a large dataset that was numerically generated representing the full wavefield of propagating elastic waves is implemented.
Additionally, generating such dataset requires a lot of time (3-months). 
As a result, we expect such a system will be an extra safety factor of functional structures hence it could reduce
maintenance costs.
