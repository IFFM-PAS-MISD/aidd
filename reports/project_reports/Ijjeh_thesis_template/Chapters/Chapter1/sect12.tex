%\section{State of the art}
%\label{sec12}
%The rapid development in the field of artificial intelligence (AI) in the recent years occurred due to the rapid growth in computational powers that boosted deep learning approach to evolve. 
%Deep learning or in other words (artificial neural network ANN) is a very promising technique due to their numerous applications such as machine translation, speech recognition, computer vision, among others. 
%We can say that the breakthroughs in computer vision began after 2012 when Alex Krizhevsky, et al. implemented a convolutional neural network (CNN) called \say{AlexNet}~\cite{Krizhevsky2012} which won the competition of 2012 ImageNet challenge. 
%AlexNet achieved state-of-the-art recognition accuracy against all traditional machine learning and computer vision techniques. 
%Therefore, AlexNet is considered a significant breakthrough in the field of machine learning and computer vision as image object detection, segmentation, video classification, object tracking, among other tasks. 
%The following years witnessed several computer vision architectures that are much more developed with a larger and deeper architectures such as VGGNet~\cite{szegedy2015going}, GoogleNet~\cite{Simonyan2015}, ResNet~\cite{he2016deep}. 
%As a result of this advancement in deep learning techniques, fields of (NDT/SHM) began to utilise ANN techniques in their approaches of damage detection and localisation, due to these techniques have the potential to overcome the conventional damage detection localisation issues by avoiding the complex process of handcrafting feature extraction and providing an automatic feature extraction solution, furthermore their capacity to adapt to big data (e.g. acquired measurements from the investigated
%structures). 
%That means the performance of ANN increases with a neural network (NN) size as well as the size of data used for supervised learning. Damage detection and localisation in composite materials by utilising elastic wave propagation have been investigated by several authors who have adapted ANN techniques in their models. 
%However, all the literature in this field were conducted on truss structures and signal data acquired by accelerometers. 
%In this project, we take a further step regarding investigating the wavefield propagation signals since it is more sensitive to local damage. 
%For this purpose, a large dataset of full wavefield of propagating elastic waves was generated to simulate the experimentally generated data which resembles measurements acquired by SLDV. 