\section{Purpose of the study}
\label{sec12}
In recent years, artificial intelligence (AI)-based approaches have been utilized by numerous scientific fields.
Accordingly, I aim to utilise AI-based approaches in the field of NDT/SHM to develop an AI-driven diagnostic system for delamination identification in composite laminates.
Adopting the AI approach will contribute to various aspects of NDT/SHM, elastic wave behaviour, image processing, and animation processing.
Therefore, I propose an end-to-end deep learning approach that is capable of performing automatic feature extraction of the delamination characteristics, implying that the supervised deep learning model will learn by itself to extract the damage features and accordingly detect and localise the damage in the investigated composite laminates.
A supervised learning-based approach was adopted for this purpose.
Accordingly, a large dataset that was numerically generated representing the full wavefield of propagating elastic waves is implemented.
It should be underlined that the computation of such a dataset is time-consuming (3 months).
As a result, I expect such a system will be an extra safety factor for operational structures, thereby reducing maintenance costs.
