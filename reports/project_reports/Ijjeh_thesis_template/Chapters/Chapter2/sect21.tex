%% SECTION HEADER ////////////////////////////////////////////////////////////////////////////////
\section[SHM for Composite Materials]{SHM for Composite Materials}
\label{sec21}
Composite materials are widely used in various industries, due to their useful characteristics. 
A composite material can be described as a compound of two or more different materials to acquire new features that cannot be achieved by specific components functioning individually.
Distinct from metallic alloys, which are isotropic materials, each material in the composite has its characteristics~\cite{Campbell2010}.
Accordingly, several advantages of these various characteristics can be obtained. 
Generally, composite materials are categorized into~\cite{Jones1999}:
\begin{itemize}
	\item Fibre-reinforced composite materials consisting of three parts: the fibres as the discontinuous phase, the matrix as the continuous phase, and the fine inter-phase region, also known as the interface~\cite{Cantwell1991}.
	\item Laminated composite materials are an assembly of multiple layers of fibre-reinforced or fabric-reinforced composite materials (e.g. plain weave, twill) that can be combined to implement necessary design features~\cite{Ramirez1999}.
	\item Particulate composite materials are characterized as being composed of particles suspended in a matrix (e.g. composite with short fibres).
\end{itemize}

When comparing composite materials to regular metallic materials, we can notice that composites have some advantages over metallic materials. 
The advantages can be summarized as~\cite{Campbell2010}:
\begin{itemize}
	\item Low density with high strength and stiffness,
	\item Greater vibration damping capacity, and more temperature resistance,
	\item Strong texture in micro-structures that makes it easy to design and 
	satisfy different application needs. 
	\item Chemical and corrosion resistance.	
\end{itemize}

However, composite materials possess some disadvantages.
Due to the nature of multiphase materials, composite materials present anisotropic characteristics. 
It is considered a disadvantage in the case of wave propagation due to the complexity of processing of registered signals. 
%Their material capacities, mainly associated with manufacturing processes, are dispersive~\cite{Awad2012}. 

Damage can accidentally occur in composite materials, either during the process of manufacturing or during the regular service life of the structure. 
Generally, impact damage in composite materials is caused by various impact events that can be resulting from the lack of reinforcement in the out-of-plane direction~\cite{Cai2012}. 
Under a high energy impact, little penetration rises in composite materials.  
Furthermore, low to medium energy impact can initiate delamination which is caused by bending cracks, matrix cracking, and shear cracks,  which mostly happen below the top surfaces and are barely visible~\cite{Cai2012}. 
Delamination can alter the compression strength of composite laminate, and it will gradually affect the composite to encounter failure by buckling~\cite{NurAzrieBtSafri2018}.
The tension encountered by the composite structure creates cracks and produces delamination between the laminates, which leads to more damage~\cite{NurAzrieBtSafri2018}. 
Furthermore, when a composite laminate encounters low- or high-velocity impact, various damage modes can appear, including fibre crack, matrix crack, delamination and fibre pullout. 
All of these damage modes are dependent on the impact parameter such as impact energy and impactor mass or impactor shape~\cite{NurAzrieBtSafri2018}.
Moreover, additional types of damage can also occur, such as debonding, which occur when an adhesive stops adhering to an adherend.
These defects can seriously decrease the performance of composites, hence, they should be detected in time to avoid catastrophic structural collapses.  

The concept of an SHM system in composite structures is to use a built-in structural diagnostic system, which usually consists of three main components~\cite{1}: 
%%%%%%%%%%%%%%%%%%%%%%%%%%%%%%%%%%%%%%%%%%%%%%%%%%%%%%%%%%%%%%%%%%%%%%%%%%%%%%%%
\begin{itemize}
	\item actuator/sensor network,
	\item supporting electronic hardware,
	\item data interpretation software for monitoring the status condition of the in-service structure
\end{itemize}
%%%%%%%%%%%%%%%%%%%%%%%%%%%%%%%%%%%%%%%%%%%%%%%%%%%%%%%%%%%%%%%%%%%%%%%%%%%%%%%%
Therefore, it is a crucial step when developing a diagnostic system to integrate and embed sensors with the composite structure. 
Hence, several types of sensors can be integrated and embedded into a composite structure, such as piezoelectric transducers (PZT), optical fibre sensors (e.g. Fiber Bragg grating (FBG)) and Microelectromechanical Systems (MEMS).

Consequently, defects can only be discovered by analysing responses of the structure, obtained by sensors, before and after it happens.
Accordingly, we cannot expect to have “damage sensors”.
The only way to detect the damage is by processing and comparing the signals received from the sensors before and after damage occurrence~\cite{s18041094}. 
Subsequently, one can attempt to classify extracted features, which are sensitive to minor damage, and can be distinguished from the response to natural and environmental disturbances~\cite{s18041094}. 
Thus, SHM  methods in composite materials are essential for damage detection and estimation since SHM implies different types of sensors mixed with damage detection techniques. 