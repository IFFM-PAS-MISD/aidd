%% SECTION HEADER ////////////////////////////////////////////////////////////////////////////////
\section{Summary}
\label{sec24}
Engineering structures are vulnerable to several types of damage that may occur naturally or artificially.
Hence, the damage will reduce the expected lifetime of the running structures, increase their maintenance costs, and sometimes may lead to catastrophic consequences. 
Therefore, to avoid such consequences, SHM techniques are applied.
%Moreover, I discussed that SHM techniques are able to detect any possible change that occurs at a structure that could decay the performance of the whole system, at the earliest possible time so that an action can be taken to reduce the downtime, operational costs and maintenance costs, consequently reducing the risk of catastrophic failure, injury, or even loss of life.

In this chapter,  several SHM approaches for detecting and localising damage within composite structures that utilise guided Lamb waves were presented. 
I illustrated that guided-wave based SHM systems can be built upon the processing of signals registered by PZTs or SLDV. 
%% SECTION HEADER ////////////////////////////////////////////////////////////////////////////////
