%% SECTION HEADER ////////////////////////////////////////////////////////////////////////////////
\section{Summary}
\label{sec24}
In this chapter, author discussed problems of ageing structures which are exposed to have several types of damage. 
Structures suffer from damage whether natural or artificial, may reduce their expected lifetime, increasing their maintenance costs, and some time may lead to catastrophic consequences. 
Therefore, to avoid such consequences SHM techniques could be applied.  
Moreover, author discussed that SHM techniques can detect any possible change that occurs at a structure which could decay the performance of the whole system, at the earliest possible time so that an action can be taken to reduce the downtime,  operational costs and maintenance costs, consequently reducing the risk of catastrophic failure, injury, or even loss of life.

In this chapter,  several SHM approaches for detecting and localising damage within composite structures that utilise guided Lamb waves were presented. 
Author illustrated that guided-wave based SHM systems can be built upon processing of signals registered by PZTs or SLDV. 
Moreover, author presented in this chapter several techniques that had studied and examined guided Lamb waves in composite materials to detect and localise the damage using signal processing techniques. 
Consequently, author concluded that those traditional techniques are complex and involve a huge numerical analysis and signal processing. Which concluded that the damage features are difficult to be extracted manually. 
Thus, new approaches that involve Machine and Deep Learning techniques are utilised are presented in this chapter. 
As a result, the process of damage features extracting became more convenient and easier since the machine is responsible for learning the new features and accordingly  detect and localise the damage. 
In consequence, it is concluded that the advantage of this approach is the improvement of feature damage extracting procedure.

Furthermore, problems with conventional damage detection techniques for SHM and the importance of the artificial intelligence approach were discussed.
Furthermore, in the second section of the chapter, the author introduced the ML approach in the SHM field.
Moreover, several techniques for feature extraction such as PCA, MSD, and GMMs were described. 
Further, several classification models such as SVM, KNN, and decision trees were introduced.
In the third section, DL approach was presented, in which techniques such as CNN  and RNN were presented.
Finally, several deep learning techniques for damage detection used regarding the SHM field based on guided waves and vibration approaches were presented.%% SECTION HEADER ////////////////////////////////////////////////////////////////////////////////
