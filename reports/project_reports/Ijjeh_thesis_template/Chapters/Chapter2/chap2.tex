\chapter[State of the in for SHM]{State of the art in Structural Health Monitoring}
\label{ch2}
Structural health monitoring (SHM) intends to describe a real-time evaluation of the materials of a structural component or the full construction during the life-cycle of the structure~\cite{Balageas2010}. 
Furthermore, SHM supports detecting and characterising defects in structures as a whole or in their parts.
Detection of structural defects is critical because they may impair the safety of the structure during its operation~\cite{Yuan2016}. 

The purpose of SHM is to identify any potential change that occurs in a structure that could affect the performance of the entire system at the earliest possible time so that an action can be taken to reduce the downtime, operational costs, and maintenance costs, consequently reducing the risk of catastrophic failure, injury, or even loss of life.
Moreover, SHM improves the work organization of maintenance services by replacing scheduled and periodic maintenance inspections with performance-based maintenance.
It decreases maintenance labour, in particular, by avoiding dismounting undamaged parts and by reducing individual involvement~\cite{Balageas2010}.

We can look at SHM as an improved method to perform Non-Destructive Evaluation (NDE).
Nonetheless, SHM involves sensors that are integrated into structures, data transmission, computational power, and processing ability within structures~\cite{Balageas2010}. 
The typical organization of an SHM system is depicted in Fig.~\ref{fig:SHMsystem}. 
Such a system is built from a diagnostic part (low level) and a prognosis part (high level).
The diagnostic part is responsible for detecting, localising, and evaluating the damage.
The prognosis part includes the production of information concerning the outcomes of the diagnosed damage.
%%%%%%%%%%%%%%%%%%%%%%%%%%%%%%%%%%%%%%%%%%%%%%%%%%%%%%%%%%%%%%%%%%%%%%%%%%%%%%%%
\begin{figure} [!ht]
	\begin{center}
		\includegraphics[width=1\textwidth]{Figures/Chapter_1/SHM_system.png}
	\end{center}
	\caption{Organization of SHM systems.} 
	\label{fig:SHMsystem}
\end{figure}
%%%%%%%%%%%%%%%%%%%%%%%%%%%%%%%%%%%%%%%%%%%%%%%%%%%%%%%%%%%%%%%%%%%%%%%%%%%%%%%%
In general, we can categorize SHM strategies into two main schemes: local and global schemes. 
Local schemes were discussed in ~\cite{Grimberg2001,Raghavan2007}
and global schemes were discussed in~\cite{Adams2002,Doebling1998,Uhl2004}. 
Local schemes aim at monitoring a small area of the structure enclosing the transducers that are used for registering the data signals after the structure is exited. 
For this purpose, a few phenomena are used, like ultrasonic waves~\cite{Raghavan2007}, eddy currents~\cite{Grimberg2001}, and acoustic emission~\cite{Grosse2008}.
On the other hand, global schemes are related to the global behaviour of the structure~\cite{Balageas2010}. 
For this purpose, vibration techniques are utilised, which can be classified as signal-based and model-based.
Signal-based approaches related to passive SHM systems analyse measured responses of the structure after ambient excitation to identify possible defects~\cite{Stepinski2013}.
While in active SHM, the structure can be externally exited using e.g. PZT transducers or shakers.
 
The model-based approaches use various types of models of a monitored structure to detect and localise damage in the structure by utilising relations between the model parameters and distinct damage features~\cite{Stepinski2013}. 
%%%%%%%%%%%%%%%%%%%%%%%%%%%%%%%%%%%%%%%%%%%%%%%%%%%%%%%%%%%%%%%%%%%%%%%%%%%%%%%%
%% SECTION HEADER ////////////////////////////////////////////////////////////////////////////////
\section[SHM for composite materials]{SHM for composite materials}
\label{sec21}
Composite materials are widely used in various industries, due to their useful characteristics. 
A composite material can be described as a compound of two or more different materials to acquire new features that cannot be achieved by specific components functioning individually.
Distinct from metallic alloys, which are isotropic materials, each material in the composite has its characteristics~\cite{Campbell2010}.
Accordingly, several advantages of these various characteristics can be obtained. 
Generally, composite materials are categorized into~\cite{Jones1999}:
\begin{itemize}
	\item Fibre-reinforced composite materials consisting of three parts: the fibres as the discontinuous phase, the matrix as the continuous phase, and the fine inter-phase region, also known as the interface~\cite{Cantwell1991}.
	\item Laminated composite materials are an assembly of multiple layers of fibre-reinforced or fabric-reinforced composite materials (e.g. plain weave, twill) that can be combined to implement necessary design features~\cite{Ramirez1999}.
	\item Particulate composite materials are characterized as being composed of particles suspended in a matrix (e.g. composite with short fibres).
\end{itemize}

When comparing composite materials to regular metallic materials, we can notice that composites have some advantages over metallic materials. 
The advantages can be summarized as~\cite{Campbell2010}:
\begin{itemize}
	\item low density with high strength and stiffness,
	\item high vibration damping capacity, and more temperature resistance,
	\item strong texture in micro-structures, making it easy to design and satisfy different application needs, 
	\item chemical and corrosion resistance.	
\end{itemize}

However, composite materials possess some disadvantages.
Due to the nature of multiphase materials, composite materials present anisotropic characteristics. 
It is considered a disadvantage in the case of wave propagation due to the complexity of processing of registered signals. 
%Their material capacities, mainly associated with manufacturing processes, are dispersive~\cite{Awad2012}. 

Damage can accidentally occur in composite materials, either during the process of manufacturing or during the regular service life of the structure. 
The lack of reinforcement in the out-of-plane direction causes that composite materials are susceptible to impact damage~\cite{Cai2012}. 
Under a high energy impact, little penetration rises in composite materials.  
Furthermore, low to medium energy impact can initiate delamination which is caused by bending cracks, matrix cracking, and shear cracks,  which mostly happen below the top surfaces and are barely visible~\cite{Cai2012}. 
Delamination can alter the compression strength of composite laminate, and it will gradually affect the composite to encounter failure by buckling~\cite{NurAzrieBtSafri2018}.
The tension encountered by the composite structure creates cracks and produces delamination between the laminates, which leads to more damage~\cite{NurAzrieBtSafri2018}. 
Furthermore, when a composite laminate encounters low- or high-velocity impact, various damage modes can appear, including fibre crack, matrix crack, delamination and fibre pullout. 
All of these damage modes are dependent on the impact parameter such as impact energy and impactor mass or impactor shape~\cite{NurAzrieBtSafri2018}.
Moreover, additional types of damage can also occur, such as debonding, which occur when an adhesive stops adhering to an adherend.
These defects can seriously decrease the performance of composites, hence, they should be detected in time to avoid catastrophic structural collapses.  

The concept of an SHM system in composite structures is to use a built-in structural diagnostic system, which usually consists of three main components~\cite{Hassani2022}: 
%%%%%%%%%%%%%%%%%%%%%%%%%%%%%%%%%%%%%%%%%%%%%%%%%%%%%%%%%%%%%%%%%%%%%%%%%%%%%%%%
\begin{itemize}
	\item Actuator/sensor technology, which can be embedded in an inspected structure to register and transmit the structural response.
	\item For real-time condition monitoring of the structure, the registered data needs to be processed by high-performance computing equipment in the control center.
	\item Data interpretation software for monitoring the registered responses from the in-service structure.
\end{itemize}
%%%%%%%%%%%%%%%%%%%%%%%%%%%%%%%%%%%%%%%%%%%%%%%%%%%%%%%%%%%%%%%%%%%%%%%%%%%%%%%%
Therefore, it is a crucial step when developing a diagnostic system to integrate and embed sensors with the composite structure. 
Hence, several types of sensors can be integrated and embedded into a composite structure, such as piezoelectric transducers (PZT), optical fibre sensors (e.g. Fiber Bragg grating (FBG)) and Microelectromechani\-cal Systems (MEMS).

Consequently, defects can only be discovered by analysing responses of the structure, obtained by sensors, before and after damage occurrence.
Accordingly, we cannot expect to have “damage sensors”.
The only way to detect the damage is by processing and comparing the signals received from the sensors before and after damage occurrence~\cite{s18041094}. 
Subsequen\-tly, one can attempt to classify extracted features, which are sensitive to minor damage, and can be distinguished from the response to natural and environmental disturbances~\cite{s18041094}. 
Thus, SHM  methods in composite materials are essential for damage detection and estimation since SHM implies different types of sensors mixed with damage detection techniques. 
%% SECTION HEADER ////////////////////////////////////////////////////////////////////////////////
\section[Guided waves based SHM]{Guided waves based SHM}
\label{sec22}
The approach behind adopting elastic waves propagation methods in SHM includes generating elastic waves in the examined structure and recording their displacement as a function of time~\cite{Ostachowicz2012}. 
The produced waves are travelling in packets, those packets keep propagating until 
they reflect from discontinuities, edges or damage in the structure. The reflected waves hold information about the location and the size of the damage. 

Designing a robust SHM system requires knowledge in various scientific fields e.g. mechanical and electrical engineering, as well as in computer science, mathematics, and physics~\cite{Willberg2013}.
Moreover, it requires a deep understanding of various material types and the design of transducers working alone and in networks. 
Also, there is a need to be familiar with signal processing methods and damage evaluation techniques~\cite{Willberg2013}.

In this literature we will focus on the guided wave-based SHM techniques for composite materials, which has brought large attention in the past two decades~\cite{Mitra2016}.
Guided waves which are essentially elastic waves propagating within bounded 
structures~\cite{Mitra2016}, e.g. in a thin-plate, they are being guided by the boundaries of the plate. 

There are a few benefits from adopting guided wave-based sche\-mes for SHM in structures over vibration based methods. 
The transducers that are utilised in SHM systems are generally affordable, also usually, due to the lightweight of those transducers, it can be implemented easily in the structure.
In addition, it is possible to scan a relatively large area compared to a little number of transducers~\cite{Mitra2016}. 
Moreover, an important advantage for guided waves over a vibration-based scheme 
is their high sensitivity for detecting small defects due to the ability to use high-frequency signals (excited and registered).
In such a case, guided waves are not so sensitive to low-frequency vibrations~\cite{Mitra2016,Croxford2007}.

Various types of guided waves have been investigated for the purpose of SHM. 
A well-known approach is the use of Lamb waves, that propagate 
within thin-plates and shells bounded by stress-free surfaces~\cite{Mitra2016}.
Lamb waves were given their name after Horace Lamb, who discovered them and 
developed a theory to describe the phenomena of their propagation 
~\cite{Ostachowicz2012}. 
However, Lamb could not able to generate those waves physically, until 
Worlton~\cite{Worlton1961} who saw the opportunity to utilise Lamb waves 
characteristics in damage detection~\cite{Ostachowicz2012}.
Lamb waves, in general, are generated and received by piezoelectric (PZT) 
transducers~\cite{Cai2012}.
Due to the multi-mode and dispersion properties, the propagation of Lamb waves 
is quite complex~\cite{Ostachowicz2012}. 
In practical applications, two forms of Lamb waves arise depending on the 
distribution of the displacement on the top and bottom bounding surfaces, these 
forms are symmetric, denoted as \(S_0,S_1,S_2,...., \)and antisymmetric, denoted as 
\(A_0,A_1,A_2,....,\) ~\cite{Ostachowicz2012}. 
Fig.~\ref{fig:LambModes} illustrates the propagation of the fundamental Lamb waves for \(A_0\) and \(S_0\) modes in a structure.
%%%%%%%%%%%%%%%%%%%%%%%%%%%%%%%%%%%%%%%%%%%%%%%%%%%%%%%%%%%%%%%%%%%%%%%%%%%%%%%%
\begin{figure}[!ht]
	\begin{center}
		\centering
		\includegraphics[width=1\textwidth]{Figures/Chapter_1/fig_Lamb_wave_modes.png}
	\end{center}
	\caption{Fundamental Lamb wave modes: (a)~\(A_0\) mode, (b)~\( S_0\) mode} 
	\label{fig:LambModes}
\end{figure} 
%%%%%%%%%%%%%%%%%%%%%%%%%%%%%%%%%%%%%%%%%%%%%%%%%%%%%%%%%%%%%%%%%%%%%%%%%%%%%%%%
\paragraph{}
Regardless of Lamb waves promising characteristics, using them for SHM 
applications hold some essential challenges. 
Among them are the dispersive nature of Lamb wave propagating modes that can convert into each other in the presence of defects and other changes in the mechanical 
impedance~\cite{Willberg2015}. 
Moreover, ascribed to some flaws in the bonding within actuators sensors and 
the structure, random noise will emerge in the relevant sensors due to the high 
sensitivity of Lamb waves toward structural perturbations. 
Also, noise arising from environmental sources, like temperature changing, or 
anisotropy of the material also summed up to the received signals making them 
very complicated and challenging to recognize and interpret~\cite{Willberg2015}.
Moreover, an essential point concerns the choice of a carrier frequency for the 
Lamb waves because the higher the frequency is, the damage detection of small 
size is more likely detected.
However,  when the frequency increases, the number of propagating wave modes will increase accordingly.
As a result, multiple wave modes propagate and each wave mode has different velocity which causes a problem with reflection identification and misinterpretation of the location and the size of the damage~\cite{Ostachowicz2012}. 
It was found that each wave mode shows a varying sensitivity to individual 
damage. 
\textcite{Kessler2002b,Ihn2008} found that \(A0\) mode is suitable for delaminations to be detected in composite materials, and \(S0\) mode was found suitable for cracks detection in metallic elements~\cite{Ihn2004,Ihn2008}.
It was also observed that the design of the transducer influences in a great manner the excited and registered wave modes~\cite{Ostachowicz2010}.
%% SECTION HEADER ////////////////////////////////////////////////////////////////////////////////
\section[Damage identification]{Damage detection and localisation using guided waves}
\label{sec23}
Damage can be defined as changes occurred in a system, either deliberately or accidentally, that adversely alter the current or future performance of the system~\cite{Farrar2012}. 
Generally, guided wave based SHM systems can be built upon processing of signals registered by different types of sensors such as PZTs, optical fibre sensors (e.g. FBG), in addition to Scanning Laser Doppler Vibrometry (SLDV), which currently is considered as Non-Destructive Test (NDT) tool.
%%%%%%%%%%%%%%%%%%%%%%%%%%%%%%%%%%%%%%%%%%%%%%%%%%%%%%%%%%%%%%%%%%%%%%%%%%%%%%%%
\subsection{Piezoelectric transducer} 
Piezoelectric transducers (piezoceramic PZT) are utilised in SHM systems to excite guided waves within structures and sensing the reflected signals. 
Based on arrangement of PZTs, two main approaches are available: \emph{pulse-echo} and \emph{pitch-catch} as presented in Fig. \ref{fig:Pulse_echo_Pitch_catch}.
In \emph{pulse-echo}, it is possible to have a group of PZTs located closely, which can excited to generate Lamb waves. 
The reflected waves from the damage are registered at the same or another PZT, this method relies on the reflection from the damage. 
While in the \emph{pitch-catch} approach, generated Lamb waves by PZTs (actuators) are transferred through the damage and registered at PZTs (sensors).
%%%%%%%%%%%%%%%%%%%%%%%%%%%%%%%%%%%%%%%%%%%%%%%%%%%%%%%%%%%%%%%%%%%%%%%%%%%%%%%%
\begin{figure}[!ht]
	\begin{center}
		\centering
		\includegraphics[width=1\textwidth]{Figures/Chapter_1/Pulse_echo_Pitch_catch.png}
	\end{center}
	\caption{(a) Pulse echo	(b) Pitch catch} 
	\label{fig:Pulse_echo_Pitch_catch}
\end{figure}
%%%%%%%%%%%%%%%%%%%%%%%%%%%%%%%%%%%%%%%%%%%%%%%%%%%%%%%%%%%%%%%%%%%%%%%%%%%%%%%%
In general, configurations of PZT transducers for damage detection and localisation for SHM can be classified into two main arrangements that are \emph{concentrated} and \emph{distributed} arrangements. 
Hence, a lot of work was performed in the literature utilising PZT configurations for generating and sensing  Lamb waves.

The following research articles are examples in which the authors used the \emph{concentrated} transducers arrangement.
\textcite{Giurgiutiu2006} implemented PZT wafer active sensor (PWAS) in phased array to investigate Lamb waves in plates.
The results that he obtained were encouraging regarding the location of the damage and its size.
Additionally,~\textcite{Wilcox2003}, investigated omni-directional wave transducer arrays for the rapid inspection of large areas of plate structures. 
In his work, two arrangements of PZTs were examined. 
The first one consists of a densely circular area with PZTs in which it presented an excellent concentrated peak at the location of the reflector, though it requires plenty of transducers. 
The other arrangement consists of a single circular ring of PZTs, which is quite efficient in any circumstance that involves various reflectors.
Moreover,~\textcite{Malinowski2009} performed a numerical analysis on an array of PZTs of a star shape for various damage scenarios. 
Their method confirmed a good damage localisation.

Furthermore, the \emph{distributed} arrangement was implemented in many research articles. 
In this arrangement, PZT transducers are spread on the entire inspected area.
\textcite{Schubert2008} tested different types of the above-mentioned arrangements. 
Moreover,~\textcite{Qiang2009} used a rectangular network of transducers on composite material, whereas a triangular network of transducers was examined in~\cite{Wandowski2009} for an isotropic specimen.

It can be concluded from previous works that using these approaches for damage detection and localization is only suitable for simple structures. 
Furthermore, the estimation of damage size is very challenging.
It is because of limited information extracted from the registered signals at discrete PZT locations. 
These challenges arise due to various limitations (e.g. the added mass and attached cables to the structure alter the propagating waves). 
Additionally, it is difficult to distinguish the registered signals among different objects (e.g. bolts and rivets), the edges, and the actual damage. 
Another challenge arises due to the effect of temperature on propagating guided waves, as it will change their amplitude and phase (the arrival time)~\cite{Putkis2015}.
Therefore, the increase in the temperature will cause the amplitude of the guided waves to decrease and the arrival time to increase.
Therefore, it becomes important to compensate for this issue~\cite{Marzani1999}.
Moreover, it is impossible to obtain high resolution damage influence maps with sparse array of sensors.
To overcome these limitations, a full wavefield measurement approach was introduced. 
As a result of utilising a full wavefield, a damage influence map is produced, which makes it possible to estimate the size of the damage~\cite{Ostachowicz2014}.
%%%%%%%%%%%%%%%%%%%%%%%%%%%%%%%%%%%%%%%%%%%%%%%%%%%
\subsection{Fibre Bragg Gratings} 
Fibre Bragg Gratings (FBG) are a sort of regular quasi-distributed fibre optic sensors (FOSs) in real-time monitoring~\cite{Cai2012}.  
FBG sensors are commonly adopted for their particular advantages, such as lightweight, small size, high stability, corrosion and electromagnetic interference resistance. 
Furthermore, FBG sensors are resistant to fluctuations in power supply and are easily embedded in different materials such as composite materials~\cite{Jang2012}. 
Applying multiplexing techniques such as wavelength division multiplexing, or time-division multiplexing, a quasi-distributed sensor network can synchronously identify multi-point monitoring of the strain and temperature inside the material, hence, enhancing the sensitivity and performance of composite SHM~\cite{Jang2012}.

FBGs have been utilised for composite materials since the 1970s~\cite{othonos1999fiber} and have been fully developed in SHM. 
Due to their unique advantages and diversity, FBGs have been broadly used in advanced spacecraft, aircraft, navigation and medical applications. 
Nowadays, FBGs are used to monitor several defects such as delamination growth, fatigue evolution, and transversal crack appearance~\cite{Kinet2014, Guemes, SelimKocaman} of composite materials such as CFRP and Glass Fiber Reinforced Plastic (GFRP).

One of the basic approaches for monitoring the damage is by exciting Lamb waves that propagate through the structure, then detected by the FBG sensor array~\cite{Soman2021, Wee2021, Soman}. 
Due to their multiplexing abilities, the FBG sensor arrays can monitor areas with a large surface~\cite{Wee2017}. 
Nonetheless, the main disadvantages of FBG sensor arrays are their high price and low sensitivity to the surface waves compared to PZTs. 
Various approaches exist to enhance this sensitivity, from modifying the spectral output of the FBG sensors to adjusting the sensor coating to making resonance conditions on the FBG sensor~\cite{Wee2017}.
%%%%%%%%%%%%%%%%%%%%%%%%%%%%%%%%%%%%%%%%%%%%%%%%%%
\subsection{Scanning Laser Doppler Vibrometry} 
Scanning Laser Doppler Vibrometer (SLDV) was developed and presented in experimen\-tal research in the earlies of 1980s. 
SLDV employs Doppler frequency shift principle to measure the velocity of a moving object in which the amount of the shifted frequency depends on the velocity of the moving object~\cite{Stanbridge1999}. 
SLDV links a computer-controlled XY scanning mirror with a camera inside the optical head, which densely scans the vibrating surface of the structure and gets a large number of high-resolution measurements\cite{Helfrick2011}. 
Essentially, the grid of points resembles a dense array of PZTs. 
Application of such a dense array of PZTs would be otherwise impractical.  
Hence, SLDV is employed for full wavefield measurements instead of PZT arrays. 
Consequently, vibrations of a structure and the propagation of guided waves can be measured accurately~\cite{Ostachowicz2014}.

However, in many situations, it is necessary to obtain information about the vibrations of the measured object in three dimensions. 
In such situations, a 3D vibrometer is used, which holds three 1D scanning vibrometer heads in addition to the data acquisition system and a control system.
A 3D vibrometer measures a location with three independent laser beams that hit the target from three different directions, which yields a measurement of the complete in-plane and out-of-plane velocity of the target.

SLDV has been broadly used for sensing Lamb waves. 
Several works in the literature are concentrated on damage imaging methods for damage identification by using the signals sensed at a grid of points and recorded by SLDV.
For instance, authors in~\cite{Yu2013} applied a frequency wavenumber domain analysis utilising a 2D Fourier transform to detect a crack in an aluminium plate. 
The method of wavenumber frequency filtering of SLDV data was applied for damage imaging in~\cite{Ruzzene2007}. 
Authors in~\cite{Kudela2015} introduced a new method of imaging crack growth in a structure.
In the proposed method, they employed full wavefield data captured by SLDV.
Also, authors in~\cite{Harb2015} utilized SLDV based measurement for inferring the dispersion curves for \(A0\) Lamb wave mode. 
Moreover, SLDV has been used to scan and capture Lamb waves in various types of composite plates for damage detection~\cite{Lamboul2013, Radzienski2019,Sohn2011, An2016,Rogge2013,  Tian2015}.

Despite all the advantages of utilising SLDV, there are some disadvantages. 
The first drawback concerns the surface of the specimen, which must be smooth and characterised by proper reflectivity. Otherwise, the captured signal to noise ratio will be decreased~\cite{Ostachowicz2014}. 
Furthermore, experimenting using  SLDV requires much time since the SLDV performs measurements at a single point in space at a time.
Due to registering a full wavefield of Lamb waves, the process of measurements must be repeated by keeping the same excitation and pause until the wave completely attenuates~\cite{Ostachowicz2014}.
%%%%%%%%%%%%%%%%%%%%%%%%%%%%%%%%%%%%%%%%%%%%%%%%%%%%%%%%%%%%%%%%%%%%%%%%%%%%%%%
\section{Compressive Sensing of wavefield}
As previously stated, guided waves, specifically Lamb waves, are frequently used for SHM and NDT.
For point-wise measurements in the former scenario, an array of transducers is typically used.
These are typically piezoelectric transducers that can function as actuators as well as sensors, as in active guided wave-based SHM.
It should be emphasized that round-robin actuator-sensor measurements can be performed very quickly, allowing for near-real-time monitoring of a structure.

There has been a lot of recent research on the application of SLDV for NDT \cite{Flynn2013,Kudela2015,Kudela2018d,Segers2021,Segers2022}.
For guided wave excitation, either a piezoelectric transducer or a pulse laser is utilized, and measurements are taken by SLDV at one location on the surface of an inspected structure.
The method is continued automatically for other points in a scanning fashion until the full wavefield of Lamb waves is obtained.

Full wavefield measurements are taken on a very dense grid of points opposite to sparsely measured signals by sensors.
Hence, deliver much more useful data from which information about damage can be extracted in comparison to signals measured by an array of transducers.
On the other hand, SLDV measurements take much more time than measurements conducted by an array of transducers.
It makes the SLDV approach unsuitable for SHM in which continuous monitoring is required.
But it is very capable for offline NDT applications.

One can imagine that in a future matrix of laser heads instead of a single laser head used nowadays will be developed to reduce SLDV measurement time.
Alternatively, compressive sensing (CS) and/or deep learning super-resolution (DLSR) can be applied.
It means that SLDV measurements can be taken on a low-resolution grid of points and then full wavefield can be reconstructed at high-resolution.

CS was originally proposed in the field of statistics~\cite{Candes2006,Donoho2006} and used for efficient acquisition and reconstruction of signals and images.
It assumes that a signal or an image can be represented in a sparse form in another domain with appropriate bases (Fourier, cosine, wavelet).
On such bases, many coefficients are close or equal to zero.
The sparsity can be exploited to recover a signal or image from fewer samples than required by the Nyquist–Shannon sampling theorem.
However, there is no unique solution for the estimation of unmeasured data.
Therefore, optimisation methods for solving under-determined systems of linear equations that promote sparsity are applied~\cite{Chen1998,VanEwoutBerg2008,VandenBerg2019}.
Moreover, a suitable sampling strategy is required.

Since then, CS has found applications in medical imaging~\cite{Lustig2007}, communication systems~\cite{Gao2018}, and seismology~\cite{Herrmann2012}.
It is also considered in the field of guided waves and ultrasonic signal processing~\cite{Harley2013,Mesnil2016,Perelli2012,Perelli2015,DiIanni2015,KeshmiriEsfandabadi2018,Chang2020}

Harley and Mura~\cite{Harley2013} utilised a general model for Lamb waves propagating in a plate structure (without defects) and $L_1$ optimisation strategies to recover their frequency-wavenumber representation. 
They applied sparse recovery by basis pursuit and sparse wavenumber synthesis.
They used a limited number of transducers and achieved a good correlation between the true and estimated responses across a wide range of frequencies.
Mensil and Ruzzene~\cite{Mesnil2016} were focused on the reconstruction of wavefield that includes the interaction of Lamb waves with delamination.
Similar to previous studies, analytic solutions were utilised to create a compressive sensing matrix.
However, the limitation of these methods is that dispersion curves of Lamb waves propagating in the analysed plate have to be known a priori.

Perelli et al.~\cite{Perelli2012} incorporated the warped frequency transform into a compressive sensing framework for improved damage localisation.
The wavelet packet transform and frequency warping was used in~\cite{Perelli2015} to generate a sparse decomposition of the acquired dispersive signal.

Di Ianni et al.~\cite{DiIanni2015} investigated various bases in compressive sensing to reduce the acquisition time of SLDV measurements.
Similarly, a damage detection and localisation technique based on a compressive sensing algorithm was presented in~\cite{KeshmiriEsfandabadi2018}.
The authors have shown that the acquisition time can be reduced significantly without losing detection accuracy.

Another application of compressive sensing was reported in~\cite{Chang2020}. 
The authors used signals registered by an array of sensors for tomography of corrosion.
They investigated the reconstruction success rate depending on the number of actuator-sensor paths.

The group of DLSR methods is applied mostly to images~\cite{Dahl2017,Zhang2018,Wang2019} and videos~\cite{Zhang2017,Yan2019}.
Image super-resolution (SR) is the process of recovering high-resolution images from low-resolution images.
A similar approach can be used in videos where data is treated as a sequence of images.
Notable applications are medical imaging, satellite imaging, surveillance and security, astronomical imaging, amongst others.
Also deep learning super sampling developed by NVIDIA and FidelityFX super-resolution developed by AMD was adopted for video games~\cite{Claypool2006}.
Mostly supervised techniques are employed
which benefit from recent advancements in deep learning methods ranging from enhanced convolutional neural networks (CNN)~\cite{Zhang2017}, through an extension of PixelCNN~\cite{Dahl2017} to generative adversarial networks (GANs)~\cite{Wang2019}, to name a few.
Nevertheless, so far neither of these methods have been applied to wavefields of propagating Lamb waves.
The exception is an enhancement of wavefields as the second step of SR followed by classic CS~\cite{Park2017a,KeshmiriEsfandabadi2020}.
%% SECTION HEADER ////////////////////////////////////////////////////////////////////////////////
\section{Summary}
\label{sec24}
Engineering structures are vulnerable to several types of damage that may occur naturally or artificially.
Hence, the damage will reduce the expected lifetime of the running structures, increase their maintenance costs, and sometimes may lead to catastrophic consequences. 
Therefore, to avoid such consequences, SHM techniques are applied.  
Moreover, I discussed that SHM techniques are able to detect any possible change that occurs at a structure that could decay the performance of the whole system, at the earliest possible time so that an action can be taken to reduce the downtime,  operational costs and maintenance costs, consequently reducing the risk of catastrophic failure, injury, or even loss of life.

In this chapter,  several SHM approaches for detecting and localising damage within composite structures that utilise guided Lamb waves were presented. 
I illustrated that guided-wave based SHM systems can be built upon the processing of signals registered by PZTs or SLDV. 
%% SECTION HEADER ////////////////////////////////////////////////////////////////////////////////

%\input{Chapters/Chapter2/sect25}
%\input{Chapters/Chapter2/sect26}
%\input{Chapters/Chapter2/sect27}
%%% SECTION HEADER ////////////////////////////////////////////////////////////////////////////////
\section{Summary}
\label{sec28}
In this chapter, author discussed problems of ageing structures which are exposed to have several types of damage. 
Structures suffer from damage whether natural or artificial, may reduce their expected lifetime, increasing their maintenance costs, and some time may lead to catastrophic consequences. 
Therefore, to avoid such consequences SHM techniques could be applied.  
Moreover, author discussed that SHM techniques can detect any possible change that occurs at a structure which could decay the performance of the whole system, at the earliest possible time so that an action can be taken to reduce the downtime,  operational costs and maintenance costs, consequently reducing the risk of catastrophic failure, injury, or even loss of life.

In this chapter,  several SHM approaches for detecting and localising damage within composite structures that utilise guided Lamb waves were presented. 
Author illustrated that guided-wave based SHM systems can be built upon processing of signals registered by PZTs or SLDV. 
Moreover, author presented in this chapter several techniques that had studied and examined guided Lamb waves in composite materials to detect and localise the damage using signal processing techniques. 
Consequently, author concluded that those traditional techniques are complex and involve a huge numerical analysis and signal processing. Which concluded that the damage features are difficult to be extracted manually. 
Thus, new approaches that involve Machine and Deep Learning techniques are utilised are presented in this chapter. 
As a result, the process of damage features extracting became more convenient and easier since the machine is responsible for learning the new features and accordingly  detect and localise the damage. 
In consequence, it is concluded that the advantage of this approach is the improvement of feature damage extracting procedure.

Furthermore, problems with conventional damage detection techniques for SHM and the importance of the artificial intelligence approach were discussed.
Furthermore, in the second section of the chapter, the author introduced the ML approach in the SHM field.
Moreover, several techniques for feature extraction such as PCA, MSD, and GMMs were described. 
Further, several classification models such as SVM, KNN, and decision trees were introduced.
In the third section, DL approach was presented, in which techniques such as CNN  and RNN were presented.
Finally, several deep learning techniques for damage detection used regarding the SHM field based on guided waves and vibration approaches were presented.%% SECTION HEADER ////////////////////////////////////////////////////////////////////////////////
