%% SECTION HEADER ////////////////////////////////////////////////////////////////////////////////
\section{Delamination identification using animation of full wavefield frames}
\label{sec44}
In this section, we used full wavefield frames of Lamb waves propagation in an end-to-end deep learning model to identify CFRP delamination instead of using RMS images as in the FCN models.
Accordingly, a many-to-one prediction scheme was adopted by employing convolutional-based recurrent neural networks (RNNs) to perform pixel-wise image segmentation.
A sequence of full wavefield frames is fed into the proposed deep learning models in order to identify the delamination.
Hence, in the segmentation problem, there are two classes: undamaged and damaged.
To the best of our knowledge, it is the first implementation of deep neural networks utilising lamb wave propagation animations for damage imaging with semantic segmentation.
The proposed model showed excellent results in identifying the delamination in the numerically generated dataset, and it also showed their capability to generalise delamination identification in real world scenarios.
%%%%%%%%%%%%%%%%%%%%%%%%%%%%%%%%%%%%%%%%%%%%%%%%%%%%%%%%%%%%%%%%%%%%%%%%%%%%%%%%
\subsection{Data preprocessing}
%%%%%%%%%%%%%%%%%%%%%%%%%%%%%%%%%%%%%%%%%%%%%%%%%%%%%%%%%%%%%%%%%%%%%%%%%%%%%%%%
%%%%%%%%%%%%%%%%%%%%%%%%%%%%%%%%%%%%%%%%%%%%%%%%%%%%%%%%%%%%%%%%%%%%%%%%%%%%%%%%
\subsection{Encoder-Decoder ConvLSTM model}
\label{proposed_approach}
%%%%%%%%%%%%%%%%%%%%%%%%%%%%%%%%%%%%%%%%%%%%%%%%%%%%%%%%%%%%%%%%%%%%%%%%%%%%%%%%
In this work, we developed an end-to-end deep learning model utilising full wavefield frames of Lamb wave propagation for delamination identification in CFRP materials as presented in Fig.~\ref{fig:proposed_convLSTM_model}.
The developed model has a scheme of many-to-one sequence prediction, which takes \(n\) number of frames representing the full wavefield propagation through time and their interaction with the delamination to extract damage features, and finally predict the delamination location, shape, and size in a single output image.
%%%%%%%%%%%%%%%%%%%%%%%%%%%%%%%%%%%%%%%%%%%%%%%%%%%%%%%%%%%%%%%%%%%%%%%%%%%%%%%%
\begin{figure} [!h]
	\centering
	\includegraphics[width=5cm]{Figures/Chapter_4/figure3b.png}
	\caption{Encoder-Decoder ConvLSTM model architecture.}
	\label{fig:proposed_convLSTM_model}
\end{figure} 
%%%%%%%%%%%%%%%%%%%%%%%%%%%%%%%%%%%%%%%%%%%%%%%%%%%%%%%%%%%%%%%%%%%%%%%%%%%%%%%%

In the implemented model, we applied an autoencoder technique (AE) which is well-known for extracting spatial features.
The idea of AE is to compress the input data within the encoding process then learn how to reconstruct it back from the reduced encoded representation (latent space) to a representation that is as close to the original input as possible. 
Hence, we have investigated the use of AE to process a sequence of input frames to perform image segmentation.
Therefore, a Time Distributed layer presented in Fig.~\ref{fig:TD} was introduced to the model, in which it distributes the input frames into the AE layers in order to process them independently.
%%%%%%%%%%%%%%%%%%%%%%%%%%%%%%%%%%%%%%%%%%%%%%%%%%%%%%%%%%%%%%%%%%%%%%%%%%%%%%%%
\begin{figure}[!h]
	\centering
	\includegraphics[width=5cm]{Figures/Chapter_4/figure4_TD.png}
	\caption{Flow of input frames using Time distributed layer.}
	\label{fig:TD}
\end{figure}
%%%%%%%%%%%%%%%%%%%%%%%%%%%%%%%%%%%%%%%%%%%%%%%%%%%%%%%%%%%%%%%%%%%%%%%%%%%%%%%%

As previously mentioned, an AE consists of three parts: the encoder, the bottleneck, and the decoder.
The encoder is responsible for learning how to reduce the input dimensions and compress the input data into an encoded representation.

The encoder part presented in~\ref{fig:proposed_convLSTM_model} consists of four levels of downsampling. 
The purpose of having different scale levels is to extract feature maps from the input image at different scales.
Every level at the encoder consists of two 2D convolution operations followed by a Batch Normalization then a Dropout is applied. 
Furthermore, at the end of each level a Maxpooling operation is applied to reduce the dimensionality of the inputs. 

The bottleneck presented in Fig.~\ref{fig:proposed_convLSTM_model} has the lowest level of dimensions of the input data, further it consists of two 2D convolution operations followed by a Batch Normalization.

The decoder part presented in Fig.~\ref{fig:proposed_convLSTM_model}, is responsible for learning how to restore the original dimensions of the input.
The decoder part consists of two 2D convolutional operations followed by Batch Normalization and Dropout, and an upsampling operation is applied at the end of each decoder level to retrieve the dimensions of its inputs.
Skip connections linking the encoder with the corresponding decoder levels were added to enhance the features extraction process.
The outputs of the decoder were forwarded into the ConvLSTM2D layer to learn long-term spatiotemporal features.

Further, we applied a 2D convolutional layer as the final output layer followed by a sigmoid activation function which outputs values in a range from \((0,1)\) to indicate the delamination probability.
Consequently, a threshold value must be chosen to classify the output into a damaged represented by (\(1\)) or undamaged represented by (\(0\)).
Hence, we set the threshold value to (\(0.5\)) to exclude all values below the threshold by considering them as undamaged and taking only those values greater than the threshold to be considered as damaged.

For evaluating the performance of the proposed model, the mean 
intersection over union \(IoU\) (was also used in FCN models and CNN classifier model) was applied as the accuracy metric. 
\(IoU\) is estimated by determining the intersection
area between the ground truth and the predicted output. 