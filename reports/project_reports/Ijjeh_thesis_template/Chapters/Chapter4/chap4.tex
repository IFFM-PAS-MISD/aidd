%% CHAPTER HEADER /////////////////////////////////////////////////////////////////////////////////////
\chapter[Methodology]{Methodology}
\label{ch4}

%% CHAPTER INTRODUCTION ///////////////////////////////////////////////////////////////////////////////
In this chapter, several DL methodologies were presented in order to detect and localise delamination in composite laminates based on non-contact ultrasonic wavefield imaging that can provide precise details about the health status of the investigated specimen.
Accordingly, a large synthetic dataset was generated representing the full wavefield of the Lamp waves propagating in a CFRP plate and their interaction with discontinuities such as the delamination and the plate edges, acquired by SLDV from the bottom surface of the plate.
The acquisition process of the synthetic dataset is explained in details in the section~\ref{sec41}.

The deep learning models were trained using a supervised scheme. 
Hence, it uses a training set to train the developed models to predict the desired output.
The developed methods are end-to-end approaches, in which the whole unprocessed training dataset is fed into the model.
Hence, it will learn by itself to identify distinct patterns and detect damage.
In section~\ref{sec42}, a fully connected CNN classifier model for detecting and localising delamination is presented in details.
In section~\ref{sec43}, I present five FCN models for delamination identification. 
Further, the developed models were trained using the RMS images in which the developed models are performing image pixel-wise image segmentation.
In section~\ref{sec44}, a deep learning model for delamination identification utilising the animation of full wavefield is presented.

As mentioned earlier, ultrasonic wavefield imaging with SLDV can provide detailed information on the health status of an inspected structure.
However, high spatial resolution, which is frequently required for precise damage estimation, typically demands a prolonged scanning time.
In section~\ref{sec45}, I present a DL model for image super-resolution reconstruction in which the full wavefields of Lamb waves are recovered from low-resolution wavefields acquired with a reduced number of measuring points into high-resolution wavefields.
Accordingly, this can allow for a precise recovery of propagating waves and their interactions with discontinuities such as delaminations and boundaries of the plate.
Consequently, this process will speed up the acquisition of data.
Furthermore, the reconstructed full wavefield can be used in damage imaging. 


All developed models were implemented and trained with the Keras API~\cite{chollet2015keras} running on top of TensorFlow~\cite{Abadi2016}.
The NVIDIA GeForce RTX 2070 with \(8\) GBs of memory was utilised to train the CNN classification models.
Furthermore, the NVIDIA Tesla V100 GPU with \(32\) GBs of memory was utilised to train the FCN models, the AE-ConvLSTM model, and the DLSR model.
%% INCLUDE SECTIONS ///////////////////////////////////////////////////////////////////////////////////

%% SECTION HEADER /////////////////////////////////////////////////////////////////////////////////////
\section{Synthetic data acquisition}
\label{sec41}
%%%%%%%%%%%%%%%%%%%%%%%%%%%%%%%%%%%%%%%%%%%%%%%%%%%%%
The crucial part in our work was in synthetically generating a dataset of a full wavefield of propagating of Lamb waves in a plate made of CFRP.
In which we
In this work, we have generated a large dataset of \(475\) cases of a full wavefield of propagating Lamb waves in a plate made of carbon fibre-reinforced plastic (CFRP).
The in-house code of the time-domain spectral element method was used for simulation of Lamb wave interaction with delamination~\cite{Kudela2020}.
It should be added that despite the utilisation of the parallel code of the spectral element method which was run on the Tesla K20X GPU card, the computation of the dataset (consisting of 475 cases) took about 3 months.
For each case, single delamination was modelled by using the method of splitting nodes between appropriate spectral elements. 
It was assumed that the composite laminate is made of eight layers of a total thickness of 3.9 mm.
The delamination was modelled between the third and fourth layer (see Fig.~\ref{fig:plate_setup} for details).
It should be noted that Fig.~\ref{fig:plate_setup} shows an exaggerated cross-section through the delamination. 
Zero-volume delamination was assumed in the model. 
Delamination spatial location was selected randomly so that the interaction of guided waves with delamination is different for each case.
It includes cases when delamination is located at the edge of the plate which is the most difficult to identify by signal processing methods.
Additionally, the size of the delamination of elliptic shape was randomly simulated by selecting the size of ellipse minor and major axis.
Also, the angle between the delamination major axis and the horizontal axis was randomly selected.
In summary the following random factors were simulated in each case:
\begin{itemize}
	\item delamination geometrical size	(ellipse minor and major axis randomly selected from the interval \(\left[10 \, \textrm{mm}, 40\, \textrm{mm}\right]\)),
	\item delamination angle (randomly selected from the interval \( \left[ 0^{\circ}, 180^{\circ} \right]\)),
	\item coordinates of the centre of delamination (randomly selected from the interval \(\left[0\, \textrm{mm}, 250\, \textrm{mm} -\delta \right]\) and \( \left[250\, \textrm{mm}+\delta, 500\, \textrm{mm} \right] \), where \(\delta = 10\, \textrm{mm}\)).
\end{itemize}
It resulted in random spatial placement of delaminations. The plate with overlayed 475 delamination cases is shown in Fig.~\ref{fig:random_delam}.
\begin{figure}
	\centering
	\includegraphics[scale=0.8]{Figures/Chapter_3/plate_delam_arrangement_MSSP.PNG}
	\caption{Setup for computing Lamb wave interactions with delamination.}
	\label{fig:plate_setup}
\end{figure}	
\begin{figure}
	\centering
	\includegraphics[scale=1]{Figures/Chapter_3/dataset2_labels_ellipses.png}
	\caption{The plate with 475 cases of random delaminations.}
	\label{fig:random_delam}
\end{figure}

Guided waves were excited at the plate centre by applying equivalent piezoelectric forces.
The excitation signal had a form of sinusoid modulated by Hann window. 
It was assumed that the carrier frequency is 50 kHz and the modulation frequency is 10 kHz.
A relatively low carrier frequency allowed for lower mesh density and significant computation time reduction in comparison to simulations of higher frequencies.
Additionally, the excitation signal was selected so that interaction of generated A0 Lamb wave mode with the smallest delamination can be still used as a feature for damage identification.

The output from the top and bottom surfaces of the plate in the form of particle velocities at the nodes of spectral elements were interpolated on the uniform grid of 500\(\times\)500 points by using shape functions of elements (see~\cite{Kudela2020} for more details).
It essentially resembles measurements acquired by SLDV in the transverse direction (perpendicular to the plate surface).
An example of the simulated full wavefield data on the top and bottom surfaces is presented in Fig.~\ref{fig:wavefield}.
It should be noted that stronger wave entrapment at delamination can be observed for the case of the wavefield at the top surface.
It is because the delamination within cross-section is located closer to the top surface.
It makes it easier to detect delamination by processing wavefield at the top surface.
It is better visible if the root mean square (RMS) according to Eq.~(\ref{eq:rms}) is applied to the wavefield.
The result of this operation is presented in Fig.~\ref{fig:rms}.
Based on the image analysis, the shape of the delamination can be easier to discern for the top case.
\begin{figure} [h!]
	\centering
	\begin{subfigure}[b]{0.32\textwidth}
		\centering
		\includegraphics[scale=1]{Figures/Chapter_3/96_flat_shell_Vz_1_500x500top.png}
		\caption{\(t=0.141\) ms}
		\label{fig:frame96top}
	\end{subfigure}
	\hfill
	\begin{subfigure}[b]{0.32\textwidth}
		\centering
		\includegraphics[scale=1]{Figures/Chapter_3/128_flat_shell_Vz_1_500x500top.png}
		\caption{\(t=0.188\) ms}
		\label{fig:frame128top}
	\end{subfigure}
	\hfill
	\begin{subfigure}[b]{0.32\textwidth}
		\centering
		\includegraphics[scale=1]{Figures/Chapter_3/164_flat_shell_Vz_1_500x500top.png}
		\caption{\(t=0.240\) ms}
		\label{fig:frame164top}
	\end{subfigure}	
	\hfill
	\begin{subfigure}[b]{0.32\textwidth}
		\centering
		\includegraphics[scale=1]{Figures/Chapter_3/96_flat_shell_Vz_1_500x500bottom.png}
		\caption{\(t=0.141\) ms}
		\label{fig:frame96bottom}
	\end{subfigure}
	\hfill
	\begin{subfigure}[b]{0.32\textwidth}
		\centering
		\includegraphics[scale=1]{Figures/Chapter_3/128_flat_shell_Vz_1_500x500bottom.png}
		\caption{\(t=0.188\) ms}
		\label{fig:frame128bottom}
	\end{subfigure}
	\hfill
	\begin{subfigure}[b]{0.32\textwidth}
		\centering
		\includegraphics[scale=1]{Figures/Chapter_3/164_flat_shell_Vz_1_500x500bottom.png}
		\caption{\(t=0.240\) ms}
		\label{fig:frame164bottom}
	\end{subfigure}
	
	\caption{Full wavefield at the top surface (a)--(c) and the bottom surface (d)--(f), respectively, at selected time instances showing the interaction of guided waves with delamination.}
	\label{fig:wavefield}
\end{figure} 

\begin{figure} [h!]
	\centering
	\begin{subfigure}[b]{0.47\textwidth}
		\centering
		\includegraphics[scale=1]{Figures/Chapter_3/RMS_flat_shell_Vz_1_500x500top.png}
		\caption{top}
		\label{fig:rmstop}
	\end{subfigure}
	\hfill
	\begin{subfigure}[b]{0.47\textwidth}
		\centering
		\includegraphics[scale=1]{Figures/Chapter_3/RMS_flat_shell_Vz_1_500x500bottom.png}
		\caption{bottom}
		\label{fig:rmsbottom}
	\end{subfigure}
	\caption{RMS of the full wavefield from the top surface of the plate (a) and the bottom surface of the plate (b).}
	\label{fig:rms}
\end{figure} 
%% SECTION HEADER /////////////////////////////////////////////////////////////////////////////////////
\section{Experimentally data acquisition}
\label{sec42}

%% SECTION HEADER /////////////////////////////////////////////////////////////////////////////////////
\section{Delamination detection using Full connected CNN with bounding boxes}
\label{sec42}

In this section, our initial attempts were on utilising a CNN model regarding delamination detection in CFRP materials is presented.
The model was trained on the on RMS images from the synthetically generated dataset of the propagating Lamb waves (from the top of the surface of the plate) to predict the delamination location using bounding boxes as shown in Fig~\ref{fig:RMS_14}, while, Fig.~\ref{fig:label_14} shows its corresponding ground truth.
%%%%%%%%%%%%%%%%%%%%%%%%%%%%%%%%%%%%%%%%%%%%%%%%%%%%%%%%%%%%%%%%%%%%%%%%%%%%%%%%
\begin{figure} [h!]
	\centering
	\begin{subfigure}[b]{0.47\textwidth}
		\centering
		\includegraphics[width=5cm]{Figures/Chapter_4/RMS_flat_shell_Vz_389_500x500top.png}
		\caption{}
		\label{fig:RMS_14}
	\end{subfigure}
	\hfill
	\begin{subfigure}[b]{0.47\textwidth}
		\centering
		\includegraphics[width=5cm]{Figures/Chapter_4/m1_rand_single_delam_389.png}
		\caption{}
		\label{fig:label_14}
	\end{subfigure}
	\caption{(a) RMS image: from the top of the plate, (b) Label}
	\label{fig:RMS_GT}
\end{figure} 
%%%%%%%%%%%%%%%%%%%%%%%%%%%%%%%%%%%%%%%%%%%%%%%%%%%%%%%%%%%%%%%%%%%%%%%%%%%%%%%%

Accordingly, a CNN model with fully connected dense layers was developed for delamination detection in CFRP.
Moreover, the developed model is based on a supervised learning therefore, with each generated case of delamination a ground truth (label) is given, hence, the model performs a classification task. 

In order to reduce the computation complexity for the model, the dataset for training the model was prepared by resizing the RMS input image to \((448\times 448)\) pixels,  then, was split it into \((14\times 14)\) blocks, and each block has a size of \((32\times 32)\) pixels as shown in Fig.~\ref{fig:RMS_49blocks}.
Consequently, the preprocessed dataset has a size of \((93100\times 32\times 32 \times 1)\), where (\(93100\)) is the total number of blocks for all \(475\) cases.

To examine the effect of increasing the resolution of the RMS image on delamination identification another  preparation was made by resizing the RMS input image to \((512\times 512)\) pixels, then it was split into \(16\times 16\) blocks, and each block has a size of \((32\times 32)\) pixels as shown in Fig.~\ref{fig:RMS_64blocks}.
The second preprocess dataset has a size of \((121600 \times 32 \times 32 \times 1)\), where (\(121600\)) is the total number of blocks for all \(475\) cases.

Further, for each block in the RMS input image there is a corresponding block in the ground truth image of size \((32\times 32)\) as presented in Figs.~\ref{fig:GT_49blocks} and~\ref{fig:GT_64blocks}, respectively.

For training purposes, the dataset was divided into two portions: \(80\%\)	training set and \(20\%\) testing set. 
Additionally, the validation set was created as a \(20\%\) of the training set.
%%%%%%%%%%%%%%%%%%%%%%%%%%%%%%%%%%%%%%%%%%%%%%%%%%%%%%%%%%%%%%%%%%%%%%%%%%%%%%%%
\begin{figure} [h!]
	\centering
	\begin{subfigure}[b]{0.47\textwidth}
		\centering
		\includegraphics[width=5cm]{Figures/Chapter_4/7_7_blocks_389.png}
		\caption{RMS image splitted into (\(14\times 14\)) blocks.}
		\label{fig:RMS_49blocks}
	\end{subfigure}
	\hfill
	\begin{subfigure}[b]{0.47\textwidth}
		\centering
		\includegraphics[width=5cm]{Figures/Chapter_4/8_8_blocks_389.png}
		\caption{RMS image splitted into (\(16\times 16\)) blocks.}
		\label{fig:RMS_64blocks}
	\end{subfigure}
	\hfill
	\begin{subfigure}[b]{0.47\textwidth}
	\centering
	\includegraphics[width=5cm]{Figures/Chapter_4/GT_7_7_389.png}
	\caption{Label image splitted into (\(14\times 14\)) blocks.}
	\label{fig:GT_49blocks}
	\end{subfigure}
	\hfill
	\begin{subfigure}[b]{0.47\textwidth}
		\centering
		\includegraphics[width=5cm]{Figures/Chapter_4/GT_7_7_389.png}
		\caption{Label image splitted into (\(16\times 16\)) blocks.}
		\label{fig:GT_64blocks}
	\end{subfigure}
	\caption{}
	\label{fig:grid_mesh}
\end{figure}
%%%%%%%%%%%%%%%%%%%%%%%%%%%%%%%%%%%%%%%%%%%%%%%%%%%%%%%%%%%%%%%%%%%%%%%%%%%%%%%%

Figure~\ref{CNN_model} presents the implemented CNN model architecture for classification purposes.
The model takes an input block of size \((32\times 32)\) pixels, followed by a convolutional layer that has (\(64\)) filters of size (\(3\times 3\)).
Moreover, in the convolution operation we applied the same padding, and the activation function was Relu.
Then, a pooling layer is applied which has a pool filter of size (\(2\times 2\)) with a stride of (\(2\)).
These operation of convolution and pooling is repeated two times.
The output of the second pooling layer is flattened and fed into the dense layers in which the model has two fully connected layers.
The first dense layer has (\(4096\)) neurons and the second dense layer has (\(1024\)) neurons.
Additionally, Relu activation function was applied for both dense layers.
Moreover, a dropout of probability (\(p = 0.5\)) was added to the model to reduce the overfitting issue.
The final layer in the model is the output layer, in which the model outputs two predictions (damaged and undamaged), hence, a softmax activation function was applied. 
Consequently, the whole block of size \((32\times 32)\) is classified as damaged if there is at least one pixel of delamination, otherwise, it is considered undamaged.
Finally, the predicted output (delamination) is surrounded by a bounding box as the final output.
%%%%%%%%%%%%%%%%%%%%%%%%%%%%%%%%%%%%%%%%%%%%%%%%%%%%%%%%%%%%%%%%%%%%%%%%%%%%%%%%
\begin{figure}[h!]
	\centering
	\includegraphics[scale=1]{Figures/Chapter_4/CNN_model.png}
	\caption{CNN classifier architecture.}
	\label{CNN_model}
\end{figure}
%%%%%%%%%%%%%%%%%%%%%%%%%%%%%%%%%%%%%%%%%%%%%%%%%%%%%%%%%%%%%%%%%%%%%%%%%%%%%%%%

Furthermore, to evaluate the predicted outputs we have utilised two accuracy metrics:
\begin{itemize}
	\item The classification accuracy to measure the capability of the model to detect the delamination.
	\item The intersection over union (IoU) to measure the intersection between the bounding box area which surrounds the predicted delamination and the ground truth delamination.
\end{itemize}

Moreover, selecting a proper loss function during training the model is important since the loss function reflects how good the model learns to predict.
In this model, we have applied a mean square error \((mse)\) loss function which calculates the sum of the squared distances between the predicted output values and the ground truth values.
Moreover, our focus during training the model was on minimizing the loss function and maximizing the accuracy metric.
Accordingly, an optimizer function is required to perform such operation.
In the developed model Adam optimizer was utilised~\cite{Kingma2015}. 

%% SECTION HEADER /////////////////////////////////////////////////////////////////////////////////////
\section{Delamination identification using FCN}
\label{sec44}

%% SECTION HEADER /////////////////////////////////////////////////////////////////////////////////////
\section{Summary}
\label{sec45}
%% SECTION HEADER /////////////////////////////////////////////////////////////////////////////////////
\section{Summary}
\label{sec46}
