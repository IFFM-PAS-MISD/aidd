\section{Summary}
\label{sec34}
In this chapter, I presented several techniques that had studied and examined guided Lamb waves in composite materials to detect and localise the damage using signal processing techniques. 
Consequently, it can be concluded that DL methods have more and more applications in SHM/NDT in recent years. 
However, these are theoretical implementation rather than practical implementation in the field which can evidence of the lack of maturity of these methods. 
The other conclusion can be that it is observed that signal processing methods based on handcrafted feature extraction have been progressed into end-to-end approaches.
%those traditional techniques are complex and involve a huge numerical analysis and signal processing. Which concluded that the damage features are difficult to be extracted manually. 
%Thus, new approaches that involve Machine and Deep Learning techniques are utilised are presented in this chapter.
 
%As a result, the process of damage feature extraction became more convenient and easier since the machine is responsible for learning the new features and accordingly  detect and localise the damage. 
%In consequence, it is concluded that the advantage of this approach is the improvement of the procedure for damage feature extraction.
%
%Furthermore, problems with conventional damage detection techniques for SHM and the importance of the artificial intelligence approach were discussed.
%Furthermore, in the second section of the chapter, I introduced the ML approach in the SHM field.
%Moreover, several techniques for feature extraction such as PCA, MSD, and GMMs were described. 
%
%Further, several classification models such as SVM, KNN, and decision trees were introduced.
%Moreover, DL approach was introduced, in which techniques such as CNN  and RNN were presented.
%Finally, several deep learning techniques for damage detection used regarding the SHM field based on guided waves and vibration approaches were presented.