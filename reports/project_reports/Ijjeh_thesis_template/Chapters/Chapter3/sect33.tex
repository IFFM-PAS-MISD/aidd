\section{Data-driven based SHM/NDT Techniques: Related work}
\label{sec33}
%%%%%%%%%%%%%%%%%%%%%%%%%%%%%%%%%%%%%%%%%%%%%%%%%%%%%%%%%%%%%%%%%%%%%%%%%%%%%%%%
The importance of SHM systems originates from their ability to monitor the condition of structures in real-time.
SHM systems can be developed using data-driven methods, which require a huge amount of data that are captured by monitoring the status of a structure.

The process of extracting features from structures in conventional techniques needs a lot of time and experts in the field.
Therefore, introducing machine learning methods to the feature extraction process became necessary.
Hence, deep learning methods can generalise and learn new features by themselves, which improves their functionality in damage estimation.

DL approach makes it possible to use registered data in their raw form without any need to perform feature extraction.
Hence, such an approach has an end-to-end structure that automatically learns and discovers the hidden features in a high dimensional input data~\cite{LeCun, Networks}.
Figure~\ref{fig:ML_vs_DL} illustrates the main differences between the conventional ML-based SHM and DL-based SHM approaches.

\begin{figure}[!ht]
	\centering
	\begin{subfigure}{1\textwidth}		
		\centering
		\includegraphics[width=1\textwidth]{Figures/Chapter_3/conventional_ML.png}
		\caption{} 
		\label{fig:ML_conventional}
	\end{subfigure}
	\\
	\begin{subfigure}{1\textwidth}
		\centering
		\includegraphics[width=1\textwidth]{Figures/Chapter_3/DL_approach.png}
		\caption{} 
		\label{fig:DL_approach}
	\end{subfigure}	
	\caption{(a) Conventional ML based SHM vs. (b) DL based SHM.}
	\label{fig:ML_vs_DL}
\end{figure}
%%%%%%%%%%%%%%%%%%%%%%%%%%%%%%%%%%%%%%%%%%%%%%%%%%%%%%%%%%%%%%%%%%%%%%%%%%%%%%%%
\textcite{Worden2007} have proposed several axioms related to SHM systems implemented using machine learning methods. 
According to them, damage detection can be perform\-ed in unsupervised learning.
However, recognising the damage type and how significant it is can not be performed without supervised learning. 
Moreover, the feature extraction process is essential for damage detection, and it can be performed by analysing and processing the signals captured by the sensors (e.g. PZT actuators), and then converting them to damage information.
Therefore, introducing machine learning methods to the feature extraction process became necessary.
Hence, machine learning methods can generalise and learn new features by themselves, which improves their functionality in damage estimation.

\subsection{Machine learning based SHM/NDT}
In recent years data-driven methods based on machine learning have increased in a significant way. 
In the following, I will present some methods for damage detection and estimation based on machine learning techniques.

%%%%%%%%%%%%%%%%%%%%%%%%%%%%PZT + SVM
\textcite{Das2010} presented a method for estimating several types of defects (delamination, saw cut, notches, and drilled holes) in composite material. 
For this purpose, a collection of PZT transducers were attached to the surface of the structure to generate and register Lamb waves propagation. 
Accordingly, a time-frequency domain was utilised to extract features related to defects from the registered response. 
Those extracted features were fed to one-class SVM, which performs classification and damage estimation. 
%%%%%%%%%%%%%%%%%%%%%%%%%%%% PZT + SVM

Moreover, \textcite{Dib2018} proposed a novelty classifier based on one-class SVM for detecting damage. 
The method was conducted by extracting data from damage impact on a glass-fibre composite plate and then evaluating the performance of the classifier. 
To extract the necessary features from the propagated wave the registered signal was segmented into L time bins, and the Fourier transform was applied to each time bin.
Accordingly, the features vector was constructed from the signal phase and the amplitude for each segmented time bin.
%%%%%%%%%%%%%%%%%%%%%%%%%%%%%%%%% PZT + PCA+ KNN + SVM

\textcite{Vitola2016} developed a damage detection and classification methodology that was examined on aluminium plates.
An array of PZT transducers was placed on the plate surface to sense wave propagation in the structure.
The methodology is based on the use of principal component analysis (PCA) and machine learning techniques for recognising patterns.
PCA means analysing a large amount of information by finding the principal components.
However, the PCA method is not invariant to scaling, thus, data must be normalized~\cite{Tibaduiza2016}.
Next, normalised data is fed to several machine learning models for training.
For this purpose, several classification algorithms were applied, decision trees, KNN and SVM.
However, only a few of these models presented good outputs in damage detection.

%%%%%%%%%%%%%%%%%%%%%%%%%%% KNN
\textcite{Godin2004} applied Acoustic Emission signals (AE) in their approach, which happen due to a sudden release of stored energy when damage occurs.
AE signals contain important information about the discriminative features 
for the damage type such as fibre breakage, de-cohesion of the interface, or a crack in the matrix in composite materials.
Authors in this work presented supervised and unsupervised classifiers to recognise different damage patterns through grouping AE signals from the tensile tests of unidirectional glass/polyester composite into several different classes. 
For clustering AE signals, a K-means algorithm was used. 
AE signals were clustered based on several metrics such as the AE signal duration, amplitude, rise time, and the number of counts to the peak.
Accordingly, the clustered labelled data is fed into a KNN supervised classifier.
A trained classifier can classify new coming data accordingly.
Regarding the unsupervised classification, the Kohonen classifier was utilised~\cite{58325}, which is a self-organising map (SOM) which is a neural network consisting of neurons as processing units. 

%%%%%%%%%%%%%%%%%%%%%%%%%%% 

\textcite{Nazarko2020} monitored the axial bolt forces using elastic wave propagation signals.
Six-bolt flange connections were put through a series of static tensile tests in the lab. 
For the accurate measurement of axial force, some bolts were equipped with washer load cells.
Additionally, a few bolts were equipped with piezoelectric transducers (actuator and sensor operating in a pitch-catch arrangement) to capture the elastic wave signals.
The outcomes of the ultrasonic testing were then integrated with the artificial neural network (ANN) for both signal compression and as a tool for user interface. 
The outcomes demonstrated that ANNs could predict the axial forces in bolts with a reasonable amount of accuracy. 
Significant potential exists for actual NDT inspections, according to the suggested method~\cite{Nazarko2020}. 

%%%%%%%%%%%%%%%%%%%%%%%%%%% KNN
\textcite{Pashmforoush2014} proposed a technique to classify damage to various lay-up configurations in glass/polyester composites.
For this purpose, the K-means algorithm with the genetic algorithm was utilised. 
PCA was used to reduce the data dimensionali\-ty.
Next, a combination of the K-means algorithm with the genetic algorithm is used for clustering the data. 
The reason for applying the genetic algorithm is to find the optimal number of cluster centres for the KNN algorithm.
Parameters of the AE signals such as peak amplitude, frequency, rise time, energy, and duration were estimated for each cluster and utilised as discriminative features. 
AE signal frequency was found to be a good feature for discrimination. Accordingly, AE signals with the highest frequency were corresponding to fibre breakage, AE signals with the lowest frequency were corresponding to matrix cracking, and the frequencies range in-between were corresponding to the debonding defect. 

%%%%%%%%%%%%%%%%%%%%%%%%%%%%%%%%%%%%%%%%%%%%%%%%%%%%%%%%%%%%%%%%
\textcite{Nazarko2016} investigated the potential of utilising artificially deteriorated signals of Lamb waves in training a novelty detection (ND) system for early damage detection.
To train auto-associative neural networks, the authors used principal components that were generated from signals that were measured experimentally.
The measurements of Lamb waves in the investigated specimens made of aluminium and glass fibre reinforced polymer serve as an excellent illustration of how the ND algorithm accurately handles both simple and complex signals.
It was also noted that the proposed ND method maintained its sensitivity and robustness when it used raw signals with a relatively low sampling rate, on a relatively narrow time window, and further noised signals.
%%%%%%%%%%%%%%%%%%%%%%%%%%%% PZT + ConvNet
%\textcite{Sammons2016} utilised X-ray computed tomography for estimating the delaminations in a CFRP. For this purpose, they utilised the Convolutional Network (ConvNet ) for performing image segmentation of the defected input images to estimate the delaminations. There ConvNet was capable of identifying  and quantifying small delaminations. 
%Unfortunately, the ConvNet could not recognise delaminations with large sizes.
%%%%%%%%%%%%%%%%%%%%%%%%%%%% PZT + ConvNet
%
%Moreover, \textcite{Chetwynd2008} have investigated curved carbon fibre composite panel for damage localisation. 
%Accordingly, stiffeners were used during the experiments to represent real-life damage. 
%For this purpose, authors attached a combination of PZT transducers on the panel used to generate and receive Lamb waves that propagate through the structure. 
%During their propagation through the structure, Lamb waves encounter defects, which affects their propagation response. 
%The collected response was transformed into a novel scaler index using outlier analysis~\cite{Beniger1980}, which was then fed to MLP. 
%The MLP used for classification and regression applications of damage detection. 
%Classification operation is responsible for predicting whether there is damage or not in a specific location. 
%Where the regression operation is responsible for the exact estimation of the damage location.
%%%%%%%%%%%%%%%%%%%%%%%%%%% Ful wavefield +ConvNets
%% SECTION HEADER ////////////////////////////////////////////////////////////////////////////////
\subsection{Deep learning based SHM/NDT}

Deep learning techniques have widely been utilised for the inspection and maintenance of civil infrastructure and have shown very promising results \cite{Cha2017b, Lin2017, liu2019computer, Beckman2019, Choi2020, Sonski2020a, Sonski2020, Sonski2019}. 

Besides the widespread applications of deep learning for SHM/NDT in civil engineering, deep learning is still less investigated for the purpose of damage detection based on guided waves in composite materials.

Guided waves approaches are widely utilised in SHM/NDT due to the fact it can detect very small damage sizes~\cite{Guemes2020}. 
Damage detection and localisation approaches using guided waves are based on the measurements of the PZT sensors, whether bonded or embedded into the investigated structure. 
PZT sensor(s) are responsible for the excitation of the structure by a short ultrasonic pulse (usually, the used frequency is in the range of hundreds of kHz) that propagates through an investigated structure such as plates or pipes as an elastic wave.
The registered signals (baseline) are stored and compared with other registered signals acquired through the lifetime of the investigated structure.
Damage detection using the baseline subtraction approach for guided waves is based on subtracting damage-free registered measurements from the newly registered measurements to obtain the new changes that occurred to the structure.
These changes are considered as damage information.
The baseline approach is effective in controlled environments where the variations of the operational/environments (i.e. considerations of multiple sensing modalities, uncertainty in material properties, bounding conditions, etc.) are negligible~\cite{Yuan2020}.  
Such variations can alter registered data leading to false alarms.
The effect of such variations can be reduced through physics-based modeling, which can simulate an undamaged scenario (baseline) for the wave propagation through the investigated structure.
Then, the simulated baseline can be used in the subtraction for damage detection.
However, in real-world structures, it is difficult to adjust the parameters of the model to match the experimental registered data.
Accordingly, data-driven techniques based on ML and DL approaches can be the solution and deliver robust models for many real-life variations.

In the following, methods for damage size estimation based on machine learning and deep learning techniques are presented, which are targeted in the field of SHM/NDT.

%\textcite{islam1994damage} presented one of the earliest research studies for assessing delamination location and size in composite structures using deep learning techniques.
%They trained a neural network model using frequencies from modal analysis data for the first five modes.
%Data were obtained using piezoceramic sensors in both damaged and undamaged composite beams.
%In the following, several approaches utilising guided waves for SHM/NDT based on data-driven techniques for damage detection and localisation are presented.

\textcite{Melville1949} proposed a CNN model for the prediction of damage state in thin metal plates to overcome the issue of inaccurate representation of guided wave propagation when applying conventional approaches. 
The model utilizes the full wavefie\-ld scans of thin plates (aluminium).
Moreover, the acquired raw data used for training the model was divided into undamaged and damaged states equally.
The model achieved higher accuracy regarding damage detection equal \(99.98\%\) when compared to SVM which achieved \(62\%\).

\textcite{Sammons2016} proposed a CNN model based on X-ray computed tomography for delamination estimation in a composite structure.
Furthermore, image segmentation was applied to the input images to identify the damage.
However, the model was only able to identify small delaminations.

Moreover,~\textcite{Chetwynd2008} presented a multi-layer perceptron (MLP) network for damage detection in curved composite panels, in which, stiffeners were added to represent the damage.
The Authors in this work investigated the propagation of Lamb waves through the panel in which they were generated and registered by a PZT array.
Furthermore, for each Lamb wave response, a novelty index was obtained.
The index value is compared to some threshold value, in which if the index value exceeds the threshold it implies that there is damage to the structure.
Accordingly, the MLP network was fed by obtained novelty indexes, and performed two operations: classification and regression.
The classification network was designed to define three convex regions of the panel and then to determine whether the panel is damaged or not.
On the other hand, the regression network is capable of estimating the exact location of the damage

Furthermore,~\textcite{DeFenza2015} proposed an artificial neural network (ANN) model for damage detection in plates made of aluminium alloys and composite utilising Lamb waves.
Response data of wave propagation were used to calculate damage indexes which were fed into the model as an input.
Accordingly, the model performs automatic feature extraction in conjunction with the probability ellipse-based method. 
The ANN model and probability ellipse (PE) method were applied to identify the damage location.
The results from the ANN model and the PE presents how it is useful to apply damage indexes as a baseline for such methods to evaluate damage in aluminium and composite structures. 
Ewald et al.~\cite{Ewald2019} presented a CNN model called (DeepSHM) for signal classification using Lamb waves.
Furthermore, the model provides an end-to-end approach for SHM by utilising response signals captured by sensors.
Moreover, response signals were preprocessed by wavelet transform to get the wavelet coefficient matrix (WCM).
Further, the CNN model was trained with the WCM to obtain neural weights.

Full wavefield scanning using SLDV is time-consuming, however, simply reducing the number of scanning points will result in low-quality images. 
\textcite{esfandabadideep} proposed a compressive Sensing technique using ConvNets to enhance the resolution for images captured by SLDV while decreasing the number of measurement scan points down to \(10\%\) of the number of the full gird scanning points. 
Although, the proposed technique enhanced the image resolution, however, there is a side effect, which resembles the fact when enhancing the resolution, the most affected region is the damaged area. 
Accordingly, the damage features will be altered.
On the other hand, this may be an indication of the location of the damage.

Furthermore,~\textcite{Melville2018} proposed a technique for damage detection in thin metal plates (aluminum and steel), using full wavefield data scanned by SLDV. 
Using this data to train a deep neural network of 4 hidden layers including 2 convolutional layers for features extraction and 2 fully connected layers. 
The developed model shown good results when compared with traditional machine learning SVM.
Moreover,~\textcite{Melville2017} introduced a method for detecting damage in structures based on the k-means algorithm. 
The method is known as~\enquote{dictionary learning} which uses full wavefield data collected from thin metal plates. 
The method was applied to structures with different material types and thicknesses that were not used during training to prove how well the model in damage detection in various conditions. 
However, their work was not implemented for a further step, which is damage localization and classification.

%\textcite{Ijjeh2021} presented a fully convolutional network (FCN)  for damage identification in composite plates base on a supervised learning approach.
%Furthermore, the authors utilised a full wavefield of Lamb waves propagation, which was numerically generated resembling measurements acquired by scanning laser Doppler vibrometer (SLDV).
%The model performs a pixel-wise segmentation that is able to identify the delamination which results in damaged and undamaged classes.
%Moreover, the model results were validated through a comparison with a conventional wavefield signal processing method i.e. adaptive wavenumber filtering~\cite{Radzienski2019,Kudela2018}.
%The proposed model achieved an accuracy of \(93.3\%\) in damage detection on numerical data compared to  \(64.8\%\) with the conventional method.
%Furthermore, the proposed model was verified on experimental data and it proved its ability for generalisation.
%%%%%%%%%%%%%%%%%%%%%%%%%%%%%%%%%%%%%%%%%%%%%%%%%%%%%%%%%%%%%%%%%%%%%%%%%%%%%%%%%%%%%%%%%%%%%%%%%%%%%%%%%%%%%%%%%%%%%%%%%%%%%%%%%%%%%%%%%%%%%%%%%%%%%%%%%%%%%%%%
%\subsection{Vibration based SHM though DL}
%\label{sec24}
%The vibration-based approach for damage assessment using ML techniques has been investigated thoroughly  for several SHM applications.
%Furthermore, introducing DL techniques for data-driven SHM applications has presented new scopes for investigating large scale structures and enhanced the process of data acquisition and processing of large datasets acquired by sensors of different types~\cite{Carden2004,Sohn1996}.
%Generally, the conventional approach for damage localisation requires prior knowledge of the approximate damage locations~\cite{Xu2018,Dorafshan2016}. 
%Therefore, the identification process regarding candidates for the damaged locations is complex and can consume plenty of time.
%Damage locations identification under the vibrational approach is based on the fact that the damage cause changes in the vibration characteristics such as modal shapes, frequencies, and damping~\cite{Doebling1998},
%which can be utilised in the identification of damaged locations from the registered data response of a structure.
%A vibration-based approach can be categorised into two classes:
%model-based (parametric) and non-model-based or (non-parametric).
%Parametric methods require computational models and associated assumptions about the investigated structure.
%In general parametric methods can achieve good accuracy, however, there is no guarantee regarding the availability of accurate information about the structural system in the real-world~\cite{Azimi2020}. 
%As a result, the non-parametric methods arise due to the challenges in developing robust computational models. 
%With non-parametric methods, there are no prior assumptions about the structural system.
%
%In the following, several vibration-based for SHM using DL techniques are presented.
%Authors in~\cite{Abdeljaber2017} introduced a damage identification approach based on output-only response data.
%In which, various damage cases (loose bolt) were investigated, accordingly training data were generated based on the acceleration response.
%Authors in this approach have trained several CNNs separately regarding each damage case, and accordingly, the probability of damage (PoD) was determined.
%By investigating scenarios of undamaged, single damage and multiple damage cases, they obtained \(0.54\%\) average error for specifically identified cases.
%
%Authors in~\cite{Lin2017} introduced a new approach to structural damage detection using CNN.
%Moreover, the authors have developed a numerical model of simply supported Euler Bernoulli beam.
%The detection model was designed to learn features and to identify damaged locations, moreover, it led to excellent results regarding the accuracy of damaged locations on the noise-free and noisy dataset.
%Wang and Cha in~\cite{Cha2018} proposed an unsupervised CNN model, that is able to extract the feature representations from the unlabelled data.
%The authors in their model used raw acceleration signals (sensitive to the damage presence) that were acquired from an intact lab-scale steel bridge.
%Then, the acquired response vector was normalised followed by applying the continuous wavelet transform (CWT) and fast Fourier transform (FFT).
%The output was then fed into a CNN auto-encoder,
%Accordingly, the extracted damage features were fed into one-class (OC) SVMs as novelty detectors corresponding to the sensors.
%Consequently, the approximation of damage location (loose-bolt) was estimated based on the locations of the sensors with the highest novelty rates.
%
%Motivated by human vision and thinking, authors in~\cite{Cha2018} presented a computer vision and deep-learning framework for anomaly detection.
%The proposed approach consists of two steps.
%In the first step, data conversion by data visualisation is carried out, in which it mimics human vision and thinking.
%In data visualisation,  the registered data response of acceleration is transformed into images plotted in gray-scale. 
%In the second step, the training dataset is labeled manually, then fed into deep convolutional neural networks (DCNNs).
%The proposed technique was tested on one-year data and achieved a global accuracy of \(87,0\%\) and it could be used for real-time SHM.
%Moreover, Tang et al. in~\cite{Tang2019} presented a DL technique for data anomaly detection which can be considered as an improved technique to the previous work in~\cite{Cha2018}.
%Initially, the raw time series measured data are split into segments, and data in the time and frequency domain are visualised. 
%Images related to each section are stacked as a single dual-channel (red and green).
%Then, the training dataset is fed into a CNN that learns how to perform data anomaly classification.
%The main difference between the previous approach and this approach was in using imbalanced data in which the number of samples of different classes was unequal, however, in this approach the used data were balanced.
%Finally, the comparison shows that this approach outperformed the previous one and achieved higher accuracy for all data anomaly patterns.
%
%Authors in~\cite{Wu2019} presented a study of the deep CNN method in estimating the dynamic response of a linear single-degree-of-freedom (SDOF) system, a nonlinear SDOF, and a multidegree of freedom (MDOF) streel frame.
%In some cases, the convolutional kernel can approximate the numerical integration operator, and the convolutional layer can be interpreted as a dominant frequency extraction operator.
%Moreover, different cases of noise-contaminated signals were investigated. 
%Additionally, MLP method was used as a reference to the proposed CNN approach.
%A comparison between the results obtained by the MLP and CNN shows that the CNN approach is more accurate and robust against noisy input data.
%
%Authors in ~\cite{Oh2019}  presented a study of the CNN technique for SHM application for response estimation of tall buildings under wind excitation.
%The proposed CNN model was trained on measured structural response data which take wind data measured as inputs in order to predict strains in future wind loads.
%In order to measure the performance of the proposed technique, it was verified with unseen data never used at the training phase and it was able to accurately estimate the maximum and minimum strains.
%Authors in~\cite{Li2020} proposed a CNN model for damage detection of a bridge structure.
%Moreover, the authors compared the performance of the CNN model with other techniques such as random forest, SVM, KNN, and decision tree, and the results showed that the accuracy was enhanced by at least \(15\%\).
%
%Since the acceleration response signal is highly prone to noise~\cite{Azimi2020}, researchers begin utilising other types of sensor data or use alternative features.  
%Li et al. in ~\cite{Li2020a} investigated damage in bridge structure accordingly, proposed a supervised learning technique based on the CNN model.
%Dataset was acquired by deflection of a scaled-down model bridge by a fibre-optic gyroscope.
%Then, the dataset was fed into a 1D-CNN model to classify three states of damage and an intact class (benchmark/damage-free).
%To investigate the performance of the proposed model, a cross-validation technique was applied. 
%It showed that the accuracy of the CNN model increased by at least \(15.3\%\) over other conventional methods such as SVM, KNN, decision trees, and random forests.
%Authors in \cite{Lopez-Pacheco2020} introduced a novel frequency-domain convolutional neural network (FDCNN) for damage detection based on Bouc-Wen hysteric model~\cite{Ismail2009}.  
%In the FDCNN method utilises only acceleration measurements for damage diagnosis, that are sensitive to environmental noise.
%Moreover, FDCNN reduces the computational time during the learning process, which increase noise robustness.
%The FDCNN introduced the spectral pooling operator responsible for attenuating the noise in measurements.
%The proposed method was validated through comparing it with different CNN model. 
%The performance of the proposed method was higher regarding damage identification in building structures.
%
%Finally, with smart monitoring as a target, authors in~\cite{Hung2020}  proposed a hybrid deep learning model for damage detection for SHM.
%The proposed model can deal with different damage levels and accurately detect damage by combining 1D-CNN and Long-Short Term Memory (LSTM) into a single end-to-end model fed by the raw time-series, and as a result, avoiding signal preprocessing step.
%Moreover, the proposed model verified that with low noise levels,  accurate damage detection can be achieved.