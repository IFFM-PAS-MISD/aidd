%% SECTION HEADER 
\section{Summary}
\label{sec55}

The results of investigating several end-to-end DL models are presented in this chapter. 
The developed DL models were trained on a large synthetically generated dataset representing the full wavefield of the Lamb waves propagating in a CFRP plate and their interaction with discontinuities such as delamination and the plate edges, acquired by SLDV. 

The first developed DL models were trained on RMS images to perform delamination detection and localisation through bounding boxes.
Hence, two CNN models were developed according to the RMS image resolution of $14\times14$ and $16\times16$ blocks with a size of $32\times32$ pixels, respectively.
The CNN models were evaluated on numerical test cases, proving their ability to detect and localise delamination in previously unseen data.
The IoU metric was applied to evaluate the obtained results.
However, these models were not able to predict delaminations properly for the experimental test cases.

The DL models for pixel-wise image segmentation were also trained on RMS images in a one-to-one scheme to perform delamination identification.
Accordingly, five FCN models were developed: Res-UNet, VGG16 encoder-decoder, FCN-DenseNet, PSPNet, and GCN.
The FCN models were evaluated on numerical test cases demonstrating their ability to identify delamination in previously unseen data. 
Furthermore, the FCN models were evaluated on an experimental test case, in which the models showed their capability to generalise.

The AE ConvLSTM model was trained on a many-to-one scheme, in which a number of full wavefield frames were input to the DL model to perform delamination identification.
To evaluate the performance of the AE ConvLSTM model, I tested it on several numerical test cases, showing its ability to identify the delamination of previously unseen data.
The AE ConvLSTM model was evaluated on several experimental test cases, indicating its capability to generalise well.

The last developed DL model (DLSR) presents a remarkable way of acquiring HR wavefields with fewer measuring points to reduce acquisition time.
The DLSR model was also numerically and experimentally evaluated. 
It demonstrates its ability to recover HR frames from the LR frames with high accuracy, resulting in a significantly faster data acquisition process.



