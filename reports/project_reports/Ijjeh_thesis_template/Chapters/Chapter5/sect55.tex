%% SECTION HEADER 
\section{Summary}
\label{sec55}

In this chapter, I investigated several end-to-end DL models that were trained on a large synthetically generated dataset representing the full wavefield of the Lamp waves propagating in a CFRP plate and their interaction with discontinuities such as the delamination and the plate edges, acquired by SLDV from the bottom surface of the plate. 
The developed models perform: 
\begin{enumerate}
	\item Delamination detection and localisation with surrounding boxes, in which two CNN models were developed.
	\item Delamination identification (pixel-wise image segmentation):
		\begin{itemize}
			\item One-to-one based models, in which, I utilised the RMS images as an input.
			Hence, five DL models have been developed: \\ Res-UNet, VGG16 encoder-decoder, FCN-DenseNet, PSPNet, and GCN.
			\item Many-to-one based model, in which a sequence of full wavefield frames were utilised as an input to the model.
		\end{itemize} 
\end{enumerate}

As mentioned, data acquisition with a high spatial resolution is frequently required for precise damage estimation and necessitates a lengthy scanning time.
This chapter studies a unique way for acquiring high-resolution wavefields with fewer measurement points in order to reduce acquisition time.
To recover high spatial frequency information from low-resolution images, such a technology combines the compressive sensing methodology outlined in the preceding chapter with convolutional neural networks.

