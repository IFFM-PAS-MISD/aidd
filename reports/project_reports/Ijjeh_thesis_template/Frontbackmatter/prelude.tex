% prelude.tex (specification of which features in `mathphdthesis.sty' you
% are using, your personal information, and your title & abstract)

% Specify features of `mathphdthesis.sty' you want to use:
\titlepgtrue 												% Main title page (required)
\signaturepagetrue 											% Page for declaration of originality (required)
\copyrighttrue 												% Copyright page (required)
\abswithesistrue 											% Abstract to be bound with thesis (optional)
\dedicatetrue 											% Dedicaiton

\acktrue 													% Acknowledgments page (optional)
\publicationtrue										    	% Publications  page (optional)
\tablecontentstrue 											% Table of contents page (required)
\tablespagetrue 											% Table of contents page for tables (required only if you have tables)
\figurespagetrue 											% Table of contents page for figures (required only if you have figures)

% Title, author, supervisors, university, date of submission
\title{Feasibility study of artificial intelligence approach for delamination identification in composite laminates}							% Thesis title
\author{Abdalraheem A. Ijjeh} 	% First name and surname of candidate (e.g. John Doe)
\prevdegrees{M.Sc. Eng.}              			% Specify your previous degrees (e.g. B.E. (Hons))
\institute{Mechanics of Intelligent Structures Department}								% Institute of department (e.g. National Centre for Maritime Engineering and Hydrodynamics)

\submittedfor{A dissertation submitted to the Scientific Board of the Szewalski Institute of Fluid-Flow Machinery, Polish Academy of Sciences in partial fulfillment of the requirements for the Degree of Doctor of Philosophy}			% Degree thesis is submitted for (e.g. Submitted in fulfillment of the requirements for the Degree of Doctor of Philosophy)
\advisor{ Pawe\l{} Kudela, D.Sc. Ph.D. Eng.} % Supervisors: (e.g. Prof. Lawrence K. Forbes)
\dept{The Szewalski Institute of Fluid-Flow Machinery, Polish Academy of Sciences}
\date{May, 2022}

% Dedicaton page
\newcommand{\dedication}
{
	\begin{center}
		\emph{To my beloved family.}
	\end{center}
}
% Acknowledgments page
\newcommand{\acknowledgement}
{
I would like to express my deepest sense of gratitude to my supervisor, Prof.~Pawe\l{} Kudela, for his guidance, encouragement, and advice from the initial stage of my Ph.D. studies till the end of helping me develop a thorough understanding of the subject and my studies.
Also, I would like to thank him for passing me his expertise and knowledge of being a scientific researcher.

I would like to express my gratitude to Dr.~Maciej Radzienski for supplying the experimental data of the full wavefield measured by SLDV.

I would like to thank my committee members for letting my defense be a good moment of time, and for their constructive comments and suggestions.

My sincere thanks to my father, Dr.~Abdullah Ijjeh, for his continuous guidance, advice, and encouragement throughout my studies.

I would like to thank my mother for her endless love, trust, encouragement, and support throughout my life.

I would like to thank my wife, Duaa, and my daughter, Raghad, for their consistent support and encouragement.

I would like to thank my brother, Dr.~Abdalraheman Ijjeh, for his consistent advice.

And finally, I would like to thank everyone who supported me during my journey of Ph.D. studies.

}

% Abstract to be bound with thesis
\newcommand{\abstextwithesis}
{
	%Basic abstract goes here. Can use paragraphs and normal \LaTeX commands.
	...
}

% Puplications page
\newcommand{\publications}
{
	\textbf{Journal papers}
	\begin{itemize}
		\item Ijjeh, Abdalraheem A., Saeed Ullah, and Pawel Kudela. \enquote{Full wavefield processing by using FCN for delamination detection.} Mechanical Systems and Signal Processing 153 (2021): 107537.
		\item Ijjeh, Abdalraheem A., and Pawel Kudela. \enqoute{Deep learning based segmentation using full wavefield processing for delamination identification: A comparative study.} Mechanical Systems and Signal Processing 168 (2022): 108671.
		\item  Saeed Ullah, Ijjeh, Abdalraheem A., and Pawel Kudela. \enquote{Deep learning approach for delamination identification using animation of Lamb waves.} Structural Health Monitoring, SAGE (Under review)
		\item Abdalraheem Ijjeh, Saeed Ullah, Maciej Radzienski and Pawel Kudela. \enqoute{Deep learning super-resolution for the reconstruction of full wavefield of Lamb waves}. Mechanical Systems and Signal Processing (Under review).
	\end{itemize}
	\textbf{Conference papers}
	\begin{itemize}
		\item Ijjeh, Abdalraheem A., and Pawel Kudela. \enquote{Feasibility Study of Full Wavefield Processing by Using CNN for Delamination Detection.} Proceedings of the International Conference on Structural Health Monitoring of Intelligent Infrastructure, Porto, Portugal, 30 June - 2 July 2021, ISSN 2564-3738, pages \(709-713\). 
		\item Ijjeh, Abdalraheem A., and Pawel Kudela.  \enqoute{Delamination identification using global convolution networks}, 10th EWSHM European Workshop on Structural Health Monitoring, Palermo, Italy, 4 to 7 July 2022.
	\end{itemize}
	\textbf{Scientific monographs}
	\begin{itemize}
		\item Abdalraheem Ijjeh [100\%], Machine Learning for SHM: Literature Review, chapter in: Wybrane zagadnienia inżynierii mechanicznej, Praca zbiorowa pod redakcją M. Mieloszyk, T. Ochrymiuka, Wydawnictwo Instytutu Maszyn Przepływowych PAN, Gdańsk, 2020, ISBN 978-83-88237-97-3, [80 pt].
		\item Abdalraheem Ijjeh [100\%], Data-driven based approach for damage detection, chapter in: Wybrane zagadnienia inżynierii mechanicznej 2021, Praca zbiorowa pod redakcją M. Mieloszyk, T. Ochrymiuka, Wydawnictwo Instytutu Maszyn Przepływowych PAN, Gdańsk, 2021, ISBN 978-83-66928-00-8 , [80 pt].
		\item Abdalraheem Ijjeh [100\%], Deep Learning based Damage Imaging
		techniques, chapter in: Wybrane zagadnienia inżynierii mechanicznej 2022, Praca zbiorowa pod redakcją M. Mieloszyk, T. Ochrymiuka, Wydawnictwo Instytutu Maszyn Przepływowych PAN, Gdańsk, 2022, ISBN 000-00-00000-00-0 , [80 pt].
	\end{itemize}
	\textbf{Internships}
	\begin{itemize}
		\item Erasmus+ (Wave propagation in phononic crystals, 05/06/2022 – 10/06/2022)	ISEN-JUNIA (CNRS-IEMN) (Lille, France).
%		\begin{itemize}
%			\item Training on the fundamentals of phononic crystals.
%			\item Training on the basics of the Finite Element software "COMSOL Multiphysics"
%			\item Training on the meshing procedure of the periodic unit cells.
%			\item Training on the parametrization of wavenumbers in the eigenvalue problems for the calculation of dispersion diagrams in the first irreducible Brillouin zone.
%			\item Training on the computation of the dispersion diagrams by using "COMSOL Multiphysics"
%		\end{itemize}
		 
	\end{itemize}

}


% Engineering guote page
\newcommand{\engineeringquote}
{
\null\vskip1.8in
\begin{quote}
\begin{flushright}
\end{flushright}
\end{quote}
}

% Bibliography Title
\renewcommand{\bibname}{References/Bibliography}
% Bibliography spacing
\setlength\bibitemsep{1.5\itemsep}

% Settings for array package
\newcolumntype{L}[1]{>{\raggedright\let\newline\\\arraybackslash\hspace{0pt}}m{#1}}
\newcolumntype{C}[1]{>{\centering\let\newline\\\arraybackslash\hspace{0pt}}m{#1}}
\newcolumntype{R}[1]{>{\raggedleft\let\newline\\\arraybackslash\hspace{0pt}}m{#1}}

% Take care of things in `mathphdthesis.sty' behind the scenes.
% Basically just does a check of all the fields that have been activated
% above and fills out the appropriate pages and adds them to the thesis.
\beforepreface
\afterpreface