\documentclass[11pt,a4paper]{report}
\usepackage{fullpage}
\usepackage{hyperref}

\renewcommand{\baselinestretch}{2}
\author{Abdalraheem Abdullah Yousef Ijjeh \\
	Supervised by: \\
	Professor Pawel Kudela}
\title{\Huge Quarterly report \\
	\huge For the period \\ 2021-03-02 to 2021-06-02
	\\ \Large In accordance with Project\\
	Feasibility studies of artificial intelligence-driven diagnostics\\ Institute of Fluid-Flow Machinery, Polish Academy of Sciences
	\\ Gdańsk, Poland}


\begin{document}
	\maketitle
	% \tableofcontents
	\newpage
	\section{Project work flow}
	\begin{itemize}
		\item Several datasets for RNN models were prepared (of full wavefield frames).
		\item Different RNN/LSTM and CNN models (using full wavefield frames) were implemented and tested to estimate the size and location of delamination.
		\item Second chapter (monograph) entitled: "Data-driven based approach for damage
		detection" was submitted and accepted by the TSD of the Polish Academy of Sciences.	
		\item Currently, I am working on implementing models that artificially generate Full wavefield frames of Lamb waves resembling its propagation in CFRP and its interaction with delamination and the boundaries of the specimen.
		Accordingly, we will be able to create a large dataset of propagating Lamb waves interacting with delaminations of various locations, sizes, and angles in a short time compared with the current numerical method.
		Furthermore, this artificially created dataset can be used in training other deep learning models.
		Moreover, the idea can be generalized to further types of damage, or in general to study different phenomena.
	\end{itemize}
%	\bibliography{quarter_report.bib}
%	\bibliographystyle{unsrt}
\end{document}