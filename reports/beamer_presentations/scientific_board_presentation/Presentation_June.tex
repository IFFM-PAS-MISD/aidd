%\PassOptionsToPackage{draft}{graphicx}
%\documentclass[10pt]{beamer} % aspect ratio 4:3, 128 mm by 96 mm
\documentclass[10pt,aspectratio=169]{beamer} % aspect ratio 16:9
%\graphicspath{{../../figures/}}
\graphicspath{{figures/}}
%\includeonlyframes{frame1,frame2,frame3}

%%%%%%%%%%%%%%%%%%%%%%%%%%%%%%%%%%%%%%%%%%%%%%%%%%
% Packages
%%%%%%%%%%%%%%%%%%%%%%%%%%%%%%%%%%%%%%%%%%%%%%%%%%
\usepackage{appendixnumberbeamer}
\usepackage{booktabs}
\usepackage{csvsimple} % for csv read
\usepackage[scale=2]{ccicons}
\usepackage{pgfplots}
\usepackage{xspace}
\usepackage{amsmath}
\usepackage{totcount}
\usepackage{tikz}
\usepackage{bm}
\usepackage{float}
\usepackage{eso-pic} 
\usepackage{wrapfig}
\usepackage{animate,media9,movie15}
\usepackage{subfig}

\usepackage{multimedia}
%\usepackage{FiraSans}

%\usepackage{comment}
%\usetikzlibrary{external} % speedup compilation
%\tikzexternalize % activate!
%\usetikzlibrary{shapes,arrows} 

%\usepackage{bibentry}
%\nobibliography*
\usepackage{ifthen}
\newcounter{angle}
\setcounter{angle}{0}
%\usepackage{bibentry}
%\nobibliography*
\usepackage{caption}%

\captionsetup[figure]{labelformat=empty}%
\graphicspath{{figures/}}
\usefonttheme{structurebold}
%%%%%%%%%%%%%%%%%%%%%%%%%%%%%%%%%%%%%%%%%%%%%%%%%%
% Metropolis theme custom modification file
%%%%%%%%%%%%%%%%%%%%%%%%%%%%%%%%%%%%%%%%%%%%%%%%%%
% Metropolis theme custom modification file
%%%%%%%%%%%%%%%%%%%%%%%%%%%%%%%%%%%%%%%%%%%%%%%%%%
% Metropolis theme custom colors
%%%%%%%%%%%%%%%%%%%%%%%%%%%%%%%%%%%%%%%%%%%%%%%%%%
\usetheme[progressbar=foot]{metropolis}
\useoutertheme{metropolis}
\useinnertheme{metropolis}
\usefonttheme{metropolis}
\setbeamercolor{background canvas}{bg=white}

%\usecolortheme{spruce}

\definecolor{myblue}{rgb}{0.19,0.55,0.91}
\definecolor{mediumblue}{rgb}{0,0,205}
\definecolor{darkblue}{rgb}{0,0,139}
\definecolor{Dodgerblue}{HTML}{1E90FF}
\definecolor{Navy}{HTML}{000080} % {rgb}{0,0,128}
\definecolor{Aliceblue}{HTML}{F0F8FF}
\definecolor{Lightskyblue}{HTML}{87CEFA}
\definecolor{logoblue}{RGB}{1,67,140}
\definecolor{Purple}{HTML}{911146}
\definecolor{Orange}{HTML}{CF4A30}

\setbeamercolor{progress bar}{bg=Lightskyblue}
\setbeamercolor{progress bar}{ fg=logoblue} 
\setbeamercolor{frametitle}{bg=logoblue}
\setbeamercolor{title separator}{fg=logoblue}
\setbeamercolor{block title}{bg=Lightskyblue!30,fg=black}
\setbeamercolor{block body}{bg=Lightskyblue!15,fg=black}
\setbeamercolor{alerted text}{fg=Purple}
% notes colors
\setbeamercolor{note page}{bg=white}
\setbeamercolor{note title}{bg=Lightskyblue}
%%%%%%%%%%%%%%%%%%%%%%%%%%%%%%%%%%%%%%%%%%%%%%%%%%
%  Theme modifications
%%%%%%%%%%%%%%%%%%%%%%%%%%%%%%%%%%%%%%%%%%%%%%%%%%
% modify progress bar linewidth
\makeatletter
\setlength{\metropolis@progressinheadfoot@linewidth}{2pt} 
\setlength{\metropolis@titleseparator@linewidth}{1pt}
\setlength{\metropolis@progressonsectionpage@linewidth}{1pt}

\setbeamertemplate{progress bar in section page}{
	\setlength{\metropolis@progressonsectionpage}{%
		\textwidth * \ratio{\thesection pt}{\totvalue{totalsection} pt}%
	}%
	\begin{tikzpicture}
		\fill[bg] (0,0) rectangle (\textwidth, 
		\metropolis@progressonsectionpage@linewidth);
		\fill[fg] (0,0) rectangle (\metropolis@progressonsectionpage, 
		\metropolis@progressonsectionpage@linewidth);
	\end{tikzpicture}%
}
\makeatother
\newcounter{totalsection}
\regtotcounter{totalsection}

\AtBeginDocument{%
	\pretocmd{\section}{\refstepcounter{totalsection}}{\typeout{Yes, prepending 
	was successful}}{\typeout{No, prepending was not successful}}%
}%
%%%%%%%%%%%%%%%%%%%%%%%%%%%%%%%%%%%%%%%%%%%%%%%%%%
%  Bibliography mods
%%%%%%%%%%%%%%%%%%%%%%%%%%%%%%%%%%%%%%%%%%%%%%%%%%
\setbeamertemplate{bibliography item}{\insertbiblabel} %% Remove book symbol 
%%from references and add number in square brackets
% kill the abominable icon (without number)
%\setbeamertemplate{bibliography item}{}
%\makeatletter
%\renewcommand\@biblabel[1]{#1.} % number only
%\makeatother
% remove line breaks in bibliography
\setbeamertemplate{bibliography entry title}{}
\setbeamertemplate{bibliography entry location}{}
%%%%%%%%%%%%%%%%%%%%%%%%%%%%%%%%%%%%%%%%%%%%%%%%%%
%  Bibliography custom commands
%%%%%%%%%%%%%%%%%%%%%%%%%%%%%%%%%%%%%%%%%%%%%%%%%%
\newcommand{\bibliotitlestyle}[1]{\textbf{#1}\par}

\newif\ifinbiblio
\newcounter{bibkey}
\newenvironment{biblio}[2][long]{%
	%\setbeamertemplate{bibliography item}{\insertbiblabel}
	\setbeamertemplate{bibliography item}{}% without numbers
	\setbeamerfont{bibliography item}{size=\footnotesize}
	\setbeamerfont{bibliography entry author}{size=\footnotesize}
	\setbeamerfont{bibliography entry title}{size=\footnotesize}
	\setbeamerfont{bibliography entry location}{size=\footnotesize}
	\setbeamerfont{bibliography entry note}{size=\footnotesize}
	\ifx!#2!\else%
	\bibliotitlestyle{#2}%
	\fi%
	\begin{thebibliography}{}%
		\inbibliotrue%
		\setbeamertemplate{bibliography entry title}[#1]%
	}{%
		\inbibliofalse%
		\setbeamertemplate{bibliography item}{}%
	\end{thebibliography}%
}

\newcommand{\biblioref}[5][short]{
	\setbeamertemplate{bibliography entry title}[#1]
	\stepcounter{bibkey}%
	\ifinbiblio%
	\bibitem{\thebibkey}%
	#2
	\newblock #4
	\ifx!#5!\else\newblock {\em #5}, #3 \fi%
	\else%
	\begin{biblio}{}
		\bibitem{\thebibkey}
		#2
		\newblock #4
		\ifx!#5!\else\newblock {\em #5}, #3\fi
	\end{biblio}
	\fi
}
%
%\newbibmacro*{hypercite}{%
%	\renewcommand{\@makefntext}[1]{\noindent\normalfont##1}%
%	\footnotetext{%
%		\blxmkbibnote{foot}{%
%			\printtext[labelnumberwidth]{%
%				\printfield{prefixnumber}%
%				\printfield{labelnumber}}%
%			\addspace
%			\fullcite{\thefield{entrykey}}}}}
%
%\DeclareCiteCommand{\hypercite}%
%{\usebibmacro{cite:init}}
%{\usebibmacro{hypercite}}
%{}
%{\usebibmacro{cite:dump}}
%
%% Redefine the \footfullcite command to use the reference number
%\renewcommand{\footfullcite}[1]{\cite{#1}\hypercite{#1}}
%\usefonttheme[onlymath]{Serif} % It should be uncommented if Fira fonts in 
%%math does not work

%%%%%%%%%%%%%%%%%%%%%%%%%%%%%%%%%%%%%%%%%%%%%%%%%%
% Custom commands
%%%%%%%%%%%%%%%%%%%%%%%%%%%%%%%%%%%%%%%%%%%%%%%%%%
% matrix command 
\newcommand{\matr}[1]{\mathbf{#1}} % bold upright (Elsevier, Springer)
%\newcommand{\matr}[1]{#1}   % pure math version
%\newcommand{\matr}[1]{\bm{#1}}  % ISO complying version
% vector command 
\newcommand{\vect}[1]{\mathbf{#1}} % bold upright (Elsevier, Springer)
% bold symbol
\newcommand{\bs}[1]{\boldsymbol{#1}}
% derivative upright command
\DeclareRobustCommand*{\drv}{\mathop{}\!\mathrm{d}}
\newcommand{\ud}{\mathrm{d}}
% 
\newcommand{\themename}{\textbf{\textsc{metropolis}}\xspace}

%%%%%%%%%%%%%%%%%%%%%%%%%%%%%%%%%%%%%%%%%%%%%%%%%%
% Title page options
%%%%%%%%%%%%%%%%%%%%%%%%%%%%%%%%%%%%%%%%%%%%%%%%%%
% \date{\today}
\date{}
%%%%%%%%%%%%%%%%%%%%%%%%%%%%%%%%%%%%%%%%%%%%%%%%%%
% option 1
%%%%%%%%%%%%%%%%%%%%%%%%%%%%%%%%%%%%%%%%%%%%%%%%%%%
\title{FEASIBILITY STUDY OF ARTIFICIAL INTELLIGENCE APPROACH FOR DELAMINATION IDENTIFICATION IN COMPOSITE LAMINATES}
\subtitle{In preparation for a Ph.D. defence}
\author{\textbf{Abdalraheem Ijjeh\\Supervisor: D.Sc. Ph.D. Eng. Paweł Kudela}}
% logo align to Institute 
\institute{Institute of Fluid Flow Machinery \\ 
	Polish Academy of Sciences \\ 
	\vspace{-1.5cm}
	\flushright 
	\includegraphics[width=6cm]{imp_logo.png}}
%%%%%%%%%%%%%%%%%%%%%%%%%%%%%%%%%%%%%%%%%%%%%%%%%%
% option 2 - authors in one line
%%%%%%%%%%%%%%%%%%%%%%%%%%%%%%%%%%%%%%%%%%%%%%%%%%
%	\title{My fancy title}
%	\subtitle{Lamb-opt}
%	\author{\textbf{Paweł Kudela}\textsuperscript{2}, Maciej 
%	Radzieński\textsuperscript{2}, Wiesław Ostachowicz\textsuperscript{2}, 
%	Zhibo Yang\textsuperscript{1} }
%	 logo align to Institute 
%	\institute{\textsuperscript{1}Xi'an Jiaotong University \\ 
%	\textsuperscript{2}Institute of Fluid Flow Machinery\\ \hspace*{1pt} Polish 
%	Academy of Sciences \\ \vspace{-1.5cm}\flushright 
%	
%\includegraphics[width=4cm]{//odroid-sensors/sensors/MISD_shared/logo/logo_eng_40mm.eps}}
%%%%%%%%%%%%%%%%%%%%%%%%%%%%%%%%%%%%%%%%%%%%%%%%%%%
% option 3 - multilogo vertical
%%%%%%%%%%%%%%%%%%%%%%%%%%%%%%%%%%%%%%%%%%%%%%%%%%
%%\title{My fancy title}
%%\subtitle{Lamb-opt}
%%	\author{\textbf{Paweł Kudela}\inst{1}, Maciej Radzieński\inst{1}, Wiesław Ostachowicz\inst{1}, Zhibo Yang\inst{2} }
%%	 logo under Institute 
%%	\institute%
%%	{ 
%%		\inst{1}%
%%		Institute of Fluid Flow Machinery\\ \hspace*{1pt} Polish Academy of Sciences \\ \includegraphics[height=0.85cm]{//odroid-sensors/sensors/MISD_shared/logo/logo_eng_40mm.eps} \\
%%		\and
%%		\inst{2}%
%%	 Xi'an Jiaotong University \\ \includegraphics[height=0.85cm]{//odroid-sensors/sensors/MISD_shared/logo/logo_box.eps}
%% }
% end od option 3
%%%%%%%%%%%%%%%%%%%%%%%%%%%%%%%%%%%%%%%%%%%%%%%%%%
%% option 4 - 3 Institutes and logos horizontal centered
%%%%%%%%%%%%%%%%%%%%%%%%%%%%%%%%%%%%%%%%%%%%%%%%%%
%\title{My fancy title}
%\subtitle{Lamb-opt }
%\author{\textbf{Paweł Kudela}\textsuperscript{1}, Maciej Radzieński\textsuperscript{1}, Marco Miniaci\textsuperscript{2}, Zhibo Yang\textsuperscript{3} }
%
%\institute{ 
%\begin{columns}[T,onlytextwidth]
%	\column{0.39\textwidth}
%	\begin{center}
%		\textsuperscript{1}Institute of Fluid Flow Machinery\\ \hspace*{3pt}Polish Academy of Sciences
%	\end{center}
%	\column{0.3\textwidth}
%	\begin{center}
%		\textsuperscript{2}Zurich University
%	\end{center}
%	\column{0.3\textwidth}
%	\begin{center}
%		\textsuperscript{3}Xi'an Jiaotong University
%	\end{center}
%\end{columns}
%\vspace{6pt}
%% logos 
%\begin{columns}[b,onlytextwidth]
%	\column{0.39\textwidth}
%		\centering 
%		\includegraphics[scale=0.9,height=0.85cm,keepaspectratio]{//odroid-sensors/sensors/MISD_shared/logo/logo_eng_40mm.eps}
%	\column{0.3\textwidth}
%		\centering 
%		\includegraphics[scale=0.9,height=0.85cm,keepaspectratio]{//odroid-sensors/sensors/MISD_shared/logo/logo_box.eps}
%	\column{0.3\textwidth}
%		\centering 
%		\includegraphics[scale=0.9,height=0.85cm,keepaspectratio]{//odroid-sensors/sensors/MISD_shared/logo/logo_box2.eps}
%\end{columns}
%}
%\makeatletter
%\setbeamertemplate{title page}{
%	\begin{minipage}[b][\paperheight]{\textwidth}
%		\centering % <-- Center here
%		\ifx\inserttitlegraphic\@empty\else\usebeamertemplate*{title graphic}\fi
%		\vfill%
%		\ifx\inserttitle\@empty\else\usebeamertemplate*{title}\fi
%		\ifx\insertsubtitle\@empty\else\usebeamertemplate*{subtitle}\fi
%		\usebeamertemplate*{title separator}
%		\ifx\beamer@shortauthor\@empty\else\usebeamertemplate*{author}\fi
%		\ifx\insertdate\@empty\else\usebeamertemplate*{date}\fi
%		\ifx\insertinstitute\@empty\else\usebeamertemplate*{institute}\fi
%		\vfill
%		\vspace*{1mm}
%	\end{minipage}
%}
%
%\setbeamertemplate{title}{
%	% \raggedright% % <-- Comment here
%	\linespread{1.0}%
%	\inserttitle%
%	\par%
%	\vspace*{0.5em}
%}
%\setbeamertemplate{subtitle}{
%	% \raggedright% % <-- Comment here
%	\insertsubtitle%
%	\par%
%	\vspace*{0.5em}
%}
%\makeatother
% end of option 4
%%%%%%%%%%%%%%%%%%%%%%%%%%%%%%%%%%%%%%%%%%%%%%%%%%
% option 5 - 2 Institutes and logos horizontal centered
%%%%%%%%%%%%%%%%%%%%%%%%%%%%%%%%%%%%%%%%%%%%%%%%%%
%\title{My fancy title}
%\subtitle{Lamb-opt }
%\author{\textbf{Paweł Kudela}\textsuperscript{1}, Maciej Radzieński\textsuperscript{1}, Marco Miniaci\textsuperscript{2}}
%
%\institute{ 
%	\begin{columns}[T,onlytextwidth]
%		\column{0.5\textwidth}
%			\centering
%			\textsuperscript{1}Institute of Fluid Flow Machinery\\ \hspace*{3pt}Polish Academy of Sciences
%		\column{0.5\textwidth}
%			\centering
%			\textsuperscript{2}Zurich University
%	\end{columns}
%	\vspace{6pt}
%	% logos 
%	\begin{columns}[b,onlytextwidth]
%		\column{0.5\textwidth}
%		\centering 
%		\includegraphics[scale=0.9,height=0.85cm,keepaspectratio]{//odroid-sensors/sensors/MISD_shared/logo/logo_eng_40mm.eps}
%		\column{0.5\textwidth}
%		\centering 
%		\includegraphics[scale=0.9,height=0.85cm,keepaspectratio]{//odroid-sensors/sensors/MISD_shared/logo/logo_box.eps}
%	\end{columns}
%}
%\makeatletter
%\setbeamertemplate{title page}{
%	\begin{minipage}[b][\paperheight]{\textwidth}
%		\centering % <-- Center here
%		\ifx\inserttitlegraphic\@empty\else\usebeamertemplate*{title graphic}\fi
%		\vfill%
%		\ifx\inserttitle\@empty\else\usebeamertemplate*{title}\fi
%		\ifx\insertsubtitle\@empty\else\usebeamertemplate*{subtitle}\fi
%		\usebeamertemplate*{title separator}
%		\ifx\beamer@shortauthor\@empty\else\usebeamertemplate*{author}\fi
%		\ifx\insertdate\@empty\else\usebeamertemplate*{date}\fi
%		\ifx\insertinstitute\@empty\else\usebeamertemplate*{institute}\fi
%		\vfill
%		\vspace*{1mm}
%	\end{minipage}
%}
%
%\setbeamertemplate{title}{
%	% \raggedright% % <-- Comment here
%	\linespread{1.0}%
%	\inserttitle%
%	\par%
%	\vspace*{0.5em}
%}
%\setbeamertemplate{subtitle}{
%	% \raggedright% % <-- Comment here
%	\insertsubtitle%
%	\par%
%	\vspace*{0.5em}
%}
%\makeatother
% end of option 5
%
%%%%%%%%%%%%%%%%%%%%%%%%%%%%%%%%%%%%%%%%%%%%%%%%%%
% End of title page options
%%%%%%%%%%%%%%%%%%%%%%%%%%%%%%%%%%%%%%%%%%%%%%%%%%
% logo option - alternative manual insertion by modification of coordinates in \put()
%\titlegraphic{%
%	%\vspace{\logoadheight}
%	\begin{picture}(0,0)
%	\put(305,-185){\makebox(0,0)[rb]{\includegraphics[width=4cm]{//odroid-sensors/sensors/MISD_shared/logo/logo_eng_40mm.eps}}}
%	\end{picture}}
%
%%%%%%%%%%%%%%%%%%%%%%%%%%%%%%%%%%%%%%%%%%%%%%%%%%
%\tikzexternalize % activate!
%%%%%%%%%%%%%%%%%%%%%%%%%%%%%%%%%%%%%%%%%%%%%%%%%%
\begin{document}
	

%%%%%%%%%%%%%%%%%%%%%%%%%%%%%%%%%%%%%%%%%%%%%%%%%%
\maketitle
\note{
	Welcome and thank you for joining me here at this remote presentation event. At the beginning I want to introduce my self, my name is Abdalrheem Ijjeh, a PhD student at the Institute of Fluid flow machinery, Polish academy of sciences, supervised by professor Pawel Kudela. Today I am going to briefly present my PhD project entitled by Feasibility studies of artificial intelligence-driven diagnostics}
%%%%%%%%%%%%%%%%%%%%%%%%%%%%%%%%%%%%%%%%%%%%%%%%%%
% SLIDES
%%%%%%%%%%%%%%%%%%%%%%%%%%%%%%%%%%%%%%%%%%%%%%%%%%
\begin{frame}[label=frame1]{Outlines}
	\setbeamertemplate{section in toc}[sections numbered]
	\tableofcontents
\end{frame}
\note{
	The presentation will be as follow: 
	In the first section, I am going to introduce and define composite materials, then I will talk briefly about defects in composite materials, and the conventional approach of damage detection in composite materials. \\
	In the second section, I will briefly introduce Artificial intelligence, machine learning and deep learning approaches.\\
	In the third section, the objectives of the project will be presented. \\
	Then I will talk about the Dataset generation in the fourth section. \\
	Finally, in the last section, I am going to talk about deep learning techniques for delamination identification.}
%%%%%%%%%%%%%%%%%%%%%%%%%%%%%%%%%%%%%%%%%%%%%%%%%%%%%%%%%%%%%%%%%%%%%%%%%%%%%%%
\section{Problem statement}
%%%%%%%%%%%%%%%%%%%%%%%%%%%%%%%%%%%%%%%%%%%%%%%%%%
\begin{frame}{Defects of composite laminates}
	\small
	Composite laminates can have different types of damage such as: \\
	\textbf{Cracks, fibre breakage, debonding, and delamination.} \\ 
	\begin{minipage}[c]{.40\textwidth}
		\begin{itemize}
			\footnotesize
			\item Delamination is a critical failure mechanism in laminated fibre-reinforced polymer matrix composites.
			\item Delamination is one of the most hazardous forms of the defects. 
			It leads to very catastrophic failures if not detected at early stages.
		\end{itemize}
	\end{minipage}
	\hfill
	\begin{minipage}[c]{0.50\textwidth}
		\subfloat{\includegraphics[width=.95\textwidth]{damage_composites.png}}
		\tiny
		(source: https://industrialndt.com/ultrasonic-testing-for-fiber-glass-reinforced-plastic/)
	\end{minipage}
\end{frame}
\note{
	Generally, impact damage in composite materials is caused by various impact events that result from the lack of reinforcement in the out-of-plane direction.
	Causing: 
	Matrix cracks, 
	Fibre breakage, 
	Debonding (occurs when an adhesive stops adhering to an adherend) 
	And delamination, as shown in the figure, which can alter the compression strength of composite laminate and gradually affect the composite to encounter failure by buckling.
	Among these defects in composites, delamination is considered one of the most hazardous types of defects. 
	That is because delamination can seriously decrease the performance of the composite.
	Therefore, delamination detection in the early stages can help to avoid structural collapses.
	}
%%%%%%%%%%%%%%%%%%%%%%%%%%%%%%%%%%%%%%%%%%%%%%%%%%
\begin{frame}{Conventional machine learning approaches}
	Conventional structural damage detection methods involve two processes:
	\begin{itemize}
		\item \textbf{Feature extraction}
		\item \textbf{Feature classification}
	\end{itemize}
	\subfloat{\includegraphics[width=.95\textwidth]{conventional_ML.png}}
		Drawbacks of Conventional methods:
	\begin{itemize}
		\item Requires a great amount of human labor and computational effort.
		\item Demands a high amount of experience of the practitioner.
		\item Inefficient with big data which requires a complex computation of damage features. 
	\end{itemize}
\end{frame}
\note{
	In conventional damage detection methods, to detect the damage, first, we must analyse the response of structure obtained by sensors such as (PZTs) that can sense for example the generated guided waves (e.g. Lamb waves). 
	Then, we need to extract the features of response (which requires large computational effort). 
	Then we can attempt to classify these features, which are sensitive to minor damage and that can be distinguished from the response to natural and environmental changes (baseline).
	}
%%%%%%%%%%%%%%%%%%%%%%%%%%%%%%%%%%%%%%%%%%%%%%%%%%
\note{
	Conventional methods of damage detection focus on patterns
	extraction from registered measurements and accordingly make decisions based on these patterns. 
	Moreover, conventional methods for pattern recognition require feature selection and classification (handcrafted features). 
	These conventional methods can perform efficient damage detection. However, these methods depend on selected features from their scope of measurement. 
	Accordingly, introducing new patterns will cause them to fail in detecting the damage.
	Furthermore, these methods could fail in detecting damage when dealing with big data requiring a complex computation of damage features.
}

%%%%%%%%%%%%%%%%%%%%%%%%%%%%%%%%%%%%%%%%%%%%%%%%%%%%%%%%%%%%%%%%%%%%%%%%%%%%%%%%
\section{Objectives}
\begin{frame}{Objectives}
	\begin{itemize}
		\item \textbf{To develop \textcolor{blue}{a novel AI-driven diagnostic system} for delamination identification in composite laminates such as carbon fibre reinforced polymers (CFRP).
			\item Using an \textcolor{blue}{end-to-end} approach in which DNN processes the animation of propagating waves (input) directly into a damage map (output).}
		\item \textbf{To address the issue of \textcolor{blue}{slow data acquisition} of high-resolution full wavefields of Lamb wave propagation.}
	\end{itemize}
\end{frame}
%%%%%%%%%%%%%%%%%%%%%%%%%%%%%%%%%%%%%%%%%%%%%%%%%%%%%%%%%%%%%%%%%%%%%%%%%%%%%%%

\setcounter{subfigure}{0}
\section{Artificial intelligence, machine learning, and deep learning}
%%%%%%%%%%%%%%%%%%%%%%%%%%%%%%%%%%%%%%%%%%%%%%%%%%
%%%%%%%%%%%%%%%%%%%%%%%%%%%%%%%%%%%%%%%%%%%%%%%%%%
\begin{frame}{What is deep learning?}
	\begin{figure}
		\centering
		\includegraphics[width=0.85\textwidth]{AI_vs_ML_vs_Deep_Learning.png}
	\end{figure}
	\tiny
	(source: https://www.ingeniovirtual.com/)
\end{frame}
%%%%%%%%%%%%%%%%%%%%%%%%%%%%%%%%%%%%%%%%%%%%%%%%%%%%%%%%%%%%%%%%%%%%%%%%%%%%%%%%
\begin{frame}{Why deep learning?}
	\centering
	\textbf{End-to-end approach} 
	\par\medskip
	\subfloat{\includegraphics[width=.95\textwidth]{DL_approach.png}}
\end{frame}
%%%%%%%%%%%%%%%%%%%%%%%%%%%%%%%%%%%%%%%%%%%%%%%%%%%%%%%%%%%%%%%%%%%%%%%%%%%%%%%%
\setcounter{subfigure}{0}
%%%%%%%%%%%%%%%%%%%%%%%%%%%%%%%%%%%%%%%%%%%%%%%%%%
\begin{frame}{Deep learning, why now?}
	\begin{minipage}[c]{0.4\textwidth}
		AI technologies are in accelerating growth due to:
		\begin{itemize}
			\item Exponential development in computer hardware industries
			 (e.g. CPUs, GPUs, FPGAs, TPUs and ASICs)
			\item Era of Big data,
		\end{itemize}
	\end{minipage}
	\begin{minipage}[c]{0.55\textwidth}
		\begin{figure}
			\centering
			\subfloat{\animategraphics[autoplay,loop,width=.9\textwidth]{10}{figures/gif_figs/gpu/gpu_-}{0}{34}}
		\end{figure}
	\tiny
	(source: https://www.techbooky.com/)
	\end{minipage}
	
\end{frame}
%%%%%%%%%%%%%%%%%%%%%%%%%%%%%%%%%%%%%%%%%%%%%%%%%%%%%%%%%%%%%%%%%%%%%%%%%%%%%%%%%%%%%%%%%%%%%%%%%%%%%%%%%%%%%%%%%%%%%%%%%%%%%%%%%%%%%%%%%%%%%%%%%%%%%%%%%%%%%%%%%
\section{Synthetic dataset generation}
%%%%%%%%%%%%%%%%%%%%%%%%%%%%%%%%%%%%%%%%%%%%%%%%%%
\setcounter{subfigure}{0}
%%%%%%%%%%%%%%%%%%%%%%%%%%%%%%%%%%%%%%%%%%%%%%%%%%
\begin{frame}{Common learning strategies}
	\centering
	\begin{figure}
		\includegraphics[width=0.9\textwidth]{learning.png}
	\end{figure}
	\tiny
	(source: https://www.aitude.com/supervised-vs-unsupervised-vs-reinforcement/)
\end{frame}
\setcounter{subfigure}{0}
%%%%%%%%%%%%%%%%%%%%%%%%%%%%%%%%%%%%%%%%%%%%%%%%%%
\subsection{Synthetic Dataset of propagating Lamb waves}
\setcounter{subfigure}{0}
%%%%%%%%%%%%%%%%%%%%%%%%%%%%%%%%%%%%%%%%%%%%%%%%%%
\begin{frame}{Dataset description}
	\centering
	\begin{minipage}[c]{0.35\textwidth}
		\begin{itemize}
			\justifying
			\item 475 cases.
			\item Delamination has a different shape, size and location for each case.
			\item CFRP is made of 8-layers.
			\item Delamination was modelled between the 3rd and 4th layer.
		\end{itemize}
	\end{minipage}
	\begin{minipage}[c]{0.6\textwidth}
		\begin{figure}
			\centering
			\subfloat[Delamination orientation \label{fig:1}]{\includegraphics[width=0.52\textwidth]{figure1.png}}\qquad
			\subfloat[all cases overlapped \label{fig:2}]{\includegraphics[width=0.35\textwidth]{figure_overlap.png}}
		\end{figure}
	\end{minipage}
\end{frame}

\note{
	Dataset generation.
	Numerically generating 475 cases of full wavefield of propagating Lamb waves in a plate made of CFRP as shown in the figure. \\
	Essentially, the output resembles measurements acquired by SLDV in the transverse direction (perpendicular to the plate surface). \\
	Each delamination with different shape, size and location was modelled randomly on the plate.\\ 
	To improve the delamination visibility Root mean square RMS was applied.
	The dataset is used to train the deep learning models in order to identify the delaminations..
}

\setcounter{subfigure}{0}
\begin{frame}{Training Sample case}
	\begin{figure}
		\centering
		\subfloat[Full wavefield $s(x,y,t_k)$ \label{fig:3}]{\animategraphics[autoplay,loop, controls,width=4cm]{16}{figures/gif_figs/7_output/flat_shell_Vz_7_500x500bottom-}{1}{512}}\qquad
		\subfloat[RMS image $\hat{s}(x,y)$ \label{fig:4}]{\includegraphics[width=4cm]{RMS_flat_shell_Vz_7_500x500bottom.png}}\qquad
		\subfloat[Ground truth (label) \label{fig:5}]{\includegraphics[width=4cm]{m1_rand_single_delam_7.png}}
	\end{figure}

The RMS is defined as:
%%%%%%%%%%%%%%%
\begin{equation}
	\hat{s}(x,y) = \sqrt{\frac{1}{N}\sum_{k-1}^{N}s(x,y,t_k)^2}, 
	\label{eqn:rms} 
\end{equation}
\end{frame}
%%%%%%%%%%%%%%%%%%%%%%%%%%%%%%%%%%%%%%%%%%%%%%%%%%%%%%
\section{Deep learning-based approaches}
\setcounter{subfigure}{0}
\begin{frame}{What is computer vision?}
	\begin{minipage}[c]{0.30\textwidth}
		Computer vision is a field of AI that enables computers and systems to derive meaningful information from digital images, videos and other visual inputs. 
	\end{minipage}
	\hfill
	\begin{minipage}[c]{0.65\textwidth}
		\begin{figure}
			\centering
			\includegraphics[width=1\textwidth]{computer_vision_tasks.png}
		\end{figure}
	\end{minipage}
\end{frame}
%%%%%%%%%%%%%%%%%%%%%%%%%%%%%%%%%%%%%%%%%%%%%%%%%%%%%%
\subsection{Semantic segmentation}
\setcounter{subfigure}{0}
\begin{frame}{Image semantic segmentation}
	\begin{minipage}[l]{0.35\textwidth}
		Segmentation schemes can be developed:
		\medskip
		\begin{enumerate}
			\item \textbf{One-to-one \\(RMS based approach)} 
			\medskip
			\item \textbf{Many-to-one \\(Full wavefield frames)}
		\end{enumerate}
	\end{minipage}
	\begin{minipage}[l]{0.6\textwidth}
		\begin{figure}
			\centering
			\subfloat[Single input]{\includegraphics[width=.32\textwidth]{RMS_flat_shell_Vz_381_500x500bottom.png}}\qquad
			\subfloat[Single output]{\includegraphics[width=.32\textwidth]{GCN_381.png}}\qquad
			\\
			\subfloat[Full wavefield frames]{\animategraphics[autoplay,loop,width=.32\textwidth]{4}{figures/gif_figs/381_output/flat_shell_Vz_381_500x500bottom-}{85}{113}}\qquad
			\subfloat[Single output]{\includegraphics[width=.32\textwidth]{GCN_381.png}}
			
		\end{figure}
	\end{minipage}
\end{frame}

\setcounter{subfigure}{0}
\subsection{Developed DL models}
%\begin{frame}{Common deep learning architectures}
%	
%	\begin{minipage}[t]{0.45\textwidth}
%		\textbf{RMS based}\\
%		\begin{itemize}
%			\item Convolutional neural networks (CNN)
%			\item Fully convolutional network (FCN)
%		\end{itemize}
%	\end{minipage}
%	\hfill
%	\begin{minipage}[t]{0.45\textwidth}
%		\textbf{Full wavefield frames}\\
%		\begin{itemize}
%			\item Recurrent neural network (RNN)
%			\item Long short-term memory (LSTM)
%			\item ConvLSTM
%		\end{itemize}
%	\end{minipage}
%\end{frame}
%%%%%%%%%%%%%%%%%%%%%%%%%%%%%%%%%%%%%%%%%%%%%%%%%%
\begin{frame}{Developed model for delamination identification}
	\begin{minipage}[t]{0.45\textwidth}
		\textbf{RMS based models: \\}
			\begin{itemize}
				\item Res-UNet
				\item VGG 16 encoder-decoder
				\item FCN-DenseNet
				\item PSPNet
				\item GCN
			\end{itemize}
		{\tiny 
			\begin{enumerate}
				\item Ijjeh AA, Kudela P. Deep learning based segmentation using full wavefield processing for delamination identification: A comparative study. Mechanical Systems and Signal Processing. 2022 Apr 1;168:108671.
				\item Ijjeh AA, Ullah S, Kudela P. Full wavefield processing by using FCN for delamination detection. Mechanical Systems and Signal Processing. 2021 May 15;153:107537.
				
			\end{enumerate}}
	\end{minipage}
	\hfill
	\begin{minipage}[t]{.45\textwidth}
	\textbf{Full wavefield frames based model:}
		\begin{itemize}
			\item Autoencoder ConvLSTM
		\end{itemize}
	\tiny
	\begin{itemize}
		\item Ullah S, Ijjeh AA, Kudela P. Deep learning approach for delamination identification using animation of Lamb waves. (submitted / under review)
	\end{itemize}
	
	\end{minipage}
\end{frame}
%%%%%%%%%%%%%%%%%%%%%%%%%%%%%%%%%%%%%%%%%%%%%%%%%%
%\setcounter{subfigure}{0}
%\begin{frame}{RMS based models}
%	\begin{minipage}[c]{0.55\textwidth}
%		\begin{figure}
%			\subfloat[Res-UNet model]{\includegraphics[width=1\textwidth]{figure4.png}}
%		\end{figure}
%	\end{minipage}
%	\begin{minipage}[c]{0.35\textwidth}
%		\begin{figure}
%			\subfloat[Data flow \& intermediate outputs of layers \label{fig:}]{\animategraphics[autoplay, controls,width=.8\textwidth]{4}{figures/gif_figs/381__inter_pred/intermediate_output-}{0}{103}}
%\end{figure}
%	\end{minipage}
%
%\end{frame}

\subsection{Numerical test cases}
\setcounter{subfigure}{0}
\begin{frame}{Numerical test cases RMS based models}
	\begin{minipage}[c]{0.32\textwidth}
		\begin{figure}[c]
			\centering
			\animategraphics[controls,width=.9\textwidth]{2}{figures/gif_figs/397/intermediate_output-}{0}{82}
			\caption{\(1^{st}\) numerical case.}
		\end{figure}
	\end{minipage}
	\hfill
	\begin{minipage}[c]{0.32\textwidth}
		\begin{figure}[c]
			\centering
			\animategraphics[controls,width=.9\textwidth]{2}{figures/gif_figs/438/intermediate_output-}{0}{82}
			\caption{\(2^{nd}\) numerical case.}
		\end{figure}
	\end{minipage}
	\hfill
	\begin{minipage}[c]{0.32\textwidth}
		\begin{figure}[c]
			\centering
			\animategraphics[controls,width=.9\textwidth]{2}{figures/gif_figs/456/intermediate_output-}{0}{82}
			\caption{\(3^{rd}\) numerical case.}
		\end{figure}
	\end{minipage}
\end{frame}


%%%%%%%%%%%%%%%%%%%%%%%%%%%%%%%%%%%%%%%%%%%%%%%%%%
\setcounter{subfigure}{0}
\begin{frame}{Numerical test cases animation of Lamb waves}
	\begin{minipage}[t]{.3\textwidth}
	\begin{figure}
		\centering
		\includegraphics[width=.8\textwidth]{figure5b.png}
	\end{figure}
	\end{minipage}
	\begin{minipage}[t]{.65\textwidth}
		\begin{figure}
			\subfloat[\(1^{st}\) case.]{\animategraphics[autoplay,loop,width=.3\textwidth]{6}{figures/gif_figs/397_convLSTM/397_convLSTM-}{0}{23}}
			,
			\subfloat[\(2^{nd}\) case.]{\animategraphics[autoplay,loop,width=.3\textwidth]{6}{figures/gif_figs/438_convLSTM/438_convLSTM-}{0}{23}}
			,
			\subfloat[\(3^{rd}\) case.]{\animategraphics[autoplay,loop,width=.3\textwidth]{6}{figures/gif_figs/456_convLSTM/456_convLSTM-}{0}{23}}
			\\
			\subfloat[Prediction]{\includegraphics[width=.3\textwidth]{figures/predicted_397.png}}
			,
			\subfloat[Prediction]{\includegraphics[width=.3\textwidth]{figures/predicted_438.png}}
			,
			\subfloat[Prediction]{\includegraphics[width=.3\textwidth]{figures/predicted_456.png}}
		\end{figure}	
	\end{minipage}
\end{frame}

\subsection{Experimental case}
\setcounter{subfigure}{0}
\begin{frame}{Experimental setup}
	\begin{minipage}[t]{0.55\textwidth}
		\begin{figure}
			\centering
			\includegraphics[width=.9\textwidth]{wibrometr-laserowy-1d_small-description.png}
		\end{figure}
	\end{minipage}
	\begin{minipage}[t]{0.4\textwidth}
		\begin{enumerate}
			\item Waveform generator
			\item Power amplifier	
			\item Specimen
			\item SLDV head
			\item DAQ
		\end{enumerate}
	\end{minipage}
\end{frame}

\begin{frame}{Experimental results RMS based}
	\centering
	\begin{figure}
		\subfloat[ERMS \& label]{\includegraphics[width=.4\textwidth]{ERMS_with_label.png}}
		\hfill
		\subfloat[IoU\(=0.723\)]{\includegraphics[width=.4\textwidth]{Fig_GCN_7.png}}\qquad
	\end{figure}
\end{frame}

\setcounter{subfigure}{0}
\begin{frame}{Experimental results full wavefield based}
	\begin{minipage}[ct]{0.5\textwidth}
	\begin{figure}
		\centering
		\includegraphics[width=.95\textwidth]{figure3.png}
	\end{figure}
	\end{minipage}
	\hfill
	\begin{minipage}[t]{0.4\textwidth}
		\centering
		\subfloat{\animategraphics[autoplay,loop,width=.40\textwidth]{12}{figures/gif_figs/exp/exp-}{0}{447}} 
		\par\medskip\subfloat{\includegraphics[width=.95\textwidth]{exp_rms_thresholded.png}}\qquad
	\end{minipage}
\end{frame}

\subsection{Super-resolution image reconstruction}
\setcounter{subfigure}{0}
\begin{frame}{Super-resolution image reconstruction}
\textbf{Single image Super-Resolution (SISR) aims to generate a visually pleasing high-resolution image from its de-graded low-resolution measurement.}
\begin{figure}
	\centering
	\subfloat[listentry][LR]{\includegraphics[width=.2\textwidth]{LR_case_1_frame_1.png}}\qquad
	\subfloat[listentry][HR]{\includegraphics[width=.2\textwidth]{SR_case_1_frame_1.png}}\qquad
\end{figure}
\textbf{DLSR model was developed to recover the high-resolution full wavefield
	frames with satisfying accuracy from the low-resolution measurement (below Nyquist sampling rate) acquired by SLDV}
\tiny
\begin{itemize}
	\item Ijjeh, Abdalraheem A., Saeed Ullah, Maciej Radzienski and Pawel Kudela. “Deep learning super-resolution for the reconstruction of full wavefield of Lamb waves.”
	Mechanical Systems and Signal Processing (Under review).
\end{itemize}
\end{frame}

\setcounter{subfigure}{0}
\begin{frame}{Numerical test cases at certain frames}
	\begin{figure}
		\centering
		\subfloat[listentry][HR ref]{\includegraphics[width=.13\textwidth]{output_397_frame_127_full_frame_GT.png}}\qquad
		\subfloat[listentry][SR $f_n$]{\includegraphics[width=.13\textwidth]{output_397_frame_127_full_frame_pred.png}}\qquad
		\subfloat[listentry][Ref]{\includegraphics[width=.13\textwidth]{output_397_frame_127_delamination_GT.png}}\qquad
		\subfloat[listentry][Pred]{\includegraphics[width=.13\textwidth]{output_397_frame_127_delamination_pred.png}}\qquad
		\\
		\subfloat[listentry][HR ref]{\includegraphics[width=.13\textwidth]{output_438_frame_154_full_frame_GT.png}}\qquad
		\subfloat[listentry][SR $f_n$]{\includegraphics[width=.13\textwidth]{output_438_frame_154_full_frame_pred.png}}\qquad
		\subfloat[listentry][Ref]{\includegraphics[width=.13\textwidth]{output_438_frame_154_delamination_GT.png}}\qquad
		\subfloat[listentry][Pred]{\includegraphics[width=.13\textwidth]{output_438_frame_154_delamination_pred.png}}\qquad
		\\				
		\subfloat[listentry][HR ref]{\includegraphics[width=.13\textwidth]{output_456_frame_159_full_frame_GT.png}}\qquad
		\subfloat[listentry][SR $f_n$]{\includegraphics[width=.13\textwidth]{output_456_frame_159_full_frame_pred.png}}\qquad
		\subfloat[listentry][Ref]{\includegraphics[width=.13\textwidth]{output_456_frame_159_delamination_GT.png}}\qquad
		\subfloat[listentry][Pred]{\includegraphics[width=.13\textwidth]{output_456_frame_159_delamination_pred.png}}\qquad
	\end{figure}
\end{frame}

\setcounter{subfigure}{0}
\begin{frame}{Experimental test case at certain frame}
	\begin{figure}
		\centering
		\centering
		\subfloat[listentry][HR ref]{\includegraphics[width=.2\textwidth]{figure10a.png}}\qquad
		\subfloat[listentry][SR $f_n$]{\includegraphics[width=.2\textwidth]{figure10e.png}}\qquad
		\\
		\subfloat[listentry][Ref]{\includegraphics[width=.2\textwidth]{figure11a.png}}\qquad
		\subfloat[listentry][Pred]{\includegraphics[width=.2\textwidth]{figure11e.png}}\qquad
	\end{figure}
\end{frame}
%%%%%%%%%%%%%%%%%%%%%%%%%%%%%%%%%%%%%%%%%%%%%%%%%%
{\setbeamercolor{palette primary}{fg=blue, bg=white}
\begin{frame}[standout]
 Thank you for your listening!\\ \vspace{12pt}
 Questions?\\ \vspace{12pt}
 \url{aijjeh@imp.gda.pl}
\end{frame}
}
\note{Than you for listening, and I am }
%%%%%%%%%%%%%%%%%%%%%%%%%%%%%%%%%%%%%%%%%%%%%%%%%%
% END OF SLIDES
%%%%%%%%%%%%%%%%%%%%%%%%%%%%%%%%%%%%%%%%%%%%%%%%%%
\end{document}