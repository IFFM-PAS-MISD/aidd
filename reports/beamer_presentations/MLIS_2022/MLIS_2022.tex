%\documentclass[10pt]{beamer} % aspect ratio 4:3, 128 mm by 96 mm
%\PassOptionsToPackage{draft}{graphicx}
\documentclass[10pt,aspectratio=169]{beamer} % aspect ratio 16:9
%\graphicspath{{../../figures/}}
\graphicspath{{figs/}}
%\includeonlyframes{frame1,frame2,frame3}

%%%%%%%%%%%%%%%%%%%%%%%%%%%%%%%%%%%%%%%%%%%%%%%%%%
% Packages
%%%%%%%%%%%%%%%%%%%%%%%%%%%%%%%%%%%%%%%%%%%%%%%%%%
\usepackage{appendixnumberbeamer}
\usepackage{booktabs}
\usepackage{csvsimple} % for csv read
\usepackage[scale=2]{ccicons}
\usepackage{pgfplots}
\usepackage{xspace}
\usepackage{amsmath}
\usepackage{totcount}
\usepackage{tikz}
\usepackage{bm}
\usepackage{float}
\usepackage{eso-pic} 
\usepackage{wrapfig}
\usepackage{animate,media9,movie15}
\usepackage{subfig}

\usepackage{multimedia}
\usepackage[draft]{graphicx}
\usepackage{ifthen}
\newcounter{angle}
\setcounter{angle}{0}
\usepackage{caption}

\captionsetup[figure]{labelformat=empty}%
\graphicspath{{figures/}}
\usefonttheme{structurebold}
%%%%%%%%%%%%%%%%%%%%%%%%%%%%%%%%%%%%%%%%%%%%%%%%%%
% Metropolis theme custom modification file
%%%%%%%%%%%%%%%%%%%%%%%%%%%%%%%%%%%%%%%%%%%%%%%%%%
% Metropolis theme custom modification file
%%%%%%%%%%%%%%%%%%%%%%%%%%%%%%%%%%%%%%%%%%%%%%%%%%
% Metropolis theme custom colors
%%%%%%%%%%%%%%%%%%%%%%%%%%%%%%%%%%%%%%%%%%%%%%%%%%
\usetheme[progressbar=foot]{metropolis}
\useoutertheme{metropolis}
\useinnertheme{metropolis}
\usefonttheme{metropolis}
\setbeamercolor{background canvas}{bg=white}

%\usecolortheme{spruce}

\definecolor{myblue}{rgb}{0.19,0.55,0.91}
\definecolor{mediumblue}{rgb}{0,0,205}
\definecolor{darkblue}{rgb}{0,0,139}
\definecolor{Dodgerblue}{HTML}{1E90FF}
\definecolor{Navy}{HTML}{000080} % {rgb}{0,0,128}
\definecolor{Aliceblue}{HTML}{F0F8FF}
\definecolor{Lightskyblue}{HTML}{87CEFA}
\definecolor{logoblue}{RGB}{1,67,140}
\definecolor{Purple}{HTML}{911146}
\definecolor{Orange}{HTML}{CF4A30}

\setbeamercolor{progress bar}{bg=Lightskyblue}
\setbeamercolor{progress bar}{ fg=logoblue} 
\setbeamercolor{frametitle}{bg=logoblue}
\setbeamercolor{title separator}{fg=logoblue}
\setbeamercolor{block title}{bg=Lightskyblue!30,fg=black}
\setbeamercolor{block body}{bg=Lightskyblue!15,fg=black}
\setbeamercolor{alerted text}{fg=Purple}
% notes colors
\setbeamercolor{note page}{bg=white}
\setbeamercolor{note title}{bg=Lightskyblue}
%%%%%%%%%%%%%%%%%%%%%%%%%%%%%%%%%%%%%%%%%%%%%%%%%%
%  Theme modifications
%%%%%%%%%%%%%%%%%%%%%%%%%%%%%%%%%%%%%%%%%%%%%%%%%%
% modify progress bar linewidth
\makeatletter
\setlength{\metropolis@progressinheadfoot@linewidth}{2pt} 
\setlength{\metropolis@titleseparator@linewidth}{1pt}
\setlength{\metropolis@progressonsectionpage@linewidth}{1pt}

\setbeamertemplate{progress bar in section page}{
	\setlength{\metropolis@progressonsectionpage}{%
		\textwidth * \ratio{\thesection pt}{\totvalue{totalsection} pt}%
	}%
	\begin{tikzpicture}
		\fill[bg] (0,0) rectangle (\textwidth, 
		\metropolis@progressonsectionpage@linewidth);
		\fill[fg] (0,0) rectangle (\metropolis@progressonsectionpage, 
		\metropolis@progressonsectionpage@linewidth);
	\end{tikzpicture}%
}
\makeatother
\newcounter{totalsection}
\regtotcounter{totalsection}

\AtBeginDocument{%
	\pretocmd{\section}{\refstepcounter{totalsection}}{\typeout{Yes, prepending 
	was successful}}{\typeout{No, prepending was not successful}}%
}%
%%%%%%%%%%%%%%%%%%%%%%%%%%%%%%%%%%%%%%%%%%%%%%%%%%
%  Bibliography mods
%%%%%%%%%%%%%%%%%%%%%%%%%%%%%%%%%%%%%%%%%%%%%%%%%%
\setbeamertemplate{bibliography item}{\insertbiblabel} %% Remove book symbol 
%%from references and add number in square brackets
% kill the abominable icon (without number)
%\setbeamertemplate{bibliography item}{}
%\makeatletter
%\renewcommand\@biblabel[1]{#1.} % number only
%\makeatother
% remove line breaks in bibliography
\setbeamertemplate{bibliography entry title}{}
\setbeamertemplate{bibliography entry location}{}
%%%%%%%%%%%%%%%%%%%%%%%%%%%%%%%%%%%%%%%%%%%%%%%%%%
%  Bibliography custom commands
%%%%%%%%%%%%%%%%%%%%%%%%%%%%%%%%%%%%%%%%%%%%%%%%%%
\newcommand{\bibliotitlestyle}[1]{\textbf{#1}\par}

\newif\ifinbiblio
\newcounter{bibkey}
\newenvironment{biblio}[2][long]{%
	%\setbeamertemplate{bibliography item}{\insertbiblabel}
	\setbeamertemplate{bibliography item}{}% without numbers
	\setbeamerfont{bibliography item}{size=\footnotesize}
	\setbeamerfont{bibliography entry author}{size=\footnotesize}
	\setbeamerfont{bibliography entry title}{size=\footnotesize}
	\setbeamerfont{bibliography entry location}{size=\footnotesize}
	\setbeamerfont{bibliography entry note}{size=\footnotesize}
	\ifx!#2!\else%
	\bibliotitlestyle{#2}%
	\fi%
	\begin{thebibliography}{}%
		\inbibliotrue%
		\setbeamertemplate{bibliography entry title}[#1]%
	}{%
		\inbibliofalse%
		\setbeamertemplate{bibliography item}{}%
	\end{thebibliography}%
}

\newcommand{\biblioref}[5][short]{
	\setbeamertemplate{bibliography entry title}[#1]
	\stepcounter{bibkey}%
	\ifinbiblio%
	\bibitem{\thebibkey}%
	#2
	\newblock #4
	\ifx!#5!\else\newblock {\em #5}, #3 \fi%
	\else%
	\begin{biblio}{}
		\bibitem{\thebibkey}
		#2
		\newblock #4
		\ifx!#5!\else\newblock {\em #5}, #3\fi
	\end{biblio}
	\fi
}
%
%\newbibmacro*{hypercite}{%
%	\renewcommand{\@makefntext}[1]{\noindent\normalfont##1}%
%	\footnotetext{%
%		\blxmkbibnote{foot}{%
%			\printtext[labelnumberwidth]{%
%				\printfield{prefixnumber}%
%				\printfield{labelnumber}}%
%			\addspace
%			\fullcite{\thefield{entrykey}}}}}
%
%\DeclareCiteCommand{\hypercite}%
%{\usebibmacro{cite:init}}
%{\usebibmacro{hypercite}}
%{}
%{\usebibmacro{cite:dump}}
%
%% Redefine the \footfullcite command to use the reference number
%\renewcommand{\footfullcite}[1]{\cite{#1}\hypercite{#1}}


%%%%%%%%%%%%%%%%%%%%%%%%%%%%%%%%%%%%%%%%%%%%%%%%%%
% Custom commands
%%%%%%%%%%%%%%%%%%%%%%%%%%%%%%%%%%%%%%%%%%%%%%%%%%
% matrix command 
\newcommand{\matr}[1]{\mathbf{#1}} % bold upright (Elsevier, Springer)
%\newcommand{\matr}[1]{#1}          % pure math version
%\newcommand{\matr}[1]{\bm{#1}}     % ISO complying version
% vector command 
\newcommand{\vect}[1]{\mathbf{#1}} % bold upright (Elsevier, Springer)
% bold symbol
\newcommand{\bs}[1]{\boldsymbol{#1}}
% derivative upright command
\DeclareRobustCommand*{\drv}{\mathop{}\!\mathrm{d}}
\newcommand{\ud}{\mathrm{d}}
% 
\newcommand{\themename}{\textbf{\textsc{metropolis}}\xspace}

\definecolor{bleudefrance}{rgb}{0.19, 0.55, 0.91}
%%%%%%%%%%%%%%%%%%%%%%%%%%%%%%%%%%%%%%%%%%%%%%%%%%
%  Title page options
%%%%%%%%%%%%%%%%%%%%%%%%%%%%%%%%%%%%%%%%%%%%%%%%%%
% \date{\today}
\date{}
%%%%%%%%%%%%%%%%%%%%%%%%%%%%%%%%%%%%%%%%%%%%%%%%%%
% option 1
%%%%%%%%%%%%%%%%%%%%%%%%%%%%%%%%%%%%%%%%%%%%%%%%%%%
\title{Convolutional LSTM for delamination imaging in composite laminates}
\subtitle{MLIS 2022, Seoul, South Korea}

\author{\textbf{Prof. Paweł Kudela\\ PhD candidate Abdalraheem Ijjeh }}
% logo align to Institute 
\institute{Institute of Fluid Flow Machinery \\ 
	Polish Academy of Sciences \\ 
	\vspace{-1.5cm}
	\flushright 
	\includegraphics[width=6cm]{imp_logo.png}}

\setbeamercolor{alerted text}{fg=bleudefrance}

\begin{document}
	

%%%%%%%%%%%%%%%%%%%%%%%%%%%%%%%%%%%%%%%%%%%%%%%%%%
\maketitle

%%%%%%%%%%%%%%%%%%%%%%%%%%%%%%%%%%%%%%%%%%%%%%%%%%
% SLIDES
%%%%%%%%%%%%%%%%%%%%%%%%%%%%%%%%%%%%%%%%%%%%%%%%%%
\begin{frame}[label=frame1]{Outlines}
	\setbeamertemplate{section in toc}[sections numbered]
	\tableofcontents
\end{frame}
%%%%%%%%%%%%%%%%%%%%%%%%%%%%%%%%%%%%%%%%%%%%%%%%%%
\section{Composite laminates}
%%%%%%%%%%%%%%%%%%%%%%%%%%%%%%%%%%%%%%%%%%%%%%%%%%
\subsection{Delamination in composite laminates}

\begin{frame}{Defects of composite laminates}
	\small		
	Composite laminates can have different types of damage such as: \\
	\textbf{\uncover<2->{Cracks, }\uncover<3->{fibre breakage,}\uncover<4->{debonding,}\uncover<5->{and \textcolor{blue}{delamination.}}} \\ 
	\begin{minipage}[c]{.40\textwidth}
		\begin{itemize}
			\footnotesize	
			\alt<6->{\item Delamination is a critical failure mechanism in laminated fibre-reinforced polymer matrix composites.}{\item \textcolor{gray}{Delamination is a critical failure mechanism in laminated fibre-reinforced polymer matrix composites.}}
			\onslide<7->
			\item Delamination is one of the most hazardous forms of the defects. 
			It leads to very catastrophic failures if not detected at early stages.
		\end{itemize}
	\end{minipage}
	\hfill
	\begin{minipage}[c]{0.50\textwidth}
		\onslide\subfloat{\includegraphics[width=.95\textwidth]{delaminated_plate1.jpg}}
	\end{minipage}
\end{frame}

%%%%%%%%%%%%%%%%%%%%%%%%%%%%%%%%%%%%%%%%%%%%%%%%%%
\subsection{Damage detection in composite laminates}
%%%%%%%%%%%%%%%%%%%%%%%%%%%%%%%%%%%%%%%%%%%%%%%%%%
\begin{frame}{Conventional approaches}
	Conventional structural damage detection methods involve two processes:
	\begin{itemize}[<alert@+>]
		\item \textbf{Feature extraction}
		\item \textbf{Feature classification}
	\end{itemize}
	\subfloat{\includegraphics[width=.95\textwidth]{conventional_ML.png}}
	Drawbacks of Conventional methods:
	\begin{itemize}[<alert@+>]
		\item Requires a great amount of human labor and computational effort.
		\item Demands a high amount of experience of the practitioner.
		\item Inefficient with big data which requires a complex computation of damage features.
	\end{itemize}
\end{frame}
%%%%%%%%%%%%%%%%%%%%%%%%%%%%%%%%%%%%%%%%%%%%%%%%%%
\setcounter{subfigure}{0}
\begin{frame}{End-to-end approach}
	\centering
	\textbf{End-to-end approach} 
	\par\medskip
	\subfloat{\includegraphics[width=.95\textwidth]{DL_approach.png}}
\end{frame}
%%%%%%%%%%%%%%%%%%%%%%%%%%%%%%%%%%%%%%%%%%%%%%%%%%
%%%%%%%%%%%%%%%%%%%%%%%%%%%%%%%%%%%%%%%%%%%%%%%%%%%%%%%%%%%%%%%%%%%%%%%%%%%%%%%%
%\subsection{Computer vision}
%\setcounter{subfigure}{0}
%\begin{frame}{Computer vision}
%	\begin{minipage}[c]{0.30\textwidth}
%		Computer vision is a field of AI that enables computers and systems to derive meaningful information from digital images, videos and other visual inputs. 
%	\end{minipage}
%	\hfill
%	\begin{minipage}[c]{0.65\textwidth}
%		\begin{figure}
%			\centering
%			\includegraphics[width=1\textwidth]{computer_vision_tasks.png}
%		\end{figure}
%	\end{minipage}
%\end{frame}

%%%%%%%%%%%%%%%%%%%%%%%%%%%%%%%%%%%%%%%%%%%%%%%%%%
\section{Synthetic Dataset generation}
\subsection{Dataset description}
\setcounter{subfigure}{0}
%%%%%%%%%%%%%%%%%%%%%%%%%%%%%%%%%%%%%%%%%%%%%%%%%%
\begin{frame}{Dataset description}
	\centering
	\begin{minipage}[c]{0.35\textwidth}
		\begin{itemize}[<alert@+>]
			\justifying
			\item 475 cases.
			\item Delamination has a different shape, size and location for each case.
			\item CFRP is made of 8-layers.
			\item Delamination was modelled between the 3rd and 4th layer.			
		\end{itemize}
	\end{minipage}
	\hfill
	\begin{minipage}[c]{0.6\textwidth}
		\begin{figure}
			\centering			
			\subfloat[Delamination orientation \label{fig:1}]{\includegraphics[width=0.52\textwidth]{figure1.png}}\qquad
			\subfloat[all cases overlapped \label{fig:2}]{\includegraphics[width=0.35\textwidth]{figure_overlap.png}}
		\end{figure}
	\end{minipage}
\end{frame}

\section{Methodology}
\subsection{Developed DL model}
\setcounter{subfigure}{0}
\begin{frame}{Sample frames of full wave propagation}
	\begin{figure}
		\begin{minipage}[l]{0.3\textwidth}
			\normalsize
			\begin{itemize}[<alert@+>]
				\item The dataset is divided into training set (80\%) and testing sets (20\%).
				\item Each case contains 512 frames.
				\item \textcolor{blue}{Many-to-one} scheme for image segmentation is used.
				\item 24 consecutive frames starting from the initial interaction with the damage are used for training.
			\end{itemize}
		\end{minipage}
		\begin{minipage}[l]{0.65\textwidth}
			\centering	
			\subfloat{\includegraphics[width=.8\textwidth]{figure2.png}} \\	
			\subfloat[Full wavefield (512 frames)]{\animategraphics[autoplay,loop,width=2cm]{32}{figures/gif_figs/7_output/flat_shell_Vz_7_500x500bottom-}{1}{512}}\qquad
			\subfloat[Training sample of 24 frames]{\animategraphics[autoplay,loop,width=2cm]{16}{figures/gif_figs/7_output/flat_shell_Vz_7_500x500bottom-}{86}{109}}\qquad
			\subfloat[Ground truth (label) \label{fig:5}]{\includegraphics[width=2cm]{m1_rand_single_delam_7.png}}
		\end{minipage}
	\end{figure}
\end{frame}
\setcounter{subfigure}{0}
\begin{frame}{DL model procedure}
	\begin{minipage}[l]{0.35\textwidth}
		The complete procedure of obtaining intermediate predictions for
		the test cases and finally calculating the RMS image (Damage map). \\
		\begin{equation*}
			RMS = \sqrt{\frac{1}{N}\sum_{k=1}^{N}\hat{Y_k}^2}.	
			\label{RMS}
		\end{equation*}
	\end{minipage}
	\begin{minipage}[l]{0.6\textwidth}
		\begin{figure}
			\centering
			\subfloat{\includegraphics[width=.95\textwidth]{figure3.png}}\qquad
		\end{figure}
	\end{minipage}	
\end{frame}
\setcounterpageref{subfigure}{0}
\begin{frame}{LSTM and ConvLSTM units}
	\begin{minipage}[l]{0.55\textwidth}
		\begin{itemize}[<alert@+>]
			\item Long short-term memory unit cell (LSTM) can keep information related to long-term dependencies. 
			\item LSTM is inefficient at capturing spatial information when the time series inputs are consecutive images.
			\item \textcolor{blue}{ConvLSTM} is a combination of a convolution operation and an LSTM cell that was introduced to solve such a problem.. 
			\item \textcolor{blue}{ConvLSTM} can capture the time-correlated and spatial features in a series of consecutive images.
		\end{itemize}
	\end{minipage}
	\begin{minipage}[l]{0.4\textwidth}
		\begin{figure}
			\centering
			\subfloat[LSTM]{\includegraphics[scale=0.5]{figure4a.png}} \\
			\subfloat[ConvLSTM]{\includegraphics[scale=0.7]{figure4b.png}}
		\end{figure}
	\end{minipage}
\end{frame}

%%%%%%%%%%%%%%%%%%%%%%%%%%%%%%%%%%%%%%%%%%%%%%%%%%
\setcounter{subfigure}{0}
\begin{frame}{Architecture of the developed ConvLSTM model}
	\begin{minipage}[l]{0.6\textwidth}
		Autoencoder technique (AE) is well-known for extracting spatial features.
		AE consists of three parts: 
		\begin{itemize}
			\item \textcolor{blue}{(Encoder)}: learns how to reduce the input dimensions and compress the input data into an encoded representation.
			\item \textcolor{blue}{(Bottleneck)}: is the lowest level of dimensions
			of the input data (latent space).
			\item \textcolor{blue}{(Decoder)}: learns how to restore the original dimensions of the input.		
		\end{itemize}
	 \textcolor{blue}{Time distributed layer}
	 is introduced to the model to distribute the input frames into the AE layers in order to process them independently.
 	\end{minipage}
	\begin{minipage}[l]{0.35\textwidth}
		\begin{figure}
			\subfloat{\includegraphics[scale=1.1]{figure5b.png}}
		\end{figure}
	\end{minipage}
\end{frame}

\setcounter{subfigure}{0}
\begin{frame}{Evaluation metrics for delamination identification}
	\begin{minipage}[c]{0.45\textwidth}
		For evaluating delamination identification
		\begin{itemize}
			\item Intersection over Union (IoU): %%%%%%%%%%%%%%%%%%%%%%%%%%%%%%%%%%%%%%%%%%%%%%%%%%%%%%%%%%%%%%%%%%%%%%%%
			\begin{equation*}
				\textup{IoU}=\frac{Intersection}{Union}=\frac{\hat{Y} \cap Y}{\hat{Y} \cup Y},
				\label{eqn:iou}
			\end{equation*}
			%%%%%%%%%%%%%%%%%%%%%%%%%%%%%%%%%%%%%%%%%%%%%%%%%%%%%%%%%%%%%%%%%%%%
		\end{itemize}
	\end{minipage}
	\fill
	\begin{minipage}[c]{0.45\textwidth}
		\begin{figure}
			\subfloat{\includegraphics[width=1\textwidth]{IoU_figure.png}}
		\end{figure}
	\end{minipage}
\end{frame}


\section{Numerical test cases}
\setcounter{subfigure}{0}
\begin{frame}{First test case}
	\begin{figure}
		\centering
		\subfloat[Full wavefield (512 frames)]{\animategraphics[autoplay,loop,height=4cm,keepaspectratio]{32}{figures/gif_figs/381_output/output_381-}{1}{512}}\quad
		\subfloat[RMS of all intermediate predictions]{\includegraphics[height=4cm,keepaspectratio]{figures/RMS_Ijjeh_num_case_381.png}}\quad
		\subfloat[Binary RMS, IoU= 0.88]{\includegraphics[height=4cm,keepaspectratio]{figures/Binary_RMS_Ijjeh_num_case381_.png}}\quad
		\end{figure}	
\end{frame}

\setcounter{subfigure}{0}
\begin{frame}{Second test case}
	\begin{figure}
		\centering
			\subfloat[Full wavefield (512 frames)]{\animategraphics[autoplay,loop,height=4cm,keepaspectratio]{32}{figures/gif_figs/385_output/output_385-}{1}{512}}\quad
		\subfloat[RMS of all intermediate predictions]{\includegraphics[height=4cm,keepaspectratio]{figures/RMS_Ijjeh_num_case_385.png}}\quad
		\subfloat[Binary RMS, IoU= 0.58]{\includegraphics[height=4cm,keepaspectratio]{figures/Binary_RMS_Ijjeh_num_case385_.png}}
	\end{figure}
\end{frame}

\setcounter{subfigure}{0}
\begin{frame}{Third test case}
	\begin{figure}
		\centering
		\subfloat[Full wavefield (512 frames)]{\animategraphics[autoplay,loop,height=4cm,keepaspectratio]{32}{figures/gif_figs/394_output/output_394-}{1}{512}}\quad
		\subfloat[RMS of all intermediate predictions]{\includegraphics[height=4cm,keepaspectratio]{figures/RMS_Ijjeh_num_case_394.png}}\quad
		\subfloat[Binary RMS, IoU= 0.8]{\includegraphics[height=4cm,keepaspectratio]{figures/Binary_RMS_Ijjeh_num_case394_.png}}
	\end{figure}
\end{frame}

\section{Experimental test case}
\setcounter{subfigure}{0}
\begin{frame}{Experimental setup}
	\begin{minipage}[t]{0.55\textwidth}
		\begin{figure}
			\centering
			\includegraphics[width=.9\textwidth]{wibrometr-laserowy-1d_small-description.png}
		\end{figure}
	\end{minipage}
	\begin{minipage}[t]{0.4\textwidth}
		\begin{enumerate}[<alert@+>]
			\item Waveform generator
			\item Power amplifier	
			\item Specimen
			\item SLDV head
			\item DAQ
		\end{enumerate}
	\end{minipage}
\end{frame}

\setcounter{subfigure}{0}
\begin{frame}{Delamination arrangement}
	\begin{minipage}[c]{0.4\textwidth}
		\begin{itemize}[<alert@+>]
			\item Teflon inserts with a thickness of \(250\ \mu\)m were used to simulate the delaminations.
			\item The average thickness of the specimen was \(3.9 \pm 0.1\) mm.
			\item The delaminations were located at the same distance, equal to \(150\) mm from the centre of the plate.
		\end{itemize}
	\end{minipage}
	\begin{minipage}[c]{0.55\textwidth}
		\centering
		\includegraphics[width=.7\textwidth]{figures/figure11.png}
	\end{minipage}
\end{frame}

\setcounter{subfigure}{0}
\begin{frame}{Experimental case: Specimen~III}
		\begin{figure}
		\centering
		\subfloat[Full wavefield (512 frames)]{\animategraphics[autoplay,loop,height=4cm]{32}{figures/gif_figs/input_specimen_3/specimen_3-}{1}{512}}\quad
%		\subfloat[Intermidate ouputs]{\animategraphics[autoplay,loop,height=3cm]{24}{figures/gif_figs/Intermediate_specimen_3/Intermediate_specimen_3-}{0}{487}}\quad
		\subfloat[RMS of all intermediate predictions]{\includegraphics[height=4cm,keepaspectratio]{figures/RMS_L3_S3_B_333x333p_50kHz_5HC_18Vpp_x10_pzt_Ijjeh_updated_results_.png}}\quad
		\subfloat[Binary RMS, IoU= 0.64]{\includegraphics[height=4cm,keepaspectratio]{figures/Binary_RMS_L3_S3_B__333x333p_50kHz_5HC_18Vpp_x10_pzt_Ijjeh_.png}}
	\end{figure}
\end{frame}


\section{Conclusions}
\begin{frame}{Conclusions}
	\begin{itemize}[<alert@+>]
		\item Full wavefields of elastic waves propagating in a composite laminate contain extensive, valuable, and complex information regarding the discontinuities in the plate, such as delamination or edges.
		\item Such information can be utilised to train deep learning models
		to perform damage identification in an {\textbf{End-to-End}} approach.
		\item With deep learning based approaches, it is possible to use registered data in its raw form without the need to perform feature engineering, extraction, and classification.
		\item The developed DL model requires more time during training, it can still sustain rapid and faster testing compared to conventional signal processing and machine learning methods.
		\item The developed model shows its ability to generalise by detecting the delamination in the unseen test cases (numerical and experimental).
	\end{itemize}
\end{frame}

%%%%%%%%%%%%%%%%%%%%%%%%%%%%%%%%%%%%%%%%%%%%%%%%%%
{\setbeamercolor{palette primary}{fg=blue, bg=white}
\begin{frame}[standout]
	\textbf{Acknowledgment} \\ 
	The research work was funded by the Polish National Science Center under grant agreement no. 2018/31/B/ST8/00454.
\end{frame}

\begin{frame}[standout]
  Thank you for your listening!\\ \vspace{12pt}
  Questions?\\ \vspace{12pt}
  \url{pk@imp.gda.pl}\\
  \url{aijjeh@imp.gda.pl}
\end{frame}
}
\note{Thank you for listening}
%%%%%%%%%%%%%%%%%%%%%%%%%%%%%%%%%%%%%%%%%%%%%%%%%%
% END OF SLIDES
%%%%%%%%%%%%%%%%%%%%%%%%%%%%%%%%%%%%%%%%%%%%%%%%%%
\end{document}