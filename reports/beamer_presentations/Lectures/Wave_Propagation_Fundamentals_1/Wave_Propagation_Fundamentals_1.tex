%\documentclass[10pt]{beamer} % aspect ratio 4:3, 128 mm by 96 mm
%\documentclass[10pt,aspectratio=169]{beamer} % aspect ratio 16:9, only frames
\documentclass[10pt,aspectratio=169,notes]{beamer} % aspect ratio 16:9, frames+notes 
%\documentclass[10pt,aspectratio=169,notes=only]{beamer} % aspect ratio 16:9, notes only 
\usepackage{pgfpages}
%\setbeameroption{show notes}
%\setbeameroption{show notes on second screen=right}
%\setbeameroption{show notes on second screen=bottom} % does not work for animations

%\graphicspath{{../../figures/}}
\graphicspath{{figs/}{figs/longitudinal-wave-png/}}
%\includeonlyframes{frame1,frame2,frame3}
%\includeonlyframes{frame10}
%%%%%%%%%%%%%%%%%%%%%%%%%%%%%%%%%%%%%%%%%%%%%%%%%%
% Packages
%%%%%%%%%%%%%%%%%%%%%%%%%%%%%%%%%%%%%%%%%%%%%%%%%%
\usepackage{appendixnumberbeamer}
\usepackage{booktabs}
\usepackage{csvsimple} % for csv read
\usepackage[scale=2]{ccicons}
\usepackage{pgfplots}
\usepackage{xspace}
\usepackage{amsmath}
\usepackage{totcount}
\usepackage{tikz}
\usepackage{bm}
%\usepackage{FiraSans}
\usepackage{mathrsfs} % for Fourier and Laplace symbols
\usepackage{verbatim}
%\usepackage{eulervm} % alternative math fonts
%\usepackage{comment}
\usetikzlibrary{external} % speedup compilation
%\tikzexternalize % activate!
%\usetikzlibrary{shapes,arrows}  
% the animations are only supported by some pdf readers (AcrobatReader, PDF-XChange, acroread, and Foxit Reader)
%\usepackage{animate}
\usepackage[draft]{animate}
\usepackage{ifthen}
\newcounter{angle}
\setcounter{angle}{0}
%\usepackage{bibentry}
%\nobibliography*
\usepackage{caption}%
\captionsetup[figure]{labelformat=empty}%
%%%%%%%%%%%%%%%%%%%%%%%%%%%%%%%%%%%%%%%%%%%%%%%%%%
% Metropolis theme custom modification file
%%%%%%%%%%%%%%%%%%%%%%%%%%%%%%%%%%%%%%%%%%%%%%%%%%
% Metropolis theme custom modification file
%%%%%%%%%%%%%%%%%%%%%%%%%%%%%%%%%%%%%%%%%%%%%%%%%%
% Metropolis theme custom colors
%%%%%%%%%%%%%%%%%%%%%%%%%%%%%%%%%%%%%%%%%%%%%%%%%%
\usetheme[progressbar=foot]{metropolis}
\useoutertheme{metropolis}
\useinnertheme{metropolis}
\usefonttheme{metropolis}
\setbeamercolor{background canvas}{bg=white}

%\usecolortheme{spruce}

\definecolor{myblue}{rgb}{0.19,0.55,0.91}
\definecolor{mediumblue}{rgb}{0,0,205}
\definecolor{darkblue}{rgb}{0,0,139}
\definecolor{Dodgerblue}{HTML}{1E90FF}
\definecolor{Navy}{HTML}{000080} % {rgb}{0,0,128}
\definecolor{Aliceblue}{HTML}{F0F8FF}
\definecolor{Lightskyblue}{HTML}{87CEFA}
\definecolor{logoblue}{RGB}{1,67,140}
\definecolor{Purple}{HTML}{911146}
\definecolor{Orange}{HTML}{CF4A30}

\setbeamercolor{progress bar}{bg=Lightskyblue}
\setbeamercolor{progress bar}{ fg=logoblue} 
\setbeamercolor{frametitle}{bg=logoblue}
\setbeamercolor{title separator}{fg=logoblue}
\setbeamercolor{block title}{bg=Lightskyblue!30,fg=black}
\setbeamercolor{block body}{bg=Lightskyblue!15,fg=black}
\setbeamercolor{alerted text}{fg=Purple}
% notes colors
\setbeamercolor{note page}{bg=white}
\setbeamercolor{note title}{bg=Lightskyblue}
%%%%%%%%%%%%%%%%%%%%%%%%%%%%%%%%%%%%%%%%%%%%%%%%%%
%  Theme modifications
%%%%%%%%%%%%%%%%%%%%%%%%%%%%%%%%%%%%%%%%%%%%%%%%%%
% modify progress bar linewidth
\makeatletter
\setlength{\metropolis@progressinheadfoot@linewidth}{2pt} 
\setlength{\metropolis@titleseparator@linewidth}{1pt}
\setlength{\metropolis@progressonsectionpage@linewidth}{1pt}

\setbeamertemplate{progress bar in section page}{
	\setlength{\metropolis@progressonsectionpage}{%
		\textwidth * \ratio{\thesection pt}{\totvalue{totalsection} pt}%
	}%
	\begin{tikzpicture}
		\fill[bg] (0,0) rectangle (\textwidth, 
		\metropolis@progressonsectionpage@linewidth);
		\fill[fg] (0,0) rectangle (\metropolis@progressonsectionpage, 
		\metropolis@progressonsectionpage@linewidth);
	\end{tikzpicture}%
}
\makeatother
\newcounter{totalsection}
\regtotcounter{totalsection}

\AtBeginDocument{%
	\pretocmd{\section}{\refstepcounter{totalsection}}{\typeout{Yes, prepending 
	was successful}}{\typeout{No, prepending was not successful}}%
}%
%%%%%%%%%%%%%%%%%%%%%%%%%%%%%%%%%%%%%%%%%%%%%%%%%%
%  Bibliography mods
%%%%%%%%%%%%%%%%%%%%%%%%%%%%%%%%%%%%%%%%%%%%%%%%%%
\setbeamertemplate{bibliography item}{\insertbiblabel} %% Remove book symbol 
%%from references and add number in square brackets
% kill the abominable icon (without number)
%\setbeamertemplate{bibliography item}{}
%\makeatletter
%\renewcommand\@biblabel[1]{#1.} % number only
%\makeatother
% remove line breaks in bibliography
\setbeamertemplate{bibliography entry title}{}
\setbeamertemplate{bibliography entry location}{}
%%%%%%%%%%%%%%%%%%%%%%%%%%%%%%%%%%%%%%%%%%%%%%%%%%
%  Bibliography custom commands
%%%%%%%%%%%%%%%%%%%%%%%%%%%%%%%%%%%%%%%%%%%%%%%%%%
\newcommand{\bibliotitlestyle}[1]{\textbf{#1}\par}

\newif\ifinbiblio
\newcounter{bibkey}
\newenvironment{biblio}[2][long]{%
	%\setbeamertemplate{bibliography item}{\insertbiblabel}
	\setbeamertemplate{bibliography item}{}% without numbers
	\setbeamerfont{bibliography item}{size=\footnotesize}
	\setbeamerfont{bibliography entry author}{size=\footnotesize}
	\setbeamerfont{bibliography entry title}{size=\footnotesize}
	\setbeamerfont{bibliography entry location}{size=\footnotesize}
	\setbeamerfont{bibliography entry note}{size=\footnotesize}
	\ifx!#2!\else%
	\bibliotitlestyle{#2}%
	\fi%
	\begin{thebibliography}{}%
		\inbibliotrue%
		\setbeamertemplate{bibliography entry title}[#1]%
	}{%
		\inbibliofalse%
		\setbeamertemplate{bibliography item}{}%
	\end{thebibliography}%
}

\newcommand{\biblioref}[5][short]{
	\setbeamertemplate{bibliography entry title}[#1]
	\stepcounter{bibkey}%
	\ifinbiblio%
	\bibitem{\thebibkey}%
	#2
	\newblock #4
	\ifx!#5!\else\newblock {\em #5}, #3 \fi%
	\else%
	\begin{biblio}{}
		\bibitem{\thebibkey}
		#2
		\newblock #4
		\ifx!#5!\else\newblock {\em #5}, #3\fi
	\end{biblio}
	\fi
}
%
%\newbibmacro*{hypercite}{%
%	\renewcommand{\@makefntext}[1]{\noindent\normalfont##1}%
%	\footnotetext{%
%		\blxmkbibnote{foot}{%
%			\printtext[labelnumberwidth]{%
%				\printfield{prefixnumber}%
%				\printfield{labelnumber}}%
%			\addspace
%			\fullcite{\thefield{entrykey}}}}}
%
%\DeclareCiteCommand{\hypercite}%
%{\usebibmacro{cite:init}}
%{\usebibmacro{hypercite}}
%{}
%{\usebibmacro{cite:dump}}
%
%% Redefine the \footfullcite command to use the reference number
%\renewcommand{\footfullcite}[1]{\cite{#1}\hypercite{#1}}
%\usefonttheme[onlymath]{Serif} 

%%%%%%%%%%%%%%%%%%%%%%%%%%%%%%%%%%%%%%%%%%%%%%%%%%
% Custom commands
%%%%%%%%%%%%%%%%%%%%%%%%%%%%%%%%%%%%%%%%%%%%%%%%%%
% matrix command 
%\newcommand{\matr}[1]{\mathbf{#1}} % bold upright (Elsevier, Springer)
%  metropolis compatible (FiraSans auto replacement)
\newcommand{\matr}[1]{\boldsymbol{#1}}
%\newcommand{\matr}[1]{#1}          % pure math version
%\newcommand{\matr}[1]{\bm{#1}}     % ISO complying version
% vector command 
%\newcommand{\vect}[1]{\mathbf{#1}} % bold upright (Elsevier, Springer)
% metropolis compatible (FiraSans auto replacement)
\newcommand{\vect}[1]{\boldsymbol{#1}}
% bold symbol
\newcommand{\bs}[1]{\boldsymbol{#1}}
% derivative upright command
\DeclareRobustCommand*{\drv}{\mathop{}\!\mathrm{d}}
\newcommand{\ud}{\mathrm{d}}
\newcommand{\myexp}{\mathrm{e}}
% 
\newcommand{\themename}{\textbf{\textsc{metropolis}}\xspace}
\renewcommand{\Re}{\operatorname{\mathbb{R}e}}
\renewcommand{\Im}{\operatorname{\mathbb{I}m}}
%%%%%%%%%%%%%%%%%%%%%%%%%%%%%%%%%%%%%%%%%%%%%%%%%%
%  Title page options
%%%%%%%%%%%%%%%%%%%%%%%%%%%%%%%%%%%%%%%%%%%%%%%%%%
% \date{\today}
\date{}
%%%%%%%%%%%%%%%%%%%%%%%%%%%%%%%%%%%%%%%%%%%%%%%%%%
% option 1
%%%%%%%%%%%%%%%%%%%%%%%%%%%%%%%%%%%%%%%%%%%%%%%%%%
\title{Fundamentals of elastic wave propagation phenomenon}
\subtitle{Lecture Series}
\author{\textbf{Paweł Kudela} }
% logo align to Institute 
\institute{Institute of Fluid Flow Machinery\\Polish Academy of Sciences \\ \vspace{-1.5cm}\flushright \includegraphics[width=4cm]{//odroid-sensors/sensors/MISD_shared/logo/logo_eng_40mm.eps}}

%%%%%%%%%%%%%%%%%%%%%%%%%%%%%%%%%%%%%%%%%%%%%%%%%%
%\tikzexternalize % activate!
%%%%%%%%%%%%%%%%%%%%%%%%%%%%%%%%%%%%%%%%%%%%%%%%%%
\begin{document}
%%%%%%%%%%%%%%%%%%%%%%%%%%%%%%%%%%%%%%%%%%%%%%%%%%
\maketitle
%%%%%%%%%%%%%%%%%%%%
\note{Welcome. My name is Pawel Kudela. 
I am an associate professor at the Institute of Fluid Machinery, Polish Academy of Sciences.
Today I will be talking about waves with the focus on elastic wave propagation phenomenon. I will talk about its usefulness in practical applications.
}
%%%%%%%%%%%%%%%%%%%%%%%%%%%%%%%%%%%%%%%%%%%%%%%%%%
% SLIDES
%%%%%%%%%%%%%%%%%%%%%%%%%%%%%%%%%%%%%%%%%%%%%%%%%%
\begin{frame}[allowframebreaks]{Table of contents}
  \setbeamertemplate{section in toc}[sections numbered]
  \tableofcontents[hideallsubsections]
\end{frame}
%%%%%%%%%%%%%%%%%%%%
\note{}
%%%%%%%%%%%%%%%%%%%%%%%%%%%%%%%%%%%%%%%%%%%%%%%%%%
\section{Types of wave}
%%%%%%%%%%%%%%%%%%%%%%%%%%%%%%%%%%%%%%%%%%%%%%%%%%
\begin{frame}[label=frame2]{Types of wave}
%%%%%%%%%%%%%%%%%%%%%%%%%%%%%%%%%%%%%%%%%%%%%%
\begin{equation*}
\frac{\partial^2 \Psi(x,t)}{\partial t^2} = \upsilon_p^2 \, \frac{\partial^2 \Psi(x,t)}{\partial x^2}
\end{equation*}
\begin{figure}
	\includegraphics[width=0.9\textwidth]{wave_types.png}
\end{figure}
\end{frame}
%%%%%%%%%%%%%%%%%%%%
\note{
This is the simples form of the wave equation. It is a second partial derivatives of physical quantity in respect to time equal to coefficient \(\upsilon_p\) square times a second partial derivative in respect to space. The coefficient  \(\upsilon_p\)  usually refers to wave velocity.
This equation can describe the problem of taut string, sound waves, electromechanical waves and longitudinal waves propagating in a rod.
Of course psi has different quantity for each penomenon; for vibrating taut string it will be the displacement, for sound waves it will be the pressure. 
Electromagnetic waves are created by a fusion of electric and magnetic fields.
Whereas propagating waves in a rod will be considered in terms of longitudinal displacements.
It is interesting to note that electromagnetic waves do not need a medium to travel.
It is also important to mention that propagating waves transmits energy from one point to another but without transporting a mass. 
Propagating waves can also be used for sending information.
 }
%%%%%%%%%%%%%%%%%%%%%%%%%%%%%%%%%%%%%%%%%%%%%%%%%%
\begin{frame}{Types of wave}
%%%%%%%%%%%%%%%%%%%%%%%%%%%%%%%%%%%%%%%%%%%%%%%%%%
	Various classifications of elastic waves can be used. 
	Based on oscillation form:
	\begin{itemize}
		\item standing waves
		\item propagating (progressing) waves
	\end{itemize}
	Based on analysed physical quantity:
	\begin{itemize}
		\item electromagnetic waves (Microwaves, X-ray, radio waves, ultraviolet waves)
		\item mechanical waves (water waves, ultrasonic waves, elastic waves, guided waves)
		\item matter waves
	\end{itemize}
\end{frame}
%%%%%%%%%%%%%%%%%%%%
\note{\footnotesize
There are many types of waves that it is easy to get confused. 
That is why we will start with the basic principles first and we will progress then towards more complex problems.
Based on oscilation form we can divide waves into standing waves and propagating or progressing waves.
An example of standing waves would be vibrating guitar string or structure vibrating at harmonic frequency.
An example of propagating waves will be sound wave or seismic wave.

Based on analysed physical quantity we can distinguish electromagnetic waves, mechanical waves and matter waves.
All light waves such as Microwaves, X-ray, radio waves, ultraviolet waves are examples of electromagnetic waves. 
They can propagate through a vacuum.
In opposite to electromagnetic waves, mechanical waves are not capable of transmitting its energy through the vacuum.
Mechanical waves require a medium in order to transport their energy from one location to another. It has a form of oscilations of particles in a medium.
Elastic and guided waves, on which I will elaborate more, belongs to the group of propagating mechanical waves. 

Matter waves sometimes are called de Broglie Matter Waves after the name of scientist who proposed a new speculative hypothesis that electrons and other particles of matter can behave like waves. 
Now it is a central part of the theory of quantum mechanics, being an example of wave–particle duality.
}
%%%%%%%%%%%%%%%%%%%%%%%%%%%%%%%%%%%%%%%%%%%%%%%%%%
\begin{frame}{Types of wave}
%%%%%%%%%%%%%%%%%%%%%%%%%%%%%%%%%%%%%%%%%%%%%%%%%%
Acoustic waves based on wave frequency can be divided into:
\begin{itemize}
		\item infrasonic waves:  0.001 Hz - 20 Hz (earthquakes monitoring)
		\item sound waves (audible): 20 Hz - 20 kHz 
		\item ultrasonic waves (non-audible for humans): above 20 kHz
		\begin{itemize}
			\item chemistry (sonication, ultrasonication, mixing, production of nanoparticles)
			\item food processing (mixing, homogenization, emulsification): 20 kHz - 1MHz
			\item medical applications (e.g. ultrasonic nebulizer) and non-destructive testing 20 kHz - 2 MHz
			\item diagnostics and NDE 2 MHz - 200 MHz
		\end{itemize}	
	\end{itemize}
\vspace{10mm}
\begin{figure}
	\includegraphics[width=0.9\textwidth]{Sound_wave_spectrum.png}
\end{figure}
\end{frame}
%%%%%%%%%%%%%%%%%%%%
\note{
Our body's physical sensors respond to oscillations with a well defined wavelength and frequency — We personally detect two kinds of waves in our environment to help us build a picture of the world we live in: sound and light. 
Our detectors for these waves — ears and eyes — are evolved to respond differently to different frequencies, giving us perception of pitch and color. 
As a result, an analysis in signals in terms of sinusoidal oscillations tends to make sense to us.
Voice frequency band ranges from approximately 300 to 3400 Hz.	
}
%%%%%%%%%%%%%%%%%%%%%%%%%%%%%%%%%%%%%%%%%%%%%%
\begin{frame}{Elastic waves}
%%%%%%%%%%%%%%%%%%%%%%%%%%%%%%%%%%%%%%%%%%%%%%
\begin{alertblock}{Elastic waves}
Elastic waves are mechanical waves propagating in an elastic medium as an effect of forces associated with volume deformation (compression and extension) and shape deformation (shear) of medium elements.
\end{alertblock}
	
\end{frame}
%%%%%%%%%%%%%%%%%%%%
\note{
}
%%%%%%%%%%%%%%%%%%%%%%%%%%%%%%%%%%%%%%%%%%%%%%
\section{Basic definitions}
%%%%%%%%%%%%%%%%%%%%%%%%%%%%%%%%%%%%%%%%%%%%%%
\begin{frame}[t]{Travelling line}
%%%%%%%%%%%%%%%%%%%%%%%%%%%%%%%%%%%%%%%%%%%%%%
	\begin{columns}[T]
		\begin{column}{0.5\textwidth}
			\only<1->{
				\begin{equation*}
					y=\frac{1}{3} x
				\end{equation*}
		    }	
		    \only<2->{
		    	\begin{equation*}
		    	x \rightarrow x - 6 t
		    	\end{equation*}
		    }	
		    \only<3->{
		    	\begin{equation*}
		    		y=\frac{1}{3} (x-6 t)
		    	\end{equation*}
		    }	
	        \only<4->{
	       	\begin{equation*}
	       	\color{logoblue}
	       	t=0 \quad y=\frac{1}{3} x
	       	\end{equation*}
	       }
		     \only<5->{
		    	\begin{equation*}
		    	\color{green}
		    	t=1 \quad y=\frac{1}{3} (x-2)
		    	\end{equation*}
		    }	
			\end{column}
		\begin{column}{0.5\textwidth}
			\only<1-2>{
			\includegraphics[width=0.9\textwidth]{travelling_line_1.png}
		  	}
	  	 	\only<3-4>{
	  	 	\includegraphics[width=0.9\textwidth]{travelling_line_2.png}
	  	 	}
  	 		\only<5-6>{
  	 			\includegraphics[width=0.9\textwidth]{travelling_line_3.png}
  	 		}
   			\only<6>{
	   		\begin{flalign*}	
	   			&+ \textrm{direction} \quad x \rightarrow x - \upsilon t\\
	   			&- \textrm{direction} \quad x \rightarrow x + \upsilon t
	   		\end{flalign*}	
	   		}
		\end{column}
	\end{columns}		
\end{frame}
%%%%%%%%%%%%%%%%%%%%
\note{We will start with very simple equation: y equal one third of x which is the equation of stright line as it is shown here.
Let's suppose that I want this line to move,
Move it with a speed of 6 meters per second.
All I have to do now is to replace the x in the equation by x minus six t.
Notice the minus sign, it means that our line will go to plus x direction.
The equation becomes: y equal one third of x minus six t.
And at the time t equal zero you already have the line.
At t equal one the line is parallel and it moved in this direction with the speed of six meters per second.
So it is telling us that if we ever wanted to move with the speed v in the plus direction all we have to do is replace in our equation x with x minus v times t
and if we want to move in minus x direction we need to replace x with x minus v times t.
}
%%%%%%%%%%%%%%%%%%%%%%%%%%%%%%%%%%%%%%%%%%%%%%%%%%
\begin{frame}[t]{Travelling wave}
	%%%%%%%%%%%%%%%%%%%%%%%%%%%%%%%%%%%%%%%%%%%%%%%%%%
	\begin{columns}[T]
		\begin{column}{0.5\textwidth}
			\only<1>{
				\begin{equation*}
				y=2 \sin (3 x)
				\end{equation*}
			}	
		   	\only<2->{
		   	\begin{equation*}
		   	y=2 \sin (\underbrace{3}_{k} x)
		   	\end{equation*}
		   }
			\only<3->{
				\begin{equation*}
				x \rightarrow x - 6 t
				\end{equation*}
			}	
			\only<4->{
				\begin{equation*}
			    y=2 \sin \left(\underbrace{3}_{k} (x -\underbrace{6}_{\upsilon} t) \right)
				\end{equation*}
			}	
			\only<5->{
				\begin{equation*}
				\color{logoblue}
				t=0 \quad y=2 \sin (3 x)
				\end{equation*}
			}
			\only<6->{
				\begin{equation*}
				\color{green}
				t=1 \quad y=2 \sin (3 x -6)
				\end{equation*}
			}	
		\end{column}
		\begin{column}{0.5\textwidth}
			\only<1-2>{
				\includegraphics[width=0.9\textwidth]{travelling_sine_1.png}
			}
		   \only<3-5>{
		   	\includegraphics[width=0.9\textwidth]{travelling_sine_2.png}
		   }
		   \only<6>{
		   	\includegraphics[width=0.9\textwidth]{travelling_sine_3.png}
		   }
		   \only<2-6>{
		   	\begin{equation*}
		   	k=\frac{2 \pi}{\lambda} = 3
		   	\end{equation*}
		   }		
		   \only<6>{
		   	\begin{flalign*}	
		   	&+ \textrm{direction} \quad x \rightarrow x - \upsilon t\\
		   	&- \textrm{direction} \quad x \rightarrow x + \upsilon t
		   	\end{flalign*}	
		   }
		\end{column}
	\end{columns}		
\end{frame}
%%%%%%%%%%%%%%%%%%%%
\note{\footnotesize Now I'm going to change to something which is real wave.
y equal to twice sine of three x; that's a wave, it is not moving yet.
And lambda which we call wavelength in this case is two pi over three.
We can also define lambda between consequtive wave peaks.
I will introduce new symbol k which we call wavenumber and k is simply defined as
two pi over lambda, so in our case it is three. If you know this number you can immediately tell what the wavelenght is.
Now I want to have this wave to move. 
I want to have travelling wave.
And I want to move it with the speed of 6 meters per secund in the positive direction.
So the receipe is very simple.
All I have to do is to replace x by x minus six t.
And now if you look at this equation and plot it a little bit later than time t zero you will see that indeed it has moved in the positive direction.
And it is moving with the speed of 6 meters per sucund.
So this equation holds all characteristics of the oscilation.
It holds the amplitude which is 2. 
k holds the information about the wavelenght and this information tells you what the speed is and a minus sign, which is important, tells you that it propagates in positive direction.
If we want to have travelling wave to the left or munus direction we have to have plus sign here.
Peaks of the wave denoted by blue dots are called wave crests and valleys denoted by red dots are called troughs.
}
%%%%%%%%%%%%%%%%%%%%%%%%%%%%%%%%%%%%%%%%%%%%%%%%%%
\begin{frame}[t]{Travelling wave - basic definitions}
%%%%%%%%%%%%%%%%%%%%%%%%%%%%%%%%%%%%%%%%%%%%%%%%%%
	\begin{columns}[T]	
		\begin{column}[T]{0.5\textwidth}
			\only<1->{
				\includegraphics[width=0.9\textwidth]{travelling_wave_string.png}
			}
			\only<1->{
			\begin{equation*}
			y=2 \sin \left(\underbrace{3}_{k} (x -\underbrace{6}_{\upsilon} t) \right)
			\end{equation*}
		   }	
			\only<2->{
			\begin{equation*}
			y = 2 \sin (\underbrace{3}_{k} x -\underbrace{18}_{\omega} t)
			\end{equation*}
			}
		\only<3>{\url{https://phet.colorado.edu/sims/html/wave-on-a-string/latest/wave-on-a-string_en.html}}
		\end{column}
		\begin{column}[T]{0.5\textwidth}
			\centering
			\only<1->{
		Period of one oscillation
			\begin{equation*}
			T=\frac{2 \pi}{\omega}
			\end{equation*}
		Wavelength
			\begin{equation*}
			\lambda = \upsilon T = \frac{\upsilon}{f}
			\end{equation*}
		Frequency 
		\begin{equation*}
			f = \frac{\upsilon}{\lambda}	
		\end{equation*}
		}
		\only<3->{
			Velocity
			\begin{equation*}
			\boxed{\upsilon = \frac{\omega}{k}}	
			\end{equation*}
		}
	 \end{column}
	\end{columns}		
\end{frame}
%%%%%%%%%%%%%%%%%%%%
\note{We can make such a travelling wave by attaching a string to a wheel rotating with angular frequency omega and radius R. 
If the radius has two units it will give us the same amplitude.
So the one period of oscilation is two pi over omega.
The wave is going to propagate with velocity v.
Hence, in order to travel the distance Lambda we need to multiply velocity v by one period T which gives us wavelength.
We can also define freqency as as ratio of velocity to wavelenght.
And now I can write the equation in somewhat different form in which we can distinct spatial part and time part.
This is wavenumber k and this is now angular frequecy omega.
Then I can find velocity which is omega over k.
}
% https://www.compadre.org/nexusph/course/Sinusoidal_waves
%https://phet.colorado.edu/sims/html/wave-on-a-string/latest/wave-on-a-string_en.html
% propagating sine wave 
% no damping, no end (infinite string under tension, oscilate)
% standing wave
% no damping, fixed end, oscilate
% first harmonic, f1=0.41 Hz
% second harmonic, f2=0.83 Hz
% third harmonic f3=1.24 Hz
% fourth harmonic f4=1.66 Hz
%%%%%%%%%%%%%%%%%%%%%%%%%%%%%%%%%%%%%%%%%%%%%%%%%%
\begin{frame}[t]{Standing waves}
%%%%%%%%%%%%%%%%%%%%%%%%%%%%%%%%%%%%%%%%%%%%%%%%%%
\only<1->{
	\begin{alertblock}{Standing waves}	
		Standing wave (stationary wave) is the wave that remains in a constant position (there is no shifting of the waveform).
	\end{alertblock}}
	\only<2->{
\begin{columns}[T]
	\begin{column}{0.5\textwidth}
		\centering
		\(\longrightarrow\)
		\begin{equation*}
		y_1 = y_0 \sin(k x - \omega t)
		\end{equation*}
	\end{column}
	\begin{column}{0.5\textwidth}
		\centering
		\(\longleftarrow\)
		\begin{equation*}
		y_1 = y_0 \sin(k x + \omega t)
		\end{equation*}
	\end{column}
\end{columns}}
\only<3->{
		\begin{equation*}
			y = y_1 + y_2
		\end{equation*}
		\begin{equation*}
			\sin(\theta_1) + \sin(\theta_2) = 2\, \sin\left( \frac{\theta_1 + \theta_2}{2} \right) \cos\left( \frac{\theta_1 - \theta_2}{2} \right)
		\end{equation*}
		\begin{equation*}
		y = 2\, y_0 \sin(k x) \, \cos(\omega t)
		\end{equation*}
}

\only<4->{
	from boundary conditions
	\begin{equation*}
	y = 2\, y_0 \sin\left(\frac{m \pi}{L} x\right) \, \cos(\omega t)
	\end{equation*}}
\end{frame}
%%%%%%%%%%%%%%%%%%%%
\note{
Standing wave (stationary wave) is the wave that remains in a constant position. 
There is no shifting of the waveform; waveform goes always throgh the same nodes.
Stadning waves can arise due to interference. The sum of two counter propagating waves (of equal amplitude and frequency) creates standing wave.
Standing waves commonly arise when a boundary blocks further propagation of the wave, thus causing wave reflection, and therefore introducing a counter propagating wave (such example is a guitar string).
	
Let's suppose that I have travelling wave in right direction and I have another wave which is exactly identical in terms of amplitude, but it is travelling in opposite direction.
This plus sign tells me that it is going in this direction.
So if this is a string the net result is the sum of the two.
By employing trigonometric manipulation like this we can get this equation:
Notice that the amplitude has doubled.
And this is very different from travelling wave.
Nowhere you will see kx minus omega t.
All the timing information is separete from the spatial information.
}
%%%%%%%%%%%%%%%%%%%%%%%%%%%%%%%%%%%%%%%%%%%%%%
\begin{frame}{Standing waves (normal modes)}
%%%%%%%%%%%%%%%%%%%%%%%%%%%%%%%%%%%%%%%%%%%%%%
\begin{figure}
	\animategraphics[controls,autoplay,loop,width=0.8\textwidth]{3}{standing_wave_}{1}{33}
\end{figure}
\end{frame}
%%%%%%%%%%%%%%%%%%%%
\note{
	
}
%%%%%%%%%%%%%%%%%%%%%%%%%%%%%%%%%%%%%%%%%%%%%%%%%%
\begin{frame}{Basic terminology}
%%%%%%%%%%%%%%%%%%%%%%%%%%%%%%%%%%%%%%%%%%%%%%%%%%
\begin{columns}[T]
		\begin{column}{0.5\textwidth}
			\begin{figure}
				\only<1>{
				\includegraphics[width=0.9\textwidth]{Fig_1_7.jpg}
				}
				\only<2>{
					\includegraphics[width=0.9\textwidth]{Fig_1_8.jpg}
				}
			\end{figure}
		\end{column}
		\begin{column}{0.5\textwidth}
		\begin{itemize}
		\item Phase velocity \(c_p=\frac{\omega}{k}\) - velocity of wave crests
		\item Group velocity \(c_g = \frac{\drv \omega}{\drv k}\) - velocity of wave packets
		\item Wavelength \(\lambda\) - distance between two consecutive crests or two consecutive troughs
		\item Wavenumber \(k\)
		\item Frequency \(f\)
		\end{itemize}
		\end{column}
\end{columns}

\end{frame}
%%%%%%%%%%%%%%%%%%%%
\note{
}
%%%%%%%%%%%%%%%%%%%%%%%%%%%%%%%%%%%%%%%%%%%%%%%%%%
\begin{frame}{Wave modulation}
%%%%%%%%%%%%%%%%%%%%%%%%%%%%%%%%%%%%%%%%%%%%%%%%%%
Consider two right propagating waves of equal amplitude, and different frequency and wavenumber
\begin{equation*}
u(x,t)=U_0 \left[ \sin(k_1 x - \omega_1 t) + \sin( k_2 x - \omega_2 t)\right]
\end{equation*}

\begin{equation*}
u(x,t)=2 U_0\, \cos\left(\frac{k_1- k_2}{2}  x - \frac{\omega_1-\omega_2}{2}  t\right) \, \sin\left(\frac{ k_1 + k_2} x -\frac{ \omega_1 + \omega_2}{2} t \right)
\end{equation*}

\begin{equation*}
u(x,t)=2 U_0\, \color{darkblue}{\underbrace{\cos(\Delta k x - \Delta \omega   t)}_{\textrm{Modulation}}} \, \color{red}{\underbrace{\sin( k_0  x - \omega t)}_{\textrm{Carrier wave}}}
\end{equation*}

\begin{itemize}
	\item Beating phenomenon
	\item Propagation of a wave packet (assembly of waves)
\end{itemize}
\end{frame}
%%%%%%%%%%%%%%%%%%%%
\note{
}
%%%%%%%%%%%%%%%%%%%%%%%%%%%%%%%%%%%%%%%%%%%%%%%%%%
\begin{frame}{Wave modulation}
%%%%%%%%%%%%%%%%%%%%%%%%%%%%%%%%%%%%%%%%%%%%%%%%%%
\begin{figure}
		\includegraphics[width=0.9\textwidth]{Fig_1_9.jpg}
	\end{figure}	
\end{frame}
%%%%%%%%%%%%%%%%%%%%
\note{
}
%%%%%%%%%%%%%%%%%%%%%%%%%%%%%%%%%%%%%%%%%%%%%%%%%%
\begin{frame}{Wave modulation}
%%%%%%%%%%%%%%%%%%%%%%%%%%%%%%%%%%%%%%%%%%%%%%%%%%
Propagation speed of modulating wave defines the propagation speed of the wave packet
\begin{equation*}
\Delta k x - \Delta \omega t = const \quad \rightarrow \quad x = \frac{\Delta \omega}{\Delta k} t + const
\end{equation*}

\begin{equation*}
c_g = \frac{\Delta \omega}{\Delta k}
\end{equation*}
To the limit \(\Delta \omega \rightarrow 0, \Delta k \rightarrow 0\)
\begin{equation*}
\hspace{2.8cm}\boxed{c_g = \frac{\drv \omega}{\drv k}} \quad \textrm{\color{red}{Group velocity}}
\end{equation*}
\begin{itemize}
	\item \makebox[3.2cm][l]{non-dispersive media} \(c_g = c_p\)
	\item \makebox[3.2cm][l]{dispersive media} \(c_g \neq c_p \)
\end{itemize}
\end{frame}
%%%%%%%%%%%%%%%%%%%%
\note{
}
%%%%%%%%%%%%%%%%%%%%%%%%%%%%%%%%%%%%%%%%%%%%%%
\begin{frame}{Group vs phase velocity (1)}
%%%%%%%%%%%%%%%%%%%%%%%%%%%%%%%%%%%%%%%%%%%%%%%%%%
	\begin{figure}
		\includegraphics[width=0.5\textwidth]{group_vs_phase_velocity.png}
	\end{figure}	
\end{frame}
%%%%%%%%%%%%%%%%%%%%
\note{
phase velocity - angle between a line connecting zero and angular frequency value and wavenumber axis
group velocity - tangent
}
%%%%%%%%%%%%%%%%%%%%%%%%%%%%%%%%%%%%%%%%%%%%%%
\begin{frame}{Group vs phase velocity (2)}
%%%%%%%%%%%%%%%%%%%%%%%%%%%%%%%%%%%%%%%%%%%%%%
\begin{columns}[T]
	\begin{column}{0.5\textwidth}
		\begin{figure}
			\animategraphics[autoplay,loop,width=0.6\textwidth]{5}{wave_phase_equal_group_velocity_}{1}{50}
			\caption{\(c_g = c_p\)}
		\end{figure}
		\begin{figure}
			\animategraphics[autoplay,loop,width=0.6\textwidth]{5}{wave_opposite_group_phase_velocity_}{1}{50}
			\caption{\(c_g>0,\,c_p<0\)}
		\end{figure}
	\end{column}
	\begin{column}{0.5\textwidth}
		\begin{figure}
			\animategraphics[autoplay,loop,width=0.6\textwidth]{5}{wave_phase_faster_than_group_velocity_}{1}{50}
			\caption{\(c_p > c_g\)}
		\end{figure}
	   \begin{figure}
	   	\animategraphics[autoplay,loop,width=0.6\textwidth]{5}{wave_phase_slower_than_group_velocity_}{1}{50}
	   	\caption{\(c_p < c_g\)}
	   \end{figure}
	\end{column}
\end{columns}
\end{frame}
%%%%%%%%%%%%%%%%%%%%
\note{
For positive group velocity and negative phase velocity the wave is dancing like Michael Jackson. It is like his feet are going to the left whereas his body is going to the right.	
}
%%%%%%%%%%%%%%%%%%%%%%%%%%%%%%%%%%%%%%%%%%%%%%
\section{What is wave dispersion?}
%%%%%%%%%%%%%%%%%%%%%%%%%%%%%%%%%%%%%%%%%%%%%%
\begin{frame}{Dispersion curves (1)}
	\begin{alertblock}{Definition}
		A \textbf{dispersion relation} relates the wavelength $\lambda$ or wavenumber $k$ of a wave to its frequency $\omega$.\\
		\begin{equation*}
		\boxed{k=k(\omega)} \quad \textrm{or} \quad \boxed{\omega = \omega(k)}
		\end{equation*}
		\vspace{10pt}
		$k(\omega)$ $[\frac{\mathrm{rad}}{\mathrm{m}}] \quad k(f)$ $[\frac{1}{\mathrm{m}}]$
	\end{alertblock}
	\begin{block}{Phase velocity}
		\begin{equation*}
		c_p = \frac{\omega}{k} = const\quad \textrm{non-dispersive}
		\end{equation*}
	\end{block}
	\begin{block}{Group velocity}
		\begin{equation*}
		c_g = \frac{\drv \omega}{\drv k} = const \quad \textrm{non-dispersive}
		\end{equation*}
	\end{block}
\end{frame}
%%%%%%%%%%%%%%%%%%%%
\note{
What is the phenomenon of dispersion?
This is the relationship between the wavenumber or wavelength and its frequency. 
I usually use the wavenumber and the linear frequency, hence the unit of the wave number is one over a meter. 
The condition necessary to have non-dispersive wave is that phase velocity must be constant.
It implies that the group velocity is also constant and actually equal to phase velocity.
}
%%%%%%%%%%%%%%%%%%%%%%%%%%%%%%%%%%%%%%%%%%%%%%
\begin{frame}{Dispersion curves (2)}
	\begin{figure}
		\includegraphics[width=0.8\textwidth]{linear_dispersion.png}
	\end{figure}
A nondispersive medium is characterized by:
	\begin{itemize}
	\item A linear dispersion relation
	\item The phase velocity is constant at all frequencies
	\end{itemize}
\end{frame}
%%%%%%%%%%%%%%%%%%%%
\note{
	
}
%%%%%%%%%%%%%%%%%%%%%%%%%%%%%%%%%%%%%%%%%%%%%%
\begin{frame}{Dispersion effect (1)}
	\begin{columns}[T]
		\column{0.5\textwidth}
		\begin{figure}
			\includegraphics[width=0.7\textwidth]{dispersion/excitation_narrow_time.png}
			\includegraphics[width=0.7\textwidth]{dispersion/excitation_wide_time.png}
		\end{figure}
		\column{0.5\textwidth}
		\newcommand{\modelname}{dispersion_effect}
		\begin{figure}
			\includegraphics[width=0.7\textwidth]{dispersion/excitation_narrow_frequency.png}
			\includegraphics[width=0.7\textwidth]{dispersion/excitation_wide_frequency.png}
		\end{figure}
	\end{columns}
\end{frame}
%%%%%%%%%%%%%%%%%%%%
\note{
To demonstrate to you what the phenomenon of wave dispersion reveals, I have prepared an example. 
Consider two signals: one with a 200 kHz carrier frequency that looks like this and the other with a 20 kHz carrier frequency that looks like this.
}
%%%%%%%%%%%%%%%%%%%%%%%%%%%%%%%%%%%%%%%%%%%%%%
\begin{frame}{Dispersion effect (2)}
%%%%%%%%%%%%%%%%%%%%%%%%%%%%%%%%%%%%%%%%%%%%%%
\begin{columns}[T]
	\column{0.5\textwidth}
	\only<1>{
		\begin{figure}
			\includegraphics[width=0.8\textwidth]{dispersion/A0_dispersion_less_dispersive.png}
			\caption{Less dispersive region}
		\end{figure}
		\column{0.5\textwidth}
		\newcommand{\modelname}{dispersion_effect}
		\begin{figure}
			\includegraphics[width=0.8\textwidth]{dispersion/A0_dispersion_dispersive.png}
			\caption{Dispersive region}
		\end{figure}
	}
	\only<2>{
		\begin{figure}
			\includegraphics[width=0.8\textwidth]{dispersion/A0_phase_velocity_less_dispersive.png}
			\caption{Less dispersive region}
		\end{figure}
		\column{0.5\textwidth}
		\newcommand{\modelname}{dispersion_effect}
		\begin{figure}
			\includegraphics[width=0.8\textwidth]{dispersion/A0_phase_velocity_dispersive.png}
			\caption{Dispersive region}
		\end{figure}
	}
	\only<3>{
		\begin{figure}
			\includegraphics[width=0.8\textwidth]{dispersion/A0_group_velocity_less_dispersive.png}
			\caption{Less dispersive region}
		\end{figure}
		\column{0.5\textwidth}
		\begin{figure}
			\includegraphics[width=0.8\textwidth]{dispersion/A0_group_velocity_dispersive.png}
			\caption{Dispersive region}
		\end{figure}
	}
\end{columns}
\begin{center}
	\only<2>{\alert{Phase velocities}}
	\only<3>{\alert{Group velocities}}
\end{center}	
\end{frame}
%%%%%%%%%%%%%%%%%%%%
\note{
Let's superimpose the frequency spectra on the dispersion curves. 
The spectrum on the left is in a region with less dispersion than the spectrum on the right. 
You can see that the frquencies of our signal on the left hand side are on almost linear portion of the dispersion curves whereas the frequencies of the signal on the right hand side are on a nonlinear portion of the dispersion curve. 

This effect is easier to explain by taking into account phase velocities. 
The flatter the curve, the smaller the wave dispersion. 
The same is true for group velocities.
So if we want to avoid or alleviete the effect of dispersion we need to use narrow frequency band excitation signal with carrier frequency which corresponds to flat region of dispersion curves expressed as relationship between wave velocity and frequencies.
}
%%%%%%%%%%%%%%%%%%%%%%%%%%%%%%%%%%%%%%%%%%%%%%
\begin{frame}{Dispersion effect (3)}
%%%%%%%%%%%%%%%%%%%%%%%%%%%%%%%%%%%%%%%%%%%%%%
	\begin{figure}
		\animategraphics[autoplay,loop,width=0.8\textwidth]{1}{dispersion/dispersion_effect_less_dispersive_L_}{1}{11}
	\end{figure}
\vspace{-5mm}
	\begin{figure}
		\animategraphics[autoplay,loop,width=0.8\textwidth]{1}{dispersion/dispersion_effect_dispersive_L_}{1}{11}
	\end{figure}
\end{frame}
%%%%%%%%%%%%%%%%%%%%
\note{
The dispersion effect is visible in this animation. 
The wave shape at the top does not change much with the propagation distance, while the wave at the bottom is dispersive and deforms with the propagation distance.
}

%%%%%%%%%%%%%%%%%%%%%%%%%%%%%%%%%%%%%%%%%%%%%%
\section{1D wave propagation}
%%%%%%%%%%%%%%%%%%%%%%%%%%%%%%%%%%%%%%%%%%%%%%
\begin{frame}[t]{Longitudinal waves in rod (1)}
%%%%%%%%%%%%%%%%%%%%%%%%%%%%%%%%%%%%%%%%%%%%%%
\begin{figure}
\includegraphics[width=0.8\textwidth]{rod_segment_with_loads.png}
\caption{Segment of a rod}
\end{figure}
\begin{columns}[T]
	\begin{column}{0.5\textwidth}
		\(E\) Young's modulus\\
		\(\rho\) mass density\\
		\(\eta\) viscous damping per unit volume\\
		\(q(x,t)\) external force per unit length\\
	\end{column}
	\begin{column}{0.5\textwidth}
		According to Newton's second law of motion:
		\begin{equation*}
		\sum Fx = m a_x
		\end{equation*}
	\end{column}
\end{columns}		


\end{frame}
%%%%%%%%%%%%%%%%%%%%
\note{
}
%%%%%%%%%%%%%%%%%%%%%%%%%%%%%%%%%%%%%%%%%%%%%%
\begin{frame}[t]{Longitudinal waves in rod (2)}
%%%%%%%%%%%%%%%%%%%%%%%%%%%%%%%%%%%%%%%%%%%%%%
	\begin{figure}
		\includegraphics[width=0.8\textwidth]{rod_segment_with_loads.png}
	\end{figure}
\begin{columns}[T]
	\begin{column}{0.5\textwidth}
			Following the assumption of only one displacement \(u(x)\), the axial strain is given by:
		\begin{equation*}
		\varepsilon_{xx}  = \frac{\partial u}{\partial x}
		\end{equation*}
	\end{column}
	\begin{column}{0.5\textwidth}
		Considering 1D form of Hooke's law (linear elastic material):
		\begin{equation*}
		\sigma_{xx} = E \varepsilon_{xx}
		\end{equation*}
	\end{column}
\end{columns}
	Resultant axial force:
	\begin{equation*}
	F=\int_A \sigma_{xx} \drv A = EA \frac{\partial u}{\partial x}
	\end{equation*}
	
	\begin{equation*}
    -F + (F + \Delta F) + q \Delta x - \eta A \Delta x \dot{u} = \rho A \Delta x \ddot{u}
	\end{equation*}
\end{frame}
%%%%%%%%%%%%%%%%%%%%
\note{
}
%%%%%%%%%%%%%%%%%%%%%%%%%%%%%%%%%%%%%%%%%%%%%%
\begin{frame}[t]{Longitudinal waves in rod (3)}
%%%%%%%%%%%%%%%%%%%%%%%%%%%%%%%%%%%%%%%%%%%%%%
\only<1->{
If \(\Delta\) quantities are small
	\begin{equation*}
	\frac{\partial F}{\partial x} = \rho A \frac{\partial^2 u}{\partial t^2} + \eta A \frac{\partial u}{\partial t} - q
	\end{equation*}	
	Substitute
	\begin{equation*}
	F= EA \frac{\partial u}{\partial x}
	\end{equation*}
	
  \begin{equation*}
  \frac{\partial }{\partial x} \left[ EA \frac{\partial u}{\partial x} \right]= \rho A \frac{\partial^2 u}{\partial t^2} + \eta A \frac{\partial u}{\partial t} - q
\end{equation*}	
}
\only<2->{
In the special case of uniform properties and no damping homogenous equation is:
\begin{equation*}
c_0^2 \frac{\partial^2 u(x,t)}{\partial x^2} - \frac{\partial^2 u(x,t)}{\partial t^2} = 0, \quad c_0 = \sqrt{\frac{EA}{\rho A}}
\end{equation*}	
D'Alembert general solution
\begin{equation*}
u(x,t) = f(x-c_0 t) + g(x+c_0 t)
\end{equation*}	
}
\end{frame}
%%%%%%%%%%%%%%%%%%%%
\note{
D'Alembert  solution does not hold for the general rods therefore will not be pursued any further. Homogenous equation can also be solved by using separation of variables. But we will consider more general case on next slides.
}
%%%%%%%%%%%%%%%%%%%%%%%%%%%%%%%%%%%%%%%%%%%%%%
\begin{frame}[t]{Elementary rod theory - spectral  analysis (1)}
%%%%%%%%%%%%%%%%%%%%%%%%%%%%%%%%%%%%%%%%%%%%%%
\begin{equation*}
\frac{\partial }{\partial x} \left[ EA \frac{\partial u}{\partial x} \right] - \rho A \frac{\partial^2 u}{\partial t^2}  - \eta A \frac{\partial u}{\partial t} = - q
\end{equation*}	
\(\xrightarrow{\mathcal{F}} \)
	\begin{equation*}
	\frac{\drv}{\drv x}\left[EA \frac{\drv \hat{u}}{\drv x}\right] + \omega^2 \rho A \hat{u} - i \omega \eta A \hat{u} = -\hat{q} 
	\end{equation*}	
Assumption: \(EA\) is not changing along the rod\\
Homogeneous equation (without external force):	
	\begin{equation*}
	EA \frac{\drv^2 \hat{u}}{\drv x^2} + \left(\omega^2 \rho A  - i \omega \eta A \right)\hat{u} = 0
	\end{equation*}	
Assume solution in the form:
	\begin{equation*}
	\hat{u}(x) = \matr{A} \myexp^{-k_1x} + \matr{B} \myexp^{+k_1x}
	\end{equation*}	
	\(\matr{A}, \matr{B}\) undetermined amplitudes at each frequency
\end{frame}
%%%%%%%%%%%%%%%%%%%%
\note{
}
%%%%%%%%%%%%%%%%%%%%%%%%%%%%%%%%%%%%%%%%%%%%%%
\begin{frame}[t]{Elementary rod theory - spectral  analysis (2)}
%%%%%%%%%%%%%%%%%%%%%%%%%%%%%%%%%%%%%%%%%%%%%%
\only<1->{
Spectrum relation:
\begin{columns}[T]
	\begin{column}{0.5\textwidth}
		\centering
	damped case:	
		\begin{equation*}
			k_1 = \sqrt{\frac{\omega^2 \rho A - i \omega \eta A}{EA}}
		\end{equation*}	
		\begin{equation*}
		 c_p = \frac{\omega}{k} = \frac{\omega}{\sqrt{\frac{\omega^2 \rho A - i \omega \eta A}{EA}}}
		 \end{equation*}
	\end{column}
	\begin{column}{0.5\textwidth}
		\centering
	undamped case:	
		\begin{equation*}
			k_1 = \sqrt{\frac{\omega^2 \rho A}{EA}}
		\end{equation*}	
			\begin{equation*}
		   c_p = \frac{\omega}{k} = \sqrt{\frac{EA}{\rho A}} = c_0, \quad c_g = \frac{\drv \omega}{\drv k}\sqrt{\frac{EA}{\rho A}} = c_0
		\end{equation*}	
	\end{column}
\end{columns}	
}
\only<2->{
\(\xrightarrow{\mathcal{F}^{-1}} \)
\begin{equation*}
u(x,t) = \sum \matr{A} \myexp^{-i(k_1 x - \omega t)} + \sum \matr{B} \myexp^{+i (k_1 x +\omega t)}
\end{equation*}	
}
	
\end{frame}
%%%%%%%%%%%%%%%%%%%%
\note{
}
%%%%%%%%%%%%%%%%%%%%%%%%%%%%%%%%%%%%%%%%%%%%%%
\begin{frame}[t]{Elementary rod theory - spectral  analysis (3)}
%%%%%%%%%%%%%%%%%%%%%%%%%%%%%%%%%%%%%%%%%%%%%%
	\begin{figure}
		\includegraphics[width=0.6\textwidth]{elastic_vs_viscoelastic_rod_Doyle.png}
		\caption{Comparison of elastic and viscoelastic materials in terms of spectrum relations for rod}
	\end{figure}	
{\scriptsize
\begin{biblio}{Source}
	\biblioref{Doyle J.F.}{1997}{ Wave Propagation in Structures: Spectral Analysis Using Fast Discrete Fourier Transforms}{Springer}
\end{biblio}}
\end{frame}
%%%%%%%%%%%%%%%%%%%%
\note{
}
%%%%%%%%%%%%%%%%%%%%%%%%%%%%%%%%%%%%%%%%%%%%%%
\begin{frame}[t]{Spectral rod element}
%%%%%%%%%%%%%%%%%%%%%%%%%%%%%%%%%%%%%%%%%%%%%%
	\begin{figure}
		\includegraphics[width=0.8\textwidth]{spectral_rod.png}
		\caption{Spectral rod element along with throw-off element}
	\end{figure}	
General longitudinal displacement
\begin{equation*}
	\hat{u}(x) = \matr{A} \myexp^{-i k_1 x} +\matr{B} \myexp^{-i k_1 (L-x)}
\end{equation*}
\( \matr{A}, \matr{B}\) determined from boundary conditions

Nodal displacements (degrees of freedom):
\begin{equation*}
\hat{u}(0) = \hat{u}_1 = \matr{A} + \matr{B} \myexp^{- i k_1 L} \quad  \hat{u}(L) = \hat{u}_2 = \matr{A} \myexp^{- i k_1 L} + \matr{B}
\end{equation*}
\end{frame}
%%%%%%%%%%%%%%%%%%%%
\note{
}
%%%%%%%%%%%%%%%%%%%%%%%%%%%%%%%%%%%%%%%%%%%%%%
\begin{frame}[t]{Spectral rod element: shape functions}
%%%%%%%%%%%%%%%%%%%%%%%%%%%%%%%%%%%%%%%%%%%%%%
	\begin{figure}
	\includegraphics[width=0.8\textwidth]{spectral_rod.png}
\end{figure}	
\begin{equation*}
\hat{u}(x) = \hat{N}_1(x) \hat{u}_1 + \hat{N}_2(x) \hat{u}_2
\end{equation*}
\begin{columns}[T]
	\begin{column}{0.5\textwidth}
		\begin{align*}
		&\hat{N}_1(x) = \left[ \myexp^{-i k _1 x} - \myexp^{-i k_1 (2L-x)}\right]/\Delta\\
		&\hat{N}_2(x) = \left[ -\myexp^{-i k _1(L+x) } + \myexp^{-i k_1 (L-x)}\right]/\Delta\\
		&\Delta = 1-\myexp^{-i 2 k_1 L}
		\end{align*}
	\end{column}
	\begin{column}{0.5\textwidth}
		\begin{align*}
	&\hat{N}_1(x) = \csc(k L) \sin\left(k (L-x)\right)\\
	&\hat{N}_2(x) = \csc(k L) \sin(k  x)
	\end{align*}	
	\end{column}
\end{columns}	
For throw-off element
\begin{equation*}
\hat{N}_1(x) = \myexp^{-i k x}
\end{equation*}	
\end{frame}
%%%%%%%%%%%%%%%%%%%%
\note{
}
%%%%%%%%%%%%%%%%%%%%%%%%%%%%%%%%%%%%%%%%%%%%%%
\begin{frame}[t]{Spectral rod element: dynamic stiffness matrix}
%%%%%%%%%%%%%%%%%%%%%%%%%%%%%%%%%%%%%%%%%%%%%%
\begin{equation*}
F= EA \frac{\partial u}{\partial x}, \quad \hat{u}(x) = \hat{N}_1(x) \hat{u}_1 + \hat{N}_2(x) \hat{u}_2
\end{equation*}
\begin{align*}
& \hat{F}_1 = -\hat{F}(0) = -\hat{E}\hat{A} \left( \frac{\partial N_1(0)}{\partial x} \hat{u}_1 +  \frac{\partial N_2(0)}{\partial x} \hat{u}_2 \right)\\
& \hat{F}_2 = +\hat{F}(L) = +\hat{E}\hat{A} \left( \frac{\partial N_1(L)}{\partial x} \hat{u}_1 +  \frac{\partial N_2(L)}{\partial x} \hat{u}_2 \right)
\end{align*}	
\begin{equation*}
\left\{
	\begin{array}{c}
	\hat{F}_1 \\ 
	\hat{F}_2 
	\end{array} 
	\right\} = \hat{E}\hat{A}  
     \left[  
     \begin{array}{rr}
      - \frac{\partial N_1(0)}{\partial x}  &  - \frac{\partial N_2(0)}{\partial x}  \\
       \frac{\partial N_1(L)}{\partial x}  &  \frac{\partial N_2(L)}{\partial x} 
       \end{array}
       \right]
       \left\{ \begin{array}{l}\hat{u}_1 \\ \hat{u}_2\end{array}\right\} = 
       \left[  \begin{array}{ll} \hat{k}_{11}^e & \hat{k}_{12}^e\\
       \hat{k}_{21}^e & \hat{k}_{22}^e \end{array}\right]
       \left\{ \begin{array}{l} \hat{u}_1 \\ \hat{u}_2\end{array}\right\}
\end{equation*}
\begin{equation*}
\begin{array}{ll}
 \hat{k}_{11}^e = k \hat{E}\hat{A} \cot (k L), \quad &\hat{k}_{12}^e = -k \hat{E}\hat{A} \csc (k L)\\
 \hat{k}_{21}^e =-k \hat{E}\hat{A} \csc (k L), \quad &\hat{k}_{22}^e = k \hat{E}\hat{A} \cot (k L)
\end{array}
\end{equation*}
throw-off \(\hat{k}^e = \hat{E} \hat{A} i k \)
\end{frame}
%%%%%%%%%%%%%%%%%%%%
\note{
}
%%%%%%%%%%%%%%%%%%%%%%%%%%%%%%%%%%%%%%%%%%%%%%
\begin{frame}{Measured waves in a rod}
%%%%%%%%%%%%%%%%%%%%%%%%%%%%%%%%%%%%%%%%%%%%%%
\begin{columns}[T]
	\begin{column}{0.5\textwidth}
		\begin{figure}
			\animategraphics[autoplay,loop,width=1.05\textwidth]{5}{animation-rod-laser-sync/along_}{1}{201}
		\end{figure}
	\end{column}
	\begin{column}{0.5\textwidth}
		\begin{figure}
			\animategraphics[autoplay,loop,width=0.9\textwidth]{5}{animation-rod-laser-sync/across_}{1}{201}
		\end{figure}
	\end{column}
\end{columns}
\end{frame}
%%%%%%%%%%%%%%%%%%%%
\note{
}
%%%%%%%%%%%%%%%%%%%%%%%%%%%%%%%%%%%%%%%%%%%%%%
\section{Guided waves}
%%%%%%%%%%%%%%%%%%%%%%%%%%%%%%%%%%%%%%%%%%%%%%
\begin{frame}{Ultrasonic testing vs guided wave testing (1)}
%%%%%%%%%%%%%%%%%%%%%%%%%%%%%%%%%%%%%%%%%%%%%%
\begin{figure}
	\includegraphics[width=0.6\textwidth]{1280px-UT_vs_GWT.jpg}
	\caption{source: wikipedia}
\end{figure}
\end{frame}
%%%%%%%%%%%%%%%%%%%%
\note{
On the upper figure, bulk waves are introduced to the structure by the transducer head.
They are propagating through the thickness of inspected element. 
Based on the characteristics of bulk waves such as velocity and amplitude of reflected signal we can assess only small area but with quite high precision regarding defect occurence. Then we need to move the transducer head to inspect other part of the structure.

On the lower figure, waves are guided by the shape of the structural element. 
Hence the name - guided waves. 
You can imagine that this is a pipe transporting for example gas or oil and there is a device in the form of a ring of transducers which excite elastic waves. 
Similar ring of transducers can be placed far away and register incoming waves.
 If there is any anomaly, change of waveform, it would mean that there is a damage in between transducer rings. 
 In this way inspected area is very large. 
 Actuators and sensors can be placed permanently and the structure can be monitored online. 
 However, the precision of inspection can be lower than in case of ultrasonic testing. 
}
%%%%%%%%%%%%%%%%%%%%%%%%%%%%%%%%%%%%%%%%%%%%%%%%%%
\begin{frame}{Ultrasonic testing vs guided wave testing (2)}
%%%%%%%%%%%%%%%%%%%%%%%%%%%%%%%%%%%%%%%%%%%%%%%%%%
\alert{Bulk waves} exist in infinite homogenous bodies and propagate indefinitely without being interrupted by boundaries or interfaces. 
These waves can be decomposed into infinite plane waves propagating along arbitrary direction within the solid.

\alert{Guided waves} are those waves that require a boundary for their existence, such as surface waves, Lamb waves, and interface waves.
\vspace{5mm}
\begin{columns}[T]
	\begin{column}{0.5\textwidth}
	\textbf{Ultrasonic waves}	
	\begin{itemize}
		\item Frequency range: 2 MHz - 200 MHz
		\item Wavelength \(\lambda << h\) thickness 
		\item shorter wavelengths
	\end{itemize}
	\end{column}
	\begin{column}{0.5\textwidth}
	\textbf{Guided waves}	
	\begin{itemize}
		\item Typical frequency range: 10 kHz - 1 MHz
		\item Wavelength \(\lambda > h\) thickness 
		\item longer wavelengths
	\end{itemize}
	\end{column}
\end{columns}			
\end{frame}
%%%%%%%%%%%%%%%%%%%%
\note{
	
}
%%%%%%%%%%%%%%%%%%%%%%%%%%%%%%%%%%%%%%%%%%%%%%
\begin{frame}{Waves used in non-destructive testing}
%%%%%%%%%%%%%%%%%%%%%%%%%%%%%%%%%%%%%%%%%%%%%%
Elastic wave propagation types depending on particle motion:
\begin{itemize}
	\item  \alert{The longitudinal wave}is a compressional wave in which the particle motion is in the same direction as the propagation of the wave
	\item \alert{The shear wave} is a wave motion in which the particle motion is perpendicular to the direction of the propagation
	\item \alert{Surface (Rayleigh) waves} have an elliptical particle motion and travel across the surface of a material. Their velocity is approximately 90\% of the shear wave velocity of the material and their depth of penetration is approximately equal to one
	wavelength
	\item \alert{Plate (Lamb) waves} have a complex vibration occurring in materials where thickness is less than the wavelength of elastic wave introduced into it.
\end{itemize}
		
\end{frame}
%%%%%%%%%%%%%%%%%%%%
\note{
	The most common methods of ultrasonic examination utilize either
	longitudinal waves or shear waves. Other forms of elastic wave propagation exist,
	including surface waves and Lamb waves. 	
}
%%%%%%%%%%%%%%%%%%%%%%%%%%%%%%%%%%%%%%%%%%%%%%%%%%
\begin{frame}{Waves used in non-destructive testing}
%%%%%%%%%%%%%%%%%%%%%%%%%%%%%%%%%%%%%%%%%%%%%%%%%%
\begin{figure}
	\animategraphics[controls,autoplay,loop,width=0.5\textwidth]{5}{frame}{1}{74}
	\caption{\alert{Longitudinal wave}  - plane pressure pulse wave (source: wikipedia)}
\end{figure}
\end{frame}
%%%%%%%%%%%%%%%%%%%%
\note{
	
}
%%%%%%%%%%%%%%%%%%%%%%%%%%%%%%%%%%%%%%%%%%%%%%%%%%
\begin{frame}{Waves used in non-destructive testing}
%%%%%%%%%%%%%%%%%%%%%%%%%%%%%%%%%%%%%%%%%%%%%%%%%%
	\begin{figure}
		\includegraphics[width=0.5\textwidth]{Fig_1_1.jpg}
		\caption{\alert{Shear horizontal wave} }
	\end{figure}
\end{frame}
%%%%%%%%%%%%%%%%%%%%
\note{
	
}
%%%%%%%%%%%%%%%%%%%%%%%%%%%%%%%%%%%%%%%%%%%%%%%%%%
\begin{frame}{Waves used in non-destructive testing}
%%%%%%%%%%%%%%%%%%%%%%%%%%%%%%%%%%%%%%%%%%%%%%%%%%
	\begin{figure}
		\includegraphics[width=0.5\textwidth]{Fig_1_3.jpg}
		\caption{\alert{Rayleigh waves} }
	\end{figure}
\end{frame}
%%%%%%%%%%%%%%%%%%%%
\note{
	
}
%%%%%%%%%%%%%%%%%%%%%%%%%%%%%%%%%%%%%%%%%%%%%%
\begin{frame}{Waves used in non-destructive testing}
%%%%%%%%%%%%%%%%%%%%%%%%%%%%%%%%%%%%%%%%%%%%%%
	\begin{figure}
		\includegraphics[width=0.5\textwidth]{Fig_1_4.jpg}
		\caption{\alert{Love waves} (surface seismic waves) named after Augustus Edward Hough Love }
	\end{figure}
\end{frame}
%%%%%%%%%%%%%%%%%%%%
\note{
	
}
%%%%%%%%%%%%%%%%%%%%%%%%%%%%%%%%%%%%%%%%%%%%%%
\begin{frame}{Waves used in non-destructive testing}
%%%%%%%%%%%%%%%%%%%%%%%%%%%%%%%%%%%%%%%%%%%%%%
\begin{columns}[T]
	\begin{column}{0.5\textwidth}
		\begin{figure}
		\includegraphics[width=0.9\textwidth]{Fig_1_5.jpg}
		\caption{Fundamental symmetric, S0, \alert{Lamb wave} mode }	
		\end{figure}
	\end{column}
	\begin{column}{0.5\textwidth}
		\begin{figure}
		\includegraphics[width=0.9\textwidth]{Fig_1_6.jpg}
		\caption{Fundamental antisymmetric, A0, \alert{Lamb wave} mode }
		\end{figure}
	\end{column}
\end{columns}	
\end{frame}
%%%%%%%%%%%%%%%%%%%%
\note{
	
}
%%%%%%%%%%%%%%%%%%%%%%%%%%%%%%%%%%%%%%%%%%%%%%
\begin{frame}{Lamb waves}
%%%%%%%%%%%%%%%%%%%%%%%%%%%%%%%%%%%%%%%%%%%%%%
\begin{alertblock}{Lamb waves}	
		Lamb waves are plane waves propagating in thin plates.\\
		Shear vertical waves in conjunction with longitudinal P waves interacts with plate surfaces resulting in complex wave mechanism which leads to creation of Lamb waves.
\end{alertblock}
Horace Lamb discovered these type of waves in 1917.
He derived theory and dispersion relations.
\begin{columns}[T]
	\begin{column}{0.5\textwidth}
		\centering
		symmetric modes
		\begin{equation*}
		  \frac{\tan(q h)}{\tan(p h)} = -\frac{4 k^2 p q}{\left(q^2 - k^2\right)^2}
		\end{equation*}
	\end{column}
	\begin{column}{0.5\textwidth}
		\centering
		antisymmetric modes
		\begin{equation*}
		\frac{\tan(q h)}{\tan(p h)} = -\frac{\left(q^2 - k^2\right)^2}{4 k^2 p q}
		\end{equation*}
	\end{column}
\end{columns}
	
\centering
\(q=q(\omega,k), \quad p=p(\omega,k) \)
\end{frame}
%%%%%%%%%%%%%%%%%%%%
\note{
	
}
%%%%%%%%%%%%%%%%%%%%%%%%%%%%%%%%%%%%%%%%%%%%%%
\begin{frame}{Dispersion curves of Lamb waves}
%%%%%%%%%%%%%%%%%%%%%%%%%%%%%%%%%%%%%%%%%%%%%%
\begin{figure}
	\only<1>{
	\includegraphics[width=0.8\textwidth]{Fig_1_12.png}	
	}
	\only<2>{
		\includegraphics[width=0.8\textwidth]{Fig_1_13.png}	
	}
\end{figure}
\end{frame}
%%%%%%%%%%%%%%%%%%%%
\note{
	
}
%%%%%%%%%%%%%%%%%%%%%%%%%%%%%%%%%%%%%%%%%%%%%%
\section{Challenges in simulations of elastic wave propagation phenomenon}
%%%%%%%%%%%%%%%%%%%%%%%%%%%%%%%%%%%%%%%%%%%%%%
\section{Excitation and measurement techniques}
%%%%%%%%%%%%%%%%%%%%%%%%%%%%%%%%%%%%%%%%%%%%%%
\begin{frame}{Piezoelectricity}
%%%%%%%%%%%%%%%%%%%%%%%%%%%%%%%%%%%%%%%%%%%%%%
	\begin{columns}[T]
		\begin{column}{0.5\textwidth}
			\begin{figure}
				\animategraphics[autoplay,loop,width=0.7\textwidth]{3}{piezoelectricity_effect/piezoelectricity_effect_}{1}{25}
				\caption{Source: \url{https://commons.wikimedia.org/wiki/File:SchemaPiezo.gif }}
			\end{figure}
		\end{column}
		\begin{column}{0.5\textwidth}
		Piezoelectricity is the electric charge that accumulates in certain solid materials (such as crystals, certain ceramics,) in response to applied mechanical stress.	\\
		Lead zirconate titanate\\
		Pb[Zr\textsubscript{x}Ti\textsubscript{1-x}]O\textsubscript{3} \\
		 more commonly known as PZT, is the most commonly used piezoelectric ceramic today
		\end{column}
	\end{columns}		
\end{frame}
%%%%%%%%%%%%%%%%%%%%
\note{
}
%%%%%%%%%%%%%%%%%%%%%%%%%%%%%%%%%%%%%%%%%%%%%%
\begin{frame}[t]{SLDV measurements}
%%%%%%%%%%%%%%%%%%%%%%%%%%%%%%%%%%%%%%%%%%%%%%
	\begin{columns}[T]
		\column{0.5\textwidth}
		\begin{figure}
			\includegraphics[width=0.8\textwidth]{wibrometr-laserowy-1d_small-description.png}
		\end{figure}
		\column{0.5\textwidth}
		\begin{enumerate}
			\item Signal generator: TTI 1241 
			\item Amplifier: Piezo Systems EPA-104-230 $\pm$200 Vp
			\item Specimen
			\item Scanning head: Polytec PSV-400
			\item DAQ system: Polytec
		\end{enumerate}
	\end{columns}
	{\small
		Vibrometer allows measurements of vibration velocities in range 0.01~$\mu$m/s $-$ 10 m/s for frequencies from DC up to 1.5~MHz for measuring distance from 40~cm up to dozens of meters. Scanning resolution is 0.002$^{\circ}$  which provides possibility of defining about 300 000 points in laser working area.}
\end{frame}
%%%%%%%%%%%%%%%%%%%%
\note{
This picture shows the elements of the review of the Lamba wave propagation measurement. These are signal generators, amplifier, sample, laser vibrometer head and analogue to digital converters. 
The laser vibrometer measures the vibration velocities along the laser beam. 
Theoretically, vibration measurement up to 1.5 MHz is included. 
In practice it depends on many parameters such as vibration amplitude, type of decoder, distance of the laser from the sample, attenuation of the material, etc.
}
%%%%%%%%%%%%%%%%%%%%%%%%%%%%%%%%%%%%%%%%%%%%%%
\section{Assignment}
%%%%%%%%%%%%%%%%%%%%%%%%%%%%%%%%%%%%%%%%%%%%%%
\begin{frame}{Assignment}
%%%%%%%%%%%%%%%%%%%%%%%%%%%%%%%%%%%%%%%%%%%%%%
Answer to the following questions:
\begin{enumerate}	
  \item What is the difference between wave propagation and vibrations?
  \item What is the difference between bulk waves and Lamb waves?
\end{enumerate}
\end{frame}
%%%%%%%%%%%%%%%%%%%%
\note{
}
%%%%%%%%%%%%%%%%%%%%%%%%%%%%%%%%%%%%%%%%%%%%%%
%%%%%%%%%%%%%%%%%%%%%%%%%%%%%%%%%%%%%%%%%%%%%%
\begin{frame}{References}
%%%%%%%%%%%%%%%%%%%%%%%%%%%%%%%%%%%%%%%%%%%%%%
\begin{biblio}{Recommended books}
	\biblioref{Rose J.L.}{1999}{ Ultrasonic Waves in Solid Media}{Cambridge University Press}
	\biblioref{Doyle J.F.}{1997}{ Wave Propagation in Structures: Spectral Analysis Using Fast Discrete Fourier Transforms}{Springer}
	\biblioref{Ostachowicz W., Kudela P., Krawczuk M., Zak A.}{2012}{ Guided Waves in Structures for SHM: The Time Domain Spectral Element Method}{Wiley}
\end{biblio}
\end{frame}
%%%%%%%%%%%%%%%%%%%%%%%%%%%%%%%%%%%%%%%%%%%%%%
{\setbeamercolor{palette primary}{fg=black, bg=white}
\begin{frame}[standout]
  Thank you for your attention!\\ \vspace{12pt}
  Questions?\\ \vspace{12pt}
  \url{pk@imp.gda.pl}
\end{frame}
}
\note{Thank you for your attention!
I hope that you have learnt something from my presentation.
So, if you have any questions please drop me a line.
See you next time!}
%%%%%%%%%%%%%%%%%%%%%%%%%%%%%%%%%%%%%%%%%%%%%%
\begin{frame}{Empty}
%%%%%%%%%%%%%%%%%%%%%%%%%%%%%%%%%%%%%%%%%%%%%%
	
\end{frame}
%%%%%%%%%%%%%%%%%%%%
\note{
}
%%%%%%%%%%%%%%%%%%%%%%%%%%%%%%%%%%%%%%%%%%%%%%
\begin{frame}{Empty 2-columns}
%%%%%%%%%%%%%%%%%%%%%%%%%%%%%%%%%%%%%%%%%%%%%%
	\begin{columns}[T]
		\begin{column}{0.5\textwidth}
			
		\end{column}
		\begin{column}{0.5\textwidth}
			
		\end{column}
	\end{columns}		
\end{frame}
%%%%%%%%%%%%%%%%%%%%
\note{
}
%%%%%%%%%%%%%%%%%%%%%%%%%%%%%%%%%%%%%%%%%%%%%%
% END OF SLIDES
%%%%%%%%%%%%%%%%%%%%%%%%%%%%%%%%%%%%%%%%%%%%%%
\end{document}