\documentclass[10pt,aspectratio=169,dvipsnames]{beamer} % aspect ratio 16:9
%\graphicspath{{../../figures/}}

%\includeonlyframes{frame1,frame2,frame3}

%%%%%%%%%%%%%%%%%%%%%%%%%%%%%%%%%%%%%%%%%%%%%%%%%%
% Packages
%%%%%%%%%%%%%%%%%%%%%%%%%%%%%%%%%%%%%%%%%%%%%%%%%%
\usepackage{appendixnumberbeamer}
\usepackage{booktabs}
\usepackage{csvsimple} % for csv read
\usepackage[scale=2]{ccicons}
\usepackage{pgfplots}
\usepackage{xspace}
\usepackage{amsmath}
\usepackage{totcount}
\usepackage{tikz}
\usepackage{bm}
\usepackage{float}
\usepackage{eso-pic} 
\usepackage{wrapfig}
\usepackage{animate,media9}
\usepackage{subfig}
\usepackage{fancybox}
%\usepackage{multimedia}
\usepackage{dashbox}
\usepackage{tcolorbox}
\usepackage{multicol}
\usepackage{multirow}
\usepackage{xcolor}
\usepackage[document]{ragged2e}
\usepackage{caption}

%\usepackage{FiraSans}

\usepackage{comment}
%\usetikzlibrary{external} % speedup compilation
%\tikzexternalize % activate!
%\usetikzlibrary{shapes,arrows} 

%\usepackage{bibentry}
%\nobibliography*
\usepackage{ifthen}
\newcounter{angle}
\setcounter{angle}{0}
%\usepackage{bibentry}
%\nobibliography*
\usepackage{caption}%

\graphicspath{{figures/}}

\captionsetup[figure]{labelformat=empty}%
\usefonttheme{structurebold}
%%%%%%%%%%%%%%%%%%%%%%%%%%%%%%%%%%%%%%%%%%%%%%%%%%
% Metropolis theme custom modification file
%%%%%%%%%%%%%%%%%%%%%%%%%%%%%%%%%%%%%%%%%%%%%%%%%%
% Metropolis theme custom modification file
%%%%%%%%%%%%%%%%%%%%%%%%%%%%%%%%%%%%%%%%%%%%%%%%%%
% Metropolis theme custom colors
%%%%%%%%%%%%%%%%%%%%%%%%%%%%%%%%%%%%%%%%%%%%%%%%%%
\usetheme[progressbar=foot]{metropolis}
\useoutertheme{metropolis}
\useinnertheme{metropolis}
\usefonttheme{metropolis}
\setbeamercolor{background canvas}{bg=white}

%\usecolortheme{spruce}

\definecolor{myblue}{rgb}{0.19,0.55,0.91}
\definecolor{mediumblue}{rgb}{0,0,205}
\definecolor{darkblue}{rgb}{0,0,139}
\definecolor{Dodgerblue}{HTML}{1E90FF}
\definecolor{Navy}{HTML}{000080} % {rgb}{0,0,128}
\definecolor{Aliceblue}{HTML}{F0F8FF}
\definecolor{Lightskyblue}{HTML}{87CEFA}
\definecolor{logoblue}{RGB}{1,67,140}
\definecolor{Purple}{HTML}{911146}
\definecolor{Orange}{HTML}{CF4A30}

\setbeamercolor{progress bar}{bg=Lightskyblue}
\setbeamercolor{progress bar}{ fg=logoblue} 
\setbeamercolor{frametitle}{bg=logoblue}
\setbeamercolor{title separator}{fg=logoblue}
\setbeamercolor{block title}{bg=Lightskyblue!30,fg=black}
\setbeamercolor{block body}{bg=Lightskyblue!15,fg=black}
\setbeamercolor{alerted text}{fg=Purple}
% notes colors
\setbeamercolor{note page}{bg=white}
\setbeamercolor{note title}{bg=Lightskyblue}
%%%%%%%%%%%%%%%%%%%%%%%%%%%%%%%%%%%%%%%%%%%%%%%%%%
%  Theme modifications
%%%%%%%%%%%%%%%%%%%%%%%%%%%%%%%%%%%%%%%%%%%%%%%%%%
% modify progress bar linewidth
\makeatletter
\setlength{\metropolis@progressinheadfoot@linewidth}{2pt} 
\setlength{\metropolis@titleseparator@linewidth}{1pt}
\setlength{\metropolis@progressonsectionpage@linewidth}{1pt}

\setbeamertemplate{progress bar in section page}{
	\setlength{\metropolis@progressonsectionpage}{%
		\textwidth * \ratio{\thesection pt}{\totvalue{totalsection} pt}%
	}%
	\begin{tikzpicture}
		\fill[bg] (0,0) rectangle (\textwidth, 
		\metropolis@progressonsectionpage@linewidth);
		\fill[fg] (0,0) rectangle (\metropolis@progressonsectionpage, 
		\metropolis@progressonsectionpage@linewidth);
	\end{tikzpicture}%
}
\makeatother
\newcounter{totalsection}
\regtotcounter{totalsection}

\AtBeginDocument{%
	\pretocmd{\section}{\refstepcounter{totalsection}}{\typeout{Yes, prepending 
	was successful}}{\typeout{No, prepending was not successful}}%
}%
%%%%%%%%%%%%%%%%%%%%%%%%%%%%%%%%%%%%%%%%%%%%%%%%%%
%  Bibliography mods
%%%%%%%%%%%%%%%%%%%%%%%%%%%%%%%%%%%%%%%%%%%%%%%%%%
\setbeamertemplate{bibliography item}{\insertbiblabel} %% Remove book symbol 
%%from references and add number in square brackets
% kill the abominable icon (without number)
%\setbeamertemplate{bibliography item}{}
%\makeatletter
%\renewcommand\@biblabel[1]{#1.} % number only
%\makeatother
% remove line breaks in bibliography
\setbeamertemplate{bibliography entry title}{}
\setbeamertemplate{bibliography entry location}{}
%%%%%%%%%%%%%%%%%%%%%%%%%%%%%%%%%%%%%%%%%%%%%%%%%%
%  Bibliography custom commands
%%%%%%%%%%%%%%%%%%%%%%%%%%%%%%%%%%%%%%%%%%%%%%%%%%
\newcommand{\bibliotitlestyle}[1]{\textbf{#1}\par}

\newif\ifinbiblio
\newcounter{bibkey}
\newenvironment{biblio}[2][long]{%
	%\setbeamertemplate{bibliography item}{\insertbiblabel}
	\setbeamertemplate{bibliography item}{}% without numbers
	\setbeamerfont{bibliography item}{size=\footnotesize}
	\setbeamerfont{bibliography entry author}{size=\footnotesize}
	\setbeamerfont{bibliography entry title}{size=\footnotesize}
	\setbeamerfont{bibliography entry location}{size=\footnotesize}
	\setbeamerfont{bibliography entry note}{size=\footnotesize}
	\ifx!#2!\else%
	\bibliotitlestyle{#2}%
	\fi%
	\begin{thebibliography}{}%
		\inbibliotrue%
		\setbeamertemplate{bibliography entry title}[#1]%
	}{%
		\inbibliofalse%
		\setbeamertemplate{bibliography item}{}%
	\end{thebibliography}%
}

\newcommand{\biblioref}[5][short]{
	\setbeamertemplate{bibliography entry title}[#1]
	\stepcounter{bibkey}%
	\ifinbiblio%
	\bibitem{\thebibkey}%
	#2
	\newblock #4
	\ifx!#5!\else\newblock {\em #5}, #3 \fi%
	\else%
	\begin{biblio}{}
		\bibitem{\thebibkey}
		#2
		\newblock #4
		\ifx!#5!\else\newblock {\em #5}, #3\fi
	\end{biblio}
	\fi
}
%
%\newbibmacro*{hypercite}{%
%	\renewcommand{\@makefntext}[1]{\noindent\normalfont##1}%
%	\footnotetext{%
%		\blxmkbibnote{foot}{%
%			\printtext[labelnumberwidth]{%
%				\printfield{prefixnumber}%
%				\printfield{labelnumber}}%
%			\addspace
%			\fullcite{\thefield{entrykey}}}}}
%
%\DeclareCiteCommand{\hypercite}%
%{\usebibmacro{cite:init}}
%{\usebibmacro{hypercite}}
%{}
%{\usebibmacro{cite:dump}}
%
%% Redefine the \footfullcite command to use the reference number
%\renewcommand{\footfullcite}[1]{\cite{#1}\hypercite{#1}}
%\usefonttheme[onlymath]{Serif} % It should be uncommented if Fira fonts in 
%%math does not work

%%%%%%%%%%%%%%%%%%%%%%%%%%%%%%%%%%%%%%%%%%%%%%%%%%
% Custom commands
%%%%%%%%%%%%%%%%%%%%%%%%%%%%%%%%%%%%%%%%%%%%%%%%%%
% matrix command 
\newcommand{\matr}[1]{\mathbf{#1}} % bold upright (Elsevier, Springer)
%\newcommand{\matr}[1]{#1}   % pure math version
%\newcommand{\matr}[1]{\bm{#1}}  % ISO complying version
% vector command 
\newcommand{\vect}[1]{\mathbf{#1}} % bold upright (Elsevier, Springer)
% bold symbol
\newcommand{\bs}[1]{\boldsymbol{#1}}
% derivative upright command
\DeclareRobustCommand*{\drv}{\mathop{}\!\mathrm{d}}
\newcommand{\ud}{\mathrm{d}}
% 
\newcommand{\themename}{\textbf{\textsc{metropolis}}\xspace}

%%%%%%%%%%%%%%%%%%%%%%%%%%%%%%%%%%%%%%%%%%%%%%%%%%
% Title page options
%%%%%%%%%%%%%%%%%%%%%%%%%%%%%%%%%%%%%%%%%%%%%%%%%%
% \date{\today}
\date{}
%%%%%%%%%%%%%%%%%%%%%%%%%%%%%%%%%%%%%%%%%%%%%%%%%%
% option 1
%%%%%%%%%%%%%%%%%%%%%%%%%%%%%%%%%%%%%%%%%%%%%%%%%%%
\title{FEASIBILITY STUDY OF ARTIFICIAL INTELLIGENCE APPROACH FOR DELAMINATION IDENTIFICATION IN COMPOSITE LAMINATES}
%\subtitle{In preparation for a Ph.D. defence}
\author{\textbf{D.Sc. Ph.D. Eng. Paweł Kudela} \and \\ \textbf{Ph.D. candidate Eng. Abdalraheem A. Ijjeh }
}
% logo align to Institute 
\institute{Institute of Fluid Flow Machinery \\ 
	Polish Academy of Sciences \\ 
	\vspace{-1.5cm}
	\flushright 
	\includegraphics[width=6cm]{imp_logo.png}}
%%%%%%%%%%%%%%%%%%%%%%%%%%%%%%%%%%%%%%%%%%%%%%%%%%
% option 2 - authors in one line
%%%%%%%%%%%%%%%%%%%%%%%%%%%%%%%%%%%%%%%%%%%%%%%%%%
%	\title{My fancy title}
%	\subtitle{Lamb-opt}
%	\author{\textbf{Paweł Kudela}\textsuperscript{2}, Maciej 
	%	Radzieński\textsuperscript{2}, Wiesław Ostachowicz\textsuperscript{2}, 
	%	Zhibo Yang\textsuperscript{1} }
%	 logo align to Institute 
%	\institute{\textsuperscript{1}Xi'an Jiaotong University \\ 
	%	\textsuperscript{2}Institute of Fluid Flow Machinery\\ \hspace*{1pt} Polish 
	%	Academy of Sciences \\ \vspace{-1.5cm}\flushright 	
	%	\includegraphics[width=6cm]{imp_logo.png}}
%%%%%%%%%%%%%%%%%%%%%%%%%%%%%%%%%%%%%%%%%%%%%%%%%%%
% option 3 - multilogo vertical
%%%%%%%%%%%%%%%%%%%%%%%%%%%%%%%%%%%%%%%%%%%%%%%%%%
%%\title{My fancy title}
%%\subtitle{Lamb-opt}
%%	\author{\textbf{Paweł Kudela}\inst{1}, Maciej Radzieński\inst{1}, Wiesław Ostachowicz\inst{1}, Zhibo Yang\inst{2} }
%%	 logo under Institute 
%%	\institute%
%%	{ 
	%%		\inst{1}%
	%%		Institute of Fluid Flow Machinery\\ \hspace*{1pt} Polish Academy of Sciences \\ \includegraphics[height=0.85cm]{//odroid-sensors/sensors/MISD_shared/logo/logo_eng_40mm.eps} \\
	%%		\and
	%%		\inst{2}%
	%%	 Xi'an Jiaotong University \\ \includegraphics[height=0.85cm]{//odroid-sensors/sensors/MISD_shared/logo/logo_box.eps}
	%% }
% end od option 3
%%%%%%%%%%%%%%%%%%%%%%%%%%%%%%%%%%%%%%%%%%%%%%%%%%
%% option 4 - 3 Institutes and logos horizontal centered
%%%%%%%%%%%%%%%%%%%%%%%%%%%%%%%%%%%%%%%%%%%%%%%%%%
%\title{My fancy title}
%\subtitle{Lamb-opt }
%\author{\textbf{Paweł Kudela}\textsuperscript{1}, Maciej Radzieński\textsuperscript{1}, Marco Miniaci\textsuperscript{2}, Zhibo Yang\textsuperscript{3} }
%
%\institute{ 
	%\begin{columns}[T,onlytextwidth]
	%	\column{0.39\textwidth}
	%	\begin{center}
		%		\textsuperscript{1}Institute of Fluid Flow Machinery\\ \hspace*{3pt}Polish Academy of Sciences
		%	\end{center}
	%	\column{0.3\textwidth}
	%	\begin{center}
		%		\textsuperscript{2}Zurich University
		%	\end{center}
	%	\column{0.3\textwidth}
	%	\begin{center}
		%		\textsuperscript{3}Xi'an Jiaotong University
		%	\end{center}
	%\end{columns}
	%\vspace{6pt}
	%% logos 
	%\begin{columns}[b,onlytextwidth]
	%	\column{0.39\textwidth}
	%		\centering 
	%		\includegraphics[scale=0.9,height=0.85cm,keepaspectratio]{//odroid-sensors/sensors/MISD_shared/logo/logo_eng_40mm.eps}
	%	\column{0.3\textwidth}
	%		\centering 
	%		\includegraphics[scale=0.9,height=0.85cm,keepaspectratio]{//odroid-sensors/sensors/MISD_shared/logo/logo_box.eps}
	%	\column{0.3\textwidth}
	%		\centering 
	%		\includegraphics[scale=0.9,height=0.85cm,keepaspectratio]{//odroid-sensors/sensors/MISD_shared/logo/logo_box2.eps}
	%\end{columns}
	%}
%\makeatletter
%\setbeamertemplate{title page}{
	%	\begin{column}[b][\paperheight]{\textwidth}
		%		\centering % <-- Center here
		%		\ifx\inserttitlegraphic\@empty\else\usebeamertemplate*{title graphic}\fi
		%		\vfill%
		%		\ifx\inserttitle\@empty\else\usebeamertemplate*{title}\fi
		%		\ifx\insertsubtitle\@empty\else\usebeamertemplate*{subtitle}\fi
		%		\usebeamertemplate*{title separator}
		%		\ifx\beamer@shortauthor\@empty\else\usebeamertemplate*{author}\fi
		%		\ifx\insertdate\@empty\else\usebeamertemplate*{date}\fi
		%		\ifx\insertinstitute\@empty\else\usebeamertemplate*{institute}\fi
		%		\vfill
		%		\vspace*{1mm}
		%	\end{column}
	%}
%
%\setbeamertemplate{title}{
	%	% \raggedright% % <-- Comment here
	%	\linespread{1.0}%
	%	\inserttitle%
	%	\par%
	%	\vspace*{0.5em}
	%}
%\setbeamertemplate{subtitle}{
	%	% \raggedright% % <-- Comment here
	%	\insertsubtitle%
	%	\par%
	%	\vspace*{0.5em}
	%}
%\makeatother
% end of option 4
%%%%%%%%%%%%%%%%%%%%%%%%%%%%%%%%%%%%%%%%%%%%%%%%%%
% option 5 - 2 Institutes and logos horizontal centered
%%%%%%%%%%%%%%%%%%%%%%%%%%%%%%%%%%%%%%%%%%%%%%%%%%
%\title{My fancy title}
%\subtitle{Lamb-opt }
%\author{\textbf{Paweł Kudela}\textsuperscript{1}, Maciej Radzieński\textsuperscript{1}, Marco Miniaci\textsuperscript{2}}
%
%\institute{ 
	%	\begin{columns}[T,onlytextwidth]
		%		\column{0.5\textwidth}
		%			\centering
		%			\textsuperscript{1}Institute of Fluid Flow Machinery\\ \hspace*{3pt}Polish Academy of Sciences
		%		\column{0.5\textwidth}
		%			\centering
		%			\textsuperscript{2}Zurich University
		%	\end{columns}
	%	\vspace{6pt}
	%	% logos 
	%	\begin{columns}[b,onlytextwidth]
		%		\column{0.5\textwidth}
		%		\centering 
		%		\includegraphics[scale=0.9,height=0.85cm,keepaspectratio]{//odroid-sensors/sensors/MISD_shared/logo/logo_eng_40mm.eps}
		%		\column{0.5\textwidth}
		%		\centering 
		%		\includegraphics[scale=0.9,height=0.85cm,keepaspectratio]{//odroid-sensors/sensors/MISD_shared/logo/logo_box.eps}
		%	\end{columns}
	%}
%\makeatletter
%\setbeamertemplate{title page}{
	%	\begin{column}[b][\paperheight]{\textwidth}
		%		\centering % <-- Center here
		%		\ifx\inserttitlegraphic\@empty\else\usebeamertemplate*{title graphic}\fi
		%		\vfill%
		%		\ifx\inserttitle\@empty\else\usebeamertemplate*{title}\fi
		%		\ifx\insertsubtitle\@empty\else\usebeamertemplate*{subtitle}\fi
		%		\usebeamertemplate*{title separator}
		%		\ifx\beamer@shortauthor\@empty\else\usebeamertemplate*{author}\fi
		%		\ifx\insertdate\@empty\else\usebeamertemplate*{date}\fi
		%		\ifx\insertinstitute\@empty\else\usebeamertemplate*{institute}\fi
		%		\vfill
		%		\vspace*{1mm}
		%	\end{column}
	%}
%
%\setbeamertemplate{title}{
	%	% \raggedright% % <-- Comment here
	%	\linespread{1.0}%
	%	\inserttitle%
	%	\par%
	%	\vspace*{0.5em}
	%}
%\setbeamertemplate{subtitle}{
	%	% \raggedright% % <-- Comment here
	%	\insertsubtitle%
	%	\par%
	%	\vspace*{0.5em}
	%}
%\makeatother
% end of option 5
%
%%%%%%%%%%%%%%%%%%%%%%%%%%%%%%%%%%%%%%%%%%%%%%%%%%
% End of title page options
%%%%%%%%%%%%%%%%%%%%%%%%%%%%%%%%%%%%%%%%%%%%%%%%%%
% logo option - alternative manual insertion by modification of coordinates in \put()
%\titlegraphic{%
	%	%\vspace{\logoadheight}
	%	\begin{picture}(0,0)
		%	\put(305,-185){\makebox(0,0)[rb]{\includegraphics[width=4cm]{//odroid-sensors/sensors/MISD_shared/logo/logo_eng_40mm.eps}}}
		%	\end{picture}}
%
%%%%%%%%%%%%%%%%%%%%%%%%%%%%%%%%%%%%%%%%%%%%%%%%%%
%\tikzexternalize % activate!
%%%%%%%%%%%%%%%%%%%%%%%%%%%%%%%%%%%%%%%%%%%%%%%%%%
\setbeamertemplate{section in toc}[sections numbered]
\setbeamertemplate{subsection in toc}[subsections numbered]

\begin{document}
	%%%%%%%%%%%%%%%%%%%%%%%%%%%%%%%%%%%%%%%%%%%%%%%%%%
	\maketitle
	%%%%%%%%%%%%%%%%%%%%%%%%%%%%%%%%%%%%%%%%%%%%%%%%%%
	% SLIDES
	%%%%%%%%%%%%%%%%%%%%%%%%%%%%%%%%%%%%%%%%%%%%%%%%%%
	\begin{frame}[label=frame1]{Outlines}
		\begin{multicols}{2}
			%		\fontsize{6pt}{8pt}\selectfont
			\setbeamertemplate{section in toc}[sections numbered]
			\setbeamertemplate{subsection in toc}[subsections numbered]
			\tableofcontents
		\end{multicols}
	\end{frame}
	%%%%%%%%%%%%%%%%%%%%%%%%%%%%%%%%%%%%%%%%%%%%%%%%%%%%%%%%%%%%%%%%%%%%%%%%%%%%%%%%
	\section{Composite laminates}
	%%%%%%%%%%%%%%%%%%%%%%%%%%%%%%%%%%%%%%%%%%%%%%%%%%%%%%%%%%%%%%%%%%%%%%%%%%%%%%%%
	%\subsection{Composite laminates}
	\begin{frame}{What are Composite Laminates?}
		\begin{columns}[T]
			\begin{column}[c]{.3\textwidth}
				\small
				\begin{itemize}
					\justifying
					%			\item Composite laminates are usually composed of plies with different directions.
					\item A \textcolor{blue}{ply} is made up of unidirectional continuous \textcolor{blue}{Fibres} held together by a \textcolor{blue}{Polymer matrix such as Epoxy resin} .			
					\item The lay-up of the plies depends on the anticipated loading of the structure where the laminate will be used.
				\end{itemize}
			\end{column}
			\hfill
			\begin{column}[c]{.65\textwidth}
				\begin{figure}
					\includegraphics[width=.95\textwidth]{compsite_laminates.png}
				\end{figure}
			\end{column}
		\end{columns}
	\end{frame}
	%%%%%%%%%%%%%%%%%%%%%%%%%%%%%%%%%%%%%%%%%%%%%%%%%%%%%%%%%%%%%%%%%%%%%%%%%%%%%%%%
	\begin{frame}{Advantages \& applications}
		\begin{columns}[T]
			\begin{column}[t]{.45\textwidth}
				\textbf{Some of the favorable characteristics of utilising composite laminates:}
				\begin{itemize}
					\item \alert{High impact damage resistance}
					\item \alert{Light weight}
					\item \alert{High strength-to-weight ratio}
					\item \alert{Low density}
					\item \alert{resistance to fatigue and corrosion}
				\end{itemize}
			\end{column}
			\begin{column}[t]{.45\textwidth}
				\textbf{Some applications of Composite laminates:}
				\begin{itemize}
					\item \alert{Aerospace structures}
					\item \alert{Automotive}
					\item \alert{Wind turbines}
					\item \alert{Pipes and tanks}
				\end{itemize}
			\end{column}
		\end{columns}
	\end{frame}
	%%%%%%%%%%%%%%%%%%%%%%%%%%%%%%%%%%%%%%%%%%%%%%%%%%%%%%%%%%%%%%%%%%%%%%%%%%%%%%%%
	\begin{frame}{Defects in composite laminates}
		\small
		Composite laminates can have different types of damage such as: \\
		\textbf{Cracks, fibre breakage, debonding, and \alert{delamination}}.
		\begin{columns}[T]
			\begin{column}[c]{.45\textwidth}
				\begin{itemize}
					\footnotesize
					\item Delamination is a critical failure mechanism in laminated fibre-reinforced polymer matrix composites.
					\item Delamination is one of the most hazardous forms of the defects. 
					It leads to very catastrophic failures if not detected at early stages.
				\end{itemize}
			\end{column}
			\begin{column}[c]{0.50\textwidth}
				\begin{figure}
					\includegraphics[width=.95\textwidth]{delaminated_plate1.jpg}
				\end{figure}
			\end{column}
		\end{columns}
	\end{frame}
	%%%%%%%%%%%%%%%%%%%%%%%%%%%%%%%%%%%%%%%%%%%%%%%%%%%%%%%%%%%%%%%%%%%%%%%%%%%%%%%%
	
	%%%%%%%%%%%%%%%%%%%%%%%%%%%%%%%%%%%%%%%%%%%%%%%%%%%%%%%%%%%%%%%%%%%%%%%%%%%%%%%%
	\section{SHM/NDE}
	%%%%%%%%%%%%%%%%%%%%%%%%%%%%%%%%%%%%%%%%%%%%%%%%%%%%%%%%%%%%%%%%%%%%%%%%%%%%%%%%
	\begin{frame}{Structural Health Monitoring (SHM)}
		\begin{figure}
			\includegraphics[height=.8\textheight]{SHM_system.png}
		\end{figure}
	\end{frame}
	%%%%%%%%%%%%%%%%%%%%%%%%%%%%%%%%%%%%%%%%%%%%%%%%%%%%%%%%%%%%%%%%%%%%%%%%%%%%%%%%
	
	\begin{frame}{Non Destructive Evaluations (NDEs)}
		\begin{columns}[T]
			\only<1>{
				\begin{column}[c]{0.32\textwidth}	
					\begin{itemize}
						\item \textbf{Visual inspection}
						\item \textbf{Eddy current}
						\item \textbf{Dye penetration}
						\item \textbf{Acoustic emission}
						\item \alert{\textbf{Ultrasonic testing}} \alert{\textbf{(local NDEs)}}
						\item \alert{\textbf{Guided wave testing (global NDEs)}}
					\end{itemize}
			\end{column}}
			\only<2->{
				\begin{column}[c]{0.32\textwidth}
					\begin{itemize}
						\item[$\times$] \alert{Labor intensive}
						\item[$\times$] \alert{Time consuming}
						\item[$\times$] \alert{Need professional and experienced personnel}
						\item[$\times$] \alert{May require the disassembly of complex structures}
					\end{itemize}
					\alert{\textbf{Local NDEs are not practical and do not fit for SHM}}
			\end{column}}
			\begin{column}[c]{0.65\textwidth}				
				\textbf{Ultrasonic testing} \hspace{50pt} \textbf{Guided wave testing}
				\begin{figure}
					\includegraphics[width=0.95\textwidth]{local_ultrasonic.png}
				\end{figure}
			\end{column}		
		\end{columns}	
	\end{frame}
	\begin{frame}{Ultrasonic testing vs guided wave testing}
		\alert{Bulk waves} exist in infinite homogeneous bodies and propagate indefinitely without being interrupted by boundaries or interfaces. 
		These waves can be decomposed into infinite plane waves propagating along arbitrary direction within the solid.
		
		\alert{Guided waves} are those waves that require a boundary for their existence, such as surface waves, Lamb waves, and interface waves.
		\vspace{5mm}
		\begin{columns}[T]
			\begin{column}{0.5\textwidth}
				\textbf{Ultrasonic waves}	
				\begin{itemize}
					\item Frequency range: 2 MHz - 200 MHz
					\item Wavelength \(\lambda << h\) thickness 
					\item shorter wavelengths
				\end{itemize}
			\end{column}
			\begin{column}{0.5\textwidth}
				\textbf{Guided waves}	
				\begin{itemize}
					\item Typical frequency range: 10 kHz - 1 MHz
					\item Wavelength \(\lambda > h\) thickness 
					\item longer wavelengths
				\end{itemize}
			\end{column}
		\end{columns}			
	\end{frame}
	%%%%%%%%%%%%%%%%%%%%%%%%%%%%%%%%%%%%%%%%%%%%%%%%%%%%%%%%%%%%%%%%%%%%%%%%%%%%%%%%
	\section{Guided waves}
	%%%%%%%%%%%%%%%%%%%%%%%%%%%%%%%%%%%%%%%%%%%%%%%%%%%%%%%%%%%%%%%%%%%%
	%%%%%%%%%%%%%%%%%%%%%%%%%%%%%%%%%%%%%%%%%%%%%%
	%\begin{frame}{Waves used in non-destructive testing}
	%	%%%%%%%%%%%%%%%%%%%%%%%%%%%%%%%%%%%%%%%%%%%%%%
	%	Elastic wave propagation types depending on particle motion:
	%	\begin{itemize}
		%		\item  \alert{The longitudinal wave} is a compressional wave in which the particle motion is in the same direction as the propagation of the wave
		%		\item \alert{The shear wave} is a wave motion in which the particle motion is perpendicular to the direction of the propagation
		%		\item \alert{Surface (Rayleigh) waves} have an elliptical particle motion and travel across the surface of a material. Their velocity is approximately 90\% of the shear wave velocity of the material and their depth of penetration is approximately equal to one
		%		wavelength
		%		\item \alert{Plate (Lamb) waves} have a complex vibration occurring in materials where thickness is less than the wavelength of elastic wave introduced into it.
		%	\end{itemize}
	%\end{frame}
	%%%%%%%%%%%%%%%%%%%%%%%%%%%%%%%%%%%%%%%%%%%%%%%%%%%
	%\setcounter{subfigure}{0}
	%\begin{frame}{Waves used in non-destructive testing}
	%	%%%%%%%%%%%%%%%%%%%%%%%%%%%%%%%%%%%%%%%%%%%%%%%%%%
	%	\begin{figure}
		%		\subfloat{\animategraphics[autoplay,loop, controls,width=0.5\textwidth]{10}{figures/gif_figs/Longitudinal_wave/Longitudinal_wave-}{0}{35}}
		%		\caption{\alert{Longitudinal wave} - plane pressure pulse wave}
		%	\end{figure}
	%\tiny 
	%(source: https://nojigon.webs.upv.es/index.php)
	%\end{frame}
	%%%%%%%%%%%%%%%%%%%%%%%%%%%%%%%%%%%%%%%%%%%%%%%%%%%%%%%%%%%%%%%%%%%%%%%%%%%%%%%%%
	%%%%%%%%%%%%%%%%%%%%%%%%%%%%%%%%%%%%%%%%%%%%%%%%%%%
	%\setcounter{subfigure}{0}
	%\begin{frame}{Waves used in non-destructive testing}
	%	%%%%%%%%%%%%%%%%%%%%%%%%%%%%%%%%%%%%%%%%%%%%%%%%%%
	%	\begin{columns}[T]
		%		\begin{column}{0.5\textwidth}
			%			\centering
			%			\begin{figure}
				%				\subfloat{\animategraphics[autoplay,loop, controls,width=0.95\textwidth]{10}{figures/gif_figs/SH_shear_wave/SH_shear-}{0}{39}}
				%				\caption{\alert{Shear horizontal wave}}
				%			\end{figure}			
			%		\end{column}
		%		\begin{column}{0.5\textwidth}
			%			\centering
			%			\begin{figure}
				%				\subfloat{\animategraphics[autoplay,loop, controls,width=0.95\textwidth]{10}{figures/gif_figs/SV_shear_wave/SV_shear-}{0}{39}}
				%				\caption{\alert{Shear vertical wave}}
				%			\end{figure}			
			%		\end{column}	
		%	\end{columns}
	%\tiny 
	%(source: https://nojigon.webs.upv.es/index.php)
	%\end{frame}
	%%%%%%%%%%%%%%%%%%%%%%%%%%%%%%%%%%%%%%%%%%%%%%%%%%%
	%\setcounter{subfigure}{0}
	%\begin{frame}{Waves used in non-destructive testing}
	%	%%%%%%%%%%%%%%%%%%%%%%%%%%%%%%%%%%%%%%%%%%%%%%%%%%
	%	\begin{figure}
		%		\centering
		%		\subfloat{\animategraphics[autoplay,loop, controls,height=0.65\textheight]{15}{figures/gif_figs/Raileigh_wave/Raileigh_wave-}{0}{67}}
		%		\caption{\alert{Rayleigh waves}}		
		%	\end{figure}			
	%\tiny 
	%(source: https://nojigon.webs.upv.es/index.php)
	%\end{frame}
	%%%%%%%%%%%%%%%%%%%%%%%%%%%%%%%%%%%%%%%%%%%%%%%
	%\setcounter{subfigure}{0}
	%\begin{frame}{Waves used in non-destructive testing}
	%	%%%%%%%%%%%%%%%%%%%%%%%%%%%%%%%%%%%%%%%%%%%%%%
	%	\begin{figure}			
		%		\centering
		%		\subfloat{\animategraphics[autoplay,loop, controls,height=0.65\textheight]{20}{figures/gif_figs/love_wave/love_wave-}{0}{107}}
		%		\caption{\alert{Love waves} (surface seismic waves) named after Augustus Edward Hough Love}		
		%	\end{figure}			
	%\tiny 
	%(source: https://nojigon.webs.upv.es/index.php)
	%\end{frame}
	%%%%%%%%%%%%%%%%%%%%%%%%%%%%%%%%%%%%%%%%%%%%%%
	\setcounter{subfigure}{0}
	\begin{frame}{Lamb waves}
		%%%%%%%%%%%%%%%%%%%%%%%%%%%%%%%%%%%%%%%%%%%%%%
		\begin{alertblock}{Lamb waves}	
			Lamb waves are plane waves propagating in thin plates.\\
			Shear vertical waves in conjunction with longitudinal P waves interacts with plate surfaces resulting in complex wave mechanism which leads to creation of Lamb waves.
		\end{alertblock}
		Horace Lamb discovered these type of waves in 1917.
		He derived theory and dispersion relations.
		\begin{columns}[T]
			\begin{column}{0.5\textwidth}
				\centering
				symmetric modes
				\begin{equation*}
					\frac{\tan(q h)}{\tan(p h)} = -\frac{4 k^2 p q}{\left(q^2 - k^2\right)^2}
				\end{equation*}
			\end{column}
			\begin{column}{0.5\textwidth}
				\centering
				antisymmetric modes
				\begin{equation*}
					\frac{\tan(q h)}{\tan(p h)} = -\frac{\left(q^2 - k^2\right)^2}{4 k^2 p q}
				\end{equation*}
			\end{column}	
		\end{columns}	
		\centering
		\(q=q(\omega,k), \quad p=p(\omega,k) \)
	\end{frame}
	%%%%%%%%%%%%%%%%%%%%%%%%%%%%%%%%%%%%%%%%%%%%%%
	\setcounter{subfigure}{0}
	\begin{frame}{Lamb waves modes}
		%%%%%%%%%%%%%%%%%%%%%%%%%%%%%%%%%%%%%%%%%%%%%%
		\begin{columns}[T]
			\begin{column}{0.3\textwidth}
				\centering
				\begin{figure}
					\animategraphics[autoplay,loop,width=1\textwidth]{10}{figures/gif_figs/S0_mode/S0_mode-}{0}{67}
					\caption{Fundamental symmetric, S0, \alert{Lamb wave} mode (in-plane motion)}
				\end{figure}
			\end{column}
			\begin{column}{0.3\textwidth}
				\centering
				\begin{figure}
					\animategraphics[autoplay,loop,width=1\textwidth]{10}{figures/gif_figs/A0_mode/A0_mode-}{0}{67}
					\caption{Fundamental antisymmetric, A0, \alert{Lamb wave} mode (out-of-plane motion)}
				\end{figure}
			\end{column}
			\only<2->{
				\begin{column}{0.37\textwidth}
					\begin{itemize}
						\item \textcolor{blue}{Travel within guides for long distances}
						\item \textcolor{blue}{Can propagate in complex structures}
						\item \textcolor{blue}{High speeds in metals and composites}
						\item \textcolor{blue}{Can be automated using software}
					\end{itemize}
					\textbf{Lamb waves are a promising global NDE solution for SHM}
			\end{column}}
		\end{columns}	
		\tiny 
		(source: https://nojigon.webs.upv.es/index.php)
	\end{frame}
	%%%%%%%%%%%%%%%%%%%%%%%%%%%%%%%%%%%%%%%%%%%%%%
	
%	\begin{frame}{Dispersion curves of Lamb waves}
%		%%%%%%%%%%%%%%%%%%%%%%%%%%%%%%%%%%%%%%%%%%%%%%
%		\begin{figure}
%			\only<1>{
%				\includegraphics[width=0.8\textwidth]{/figs/Fig_1_12.png}	
%			}
%			\only<2>{
%				\includegraphics[width=0.8\textwidth]{/figs/Fig_1_13.png}	
%			}
%		\end{figure}
%	\end{frame}
	%%%%%%%%%%%%%%%%%%%%%%%%%%%%%%%%%%%%%%%%%%%%%%%%%%%%%%%%%%%%%%%%%%%%%%%%%%%%%%%%
	\section{Damage detection approaches}
	%%%%%%%%%%%%%%%%%%%%%%%%%%%%%%%%%%%%%%%%%%%%%%%%%%%%%%%%%%%%%%%%%%%%%%%%%%%%%%%%
	\begin{frame}{Conventional approaches}
		Conventional structural damage detection methods involve two processes:
		\begin{itemize}
			\item \alert{\textbf{Feature extraction}} \(\rightarrow\) includes signals preprocessing (signal denoising and averaging), then extracting \textbf{damage indexes (DIs)}.
			\item \alert{\textbf{Feature classification}}\(\rightarrow\) the extracted features are classified, for instance, into healthy or damaged states.
		\end{itemize}
		\begin{figure}
			\centering
			\includegraphics[width=.8\textwidth]{conventional_ML.png}
		\end{figure}	
		\textbf{Drawbacks of Conventional methods:}
		\begin{itemize}
			\item[$\times$]\alert{Requires a great amount of human labor and computational effort.}
			\item[$\times$]\alert{Demands a high amount of experience of the practitioner.}
			\item[$\times$]\alert{Inefficient with big data which requires a complex computation of damage features.} 
		\end{itemize}
	\end{frame}
	
	%%%%%%%%%%%%%%%%%%%%%%%%%%%%%%%%%%%%%%%%%%%%%%%%%%%%%%%%%%%%%%%%%%%%%%%%%%%%%%%%
	\begin{frame}{Deep learning approach}
		\textbf{Deep learning (DL) technologies are in accelerating growth due to:}
		\begin{itemize}
			\item \textcolor{blue}{Exponential development in computer hardware/software industries.}
			\item \textcolor{blue}{Machine learning algorithms.}
			\item \textcolor{blue}{Era of Big data.}
		\end{itemize}	
		With \textbf{DL}, feature extraction and classification are done \alert{\textbf{automatically}} without human intervention in an \alert{\textbf{End-to-end}} scheme.
		\begin{figure}
			\includegraphics[width=.95\textwidth]{DL_approach.png}
		\end{figure}
	\end{frame}
	%%%%%%%%%%%%%%%%%%%%%%%%%%%%%%%%%%%%%%%%%%%%%%%%%%%%%%%%%%%%%%%%%%%%%%%%%%%%%%%%
	\section{Objectives}
	\begin{frame}{Objectives}
		\textbf{To develop \textcolor{blue}{a novel AI-driven diagnostic system} for delamination identification in composite laminates such as carbon fibre reinforced polymers (CFRP).}
		\vfil
		\textbf{To address the issue of \textcolor{blue}{slow data acquisition} of high-resolution full wavefields of Lamb wave propagation.}
		\begin{alertblock}{Thesis}
			It is possible to use an end-to-end approach in which DNN 
			processes the animation of propagating waves (input) directly into a damage map (output).
		\end{alertblock}
	\end{frame}
	%%%%%%%%%%%%%%%%%%%%%%%%%%%%%%%%%%%%%%%%%%%%%%%%%%%%%%%%%%%%%%%%%%%%%%%%%%%%%%%
	%\setcounter{subfigure}{0}
	%\section{Artificial intelligence, machine learning, and deep learning}
	%%%%%%%%%%%%%%%%%%%%%%%%%%%%%%%%%%%%%%%%%%%%%%%%%%%
	%%%%%%%%%%%%%%%%%%%%%%%%%%%%%%%%%%%%%%%%%%%%%%%%%%%
	%\begin{frame}{What is deep learning?}
	%	\begin{figure}
		%		\centering
		%		\includegraphics[width=0.85\textwidth]{AI_vs_ML_vs_Deep_Learning.png}
		%	\end{figure}
	%	\tiny
	%	(source: https://www.ingeniovirtual.com/)
	%\end{frame}
	%
	%%%%%%%%%%%%%%%%%%%%%%%%%%%%%%%%%%%%%%%%%%%%%%%%%%%%%%%%%%%%%%%%%%%%%%%%%%%%%%%%%
	%\setcounter{subfigure}{0}
	%%%%%%%%%%%%%%%%%%%%%%%%%%%%%%%%%%%%%%%%%%%%%%%%%%%
	%\begin{frame}{Deep learning, why now?}
	%	\begin{column}[c]{0.4\textwidth}
		%		AI technologies are in accelerating growth due to:
		%		\begin{itemize}
			%			\item Exponential development in computer hardware industries
			%			 (e.g. CPUs, GPUs, FPGAs, TPUs and ASICs)
			%			\item Era of Big data.
			%		\end{itemize}
		%	\end{column}
	%	\begin{column}[c]{0.55\textwidth}
		%		\begin{figure}
			%			\centering
			%			\subfloat{\animategraphics[autoplay,loop,width=.9\textwidth]{10}{gif_figs/gpu/gpu_-}{0}{34}}
			%		\end{figure}
		%	\tiny
		%	(source: https://www.techbooky.com/)
		%	\end{column}
	%	
	%\end{frame}
	%%%%%%%%%%%%%%%%%%%%%%%%%%%%%%%%%%%%%%%%%%%%%%%%%%%
	%\setcounter{subfigure}{0}
	%%%%%%%%%%%%%%%%%%%%%%%%%%%%%%%%%%%%%%%%%%%%%%%%%%%
	%\begin{frame}{Common learning strategies}
	%	\centering
	%	\begin{figure}
		%		\includegraphics[width=0.9\textwidth]{learning.png}
		%	\end{figure}
	%	\tiny
	%	(source: https://www.aitude.com/supervised-vs-unsupervised-vs-reinforcement/)
	%\end{frame}
	%%%%%%%%%%%%%%%%%%%%%%%%%%%%%%%%%%%%%%%%%%%%%%%%%%%%%%%%%%%%%%%%%%%%%%%%%%%%%%%%%%
	\setcounter{subfigure}{0}
	\section{Synthetic dataset generation}
	%%%%%%%%%%%%%%%%%%%%%%%%%%%%%%%%%%%%%%%%%%%%%%%%%%
	%\subsection{Synthetic Dataset of propagating Lamb waves}
	%\setcounter{subfigure}{0}
	%%%%%%%%%%%%%%%%%%%%%%%%%%%%%%%%%%%%%%%%%%%%%%%%%%
	%%%%%%%%%%%%%%%%%%%%%%%%%%%%%%%%%%%%%%%%%%%%%%%%%%%%%%%%%%%%%%%%%%%%
	\begin{frame}{Dataset description}
		\begin{columns}[T]
			\only<1-5>{\begin{column}[c]{0.48\textwidth}
					\begin{itemize}
						\only<1>
						{\item 475 delamination scenarios
						\item Delamination size min 10 mm, max  40 mm
						\item \textbf{3-months of computing}}
						\only<2>{\item \alert{Delamination geometrical size} (ellipse minor and major axis) randomly selected from [10 mm, 40 mm].}
						\only<3>{\item \alert{Delamination angle} randomly selected from the interval $[0^\circ,180^\circ]$}
						\only<4>{\item \alert{Coordinates of the centre of delamination \((x_{c},y_{c})\)} randomly selected from the interval [0 mm, 250 mm - $\delta$] and [250 mm + $\delta$, 500 mm], where $\delta$ = 10 mm.}
						\only<5>{\item Delamination was modelled between the 3rd and 4th layer
						\item CFRP is made of 8-layers}
					\end{itemize}
			\end{column}}
			\only<1-4>{\begin{column}[c]{0.48\textwidth}
				\begin{figure}					
					\centering
					\only<1>{\subfloat[All random delaminations]{\includegraphics[width=0.7\textwidth]{figure_overlap.png}}}
					\only<2-4>{\subfloat[Delamination orientation]{\includegraphics[width=0.90\textwidth]{figure1.png}}}
				\end{figure}
			\end{column}}
			\only<5>{\begin{column}[c]{0.48\textwidth}
				\begin{figure}
					\centering
					\includegraphics[width=0.8\textwidth]{delamination_placement.png}
					\caption{Delamination placement}
				\end{figure}
			\end{column}}
		\end{columns}
	\end{frame}
	%%%%%%%%%%%%%%%%%%%%%%%%%%%%%%%%%%%%%%%%%%%%%%%%%%%%%%%%%%%%%%%%%%%%
	%%%%%%%%%%%%%%%%%%%%%%%%%%%%%%%%%%%%%%%%%%%%%%%%%%%%%%%%%%%%%%%%%%%%
	\setcounter{subfigure}{0}
	\begin{frame}{Training Sample case}
		\begin{columns}[T]
			\begin{column}[c]{.32\textwidth}
				\begin{figure}
					\centering
					\animategraphics[autoplay,loop,width=0.95 \textwidth]{16}{figures/gif_figs/7_output/flat_shell_Vz_7_500x500bottom-}{1}{512}
					\caption{Full wavefield $s(x,y,t_k)$}
				\end{figure}
			\end{column}
			\begin{column}[c]{.32\textwidth}
				\begin{figure}
					\centering
					\includegraphics[width=0.95 \textwidth]{RMS_flat_shell_Vz_7_500x500bottom.png}
					\caption{RMS image $\hat{s}(x,y)$}
				\end{figure}
			\end{column}
			\begin{column}[c]{.32\textwidth}
				\begin{figure}
					\centering
					\includegraphics[width=0.95 \textwidth]{m1_rand_single_delam_7.png}
					\caption{Ground truth (label)}
				\end{figure}
			\end{column}
		\end{columns}
		The RMS is defined as:
		\begin{equation}
			\hat{s}(x,y) = \sqrt{\frac{1}{N}\sum_{k=1}^{N}s(x,y,t_k)^2}, 
			\label{eqn:rms} 
		\end{equation}
	\end{frame}
	%%%%%%%%%%%%%%%%%%%%%%%%%%%%%%%%%%%%%%%%%%%%%%%%%%%%%%
	\section{DL-based approaches for damage detection and localisation}
	\setcounter{subfigure}{0}
	\begin{frame}{What is computer vision?}
		\begin{columns}[T]
			\begin{column}[c]{0.30\textwidth}
				\justifying
				\alert{\textbf{Computer vision} is a field of AI that enables computers and systems to derive meaningful information from digital images, videos and other visual inputs.} 
			\end{column}
			\hfill
			\begin{column}[c]{0.65\textwidth}
				\begin{figure}
					\centering
					\includegraphics[width=1\textwidth]{computer_vision_tasks.png}
				\end{figure}
			\end{column}
		\end{columns}
	\end{frame}
	%%%%%%%%%%%%%%%%%%%%%%%%%%%%%%%%%%%%%%%%%%%%%%%%%%%%%%
	\subsection{Semantic segmentation}
	\begin{frame}{Image semantic segmentation}
		\begin{columns}[T]
			\begin{column}[c]{0.48\textwidth}		
				\centering		
				\textbf{One-to-one (RMS based approach)}
				\begin{figure}
					\subfloat[Single input]{\includegraphics[width=.45\textwidth]{RMS_flat_shell_Vz_381_500x500bottom.png}}\quad
					\subfloat[Single output]{\includegraphics[width=.45\textwidth]{GCN_381.png}}
				\end{figure}
			\end{column}
			\begin{column}[c]{0.48\textwidth}
				\centering
				\textbf{Many-to-one (Full wavefield frames)}
				\centering
				\begin{figure}				
					\subfloat[Full wavefield]{\animategraphics[autoplay,loop,width=.45\textwidth]{4}{figures/gif_figs/381_output/output_381-}{85}{113}}\quad
					\subfloat[Single output]{\includegraphics[width=.45\textwidth]{GCN_381.png}}
				\end{figure}
			\end{column}
		\end{columns}
	\end{frame}
	
	\subsection{Developed DL models}
	%\begin{frame}{Common deep learning architectures}
	%	
	%	\begin{column}[t]{0.45\textwidth}
		%		\textbf{RMS based}\\
		%		\begin{itemize}
			%			\item Convolutional neural networks (CNN)
			%			\item Fully convolutional network (FCN)
			%		\end{itemize}
		%	\end{column}
	%	\hfill
	%	\begin{column}[t]{0.45\textwidth}
		%		\textbf{Full wavefield frames}\\
		%		\begin{itemize}
			%			\item Recurrent neural network (RNN)
			%			\item Long short-term memory (LSTM)
			%			\item ConvLSTM
			%		\end{itemize}
		%	\end{column}
	%\end{frame}
	%%%%%%%%%%%%%%%%%%%%%%%%%%%%%%%%%%%%%%%%%%%%%%%%%%
	\begin{frame}{Developed model for delamination identification}
		\begin{columns}[T]
			\begin{column}[t]{0.45\textwidth}
				\textbf{RMS based models: }
				\medskip
				\begin{itemize}
					\item Res-UNet
					\item VGG 16 encoder-decoder
					\item FCN-DenseNet
					\item PSPNet
					\item GCN
				\end{itemize}
				\biblioref{Ijjeh, A. A., Kudela P.}{2022 Apr 1}{Deep learning based segmentation using full wavefield processing for delamination identification: A comparative study} {Mechanical Systems and Signal Processing. 168:108671}
				\biblioref{Ijjeh, A. A., Ullah, S., Kudela P.}{2021 May 15}{Full wavefield processing by using FCN for delamination detection}{Mechanical Systems and Signal Processing.  153:107537} 

			\end{column}
			\hfill
			\begin{column}[t]{.45\textwidth}
				\textbf{Full wavefield frames based model:}
				\begin{itemize}
					\item Autoencoder ConvLSTM
				\end{itemize}
				\biblioref{Ullah, S., Ijjeh, A. A., Kudela, P.}{2023}{Deep learning approach for delamination identification using animation of Lamb waves}{Engineering Applications of Artificial Intelligence, 117, 105520}
			\end{column}
		\end{columns}
	\end{frame}
	
	\setcounter{subfigure}{0}
	\subsection{RMS based models}
	
	\begin{frame}{Residual UNet}
		\begin{figure}
			\centering
			\includegraphics[width=.6\textwidth]{figure4.png}
		\end{figure}
	\end{frame}
	
	\begin{frame}{VGG16 encoder-decoder}
		\begin{figure}
			\centering
			\includegraphics[width=.6\textwidth]{figure5.png}
		\end{figure}
	\end{frame}
	
	\begin{frame}{FCN-DenseNet}
		\begin{columns}[T]
			\begin{column}[c]{0.48\textwidth}
				\begin{figure}[h!]
					\includegraphics[height=.8\textheight]{FCN_dense_net.png}
					\caption{FCN-DenseNet architecture.} 
					\label{fcn}
				\end{figure}
			\end{column}
			\hfill
			\begin{column}[c]{0.48\textwidth}
				\begin{figure}[h!]
					\centering
					\includegraphics[width=0.5\textwidth,angle=-90]{figure6.png}
					\caption{Dense block architecture.} 
				\end{figure}
			\end{column}
		\end{columns}
	\end{frame}
	
	\begin{frame}{Pyramid Scene Parsing Network}
		\begin{figure} [h!]
			\centering
			\includegraphics[height=.7\textheight]{figure7.png}
			\caption{PSPNet architecture.} 
		\end{figure}
	\end{frame}
	
	\begin{frame}{Global Convolution Network}
		\begin{columns}[T]
			\begin{column}[c]{0.55\textwidth}
				\begin{figure}
					\centering
					\includegraphics[width=.75\textwidth]{figure8.png}
					\caption{GCN architecture.} 
				\end{figure}	
			\end{column}
			\begin{column}[c]{0.45\textwidth}
				\begin{figure}
					\centering
					\includegraphics[width=.9\textwidth]{figure9.png}
					\caption{(a) Residual block, (b) GCN block, (c) Boundary Refinement.} 
				\end{figure}	
			\end{column}
		\end{columns}
	\end{frame}	
	%%%%%%%%%%%%%%%%%%%%%%%%%%%%%%%%%%%%%%%%%%%%%%%%%%%%%%%%%%%%%%%%%%%%
	%\setcounter{subfigure}{0}
	%\begin{frame}{RMS based models}
	%	\begin{column}[c]{0.55\textwidth}
		%		\begin{figure}
			%			\subfloat[Res-UNet model]{\includegraphics[width=1\textwidth]{figure4.png}}
			%		\end{figure}
		%	\end{column}
	%	\begin{column}[c]{0.35\textwidth}
		%		\begin{figure}
			%			\subfloat[Data flow \& intermediate outputs of layers \label{fig:}]{\animategraphics[autoplay, controls,width=.8\textwidth]{4}{figures/gif_figs/381__inter_pred/intermediate_output-}{0}{103}}
			%\end{figure}
			%	\end{column}
		%
		%\end{frame}
			
	\setcounter{subfigure}{0}
	%%%%%%%%%%%%%%%%%%%%%%%%%%%%%%%%%%%%%%%%%%%%%%%%%%%%%%%%%%%%%%%%%%%	
	\subsection{Full wavefield frames based model}
		\begin{frame}{Autoencoder ConvLSTM}
			\begin{columns}[T]		
				\begin{column}[c]{0.5\textwidth}
					\only<1>{
						\begin{figure}[c]
							\centering
							\subfloat{\includegraphics[width=.8\textwidth]{figure2.png}}
							\caption{Sample frames of full wave propagation.}
					\end{figure}}
					\only<2->{
						\begin{figure}[c]
							\centering
							\subfloat{\includegraphics[height=.6\textheight]{figure3.png}}
							\caption{The procedure of calculating the RMS prediction image (damage map).}
					\end{figure}}
				\end{column}
				\begin{column}[c]{0.45\textwidth}
					\begin{figure}
						\centering
						\subfloat{\includegraphics[width=.8\textheight]{figure5b.png}}
						\caption{Autoencoder ConvLSTM model}
					\end{figure}
				\end{column}
			\end{columns}
		\end{frame}
		%%%%%%%%%%%%%%%%%%%%%%%%%%%%%%%%%%%%%%%%%%%%%%%%%%%%%%%%%%%%%%%%
		\begin{frame}{Evaluation metrics for delamination identification}
			\begin{columns}[T]
				\begin{column}[c]{0.45\textwidth}
					For evaluating delamination identification
					\begin{itemize}
						\item Intersection over Union (IoU): 
						\begin{equation*}
							\textup{IoU}=\frac{Intersection}{Union}=\frac{\hat{Y} \cap Y}{\hat{Y} \cup Y},
							\label{eqn:iou}
						\end{equation*}
						\item Percentage area error $\epsilon$:
						\begin{equation*}
							\epsilon=\frac{|A-\hat{A}|}{A} \times 100\%,
							\label{eqn:mean_size_error}
						\end{equation*}
					\end{itemize}
				\end{column}
				\begin{column}[c]{0.45\textwidth}
					\begin{figure}
						\centering
						\includegraphics[width=1.0\textwidth]{IoU_figure.png}			
					\end{figure}
				\end{column}
			\end{columns}
		\end{frame}
		%%%%%%%%%%%%%%%%%%%%%%%%%%%%%%%%%%%%%%%%%%%%%%%%%%%%%%%%%%%%%%%%
		\section{Numerical test cases}
		%%%%%%%%%%%%%%%%%%%%%%%%%%%%%%%%%%%%%%%%%%%%%%%%%%%%%%%%%%%%%%%%
		\begin{frame}{Numerical test cases RMS based models (GCN model)}
			\begin{columns}[T]
				\begin{column}[c]{0.32\textwidth}
					\begin{figure}[c]
						\centering
						\animategraphics[controls,width=.9\textwidth]{2}{figures/gif_figs/397/intermediate_output-}{0}{82}
						\caption{\(1^{st}\) numerical case.}
					\end{figure}
				\end{column}
				\hfill
				\begin{column}[c]{0.32\textwidth}
					\begin{figure}[c]
						\centering
						\animategraphics[controls,width=.9\textwidth]{2}{figures/gif_figs/438/intermediate_output-}{0}{82}
						\caption{\(2^{nd}\) numerical case.}
					\end{figure}
				\end{column}
				\hfill
				\begin{column}[c]{0.32\textwidth}
					\begin{figure}[c]
						\centering
						\animategraphics[controls,width=.9\textwidth]{2}{figures/gif_figs/456/intermediate_output-}{0}{82}
						\caption{\(3^{rd}\) numerical case.}
					\end{figure}
				\end{column}
			\end{columns}
		\end{frame}
		%%%%%%%%%%%%%%%%%%%%%%%%%%%%%%%%%%%%%%%%%%%%%%%%%%%%%%%%%%%%%%%%
		\begin{frame}{RMS based: Analysis of numerical cases}
			\begin{table}[ht!]
				\centering
				\caption{Analysis of numerical cases.}
				\label{tab:table_all_numerical_cases}	
				\begin{tabular}{lcc}
					\toprule
					Model & mean \(IoU\) & max \(IoU\) \\ 
					\midrule 
					Res-UNet & \(0.66\) & \(0.89\) \\ 
					VGG16 encoder-decoder & \(0.57\) & \(0.84\) \\ 
					FCN-DenseNet & \(0.68\) & \(0.92\) \\ 
					PSPNet & \(0.55\) & \(0.91\) \\ 
					GCN & \(0.76\) & \(0.93\) \\ 
					\bottomrule
				\end{tabular}
			\end{table}
		\end{frame}
		%%%%%%%%%%%%%%%%%%%%%%%%%%%%%%%%%%%%%%%%%%%%%%%%%%%%%%%%%%%%%%%%
		\begin{frame}{Numerical test cases animation of Lamb waves}
			\setcounter{subfigure}{0}
			\only<1>{
				\textbf{First test case}
				\begin{figure}
					\centering
					\subfloat[Full wavefield (512 frames)]{\animategraphics[autoplay,loop,height=4cm,keepaspectratio]{32}{figures/gif_figs/381_output/output_381-}{1}{512}}\quad
					\subfloat[RMS of all intermediate predictions]{\includegraphics[height=4.1cm,keepaspectratio]{figures/RMS_Ijjeh_num_case_381.png}}\quad
					\subfloat[Binary RMS, IoU= 0.88]{\includegraphics[height=4cm,keepaspectratio]{figures/Binary_RMS_Ijjeh_num_case381_.png}}\quad
				\end{figure}}
			\setcounter{subfigure}{0}
			\only<2>{
				\textbf{Second test case}
				\begin{figure}
					\centering
					\subfloat[Full wavefield (512 frames)]{\animategraphics[autoplay,loop,height=4cm,keepaspectratio]{32}{figures/gif_figs/385_output/output_385-}{1}{512}}\quad
					\subfloat[RMS of all intermediate predictions]{\includegraphics[height=4.1cm,keepaspectratio]{figures/RMS_Ijjeh_num_case_385.png}}\quad
					\subfloat[Binary RMS, IoU= 0.58]{\includegraphics[height=4cm,keepaspectratio]{figures/Binary_RMS_Ijjeh_num_case385_.png}}
			\end{figure}}
			\setcounter{subfigure}{0}
			\only<3>{
				\textbf{Third test case}
				\begin{figure}
					\centering
					\subfloat[Full wavefield (512 frames)]{\animategraphics[autoplay,loop,height=4cm,keepaspectratio]{32}{figures/gif_figs/394_output/output_394-}{1}{512}}\quad
					\subfloat[RMS of all intermediate predictions]{\includegraphics[height=4.1cm,keepaspectratio]{figures/RMS_Ijjeh_num_case_394.png}}\quad
					\subfloat[Binary RMS, IoU= 0.8]{\includegraphics[height=4cm,keepaspectratio]{figures/Binary_RMS_Ijjeh_num_case394_.png}}
			\end{figure}}
		\end{frame}
		%%%%%%%%%%%%%%%%%%%%%%%%%%%%%%%%%%%%%%%%%%%%%%%%%%%%%%%%%%%%%%%%
		\section{Experimental case}
		%%%%%%%%%%%%%%%%%%%%%%%%%%%%%%%%%%%%%%%%%%%%%%%%%%%%%%%%%%%%%%%%
		\begin{frame}[t]{Composite specimen}
			\begin{columns}[T]
				\column{0.7\textwidth}
				{\small
					\begin{itemize}
						\item 16 layers set at the same angle \\
						\item carbon: Prepreg GG 205  P (fibres Toray FT 300 - 3K 200 tex), $E=230$ GPa
						\item epoxy resin: IMP503Z-HT by Impregnatex Compositi 
						\item dimensions: 500$\times$500$\times$3.9 mm\\
						\item density: 1522.4~kg/m\textsuperscript{3}
					\end{itemize}
				}
				\column{0.3\textwidth}
				\begin{figure}
					\includegraphics[width=0.6\textwidth]{weave-1.jpg}
					\caption{Plain weave fabric}
				\end{figure}
			\end{columns}
			\begin{table}[h]
				\renewcommand{\arraystretch}{1.1}
				\centering \footnotesize
				\caption{Geometry of a plain weave fabric reinforced composite [mm]}
				\begin{tabular}{cccccc} 
					%\hline
					\toprule
					\multicolumn{4}{c}{\textbf{width} }	& \multicolumn{2}{c}{\textbf{thickness} }  \\ 
					%	\hline \hline
					\cmidrule(lr){1-4} \cmidrule(lr){5-6} 
					fill & warp & fill gap& warp gap& fill & warp\\
					%\hline
					$a_f$ &$a_w$& $g_f$  & $g_w$  & $h_f$& $h_w$ \\ 
					%\hline
					%\midrule
					\cmidrule(lr){1-2} \cmidrule(lr){3-4} \cmidrule(lr){5-6}
					1.92 &2.0& 0.05& 0.05 & 0.121875 & 0.121875 \\
					%\hline 
					\bottomrule 
				\end{tabular} 
				\label{tab:weave_geo}
			\end{table}
		\end{frame}
		%%%%%%%%%%%%%%%%%%%%%%%%%%%%%%%%%%%%%%%%%%%%%%%%%%%%%%%%%%%%%%%%
		\begin{frame}[t]{Specimens with defects}
			\vspace{-0.5cm}
			\begin{columns}[T]
				\column{0.5\textwidth}
				\begin{figure}
					\includegraphics[scale=0.36]{plate_multi_delam_arrangement_large_fonts.png}
				\end{figure}
				\column{0.5\textwidth}
				\begin{figure}
					\includegraphics[scale=0.36]{plate_single_delam_arrangement_large_fonts.png}
				\end{figure}
			\end{columns}
		\end{frame}
		%%%%%%%%%%%%%%%%%%%%%%%%%%%%%%%%%%%%%%%%%%%%%%%%%%%%%%%%%%%%%%%%
		\begin{frame}[t]{SLDV measurements: setup}
			\begin{figure}
				\includegraphics[width=0.7\textwidth]{sensors_fig4_setup.png}
			\end{figure}
		\end{frame}
		%%%%%%%%%%%%%%%%%%%%%%%%%%%%%%%%%%%%%%%%%%%%%%%%%%%%%%%%%%%%%%%%
		\begin{frame}[t]{SLDV measurements: laboratory}
			\begin{columns}[T]
				\column{0.5\textwidth}
				\begin{figure}
					\includegraphics[width=0.8\textwidth]{wibrometr-laserowy-1d_small-description.png}
				\end{figure}
				\column{0.5\textwidth}
				\begin{enumerate}
					\item Signal generator: TTI 1241 
					\item Amplifier: Piezo Systems EPA-104-230 $\pm$200 Vp
					\item Specimen
					\item Scanning head: Polytec PSV-400
					\item DAQ system: Polytec
				\end{enumerate}
			\end{columns}
			{\small
				Measurements were taken on a uniform grid of \textbf{333$\times$333 points}.\\
				Excitation in the form of Hann windowed sine signal of carrier frequency \textbf{50 kHz} was applied to piezoelectric transducer.}
		\end{frame}
		
%		\begin{frame}{Experimental setup}
%			\begin{columns}[T]
%				\begin{column}[t]{0.55\textwidth}
%					\begin{figure}
%						\centering
%						\includegraphics[width=.9\textwidth]{wibrometr-laserowy-1d_small-description.png}
%					\end{figure}
%				\end{column}
%				\begin{column}[t]{0.4\textwidth}
%					\begin{enumerate}
%						\item Waveform generator
%						\item Power amplifier	
%						\item Specimen
%						\item SLDV head
%						\item DAQ
%					\end{enumerate}
%				\end{column}
%			\end{columns}
%		\end{frame}
	
%		\begin{frame}{Single delamination arrangement}
%			\begin{minipage}[c]{0.4\textwidth}
%				\begin{itemize}[<alert@+>]
%					\item 
%					\item 
%					\item 
%				\end{itemize}
%			\end{minipage}
%			\begin{minipage}[c]{0.55\textwidth}
%				\centering
%				\includegraphics[width=.7\textwidth]{plate_single_delam_arrangement_large_fonts.jpg}
%			\end{minipage}
%		\end{frame}
		%%%%%%%%%%%%%%%%%%%%%%%%%%%%%%%%%%%%%%%%%%%%%%%%%%%%%%%%%%%%%%%%
		\setcounter{subfigure}{0}
		\begin{frame}{Experimental results RMS based (Single delamination)}
			\centering
			\begin{figure}
				\subfloat[ERMS \& label]{\includegraphics[width=.18\textwidth]{ERMS_with_label.png}}\qquad
				\subfloat[ERMSF]{\includegraphics[width=.18\textwidth]{ERMSF_CFRP_teflon_3o_375_375p_50kHz_5HC_x12_15Vpp.png}}\qquad
				\subfloat[Binary ERMSF: IoU=$0.401$]{\includegraphics[width=.18\textwidth]{Binary_ERMSF_CFRP_teflon_3o_375_375p_50kHz_5HC_x12_15Vpp.png}}\qquad
				\\
				\subfloat[GCN: IoU\(=0.723\)]{\includegraphics[width=.18\textwidth]{Fig_GCN_7.png}}\qquad
				\subfloat[FCN-DenseNet: IoU=$0.54$]{\includegraphics[width=.18\textwidth]{Fig_FCN_DenseNet_7.png}}\qquad
			\end{figure}
		\end{frame}
		%%%%%%%%%%%%%%%%%%%%%%%%%%%%%%%%%%%%%%%%%%%%%%%%%%%%%%%%%%%%%%%%
		\begin{frame}{RMS based: Analysis of experimental case}
			\begin{table}[!ht]
				\centering
				\caption{Evaluation metrics of the experimental case.}
				\label{tab:rms_exp_case}
				\begin{tabular}{l@{\ }cccc}
					\toprule
					\multicolumn{1}{l}{Model} & \multicolumn{1}{c}{A [mm\textsuperscript{2}]} & \multicolumn{3}{c}{Predicted output} \\ 
					\cmidrule(lr){3-5} & & \multicolumn{1}{c}{IoU} & \multicolumn{1}{c}{\(\hat{A}\) [mm\textsuperscript{2}]} & \(\epsilon\) \\ \midrule
					Res-UNet & \multicolumn{1}{c}{\multirow{5}{*}{210}} & \multicolumn{1}{c}{0.58} & \multicolumn{1}{c}{323}  & \(53.8\%\) \\ 
					VGG16 encoder-decoder &  & \multicolumn{1}{c}{0.62} & \multicolumn{1}{c}{320} & \(52.4\%\) 
					\\ 
					FCN-DenseNet &  & \multicolumn{1}{c}{0.54} & \multicolumn{1}{c}{386} & \(83.8\%\) \\ 
					PSPNet &  & \multicolumn{1}{c}{0.49} & \multicolumn{1}{c}{580} & \(176.2\%\) 
					\\ 
					GCN &  & \multicolumn{1}{c}{0.72} & \multicolumn{1}{c}{309} & \(47.1\%\) 
					\\ 
					\bottomrule
				\end{tabular}		
			\end{table}
		\end{frame}
		%%%%%%%%%%%%%%%%%%%%%%%%%%%%%%%%%%%%%%%%%%%%%%%%%%%%%%%%%%%%%%%%
		\begin{frame}{Experimental results full wavefield based (Single delamination)}
			\begin{figure}
				\centering
				\subfloat[Full wavefield (512 frames)]{\animategraphics[autoplay,loop,height=3cm]{32}{figures/gif_figs/CFRP_teflon_3o_375_375p_50kHz_5HC_x12_15Vpp/CFRP_teflon_30-}{1}{256}}\quad
				\subfloat[Intermidate ouputs]{\animategraphics[autoplay,loop,height=3cm]{24}{figures/gif_figs/CFRP_ijjeh_single_delamination/intermediate_output-}{0}{231}}\quad
				\subfloat[RMS of all intermediate predictions]{\includegraphics[height=3cm,keepaspectratio]{figures/RMS_CFRP_teflon_3o_375_375p_50kHz_5HC_x12_15Vpp_Ijjeh_updated_results_.png}}\quad
				\subfloat[Binary RMS]{\includegraphics[height=3cm,keepaspectratio]{figures/Binary_RMS_CFRP_teflon_3o__375_375p_50kHz_5HC_x12_15Vpp_Ijjeh_.png}}
			\end{figure}
			IoU= $0.41$ and $\epsilon=71.56\%$  for the thresholded damage map.
		\end{frame}
		%%%%%%%%%%%%%%%%%%%%%%%%%%%%%%%%%%%%%%%%%%%%%%%%%%%%%%%%%%%%%%%%
%		\setcounter{subfigure}{0}
%		\begin{frame}{Multiple delamination arrangement}
%			\begin{minipage}[c]{0.4\textwidth}
%				\begin{itemize}[<alert@+>]
%					\item Teflon inserts with a thickness of \(250\ \mu\)m were used to simulate the delaminations.
%					\item The average thickness of the specimen was \(3.9 \pm 0.1\) mm.
%					\item The delaminations were located at the same distance, equal to \(150\) mm from the centre of the plate.
%				\end{itemize}
%			\end{minipage}
%			\begin{minipage}[c]{0.55\textwidth}
%				\centering
%				\includegraphics[width=.7\textwidth]{plate_multi_delam_arrangement_large_fonts.png}
%			\end{minipage}
%		\end{frame}
		%%%%%%%%%%%%%%%%%%%%%%%%%%%%%%%%%%%%%%%%%%%%%%%%%%%%%%%%%%%%%%%%
		\setcounter{subfigure}{0}	
		\begin{frame}{Experimental results full wavefield based}
			\begin{figure}
				\centering
				\subfloat[Full wavefield (512 frames)]{\animategraphics[autoplay,loop,height=3cm]{32}{figures/gif_figs/input_specimen_3/specimen_3-}{1}{512}}\quad
				\subfloat[Intermidate ouputs]{\animategraphics[autoplay,loop,height=3cm]{24}{figures/gif_figs/Intermediate_specimen_3/Intermediate_specimen_3-}{0}{487}}\quad
				\subfloat[RMS of all intermediate predictions]{\includegraphics[height=3cm,keepaspectratio]{figures/RMS_L3_S3_B_333x333p_50kHz_5HC_18Vpp_x10_pzt_Ijjeh_updated_results_.png}}\quad
				\subfloat[Binary RMS]{\includegraphics[height=3cm,keepaspectratio]{figures/Binary_RMS_L3_S3_B__333x333p_50kHz_5HC_18Vpp_x10_pzt_Ijjeh_.png}}
			\end{figure}
			IoU= $0.64$ and $\epsilon=1.69\%$  for the thresholded damage map.
		\end{frame}
		%%%%%%%%%%%%%%%%%%%%%%%%%%%%%%%%%%%%%%%%%%%%%%%%%%%%%%%%%%%%%%%%
		\setcounter{subfigure}{0}
		\section{Super-resolution image reconstruction}
		\begin{frame}{Super-resolution image reconstruction}
			\begin{columns}[T]
				\begin{column}[c]{0.4\textwidth}
					\justifying
					\only<1>{\alert{Single image Super-Resolution (SISR)} aims to generate a visually pleasing high-resolution image from its de-graded low-resolution measurement.}

					\only<2>{DLSR model was developed to recover the high-resolution full wavefield frames with satisfying accuracy from the low-resolution measurement (below Nyquist sampling rate) acquired by SLDV.}
					
					\tiny
					\only<2>{
						\bigskip 
						\biblioref{Abdalraheem Ijjeh, Saeed Ullah, Maciej Radzienski, Pawel Kudela}{2023}{Deep learning super-resolution for the reconstruction of full wavefield of Lamb waves}{Mechanical Systems and Signal Processing 186: 109878.}}
				\end{column}
				\begin{column}[c]{0.55\textwidth}
					\begin{figure}
						\subfloat{\includegraphics[width=0.8\textwidth]{superresolution_flowchart.jpg}}
					\end{figure}
				\end{column}
			\end{columns}
		\end{frame}
		%%%%%%%%%%%%%%%%%%%%%%%%%%%%%%%%%%%%%%%%%%%%%%%%%%%%%%%%%%%%%%%%
		\begin{frame}{Evaluation metrics for SR model}
			For evaluating the reconstructed full wavefield
			\begin{itemize}
				\item Peak signal-to-noise ratio (PSNR):
				\begin{equation*}
					PSNR=20\log_{10}\left(\frac{R}{\sqrt{MSE}}\right),
					\label{PSNR}
				\end{equation*}
				\item Pearson correlation coefficient:
				\begin{equation*}
					r_{xy} = \frac{\sum_{i=1}^{n}(x_i - \bar{x})(y_i-\bar{y})}{\sqrt{\sum_{i=1}^{n}(x_i - \bar{x})^2}\sqrt{\sum_{i=1}^{n}(y_i - \bar{y})^2}},
					\label{Pearson}
				\end{equation*}
			\end{itemize}
		\end{frame}
		%%%%%%%%%%%%%%%%%%%%%%%%%%%%%%%%%%%%%%%%%%%%%%%%%%%%%%%%%%%%%%%%
		\setcounter{subfigure}{0}
		\begin{frame}{Numerical test cases at certain frames}
			\begin{columns}[T]
				\begin{column}[c]{0.25\textwidth}
					\centering
					\only<1>{\textbf{First test case}}
					\only<2>{\textbf{Second test case}}
					\only<3>{\textbf{Third test case}}
					\begin{figure}
						\centering
						\only<1>{\subfloat[LR input, $f_n=127$]{\includegraphics[height=.45\textheight]{LR_397_frame_127_input.png}}}
						
						\only<2>{\subfloat[LR input, $f_n=154$]{\includegraphics[height=.45\textheight]{LR_438_frame_154_input.png}}}
						
						\only<3>{\subfloat[LR input, $f_n=159$]{\includegraphics[height=.45\textheight]{LR_456_frame_159_input.png}}}
					\end{figure}
				\end{column}
				\begin{column}[c]{0.7\textwidth}
					\only<1>{
						\begin{figure}
							\centering
							\subfloat[listentry][HR ref]{\includegraphics[height=.35\textheight]{output_397_frame_127_full_frame_GT.png}}\quad
							\subfloat[listentry][SR $f_n$]{\includegraphics[height=.35\textheight]{output_397_frame_127_full_frame_pred.png}}
							\\
							\subfloat[listentry][Ref]{\includegraphics[height=.35\textheight]{output_397_frame_127_delamination_GT.png}}\quad
							\subfloat[listentry][Pred]{\includegraphics[height=.35\textheight]{output_397_frame_127_delamination_pred.png}}
						\end{figure}}
					
						\only<2>{
						\begin{figure}
							\subfloat[listentry][HR ref]{\includegraphics[height=.35\textheight]{output_438_frame_154_full_frame_GT.png}}\quad
							\subfloat[listentry][SR $f_n$]{\includegraphics[height=.35\textheight]{output_438_frame_154_full_frame_pred.png}}
							\\
							\subfloat[listentry][Ref]{\includegraphics[height=.35\textheight]{output_438_frame_154_delamination_GT.png}}\quad	
							\subfloat[listentry][Pred]{\includegraphics[height=.35\textheight]{output_438_frame_154_delamination_pred.png}}
						\end{figure}}

						\only<3>{
						\begin{figure}
							\subfloat[listentry][HR ref]{\includegraphics[height=.35\textheight]{output_456_frame_159_full_frame_GT.png}}\quad
							\subfloat[listentry][SR $f_n$]{\includegraphics[height=.35\textheight]{output_456_frame_159_full_frame_pred.png}}
							\\
							\subfloat[listentry][Ref]{\includegraphics[height=.35\textheight]{output_456_frame_159_delamination_GT.png}}\quad
							\subfloat[listentry][Pred]{\includegraphics[height=.35\textheight]{output_456_frame_159_delamination_pred.png}}
						\end{figure}}
				\end{column}
			\end{columns}
		\end{frame}
		%%%%%%%%%%%%%%%%%%%%%%%%%%%%%%%%%%%%%%%%%%%%%%%%%%%%%%%%%%%%%%%%
		\begin{frame}{Experimental setup}
			\begin{figure}
				\centering
				\includegraphics[height=0.8\textheight]{sensors_fig4_setup.png}
			\end{figure}
		\end{frame}
		%%%%%%%%%%%%%%%%%%%%%%%%%%%%%%%%%%%%%%%%%%%%%%%%%%%%%%%%%%%%%%%%
		\setcounter{subfigure}{0}
		\begin{frame}{Experimental test case at certain frame}
			Low-resolution measurements (Input): \(32\times32=1024\) points. (Nf = 110.)\\
			High-resolution (Output): \(512\times512=262144\) points (Nf = 110.).
			\begin{columns}[T]
				\begin{column}[c]{0.3\textwidth}
					\begin{figure}						
						\subfloat[LR input]{						\includegraphics[height=.5\textheight]{frame110_32x32.png}}
					\end{figure}
				\end{column}
				\begin{column}[c]{0.66\textwidth}
					\begin{figure}
						\subfloat[HR ref]{\includegraphics[height=.27\textheight]{figure10a.png}}\quad
						\subfloat[CS]{\includegraphics[height=.27\textheight]{figure10b.png}}\quad
						\subfloat[SR]{\includegraphics[height=.27\textheight]{figure10e.png}}\quad
						\subfloat[Ref]{\includegraphics[height=.27\textheight]{figure11a.png}}\quad
						\subfloat[CS]{\includegraphics[height=.27\textheight]{figure11b.png}}\quad
						\subfloat[Pred]{\includegraphics[height=.27\textheight]{figure11e.png}}\quad
					\end{figure}
				\end{column}				
			\end{columns}
		\end{frame}
		%%%%%%%%%%%%%%%%%%%%%%%%%%%%%%%%%%%%%%%%%%%%%%%%%%%%%%%%%%%%%%%%
		\section{Conclusions}
		%%%%%%%%%%%%%%%%%%%%%%%%%%%%%%%%%%%%%%%%%%%%%%%%%%%%%%%%%%%%%%%%
		\begin{frame}{Conclusions}
			\footnotesize
			\begin{itemize}
				\item Full wavefields of elastic waves propagating in a composite laminate contain extensive, valuable, and complex information regarding the discontinuities in the plate, such as delamination or edges.
				\item Such information can be utilised to train deep learning models
				to perform damage identification in an end-to-end approach.
				\item With deep	learning approaches, it is possible to use registered data in its raw form without the need to perform feature engineering, extraction, and classification. 
				Hence, such an approach has an end-to-end structure that automatically learns and discovers the hidden features in high-dimensional input data.
				\item Deep learning approaches surpass the conventional signal processing techniques.
				\item The DLSR model can recover the high-resolution full wavefield
				frames with high accuracy from the low-resolution acquired full wavefield by SLDV.
			\end{itemize}
		\end{frame}		
		%%%%%%%%%%%%%%%%%%%%%%%%%%%%%%%%%%%%%%%%%%%%%%%%%%
		{\setbeamercolor{palette primary}{fg=blue, bg=white}
			\begin{frame}[standout]
				Thank you for your listening!\\ \vspace{12pt}
				Questions?\\ \vspace{12pt}
				\url{pk@imp.gda.pl} 
				\par\medskip
				\url{aijjeh@imp.gda.pl}
				
				
				\par\medskip
				\par\medskip
				\footnotesize
				The research work was funded by the Polish National Science Center under grant agreement no. 2018/31/B/ST8/00454.
			\end{frame}
		}
		%%%%%%%%%%%%%%%%%%%%%%%%%%%%%%%%%%%%%%%%%%%%%%%%%%
		% END OF SLIDES
		%%%%%%%%%%%%%%%%%%%%%%%%%%%%%%%%%%%%%%%%%%%%%%%%%%
	\end{document}