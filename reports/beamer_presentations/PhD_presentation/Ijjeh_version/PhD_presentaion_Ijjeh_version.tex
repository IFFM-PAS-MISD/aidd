%\PassOptionsToPackage{draft}{graphicx}
\documentclass[10pt,aspectratio=169,dvipsnames]{beamer} % aspect ratio 16:9
%\graphicspath{{../../figures/}}

%\includeonlyframes{frame1,frame2,frame3}

%%%%%%%%%%%%%%%%%%%%%%%%%%%%%%%%%%%%%%%%%%%%%%%%%%
% Packages
%%%%%%%%%%%%%%%%%%%%%%%%%%%%%%%%%%%%%%%%%%%%%%%%%%
\usepackage{appendixnumberbeamer}
\usepackage{booktabs}
\usepackage{csvsimple} % for csv read
\usepackage[scale=2]{ccicons}
\usepackage{pgfplots}
\usepackage{xspace}
\usepackage{amsmath}
\usepackage{totcount}
\usepackage{tikz}
\usepackage{bm}
\usepackage{float}
\usepackage{eso-pic} 
\usepackage{wrapfig}
\usepackage{animate,media9}
\usepackage{subfig}
\usepackage{fancybox}
%\usepackage{multimedia}
\usepackage{dashbox}
\usepackage{tcolorbox}
\usepackage{multicol}
\usepackage{multirow}
\usepackage{xcolor}
\usepackage[document]{ragged2e}
\usepackage{caption}
\usepackage{comment}
%\usepackage[export]{adjustbox}

%\usepackage{FiraSans}

%\usepackage{comment}
%\usetikzlibrary{external} % speedup compilation
%\tikzexternalize % activate!
%\usetikzlibrary{shapes,arrows} 

%\usepackage{bibentry}
%\nobibliography*
\usepackage{ifthen}
\newcounter{angle}
\setcounter{angle}{0}
%\usepackage{bibentry}
%\nobibliography*
\usepackage{caption}%

\graphicspath{{figures/}}

\captionsetup[figure]{labelformat=empty}%
\usefonttheme{structurebold}
%%%%%%%%%%%%%%%%%%%%%%%%%%%%%%%%%%%%%%%%%%%%%%%%%%
% Metropolis theme custom modification file
%%%%%%%%%%%%%%%%%%%%%%%%%%%%%%%%%%%%%%%%%%%%%%%%%%
% Metropolis theme custom modification file
%%%%%%%%%%%%%%%%%%%%%%%%%%%%%%%%%%%%%%%%%%%%%%%%%%
% Metropolis theme custom colors
%%%%%%%%%%%%%%%%%%%%%%%%%%%%%%%%%%%%%%%%%%%%%%%%%%
\usetheme[progressbar=foot]{metropolis}
\useoutertheme{metropolis}
\useinnertheme{metropolis}
\usefonttheme{metropolis}
\setbeamercolor{background canvas}{bg=white}

%\usecolortheme{spruce}

\definecolor{myblue}{rgb}{0.19,0.55,0.91}
\definecolor{mediumblue}{rgb}{0,0,205}
\definecolor{darkblue}{rgb}{0,0,139}
\definecolor{Dodgerblue}{HTML}{1E90FF}
\definecolor{Navy}{HTML}{000080} % {rgb}{0,0,128}
\definecolor{Aliceblue}{HTML}{F0F8FF}
\definecolor{Lightskyblue}{HTML}{87CEFA}
\definecolor{logoblue}{RGB}{1,67,140}
\definecolor{Purple}{HTML}{911146}
\definecolor{Orange}{HTML}{CF4A30}

\setbeamercolor{progress bar}{bg=Lightskyblue}
\setbeamercolor{progress bar}{ fg=logoblue} 
\setbeamercolor{frametitle}{bg=logoblue}
\setbeamercolor{title separator}{fg=logoblue}
\setbeamercolor{block title}{bg=Lightskyblue!30,fg=black}
\setbeamercolor{block body}{bg=Lightskyblue!15,fg=black}
\setbeamercolor{alerted text}{fg=Purple}
% notes colors
\setbeamercolor{note page}{bg=white}
\setbeamercolor{note title}{bg=Lightskyblue}
%%%%%%%%%%%%%%%%%%%%%%%%%%%%%%%%%%%%%%%%%%%%%%%%%%
%  Theme modifications
%%%%%%%%%%%%%%%%%%%%%%%%%%%%%%%%%%%%%%%%%%%%%%%%%%
% modify progress bar linewidth
\makeatletter
\setlength{\metropolis@progressinheadfoot@linewidth}{2pt} 
\setlength{\metropolis@titleseparator@linewidth}{1pt}
\setlength{\metropolis@progressonsectionpage@linewidth}{1pt}

\setbeamertemplate{progress bar in section page}{
	\setlength{\metropolis@progressonsectionpage}{%
		\textwidth * \ratio{\thesection pt}{\totvalue{totalsection} pt}%
	}%
	\begin{tikzpicture}
		\fill[bg] (0,0) rectangle (\textwidth, 
		\metropolis@progressonsectionpage@linewidth);
		\fill[fg] (0,0) rectangle (\metropolis@progressonsectionpage, 
		\metropolis@progressonsectionpage@linewidth);
	\end{tikzpicture}%
}
\makeatother
\newcounter{totalsection}
\regtotcounter{totalsection}

\AtBeginDocument{%
	\pretocmd{\section}{\refstepcounter{totalsection}}{\typeout{Yes, prepending 
	was successful}}{\typeout{No, prepending was not successful}}%
}%
%%%%%%%%%%%%%%%%%%%%%%%%%%%%%%%%%%%%%%%%%%%%%%%%%%
%  Bibliography mods
%%%%%%%%%%%%%%%%%%%%%%%%%%%%%%%%%%%%%%%%%%%%%%%%%%
\setbeamertemplate{bibliography item}{\insertbiblabel} %% Remove book symbol 
%%from references and add number in square brackets
% kill the abominable icon (without number)
%\setbeamertemplate{bibliography item}{}
%\makeatletter
%\renewcommand\@biblabel[1]{#1.} % number only
%\makeatother
% remove line breaks in bibliography
\setbeamertemplate{bibliography entry title}{}
\setbeamertemplate{bibliography entry location}{}
%%%%%%%%%%%%%%%%%%%%%%%%%%%%%%%%%%%%%%%%%%%%%%%%%%
%  Bibliography custom commands
%%%%%%%%%%%%%%%%%%%%%%%%%%%%%%%%%%%%%%%%%%%%%%%%%%
\newcommand{\bibliotitlestyle}[1]{\textbf{#1}\par}

\newif\ifinbiblio
\newcounter{bibkey}
\newenvironment{biblio}[2][long]{%
	%\setbeamertemplate{bibliography item}{\insertbiblabel}
	\setbeamertemplate{bibliography item}{}% without numbers
	\setbeamerfont{bibliography item}{size=\footnotesize}
	\setbeamerfont{bibliography entry author}{size=\footnotesize}
	\setbeamerfont{bibliography entry title}{size=\footnotesize}
	\setbeamerfont{bibliography entry location}{size=\footnotesize}
	\setbeamerfont{bibliography entry note}{size=\footnotesize}
	\ifx!#2!\else%
	\bibliotitlestyle{#2}%
	\fi%
	\begin{thebibliography}{}%
		\inbibliotrue%
		\setbeamertemplate{bibliography entry title}[#1]%
	}{%
		\inbibliofalse%
		\setbeamertemplate{bibliography item}{}%
	\end{thebibliography}%
}

\newcommand{\biblioref}[5][short]{
	\setbeamertemplate{bibliography entry title}[#1]
	\stepcounter{bibkey}%
	\ifinbiblio%
	\bibitem{\thebibkey}%
	#2
	\newblock #4
	\ifx!#5!\else\newblock {\em #5}, #3 \fi%
	\else%
	\begin{biblio}{}
		\bibitem{\thebibkey}
		#2
		\newblock #4
		\ifx!#5!\else\newblock {\em #5}, #3\fi
	\end{biblio}
	\fi
}
%
%\newbibmacro*{hypercite}{%
%	\renewcommand{\@makefntext}[1]{\noindent\normalfont##1}%
%	\footnotetext{%
%		\blxmkbibnote{foot}{%
%			\printtext[labelnumberwidth]{%
%				\printfield{prefixnumber}%
%				\printfield{labelnumber}}%
%			\addspace
%			\fullcite{\thefield{entrykey}}}}}
%
%\DeclareCiteCommand{\hypercite}%
%{\usebibmacro{cite:init}}
%{\usebibmacro{hypercite}}
%{}
%{\usebibmacro{cite:dump}}
%
%% Redefine the \footfullcite command to use the reference number
%\renewcommand{\footfullcite}[1]{\cite{#1}\hypercite{#1}}
%\usefonttheme[onlymath]{Serif} % It should be uncommented if Fira fonts in 
%%math does not work

%%%%%%%%%%%%%%%%%%%%%%%%%%%%%%%%%%%%%%%%%%%%%%%%%%
% Custom commands
%%%%%%%%%%%%%%%%%%%%%%%%%%%%%%%%%%%%%%%%%%%%%%%%%%
% matrix command 
\newcommand{\matr}[1]{\mathbf{#1}} % bold upright (Elsevier, Springer)
%\newcommand{\matr}[1]{#1}   % pure math version
%\newcommand{\matr}[1]{\bm{#1}}  % ISO complying version
% vector command 
\newcommand{\vect}[1]{\mathbf{#1}} % bold upright (Elsevier, Springer)
% bold symbol
\newcommand{\bs}[1]{\boldsymbol{#1}}
% derivative upright command
\DeclareRobustCommand*{\drv}{\mathop{}\!\mathrm{d}}
\newcommand{\ud}{\mathrm{d}}
% 
\newcommand{\themename}{\textbf{\textsc{metropolis}}\xspace}

%\usepackage{pgfpages}
%\setbeameroption{show notes}
%\setbeameroption{show notes on second screen=left}
\setbeamertemplate{note page}{\insertnote}
%%%%%%%%%%%%%%%%%%%%%%%%%%%%%%%%%%%%%%%%%%%%%%%%%%
% Title page options
%%%%%%%%%%%%%%%%%%%%%%%%%%%%%%%%%%%%%%%%%%%%%%%%%%
% \date{\today}
\date{}
%%%%%%%%%%%%%%%%%%%%%%%%%%%%%%%%%%%%%%%%%%%%%%%%%%
% option 1
%%%%%%%%%%%%%%%%%%%%%%%%%%%%%%%%%%%%%%%%%%%%%%%%%%%
\title{FEASIBILITY STUDY OF ARTIFICIAL INTELLIGENCE APPROACH FOR DELAMINATION IDENTIFICATION IN COMPOSITE LAMINATES}
%\subtitle{In preparation for a Ph.D. defence}
\author{\textbf{D.Sc. Ph.D. Eng. Paweł Kudela} \and \\ \textbf{Ph.D. candidate Eng. Abdalraheem A. Ijjeh }
}
% logo align to Institute 
\institute{Institute of Fluid Flow Machinery \\ 
	Polish Academy of Sciences \\ 
	\vspace{-1.5cm}
	\flushright 
	\includegraphics[width=6cm]{imp_logo.png}}
%%%%%%%%%%%%%%%%%%%%%%%%%%%%%%%%%%%%%%%%%%%%%%%%%%
% option 2 - authors in one line
%%%%%%%%%%%%%%%%%%%%%%%%%%%%%%%%%%%%%%%%%%%%%%%%%%
%	\title{My fancy title}
%	\subtitle{Lamb-opt}
%	\author{\textbf{Paweł Kudela}\textsuperscript{2}, Maciej 
	%	Radzieński\textsuperscript{2}, Wiesław Ostachowicz\textsuperscript{2}, 
	%	Zhibo Yang\textsuperscript{1} }
%	 logo align to Institute 
%	\institute{\textsuperscript{1}Xi'an Jiaotong University \\ 
	%	\textsuperscript{2}Institute of Fluid Flow Machinery\\ \hspace*{1pt} Polish 
	%	Academy of Sciences \\ \vspace{-1.5cm}\flushright 	
	%	\includegraphics[width=6cm]{imp_logo.png}}
%%%%%%%%%%%%%%%%%%%%%%%%%%%%%%%%%%%%%%%%%%%%%%%%%%%
% option 3 - multilogo vertical
%%%%%%%%%%%%%%%%%%%%%%%%%%%%%%%%%%%%%%%%%%%%%%%%%%
%%\title{My fancy title}
%%\subtitle{Lamb-opt}
%%	\author{\textbf{Paweł Kudela}\inst{1}, Maciej Radzieński\inst{1}, Wiesław Ostachowicz\inst{1}, Zhibo Yang\inst{2} }
%%	 logo under Institute 
%%	\institute%
%%	{ 
	%%		\inst{1}%
	%%		Institute of Fluid Flow Machinery\\ \hspace*{1pt} Polish Academy of Sciences \\ \includegraphics[height=0.85cm]{//odroid-sensors/sensors/MISD_shared/logo/logo_eng_40mm.eps} \\
	%%		\and
	%%		\inst{2}%
	%%	 Xi'an Jiaotong University \\ \includegraphics[height=0.85cm]{//odroid-sensors/sensors/MISD_shared/logo/logo_box.eps}
	%% }
% end od option 3
%%%%%%%%%%%%%%%%%%%%%%%%%%%%%%%%%%%%%%%%%%%%%%%%%%
%% option 4 - 3 Institutes and logos horizontal centered
%%%%%%%%%%%%%%%%%%%%%%%%%%%%%%%%%%%%%%%%%%%%%%%%%%
%\title{My fancy title}
%\subtitle{Lamb-opt }
%\author{\textbf{Paweł Kudela}\textsuperscript{1}, Maciej Radzieński\textsuperscript{1}, Marco Miniaci\textsuperscript{2}, Zhibo Yang\textsuperscript{3} }
%
%\institute{ 
	%\begin{columns}[T,onlytextwidth]
	%	\column{0.39\textwidth}
	%	\begin{center}
		%		\textsuperscript{1}Institute of Fluid Flow Machinery\\ \hspace*{3pt}Polish Academy of Sciences
		%	\end{center}
	%	\column{0.3\textwidth}
	%	\begin{center}
		%		\textsuperscript{2}Zurich University
		%	\end{center}
	%	\column{0.3\textwidth}
	%	\begin{center}
		%		\textsuperscript{3}Xi'an Jiaotong University
		%	\end{center}
	%\end{columns}
	%\vspace{6pt}
	%% logos 
	%\begin{columns}[b,onlytextwidth]
	%	\column{0.39\textwidth}
	%		\centering 
	%		\includegraphics[scale=0.9,height=0.85cm,keepaspectratio]{//odroid-sensors/sensors/MISD_shared/logo/logo_eng_40mm.eps}
	%	\column{0.3\textwidth}
	%		\centering 
	%		\includegraphics[scale=0.9,height=0.85cm,keepaspectratio]{//odroid-sensors/sensors/MISD_shared/logo/logo_box.eps}
	%	\column{0.3\textwidth}
	%		\centering 
	%		\includegraphics[scale=0.9,height=0.85cm,keepaspectratio]{//odroid-sensors/sensors/MISD_shared/logo/logo_box2.eps}
	%\end{columns}
	%}
%\makeatletter
%\setbeamertemplate{title page}{
	%	\begin{column}[b][\paperheight]{\textwidth}
		%		\centering % <-- Center here
		%		\ifx\inserttitlegraphic\@empty\else\usebeamertemplate*{title graphic}\fi
		%		\vfill%
		%		\ifx\inserttitle\@empty\else\usebeamertemplate*{title}\fi
		%		\ifx\insertsubtitle\@empty\else\usebeamertemplate*{subtitle}\fi
		%		\usebeamertemplate*{title separator}
		%		\ifx\beamer@shortauthor\@empty\else\usebeamertemplate*{author}\fi
		%		\ifx\insertdate\@empty\else\usebeamertemplate*{date}\fi
		%		\ifx\insertinstitute\@empty\else\usebeamertemplate*{institute}\fi
		%		\vfill
		%		\vspace*{1mm}
		%	\end{column}
	%}
%
%\setbeamertemplate{title}{
	%	% \raggedright% % <-- Comment here
	%	\linespread{1.0}%
	%	\inserttitle%
	%	\par%
	%	\vspace*{0.5em}
	%}
%\setbeamertemplate{subtitle}{
	%	% \raggedright% % <-- Comment here
	%	\insertsubtitle%
	%	\par%
	%	\vspace*{0.5em}
	%}
%\makeatother
% end of option 4
%%%%%%%%%%%%%%%%%%%%%%%%%%%%%%%%%%%%%%%%%%%%%%%%%%
% option 5 - 2 Institutes and logos horizontal centered
%%%%%%%%%%%%%%%%%%%%%%%%%%%%%%%%%%%%%%%%%%%%%%%%%%
%\title{My fancy title}
%\subtitle{Lamb-opt }
%\author{\textbf{Paweł Kudela}\textsuperscript{1}, Maciej Radzieński\textsuperscript{1}, Marco Miniaci\textsuperscript{2}}
%
%\institute{ 
	%	\begin{columns}[T,onlytextwidth]
		%		\column{0.5\textwidth}
		%			\centering
		%			\textsuperscript{1}Institute of Fluid Flow Machinery\\ \hspace*{3pt}Polish Academy of Sciences
		%		\column{0.5\textwidth}
		%			\centering
		%			\textsuperscript{2}Zurich University
		%	\end{columns}
	%	\vspace{6pt}
	%	% logos 
	%	\begin{columns}[b,onlytextwidth]
		%		\column{0.5\textwidth}
		%		\centering 
		%		\includegraphics[scale=0.9,height=0.85cm,keepaspectratio]{//odroid-sensors/sensors/MISD_shared/logo/logo_eng_40mm.eps}
		%		\column{0.5\textwidth}
		%		\centering 
		%		\includegraphics[scale=0.9,height=0.85cm,keepaspectratio]{//odroid-sensors/sensors/MISD_shared/logo/logo_box.eps}
		%	\end{columns}
	%}
%\makeatletter
%\setbeamertemplate{title page}{
	%	\begin{column}[b][\paperheight]{\textwidth}
		%		\centering % <-- Center here
		%		\ifx\inserttitlegraphic\@empty\else\usebeamertemplate*{title graphic}\fi
		%		\vfill%
		%		\ifx\inserttitle\@empty\else\usebeamertemplate*{title}\fi
		%		\ifx\insertsubtitle\@empty\else\usebeamertemplate*{subtitle}\fi
		%		\usebeamertemplate*{title separator}
		%		\ifx\beamer@shortauthor\@empty\else\usebeamertemplate*{author}\fi
		%		\ifx\insertdate\@empty\else\usebeamertemplate*{date}\fi
		%		\ifx\insertinstitute\@empty\else\usebeamertemplate*{institute}\fi
		%		\vfill
		%		\vspace*{1mm}
		%	\end{column}
	%}
%
%\setbeamertemplate{title}{
	%	% \raggedright% % <-- Comment here
	%	\linespread{1.0}%
	%	\inserttitle%
	%	\par%
	%	\vspace*{0.5em}
	%}
%\setbeamertemplate{subtitle}{
	%	% \raggedright% % <-- Comment here
	%	\insertsubtitle%
	%	\par%
	%	\vspace*{0.5em}
	%}
%\makeatother
% end of option 5
%
%%%%%%%%%%%%%%%%%%%%%%%%%%%%%%%%%%%%%%%%%%%%%%%%%%
% End of title page options
%%%%%%%%%%%%%%%%%%%%%%%%%%%%%%%%%%%%%%%%%%%%%%%%%%
% logo option - alternative manual insertion by modification of coordinates in \put()
%\titlegraphic{%
	%	%\vspace{\logoadheight}
	%	\begin{picture}(0,0)
		%	\put(305,-185){\makebox(0,0)[rb]{\includegraphics[width=4cm]{//odroid-sensors/sensors/MISD_shared/logo/logo_eng_40mm.eps}}}
		%	\end{picture}}
%
%%%%%%%%%%%%%%%%%%%%%%%%%%%%%%%%%%%%%%%%%%%%%%%%%%
%\tikzexternalize % activate!
%%%%%%%%%%%%%%%%%%%%%%%%%%%%%%%%%%%%%%%%%%%%%%%%%%
\setbeamertemplate{section in toc}[sections numbered]
\setbeamertemplate{subsection in toc}[subsections numbered]

\begin{document}
	%%%%%%%%%%%%%%%%%%%%%%%%%%%%%%%%%%%%%%%%%%%%%%%%%%
	\maketitle
	%%%%%%%%%%%%%%%%%%%%%%%%%%%%%%%%%%%%%%%%%%%%%%%%%%%%%%%%%%%%%%%%%%%%%%%%%%%%
	\note{I would like to thank you all for attending my doctoral defence}
	%%%%%%%%%%%%%%%%%%%%%%%%%%%%%%%%%%%%%%%%%%%%%%%%%%%%%%%%%%%%%%%%%%%%%%%%%%%%
	%%%%%%%%%%%%%%%%%%%%%%%%%%%%%%%%%%%%%%%%%%%%%%%%%%
	% SLIDES
	%%%%%%%%%%%%%%%%%%%%%%%%%%%%%%%%%%%%%%%%%%%%%%%%%%
	\begin{frame}[label=frame1]{Outlines}
		\begin{multicols}{2}
			%		\fontsize{6pt}{8pt}\selectfont
			\setbeamertemplate{section in toc}[sections numbered]
			\setbeamertemplate{subsection in toc}[subsections numbered]
			\tableofcontents
		\end{multicols}
	\end{frame}	
	%%%%%%%%%%%%%%%%%%%%%%%%%%%%%%%%%%%%%%%%%%%%%%%%%%%%%%%%%%%%%%%%%%%%%%%%%%%%
	\note{}
	%%%%%%%%%%%%%%%%%%%%%%%%%%%%%%%%%%%%%%%%%%%%%%%%%%%%%%%%%%%%%%%%%%%%%%%%%%%%
	\section{Motivation}
	\begin{frame}{Defects in composite laminates}
		\small
		Composite laminates can have different types of damage such as: \\
		\textbf{Cracks, fibre breakage, debonding, and \alert{delamination}}.
		\begin{columns}[T]
			\begin{column}[c]{.45\textwidth}
				\begin{itemize}
					\footnotesize
					\item Delamination is a critical failure mechanism in laminated fibre-reinforced polymer matrix composites.
					\item Delamination is one of the most hazardous forms of the defects. 
					It leads to very catastrophic failures if not detected at early stages.
				\end{itemize}
			\end{column}
			\begin{column}[c]{0.50\textwidth}
				\begin{figure}
					\includegraphics[width=.95\textwidth]{delaminated_plate1.jpg}
				\end{figure}
			\end{column}
		\end{columns}
	\end{frame}
	%%%%%%%%%%%%%%%%%%%%%%%%%%%%%%%%%%%%%%%%%%%%%%%%%%%%%%%%%%%%%%%%%%%%%%%%%%%%
	\note{note text}
	%%%%%%%%%%%%%%%%%%%%%%%%%%%%%%%%%%%%%%%%%%%%%%%%%%%%%%%%%%%%%%%%%%%%%%%%%%%%
	\section{Objectives}
	\begin{frame}{Objectives}
		\textbf{To develop \textcolor{blue}{a novel AI-driven diagnostic system} for delamination identification in composite laminates such as carbon fibre reinforced polymers (CFRP).}
		\vfil
		\textbf{To address the issue of \textcolor{blue}{slow data acquisition} by SLDV of high-resolution full wavefields of Lamb wave propagation.}
		\begin{alertblock}{Thesis}
			It is possible to use an end-to-end approach in which DNN 
			processes the animation of propagating waves (input) directly into a damage map (output).
		\end{alertblock}
	\end{frame}
	%%%%%%%%%%%%%%%%%%%%%%%%%%%%%%%%%%%%%%%%%%%%%%%%%%%%%%%%%%%%%%%%%%%%%%%%%%%%
	\note{note text}
	%%%%%%%%%%%%%%%%%%%%%%%%%%%%%%%%%%%%%%%%%%%%%%%%%%%%%%%%%%%%%%%%%%%%%%%%%%%%
	\section{SHM/NDE}
	%%%%%%%%%%%%%%%%%%%%%%%%%%%%%%%%%%%%%%%%%%%%%%%%%%%%%%%%%%%%%%%%%%%%%%%%%%%%
	%\subsection{Composite laminates}
%	\begin{frame}{What are Composite Laminates?}
%		\begin{columns}[T]
%			\begin{column}[c]{.3\textwidth}
%				\small
%				\begin{itemize}
%					\justifying
%					%			\item Composite laminates are usually composed of plies with different directions.
%					\item A \textcolor{blue}{ply} is made up of unidirectional continuous \textcolor{blue}{Fibres} held together by a \textcolor{blue}{Polymer matrix such as Epoxy resin} .			
%					\item The lay-up of the plies depends on the anticipated loading of the structure where the laminate will be used.
%				\end{itemize}
%			\end{column}
%			\hfill
%			\begin{column}[c]{.65\textwidth}
%				\begin{figure}
%					\includegraphics[width=.95\textwidth]{compsite_laminates.png}
%				\end{figure}
%			\end{column}
%		\end{columns}
%	\end{frame}
	%%%%%%%%%%%%%%%%%%%%%%%%%%%%%%%%%%%%%%%%%%%%%%%%%%%%%%%%%%%%%%%%%%%%%%%%%%%%%%%%

	%%%%%%%%%%%%%%%%%%%%%%%%%%%%%%%%%%%%%%%%%%%%%%%%%%%%%%%%%%%%%%%%%%%%%%%%%%%%%%%%
%	\section{SHM/NDE}
	%%%%%%%%%%%%%%%%%%%%%%%%%%%%%%%%%%%%%%%%%%%%%%%%%%%%%%%%%%%%%%%%%%%%%%%%%%%%%%%%
	\begin{frame}{Structural Health Monitoring (SHM)}
		\begin{figure}
			\includegraphics[height=.8\textheight]{SHM_system.png}
		\end{figure}
	\end{frame}
	%%%%%%%%%%%%%%%%%%%%%%%%%%%%%%%%%%%%%%%%%%%%%%%%%%%%%%%%%%%%%%%%%%%%%%%%%%%%
	\note{note text}
	%%%%%%%%%%%%%%%%%%%%%%%%%%%%%%%%%%%%%%%%%%%%%%%%%%%%%%%%%%%%%%%%%%%%%%%%%%%%
	\begin{frame}{Non Destructive Testing}
		\begin{columns}[T]
%			\only<1>{
%				\begin{column}[c]{0.32\textwidth}	
%					\begin{itemize}
%						\item \textbf{Visual inspection}
%						\item \textbf{Eddy current}
%						\item \textbf{Dye penetration}
%						\item \textbf{Acoustic emission}
%						\item \alert{\textbf{Ultrasonic testing}} \alert{\textbf{(local NDEs)}}
%						\item \alert{\textbf{Guided wave testing (global NDEs)}}
%					\end{itemize}
%			\end{column}}
			\only<1->{
			\begin{column}[c]{0.32\textwidth}
					\begin{itemize}
						\item[$\times$] \alert{Labor intensive}
						\item[$\times$] \alert{Time consuming}
						\item[$\times$] \alert{Need professional and experienced personnel}
						\item[$\times$] \alert{May require the disassembly of complex structures}
					\end{itemize}
%					\alert{\textbf{Local NDEs are not practical and do not fit for SHM}}
			\end{column}}
			\begin{column}[c]{0.65\textwidth}				
				\textbf{Ultrasonic testing} \hspace{50pt} \textbf{Guided wave testing}
				\begin{figure}
					\includegraphics[width=0.95\textwidth]{local_ultrasonic.png}
				\end{figure}
			\end{column}		
		\end{columns}	
	\end{frame}
	\begin{frame}{Ultrasonic testing vs guided wave testing}
		\alert{Bulk waves} exist in infinite homogeneous bodies and propagate indefinitely without being interrupted by boundaries or interfaces. 
		These waves can be decomposed into infinite plane waves propagating along arbitrary direction within the solid.
		
		\alert{Guided waves} are those waves that require a boundary for their existence, such as surface waves, Lamb waves, and interface waves.
		\vspace{5mm}
		\begin{columns}[T]
			\begin{column}{0.5\textwidth}
				\textbf{Ultrasonic waves}	
				\begin{itemize}
					\item Frequency range: 2 MHz - 200 MHz
					\item Wavelength \(\lambda << h\) thickness 
					\item shorter wavelengths
				\end{itemize}
			\end{column}
			\begin{column}{0.5\textwidth}
				\textbf{Guided waves}	
				\begin{itemize}
					\item Typical frequency range: 10 kHz - 1 MHz
					\item Wavelength \(\lambda > h\) thickness 
					\item longer wavelengths
				\end{itemize}
			\end{column}
		\end{columns}			
	\end{frame}
	%%%%%%%%%%%%%%%%%%%%%%%%%%%%%%%%%%%%%%%%%%%%%%%%%%%%%%%%%%%%%%%%%%%%%%%%%%%%
	\note{note text}
	%%%%%%%%%%%%%%%%%%%%%%%%%%%%%%%%%%%%%%%%%%%%%%%%%%%%%%%%%%%%%%%%%%%%%%%%%%%%
	\section{Guided waves}
	%%%%%%%%%%%%%%%%%%%%%%%%%%%%%%%%%%%%%%%%%%%%%%%%%%%%%%%%%%%%%%%%%%%%
	%%%%%%%%%%%%%%%%%%%%%%%%%%%%%%%%%%%%%%%%%%%%%%
	%\begin{frame}{Waves used in non-destructive testing}
	%	%%%%%%%%%%%%%%%%%%%%%%%%%%%%%%%%%%%%%%%%%%%%%%
	%	Elastic wave propagation types depending on particle motion:
	%	\begin{itemize}
		%		\item  \alert{The longitudinal wave} is a compressional wave in which the particle motion is in the same direction as the propagation of the wave
		%		\item \alert{The shear wave} is a wave motion in which the particle motion is perpendicular to the direction of the propagation
		%		\item \alert{Surface (Rayleigh) waves} have an elliptical particle motion and travel across the surface of a material. Their velocity is approximately 90\% of the shear wave velocity of the material and their depth of penetration is approximately equal to one
		%		wavelength
		%		\item \alert{Plate (Lamb) waves} have a complex vibration occurring in materials where thickness is less than the wavelength of elastic wave introduced into it.
		%	\end{itemize}
	%\end{frame}
	%%%%%%%%%%%%%%%%%%%%%%%%%%%%%%%%%%%%%%%%%%%%%%%%%%%
	%\setcounter{subfigure}{0}
	%\begin{frame}{Waves used in non-destructive testing}
	%	%%%%%%%%%%%%%%%%%%%%%%%%%%%%%%%%%%%%%%%%%%%%%%%%%%
	%	\begin{figure}
		%		\subfloat{\animategraphics[autoplay,loop, controls,width=0.5\textwidth]{10}{figures/gif_figs/Longitudinal_wave/Longitudinal_wave-}{0}{35}}
		%		\caption{\alert{Longitudinal wave} - plane pressure pulse wave}
		%	\end{figure}
	%\tiny 
	%(source: https://nojigon.webs.upv.es/index.php)
	%\end{frame}
	%%%%%%%%%%%%%%%%%%%%%%%%%%%%%%%%%%%%%%%%%%%%%%%%%%%%%%%%%%%%%%%%%%%%%%%%%%%%%%%%%
	%%%%%%%%%%%%%%%%%%%%%%%%%%%%%%%%%%%%%%%%%%%%%%%%%%%
	%\setcounter{subfigure}{0}
	%\begin{frame}{Waves used in non-destructive testing}
	%	%%%%%%%%%%%%%%%%%%%%%%%%%%%%%%%%%%%%%%%%%%%%%%%%%%
	%	\begin{columns}[T]
		%		\begin{column}{0.5\textwidth}
			%			\centering
			%			\begin{figure}
				%				\subfloat{\animategraphics[autoplay,loop, controls,width=0.95\textwidth]{10}{figures/gif_figs/SH_shear_wave/SH_shear-}{0}{39}}
				%				\caption{\alert{Shear horizontal wave}}
				%			\end{figure}			
			%		\end{column}
		%		\begin{column}{0.5\textwidth}
			%			\centering
			%			\begin{figure}
				%				\subfloat{\animategraphics[autoplay,loop, controls,width=0.95\textwidth]{10}{figures/gif_figs/SV_shear_wave/SV_shear-}{0}{39}}
				%				\caption{\alert{Shear vertical wave}}
				%			\end{figure}			
			%		\end{column}	
		%	\end{columns}
	%\tiny 
	%(source: https://nojigon.webs.upv.es/index.php)
	%\end{frame}
	%%%%%%%%%%%%%%%%%%%%%%%%%%%%%%%%%%%%%%%%%%%%%%%%%%%
	%\setcounter{subfigure}{0}
	%\begin{frame}{Waves used in non-destructive testing}
	%	%%%%%%%%%%%%%%%%%%%%%%%%%%%%%%%%%%%%%%%%%%%%%%%%%%
	%	\begin{figure}
		%		\centering
		%		\subfloat{\animategraphics[autoplay,loop, controls,height=0.65\textheight]{15}{figures/gif_figs/Raileigh_wave/Raileigh_wave-}{0}{67}}
		%		\caption{\alert{Rayleigh waves}}		
		%	\end{figure}			
	%\tiny 
	%(source: https://nojigon.webs.upv.es/index.php)
	%\end{frame}
	%%%%%%%%%%%%%%%%%%%%%%%%%%%%%%%%%%%%%%%%%%%%%%%
	%\setcounter{subfigure}{0}
	%\begin{frame}{Waves used in non-destructive testing}
	%	%%%%%%%%%%%%%%%%%%%%%%%%%%%%%%%%%%%%%%%%%%%%%%
	%	\begin{figure}			
		%		\centering
		%		\subfloat{\animategraphics[autoplay,loop, controls,height=0.65\textheight]{20}{figures/gif_figs/love_wave/love_wave-}{0}{107}}
		%		\caption{\alert{Love waves} (surface seismic waves) named after Augustus Edward Hough Love}		
		%	\end{figure}			
	%\tiny 
	%(source: https://nojigon.webs.upv.es/index.php)
	%\end{frame}
	%%%%%%%%%%%%%%%%%%%%%%%%%%%%%%%%%%%%%%%%%%%%%%
	\setcounter{subfigure}{0}
	\begin{frame}{Lamb waves}
		%%%%%%%%%%%%%%%%%%%%%%%%%%%%%%%%%%%%%%%%%%%%%%
		\begin{alertblock}{Lamb waves}	
			Lamb waves are plane waves propagating in thin plates.\\
			Shear vertical waves in conjunction with longitudinal P waves interacts with plate surfaces resulting in complex wave mechanism which leads to creation of Lamb waves.
		\end{alertblock}
		Horace Lamb discovered these type of waves in 1917.
		He derived theory and dispersion relations.
		\begin{columns}[T]
			\begin{column}{0.5\textwidth}
				\centering
				symmetric modes
				\begin{equation*}
					\frac{\tan(q h)}{\tan(p h)} = -\frac{4 k^2 p q}{\left(q^2 - k^2\right)^2}
				\end{equation*}
			\end{column}
			\begin{column}{0.5\textwidth}
				\centering
				antisymmetric modes
				\begin{equation*}
					\frac{\tan(q h)}{\tan(p h)} = -\frac{\left(q^2 - k^2\right)^2}{4 k^2 p q}
				\end{equation*}
			\end{column}	
		\end{columns}	
		\centering
%		\(q=q(\omega,k), \quad p=p(\omega,k) \)
		\begin{gather*}
			\centering
			p^2 = \frac{\omega^2}{c_{L}^2}-k^2,\ q^2 = \frac{\omega^2}{c_{S}^2}-k^2,\ k = \frac{2\pi}{\lambda},\ f=\frac{\omega}{2\pi}
		\end{gather*}
		\newline
		\begin{gather*}
			\centering
			c_L=\sqrt{\frac{2\mu (1-\nu)}{\rho(1-2\nu)}},\ c_S=\sqrt{\frac{\mu}{\rho}}
		\end{gather*}
			
		
%		\(p^2 = \frac{\omega^2}{c_{L}^2}-k^2,\ q^2 = \frac{\omega^2}{c_{S}^2}-k^2,\ k = \frac{2\pi}{\lambda},\ f=\frac{\omega}{2\pi} \newline
%		c_L=\sqrt{\frac{2\mu (1-\nu)}{\rho(1-2\nu)}},\ c_S=\sqrt{\frac{\mu}{\rho}} \)
	\end{frame}
	%%%%%%%%%%%%%%%%%%%%%%%%%%%%%%%%%%%%%%%%%%%%%%%%%%%%%%%%%%%%%%%%%%%%%%%%%%%%
	\note{note text}
	%%%%%%%%%%%%%%%%%%%%%%%%%%%%%%%%%%%%%%%%%%%%%%%%%%%%%%%%%%%%%%%%%%%%%%%%%%%%
	\setcounter{subfigure}{0}
	\begin{frame}{Lamb waves modes}
		%%%%%%%%%%%%%%%%%%%%%%%%%%%%%%%%%%%%%%%%%%%%%%
		\begin{columns}[T]
			\begin{column}{0.3\textwidth}
				\centering
				\begin{figure}
					\animategraphics[autoplay,loop,width=1\textwidth]{10}{figures/gif_figs/S0_mode/S0_mode-}{0}{67}
					\caption{Fundamental symmetric, S0, \alert{Lamb wave} mode (in-plane motion)}
				\end{figure}
			\end{column}
			\begin{column}{0.3\textwidth}
				\centering
				\begin{figure}
					\animategraphics[autoplay,loop,width=1\textwidth]{10}{figures/gif_figs/A0_mode/A0_mode-}{0}{67}
					\caption{Fundamental antisymmetric, A0, \alert{Lamb wave} mode (out-of-plane motion)}
				\end{figure}
			\end{column}
			\hfill
			\begin{column}{0.37\textwidth}
					\begin{itemize}
						\item \textcolor{blue}{Travel within guides for long distances}
						\item \textcolor{blue}{Can propagate in complex structures}
						\item \textcolor{blue}{High speeds in metals and composites}
						\item \textcolor{blue}{Can be automated using software}
					\end{itemize}
					\textbf{Lamb waves are a promising global NDE solution for SHM}
			\end{column}
		\end{columns}	
		\tiny 
		(source: https://nojigon.webs.upv.es/index.php)
	\end{frame}
	%%%%%%%%%%%%%%%%%%%%%%%%%%%%%%%%%%%%%%%%%%%%%%%%%%%%%%%%%%%%%%%%%%%%%%%%%%%%
	\note{note text}
	%%%%%%%%%%%%%%%%%%%%%%%%%%%%%%%%%%%%%%%%%%%%%%%%%%%%%%%%%%%%%%%%%%%%%%%%%%%%
	
%	\begin{frame}{Dispersion curves of Lamb waves}
%		%%%%%%%%%%%%%%%%%%%%%%%%%%%%%%%%%%%%%%%%%%%%%%
%		\begin{figure}
%			\only<1>{
%				\includegraphics[width=0.8\textwidth]{/figs/Fig_1_12.png}	
%			}
%			\only<2>{
%				\includegraphics[width=0.8\textwidth]{/figs/Fig_1_13.png}	
%			}
%		\end{figure}
%	\end{frame}
	%%%%%%%%%%%%%%%%%%%%%%%%%%%%%%%%%%%%%%%%%%%%%%%%%%%%%%%%%%%%%%%%%%%%%%%%%%%%%%%%
	\section{Damage detection approaches}
	%%%%%%%%%%%%%%%%%%%%%%%%%%%%%%%%%%%%%%%%%%%%%%%%%%%%%%%%%%%%%%%%%%%%%%%%%%%%
	\begin{frame}{Conventional vs deep learning approach}
		Conventional methods involve two processes:
		\alert{\textbf{Feature extraction and classification}}
%		\begin{itemize}
%			\item \alert{\textbf{Feature extraction}} \(\rightarrow\) includes signals preprocessing (signal denoising and averaging), then extracting \textbf{damage indexes (DIs)}.
%			\item \alert{\textbf{Feature classification}}\(\rightarrow\) the extracted features are classified, for instance, into healthy or damaged states.
%		\end{itemize}
		\begin{figure}
			\centering
			\includegraphics[width=.95\textwidth]{conventional_ML.png}
		\end{figure}	
%		\textbf{Drawbacks of Conventional methods:}
%		\begin{itemize}
%			\item[$\times$]\alert{Requires a great amount of human labor and computational effort.}
%			\item[$\times$]\alert{Demands a high amount of experience of the practitioner.}
%			\item[$\times$]\alert{Inefficient with big data which requires a complex computation of damage features.} 
%		\end{itemize}
	
		Deep learning offers an \alert{\textbf{end-to-end}} approach: \alert{\textbf{Automatic}} feature extraction and classification.
		\begin{figure}
			\includegraphics[width=.95\textwidth]{DL_approach.png}
		\end{figure}
	\end{frame}
	%%%%%%%%%%%%%%%%%%%%%%%%%%%%%%%%%%%%%%%%%%%%%%%%%%%%%%%%%%%%%%%%%%%%%%%%%%%%
	\note{note text}
	%%%%%%%%%%%%%%%%%%%%%%%%%%%%%%%%%%%%%%%%%%%%%%%%%%%%%%%%%%%%%%%%%%%%%%%%%%%%
	
	%%%%
	%\setcounter{subfigure}{0}
	%\section{Artificial intelligence, machine learning, and deep learning}
	%%%%%%%%%%%%%%%%%%%%%%%%%%%%%%%%%%%%%%%%%%%%%%%%%%%
	%%%%%%%%%%%%%%%%%%%%%%%%%%%%%%%%%%%%%%%%%%%%%%%%%%%
	%\begin{frame}{What is deep learning?}
	%	\begin{figure}
		%		\centering
		%		\includegraphics[width=0.85\textwidth]{AI_vs_ML_vs_Deep_Learning.png}
		%	\end{figure}
	%	\tiny
	%	(source: https://www.ingeniovirtual.com/)
	%\end{frame}
	%
	%%%%%%%%%%%%%%%%%%%%%%%%%%%%%%%%%%%%%%%%%%%%%%%%%%%%%%%%%%%%%%%%%%%%%%%%%%%%%%%%%
	%\setcounter{subfigure}{0}
	%%%%%%%%%%%%%%%%%%%%%%%%%%%%%%%%%%%%%%%%%%%%%%%%%%%
	%\begin{frame}{Deep learning, why now?}
	%	\begin{column}[c]{0.4\textwidth}
		%		AI technologies are in accelerating growth due to:
		%		\begin{itemize}
			%			\item Exponential development in computer hardware industries
			%			 (e.g. CPUs, GPUs, FPGAs, TPUs and ASICs)
			%			\item Era of Big data.
			%		\end{itemize}
		%	\end{column}
	%	\begin{column}[c]{0.55\textwidth}
		%		\begin{figure}
			%			\centering
			%			\subfloat{\animategraphics[autoplay,loop,width=.9\textwidth]{10}{gif_figs/gpu/gpu_-}{0}{34}}
			%		\end{figure}
		%	\tiny
		%	(source: https://www.techbooky.com/)
		%	\end{column}
	%	
	%\end{frame}
	%%%%%%%%%%%%%%%%%%%%%%%%%%%%%%%%%%%%%%%%%%%%%%%%%%%
	%\setcounter{subfigure}{0}
	%%%%%%%%%%%%%%%%%%%%%%%%%%%%%%%%%%%%%%%%%%%%%%%%%%%
	%\begin{frame}{Common learning strategies}
	%	\centering
	%	\begin{figure}
		%		\includegraphics[width=0.9\textwidth]{learning.png}
		%	\end{figure}
	%	\tiny
	%	(source: https://www.aitude.com/supervised-vs-unsupervised-vs-reinforcement/)
	%\end{frame}
	%%%%%%%%%%%%%%%%%%%%%%%%%%%%%%%%%%%%%%%%%%%%%%%%%%%%%%%%%%%%%%%%%%%%%%%%%%%%%%%%%%
	
	%%%%%%%%%%%%%%%%%%%%%%%%%%%%%%%%%%%%%%%%%%%%%%%%%%%%%%
	\section{DL-based approaches for damage identification}
	\setcounter{subfigure}{0}
%	\begin{frame}{Deep learning approach}
%		\textbf{Deep learning (DL) technologies are in accelerating growth due to:}
%		\begin{itemize}
%			\item \textcolor{blue}{Exponential development in computer hardware/software industries.}
%			\item \textcolor{blue}{Machine learning algorithms.}
%			\item \textcolor{blue}{Era of Big data.}
%		\end{itemize}	
%		Deep learning offers an \alert{\textbf{end-to-end}} approach: \alert{\textbf{Automatic}} feature extraction and classification.
%		\begin{figure}
%			\includegraphics[width=.95\textwidth]{DL_approach.png}
%		\end{figure}
%	\end{frame}

	\setcounter{subfigure}{0}
	\begin{frame}{Supervised deep learning}
		\begin{columns}[T]
			\begin{column}[t]{.4\textwidth}
				\begin{itemize}
					\item \alert{Labeled data} $\rightarrow$ (input data has labels/ground truths)
					\item \alert{Loss function} $\rightarrow$ measures error between predicted and the ground truth values
					\item \alert{Learnable parameters} $\rightarrow$ updated during the backpropagation step (optimization e.g. Gradient descent )					
				\end{itemize}
			\end{column}
			\begin{column}[t]{.55\textwidth}
				\begin{figure}[t]
					\centering
					\animategraphics[autoplay,loop,width =1.0\textwidth]{1}{figures/gif_figs/BP/png/BP_technique_}{0}{14}
				\end{figure}
			\end{column}
		\end{columns}		
	\end{frame}
	%%%%%%%%%%%%%%%%%%%%%%%%%%%%%%%%%%%%%%%%%%%%%%%%%%%%%%%%%%%%%%%%%%%%%%%%%%%%
	\note{note text}
	%%%%%%%%%%%%%%%%%%%%%%%%%%%%%%%%%%%%%%%%%%%%%%%%%%%%%%%%%%%%%%%%%%%%%%%%%%%%
	\setcounter{subfigure}{0}
	\begin{frame}{Computer vision}
		\begin{columns}[T]
			\begin{column}[c]{0.27\textwidth}
				\justifying
				\alert {\textbf{Computer vision}} is a field of AI that enables computers and systems to derive meaningful information from digital images, videos and other visual inputs. 
			\end{column}
			\quad
			\begin{column}[c]{0.7\textwidth}
				\begin{figure}
					\centering
					\includegraphics[width=1\textwidth]{computer_vision_tasks.png}
				\end{figure}
			\end{column}
		\end{columns}
	\end{frame}
	%%%%%%%%%%%%%%%%%%%%%%%%%%%%%%%%%%%%%%%%%%%%%%%%%%%%%%%%%%%%%%%%%%%%%%%%%%%%
	\note{note text}
	%%%%%%%%%%%%%%%%%%%%%%%%%%%%%%%%%%%%%%%%%%%%%%%%%%%%%%%%%%%%%%%%%%%%%%%%%%%%
	\begin{frame}{Convolution operation (Feature extractor)}
		\noindent Convolution in DL is a cross-correlation operation \alert{(sliding dot product)}. \\
		The kernel weights \alert{(learnable weights)} are updated during training phase.
		\begin{figure}[t]
			\centering
			\animategraphics[autoplay,loop,width =1.0\textwidth]{2}{figures/gif_figs/files/plot_convolution_process_}{0}{32}
		\end{figure}
	\end{frame}
	%%%%%%%%%%%%%%%%%%%%%%%%%%%%%%%%%%%%%%%%%%%%%%%%%%%%%%%%%%%%%%%%%%%%%%%%%%%%
	\note{note text}
	%%%%%%%%%%%%%%%%%%%%%%%%%%%%%%%%%%%%%%%%%%%%%%%%%%%%%%%%%%%%%%%%%%%%%%%%%%%%
	\setcounter{subfigure}{0}
	\begin{frame}{Encoder-decoder (Autoencoder)}
		\begin{columns}[T]
			\begin{column}[t]{.35\textwidth}
				\begin{itemize}
					\item \alert{Encoder} learns to extract features.
					\item \alert{Latent space} has condensed feature maps.
					\item \alert{Decoder} learns to locate the features learned by the encoder.
				\end{itemize}	
			\end{column}
			\hfill
			\begin{column}[t]{.6\textwidth}
				\begin{figure}
					\centering
					\includegraphics[width=1\textwidth]{nn_encoder_decoder.png}
				\end{figure}	
			\end{column}
		\end{columns}			
	\end{frame}
	%%%%%%%%%%%%%%%%%%%%%%%%%%%%%%%%%%%%%%%%%%%%%%%%%%%%%%%%%%%%%%%%%%%%%%%%%%%%
	\note{note text}
	%%%%%%%%%%%%%%%%%%%%%%%%%%%%%%%%%%%%%%%%%%%%%%%%%%%%%%%%%%%%%%%%%%%%%%%%%%%%
	\setcounter{subfigure}{0}
	\section{Synthetic dataset generation}
	%%%%%%%%%%%%%%%%%%%%%%%%%%%%%%%%%%%%%%%%%%%%%%%%%%
	%\subsection{Synthetic Dataset of propagating Lamb waves}
	%\setcounter{subfigure}{0}
	%%%%%%%%%%%%%%%%%%%%%%%%%%%%%%%%%%%%%%%%%%%%%%%%%%
	%%%%%%%%%%%%%%%%%%%%%%%%%%%%%%%%%%%%%%%%%%%%%%%%%%%%%%%%%%%%%%%%%%%%
	\begin{frame}{The time domain spectral element method}
	\begin{columns}[T]
		\begin{column}{0.47\textwidth}
			\begin{itemize}
				\item Mindlin-Reissner plate theory
				\item Splitting elements and nodes at delamination
				\item GMSH software was used for meshing quads than converted to spectral elements
			\end{itemize}	
			\begin{figure}
				\subfloat{\includegraphics[width=0.9\textwidth]{shell.png}}	
			\end{figure}
		\end{column}
		\begin{column}{0.47\textwidth}	
			\begin{figure}
				\animategraphics[controls,autoplay,loop,width=0.9\textwidth]{1}{/gif_figs/mesh/m1_rand_single_delam_}{1}{20}
			\end{figure}	
		\end{column}
	\end{columns}	
	\end{frame}
	%%%%%%%%%%%%%%%%%%%%%%%%%%%%%%%%%%%%%%%%%%%%%%%%%%%%%%%%%%%%%%%%%%%%%%%%%%%%
	\note{note text}
	%%%%%%%%%%%%%%%%%%%%%%%%%%%%%%%%%%%%%%%%%%%%%%%%%%%%%%%%%%%%%%%%%%%%%%%%%%%%
	\begin{frame}{Dataset description}
		\begin{columns}[T]
			\only<1-2>{\begin{column}[c]{0.48\textwidth}
					\begin{itemize}
						\item 475 delamination scenarios
						\item CFRP is made of 8-layers
						\item Delamination modelled between the 3rd and 4th layer
						\item Delamination size min 10 mm, max  40 mm
						\item \textbf{3-months of computing}
					\end{itemize}
			\end{column}}
			\only<1>{\begin{column}[c]{0.48\textwidth}
					\begin{figure}					
						\centering
%						\only<1>{\subfloat[All random delaminations]{\includegraphics[width=0.7\textwidth]{figure_overlap.png}}}
						\subfloat[Delamination placement]{\includegraphics[width=0.8\textwidth]{delamination_placement.png}}
					\end{figure}
			\end{column}}
			\only<2>{\begin{column}[c]{0.48\textwidth}
					\begin{figure}
						\centering
						\subfloat[Delamination orientation]{\includegraphics[width=0.90\textwidth]{figure1.png}}						
					\end{figure}
			\end{column}}
		\end{columns}
	\end{frame}
	%%%%%%%%%%%%%%%%%%%%%%%%%%%%%%%%%%%%%%%%%%%%%%%%%%%%%%%%%%%%%%%%%%%%%%%%%%%%
	\note{note text}
	%%%%%%%%%%%%%%%%%%%%%%%%%%%%%%%%%%%%%%%%%%%%%%%%%%%%%%%%%%%%%%%%%%%%%%%%%%%%
	\setcounter{subfigure}{0}
	\begin{frame}{Training Sample case}
		\begin{columns}[T]
			\begin{column}[c]{.32\textwidth}
				\begin{figure}
					\centering
					\animategraphics[autoplay,loop,width=0.95 \textwidth]{16}{figures/gif_figs/7_output/flat_shell_Vz_7_500x500bottom-}{1}{512}
					\caption{Full wavefield $s(x,y,t_k)$}
				\end{figure}
			\end{column}
			\begin{column}[c]{.32\textwidth}
				\begin{figure}
					\centering
					\includegraphics[width=0.95 \textwidth]{RMS_flat_shell_Vz_7_500x500bottom.png}
					\caption{RMS image $\hat{s}(x,y)$}
				\end{figure}
			\end{column}
			\begin{column}[c]{.32\textwidth}
				\begin{figure}
					\centering
					\includegraphics[width=0.95 \textwidth]{m1_rand_single_delam_7.png}
					\caption{Ground truth (label)}
				\end{figure}
			\end{column}
		\end{columns}
		The RMS is defined as:
		\begin{equation*}
			\hat{s}(x,y) = \sqrt{\frac{1}{N}\sum_{k=1}^{N}s(x,y,t_k)^2} 
			\label{eqn:rms} 
		\end{equation*}
	\end{frame}
	%%%%%%%%%%%%%%%%%%%%%%%%%%%%%%%%%%%%%%%%%%%%%%%%%%%%%%%%%%%%%%%%%%%%%%%%%%%%
	\note{note text}
	%%%%%%%%%%%%%%%%%%%%%%%%%%%%%%%%%%%%%%%%%%%%%%%%%%%%%%%%%%%%%%%%%%%%%%%%%%%%
	\setcounter{subfigure}{0}
	\section{Semantic segmentation}
	\begin{frame}{Semantic segmentation}
		\begin{columns}[T]
			\begin{column}[c]{0.47\textwidth}
				\centering
				\textbf{One-to-one \\image-based approach (RMS)} 
				\begin{figure}
					\centering
					\subfloat[Single input (image)]{\includegraphics[width=.45\textwidth]{RMS_flat_shell_Vz_381_500x500bottom.png}}\quad
					\subfloat[Single output]{\includegraphics[width=.45\textwidth]{GCN_381.png}}
				\end{figure}
			\end{column}
			\hfill
			\begin{column}[c]{0.47\textwidth}
				\centering
				\textbf{Many-to-one \\animation-based approach}
				\begin{figure}
					\centering
					\subfloat[Multiple frames animation]{\animategraphics[autoplay,loop,width=.45\textwidth]{4}{figures/gif_figs/381_output/output_381-}{85}{113}}\quad
					\subfloat[Single output]{\includegraphics[width=.45\textwidth]{GCN_381.png}}
				\end{figure}
			\end{column}	
		\end{columns}		
	\end{frame}	
	%%%%%%%%%%%%%%%%%%%%%%%%%%%%%%%%%%%%%%%%%%%%%%%%%%%%%%%%%%%%%%%%%%%%%%%%%%%%
	\note{note text}
	%%%%%%%%%%%%%%%%%%%%%%%%%%%%%%%%%%%%%%%%%%%%%%%%%%%%%%%%%%%%%%%%%%%%%%%%%%%%
%	\subsection{Developed DL models}
	%\begin{frame}{Common deep learning architectures}
	%	
	%	\begin{column}[t]{0.45\textwidth}
		%		\textbf{RMS based}\\
		%		\begin{itemize}
			%			\item Convolutional neural networks (CNN)
			%			\item Fully convolutional network (FCN)
			%		\end{itemize}
		%	\end{column}
	%	\hfill
	%	\begin{column}[t]{0.45\textwidth}
		%		\textbf{Full wavefield frames}\\
		%		\begin{itemize}
			%			\item Recurrent neural network (RNN)
			%			\item Long short-term memory (LSTM)
			%			\item ConvLSTM
			%		\end{itemize}
		%	\end{column}
	%\end{frame}
	%%%%%%%%%%%%%%%%%%%%%%%%%%%%%%%%%%%%%%%%%%%%%%%%%%
	\begin{frame}{Developed model for delamination identification}
		\begin{columns}[T]
			\begin{column}[t]{0.45\textwidth}
				\textbf{RMS based models: }
				\medskip
				\begin{itemize}
					\item Res-UNet
					\item VGG 16 encoder-decoder
					\item FCN-DenseNet
					\item PSPNet
					\item GCN
				\end{itemize}
				\biblioref{Ijjeh, A. A., Kudela P.}{2022}{Deep learning based segmentation using full wavefield processing for delamination identification: A comparative study} {Mechanical Systems and Signal Processing. 168:108671}
				\biblioref{Ijjeh, A. A., Ullah, S., Kudela P.}{2021}{Full wavefield processing by using FCN for delamination detection}{Mechanical Systems and Signal Processing.  153:107537} 

			\end{column}
			\hfill
			\begin{column}[t]{.45\textwidth}
				\textbf{Full wavefield frames based model:}
				\begin{itemize}
					\item Autoencoder ConvLSTM
				\end{itemize}
				\biblioref{Ullah, S., Ijjeh, A. A., Kudela, P.}{2023}{Deep learning approach for delamination identification using animation of Lamb waves}{Engineering Applications of Artificial Intelligence, 117, 105520}
			\end{column}
		\end{columns}
	\end{frame}	
	%%%%%%%%%%%%%%%%%%%%%%%%%%%%%%%%%%%%%%%%%%%%%%%%%%%%%%%%%%%%%%%%%%%%%%%%%%%%
	\note{note text}
	%%%%%%%%%%%%%%%%%%%%%%%%%%%%%%%%%%%%%%%%%%%%%%%%%%%%%%%%%%%%%%%%%%%%%%%%%%%%

	\setcounter{subfigure}{0}
	\subsection{Developed RMS based models}	
	\begin{frame}{Residual UNet}
		\begin{figure}
			\centering
			\includegraphics[width=.6\textwidth]{figure4.png}
		\end{figure}
	\end{frame}
	%%%%%%%%%%%%%%%%%%%%%%%%%%%%%%%%%%%%%%%%%%%%%%%%%%%%%%%%%%%%%%%%%%%%%%%%%%%%
	\note{note text}
	%%%%%%%%%%%%%%%%%%%%%%%%%%%%%%%%%%%%%%%%%%%%%%%%%%%%%%%%%%%%%%%%%%%%%%%%%%%%
	\begin{frame}{VGG16 encoder-decoder}
		\begin{figure}
			\centering
			\includegraphics[width=.6\textwidth]{figure5.png}
		\end{figure}
	\end{frame}
	%%%%%%%%%%%%%%%%%%%%%%%%%%%%%%%%%%%%%%%%%%%%%%%%%%%%%%%%%%%%%%%%%%%%%%%%%%%%
	\note{note text}
	%%%%%%%%%%%%%%%%%%%%%%%%%%%%%%%%%%%%%%%%%%%%%%%%%%%%%%%%%%%%%%%%%%%%%%%%%%%%
	\begin{frame}{FCN-DenseNet}
		\begin{columns}[T]
			\begin{column}[c]{0.48\textwidth}
				\begin{figure}[h!]
					\includegraphics[height=.8\textheight]{FCN_dense_net.png}
					\caption{FCN-DenseNet architecture.} 
					\label{fcn}
				\end{figure}
			\end{column}
			\hfill
			\begin{column}[c]{0.48\textwidth}
				\begin{figure}[h!]
					\centering
					\includegraphics[width=0.5\textwidth,angle=-90]{figure6.png}
					\caption{Dense block architecture.} 
				\end{figure}
			\end{column}
		\end{columns}
	\end{frame}
	%%%%%%%%%%%%%%%%%%%%%%%%%%%%%%%%%%%%%%%%%%%%%%%%%%%%%%%%%%%%%%%%%%%%%%%%%%%%
	\note{note text}
	%%%%%%%%%%%%%%%%%%%%%%%%%%%%%%%%%%%%%%%%%%%%%%%%%%%%%%%%%%%%%%%%%%%%%%%%%%%%
	
	\begin{frame}{Pyramid Scene Parsing Network}
		\begin{figure} [h!]
			\centering
			\includegraphics[height=.7\textheight]{figure7.png}
			\caption{PSPNet architecture.} 
		\end{figure}
	\end{frame}
	%%%%%%%%%%%%%%%%%%%%%%%%%%%%%%%%%%%%%%%%%%%%%%%%%%%%%%%%%%%%%%%%%%%%%%%%%%%%
	\note{note text}
	%%%%%%%%%%%%%%%%%%%%%%%%%%%%%%%%%%%%%%%%%%%%%%%%%%%%%%%%%%%%%%%%%%%%%%%%%%%%
	
	\begin{frame}{Global Convolution Network}
		\begin{columns}[T]
			\begin{column}[c]{0.55\textwidth}
				\begin{figure}
					\centering
					\includegraphics[width=.75\textwidth]{figure8.png}
					\caption{GCN architecture.} 
				\end{figure}	
			\end{column}
			\begin{column}[c]{0.45\textwidth}
				\begin{figure}
					\centering
					\includegraphics[width=.9\textwidth]{figure9.png}
					\caption{(a) Residual block, (b) GCN block, (c) Boundary Refinement.} 
				\end{figure}	
			\end{column}
		\end{columns}
	\end{frame}	
	%%%%%%%%%%%%%%%%%%%%%%%%%%%%%%%%%%%%%%%%%%%%%%%%%%%%%%%%%%%%%%%%%%%%%%%%%%%%
	\note{note text}
	%%%%%%%%%%%%%%%%%%%%%%%%%%%%%%%%%%%%%%%%%%%%%%%%%%%%%%%%%%%%%%%%%%%%%%%%%%%%
	
	%%%%%%%%%%%%%%%%%%%%%%%%%%%%%%%%%%%%%%%%%%%%%%%%%%%%%%%%%%%%%%%%%%%%
	%\setcounter{subfigure}{0}
	%\begin{frame}{RMS based models}
	%	\begin{column}[c]{0.55\textwidth}
		%		\begin{figure}
			%			\subfloat[Res-UNet model]{\includegraphics[width=1\textwidth]{figure4.png}}
			%		\end{figure}
		%	\end{column}
	%	\begin{column}[c]{0.35\textwidth}
		%		\begin{figure}
			%			\subfloat[Data flow \& intermediate outputs of layers \label{fig:}]{\animategraphics[autoplay, controls,width=.8\textwidth]{4}{figures/gif_figs/381__inter_pred/intermediate_output-}{0}{103}}
			%\end{figure}
			%	\end{column}
		%
		%\end{frame}
			
	\setcounter{subfigure}{0}
	%%%%%%%%%%%%%%%%%%%%%%%%%%%%%%%%%%%%%%%%%%%%%%%%%%%%%%%%%%%%%%%%%%%	
	\subsection{Full wavefield frames based model}
	\begin{frame}{Autoencoder ConvLSTM}
		\begin{columns}[T]		
			\begin{column}[c]{0.5\textwidth}
				\only<1>{
					\begin{figure}[c]
						\centering
						\subfloat{\includegraphics[width=.8\textwidth]{figure2.png}}
						\caption{Sample frames of full wave propagation.}
				\end{figure}}
				\only<2->{
					\begin{figure}[c]
						\centering
						\subfloat{\includegraphics[height=.6\textheight]{figure3.png}}
						\caption{The procedure of calculating the RMS prediction image (damage map).}
					\end{figure}}
				\end{column}
			\begin{column}[c]{0.45\textwidth}
				\begin{figure}
					\centering
					\subfloat{\includegraphics[width=.8\textheight]{figure5b.png}}
					\caption{Autoencoder ConvLSTM model}
				\end{figure}
			\end{column}
		\end{columns}
	\end{frame}
	%%%%%%%%%%%%%%%%%%%%%%%%%%%%%%%%%%%%%%%%%%%%%%%%%%%%%%%%%%%%%%%%%%%%%%%%%%%%
	\note{note text}
	%%%%%%%%%%%%%%%%%%%%%%%%%%%%%%%%%%%%%%%%%%%%%%%%%%%%%%%%%%%%%%%%%%%%%%%%%%%%
	\begin{frame}{Evaluation metrics for delamination identification}
		\begin{columns}[T]
			\begin{column}[c]{0.45\textwidth}
				For evaluating delamination identification
				\begin{itemize}
					\item Intersection over Union (IoU): 
					\begin{equation*}
						\textup{IoU}=\frac{Intersection}{Union}=\frac{\hat{Y} \cap Y}{\hat{Y} \cup Y}
						\label{eqn:iou}
					\end{equation*}
					\item Percentage area error $\epsilon$:
					\begin{equation*}
						\epsilon=\frac{|A-\hat{A}|}{A} \times 100\%
						\label{eqn:mean_size_error}
					\end{equation*}
				\end{itemize}
			\end{column}
			\begin{column}[c]{0.45\textwidth}
				\begin{figure}
					\centering
					\includegraphics[width=1.0\textwidth]{IoU_figure.png}		
				\end{figure}
			\end{column}
		\end{columns}
	\end{frame}
	%%%%%%%%%%%%%%%%%%%%%%%%%%%%%%%%%%%%%%%%%%%%%%%%%%%%%%%%%%%%%%%%%%%%%%%%%%%%
	\note{note text}
	%%%%%%%%%%%%%%%%%%%%%%%%%%%%%%%%%%%%%%%%%%%%%%%%%%%%%%%%%%%%%%%%%%%%%%%%%%%%
	\section{Evaluation}
	\subsection{Numerical test cases}
	%%%%%%%%%%%%%%%%%%%%%%%%%%%%%%%%%%%%%%%%%%%%%%%%%%%%%%%%%%%%%%%%
	\begin{frame}{Numerical test cases RMS based models (GCN model)}
		\begin{columns}[T]
			\begin{column}[c]{0.32\textwidth}
				\begin{figure}[c]
					\centering
					\animategraphics[controls,width=.9\textwidth]{2}{figures/gif_figs/456/intermediate_output-}{0}{82}
					\caption{\(1^{st}\) numerical case, IoU=0.71}
				\end{figure}
			\end{column}
			\hfill
			\begin{column}[c]{0.32\textwidth}
				\begin{figure}[c]
					\centering
					\animategraphics[controls,width=.9\textwidth]{2}{figures/gif_figs/438/intermediate_output-}{0}{82}
					\caption{\(2^{nd}\) numerical case, IoU=0.72}
				\end{figure}
			\end{column}
			\hfill
			\begin{column}[c]{0.32\textwidth}
				\begin{figure}[c]
					\centering
					\animategraphics[controls,width=.9\textwidth]{2}{figures/gif_figs/397/intermediate_output-}{0}{82}
					\caption{\(3^{rd}\) numerical case, IoU=0.86}
				\end{figure}					
			\end{column}
		\end{columns}
	\end{frame}
	%%%%%%%%%%%%%%%%%%%%%%%%%%%%%%%%%%%%%%%%%%%%%%%%%%%%%%%%%%%%%%%%%%%%%%%%%%%%
	\note{note text}
	%%%%%%%%%%%%%%%%%%%%%%%%%%%%%%%%%%%%%%%%%%%%%%%%%%%%%%%%%%%%%%%%%%%%%%%%%%%%
	\begin{frame}{RMS based: Analysis of numerical cases}
		\begin{columns}[T]
%				\tiny
%				\begin{column}[c]{0.48\textwidth}
%					%%%%%%%%%%%%%%%%%%%%%%%%%%%%%%%%%%%%%%%%%%%%%%%%%%%%%%%%%%%%
%					\begin{table}[ht!]
%						\centering
%						\caption{Evaluation metrics of the three numerical cases.}
%						\label{tab:RMS_num_cases}
%						\begin{tabular}{cccccc}
%							\toprule[1.5pt]
%							\multirow{2}{*}{Model} & \multirow{2}{*}{case number} & \multicolumn{1}{c}{\multirow{2}{*}{A [mm\textsuperscript{2}]}} & \multicolumn{3}{c}{Predicted output} \\ 
%							\cmidrule(lr){4-6} & & & \multicolumn{1}{c}{IoU} & \multicolumn{1}{c}{\(\hat{A}\) [mm\textsuperscript{2}]} & \(\epsilon\) \\
%							\midrule
%							\multirow{3}{*}{Res-UNet} 							
%							& 1 & 257 & \multicolumn{1}{c}{0.45} & \multicolumn{1}{c}{143} & \(44.36\%\) \\ 
%							& 2 & 105 & \multicolumn{1}{c}{0.67} & \multicolumn{1}{c}{88} & \(16.19\%\) \\ 
%							& 3 & 537 & \multicolumn{1}{c}{0.80} & \multicolumn{1}{c}{478} & \(10.99\%\) \\ 
%							\midrule
%							\multirow{3}{*}{VGG16 encoder-decoder} 
%							& 1 & 257 & \multicolumn{1}{c}{0.69} & \multicolumn{1}{c}{203} & \(21.01\%\) \\ 
%							& 2 & 105 & \multicolumn{1}{c}{0.75} & \multicolumn{1}{c}{117} & \(11.43\%\) \\ 
%							& 3 & 537 & \multicolumn{1}{c}{0.65} & \multicolumn{1}{c}{385} & \(28.31\%\) \\ 
%							\midrule
%							\multirow{3}{*}{FCN-DenseNet} 
%							& 1 & 257 & \multicolumn{1}{c}{0.52} & \multicolumn{1}{c}{505} & \(96.50\%\) \\ 
%							& 2 & 105 & \multicolumn{1}{c}{0.66} & \multicolumn{1}{c}{118} & \(12.38\%\) \\ 
%							& 3 & 537 & \multicolumn{1}{c}{0.72} & \multicolumn{1}{c}{815} & \(51.77\%\) \\ 
%							\midrule
%							\multirow{3}{*}{PSPNet} 
%							& 1 & 257 & \multicolumn{1}{c}{0.00} & \multicolumn{1}{c}{0} & \(-\%\) \\ 
%							& 2 & 105 & \multicolumn{1}{c}{0.44} & \multicolumn{1}{c}{156} & \(48.57\%\) \\ 
%							& 3 & 537 & \multicolumn{1}{c}{0.77} & \multicolumn{1}{c}{610} & \(13.59\%\) \\ 
%							\midrule
%							\multirow{3}{*}{GCN} 
%							& 1 & 257 & \multicolumn{1}{c}{0.71} & \multicolumn{1}{c}{215} & \(16.34\%\) \\ 
%							& 2 & 105 & \multicolumn{1}{c}{0.72} & \multicolumn{1}{c}{177} & \(68.57\%\) \\ 
%							& 3 & 537 & \multicolumn{1}{c}{0.86} & \multicolumn{1}{c}{523} & \(2.61\%\) \\ 
%							\bottomrule[1.5pt]
%						\end{tabular}	
%					\end{table}
					%%%%%%%%%%%%%%%%%%%%%%%%%%%%%%%%%%%%%%%%%%%%%%%%%%%%%%%%%%%	
%				\end{column}
%			\hfill
		\begin{column}[c]{0.9\textwidth}
			\begin{table}[ht!]
				\centering
				\caption{Analysis of numerical cases.}
				\label{tab:table_all_numerical_cases}	
				\begin{tabular}{lcc}
					\toprule[1.5pt]
					Model & mean IoU & max IoU \\ 
					\midrule 
					Res-UNet & \(0.66\) & \(0.89\) \\ 
					VGG16 encoder-decoder & \(0.57\) & \(0.84\) \\ 
					FCN-DenseNet & \(0.68\) & \(0.92\) \\ 
					PSPNet & \(0.55\) & \(0.91\) \\ 
					GCN & \(0.76\) & \(0.93\) \\ 
					\bottomrule[1.5pt]
				\end{tabular}
			\end{table}
		\end{column}
		\end{columns}
	\end{frame}
	%%%%%%%%%%%%%%%%%%%%%%%%%%%%%%%%%%%%%%%%%%%%%%%%%%%%%%%%%%%%%%%%%%%%%%%%%%%%
	\note{note text}
	%%%%%%%%%%%%%%%%%%%%%%%%%%%%%%%%%%%%%%%%%%%%%%%%%%%%%%%%%%%%%%%%%%%%%%%%%%%%
	\begin{frame}{Numerical test cases animation of Lamb waves}
		\setcounter{subfigure}{0}
		\only<1>{
			\textbf{First test case}
			\begin{figure}
				\centering
				\subfloat[Full wavefield (512 frames)]{\animategraphics[autoplay,loop,height=4cm,keepaspectratio]{32}{figures/gif_figs/381_output/output_381-}{1}{512}}\quad
				\subfloat[RMS of all intermediate predictions]{\includegraphics[height=4.1cm,keepaspectratio]{figures/RMS_Ijjeh_num_case_381.png}}\quad
				\subfloat[Binary RMS, IoU= 0.88]{\includegraphics[height=4cm,keepaspectratio]{figures/Binary_RMS_Ijjeh_num_case381_.png}}\quad
			\end{figure}}
		\setcounter{subfigure}{0}
		\only<2>{
		\textbf{Second test case}
		\begin{figure}
			\centering
			\subfloat[Full wavefield (512 frames)]{\animategraphics[autoplay,loop,height=4cm,keepaspectratio]{32}{figures/gif_figs/385_output/output_385-}{1}{512}}\quad
			\subfloat[RMS of all intermediate predictions]{\includegraphics[height=4.1cm,keepaspectratio]{figures/RMS_Ijjeh_num_case_385.png}}\quad
			\subfloat[Binary RMS, IoU= 0.58]{\includegraphics[height=4cm,keepaspectratio]{figures/Binary_RMS_Ijjeh_num_case385_.png}}
		\end{figure}}
		\setcounter{subfigure}{0}
		\only<3>{
		\textbf{Third test case}
		\begin{figure}
			\centering
			\subfloat[Full wavefield (512 frames)]{\animategraphics[autoplay,loop,height=4cm,keepaspectratio]{32}{figures/gif_figs/394_output/output_394-}{1}{512}}\quad
			\subfloat[RMS of all intermediate predictions]{\includegraphics[height=4.1cm,keepaspectratio]{figures/RMS_Ijjeh_num_case_394.png}}\quad
			\subfloat[Binary RMS, IoU= 0.8]{\includegraphics[height=4cm,keepaspectratio]{figures/Binary_RMS_Ijjeh_num_case394_.png}}
		\end{figure}}
	\end{frame}
	%%%%%%%%%%%%%%%%%%%%%%%%%%%%%%%%%%%%%%%%%%%%%%%%%%%%%%%%%%%%%%%%%%%%%%%%%%%%
	\note{note text}
	%%%%%%%%%%%%%%%%%%%%%%%%%%%%%%%%%%%%%%%%%%%%%%%%%%%%%%%%%%%%%%%%%%%%%%%%%%%%
	\begin{frame}{Animation based: Analysis of numerical cases}
		%%%%%%%%%%%%%%%%%%%%%%%%%%%%%%%%%%%%%%%%%%%%%%%%%%%%%%%%%%%%%%%%%%%%
		\begin{table}[!h]
			\centering
			\caption{Evaluation metrics of the three numerical cases.}
			\begin{tabular}{ccccc}
				\toprule[1.5pt]
				\multirow{2}{*}{case number} & \multicolumn{1}{c}{\multirow{2}{*}{A [mm\textsuperscript{2}]}} & \multicolumn{3}{c}{Predicted output} \\ 
				\cmidrule(lr){3-5} & & \multicolumn{1}{c}{IoU} & \multicolumn{1}{c}{\(\hat{A}\) [mm\textsuperscript{2}]} & \(\epsilon\) \\
				\midrule
				1 & 763 & \multicolumn{1}{c}{0.88} & \multicolumn{1}{c}{735} & \(3.67\%\) \\ 
				2 & 388 & \multicolumn{1}{c}{0.58} & \multicolumn{1}{c}{248} & \(36.08\%\) \\ 
				3 & 297 & \multicolumn{1}{c}{0.80} & \multicolumn{1}{c}{280} & \(5.72\%\) \\ 					
				\bottomrule[1.5pt]
			\end{tabular}	
			\label{tab:num_cases}
		\end{table}			
	\end{frame}
	%%%%%%%%%%%%%%%%%%%%%%%%%%%%%%%%%%%%%%%%%%%%%%%%%%%%%%%%%%%%%%%%%%%%%%%%%%%%
	\note{note text}
	%%%%%%%%%%%%%%%%%%%%%%%%%%%%%%%%%%%%%%%%%%%%%%%%%%%%%%%%%%%%%%%%%%%%%%%%%%%%
	\subsection{Experimental cases}
	\begin{frame}[t]{Composite specimen}
		\begin{columns}[T]
			\column{0.7\textwidth}
			{\small
				\begin{itemize}
					\item 16 layers set at the same angle \\
					\item carbon: Prepreg GG 205  P (fibres Toray FT 300 - 3K 200 tex), $E=230$ GPa
					\item epoxy resin: IMP503Z-HT by Impregnatex Compositi 
					\item dimensions: 500$\times$500$\times$3.9 mm\\
					\item density: 1522.4~kg/m\textsuperscript{3}
				\end{itemize}
			}
			\column{0.3\textwidth}
			\begin{figure}
				\includegraphics[width=0.6\textwidth]{weave-1.jpg}
				\caption{Plain weave fabric}
			\end{figure}
		\end{columns}
		\begin{table}[h]
			\renewcommand{\arraystretch}{1.1}
			\centering \footnotesize
			\caption{Geometry of a plain weave fabric reinforced composite [mm]}
			\begin{tabular}{cccccc} 
				%\hline
				\toprule[1.5pt]
				\multicolumn{4}{c}{\textbf{width} }	& \multicolumn{2}{c}{\textbf{thickness} }  \\ 
				%	\hline \hline
				\cmidrule(lr){1-4} \cmidrule(lr){5-6} 
				fill & warp & fill gap& warp gap& fill & warp\\
				%\hline
				$a_f$ &$a_w$& $g_f$  & $g_w$  & $h_f$& $h_w$ \\ 
				%\hline
				%\midrule
				\cmidrule(lr){1-2} \cmidrule(lr){3-4} \cmidrule(lr){5-6}
				1.92 &2.0& 0.05& 0.05 & 0.121875 & 0.121875 \\
				%\hline 
				\bottomrule[1.5pt] 
			\end{tabular} 
		\label{tab:weave_geo}
		\end{table}
	\end{frame}
	%%%%%%%%%%%%%%%%%%%%%%%%%%%%%%%%%%%%%%%%%%%%%%%%%%%%%%%%%%%%%%%%%%%%%%%%%%%%
	\note{note text}
	%%%%%%%%%%%%%%%%%%%%%%%%%%%%%%%%%%%%%%%%%%%%%%%%%%%%%%%%%%%%%%%%%%%%%%%%%%%%
	\begin{frame}[t]{Specimens with defects}
		\vspace{-0.5cm}
		\begin{columns}[T]
			\column{0.5\textwidth}
			\begin{figure}
				\includegraphics[scale=0.36]{plate_multi_delam_arrangement_large_fonts.png}
			\end{figure}
			\column{0.5\textwidth}
			\begin{figure}
				\includegraphics[scale=0.36]{plate_single_delam_arrangement_large_fonts.png}
			\end{figure}
		\end{columns}
	\end{frame}
	%%%%%%%%%%%%%%%%%%%%%%%%%%%%%%%%%%%%%%%%%%%%%%%%%%%%%%%%%%%%%%%%%%%%%%%%%%%%
	\note{note text}
	%%%%%%%%%%%%%%%%%%%%%%%%%%%%%%%%%%%%%%%%%%%%%%%%%%%%%%%%%%%%%%%%%%%%%%%%%%%%
	\begin{frame}[t]{SLDV measurements: setup}
		\begin{figure}
			\includegraphics[width=0.7\textwidth]{sensors_fig4_setup.png}
		\end{figure}
	\end{frame}
	%%%%%%%%%%%%%%%%%%%%%%%%%%%%%%%%%%%%%%%%%%%%%%%%%%%%%%%%%%%%%%%%%%%%%%%%%%%%
	\note{note text}
	%%%%%%%%%%%%%%%%%%%%%%%%%%%%%%%%%%%%%%%%%%%%%%%%%%%%%%%%%%%%%%%%%%%%%%%%%%%%
	\begin{frame}[t]{SLDV measurements: laboratory}
		\begin{columns}[T]
			\column{0.5\textwidth}
			\begin{figure}
				\includegraphics[width=0.8\textwidth]{wibrometr-laserowy-1d_small-description.png}
			\end{figure}
			\column{0.5\textwidth}
			\begin{enumerate}
				\item Signal generator: TTI 1241 
				\item Amplifier: Piezo Systems EPA-104-230 $\pm$200 Vp
				\item Specimen
				\item Scanning head: Polytec PSV-400
				\item DAQ system: Polytec
			\end{enumerate}
		\end{columns}
		{\small
		Measurements were taken on a uniform grid of \textbf{333$\times$333 points}.\\
			Excitation in the form of Hann windowed sine signal of carrier frequency \textbf{50 kHz} was applied to piezoelectric transducer.}
	\end{frame}
	%%%%%%%%%%%%%%%%%%%%%%%%%%%%%%%%%%%%%%%%%%%%%%%%%%%%%%%%%%%%%%%%%%%%%%%%%%%%
	\note{note text}
	%%%%%%%%%%%%%%%%%%%%%%%%%%%%%%%%%%%%%%%%%%%%%%%%%%%%%%%%%%%%%%%%%%%%%%%%%%%%
		
%		\begin{frame}{Experimental setup}
%			\begin{columns}[T]
%				\begin{column}[t]{0.55\textwidth}
%					\begin{figure}
%						\centering
%						\includegraphics[width=.9\textwidth]{wibrometr-laserowy-1d_small-description.png}
%					\end{figure}
%				\end{column}
%				\begin{column}[t]{0.4\textwidth}
%					\begin{enumerate}
%						\item Waveform generator
%						\item Power amplifier	
%						\item Specimen
%						\item SLDV head
%						\item DAQ
%					\end{enumerate}
%				\end{column}
%			\end{columns}
%		\end{frame}
	
%		\begin{frame}{Single delamination arrangement}
%			\begin{minipage}[c]{0.4\textwidth}
%				\begin{itemize}[<alert@+>]
%					\item 
%					\item 
%					\item 
%				\end{itemize}
%			\end{minipage}
%			\begin{minipage}[c]{0.55\textwidth}
%				\centering
%				\includegraphics[width=.7\textwidth]{plate_single_delam_arrangement_large_fonts.jpg}
%			\end{minipage}
%		\end{frame}
		%%%%%%%%%%%%%%%%%%%%%%%%%%%%%%%%%%%%%%%%%%%%%%%%%%%%%%%%%%%%%%%%
	\setcounter{subfigure}{0}
	\begin{frame}{Experimental results RMS based (Single delamination)}
		\begin{columns}[T]
			\begin{column}[c]{0.28\textwidth}
				\small
				Adaptive wavenumber filtering (AWF) \alert{(Conventional signal processing)}.
				\begin{small}
					\biblioref{Kudela P, Radzienski M, Ostachowicz W.}{2018}{Impact induced damage assessment by means of Lamb wave image processing}{Mechanical Systems and Signal Processing,102:23-36}						
%					\biblioref{Kudela, P., Radzieński, M. and Ostachowicz, W.}{2015}{ Identification of cracks in thin-walled structures by means of wavenumber filtering}{Mechanical Systems and Signal Processing, 50, pp.456-466}					.
				\end{small}
			\end{column}
			\hfill
			\begin{column}[c]{0.7\textwidth}
				\centering
				\begin{figure}{ht!}
					\subfloat[ERMS \& label]{\includegraphics[width=.27\textwidth]{ERMS_with_label.png}}\quad
					\subfloat[AWF]{\includegraphics[width=.27\textwidth]{ERMSF_CFRP_teflon_3o_375_375p_50kHz_5HC_x12_15Vpp.png}}\quad
					\subfloat[Binary AWF: IoU=$0.401$]{\includegraphics[width=.27\textwidth]{Binary_ERMSF_CFRP_teflon_3o_375_375p_50kHz_5HC_x12_15Vpp.png}}\quad
					\subfloat[GCN: IoU\(=0.723\)]{\includegraphics[width=.27\textwidth]{Fig_GCN_7.png}}\quad
					\subfloat[FCN-DenseNet: IoU=$0.54$]{\includegraphics[width=.27\textwidth]{Fig_FCN_DenseNet_7.png}}\qquad
				\end{figure} 
			\end{column}
		\end{columns}	
	\end{frame}
	%%%%%%%%%%%%%%%%%%%%%%%%%%%%%%%%%%%%%%%%%%%%%%%%%%%%%%%%%%%%%%%%%%%%%%%%%%%%
	\note{note text}
	%%%%%%%%%%%%%%%%%%%%%%%%%%%%%%%%%%%%%%%%%%%%%%%%%%%%%%%%%%%%%%%%%%%%%%%%%%%%
	\setcounter{subfigure}{0}
	\begin{frame}{RMS based: Analysis of experimental case}
		\begin{table}[!ht]
			\centering
			\caption{Evaluation metrics of the experimental case.}
			\label{tab:rms_exp_case}
			\begin{tabular}{lc}
				\toprule[1.5pt]
				Model & IoU  	\\			
				\midrule
				Res-UNet & 0.58 \\ 
				VGG16 encoder-decoder & 0.62 \\ 
				FCN-DenseNet & 0.54 \\ 
				PSPNet & 0.49 \\ 
				GCN & 0.72\\ 
				\bottomrule[1.5pt]
			\end{tabular}		
		\end{table}
%			\begin{table}[!ht]
%				\centering
%				\caption{Evaluation metrics of the experimental case.}
%				\label{tab:rms_exp_case}
%				\begin{tabular}{l@{\ }cccc}
%					\toprule
%					\multicolumn{1}{l}{Model} & \multicolumn{1}{c}{A [mm\textsuperscript{2}]} & \multicolumn{3}{c}{Predicted output} \\ 
%					\cmidrule(lr){3-5} & & \multicolumn{1}{c}{IoU} & \multicolumn{1}{c}{\(\hat{A}\) [mm\textsuperscript{2}]} & \(\epsilon\) \\ \midrule
%					Res-UNet & \multicolumn{1}{c}{\multirow{5}{*}{210}} & \multicolumn{1}{c}{0.58} & \multicolumn{1}{c}{323}  & \(53.8\%\) \\ 
%					VGG16 encoder-decoder &  & \multicolumn{1}{c}{0.62} & \multicolumn{1}{c}{320} & \(52.4\%\) 
%					\\ 
%					FCN-DenseNet &  & \multicolumn{1}{c}{0.54} & \multicolumn{1}{c}{386} & \(83.8\%\) \\ 
%					PSPNet &  & \multicolumn{1}{c}{0.49} & \multicolumn{1}{c}{580} & \(176.2\%\) 
%					\\ 
%					GCN &  & \multicolumn{1}{c}{0.72} & \multicolumn{1}{c}{309} & \(47.1\%\) 
%					\\ 
%					\bottomrule
%				\end{tabular}		
%			\end{table}
	\end{frame}
	%%%%%%%%%%%%%%%%%%%%%%%%%%%%%%%%%%%%%%%%%%%%%%%%%%%%%%%%%%%%%%%%%%%%%%%%%%%%
	\note{note text}
	%%%%%%%%%%%%%%%%%%%%%%%%%%%%%%%%%%%%%%%%%%%%%%%%%%%%%%%%%%%%%%%%%%%%%%%%%%%%
	\setcounter{subfigure}{0}
	\begin{frame}{Experimental results full wavefield based (Single delamination)}
		\begin{figure}[ht!]
			\centering
			\subfloat[Full wavefield (512 frames)]{\animategraphics[autoplay,loop,height=3cm]{32}{figures/gif_figs/CFRP_teflon_3o_375_375p_50kHz_5HC_x12_15Vpp/CFRP_teflon_30-}{1}{256}}\quad
			\subfloat[Intermidate ouputs]{\animategraphics[autoplay,loop,height=3cm]{24}{figures/gif_figs/CFRP_ijjeh_single_delamination/intermediate_output-}{0}{231}}\quad
			\subfloat[RMS]{\includegraphics[height=3cm,keepaspectratio]{figures/RMS_CFRP_teflon_3o_375_375p_50kHz_5HC_x12_15Vpp_Ijjeh_updated_results_.png}}\quad
			\subfloat[Binary RMS]{\includegraphics[height=3cm,keepaspectratio]{figures/Binary_RMS_CFRP_teflon_3o__375_375p_50kHz_5HC_x12_15Vpp_Ijjeh_.png}}
		\end{figure}
		IoU= $0.41$ for the thresholded damage map.% and $\epsilon=71.56\%$  
	\end{frame}
	%%%%%%%%%%%%%%%%%%%%%%%%%%%%%%%%%%%%%%%%%%%%%%%%%%%%%%%%%%%%%%%%%%%%%%%%%%%%
	\note{note text}
	%%%%%%%%%%%%%%%%%%%%%%%%%%%%%%%%%%%%%%%%%%%%%%%%%%%%%%%%%%%%%%%%%%%%%%%%%%%%
%		\setcounter{subfigure}{0}
%		\begin{frame}{Multiple delamination arrangement}
%			\begin{minipage}[c]{0.4\textwidth}
%				\begin{itemize}[<alert@+>]
%					\item Teflon inserts with a thickness of \(250\ \mu\)m were used to simulate the delaminations.
%					\item The average thickness of the specimen was \(3.9 \pm 0.1\) mm.
%					\item The delaminations were located at the same distance, equal to \(150\) mm from the centre of the plate.
%				\end{itemize}
%			\end{minipage}
%			\begin{minipage}[c]{0.55\textwidth}
%				\centering
%				\includegraphics[width=.7\textwidth]{plate_multi_delam_arrangement_large_fonts.png}
%			\end{minipage}
%		\end{frame}
		%%%%%%%%%%%%%%%%%%%%%%%%%%%%%%%%%%%%%%%%%%%%%%%%%%%%%%%%%%%%%%%%
	\setcounter{subfigure}{0}	
	\begin{frame}{Experimental results full wavefield based (Multiple delaminations)}
		\begin{columns}[T]
			\begin{column}[t]{0.24\textwidth}
				\begin{figure}[ht!]
					\centering
					\subfloat[Full wavefield (512 frames)]{\animategraphics[autoplay,loop,height=0.25\textheight]{32}{figures/gif_figs/input_specimen_3/specimen_3-}{1}{512}}
				\end{figure}
			\end{column}	
%			\quad				
			\begin{column}[t]{0.24\textwidth}
				\begin{figure}[ht!]
					\centering
					\subfloat[AWF]{\includegraphics[height=0.25\textheight]{figures/mul/figure17a.png}}
					\\
					\subfloat[Binary AWF, IoU$=0.04$]{\includegraphics[height=0.25\textheight]{figures/mul/figure17b.png}}								
				\end{figure}
			\end{column}
%				\quad
%				\begin{column}[t]{0.16\textwidth}
%					\begin{figure}
%						\centering
%						
%					\end{figure}
%				\end{column}				
%				\quad
			\begin{column}[t]{0.24\textwidth}
				\begin{figure}[ht!]
					\centering
					\subfloat[Intermidate ouputs]{\animategraphics[autoplay,loop,height=0.25\textheight]{24}{figures/gif_figs/Intermediate_specimen_3/Intermediate_specimen_3-}{0}{487}}
				\end{figure}
			\end{column}
%			\quad
			\begin{column}[t]{0.24\textwidth}
				\begin{figure}[ht!]
					\subfloat[RMS]{\includegraphics[height=0.25\textheight,keepaspectratio]{figures/RMS_L3_S3_B_333x333p_50kHz_5HC_18Vpp_x10_pzt_Ijjeh_updated_results_.png}}
					\\
					\subfloat[Binary RMS]{\includegraphics[height=0.25\textheight,keepaspectratio]{figures/Binary_RMS_L3_S3_B__333x333p_50kHz_5HC_18Vpp_x10_pzt_Ijjeh_.png}}
				\end{figure}
			\end{column}
%			\quad
%			\begin{column}[t]{0.16\textwidth}
%				\begin{figure}
%					\centering
%					
%				\end{figure}
%			\end{column}				
		\end{columns}	
		IoU= $0.64$ and $\epsilon=1.69\%$  for the thresholded damage map (Binary RMS).		
	\end{frame}
	%%%%%%%%%%%%%%%%%%%%%%%%%%%%%%%%%%%%%%%%%%%%%%%%%%%%%%%%%%%%%%%%%%%%%%%%%%%%
	\note{note text}
	%%%%%%%%%%%%%%%%%%%%%%%%%%%%%%%%%%%%%%%%%%%%%%%%%%%%%%%%%%%%%%%%%%%%%%%%%%%%
	\setcounter{subfigure}{0}
	\section{Super-resolution image reconstruction}
	\begin{frame}{Super-resolution image reconstruction}
		\begin{columns}[T]
			\begin{column}[c]{0.35\textwidth}
				\justifying	
				\small	
				DLSR model aims to speed up the data acquisition process for SLDV by recovering HR full wavefield frames with satisfying accuracy from the LR measurement (below Nyquist sampling rate) SLDV.
				\bigskip 
				\biblioref{Abdalraheem Ijjeh, Saeed Ullah, Maciej Radzienski, Pawel Kudela}{2023}{Deep learning super-resolution for the reconstruction of full wavefield of Lamb waves}{Mechanical Systems and Signal Processing 186: 109878.}
			\end{column}
			\begin{column}[c]{0.6\textwidth}
				\begin{figure}
					\subfloat{\includegraphics[width=0.95\textwidth]{superresolution_flowchart.jpeg}}
				\end{figure}
			\end{column}
		\end{columns}
	\end{frame}
	%%%%%%%%%%%%%%%%%%%%%%%%%%%%%%%%%%%%%%%%%%%%%%%%%%%%%%%%%%%%%%%%%%%%%%%%%%%%
	\note{note text}
	%%%%%%%%%%%%%%%%%%%%%%%%%%%%%%%%%%%%%%%%%%%%%%%%%%%%%%%%%%%%%%%%%%%%%%%%%%%%
	\begin{frame}{Evaluation metrics for SR model}
		For evaluating the reconstructed full wavefield
		\begin{itemize}
			\item Peak signal-to-noise ratio (PSNR):
			\begin{equation*}
				PSNR=20\log_{10}\left(\frac{R}{\sqrt{MSE}}\right)
				\label{PSNR}
			\end{equation*}
			\item Pearson correlation coefficient:
			\begin{equation*}
				r_{xy} = \frac{\sum_{i=1}^{n}(x_i - \bar{x})(y_i-\bar{y})}{\sqrt{\sum_{i=1}^{n}(x_i - \bar{x})^2}\sqrt{\sum_{i=1}^{n}(y_i - \bar{y})^2}}
				\label{Pearson}
			\end{equation*}
		\end{itemize}
	\end{frame}
	%%%%%%%%%%%%%%%%%%%%%%%%%%%%%%%%%%%%%%%%%%%%%%%%%%%%%%%%%%%%%%%%%%%%%%%%%%%%
	\note{note text}
	%%%%%%%%%%%%%%%%%%%%%%%%%%%%%%%%%%%%%%%%%%%%%%%%%%%%%%%%%%%%%%%%%%%%%%%%%%%%
	\setcounter{subfigure}{0}
	\begin{frame}{Numerical test cases at certain frames}
		\begin{columns}[T]
			\begin{column}[c]{0.25\textwidth}
				\centering
				\only<1>{\textbf{First test case}}
				\only<2>{\textbf{Second test case}}
				\only<3>{\textbf{Third test case}}
				\begin{figure}
					\centering
					\only<1>{\subfloat[LR input, $f_n=127$]{\includegraphics[height=.45\textheight]{LR_397_frame_127_input.png}}}
					
					\only<2>{\subfloat[LR input, $f_n=154$]{\includegraphics[height=.45\textheight]{LR_438_frame_154_input.png}}}
					
					\only<3>{\subfloat[LR input, $f_n=159$]{\includegraphics[height=.45\textheight]{LR_456_frame_159_input.png}}}
				\end{figure}
			\end{column}
			\begin{column}[c]{0.7\textwidth}
				\only<1>{
				\begin{figure}
					\centering
					\subfloat[listentry][HR ref]{\includegraphics[height=.35\textheight]{output_397_frame_127_full_frame_GT.png}}\quad
					\subfloat[listentry][SR $f_n$]{\includegraphics[height=.35\textheight]{output_397_frame_127_full_frame_pred.png}}
					\\
					\subfloat[listentry][Ref]{\includegraphics[height=.35\textheight]{output_397_frame_127_delamination_GT.png}}\quad
					\subfloat[listentry][Pred]{\includegraphics[height=.35\textheight]{output_397_frame_127_delamination_pred.png}}
				\end{figure}}
				\only<2>{
				\begin{figure}
					\subfloat[listentry][HR ref]{\includegraphics[height=.35\textheight]{output_438_frame_154_full_frame_GT.png}}\quad
					\subfloat[listentry][SR $f_n$]{\includegraphics[height=.35\textheight]{output_438_frame_154_full_frame_pred.png}}
					\\
					\subfloat[listentry][Ref]{\includegraphics[height=.35\textheight]{output_438_frame_154_delamination_GT.png}}\quad	
					\subfloat[listentry][Pred]{\includegraphics[height=.35\textheight]{output_438_frame_154_delamination_pred.png}}
				\end{figure}}
				\only<3>{
				\begin{figure}
					\subfloat[listentry][HR ref]{\includegraphics[height=.35\textheight]{output_456_frame_159_full_frame_GT.png}}\quad
					\subfloat[listentry][SR $f_n$]{\includegraphics[height=.35\textheight]{output_456_frame_159_full_frame_pred.png}}
					\\
					\subfloat[listentry][Ref]{\includegraphics[height=.35\textheight]{output_456_frame_159_delamination_GT.png}}\quad
					\subfloat[listentry][Pred]{\includegraphics[height=.35\textheight]{output_456_frame_159_delamination_pred.png}}
				\end{figure}}
			\end{column}
		\end{columns}
	\end{frame}
	%%%%%%%%%%%%%%%%%%%%%%%%%%%%%%%%%%%%%%%%%%%%%%%%%%%%%%%%%%%%%%%%%%%%%%%%%%%%
	\note{note text}
	%%%%%%%%%%%%%%%%%%%%%%%%%%%%%%%%%%%%%%%%%%%%%%%%%%%%%%%%%%%%%%%%%%%%%%%%%%%%
	\setcounter{subfigure}{0}
	\begin{frame}{Analysis of numerical cases}
		\begin{table}[!h]
			\centering \footnotesize
			\caption{Quality metrics for the DLSR model for three numerical test cases calculated at frames $N_f$ per case.}	
			\begin{tabular}{lccccc}
				\toprule[1.5pt]
				& & \multicolumn{2}{c}{plate} & \multicolumn{2}{c}{delamination} \\
				\cmidrule(lr){3-4} \cmidrule(lr){5-6}
				Case & $N_f$ & PSNR & PEARSON CC & PSNR & PEARSON CC \\ 
				\midrule
				1  & 127  & 42.95 & 0.999 & 33.02 & 0.993 \\					
				\midrule
				2  & 154 & 47.00 & 0.998 & 38.52 & 0.995 
				\\
				\midrule					
				3  & 159 & 48.60 & 0.998 & 46.67 & 0.998 \\					
				\bottomrule[1.5pt]
			\end{tabular}
			\label{tab:num_DLSR_results}
		\end{table}			
	\end{frame}
	%%%%%%%%%%%%%%%%%%%%%%%%%%%%%%%%%%%%%%%%%%%%%%%%%%%%%%%%%%%%%%%%%%%%%%%%%%%%
	\note{note text}
	%%%%%%%%%%%%%%%%%%%%%%%%%%%%%%%%%%%%%%%%%%%%%%%%%%%%%%%%%%%%%%%%%%%%%%%%%%%%
%		\begin{frame}{Experimental setup}
%			\begin{figure}
%				\centering
%				\includegraphics[height=0.8\textheight]{sensors_fig4_setup.png}
%			\end{figure}
%		\end{frame}
		%%%%%%%%%%%%%%%%%%%%%%%%%%%%%%%%%%%%%%%%%%%%%%%%%%%%%%%%%%%%%%%%
	\setcounter{subfigure}{0}
	\begin{frame}{Experimental test case at certain frame}
		Low-resolution measurements (Input): \(32\times32=1024\) points. (Nf = 110.)\\
		High-resolution (Output): \(512\times512=262144\) points (Nf = 110.).
		\begin{columns}[T]
			\begin{column}[c]{0.19\textwidth}
				\begin{figure}						
					\subfloat[LR input]{						\includegraphics[width=1\textwidth]{frame110_32x32.png}}
				\end{figure}
			\end{column}
			\begin{column}[c]{0.8\textwidth}
				\begin{figure}[ht!]
					\subfloat[HR ref]{\includegraphics[height=.22\textheight]{figure10a.png}}\quad
					\subfloat[CS: 1024 points]{\includegraphics[height=.22\textheight]{figure10b.png}}\quad
					\subfloat[CS: 3000 points]{\includegraphics[height=.22\textheight]{figure10c.png}}\quad
					\subfloat[CS: 4000 points]{\includegraphics[height=.22\textheight]{figure10d.png}}\quad
					\subfloat[DLSR]{\includegraphics[height=.22\textheight]{figure10e.png}}\quad
					
					\subfloat[Ref]{\includegraphics[height=.22\textheight]{figure11a.png}}\quad
					\subfloat[CS: 1024 points]{\includegraphics[height=.22\textheight]{figure11b.png}}\quad
					\subfloat[CS: 3000 points]{\includegraphics[height=.22\textheight]{figure11c.png}}\quad
					\subfloat[CS: 4000 points]{\includegraphics[height=.22\textheight]{figure11d.png}}\quad
					\subfloat[DLSR]{\includegraphics[height=.22\textheight]{figure11e.png}}\quad
				\end{figure}
			\end{column}				
		\end{columns}
	\end{frame}
	%%%%%%%%%%%%%%%%%%%%%%%%%%%%%%%%%%%%%%%%%%%%%%%%%%%%%%%%%%%%%%%%%%%%%%%%%%%%
	\note{note text}
	%%%%%%%%%%%%%%%%%%%%%%%%%%%%%%%%%%%%%%%%%%%%%%%%%%%%%%%%%%%%%%%%%%%%%%%%%%%%
	\begin{frame}{Analysis of experimental case}
		\begin{table}[!ht]
			\renewcommand{\arraystretch}{1.3}
			\centering \footnotesize
			\caption{Quality metrics for tested methods for various number of points $N_p$ and corresponding compression ratios CR calculated for the frame no $N_f=110$.}	
			\begin{tabular}{lrrrcrc} 
				\toprule[1.5pt]
				& & & \multicolumn{2}{c}{plate} & \multicolumn{2}{c}{delamination} \\
				\cmidrule(lr){4-5} \cmidrule(lr){6-7}
				Method & $N_p$ & CR [\%] & PSNR & PEARSON CC& PSNR & PEARSON CC \\
				\midrule
				\csvreader
				[table head=\toprule,
				late after line=\\ 
				]{table_metrics.csv}{
					1=\one, 2=\two, 3=\three, 4=\four, 5=\five, 6=\six, 7=\seven
				}%
				{\one & \two & \three & \four & \five & \six & \seven }%	
				\bottomrule[1.5pt]
			\end{tabular}	
			\label{tab:csv_results}
		\end{table}
	\end{frame}
	%%%%%%%%%%%%%%%%%%%%%%%%%%%%%%%%%%%%%%%%%%%%%%%%%%%%%%%%%%%%%%%%%%%%%%%%%%%%
	\note{note text}
	%%%%%%%%%%%%%%%%%%%%%%%%%%%%%%%%%%%%%%%%%%%%%%%%%%%%%%%%%%%%%%%%%%%%%%%%%%%%
	%%%%%%%%%%%%%%%%%%%%%%%%%%%%%%%%%%%%%%%%%%%%%%%%%%%%%%%%%%%%%%%%
	\section{Conclusions}
	%%%%%%%%%%%%%%%%%%%%%%%%%%%%%%%%%%%%%%%%%%%%%%%%%%%%%%%%%%%%%%%%
	\begin{frame}{Conclusions}
		\footnotesize
		\begin{itemize}
			\item Full wavefields contain rich damage-related information
			\item Full wavefields can be utilised to train deep learning models to perform damage identification in an end-to-end approach
			\item Deep learning models trained on synthetic dataset generalise well and can be applied directly to experimental wavefields
			\item Animation-based deep learning models perform better than image-based models but are more complex and require longer time for training
			\item Deep learning approaches surpass the conventional signal processing techniques
			\item The DLSR model can speed up the process of data acquisition by SLDV (from LR measurements to HR measurements).
		\end{itemize}
	\end{frame}		
	%%%%%%%%%%%%%%%%%%%%%%%%%%%%%%%%%%%%%%%%%%%%%%%%%%
	{\setbeamercolor{palette primary}{fg=blue, bg=white}
		\begin{frame}[standout]
			Thank you for your listening!\\ \vspace{12pt}
			Questions?\\ \vspace{12pt}
%			\url{pk@imp.gda.pl} 
%			\par\medskip
			\url{aijjeh@imp.gda.pl}
			
			
			\par\medskip
			\par\medskip
			\footnotesize
			The research work was funded by the Polish National Science Center under grant agreement no. 2018/31/B/ST8/00454.
		\end{frame}}
	\begin{frame}[standout]{Composite laminates advantages \& applications}
		\begin{columns}[T]
			\begin{column}[t]{.45\textwidth}
				\textbf{Some of the favorable characteristics of utilising composite laminates:}
				\begin{itemize}
					\item \alert{High impact damage resistance}
					\item \alert{Light weight}
					\item \alert{High strength-to-weight ratio}
					\item \alert{Low density}
					\item \alert{resistance to fatigue and corrosion}
				\end{itemize}
			\end{column}
			\begin{column}[t]{.45\textwidth}
				\textbf{Some applications of Composite laminates:}
				\begin{itemize}
					\item \alert{Aerospace structures}
					\item \alert{Automotive}
					\item \alert{Wind turbines}
					\item \alert{Pipes and tanks}
				\end{itemize}
			\end{column}
		\end{columns}
	\end{frame}
		%%%%%%%%%%%%%%%%%%%%%%%%%%%%%%%%%%%%%%%%%%%%%%%%%%
		% END OF SLIDES
		%%%%%%%%%%%%%%%%%%%%%%%%%%%%%%%%%%%%%%%%%%%%%%%%%%
	\end{document}