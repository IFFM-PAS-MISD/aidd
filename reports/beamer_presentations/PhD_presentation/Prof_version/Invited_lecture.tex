%\PassOptionsToPackage{draft}{graphicx}
%\documentclass[10pt]{beamer} % aspect ratio 4:3, 128 mm by 96 mm
\documentclass[10pt,aspectratio=169,dvipsnames]{beamer} % aspect ratio 16:9
%\graphicspath{{../../figures/}}

%\includeonlyframes{frame1,frame2,frame3}

%%%%%%%%%%%%%%%%%%%%%%%%%%%%%%%%%%%%%%%%%%%%%%%%%%
% Packages
%%%%%%%%%%%%%%%%%%%%%%%%%%%%%%%%%%%%%%%%%%%%%%%%%%
\usepackage{appendixnumberbeamer}
\usepackage{booktabs}
\usepackage{csvsimple} % for csv read
\usepackage[scale=2]{ccicons}
\usepackage{pgfplots}
\usepackage{xspace}
\usepackage{amsmath, nccmath} % fleqn
\usepackage{totcount}
\usepackage{tikz}
\usepackage{bm}
\usepackage{float}
\usepackage{eso-pic} 
\usepackage{wrapfig}
\usepackage{animate,media9,movie15}
\usepackage{subfig}
\usepackage{fancybox}
%\usepackage{multimedia}
\usepackage{dashbox}
\usepackage{tcolorbox}
\usepackage{multicol}
\usepackage{tikz}

\usepackage{subfig}

\usepackage{xcolor}

%\usepackage{FiraSans}

%\usepackage{comment}
%\usetikzlibrary{external} % speedup compilation
%\tikzexternalize % activate!
%\usetikzlibrary{shapes,arrows} 

%\usepackage{bibentry}
%\nobibliography*
\usepackage{ifthen}
\newcounter{angle}
\setcounter{angle}{0}
%\usepackage{bibentry}
%\nobibliography*
\usepackage{caption}%

\graphicspath{{figures/}}

\captionsetup[figure]{labelformat=empty}%
\usefonttheme{structurebold}
%%%%%%%%%%%%%%%%%%%%%%%%%%%%%%%%%%%%%%%%%%%%%%%%%%
% Metropolis theme custom modification file
%%%%%%%%%%%%%%%%%%%%%%%%%%%%%%%%%%%%%%%%%%%%%%%%%%
% Metropolis theme custom modification file
%%%%%%%%%%%%%%%%%%%%%%%%%%%%%%%%%%%%%%%%%%%%%%%%%%
% Metropolis theme custom colors
%%%%%%%%%%%%%%%%%%%%%%%%%%%%%%%%%%%%%%%%%%%%%%%%%%
\usetheme[progressbar=foot]{metropolis}
\useoutertheme{metropolis}
\useinnertheme{metropolis}
\usefonttheme{metropolis}
\setbeamercolor{background canvas}{bg=white}

%\usecolortheme{spruce}

\definecolor{myblue}{rgb}{0.19,0.55,0.91}
\definecolor{mediumblue}{rgb}{0,0,205}
\definecolor{darkblue}{rgb}{0,0,139}
\definecolor{Dodgerblue}{HTML}{1E90FF}
\definecolor{Navy}{HTML}{000080} % {rgb}{0,0,128}
\definecolor{Aliceblue}{HTML}{F0F8FF}
\definecolor{Lightskyblue}{HTML}{87CEFA}
\definecolor{logoblue}{RGB}{1,67,140}
\definecolor{Purple}{HTML}{911146}
\definecolor{Orange}{HTML}{CF4A30}

\setbeamercolor{progress bar}{bg=Lightskyblue}
\setbeamercolor{progress bar}{ fg=logoblue} 
\setbeamercolor{frametitle}{bg=logoblue}
\setbeamercolor{title separator}{fg=logoblue}
\setbeamercolor{block title}{bg=Lightskyblue!30,fg=black}
\setbeamercolor{block body}{bg=Lightskyblue!15,fg=black}
\setbeamercolor{alerted text}{fg=Purple}
% notes colors
\setbeamercolor{note page}{bg=white}
\setbeamercolor{note title}{bg=Lightskyblue}
%%%%%%%%%%%%%%%%%%%%%%%%%%%%%%%%%%%%%%%%%%%%%%%%%%
%  Theme modifications
%%%%%%%%%%%%%%%%%%%%%%%%%%%%%%%%%%%%%%%%%%%%%%%%%%
% modify progress bar linewidth
\makeatletter
\setlength{\metropolis@progressinheadfoot@linewidth}{2pt} 
\setlength{\metropolis@titleseparator@linewidth}{1pt}
\setlength{\metropolis@progressonsectionpage@linewidth}{1pt}

\setbeamertemplate{progress bar in section page}{
	\setlength{\metropolis@progressonsectionpage}{%
		\textwidth * \ratio{\thesection pt}{\totvalue{totalsection} pt}%
	}%
	\begin{tikzpicture}
		\fill[bg] (0,0) rectangle (\textwidth, 
		\metropolis@progressonsectionpage@linewidth);
		\fill[fg] (0,0) rectangle (\metropolis@progressonsectionpage, 
		\metropolis@progressonsectionpage@linewidth);
	\end{tikzpicture}%
}
\makeatother
\newcounter{totalsection}
\regtotcounter{totalsection}

\AtBeginDocument{%
	\pretocmd{\section}{\refstepcounter{totalsection}}{\typeout{Yes, prepending 
	was successful}}{\typeout{No, prepending was not successful}}%
}%
%%%%%%%%%%%%%%%%%%%%%%%%%%%%%%%%%%%%%%%%%%%%%%%%%%
%  Bibliography mods
%%%%%%%%%%%%%%%%%%%%%%%%%%%%%%%%%%%%%%%%%%%%%%%%%%
\setbeamertemplate{bibliography item}{\insertbiblabel} %% Remove book symbol 
%%from references and add number in square brackets
% kill the abominable icon (without number)
%\setbeamertemplate{bibliography item}{}
%\makeatletter
%\renewcommand\@biblabel[1]{#1.} % number only
%\makeatother
% remove line breaks in bibliography
\setbeamertemplate{bibliography entry title}{}
\setbeamertemplate{bibliography entry location}{}
%%%%%%%%%%%%%%%%%%%%%%%%%%%%%%%%%%%%%%%%%%%%%%%%%%
%  Bibliography custom commands
%%%%%%%%%%%%%%%%%%%%%%%%%%%%%%%%%%%%%%%%%%%%%%%%%%
\newcommand{\bibliotitlestyle}[1]{\textbf{#1}\par}

\newif\ifinbiblio
\newcounter{bibkey}
\newenvironment{biblio}[2][long]{%
	%\setbeamertemplate{bibliography item}{\insertbiblabel}
	\setbeamertemplate{bibliography item}{}% without numbers
	\setbeamerfont{bibliography item}{size=\footnotesize}
	\setbeamerfont{bibliography entry author}{size=\footnotesize}
	\setbeamerfont{bibliography entry title}{size=\footnotesize}
	\setbeamerfont{bibliography entry location}{size=\footnotesize}
	\setbeamerfont{bibliography entry note}{size=\footnotesize}
	\ifx!#2!\else%
	\bibliotitlestyle{#2}%
	\fi%
	\begin{thebibliography}{}%
		\inbibliotrue%
		\setbeamertemplate{bibliography entry title}[#1]%
	}{%
		\inbibliofalse%
		\setbeamertemplate{bibliography item}{}%
	\end{thebibliography}%
}

\newcommand{\biblioref}[5][short]{
	\setbeamertemplate{bibliography entry title}[#1]
	\stepcounter{bibkey}%
	\ifinbiblio%
	\bibitem{\thebibkey}%
	#2
	\newblock #4
	\ifx!#5!\else\newblock {\em #5}, #3 \fi%
	\else%
	\begin{biblio}{}
		\bibitem{\thebibkey}
		#2
		\newblock #4
		\ifx!#5!\else\newblock {\em #5}, #3\fi
	\end{biblio}
	\fi
}
%
%\newbibmacro*{hypercite}{%
%	\renewcommand{\@makefntext}[1]{\noindent\normalfont##1}%
%	\footnotetext{%
%		\blxmkbibnote{foot}{%
%			\printtext[labelnumberwidth]{%
%				\printfield{prefixnumber}%
%				\printfield{labelnumber}}%
%			\addspace
%			\fullcite{\thefield{entrykey}}}}}
%
%\DeclareCiteCommand{\hypercite}%
%{\usebibmacro{cite:init}}
%{\usebibmacro{hypercite}}
%{}
%{\usebibmacro{cite:dump}}
%
%% Redefine the \footfullcite command to use the reference number
%\renewcommand{\footfullcite}[1]{\cite{#1}\hypercite{#1}}
%\usefonttheme[onlymath]{Serif} % It should be uncommented if Fira fonts in 
%%math does not work

%%%%%%%%%%%%%%%%%%%%%%%%%%%%%%%%%%%%%%%%%%%%%%%%%%
% Custom commands
%%%%%%%%%%%%%%%%%%%%%%%%%%%%%%%%%%%%%%%%%%%%%%%%%%
% matrix command 
\newcommand{\matr}[1]{\mathbf{#1}} % bold upright (Elsevier, Springer)
%\newcommand{\matr}[1]{#1}   % pure math version
%\newcommand{\matr}[1]{\bm{#1}}  % ISO complying version
% vector command 
\newcommand{\vect}[1]{\mathbf{#1}} % bold upright (Elsevier, Springer)
% bold symbol
\newcommand{\bs}[1]{\boldsymbol{#1}}
% derivative upright command
\DeclareRobustCommand*{\drv}{\mathop{}\!\mathrm{d}}
\newcommand{\ud}{\mathrm{d}}
% 
\newcommand{\themename}{\textbf{\textsc{metropolis}}\xspace}

\newenvironment{elaboration}{%
		\end{minipage}\egroup;
		\draw[dashed] (m.south west) rectangle (m.north east);
	\end{tikzpicture}
}

%%%%%%%%%%%%%%%%%%%%%%%%%%%%%%%%%%%%%%%%%%%%%%%%%%
% Title page options
%%%%%%%%%%%%%%%%%%%%%%%%%%%%%%%%%%%%%%%%%%%%%%%%%%
% \date{\today}
\date{}
%%%%%%%%%%%%%%%%%%%%%%%%%%%%%%%%%%%%%%%%%%%%%%%%%%
% option 1
%%%%%%%%%%%%%%%%%%%%%%%%%%%%%%%%%%%%%%%%%%%%%%%%%%%
\title{Deep neural networks in structural diagnostic applications}
%\subtitle{In preparation for a Ph.D. defence}
\author{\textbf{D.Sc. Ph.D. Eng. Paweł Kudela}\\Ph.D. candidate Eng. Abdalraheem Ijjeh 
}
% logo align to Institute 
\institute{Institute of Fluid Flow Machinery \\ 
	Polish Academy of Sciences \\ 
	\vspace{-1.5cm}
	\flushright 
	\includegraphics[width=6cm]{imp_logo.png}}
%%%%%%%%%%%%%%%%%%%%%%%%%%%%%%%%%%%%%%%%%%%%%%%%%%
% option 2 - authors in one line
%%%%%%%%%%%%%%%%%%%%%%%%%%%%%%%%%%%%%%%%%%%%%%%%%%
%	\title{My fancy title}
%	\subtitle{Lamb-opt}
%	\author{\textbf{Paweł Kudela}\textsuperscript{2}, Maciej 
%	Radzieński\textsuperscript{2}, Wiesław Ostachowicz\textsuperscript{2}, 
%	Zhibo Yang\textsuperscript{1} }
%	 logo align to Institute 
%	\institute{\textsuperscript{1}Xi'an Jiaotong University \\ 
%	\textsuperscript{2}Institute of Fluid Flow Machinery\\ \hspace*{1pt} Polish 
%	Academy of Sciences \\ \vspace{-1.5cm}\flushright 	
%	\includegraphics[width=6cm]{imp_logo.png}}
%%%%%%%%%%%%%%%%%%%%%%%%%%%%%%%%%%%%%%%%%%%%%%%%%%%
% option 3 - multilogo vertical
%%%%%%%%%%%%%%%%%%%%%%%%%%%%%%%%%%%%%%%%%%%%%%%%%%
%%\title{My fancy title}
%%\subtitle{Lamb-opt}
%%	\author{\textbf{Paweł Kudela}\inst{1}, Maciej Radzieński\inst{1}, Wiesław Ostachowicz\inst{1}, Zhibo Yang\inst{2} }
%%	 logo under Institute 
%%	\institute%
%%	{ 
%%		\inst{1}%
%%		Institute of Fluid Flow Machinery\\ \hspace*{1pt} Polish Academy of Sciences \\ \includegraphics[height=0.85cm]{//odroid-sensors/sensors/MISD_shared/logo/logo_eng_40mm.eps} \\
%%		\and
%%		\inst{2}%
%%	 Xi'an Jiaotong University \\ \includegraphics[height=0.85cm]{//odroid-sensors/sensors/MISD_shared/logo/logo_box.eps}
%% }
% end od option 3
%%%%%%%%%%%%%%%%%%%%%%%%%%%%%%%%%%%%%%%%%%%%%%%%%%
%% option 4 - 3 Institutes and logos horizontal centered
%%%%%%%%%%%%%%%%%%%%%%%%%%%%%%%%%%%%%%%%%%%%%%%%%%
%\title{My fancy title}
%\subtitle{Lamb-opt }
%\author{\textbf{Paweł Kudela}\textsuperscript{1}, Maciej Radzieński\textsuperscript{1}, Marco Miniaci\textsuperscript{2}, Zhibo Yang\textsuperscript{3} }
%
%\institute{ 
%\begin{columns}[T,onlytextwidth]
%	\column{0.39\textwidth}
%	\begin{center}
%		\textsuperscript{1}Institute of Fluid Flow Machinery\\ \hspace*{3pt}Polish Academy of Sciences
%	\end{center}
%	\column{0.3\textwidth}
%	\begin{center}
%		\textsuperscript{2}Zurich University
%	\end{center}
%	\column{0.3\textwidth}
%	\begin{center}
%		\textsuperscript{3}Xi'an Jiaotong University
%	\end{center}
%\end{columns}
%\vspace{6pt}
%% logos 
%\begin{columns}[b,onlytextwidth]
%	\column{0.39\textwidth}
%		\centering 
%		\includegraphics[scale=0.9,height=0.85cm,keepaspectratio]{//odroid-sensors/sensors/MISD_shared/logo/logo_eng_40mm.eps}
%	\column{0.3\textwidth}
%		\centering 
%		\includegraphics[scale=0.9,height=0.85cm,keepaspectratio]{//odroid-sensors/sensors/MISD_shared/logo/logo_box.eps}
%	\column{0.3\textwidth}
%		\centering 
%		\includegraphics[scale=0.9,height=0.85cm,keepaspectratio]{//odroid-sensors/sensors/MISD_shared/logo/logo_box2.eps}
%\end{columns}
%}
%\makeatletter
%\setbeamertemplate{title page}{
%	\begin{minipage}[b][\paperheight]{\textwidth}
%		\centering % <-- Center here
%		\ifx\inserttitlegraphic\@empty\else\usebeamertemplate*{title graphic}\fi
%		\vfill%
%		\ifx\inserttitle\@empty\else\usebeamertemplate*{title}\fi
%		\ifx\insertsubtitle\@empty\else\usebeamertemplate*{subtitle}\fi
%		\usebeamertemplate*{title separator}
%		\ifx\beamer@shortauthor\@empty\else\usebeamertemplate*{author}\fi
%		\ifx\insertdate\@empty\else\usebeamertemplate*{date}\fi
%		\ifx\insertinstitute\@empty\else\usebeamertemplate*{institute}\fi
%		\vfill
%		\vspace*{1mm}
%	\end{minipage}
%}
%
%\setbeamertemplate{title}{
%	% \raggedright% % <-- Comment here
%	\linespread{1.0}%
%	\inserttitle%
%	\par%
%	\vspace*{0.5em}
%}
%\setbeamertemplate{subtitle}{
%	% \raggedright% % <-- Comment here
%	\insertsubtitle%
%	\par%
%	\vspace*{0.5em}
%}
%\makeatother
% end of option 4
%%%%%%%%%%%%%%%%%%%%%%%%%%%%%%%%%%%%%%%%%%%%%%%%%%
% option 5 - 2 Institutes and logos horizontal centered
%%%%%%%%%%%%%%%%%%%%%%%%%%%%%%%%%%%%%%%%%%%%%%%%%%
%\title{My fancy title}
%\subtitle{Lamb-opt }
%\author{\textbf{Paweł Kudela}\textsuperscript{1}, Maciej Radzieński\textsuperscript{1}, Marco Miniaci\textsuperscript{2}}
%
%\institute{ 
%	\begin{columns}[T,onlytextwidth]
%		\column{0.5\textwidth}
%			\centering
%			\textsuperscript{1}Institute of Fluid Flow Machinery\\ \hspace*{3pt}Polish Academy of Sciences
%		\column{0.5\textwidth}
%			\centering
%			\textsuperscript{2}Zurich University
%	\end{columns}
%	\vspace{6pt}
%	% logos 
%	\begin{columns}[b,onlytextwidth]
%		\column{0.5\textwidth}
%		\centering 
%		\includegraphics[scale=0.9,height=0.85cm,keepaspectratio]{//odroid-sensors/sensors/MISD_shared/logo/logo_eng_40mm.eps}
%		\column{0.5\textwidth}
%		\centering 
%		\includegraphics[scale=0.9,height=0.85cm,keepaspectratio]{//odroid-sensors/sensors/MISD_shared/logo/logo_box.eps}
%	\end{columns}
%}
%\makeatletter
%\setbeamertemplate{title page}{
%	\begin{minipage}[b][\paperheight]{\textwidth}
%		\centering % <-- Center here
%		\ifx\inserttitlegraphic\@empty\else\usebeamertemplate*{title graphic}\fi
%		\vfill%
%		\ifx\inserttitle\@empty\else\usebeamertemplate*{title}\fi
%		\ifx\insertsubtitle\@empty\else\usebeamertemplate*{subtitle}\fi
%		\usebeamertemplate*{title separator}
%		\ifx\beamer@shortauthor\@empty\else\usebeamertemplate*{author}\fi
%		\ifx\insertdate\@empty\else\usebeamertemplate*{date}\fi
%		\ifx\insertinstitute\@empty\else\usebeamertemplate*{institute}\fi
%		\vfill
%		\vspace*{1mm}
%	\end{minipage}
%}
%
%\setbeamertemplate{title}{
%	% \raggedright% % <-- Comment here
%	\linespread{1.0}%
%	\inserttitle%
%	\par%
%	\vspace*{0.5em}
%}
%\setbeamertemplate{subtitle}{
%	% \raggedright% % <-- Comment here
%	\insertsubtitle%
%	\par%
%	\vspace*{0.5em}
%}
%\makeatother
% end of option 5
%
%%%%%%%%%%%%%%%%%%%%%%%%%%%%%%%%%%%%%%%%%%%%%%%%%%
% End of title page options
%%%%%%%%%%%%%%%%%%%%%%%%%%%%%%%%%%%%%%%%%%%%%%%%%%
% logo option - alternative manual insertion by modification of coordinates in \put()
%\titlegraphic{%
%	%\vspace{\logoadheight}
%	\begin{picture}(0,0)
%	\put(305,-185){\makebox(0,0)[rb]{\includegraphics[width=4cm]{//odroid-sensors/sensors/MISD_shared/logo/logo_eng_40mm.eps}}}
%	\end{picture}}
%
%%%%%%%%%%%%%%%%%%%%%%%%%%%%%%%%%%%%%%%%%%%%%%%%%%
%\tikzexternalize % activate!
%%%%%%%%%%%%%%%%%%%%%%%%%%%%%%%%%%%%%%%%%%%%%%%%%%
\begin{document}
%%%%%%%%%%%%%%%%%%%%%%%%%%%%%%%%%%%%%%%%%%%%%%%%%%
\maketitle
%%%%%%%%%%%%%%%%%%%%%%%%%%%%%%%%%%%%%%%%%%%%%%%%%%
% SLIDES
%%%%%%%%%%%%%%%%%%%%%%%%%%%%%%%%%%%%%%%%%%%%%%%%%%
\begin{frame}[label=frame1]{Outline}
	\begin{multicols}{2}
%		\fontsize{6pt}{8pt}\selectfont
		\setbeamertemplate{section in toc}[sections numbered]
		\setbeamertemplate{subsection in toc}[subsections numbered]
		\tableofcontents
	\end{multicols}
\end{frame}
%%%%%%%%%%%%%%%%%%%%%%%%%%%%%%%%%%%%%%%%%%%%%%%%%%%%%%%%%%%%%%%%%%%%%%%%%%%%%%%%
\begin{frame}{Genesis (1)}
		\begin{itemize}
			\item 2004: Lectures by Prof. Zenon Waszczyszyn (perceptron, shallow networks, backprop)
			\vspace{-0.3cm}
			\begin{figure}
				\subfloat{\includegraphics[scale=0.7]{/figs2/nn.png}}	
			\end{figure}
			\item 2004: MNIST (Modified National Institute of Standards and Technology) dataset,\\ 28$\times$28 pixels, error rate 0.42\%
			\vspace{-0.5cm}
			\begin{figure}
				\subfloat{\includegraphics[scale=0.4]{/figs2/MnistExamples_cut.png}}	
			\end{figure}
			\item 2012: AlexNet (Alex Krizhevsky, Ilya Sutskever and Geoffrey Hinton), ImageNet Large Scale Visual Recognition Challenge, 227$\times$227 pixels
		\end{itemize}
\end{frame}
%%%%%%%%%%%%%%%%%%%%%%%%%%%%%%%%%%%%%%%%%%%%%%%%%%%%%%%%%%%%%%%%%%%%%%%%%%%%%%%%
\begin{frame}{Genesis (2)}
	\begin{itemize}
		\item 2012: Tesla Model S production
		\item 2014: Tesla Autopilot		
		\item 2016: Tesla's Full Self-Driving, 8 cameras, 1280$\times$960 pixels 
	\end{itemize}
		\vspace{-0.5cm}
		\begin{figure}
		\subfloat{\includegraphics[width=.85\textwidth]{/figs2/TeslaS_Ioniq5_cameras.png}}	
		\end{figure}		
\end{frame}
%%%%%%%%%%%%%%%%%%%%%%%%%%%%%%%%%%%%%%%%%%%%%%%%%%%%%%%%%%%%%%%%%%%%%%%%%%%%%%%%
\section{Motivation}
%%%%%%%%%%%%%%%%%%%%%%%%%%%%%%%%%%%%%%%%%%%%%%%%%%%%%%%%%%%%%%%%%%%%%%%%%%%%%%%%
\begin{frame}{Defects in composite laminates}
	\small
	Composite laminates can have different types of damage such as: \\
	\textbf{Cracks, fibre breakage, debonding, and \alert{delamination}.} \\ 
	\begin{minipage}[c]{0.45\textwidth}
		\begin{itemize}
			\footnotesize
			\item Delamination is a critical failure mechanism in laminated fibre-reinforced polymer matrix composites.
			\item Delamination is one of the most hazardous forms of the defects. 
			It leads to very catastrophic failures if not detected at early stages.
			\item Delamination can be invisible from the outside.
		\end{itemize}
	\end{minipage}
	\hfill
	\begin{minipage}[c]{0.45\textwidth}
		\begin{figure}
			\subfloat{\includegraphics[width=.95\textwidth]{delaminated_plate1.jpg}}
		\end{figure}
	\end{minipage}
\end{frame}
%%%%%%%%%%%%%%%%%%%%%%%%%%%%%%%%%%%%%%%%%%%%%%%%%%%%%%%%%%%%%%%%%%%%%%%%%%%%%%%%
\section{SHM/NDE}
%%%%%%%%%%%%%%%%%%%%%%%%%%%%%%%%%%%%%%%%%%%%%%%%%%%%%%%%%%%%%%%%%%%%%%%%%%%%%%%%
\begin{frame}{Non Destructive Testing}
%%%%%%%%%%%%%%%%%%%%%%%%%%%%%%%%%%%%%%%%%%%%%%%%%%%%%%%%%%%%%%%%%%%%%%%%%%%%%%%%
	\begin{columns}[T]
		\begin{column}{0.5\textwidth}
			\centering
			\hspace{2cm}\textbf{Ultrasonic testing}	
		\end{column}
		\begin{column}{0.5\textwidth}
			\centering
			\hspace{-2cm}\textbf{Guided wave testing}	
		\end{column}
	\end{columns}	
\begin{figure}
\subfloat{\includegraphics[width=0.7\textwidth]{local_ultrasonic.png}}	
\end{figure}
\end{frame}
%%%%%%%%%%%%%%%%%%%%%%%%%%%%%%%%%%%%%%%%%%%%%%%%%%%%%%%%%%%%%%%%%%%%%%%%%%%%%%%%
\begin{frame}{Ultrasonic and guided waves}
%%%%%%%%%%%%%%%%%%%%%%%%%%%%%%%%%%%%%%%%%%%%%%%%%%%%%%%%%%%%%%%%%%%%%%%%%%%%%%%%
\alert{Bulk waves} exist in infinite homogenous bodies and propagate indefinitely without being interrupted by boundaries or interfaces. 
These waves can be decomposed into infinite plane waves propagating along arbitrary direction within the solid.
	
\alert{Guided waves} are those waves that require a boundary for their existence, such as surface waves, Lamb waves, and interface waves.
	\vspace{5mm}
	\begin{columns}[T]
		\begin{column}{0.45\textwidth}
			\textbf{Ultrasonic waves}	
			\begin{itemize}
				\item Frequency range: 2 MHz - 200 MHz
				\item Wavelength \(\lambda << h\) thickness 
				\item shorter wavelengths
			\end{itemize}
		\end{column}
		\begin{column}{0.45\textwidth}
			\textbf{Guided waves}	
			\begin{itemize}
				\item Frequency range: 10 kHz - 1 MHz
				\item Wavelength \(\lambda > h\) thickness 
				\item longer wavelengths
			\end{itemize}
		\end{column}
	\end{columns}			
\end{frame}
%%%%%%%%%%%%%%%%%%%%%%%%%%%%%%%%%%%%%%%%%%%%%%%%%%%%%%%%%%%%%%%%%%%%%%%%%%%%%%%%
\section{Guided waves}
%%%%%%%%%%%%%%%%%%%%%%%%%%%%%%%%%%%%%%%%%%%%%%%%%%%%%%%%%%%%%%%%%%%%%%%%%%%%%%%%
\setcounter{subfigure}{0}
%%%%%%%%%%%%%%%%%%%%%%%%%%%%%%%%%%%%%%%%%%%%%%%%%%%%%%%%%%%%%%%%%%%%%%%%%%%%%%%%
\begin{frame}{Lamb waves}
%%%%%%%%%%%%%%%%%%%%%%%%%%%%%%%%%%%%%%%%%%%%%%%%%%%%%%%%%%%%%%%%%%%%%%%%%%%%%%%%
	\begin{alertblock}{Lamb waves}	
		Lamb waves are plane waves propagating in thin plates.\\
		Shear vertical waves in conjunction with longitudinal P waves interacts with plate surfaces resulting in complex wave mechanism which leads to creation of Lamb waves.
	\end{alertblock}
	Horace Lamb discovered these type of waves in 1917.
	He derived theory and dispersion relations.
	\begin{columns}[T]
		\begin{column}{0.5\textwidth}
			\centering
			symmetric modes
			\begin{equation*}
				\frac{\tan(q h)}{\tan(p h)} = -\frac{4 k^2 p q}{\left(q^2 - k^2\right)^2}
			\end{equation*}
		\end{column}
		\begin{column}{0.5\textwidth}
			\centering
			antisymmetric modes
			\begin{equation*}
				\frac{\tan(q h)}{\tan(p h)} = -\frac{\left(q^2 - k^2\right)^2}{4 k^2 p q}
			\end{equation*}
		\end{column}	
	\end{columns}	
	\vspace{5mm}
	\centering
	\(q=q(\omega,k), \quad p=p(\omega,k) \)\\
	\vspace{5mm}
	$\omega$ - angular frequency, $k$ - wavenumber
\end{frame}
%%%%%%%%%%%%%%%%%%%%%%%%%%%%%%%%%%%%%%%%%%%%%%%%%%%%%%%%%%%%%%%%%%%%%%%%%%%%%%%%
\begin{frame}{Dispersion curves of Lamb waves}
%%%%%%%%%%%%%%%%%%%%%%%%%%%%%%%%%%%%%%%%%%%%%%%%%%%%%%%%%%%%%%%%%%%%%%%%%%%%%%%%
	\begin{columns}[T]
			\only<1>{
				\column{0.1\textwidth}
				\centering
				\begin{equation*}
					c_p = \frac{\omega}{k}
				\end{equation*}
				\column{0.9\textwidth}
				\begin{figure}
					\includegraphics[width=0.9\textwidth]{/figs/Fig_1_12.png}
				\end{figure}
			}
			\only<2>{
				\column{0.1\textwidth}
				\centering
				\begin{equation*}
					c_g = \frac{\textrm{d}\omega}{\textrm{d}k}
				\end{equation*}
				\column{0.9\textwidth}
				\begin{figure}
					\includegraphics[width=0.9\textwidth]{/figs/Fig_1_13.png}
				\end{figure}
			}
	\end{columns}
\end{frame}
%%%%%%%%%%%%%%%%%%%%%%%%%%%%%%%%%%%%%%%%%%%%%%%%%%%%%%%%%%%%%%%%%%%%%%%%%%%%%%%%
\begin{frame}{Semi Analytical Spectral Element Method (SASE)}
%%%%%%%%%%%%%%%%%%%%%%%%%%%%%%%%%%%%%%%%%%%%%%%%%%%%%%%%%%%%%%%%%%%%%%%%%%%%%%%%
	\begin{figure}
		\includegraphics[width=\textwidth]{/figs2/SASE_model.png}
	\end{figure}
	\begin{equation*}
		\vect{u}(x,y,z,t) = \matr{U}(x) \exp \left[ i (\omega t + k \sin (\beta) y - k \cos (\beta) z)\right]
	\end{equation*}
\end{frame}
%%%%%%%%%%%%%%%%%%%%%%%%%%%%%%%%%%%%%%%%%%%%%%%%%%%%%%%%%%%%%%%%%%%%%%%%%%%%%%%%
\begin{frame}[t]{Semi Analytical Spectral Element Method (SASE)}
%%%%%%%%%%%%%%%%%%%%%%%%%%%%%%%%%%%%%%%%%%%%%%%%%%%%%%%%%%%%%%%%%%%%%%%%%%%%%%%%
		\begin{equation*}
			\left[\matr{A} - \omega^2\matr{M} \right] \vect{U} =0,
			\label{eq:eig_dispersion}
		\end{equation*}
		where $\omega$ is the angular frequency, $\matr{M}$ is the mass matrix, $\matr{U}$ is the nodal displacement vector, and the matrix $\matr{A}$ can be defined as:
		\begin{equation*}
			\begin{aligned}
				\matr{A} & =  k^2\left(s^2 \,\matr{K}_{22} + c^2\, \matr{K}_{33} - c s\, \matr{K}_{23} - c s\, \matr{K}_{32}\right) \\
				& + i k\, \matr{T}^T\left(-c\, \matr{K}_{13} - s\, \matr{K}_{21} + s\, \matr{K}_{12} + c\, \matr{K}_{31}\right) \matr{T} +\matr{K}_{11},
			\end{aligned}
			\label{eq:dispersion}
		\end{equation*}
		where  $s = \sin(\beta)$, $c = \cos(\beta)$, $i = \sqrt{-1}$.
		\begin{equation*}
			\matr{K}_{mn}^e= \int \limits_{(e)} \matr{B}_m^{T} \matr{C}^e \, \matr{B}_n\, \ud x
			\label{eq:stiffness_matrix_e}
		\end{equation*}
		Possible solutions:
		\begin{columns}[T]
			\column{0.5\textwidth}
			\begin{itemize}
				\item standard eigenvalue problem $\omega (k)$
			\end{itemize}
			\column{0.5\textwidth}
			\begin{itemize}
				\item second-order polynomial eigenvalue problem $k(\omega)$
			\end{itemize}
		\end{columns}
\end{frame}
%%%%%%%%%%%%%%%%%%%%%%%%%%%%%%%%%%%%%%%%%%%%%%%%%%
\begin{frame}[t,label=frame11]{Semi Analytical Spectral Element Method (SASE)}
	\vspace{-2mm}
		\begin{columns}[T]
			\column{0.5\textwidth}
			\textbf{standard eigenvalue problem} $\omega (k)$
			\begin{itemize}
				\item real $k$ -- real $\omega$
				\item only dispersion curves
			\end{itemize}
			\column{0.5\textwidth}
			\textbf{second-order polynomial eigenvalue problem} $k(\omega)$
			\begin{itemize}
				\item real $\omega$ -- complex $k$
				\item dispersion curves and attenuation (complex $\matr{C}$)
			\end{itemize}
		\end{columns}
		\vspace{2pt}
		\begin{columns}[T]
			\column{0.5\textwidth}
			\begin{equation*}
				\left[\matr{A} - \omega^2\matr{M} \right]_{\alert{M}} \vect{U} =0
			\end{equation*}
			\column{0.5\textwidth}
			\begin{equation*}
				\left[\hat{\matr{A}} - k \hat{\matr{D}} \right]_{\alert{2M}} \hat{\vect{Q}} =0
			\end{equation*}
			\begin{equation*}
				\hat{\vect{Q}} =\left[\begin{array}{c} 
					\vect{U}\\
					k \vect{U}
				\end{array} \right]
			\end{equation*}
		\end{columns}
		\begin{flalign*}
			&\hat{\matr{A}} =\left[\begin{array}{cc} 
				0 & \matr{K}_{11} - \omega^2 \matr{M}\\
				\matr{K}_{11} - \omega^2 \matr{M} & -i \left( c	\, \matr{K}_{13} - s\, \matr{K}_{12}  + s\, \matr{K}_{21} - c \, \matr{K}_{31}   \right)
			\end{array} \right]   \\
			&\hat{\matr{D}} =\left[\begin{array}{cc} 
				\matr{K}_{11} - \omega^2 \matr{M} & 0\\
				0& - \left( s^2 \, \matr{K}_{22} + c^2 \,  \matr{K}_{33}  -s c \,  \matr{K}_{23}  -sc \, \matr{K}_{32}  \right)
			\end{array} \right]
		\end{flalign*}
\end{frame}
%%%%%%%%%%%%%%%%%%%%%%%%%%%%%%%%%%%%%%%%%%%%%%
\begin{frame}{A0 mode dispersion curves of CFRP composite}
	%%%%%%%%%%%%%%%%%%%%%%%%%%%%%%%%%%%%%%%%%%%%%%
	\begin{columns}[T]
		\column{0.5\textwidth}
			\begin{figure}
				\includegraphics[width=0.9\textwidth]{/figs/dispersion/A0_group_velocity_less_dispersive.png}
				\caption{Less dispersive region}
			\end{figure}
			\column{0.5\textwidth}
			\begin{figure}
				\includegraphics[width=0.9\textwidth]{/figs/dispersion/A0_group_velocity_dispersive.png}
				\caption{Dispersive region}
			\end{figure}
	\end{columns}
	\begin{center}
	\alert{Group velocities}
	\end{center}	
\end{frame}
%%%%%%%%%%%%%%%%%%%%%%%%%%%%%%%%%%%%%%%%%%%%%%%%%%%%%%%%%%%%%%%%%%%%%%%%%%%%%%%%
\begin{frame}{Dispersion effect}
%%%%%%%%%%%%%%%%%%%%%%%%%%%%%%%%%%%%%%%%%%%%%%%%%%%%%%%%%%%%%%%%%%%%%%%%%%%%%%%%
	\begin{columns}[c]
		\begin{column}{0.2\textwidth}
			\centering
			A0 mode \\carrier frequency \\$f_c$ = 200 kHz
		\end{column}
		\begin{column}{0.8\textwidth}
			\begin{figure}
				\animategraphics[autoplay,loop,width=0.95\textwidth]{1}{/figs/dispersion/dispersion_effect_less_dispersive_L_}{1}{11}
			\end{figure}
		\end{column}
	\end{columns}
	\begin{columns}[c]
		\begin{column}{0.2\textwidth}
			\centering
			A0 mode \\carrier frequency \\$f_c$ = 50 kHz
		\end{column}
		\begin{column}{0.8\textwidth}
			\begin{figure}
				\animategraphics[autoplay,loop,width=0.95\textwidth]{1}{/figs/dispersion/dispersion_effect_dispersive_L_}{1}{11}
			\end{figure}
		\end{column}
	\end{columns}
\end{frame}
%%%%%%%%%%%%%%%%%%%%%%%%%%%%%%%%%%%%%%%%%%%%%%%%%%%%%%%%%%%%%%%%%%%%%%%%%%%%%%%%
\section{Experimental measurements}
%%%%%%%%%%%%%%%%%%%%%%%%%%%%%%%%%%%%%%%%%%%%%%%%%%%%%%%%%%%%%%%%%%%%%%%%%%%%%%%%
\begin{frame}[t]{Composite specimen}
	\begin{columns}[T]
		\column{0.7\textwidth}
		{\small
			\begin{itemize}
				\item 16 layers set at the same angle \\
				\item carbon: Prepreg GG 205  P (fibres Toray FT 300 - 3K 200 tex), $E=230$ GPa
				\item epoxy resin: IMP503Z-HT by Impregnatex Compositi 
				\item dimensions: 500$\times$500$\times$3.9 mm\\
				\item density: 1522.4~kg/m\textsuperscript{3}
			\end{itemize}
		}
		\column{0.3\textwidth}
		\begin{figure}
			\includegraphics[width=0.6\textwidth]{/figs2/weave-1.jpg}
			\caption{Plain weave fabric}
		\end{figure}
	\end{columns}
	\begin{table}[h]
		\renewcommand{\arraystretch}{1.1}
		\centering \footnotesize
		\caption{Geometry of a plain weave fabric reinforced composite [mm]}
		\begin{tabular}{cccccc} 
			%\hline
			\toprule
			\multicolumn{4}{c}{\textbf{width} }	& \multicolumn{2}{c}{\textbf{thickness} }  \\ 
			%	\hline \hline
			\cmidrule(lr){1-4} \cmidrule(lr){5-6} 
			fill & warp & fill gap& warp gap& fill & warp\\
			%\hline
			$a_f$ &$a_w$& $g_f$  & $g_w$  & $h_f$& $h_w$ \\ 
			%\hline
			%\midrule
			\cmidrule(lr){1-2} \cmidrule(lr){3-4} \cmidrule(lr){5-6}
			1.92 &2.0& 0.05& 0.05 & 0.121875 & 0.121875 \\
			%\hline 
			\bottomrule 
		\end{tabular} 
		\label{tab:weave_geo}
	\end{table}
\end{frame}
%%%%%%%%%%%%%%%%%%%%%%%%%%%%%%%%%%%%%%%%%%%%%%%%%%%%%%%%%%%%%%%%%%%%%%%%%%%%%%%%
\begin{frame}[t]{Specimens with defects}
%%%%%%%%%%%%%%%%%%%%%%%%%%%%%%%%%%%%%%%%%%%%%%%%%%%%%%%%%%%%%%%%%%%%%%%%%%%%%%%%
\vspace{-0.5cm}
\begin{columns}[T]
	\column{0.5\textwidth}
	\begin{figure}
		\includegraphics[scale=0.36]{figs2/plate_multi_delam_arrangement_large_fonts.png}
	\end{figure}
	\column{0.5\textwidth}
	\begin{figure}
		\includegraphics[scale=0.36]{figs2/plate_single_delam_arrangement_large_fonts.png}
	\end{figure}
\end{columns}
\end{frame}
%%%%%%%%%%%%%%%%%%%%%%%%%%%%%%%%%%%%%%%%%%%%%%%%%%%%%%%%%%%%%%%%%%%%%%%%%%%%%%%%
\begin{frame}[t]{SLDV measurements: setup}
%%%%%%%%%%%%%%%%%%%%%%%%%%%%%%%%%%%%%%%%%%%%%%%%%%%%%%%%%%%%%%%%%%%%%%%%%%%%%%%%
	\begin{figure}
		\includegraphics[width=0.7\textwidth]{figs2/sensors_fig4_setup.png}
	\end{figure}
\end{frame}
%%%%%%%%%%%%%%%%%%%%%%%%%%%%%%%%%%%%%%%%%%%%%%%%%%%%%%%%%%%%%%%%%%%%%%%%%%%%%%%%
\begin{frame}[t]{SLDV measurements: laboratory}
%%%%%%%%%%%%%%%%%%%%%%%%%%%%%%%%%%%%%%%%%%%%%%%%%%%%%%%%%%%%%%%%%%%%%%%%%%%%%%%%
	\begin{columns}[T]
		\column{0.5\textwidth}
		\begin{figure}
			\includegraphics[width=0.8\textwidth]{wibrometr-laserowy-1d_small-description.png}
		\end{figure}
		\column{0.5\textwidth}
		\begin{enumerate}
			\item Signal generator: TTI 1241 
			\item Amplifier: Piezo Systems EPA-104-230 $\pm$200 Vp
			\item Specimen
			\item Scanning head: Polytec PSV-400
			\item DAQ system: Polytec
		\end{enumerate}
	\end{columns}
	{\small
		Measurements were taken on a uniform grid of \textbf{333$\times$333 points}.\\
		Excitation in the form of Hann windowed sine signal of carrier frequency \textbf{50 kHz} was applied to piezoelectric transducer.}
\end{frame}
%%%%%%%%%%%%%%%%%%%%%%%%%%%%%%%%%%%%%%%%%%%%%%%%%%%%%%%%%%%%%%%%%%%%%%%%%%%%%%%%
\begin{frame}[t]{Full wavefield measured by SLDV}
	%%%%%%%%%%%%%%%%%%%%%%%%%%%%%%%%%%%%%%%%%%%%%%%%%%%%%%%%%%%%%%%%%%%%%%%%%%%%%%%%
	\begin{columns}[T]
		\column{0.5\textwidth}
		\only<1>{
			\vspace{-1mm}
			\begin{figure}
				\includegraphics[scale=0.5]{figs2/specimenIII_multi_delam_arrangement_large_fonts.png}
			\end{figure}
		}
		\only<2>{
			\vspace{-1pt}
			\begin{flushleft}	
				\begin{figure}
					\hspace{16mm}\animategraphics[autoplay,loop,scale=0.29]{8}{/figs2/L3_S3_B_frames/L3_S3_B_frame_}{10}{256}
					\caption{\hspace{26mm}SLDV Wavefield\\ \hspace{26mm}(experiment: specimen III)}
				\end{figure}
			\end{flushleft}	
		}
		\column{0.5\textwidth}
		\only<1>{
			\vspace{-1mm}
			\begin{figure}
				\includegraphics[scale=0.5]{figs2/specimenV_single_delam_arrangement_large_fonts.png}	
			\end{figure}
		}
		\only<2>{
			\vspace{-1pt}
			\begin{flushleft}
				\begin{figure}
					\hspace{1pt}\hspace{16mm}\animategraphics[autoplay,loop,scale=0.29]{8}{/figs2/Specimen_V_frames/Specimen_V_frame_}{10}{256}
					\caption{\hspace{26mm}SLDV wavefield\\ \hspace{26mm}(experiment: specimen V)}
				\end{figure}
			\end{flushleft}	
		}
	\end{columns}
\end{frame}
%%%%%%%%%%%%%%%%%%%%%%%%%%%%%%%%%%%%%%%%%%%%%%%%%%
\section{Wavefield imaging}
%%%%%%%%%%%%%%%%%%%%%%%%%%%%%%%%%%%%%%%%%%%%%%%%%%
%%%%%%%%%%%%%%%%%%%%%%%%%%%%%%%%%%%%%%%%%%%%%%%%%%
\begin{frame}{Image processing}
%%%%%%%%%%%%%%%%%%%%%%%%%%%%%%%%%%%%%%%%%%%%%%%%%%
	\begin{columns}[T]
		\begin{column}{0.4\textwidth}
			\begin{figure}
				\animategraphics[autoplay,loop,width=0.9\textwidth]{3}{/figs2/Lenna/Lena}{1}{46}
				%\includegraphics[width=0.8\textwidth]{/figs2/Lena1.png}
				\caption{Lenna}
			\end{figure}
		\end{column}
		\begin{column}{0.6\textwidth}
			2D FFT
			\begin{equation*}
				f(x,y) \xrightarrow{\mathcal{F}} \hat{f}(k_x,k_y)
			\end{equation*}
			\(k_x,\, k_y\) - wavenumbers in \(x\) and \(y\) direction, respectively.\\
			Image processing strategy:
			\begin{equation*}
				\alert{I(x,y) \xrightarrow{\mathcal{F}} \hat{I}(k_x,k_y) \rightarrow \hat{I}(k_x,k_y) \cdot \hat{M}(k_x,k_y)  \xrightarrow{\mathcal{F}^{-1}}  I(x,y)}
			\end{equation*}
			\(\cdot\) is element-wise multiplication; (.* in Matlab)
		\end{column}
	\end{columns}
\end{frame}
%%%%%%%%%%%%%%%%%%%%%%%%%%%%%%%%%%%%%%%%%%%%%%%%%%%%%%%%%%%%%%%%%%%%%%%%%%%%%%%%
\begin{frame}[t]{Conventional signal processing and wavefield imaging}
	\begin{figure}
		\includegraphics[width=0.8\textwidth]{figs2/sensors_fig1_algorithm.png}
	\end{figure}
\biblioref{M. Radzienski, P. Kudela, A. Marzani, L. de Marchi, W. Ostachowicz}{2019}{ Damage Identification in Various Types of Composite Plates Using Guided Waves Excited by a Piezoelectric Transducer and Measured by a Laser Vibrometer}{Sensors, 19, 1958}
\end{frame}
%%%%%%%%%%%%%%%%%%%%%%%%%%%%%%%%%%%%%%%%%%%%%%%%%%%%%%%%%%%%%%%%%%%%%%%%%%%%%%%%
\begin{frame}[t]{Wavefield imaging results}
%%%%%%%%%%%%%%%%%%%%%%%%%%%%%%%%%%%%%%%%%%%%%%%%%%%%%%%%%%%%%%%%%%%%%%%%%%%%%%%%
	\begin{columns}[T]
		\column{0.5\textwidth}
		\only<1>{
	    \vspace{-1mm}
		\begin{figure}
			\includegraphics[scale=0.5]{figs2/specimenIII_multi_delam_arrangement_large_fonts.png}
		\end{figure}
		}
		\only<2>{
		\vspace{-1pt}
			\begin{flushleft}	
				\begin{figure}
				\hspace{16mm}\includegraphics[scale=1]{figs2/ERMSF_specimen_III.png}
				\caption{\hspace{26mm}Wavenumber imaging\\ \hspace{26mm}(experiment: specimen III)}
				\end{figure}
			\end{flushleft}	
		}
		\column{0.5\textwidth}
		\only<1>{
		\vspace{-1mm}
		\begin{figure}
			\includegraphics[scale=0.5]{figs2/specimenV_single_delam_arrangement_large_fonts.png}	
		\end{figure}
		}
		\only<2>{
		\vspace{-1pt}
		\begin{flushleft}
			\begin{figure}
				\hspace{1pt}\hspace{16mm}\includegraphics[scale=1]{figs2/ERMSF_specimen_V.png}
				\caption{\hspace{26mm}Wavenumber imaging\\ \hspace{26mm}(experiment: specimen V)}
			\end{figure}
		\end{flushleft}	
		}
	\end{columns}
\end{frame}
%%%%%%%%%%%%%%%%%%%%%%%%%%%%%%%%%%%%%%%%%%%%%%%%%%%%%%%%%%%%%%%%%%%%%%%%%%%%%%%%
\section{Damage detection approaches}
%%%%%%%%%%%%%%%%%%%%%%%%%%%%%%%%%%%%%%%%%%%%%%%%%%%%%%%%%%%%%%%%%%%%%%%%%%%%%%%%
\begin{frame}{Conventional vs deep learning approach}
	Conventional: \alert{Feature extraction and classification}
	\begin{figure}
		\subfloat{\includegraphics[scale=1]{conventional_ML.png}}
	\end{figure}	
	Deep learning \alert{end-to-end approach: automatic feature extraction and classification}
	\begin{figure}
		\subfloat{\includegraphics[scale=1]{DL_approach.png}}
	\end{figure}
\end{frame}
%\setcounter{subfigure}{0}
%\section{Artificial intelligence, machine learning, and deep learning}
%%%%%%%%%%%%%%%%%%%%%%%%%%%%%%%%%%%%%%%%%%%%%%%%%%%
%%%%%%%%%%%%%%%%%%%%%%%%%%%%%%%%%%%%%%%%%%%%%%%%%%%
%\begin{frame}{What is deep learning?}
%	\begin{figure}
%		\centering
%		\includegraphics[width=0.85\textwidth]{AI_vs_ML_vs_Deep_Learning.png}
%	\end{figure}
%	\tiny
%	(source: https://www.ingeniovirtual.com/)
%\end{frame}
%
%%%%%%%%%%%%%%%%%%%%%%%%%%%%%%%%%%%%%%%%%%%%%%%%%%%%%%%%%%%%%%%%%%%%%%%%%%%%%%%%%
%\setcounter{subfigure}{0}
%%%%%%%%%%%%%%%%%%%%%%%%%%%%%%%%%%%%%%%%%%%%%%%%%%%
%\begin{frame}{Deep learning, why now?}
%	\begin{minipage}[c]{0.4\textwidth}
%		AI technologies are in accelerating growth due to:
%		\begin{itemize}
%			\item Exponential development in computer hardware industries
%			 (e.g. CPUs, GPUs, FPGAs, TPUs and ASICs)
%			\item Era of Big data.
%		\end{itemize}
%	\end{minipage}
%	\begin{minipage}[c]{0.55\textwidth}
%		\begin{figure}
%			\centering
%			\subfloat{\animategraphics[autoplay,loop,width=.9\textwidth]{10}{gif_figs/gpu/gpu_-}{0}{34}}
%		\end{figure}
%	\tiny
%	(source: https://www.techbooky.com/)
%	\end{minipage}
%	
%\end{frame}
%%%%%%%%%%%%%%%%%%%%%%%%%%%%%%%%%%%%%%%%%%%%%%%%%%%
%\setcounter{subfigure}{0}
%%%%%%%%%%%%%%%%%%%%%%%%%%%%%%%%%%%%%%%%%%%%%%%%%%%
%\begin{frame}{Common learning strategies}
%	\centering
%	\begin{figure}
%		\includegraphics[width=0.9\textwidth]{learning.png}
%	\end{figure}
%	\tiny
%	(source: https://www.aitude.com/supervised-vs-unsupervised-vs-reinforcement/)
%\end{frame}
%%%%%%%%%%%%%%%%%%%%%%%%%%%%%%%%%%%%%%%%%%%%%%%%%%%%%%%%%%%%%%%%%%%%%%%%%%%%%%%%%%
\section{Synthetic dataset generation}
\setcounter{subfigure}{0}
%%%%%%%%%%%%%%%%%%%%%%%%%%%%%%%%%%%%%%%%%%%%%%%%%%
%\subsection{Synthetic Dataset of propagating Lamb waves}
%\setcounter{subfigure}{0}
%%%%%%%%%%%%%%%%%%%%%%%%%%%%%%%%%%%%%%%%%%%%%%%%%%
%%%%%%%%%%%%%%%%%%%%%%%%%%%%%%%%%%%%%%%%%%%%%%%%%%%%%%%%%%%%%%%%%%%%%%%%%%%%%%%%
\begin{frame}{The time domain spectral element method (1)}
	%%%%%%%%%%%%%%%%%%%%%%%%%%%%%%%%%%%%%%%%%%%%%%%%%%%%%%%%%%%%%%%%%%%%%%%%%%%%%%%%
	\begin{columns}[T]
		\begin{column}{0.65\textwidth}
			\begin{figure}
				\subfloat{\includegraphics[width=0.9\textwidth]{/figs2/pzt_plate_eng_complex.png}}	
			\end{figure}
		\end{column}
		\begin{column}{0.3\textwidth}	
			\begin{figure}
				\subfloat{\includegraphics[width=0.7\textwidth]{/figs2/Wiley-cover.jpg}}	
			\end{figure}
		\end{column}
	\end{columns}	
\end{frame}
%%%%%%%%%%%%%%%%%%%%%%%%%%%%%%%%%%%%%%%%%%%%%%%%%%%%%%%%%%%%%%%%%%%%%%%%%%%%%%%%
\begin{frame}{The time domain spectral element method (2)}
%%%%%%%%%%%%%%%%%%%%%%%%%%%%%%%%%%%%%%%%%%%%%%%%%%%%%%%%%%%%%%%%%%%%%%%%%%%%%%%%
	\begin{columns}[T]
		\begin{column}{0.47\textwidth}
			\begin{itemize}
				\item Mindlin-Reissner plate theory
				\item Splitting elements and nodes at delamination
				\item GMSH software was used for meshing quads than converted to spectral elements
			\end{itemize}	
			\begin{figure}
				\subfloat{\includegraphics[width=0.9\textwidth]{/figs2/shell.png}}	
			\end{figure}
		\end{column}
		\begin{column}{0.47\textwidth}	
			\begin{figure}
				\animategraphics[controls,autoplay,loop,width=0.9\textwidth]{1}{/figs2/mesh/m1_rand_single_delam_}{1}{20}
			\end{figure}	
		\end{column}
	\end{columns}	
\end{frame}
%%%%%%%%%%%%%%%%%%%%%%%%%%%%%%%%%%%%%%%%%%%%%%%%%%%%%%%%%%%%%%%%%%%%%%%%%%%%%%%%
\begin{frame}{Dataset description (1)}
%%%%%%%%%%%%%%%%%%%%%%%%%%%%%%%%%%%%%%%%%%%%%%%%%%%%%%%%%%%%%%%%%%%%%%%%%%%%%%%%
	\begin{minipage}[l]{0.5\textwidth}
		\begin{itemize}
			\item 475 delamination scenarios
			\item CFRP is made of 8-layers
			\item Delamination modelled between the 3rd and 4th layer
			\item Delamination size min 10 mm, max  40 mm
			\item \textbf{3-months of computing}
		\end{itemize}
	\end{minipage}
	\begin{minipage}[c]{0.45\textwidth}
		\begin{figure}
			\centering
			\subfloat{\includegraphics[width=0.7\textwidth]{figure_overlap.png}}
			\caption{All random delaminations}
		\end{figure}
	\end{minipage}
\end{frame}
%%%%%%%%%%%%%%%%%%%%%%%%%%%%%%%%%%%%%%%%%%%%%%%%%%%%%%%%%%%%%%%%%%%%%%%%%%%%%%%%
\begin{frame}{Dataset description (2)}
	%%%%%%%%%%%%%%%%%%%%%%%%%%%%%%%%%%%%%%%%%%%%%%%%%%%%%%%%%%%%%%%%%%%%%%%%%%%%%%%%
	\begin{minipage}[l]{0.59\textwidth}
		First trial - \alert{failure!}
		\begin{itemize}
			\item \textbf{Delamination locations on uniform grid of coordinates}
			\item Random delamination sizes and angles
			\item Plain weave approximated by [0/90]\textsubscript{4}; rule of mixture and homogenisation; model fitting to experimental data by changing volume fraction of reinforcing fibres 
		\end{itemize}
	Second trial - \alert{success!}
	\begin{itemize}
		\item \textbf{Random delamination locations}
		\item Random delamination sizes and angles
		\item Elastic constants identified by using dispersion curves and GA algorithm
		\item A0 mode $\lambda_{num}$ = 19.5 mm, $\lambda_{exp}$ = 21.2 mm
	\end{itemize}
	\end{minipage}
	\begin{minipage}[c]{0.39\textwidth}
		\begin{figure}
			\centering
			\subfloat{\includegraphics[width=0.9\textwidth]{/figs2/elastic_C_identif_dispersion.png}}
			\caption{\hspace{10mm}SASE dispersion curves fitted \\ \hspace{10mm}to the experiment}
		\end{figure}
	\biblioref{P. Kudela, M. Radzienski, P. Fiborek, T. Wandowski}{2020}{Elastic constants identification of woven fabric reinforced composites by
using guided wave dispersion curves and genetic algorithm}{Composite Structures, 249, 112569}	
	\end{minipage}
\end{frame}
%%%%%%%%%%%%%%%%%%%%%%%%%%%%%%%%%%%%%%%%%%%%%%%%%%%%%%%%%%%%%%%%%%%%%%%%%%%%%%%%
\begin{frame}{Parallel implementation (1)}
%%%%%%%%%%%%%%%%%%%%%%%%%%%%%%%%%%%%%%%%%%%%%%%%%%%%%%%%%%%%%%%%%%%%%%%%%%%%%%%%
	\begin{columns}[T]
		\begin{column}{0.4\textwidth}
			\begin{figure}
				\subfloat{\includegraphics[scale=0.7]{/figs2/CPUvsGPU.png}}	
			\end{figure}
		\end{column}
		\begin{column}{0.6\textwidth}
			\begin{minipage}[c]{0.47\textwidth}
				Nvidia Tesla K20X
				\begin{itemize}
					\item 2688 cores
					\item 1313 GFLOPS \\(double precision)
					\item 6GB memory
				\end{itemize}	
			\end{minipage}
			\begin{minipage}[c]{0.47\textwidth}
				Nvidia Tesla V100
				\begin{itemize}
					\item 5120 cores
					\item 6605 GFLOPS \\(double precision)
					\item 32GB memory
				\end{itemize}	
			\end{minipage}
		\begin{figure}
			\subfloat{\includegraphics[scale=0.8]{/figs2/speedup.png}}	
		\end{figure}
		\end{column}
	\end{columns}	
\end{frame}
%%%%%%%%%%%%%%%%%%%%%%%%%%%%%%%%%%%%%%%%%%%%%%%%%%%%%%%%%%%%%%%%%%%%%%%%%%%%%%%%
\begin{frame}{Parallel implementation (2)}
%%%%%%%%%%%%%%%%%%%%%%%%%%%%%%%%%%%%%%%%%%%%%%%%%%%%%%%%%%%%%%%%%%%%%%%%%%%%%%%%
	\begin{columns}[T]
		\begin{column}{0.45\textwidth}
			\setlength{\abovedisplayskip}{-\baselineskip}
				\begin{fleqn}
				\begin{equation*}
					\matr{M} \vect{\ddot{U}} + \matr{C} \vect{\dot{U}} + \matr{K} \vect{U} = \vect{F} \label{eq:motion}
				\end{equation*}  
				\begin{equation*}
					\ddot{\vect{U}}\simeq \frac{1}{\Delta t^2} \left(\vect{u}_{t+\Delta t} - 2\,\vect{u}_t + \vect{u}_{t-\Delta t}\right) \label{eq:central_scheme}
				\end{equation*}
				\begin{equation*}
					\dot{\vect{U}}\simeq \frac{\vect{u}_{t+\Delta t} -\vect{u}_{t-\Delta t}}{2 \Delta t}
					\label{eq:first_derivative_scheme}
				\end{equation*}
				\end{fleqn}
		\end{column}
		\begin{column}{0.45\textwidth}	
			Simplifications:
			\begin{itemize}
				\item Equivalent piezoelectric forces, $\vect{F}$
				\item Damping proportional to mass matrix, $\matr{C}=\alpha\vect{M}$
			\end{itemize}	
		\end{column}
	\end{columns}
\begin{fleqn}
\begin{equation*}
	\underbrace{\left(\frac{1}{\Delta t^2} \,\matr{M} + \frac{1}{2 \Delta t} \matr{C}\right)}_{\vect{M}_0} \vect{u}_{t+\Delta t} = \vect{F}_t - \alert{\underbrace{\matr{K} \vect{u}_t}_{\vect{F}^i}} + \underbrace{\left(\frac{2}{\Delta t^2} \,\matr{M} \right)}_{\vect{M}_1}\vect{u}_t 
	+ \underbrace{\left(- \frac{1}{\Delta t^2} \,\matr{M} + \frac{1}{2 \Delta t} \matr{C}\right)}_{\vect{M}_2} \vect{u}_{t-\Delta t}
	\label{eq:explicit_integration}
\end{equation*}
\alert{$\vect{M}_0,\, \vect{M}_1,\, \vect{M}_2$ are diagonal!};\vspace{1mm} Parallel solution on dof level: $\vect{F}^i \leftarrow \bf{\sigma} \leftarrow \bf{\epsilon} \leftarrow \vect{u}$; \vspace{1mm}$\matr{K}$ not assembled!
\end{fleqn}	
\biblioref{P. Kudela}{2016}{Parallel implementation of spectral element method for Lamb
wave propagation modeling}{International Journal for Numerical Methods in Engineering, 106, 413-429}	
\biblioref{P. Kudela, J. Moll, P. Fiborek}{2020}{Parallel spectral element method for guided wave based structural
health monitoring}{Smart Materials and Structures, 29, 095010}	
\end{frame}
%%%%%%%%%%%%%%%%%%%%%%%%%%%%%%%%%%%%%%%%%%%%%%%%%%%%%%%%%%%%%%%%%%%%%%%%%%%%%%%%
\setcounter{subfigure}{0}
\begin{frame}{Training Sample case}
%%%%%%%%%%%%%%%%%%%%%%%%%%%%%%%%%%%%%%%%%%%%%%%%%%%%%%%%%%%%%%%%%%%%%%%%%%%%%%%%
	\begin{figure}
		\centering
		\subfloat[Full wavefield $s(x,y,t_k)$ \label{fig:3}]{\animategraphics[autoplay,loop, controls,width=4cm]{16}{figures/gif_figs/7_output/flat_shell_Vz_7_500x500bottom-}{1}{512}}\qquad
		\subfloat[RMS image $\hat{s}(x,y)$ \label{fig:4}]{\includegraphics[width=4cm]{RMS_flat_shell_Vz_7_500x500bottom.png}}\qquad
		\subfloat[Ground truth (label) \label{fig:5}]{\includegraphics[width=4cm]{m1_rand_single_delam_7.png}}
	\end{figure}

The RMS is defined as:
%%%%%%%%%%%%%%%
\begin{equation*}
	\hat{s}(x,y) = \sqrt{\frac{1}{N}\sum_{k=1}^{N}s(x,y,t_k)^2} 
	\label{eqn:rms} 
\end{equation*}
\end{frame}
%%%%%%%%%%%%%%%%%%%%%%%%%%%%%%%%%%%%%%%%%%%%%%%%%%%%%%
\section{Deep learning-based approaches}
%%%%%%%%%%%%%%%%%%%%%%%%%%%%%%%%%%%%%%%%%%%%%%%%%%%%%%
\setcounter{subfigure}{0}
\begin{frame}{What is computer vision?}
	\begin{minipage}[c]{0.30\textwidth}
		Computer vision is a field of AI that enables computers and systems to derive meaningful information from digital images, videos and other visual inputs. 
	\end{minipage}
	\hfill
	\begin{minipage}[c]{0.65\textwidth}
		\begin{figure}
			\centering
			\includegraphics[width=1\textwidth]{computer_vision_tasks.png}
		\end{figure}
	\end{minipage}
\end{frame}
%%%%%%%%%%%%%%%%%%%%%%%%%%%%%%%%%%%%%%%%%%%%%%%%%%%%%%
\subsection{Semantic segmentation}
%%%%%%%%%%%%%%%%%%%%%%%%%%%%%%%%%%%%%%%%%%%%%%%%%%%%%%
\setcounter{subfigure}{0}
\begin{frame}{Semantic segmentation}
	\begin{minipage}[c]{0.47\textwidth}
		\centering
		\textbf{One-to-one \\image-based approach (RMS)} 
		\begin{figure}
		\captionsetup[subfigure]{labelformat=empty}
		\subfloat[Single input\\ (image)]{\includegraphics[width=.4\textwidth]{RMS_flat_shell_Vz_381_500x500bottom.png}}\qquad
		\subfloat[Single output]{\includegraphics[width=.4\textwidth]{GCN_381.png}}
		\end{figure}
	\end{minipage}
	\qquad
	\begin{minipage}[c]{0.47\textwidth}
		\centering
		\textbf{Many-to-one \\animation-based approach}
		\begin{figure}
			\captionsetup[subfigure]{labelformat=empty}
			\subfloat[Multiple frames \\animation]{\animategraphics[autoplay,loop,width=.4\textwidth]{4}{figures/gif_figs/381_output/flat_shell_Vz_381_500x500bottom-}{85}{113}}\qquad
			\subfloat[Single output]{\includegraphics[width=.4\textwidth]{GCN_381.png}}
		\end{figure}
	\end{minipage}	
\end{frame}
\setcounter{subfigure}{0}
%%%%%%%%%%%%%%%%%%%%%%%%%%%%%%%%%%%%%%%%%%%%%%%%%%%%%%
\subsection{Developed DL models}
%%%%%%%%%%%%%%%%%%%%%%%%%%%%%%%%%%%%%%%%%%%%%%%%%%%%%%
%\begin{frame}{Common deep learning architectures}
%	
%	\begin{minipage}[t]{0.45\textwidth}
%		\textbf{RMS based}\\
%		\begin{itemize}
%			\item Convolutional neural networks (CNN)
%			\item Fully convolutional network (FCN)
%		\end{itemize}
%	\end{minipage}
%	\hfill
%	\begin{minipage}[t]{0.45\textwidth}
%		\textbf{Full wavefield frames}\\
%		\begin{itemize}
%			\item Recurrent neural network (RNN)
%			\item Long short-term memory (LSTM)
%			\item ConvLSTM
%		\end{itemize}
%	\end{minipage}
%\end{frame}
%%%%%%%%%%%%%%%%%%%%%%%%%%%%%%%%%%%%%%%%%%%%%%%%%%
\begin{frame}{Developed models for delamination identification}
%%%%%%%%%%%%%%%%%%%%%%%%%%%%%%%%%%%%%%%%%%%%%%%%%%
	\begin{minipage}[t]{0.45\textwidth}
		\textbf{Image-based models:}
			\begin{itemize}
				\item Res-UNet
				\item VGG 16 encoder-decoder
				\item FCN-DenseNet
				\item PSPNet
				\item GCN
			\end{itemize}
		\vspace{5mm}
		\biblioref{A. Ijjeh, P. Kudela}{2022}{Deep learning based segmentation using full wavefield processing for delamination identification: A comparative study}{Mechanical Systems and Signal Processing, 168, 108671}
		\biblioref{A. Ijjeh, S. Ullah, P. Kudela}{2021}{Full wavefield processing by using FCN for delamination detection}{Mechanical Systems and Signal Processing, 153, 107537}
	\end{minipage}
	\hfill
	\begin{minipage}[t]{.45\textwidth}
	\textbf{Animation-based model:}
		\begin{itemize}
			\item Autoencoder ConvLSTM
		\end{itemize}
	\vspace{5mm}
	\biblioref{S. Ullah, A. Ijjeh, P. Kudela}{2023}{Deep learning approach for delamination identification using animation of Lamb waves}{Engineering Applications of Artificial Intelligence, 117, 105520}
	\end{minipage}
\end{frame}
\setcounter{subfigure}{0}
%%%%%%%%%%%%%%%%%%%%%%%%%%%%%%%%%%%%%%%%%%%%%%%%%%
\subsection{RMS based models}
%%%%%%%%%%%%%%%%%%%%%%%%%%%%%%%%%%%%%%%%%%%%%%%%%%
\begin{frame}{Residual UNet}
%%%%%%%%%%%%%%%%%%%%%%%%%%%%%%%%%%%%%%%%%%%%%%%%%%
	\begin{figure}
		\centering
		\includegraphics[width=.6\textwidth]{figure4.png}
	\end{figure}
\end{frame}

\begin{frame}{VGG16 encoder-decoder}
	\begin{figure}
		\centering
		\includegraphics[width=.6\textwidth]{figure5.png}
	\end{figure}
\end{frame}
%%%%%%%%%%%%%%%%%%%%%%%%%%%%%%%%%%%%%%%%%%%%%%%%%%%%%%%%%%%%%%%%%%%%%%%%%%%%%%%%
\begin{frame}{FCN-DenseNet}
%%%%%%%%%%%%%%%%%%%%%%%%%%%%%%%%%%%%%%%%%%%%%%%%%%%%%%%%%%%%%%%%%%%%%%%%%%%%%%%%
	\begin{minipage}[c]{0.48\textwidth}
		\begin{figure} [h!]
			\includegraphics[width=.7\textwidth]{FCN_dense_net.png}
			\caption{FCN-DenseNet architecture} 
			\label{fcn}
		\end{figure}
	\end{minipage}
	\hfill
	\begin{minipage}[c]{0.48\textwidth}
		\begin{figure} [h!]
			\centering
			\includegraphics[width=0.5\textwidth,angle=-90]{figure6.png}
			\caption{Dense block architecture} 
		\end{figure}
	\end{minipage}
\end{frame}
%%%%%%%%%%%%%%%%%%%%%%%%%%%%%%%%%%%%%%%%%%%%%%%%%%%%%%%%%%%%%%%%%%%%%%%%%%%%%%%%
\begin{frame}{Pyramid Scene Parsing Network (PSPNet)}
%%%%%%%%%%%%%%%%%%%%%%%%%%%%%%%%%%%%%%%%%%%%%%%%%%%%%%%%%%%%%%%%%%%%%%%%%%%%%%%%
	\begin{figure} [h!]
		\centering
		\includegraphics[width=.65\textwidth]{figure7.png}
		\caption{PSPNet architecture} 
	\end{figure}
\end{frame}
%%%%%%%%%%%%%%%%%%%%%%%%%%%%%%%%%%%%%%%%%%%%%%%%%%%%%%%%%%%%%%%%%%%%%%%%%%%%%%%%
\begin{frame}{Global Convolution Network (GCN)}
%%%%%%%%%%%%%%%%%%%%%%%%%%%%%%%%%%%%%%%%%%%%%%%%%%%%%%%%%%%%%%%%%%%%%%%%%%%%%%%%
		\begin{minipage}[c]{0.55\textwidth}
			\begin{figure} [h!]
				\centering
				\includegraphics[width=.9\textwidth]{figure8.png}
				\caption{GCN architecture} 
			\end{figure}	
		\end{minipage}
	\hfill
	\begin{minipage}[c]{0.4\textwidth}
		\begin{figure} [h!]
			\centering
			\includegraphics[width=.9\textwidth]{figure9.png}
			\caption{(a) Residual block, (b) GCN block, (c) Boundary Refinement} 
		\end{figure}	
	\end{minipage}
\end{frame}
%%%%%%%%%%%%%%%%%%%%%%%%%%%%%%%%%%%%%%%%%%%%%%%%%%%%%%%%%%%%%%%%%%%%%%%%%%%%%%%%
\begin{frame}{Evaluation metrics for delamination identification}
%%%%%%%%%%%%%%%%%%%%%%%%%%%%%%%%%%%%%%%%%%%%%%%%%%%%%%%%%%%%%%%%%%%%%%%%%%%%%%%%
	\begin{minipage}[c]{0.45\textwidth}
	\begin{itemize}
		\item Intersection over Union (IoU): %%%%%%%%%%%%%%%%%%%%%%%%%%%%%%%%%%%%%%%%%%%%%%%%%%%%%%%%%%%%%%%%%%%%%%%%
			\begin{equation*}
				\textup{IoU}=\frac{Intersection}{Union}=\frac{\hat{Y} \cap Y}{\hat{Y} \cup Y},
				\label{eqn:iou}
			\end{equation*}
			%%%%%%%%%%%%%%%%%%%%%%%%%%%%%%%%%%%%%%%%%%%%%%%%%%%%%%%%%%%%%%%%%%%%
			\item Percentage area error $\epsilon$:
			\begin{equation*}
				\epsilon=\frac{|A-\hat{A}|}{A} \times 100\%,
				\label{eqn:mean_size_error}
			\end{equation*}
			%%%%%%%%%%%%%%%%%%%%%%%%%%%%%%%%%%%%%%%%%%%%%%%%%%%%%%%%%%%%%%%%%%%%
	\end{itemize}
	\end{minipage}
	\begin{minipage}[c]{0.45\textwidth}
		\begin{figure}
			\subfloat{\includegraphics[width=1.0\textwidth]{IoU_figure.png}}
		\end{figure}
	\end{minipage}
\end{frame}
%%%%%%%%%%%%%%%%%%%%%%%%%%%%%%%%%%%%%%%%%%%%%%%%%%
%\setcounter{subfigure}{0}
%\begin{frame}{RMS based models}
%	\begin{minipage}[c]{0.55\textwidth}
%		\begin{figure}
%			\subfloat[Res-UNet model]{\includegraphics[width=1\textwidth]{figure4.png}}
%		\end{figure}
%	\end{minipage}
%	\begin{minipage}[c]{0.35\textwidth}
%		\begin{figure}
%			\subfloat[Data flow \& intermediate outputs of layers \label{fig:}]{\animategraphics[autoplay, controls,width=.8\textwidth]{4}{figures/gif_figs/381__inter_pred/intermediate_output-}{0}{103}}
%\end{figure}
%	\end{minipage}
%
%\end{frame}
%%%%%%%%%%%%%%%%%%%%%%%%%%%%%%%%%%%%%%%%%%%%%%%%%%
\subsection{Numerical test cases (GCN model)}
%%%%%%%%%%%%%%%%%%%%%%%%%%%%%%%%%%%%%%%%%%%%%%%%%%
\setcounter{subfigure}{0}
%%%%%%%%%%%%%%%%%%%%%%%%%%%%%%%%%%%%%%%%%%%%%%%%%%%%%%%%%%%%%%%%%%%%%%%%%%%%%%%%
\begin{frame}{Numerical test cases: image-based models (GCN)}
%%%%%%%%%%%%%%%%%%%%%%%%%%%%%%%%%%%%%%%%%%%%%%%%%%%%%%%%%%%%%%%%%%%%%%%%%%%%%%%%
	\begin{minipage}[c]{0.32\textwidth}
		\begin{figure}[c]
			\centering
			\animategraphics[controls,width=.9\textwidth]{2}{figures/gif_figs/397/intermediate_output-}{0}{82}
			\caption{\(1^{st}\) numerical case}
		\end{figure}
	\end{minipage}
	\hfill
	\begin{minipage}[c]{0.32\textwidth}
		\begin{figure}[c]
			\centering
			\animategraphics[controls,width=.9\textwidth]{2}{figures/gif_figs/438/intermediate_output-}{0}{82}
			\caption{\(2^{nd}\) numerical case}
		\end{figure}
	\end{minipage}
	\hfill
	\begin{minipage}[c]{0.32\textwidth}
		\begin{figure}[c]
			\centering
			\animategraphics[controls,width=.9\textwidth]{2}{figures/gif_figs/456/intermediate_output-}{0}{82}
			\caption{\(3^{rd}\) numerical case}
		\end{figure}
	\end{minipage}
\end{frame}
%%%%%%%%%%%%%%%%%%%%%%%%%%%%%%%%%%%%%%%%%%%%%%%%%%%%%%%%%%%%%%%%%%%%%%%%%%%%%%%%
\begin{frame}{Analysis of numerical results: image-based models}
%%%%%%%%%%%%%%%%%%%%%%%%%%%%%%%%%%%%%%%%%%%%%%%%%%%%%%%%%%%%%%%%%%%%%%%%%%%%%%%%
	\begin{table}[ht!]
		\centering
		\caption{Analysis of numerical cases}
		\label{tab:table_all_numerical_cases}	
		\begin{tabular}{lcc}
			\toprule
			Model & mean \(IoU\) & max \(IoU\) \\ 
			\midrule 
			Res-UNet & \(0.66\) & \(0.89\) \\ 
			VGG16 encoder-decoder & \(0.57\) & \(0.84\) \\ 
			FCN-DenseNet & \(0.68\) & \(0.92\) \\ 
			PSPNet & \(0.55\) & \(0.91\) \\ 
			GCN & \(0.76\) & \(0.93\) \\ 
			\bottomrule
		\end{tabular}
	\end{table}
\end{frame}
%%%%%%%%%%%%%%%%%%%%%%%%%%%%%%%%%%%%%%%%%%%%%%%%%%
\setcounter{subfigure}{0}
%%%%%%%%%%%%%%%%%%%%%%%%%%%%%%%%%%%%%%%%%%%%%%%%%%%%%%%%%%%%%%%%%%%%%%%%%%%%%%%%
\subsection{Full wavefield frames based model}
%%%%%%%%%%%%%%%%%%%%%%%%%%%%%%%%%%%%%%%%%%%%%%%%%%%%%%%%%%%%%%%%%%%%%%%%%%%%%%%%
\begin{frame}{Autoencoder ConvLSTM}
	\begin{columns}[T]		
		\begin{column}{0.5\textwidth}
			\only<1>{
			\begin{figure}[c]
				\centering
				\subfloat{\includegraphics[width=.8\textwidth]{figure2.png}}
				\caption{Sample frames of full wavefield propagation}
			\end{figure}}
			\only<2->{
			\begin{figure}[c]
					\centering
					\subfloat{\includegraphics[height=.6\textheight]{figure3.png}}
					\caption{The procedure of calculating the RMS prediction image (damage map)}
				\end{figure}}
		\end{column}
		\begin{column}{0.45\textwidth}
			\begin{figure}
				\centering
				\subfloat{\includegraphics[width=.8\textheight]{figure5b.png}}
				\caption{Autoencoder ConvLSTM model}
			\end{figure}
		\end{column}
	\end{columns}
\end{frame}
\setcounter{subfigure}{0}
%%%%%%%%%%%%%%%%%%%%%%%%%%%%%%%%%%%%%%%%%%%%%%%%%%%%%%%%%%%%%%%%%%%%%%%%%%%%%%%%
\begin{frame}{Numerical test cases: animation-based model}
%%%%%%%%%%%%%%%%%%%%%%%%%%%%%%%%%%%%%%%%%%%%%%%%%%%%%%%%%%%%%%%%%%%%%%%%%%%%%%%%
		\begin{figure}
			\subfloat[\(1^{st}\) case]{\animategraphics[autoplay,loop,width=.2\textwidth]{6}{figures/gif_figs/397_convLSTM/397_convLSTM-}{0}{23}}
			\quad
			\subfloat[\(2^{nd}\) case]{\animategraphics[autoplay,loop,width=.2\textwidth]{6}{figures/gif_figs/438_convLSTM/438_convLSTM-}{0}{16}}
			\quad
			\subfloat[\(3^{rd}\) case]{\animategraphics[autoplay,loop,width=.2\textwidth]{6}{figures/gif_figs/456_convLSTM/456_convLSTM-}{0}{23}}
			\\
			\subfloat[Prediction \(1^{st}\)]{\includegraphics[width=.2\textwidth]{figures/predicted_397.png}}
			\quad
			\subfloat[Prediction \(2^{nd}\)]{\includegraphics[width=.2\textwidth]{figures/predicted_438.png}}
			\quad
			\subfloat[Prediction \(3^{rd}\)]{\includegraphics[width=.2\textwidth]{figures/predicted_456.png}}
		\end{figure}	
\end{frame}
%%%%%%%%%%%%%%%%%%%%%%%%%%%%%%%%%%%%%%%%%%%%%%%%%%%%%%%%%%%%%%%%%%%%%%%%%%%%%%%%
\subsection{Experimental case}
%%%%%%%%%%%%%%%%%%%%%%%%%%%%%%%%%%%%%%%%%%%%%%%%%%%%%%%%%%%%%%%%%%%%%%%%%%%%%%%%
\setcounter{subfigure}{0}
%%%%%%%%%%%%%%%%%%%%%%%%%%%%%%%%%%%%%%%%%%%%%%%%%%%%%%%%%%%%%%%%%%%%%%%%%%%%%%%%
\begin{frame}{Experimental results: image-based models}
%%%%%%%%%%%%%%%%%%%%%%%%%%%%%%%%%%%%%%%%%%%%%%%%%%%%%%%%%%%%%%%%%%%%%%%%%%%%%%%%
	\centering
	\begin{figure}
		\subfloat[ERMS \& label]{\includegraphics[width=.18\textwidth]{ERMS_with_label.png}}\qquad
		\subfloat[ERMSF]{\includegraphics[width=.18\textwidth]{ERMSF_CFRP_teflon_3o_375_375p_50kHz_5HC_x12_15Vpp.png}}\qquad
		\subfloat[Binary ERMSF:\\ \hspace{5mm}IoU=$0.401$]{\includegraphics[width=.18\textwidth]{Binary_ERMSF_CFRP_teflon_3o_375_375p_50kHz_5HC_x12_15Vpp.png}}\qquad
		\\
		\subfloat[GCN: \\ \hspace{5mm}IoU\(=0.723\)]{\includegraphics[width=.18\textwidth]{Fig_GCN_7.png}}\qquad
		\subfloat[FCN-DenseNet:\\ \hspace{5mm}IoU=$0.54$]{\includegraphics[width=.18\textwidth]{Fig_FCN_DenseNet_7.png}}\qquad
	\end{figure}
\end{frame}
%%%%%%%%%%%%%%%%%%%%%%%%%%%%%%%%%%%%%%%%%%%%%%%%%%%%%%%%%%%%%%%%%%%%%%%%%%%%
\begin{frame}{Analysis of experimental results: image-based models}
%%%%%%%%%%%%%%%%%%%%%%%%%%%%%%%%%%%%%%%%%%%%%%%%%%%%%%%%%%%%%%%%%%%%%%%%%%%%
	\begin{table}[!ht]
		\centering
		\caption{Evaluation metrics of the experimental case}
		\label{tab:rms_exp_case}
		\begin{tabular}{l@{\ }cccc}
			\toprule
			\multicolumn{1}{l}{Model} & \multicolumn{1}{c}{A [mm\textsuperscript{2}]} & \multicolumn{3}{c}{Predicted output} \\ 
			\cmidrule(lr){3-5} & & \multicolumn{1}{c}{IoU} & \multicolumn{1}{c}{\(\hat{A}\) [mm\textsuperscript{2}]} & \(\epsilon\) \\ \midrule
			Res-UNet & \multicolumn{1}{c}{210} & \multicolumn{1}{c}{0.58} & \multicolumn{1}{c}{323}  & \(53.8\%\) \\ 
			VGG16 encoder-decoder &  & \multicolumn{1}{c}{0.62} & \multicolumn{1}{c}{320} & \(52.4\%\) 
			\\ 
			FCN-DenseNet &  & \multicolumn{1}{c}{0.54} & \multicolumn{1}{c}{386} & \(83.8\%\) \\ 
			PSPNet &  & \multicolumn{1}{c}{0.49} & \multicolumn{1}{c}{580} & \(176.2\%\) 
			\\ 
			GCN &  & \multicolumn{1}{c}{0.72} & \multicolumn{1}{c}{309} & \(47.1\%\) 
			\\ 
			\bottomrule
		\end{tabular}		
	\end{table}
\end{frame}
\setcounter{subfigure}{0}
%%%%%%%%%%%%%%%%%%%%%%%%%%%%%%%%%%%%%%%%%%%%%%%%%%%%%%%%%%%%%%%%%%%%%%%%%%%%
\begin{frame}{Experimental results: animation-based model}
%%%%%%%%%%%%%%%%%%%%%%%%%%%%%%%%%%%%%%%%%%%%%%%%%%%%%%%%%%%%%%%%%%%%%%%%%%%%
		\centering
		\begin{columns}[T]		
			\begin{column}{0.32\textwidth}
				\begin{figure}
					\subfloat{\animategraphics[autoplay,loop,width=.75\textwidth]{12}{figures/gif_figs/exp/exp-}{0}{447}} 
					\caption{Intermediate predictions}
				\end{figure}	
			\end{column}
		\begin{column}{0.65\textwidth}
			\begin{figure}
				\subfloat{\includegraphics[width=.9\textwidth]{exp_rms_thresholded.png}}
				\caption{Damage map (RMS) and thresholded damage map}	
			\end{figure}
		
		\end{column}
	\end{columns}
	
IoU= $0.23$ and $\epsilon=10.16\%$ for thresholded damage map.
\end{frame}
%%%%%%%%%%%%%%%%%%%%%%%%%%%%%%%%%%%%%%%%%%%%%%%%%%%%%%%%%%%%%%%%%%%%%%%%%%%%
\section{Conclusions}
%%%%%%%%%%%%%%%%%%%%%%%%%%%%%%%%%%%%%%%%%%%%%%%%%%%%%%%%%%%%%%%%%%%%%%%%%%%%
\begin{frame}{Conclusions}
%%%%%%%%%%%%%%%%%%%%%%%%%%%%%%%%%%%%%%%%%%%%%%%%%%%%%%%%%%%%%%%%%%%%%%%%%%%%
	\begin{itemize}
		\item Full wavefields contain rich damage-related information
		\item Full wavefields can be utilised to train deep learning models to perform damage identification in an end-to-end approach
		\item Deep learning models trained on synthetic dataset generalise well and can be applied directly to experimental wavefields
		\item Animation-based deep learning models perform better than image-based models but are more complex and require longer time for training
		\item Deep learning approaches surpass the conventional signal processing techniques
	\end{itemize}
\end{frame}
%%%%%%%%%%%%%%%%%%%%%%%%%%%%%%%%%%%%%%%%%%%%%%%%%%%%%%%%%%%%%%%%%%%%%%%%%%%%
{\setbeamercolor{palette primary}{fg=blue, bg=white}
\begin{frame}[standout]
	Thank you for your listening!\\ \vspace{12pt}
	Questions?\\ \vspace{12pt}
	\url{pk@imp.gda.pl} 
	\par\medskip
	\url{aijjeh@imp.gda.pl}
	
 
\par\medskip
\par\medskip
\footnotesize
The research work was funded by the Polish National Science Center under grant agreement no. 2018/31/B/ST8/00454.
\end{frame}
}
%%%%%%%%%%%%%%%%%%%%%%%%%%%%%%%%%%%%%%%%%%%%%%%%%%
% END OF SLIDES
%%%%%%%%%%%%%%%%%%%%%%%%%%%%%%%%%%%%%%%%%%%%%%%%%%
\end{document}