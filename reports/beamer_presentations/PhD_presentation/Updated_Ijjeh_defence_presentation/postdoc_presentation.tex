%\PassOptionsToPackage{draft}{graphicx}
\documentclass[10pt,aspectratio=169,dvipsnames]{beamer} % , notes=only aspect ratio 16:9
%\graphicspath{{../../figures/}}

%\includeonlyframes{frame1,frame2,frame3}

%%%%%%%%%%%%%%%%%%%%%%%%%%%%%%%%%%%%%%%%%%%%%%%%%%
% Packages
%%%%%%%%%%%%%%%%%%%%%%%%%%%%%%%%%%%%%%%%%%%%%%%%%%
\usepackage{appendixnumberbeamer}
\usepackage{booktabs}
\usepackage{csvsimple} % for csv read
\usepackage[scale=2]{ccicons}
\usepackage{pgfplots}
\usepackage{xspace}
%\usepackage{amsmath}
\usepackage{totcount}
\usepackage{tikz}
\usepackage{bm}
\usepackage{float}
\usepackage{eso-pic} 
\usepackage{wrapfig}
\usepackage{animate,media9}
\usepackage{subfig}
\usepackage{fancybox}
%\usepackage{multimedia}
\usepackage{dashbox}
\usepackage{tcolorbox}
\usepackage{multicol}
\usepackage{multirow}
\usepackage{xcolor}
\usepackage[document]{ragged2e}
\usepackage[labelformat=empty]{caption}
\usepackage{comment}
\usepackage{mathtools}% Loads amsmath
\usepackage{efbox,graphicx}
\usepackage{pgfpages}
%\setbeameroption{show notes}
\captionsetup[figure]{labelformat=empty}%
\usefonttheme{structurebold}
\efboxsetup{linecolor=Lightskyblue,linewidth=2pt}
%%%%%%%%%%%%%%%%%%%%%%%%%%%%%%%%%%%%%%%%%%%%%%%%%%%%%%%%%%%%%%%%%%%%%%%%%%%%%%%%
\makeatletter
\renewcommand\tcbtitle{\ifx\tcbtitletext\@empty\else%
	{\kvtcb@fonttitle\kvtcb@haligntitle\kvtcb@before@title
		\leavevmode\color{tcbcol@title}\tcbtitletext\kvtcb@after@title}\fi}
\makeatother
%%%%%%%%%%%%%%%%%%%%%%%%%%%%%%%%%%%%%%%%%%%%%%%%%%%%%%%%%%%%%%%%%%%%%%%%%%%%%%%%
%\usepackage[font=footnotesize,labelfont=bf]{caption}
%\captionsetup[figure]{font=small}

%\usepackage[export]{adjustbox}
%\usepackage{background}
%\backgroundsetup{contents=preliminary,placement=bottom,color=blue}
%\usepackage{FiraSans}

%\usepackage{comment}
%\usetikzlibrary{external} % speedup compilation
%\tikzexternalize % activate!
%\usetikzlibrary{shapes,arrows} 

%\usepackage{bibentry}
%\nobibliography*
\usepackage{ifthen}
\newcounter{angle}
\setcounter{angle}{0}
%\usepackage{bibentry}
%\nobibliography*
\usepackage{caption}%


%\usepackage{etoolbox}
%\apptocmd{\frame}{}{\justifying}{} 

\graphicspath{{figures/}}

%%%%%%%%%%%%%%%%%%%%%%%%%%%%%%%%%%%%%%%%%%%%%%%%%%
% Metropolis theme custom modification file
%%%%%%%%%%%%%%%%%%%%%%%%%%%%%%%%%%%%%%%%%%%%%%%%%%
% Metropolis theme custom modification file
%%%%%%%%%%%%%%%%%%%%%%%%%%%%%%%%%%%%%%%%%%%%%%%%%%
% Metropolis theme custom colors
%%%%%%%%%%%%%%%%%%%%%%%%%%%%%%%%%%%%%%%%%%%%%%%%%%
\usetheme[progressbar=foot]{metropolis}
\useoutertheme{metropolis}
\useinnertheme{metropolis}
\usefonttheme{metropolis}
\setbeamercolor{background canvas}{bg=white}

%\usecolortheme{spruce}

\definecolor{myblue}{rgb}{0.19,0.55,0.91}
\definecolor{mediumblue}{rgb}{0,0,205}
\definecolor{darkblue}{rgb}{0,0,139}
\definecolor{Dodgerblue}{HTML}{1E90FF}
\definecolor{Navy}{HTML}{000080} % {rgb}{0,0,128}
\definecolor{Aliceblue}{HTML}{F0F8FF}
\definecolor{Lightskyblue}{HTML}{87CEFA}
\definecolor{logoblue}{RGB}{1,67,140}
\definecolor{Purple}{HTML}{911146}
\definecolor{Orange}{HTML}{CF4A30}

\setbeamercolor{progress bar}{bg=Lightskyblue}
\setbeamercolor{progress bar}{ fg=logoblue} 
\setbeamercolor{frametitle}{bg=logoblue}
\setbeamercolor{title separator}{fg=logoblue}
\setbeamercolor{block title}{bg=Lightskyblue!30,fg=black}
\setbeamercolor{block body}{bg=Lightskyblue!15,fg=black}
\setbeamercolor{alerted text}{fg=Purple}
% notes colors
\setbeamercolor{note page}{bg=white}
\setbeamercolor{note title}{bg=Lightskyblue}
%%%%%%%%%%%%%%%%%%%%%%%%%%%%%%%%%%%%%%%%%%%%%%%%%%
%  Theme modifications
%%%%%%%%%%%%%%%%%%%%%%%%%%%%%%%%%%%%%%%%%%%%%%%%%%
% modify progress bar linewidth
\makeatletter
\setlength{\metropolis@progressinheadfoot@linewidth}{2pt} 
\setlength{\metropolis@titleseparator@linewidth}{1pt}
\setlength{\metropolis@progressonsectionpage@linewidth}{1pt}

\setbeamertemplate{progress bar in section page}{
	\setlength{\metropolis@progressonsectionpage}{%
		\textwidth * \ratio{\thesection pt}{\totvalue{totalsection} pt}%
	}%
	\begin{tikzpicture}
		\fill[bg] (0,0) rectangle (\textwidth, 
		\metropolis@progressonsectionpage@linewidth);
		\fill[fg] (0,0) rectangle (\metropolis@progressonsectionpage, 
		\metropolis@progressonsectionpage@linewidth);
	\end{tikzpicture}%
}
\makeatother
\newcounter{totalsection}
\regtotcounter{totalsection}

\AtBeginDocument{%
	\pretocmd{\section}{\refstepcounter{totalsection}}{\typeout{Yes, prepending 
	was successful}}{\typeout{No, prepending was not successful}}%
}%
%%%%%%%%%%%%%%%%%%%%%%%%%%%%%%%%%%%%%%%%%%%%%%%%%%
%  Bibliography mods
%%%%%%%%%%%%%%%%%%%%%%%%%%%%%%%%%%%%%%%%%%%%%%%%%%
\setbeamertemplate{bibliography item}{\insertbiblabel} %% Remove book symbol 
%%from references and add number in square brackets
% kill the abominable icon (without number)
%\setbeamertemplate{bibliography item}{}
%\makeatletter
%\renewcommand\@biblabel[1]{#1.} % number only
%\makeatother
% remove line breaks in bibliography
\setbeamertemplate{bibliography entry title}{}
\setbeamertemplate{bibliography entry location}{}
%%%%%%%%%%%%%%%%%%%%%%%%%%%%%%%%%%%%%%%%%%%%%%%%%%
%  Bibliography custom commands
%%%%%%%%%%%%%%%%%%%%%%%%%%%%%%%%%%%%%%%%%%%%%%%%%%
\newcommand{\bibliotitlestyle}[1]{\textbf{#1}\par}

\newif\ifinbiblio
\newcounter{bibkey}
\newenvironment{biblio}[2][long]{%
	%\setbeamertemplate{bibliography item}{\insertbiblabel}
	\setbeamertemplate{bibliography item}{}% without numbers
	\setbeamerfont{bibliography item}{size=\footnotesize}
	\setbeamerfont{bibliography entry author}{size=\footnotesize}
	\setbeamerfont{bibliography entry title}{size=\footnotesize}
	\setbeamerfont{bibliography entry location}{size=\footnotesize}
	\setbeamerfont{bibliography entry note}{size=\footnotesize}
	\ifx!#2!\else%
	\bibliotitlestyle{#2}%
	\fi%
	\begin{thebibliography}{}%
		\inbibliotrue%
		\setbeamertemplate{bibliography entry title}[#1]%
	}{%
		\inbibliofalse%
		\setbeamertemplate{bibliography item}{}%
	\end{thebibliography}%
}

\newcommand{\biblioref}[5][short]{
	\setbeamertemplate{bibliography entry title}[#1]
	\stepcounter{bibkey}%
	\ifinbiblio%
	\bibitem{\thebibkey}%
	#2
	\newblock #4
	\ifx!#5!\else\newblock {\em #5}, #3 \fi%
	\else%
	\begin{biblio}{}
		\bibitem{\thebibkey}
		#2
		\newblock #4
		\ifx!#5!\else\newblock {\em #5}, #3\fi
	\end{biblio}
	\fi
}
%
%\newbibmacro*{hypercite}{%
%	\renewcommand{\@makefntext}[1]{\noindent\normalfont##1}%
%	\footnotetext{%
%		\blxmkbibnote{foot}{%
%			\printtext[labelnumberwidth]{%
%				\printfield{prefixnumber}%
%				\printfield{labelnumber}}%
%			\addspace
%			\fullcite{\thefield{entrykey}}}}}
%
%\DeclareCiteCommand{\hypercite}%
%{\usebibmacro{cite:init}}
%{\usebibmacro{hypercite}}
%{}
%{\usebibmacro{cite:dump}}
%
%% Redefine the \footfullcite command to use the reference number
%\renewcommand{\footfullcite}[1]{\cite{#1}\hypercite{#1}}
%\usefonttheme[onlymath]{Serif} % It should be uncommented if Fira fonts in 
%%math does not work

%%%%%%%%%%%%%%%%%%%%%%%%%%%%%%%%%%%%%%%%%%%%%%%%%%
% Custom commands
%%%%%%%%%%%%%%%%%%%%%%%%%%%%%%%%%%%%%%%%%%%%%%%%%%
% matrix command 
\newcommand{\matr}[1]{\mathbf{#1}} % bold upright (Elsevier, Springer)
%\newcommand{\matr}[1]{#1} % pure math version
%\newcommand{\matr}[1]{\bm{#1}} % ISO complying version
% vector command 
\newcommand{\vect}[1]{\mathbf{#1}} % bold upright (Elsevier, Springer)
% bold symbol
\newcommand{\bs}[1]{\boldsymbol{#1}}
% derivative upright command
\DeclareRobustCommand*{\drv}{\mathop{}\!\mathrm{d}}
\newcommand{\ud}{\mathrm{d}}
% 
\newcommand{\themename}{\textbf{\textsc{metropolis}}\xspace}

\def\checkmark{\tikz\fill[scale=0.4](0,.35) -- (.25,0) -- (1,.7) -- (.25,.15) -- cycle;} 

%
%\setbeameroption{show notes on second screen}
%\setbeamertemplate{note page}{\insertnote}
%%%%%%%%%%%%%%%%%%%%%%%%%%%%%%%%%%%%%%%%%%%%%%%%%%
% Title page options
%%%%%%%%%%%%%%%%%%%%%%%%%%%%%%%%%%%%%%%%%%%%%%%%%%
\date{February 13, 2023}
%\date{}
%%%%%%%%%%%%%%%%%%%%%%%%%%%%%%%%%%%%%%%%%%%%%%%%%%
% option 1
%%%%%%%%%%%%%%%%%%%%%%%%%%%%%%%%%%%%%%%%%%%%%%%%%%%
\justifying\title{Postdoctoral position}
%\subtitle{In preparation for a Ph.D. defence}
\author{\textbf{Ph.D. Eng. Abdalraheem A. Ijjeh }} 
% logo align to Institute 
\institute{Institute of Fluid Flow Machinery \\ 
	Polish Academy of Sciences \\ 
	Gdansk, Poland \\
	\vspace{-1.5cm}
	\flushright 
	\includegraphics[width=6cm]{imp_logo.png}}

%%%%%%%%%%%%%%%%%%%%%%%%%%%%%%%%%%%%%%%%%%%%%%%%%%
%\tikzexternalize % activate!
%%%%%%%%%%%%%%%%%%%%%%%%%%%%%%%%%%%%%%%%%%%%%%%%%%
\setbeamertemplate{section in toc}[sections numbered]
\setbeamertemplate{subsection in toc}[subsections numbered]

\begin{document}
	\maketitle
	%%%%%%%%%%%%%%%%%%%%%%%%%%%%%%%%%%%%%%%%%%%%%%%%%%%%%%%%%%%%%%%%%%%%
	% SLIDES		
	\begin{frame}[label=frame1]{Outlines}
		\setbeamertemplate{section in toc}[sections numbered]
		\setbeamertemplate{subsection in toc}[subsections numbered]
		\tableofcontents
	\end{frame}	
	%%%%%%%%%%%%%%%%%%%%%%%%%%%%%%%%%%%%%%%%%%%%%%%%%%%%%%%%%%%%%%%%%%%%
	\section{About me!}
	\begin{frame}{Introducing myself}
		\begin{columns}[T]
			\footnotesize
			\justifying
			\begin{column}[t]{.6\textwidth}
				\footnotesize				
				\begin{tcolorbox}
					{\textbf{Abdalraheem Ijjeh} was born in Irbid, Jordan in 1988.}
				\end{tcolorbox}
				\begin{tcolorbox}
					\textbf{Education}
					\begin{itemize}
						\item 12/2019 - 02/2023	\textbf{Ph.D. in mechanical engineering}, Institute of Fluid-Flow Machinery, Polish Academy of Sciences, Gdansk, Poland.				
						\item 09/2010 - 01/2014 \textbf{M.Sc. in Computer engineering}, Jordan University Of Science \& Technology, Jordan.				
						\item {09/2005 - 06/2010} \textbf{B.Sc. in Computer engineering}, Yarmouk University, Jordan.
						\item {09/2004 - 06/2005} \textbf{High school}, Model School of Yarmouk University, Jordan.
					\end{itemize}
				\end{tcolorbox}
			\end{column}
			\begin{column}[t]{0.39\textwidth}
				\begin{tcolorbox}
					\textbf{Skills and expertise}
					\begin{itemize}								
						\item Data science, big data, 
						\item Computer vision
						\item Parallel computing, CUDA
						\item Python
						\item TensorFlow, Keras
						\item Signal processing \& Image Signal Processing							
						\item{Linux} 
						\item{Latex} 
					\end{itemize}
				\end{tcolorbox}		
			\end{column}
		\end{columns}
	\end{frame}
	\section{My Ph.D. work in brief}
	%%%%%%%%%%%%%%%%%%%%%%%%%%%%%%%%%%%%%%%%%%%%%%%%%%%%%%%%%%%%%%%%%%%%
	\begin{frame}{My work in short!}
		\begin{columns}[T]
			\begin{column}[c]{0.9\textwidth}
				\begin{figure}
					\centering
					\includegraphics[height=.85\textheight]{full_procedure.png}	
				\end{figure}		
			\end{column}
		\end{columns}		
	\end{frame}
	\setcounter{subfigure}{0}	
	%%%%%%%%%%%%%%%%%%%%%%%%%%%%%%%%%%%%%%%%%%%%%%%%%%%%%%%%%
	\section{Part I: Delamination identification}
	\begin{frame}{Delamination identification approaches}
		\begin{columns}[T]
			\begin{column}[c]{0.47\textwidth}
				\centering
				\textbf{One-to-one \\image-based approach (RMS)} 
				\begin{figure}
					\centering
					\captionsetup{justification=centering}				
					\subfloat[Single input (image)]{\includegraphics[width=.45\textwidth]{RMS_flat_shell_Vz_381_500x500bottom.png}}\quad
					\subfloat[Single output]{\includegraphics[width=.45\textwidth]{GCN_381.png}}
				\end{figure}
			\end{column}
			\hfill
			\begin{column}[c]{0.47\textwidth}
				\centering
				\textbf{Many-to-one \\animation-based approach}
				\begin{figure}
					\centering
					\captionsetup{justification=centering}					
					\subfloat[Multiple frames (animation)]{\animategraphics[autoplay,loop,width=.45\textwidth]{16}{figures/gif_figs/381_output/output_381-}{1}{512}}\quad
					\subfloat[Single output]{\includegraphics[width=.45\textwidth]{GCN_381.png}}
				\end{figure}
			\end{column}	
		\end{columns}		
	\end{frame}	
	%%%%%%%%%%%%%%%%%%%%%%%%%%%%%%%%%%%%%%%%%%%%%%%%%%%%%%%%%%%%%%%%
	\begin{frame}{Numerical test cases RMS based models (GCN model)}
		\begin{columns}[T]
			\begin{column}[c]{1.0\textwidth}
				\begin{figure}[c]
					\centering
					\captionsetup{justification=centering}					
					\animategraphics[controls,height=.75\textheight]{8}{figures/gif_figs/456/intermediate_output-}{0}{82}
					\caption{IoU=0.71}
				\end{figure}
			\end{column}
		\end{columns}
	\end{frame}
	%%%%%%%%%%%%%%%%%%%%%%%%%%%%%%%%%%%%%%%%%%%%%%%%%%%%%%%%%%%%%%%%%%%%
	\begin{frame}{Numerical test case animation of Lamb waves}
		\setcounter{subfigure}{0}
		\only<1>{
			\begin{alertblock}{Test case}
				\begin{figure}
					\centering
					\captionsetup{justification=centering}
					\subfloat[Full wavefield (512 frames)]{\animategraphics[autoplay,loop,height=3cm,keepaspectratio]{32}{figures/gif_figs/381_output/output_381-}{1}{512}}\quad
					\subfloat[Intermediate outputs]{\animategraphics[autoplay,loop,height=3cm,keepaspectratio]{31}{figures/gif_figs/Numerical_case_381/num_case_381_frame_num-}{0}{487}}\quad
					\subfloat[RMS (damage map)]{\includegraphics[height=3.05cm,keepaspectratio]{figures/RMS_Ijjeh_num_case_381.png}}\quad
					\subfloat[Binary RMS, IoU= 0.88]{\includegraphics[height=3cm,keepaspectratio]{figures/Binary_RMS_Ijjeh_num_case381_.png}}\quad
				\end{figure}
		\end{alertblock}}
		\setcounter{subfigure}{0}
%		\only<2>{
%			\begin{alertblock}{Second test case}
%				\begin{figure}
%					\centering
%					\captionsetup{justification=centering}
%					\subfloat[Full wavefield (512 frames)]{\animategraphics[autoplay,loop,height=3cm,keepaspectratio]{32}{figures/gif_figs/385_output/output_385-}{1}{512}}\quad		
%					\subfloat[Intermediate outputs]{\animategraphics[autoplay,loop,height=3cm,keepaspectratio]{31}{figures/gif_figs/Numerical_case_385/num_case_385_frame_num-}{0}{487}}\quad			
%					\subfloat[RMS (damage map)]{\includegraphics[height=3.05cm,keepaspectratio]{figures/RMS_Ijjeh_num_case_385.png}}\quad
%					\subfloat[Binary RMS, IoU= 0.58]{\includegraphics[height=3cm,keepaspectratio]{figures/Binary_RMS_Ijjeh_num_case385_.png}}
%				\end{figure}
%		\end{alertblock}}
%		\setcounter{subfigure}{0}
%		\only<3>{
%			\begin{alertblock}{Third test case}
%				\begin{figure}
%					\centering
%					\captionsetup{justification=centering}
%					\subfloat[Full wavefield (512 frames)]{\animategraphics[autoplay,loop,height=3cm,keepaspectratio]{32}{figures/gif_figs/394_output/output_394-}{1}{512}}\quad
%					\subfloat[Intermediate outputs]{\animategraphics[autoplay,loop,height=3cm,keepaspectratio]{31}{figures/gif_figs/Numerical_case_394/num_case_394_frame_num-}{0}{487}}\quad
%					\subfloat[RMS (damage map)]{\includegraphics[height=3.05cm,keepaspectratio]{figures/RMS_Ijjeh_num_case_394.png}}\quad
%					\subfloat[Binary RMS, IoU= 0.8]{\includegraphics[height=3cm,keepaspectratio]{figures/Binary_RMS_Ijjeh_num_case394_.png}}
%				\end{figure}
%		\end{alertblock}}
	\end{frame}
	%%%%%%%%%%%%%%%%%%%%%%%%%%%%%%%%%%%%%%%%%%%%%%%%%%%%%%%%%%%%%%%%%%%%
	\setcounter{subfigure}{0}		%%%%%%%%%%%%%%%%%%%%%%%%%%%%%%%%%%%%%%%%%%%%%%%%%%%%%%%%%%%%%%%%%%%%
	\begin{frame}{Experimental results: RMS image-based (Single delamination)}
		\begin{columns}[T]
			\begin{column}[t]{.25\textwidth}
				\begin{figure}[ht!]
					\centering
					\captionsetup{justification=centering}
					\includegraphics[height=.35\textheight]{ERMS_with_label.png}
					\caption{ERMS \& label}
				\end{figure}
				\justifying
				\tiny
				Kudela, P., Radzienski, M. and Ostachowicz, W., 2018. \textbf{Impact induced damage assessment by means of Lamb wave image processing}. \textit{Mechanical Systems and Signal Processing}, 102, pp.23-36.
			\end{column}
			\begin{column}[t]{0.5\textwidth}
				\begin{block}{Adaptive wavenumber filtering}
					\centering
					\footnotesize
					IoU=$0.401$
					\begin{figure}[ht!]
						\centering
						\captionsetup{justification=centering}
						\subfloat{\includegraphics[height=.35\textheight]{ERMSF_CFRP_teflon_3o_375_375p_50kHz_5HC_x12_15Vpp.png}}
						\quad
						\subfloat{\includegraphics[height=.35\textheight]{Binary_ERMSF_CFRP_teflon_3o_375_375p_50kHz_5HC_x12_15Vpp.png}}
					\end{figure}
				\end{block}					
			\end{column}		
			\begin{column}[t]{0.25\textwidth}
				\begin{alertblock}{DL approach: GCN}
					\centering
					\footnotesize
					IoU\(=0.723\)
					\begin{figure}[ht!]	
						\centering				
						\subfloat{\includegraphics[height=.35\textheight]{Fig_GCN_7.png}}
					\end{figure} 	
				\end{alertblock}				
			\end{column}
		\end{columns}	
	\end{frame}
	%%%%%%%%%%%%%%%%%%%%%%%%%%%%%%%%%%%%%%%%%%%%%%%%%%%%%%%%%%%%%%%%%%%%
	%%%%%%%%%%%%%%%%%%%%%%%%%%%%%%%%%%%%%%%%%%%%%%%%%%%%%%%%%%%%%%%%%%%%
	\setcounter{subfigure}{0}
	\begin{frame}{Experimental results: Full wavefield based (Single delamination)}		
		\begin{alertblock}{DL approach}
			IoU= $0.41$% and $\epsilon=71.56\%$ 
			\begin{figure}[ht!]
				\centering
				\subfloat[Input]{\animategraphics[autoplay,loop,height=3cm]{16}{figures/gif_figs/CFRP_teflon_3o_375_375p_50kHz_5HC_x12_15Vpp/CFRP_teflon_30-}{1}{256}}\quad
				\subfloat[Intermidate ouputs]{\animategraphics[autoplay,loop,height=3cm]{15}{figures/gif_figs/CFRP_ijjeh_single_delamination/intermediate_output-}{0}{231}}\quad
				\subfloat[RMS]{\includegraphics[height=3cm,keepaspectratio]{figures/RMS_CFRP_teflon_3o_375_375p_50kHz_5HC_x12_15Vpp_Ijjeh_updated_results_.png}}\quad
				\subfloat[Binary RMS]{\includegraphics[height=3cm,keepaspectratio]{figures/Binary_RMS_CFRP_teflon_3o__375_375p_50kHz_5HC_x12_15Vpp_Ijjeh_.png}}
			\end{figure}			
		\end{alertblock}	
	\end{frame}
	%%%%%%%%%%%%%%%%%%%%%%%%%%%%%%%%%%%%%%%%%%%%%%%%%%%%%%%%%%%%%%%%%%%%
	\note{
		In this slide, I present the predicted results using the autoencoder ConvLSTM model regarding the single delamination case.
		
		Animation (a) shows the full wavefield measured by SLDV with 256 frames.
		
		Animation (b) shows the intermediate predictions of the model. 
		
		Figure (c) shows the RMS of all intermediate predictions.
		
		And finally, Figure (d) shows the binary RMS with IoU= 0.41		
	}
	%%%%%%%%%%%%%%%%%%%%%%%%%%%%%%%%%%%%%%%%%%%%%%%%%%%%%%%%%%%%%%%%%%%%
	\setcounter{subfigure}{0}	
	\begin{frame}{Experimental results: Full wavefield based (Multiple delaminations)}
		\begin{columns}[T]
			\begin{column}[t]{0.20\textwidth}
				\begin{block}{Input}
					\footnotesize Full wavefield					
					\begin{figure}[ht!]	
						\centering						
						\subfloat{\animategraphics[autoplay,loop,height=0.32\textheight]{32}{figures/gif_figs/input_specimen_3/specimen_3-}{1}{512}}
					\end{figure}
				\end{block}				
			\end{column}
			\begin{column}[t]{0.40\textwidth}				
				\begin{block}{Adaptive wavenumber filtering}
					\footnotesize IoU$=0.04$					
					\begin{figure}[ht!]	
						\centering
						\subfloat{\includegraphics[height=0.32\textheight]{figures/mul/figure17a.png}}
						\quad
						\centering
						\subfloat{\includegraphics[height=0.32\textheight]{figures/mul/figure17b.png}}								
					\end{figure}
				\end{block}	
			\end{column}
			\begin{column}[t]{0.40\textwidth}				
				\begin{alertblock}{DL approach}	
					\footnotesize IoU= $0.64$						
					\begin{figure}[ht!]	
						\centering
						\subfloat{\includegraphics[height=0.32\textheight]{figures/RMS_L3_S3_B_333x333p_50kHz_5HC_18Vpp_x10_pzt_Ijjeh_updated_results_.png}}
						\quad
						\subfloat{\includegraphics[height=0.32\textheight]{figures/Binary_RMS_L3_S3_B__333x333p_50kHz_5HC_18Vpp_x10_pzt_Ijjeh_.png}}						
					\end{figure}				
				\end{alertblock}				
			\end{column}				
		\end{columns}	
	\end{frame}
	%%%%%%%%%%%%%%%%%%%%%%%%%%%%%%%%%%%%%%%%%%%%%%%%%%%%%%%%%%%%%%%%%%%%
	\section{Part II: Super-resolution image reconstruction}
	\begin{frame}{Super-resolution image reconstruction}
		\begin{columns}[T]
			\begin{column}[t]{0.6\textwidth}
				\begin{alertblock}{Deep learning super-resolution model (DLSR)}					
					\begin{footnotesize}
						\justifying
						\settowidth{\leftmargini}{\usebeamertemplate{itemize item}}
						\addtolength{\leftmargini}{\labelsep}
						\begin{itemize}
							\item Registering HR full wavefield with an SLDV is a~time-consuming process.
							\item DLSR model aims to recover HR full wavefield scans from a~LR measurements (below the Nyquist-Shannon sampling rate).
						\end{itemize} 
					\end{footnotesize}					
				\end{alertblock}						
				\begin{exampleblock}{Compressive sensing (CS) theory}
					\footnotesize
					\justifying
					Any natural signal (\(x\)), e.g. (sounds, images) can be recovered using considerably fewer measurements (\(y\)) than standard methods.
					\vfill
					\begin{figure}[ht!]
						\centering
						\includegraphics[width=.45\textwidth]{matrix_mask.png}
					\end{figure}
				\end{exampleblock}							
			\end{column}
			\begin{column}[t]{0.4\textwidth}
				\begin{figure}[ht!]
					\centering
					\includegraphics[width=1\textwidth]{superresolution_flowchart.png}
				\end{figure}
			\end{column}
		\end{columns}		
	\end{frame}
	%%%%%%%%%%%%%%%%%%%%%%%%%%%%%%%%%%%%%%%%%%%%%%%%%%%%%%%%%%%%%%%%%%%%
	%%%%%%%%%%%%%%%%%%%%%%%%%%%%%%%%%%%%%%%%%%%%%%%%%%%%%%%%%%%%%%%%%%%%
	\setcounter{subfigure}{0}
	\begin{frame}{Experimental test case}		
		\begin{columns}[T]
			%%%%%%%%%%%%%%%%%%%%%%%%%%%%%%%%%%%%%%%%%%%%%%%%%%%%%%%%%%%%
			\begin{column}[t]{0.25\textwidth}				
				\begin{figure}	
					\centering					
					\includegraphics[width=1\textwidth]{frame110_32x32.png}
					%					\caption{LR input \((N_f = 110)\)}
				\end{figure}
				\footnotesize
				LR measurements (Input): \(32\times32=1024\)p. \\
				HR (Output): \(512\times512=262144\)p.
			\end{column}
			%%%%%%%%%%%%%%%%%%%%%%%%%%%%%%%%%%%%%%%%%%%%%%%%%%%%%%%%%%%%
			\begin{column}[t]{.25\textwidth}
				\begin{block}{HR label}
					\begin{figure}
						\centering
						\subfloat{\includegraphics[width=0.75\textwidth]{figure10a.png}}
						\vfill
						\subfloat{\includegraphics[width=0.75\textwidth]{figure11a.png}}
					\end{figure}
				\end{block}
			\end{column}
			%%%%%%%%%%%%%%%%%%%%%%%%%%%%%%%%%%%%%%%%%%%%%%%%%%%%%%%%%%%%
			\begin{column}[t]{.25\textwidth}
				\begin{block}{CS: 1024p}
					\begin{figure}
						\centering
						\subfloat{\includegraphics[width=0.75\textwidth]{figure10b.png}}
						\vfill						
						\subfloat{\includegraphics[width=0.75\textwidth]{figure11b.png}}
					\end{figure}
				\end{block}				
			\end{column}
			%%%%%%%%%%%%%%%%%%%%%%%%%%%%%%%%%%%%%%%%%%%%%%%%%%%%%%%%%%%%
			\begin{column}[t]{.25\textwidth}
				\begin{alertblock}{DLSR}
					\begin{figure}
						\centering
						\subfloat{\includegraphics[width=0.75\textwidth]{figure10e.png}}
						\vfill			
						\subfloat{\includegraphics[width=0.75\textwidth]{figure11e.png}}\quad
					\end{figure}
				\end{alertblock}				
				%%%%%%%%%%%%%%%%%%%%%%%%%%%%%%%%%%%%%%%%%%%%%%%%%%%%%%%%%%%%
			\end{column}				
		\end{columns}
	\end{frame}
	%%%%%%%%%%%%%%%%%%%%%%%%%%%%%%%%%%%%%%%%%%%%%%%%%%%%%%%%%%%%%%%%
	\section{My current work}
	%%%%%%%%%%%%%%%%%%%%%%%%%%%%%%%%%%%%%%%%%%%%%%%%%%%%%%%%%%%%%%%%
	\begin{frame}{Current research}
		\begin{columns}[T]
			\begin{column}[t]{0.49\textwidth}
				\centering
				\textbf{\underline{Metamaterials: Dispersion diagrams}}
				\begin{figure}
					\centering
					\efbox{\includegraphics[width=1\textwidth]{Surrogate_DL_model_for_PC_Abdalraheem.png}}
				\end{figure}
			\end{column}
			\begin{column}[t]{0.49\textwidth}
				\centering
				\textbf{\underline{COMSOL vs. surrogate DL model predictions}}
				\begin{figure}
					\centering
					\efbox{\includegraphics[width=1\textwidth]{plot_1029_4672_10425_KF_DL_FEM_BG_triple_tile.png}}							
				\end{figure}			
			\end{column}		
		\end{columns}				
	\end{frame}	
	%%%%%%%%%%%%%%%%%%%%%%%%%%%%%%%%%%%%%%%%%%%%%%%%%%%%%%%%%%%%%%%%
	
	\begin{frame}{Publications}
		\vspace{5pt}
		\settowidth{\leftmargini}{\usebeamertemplate{itemize item}}
		\addtolength{\leftmargini}{\labelsep}
		\begin{tiny}					
			\begin{columns}[T]
				\begin{column}[t]{0.48\textwidth}
					\underline{\textbf{Journal articles}}
					\begin{enumerate}
						\justifying
						%%%%%%%%%%%%%%%%%%%%%%%%%%%%%%%%%%%%%%%%%%%%%%%%
						\item Ijjeh, A., Ullah, S., Radzienski, M. and Kudela, P., 2023. Deep learning super-resolution for the reconstruction of full wavefield of Lamb waves. \textbf{\textit{Mechanical Systems and Signal Processing}}, 186, p.109878.						
						\textbf{[IF:8.934]}
						%%%%%%%%%%%%%%%%%%%%%%%%%%%%%%%%%%%%%%%%%%%%%%%%
						\item Ullah, S., Ijjeh, A.A. and Kudela, P., 2023. Deep learning approach for delamination identification using animation of Lamb waves. 						
						\textbf{\textit{Engineering Applications of Artificial Intelligence}}, 117, p.105520.		
						\textbf{[IF:7.802]}
						%%%%%%%%%%%%%%%%%%%%%%%%%%%%%%%%%%%%%%%%%%%%%%%%
						\item Ijjeh, A.A. and Kudela, P., 2022. Deep learning based segmentation using full wavefield processing for delamination identification: A comparative study. \textbf{\textit{Mechanical Systems and Signal Processing}}, 168, p.108671. \textbf{[IF:8.934]}
						%%%%%%%%%%%%%%%%%%%%%%%%%%%%%%%%%%%%%%%%%%%%%%%%
						\item Ijjeh, A.A., Ullah, S. and Kudela, P., 2021. Full wavefield processing by using FCN for delamination detection. \textbf{\textit{Mechanical Systems and Signal Processing}}, 153, p.107537.		
						\textbf{[IF:8.934]}	
						%%%%%%%%%%%%%%%%%%%%%%%%%%%%%%%%%%%%%%%%%%%%%%%%
					\end{enumerate}					
				\end{column}
				\begin{column}[t]{0.48\textwidth}
					\underline{\textbf{Conference papers}}
					\begin{enumerate}
						\justifying
						\item {Ijjeh, A.}, Kudela, P. Convolutional LSTM for delamination imaging in composite laminates. 
						The 4th International Conference on Machine Learning and Intelligent Systems (MLIS 2022), November \(8^{th}\) - \(11^{th}\), 2022, Seoul, Republic of Korea.
						\item Ijjeh, A. and Kudela, P., 2022, June. Delamination Identification Using Global Convolution Networks. 
						In European Workshop on Structural Health Monitoring: EWSHM 2022-Volume 3 (pp. 521-529). Cham: Springer International Publishing.		
						\item {Ijjeh, A.}, Kudela, P. Feasibility Study of Full Wavefield Processing by Using CNN for Delamination Detection. 
						Proceedings of the International Conference on Structural Health Monitoring of Intelligent
						Infrastructure, June \(30^{th}\) - July \(2^{nd}\), 2021, Porto, Portugal, ISSN 2564-3738, pages 709-713.
					\end{enumerate}		
					\underline{\textbf{Chapters}}					
					\begin{enumerate}
						\justifying
						\item {Abdalraheem Ijjeh}, Deep Learning based Damage Imaging techniques, chapter in: Wybrane zagadnienia
						inżynierii mechanicznej, Praca zbiorowa pod redakcja M.~Mieloszyk, T. Ochrymiuka, Wydawnictwo Instytutu
						Maszyn Przepływowych PAN, Gdańsk, 2022, ISBN: 978-83-66928-09-1.
						\item {Abdalraheem Ijjeh}, Data-driven based approach for damage detection, chapter in: Wybrane zagadnienia
						inżynierii mechanicznej, Praca zbiorowa pod redakcja M.~Mieloszyk, T. Ochrymiuka, Wydawnictwo Instytutu
						Maszyn Przepływowych PAN, Gdańsk, 2021, ISBN: 978-83-66928-00-8.				
						\item {Abdalraheem Ijjeh}, Machine Learning for SHM: Literature Review, chapter in: Wybrane zagadnienia
						inżynierii mechanicznej, Praca zbiorowa pod redakcja M.~Mieloszyk, T. Ochrymiuka, Wydawnictwo Instytutu
						Maszyn Przepływowych PAN, Gdańsk, 2020, ISBN: 978-83-88237-97-3.
					\end{enumerate}
				\end{column}		
			\end{columns}
		\end{tiny}
	\end{frame}	
	%%%%%%%%%%%%%%%%%%%%%%%%%%%%%%%%%%%%%%%%%%%%%%%%%%%%%%%%%%%%%%%%%%%%
	\setcounter{subfigure}{0}
	%%%%%%%%%%%%%%%%%%%%%%%%%%%%%%%%%%%%%%%%%%%%%%%%%%%%%%%%%%%%%%%%%%%%
	{
		\setbeamercolor{palette primary}{fg=blue, bg=white}
		\begin{frame}[standout]
			\centering
			Thank you for your listening!\\ \vspace{12pt}
		\end{frame}
	}
\end{document}
