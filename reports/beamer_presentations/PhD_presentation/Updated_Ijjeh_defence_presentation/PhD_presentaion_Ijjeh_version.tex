%\PassOptionsToPackage{draft}{graphicx}
\documentclass[10pt,aspectratio=169,dvipsnames]{beamer} % aspect ratio 16:9
%\graphicspath{{../../figures/}}

%\includeonlyframes{frame1,frame2,frame3}

%%%%%%%%%%%%%%%%%%%%%%%%%%%%%%%%%%%%%%%%%%%%%%%%%%
% Packages
%%%%%%%%%%%%%%%%%%%%%%%%%%%%%%%%%%%%%%%%%%%%%%%%%%
\usepackage{appendixnumberbeamer}
\usepackage{booktabs}
\usepackage{csvsimple} % for csv read
\usepackage[scale=2]{ccicons}
\usepackage{pgfplots}
\usepackage{xspace}
%\usepackage{amsmath}
\usepackage{totcount}
\usepackage{tikz}
\usepackage{bm}
\usepackage{float}
\usepackage{eso-pic} 
\usepackage{wrapfig}
\usepackage{animate,media9}
\usepackage{subfig}
\usepackage{fancybox}
%\usepackage{multimedia}
\usepackage{dashbox}
\usepackage{tcolorbox}
\usepackage{multicol}
\usepackage{multirow}
\usepackage{xcolor}
\usepackage[document]{ragged2e}
\usepackage[labelformat=empty]{caption}
\usepackage{comment}
\usepackage{mathtools}% Loads amsmath

\usepackage{efbox,graphicx}

\usepackage{pgfpages}
%\setbeameroption{show notes}

\captionsetup[figure]{labelformat=empty}%
\usefonttheme{structurebold}

\efboxsetup{linecolor=Lightskyblue,linewidth=2pt}

%\usepackage[font=footnotesize,labelfont=bf]{caption}
%\captionsetup[figure]{font=small}

%\usepackage[export]{adjustbox}
%\usepackage{background}
%\backgroundsetup{contents=preliminary,placement=bottom,color=blue}
%\usepackage{FiraSans}

%\usepackage{comment}
%\usetikzlibrary{external} % speedup compilation
%\tikzexternalize % activate!
%\usetikzlibrary{shapes,arrows} 

%\usepackage{bibentry}
%\nobibliography*
\usepackage{ifthen}
\newcounter{angle}
\setcounter{angle}{0}
%\usepackage{bibentry}
%\nobibliography*
\usepackage{caption}%


%\usepackage{etoolbox}
%\apptocmd{\frame}{}{\justifying}{} 

\graphicspath{{figures/}}

%%%%%%%%%%%%%%%%%%%%%%%%%%%%%%%%%%%%%%%%%%%%%%%%%%
% Metropolis theme custom modification file
%%%%%%%%%%%%%%%%%%%%%%%%%%%%%%%%%%%%%%%%%%%%%%%%%%
% Metropolis theme custom modification file
%%%%%%%%%%%%%%%%%%%%%%%%%%%%%%%%%%%%%%%%%%%%%%%%%%
% Metropolis theme custom colors
%%%%%%%%%%%%%%%%%%%%%%%%%%%%%%%%%%%%%%%%%%%%%%%%%%
\usetheme[progressbar=foot]{metropolis}
\useoutertheme{metropolis}
\useinnertheme{metropolis}
\usefonttheme{metropolis}
\setbeamercolor{background canvas}{bg=white}

%\usecolortheme{spruce}

\definecolor{myblue}{rgb}{0.19,0.55,0.91}
\definecolor{mediumblue}{rgb}{0,0,205}
\definecolor{darkblue}{rgb}{0,0,139}
\definecolor{Dodgerblue}{HTML}{1E90FF}
\definecolor{Navy}{HTML}{000080} % {rgb}{0,0,128}
\definecolor{Aliceblue}{HTML}{F0F8FF}
\definecolor{Lightskyblue}{HTML}{87CEFA}
\definecolor{logoblue}{RGB}{1,67,140}
\definecolor{Purple}{HTML}{911146}
\definecolor{Orange}{HTML}{CF4A30}

\setbeamercolor{progress bar}{bg=Lightskyblue}
\setbeamercolor{progress bar}{ fg=logoblue} 
\setbeamercolor{frametitle}{bg=logoblue}
\setbeamercolor{title separator}{fg=logoblue}
\setbeamercolor{block title}{bg=Lightskyblue!30,fg=black}
\setbeamercolor{block body}{bg=Lightskyblue!15,fg=black}
\setbeamercolor{alerted text}{fg=Purple}
% notes colors
\setbeamercolor{note page}{bg=white}
\setbeamercolor{note title}{bg=Lightskyblue}
%%%%%%%%%%%%%%%%%%%%%%%%%%%%%%%%%%%%%%%%%%%%%%%%%%
%  Theme modifications
%%%%%%%%%%%%%%%%%%%%%%%%%%%%%%%%%%%%%%%%%%%%%%%%%%
% modify progress bar linewidth
\makeatletter
\setlength{\metropolis@progressinheadfoot@linewidth}{2pt} 
\setlength{\metropolis@titleseparator@linewidth}{1pt}
\setlength{\metropolis@progressonsectionpage@linewidth}{1pt}

\setbeamertemplate{progress bar in section page}{
	\setlength{\metropolis@progressonsectionpage}{%
		\textwidth * \ratio{\thesection pt}{\totvalue{totalsection} pt}%
	}%
	\begin{tikzpicture}
		\fill[bg] (0,0) rectangle (\textwidth, 
		\metropolis@progressonsectionpage@linewidth);
		\fill[fg] (0,0) rectangle (\metropolis@progressonsectionpage, 
		\metropolis@progressonsectionpage@linewidth);
	\end{tikzpicture}%
}
\makeatother
\newcounter{totalsection}
\regtotcounter{totalsection}

\AtBeginDocument{%
	\pretocmd{\section}{\refstepcounter{totalsection}}{\typeout{Yes, prepending 
	was successful}}{\typeout{No, prepending was not successful}}%
}%
%%%%%%%%%%%%%%%%%%%%%%%%%%%%%%%%%%%%%%%%%%%%%%%%%%
%  Bibliography mods
%%%%%%%%%%%%%%%%%%%%%%%%%%%%%%%%%%%%%%%%%%%%%%%%%%
\setbeamertemplate{bibliography item}{\insertbiblabel} %% Remove book symbol 
%%from references and add number in square brackets
% kill the abominable icon (without number)
%\setbeamertemplate{bibliography item}{}
%\makeatletter
%\renewcommand\@biblabel[1]{#1.} % number only
%\makeatother
% remove line breaks in bibliography
\setbeamertemplate{bibliography entry title}{}
\setbeamertemplate{bibliography entry location}{}
%%%%%%%%%%%%%%%%%%%%%%%%%%%%%%%%%%%%%%%%%%%%%%%%%%
%  Bibliography custom commands
%%%%%%%%%%%%%%%%%%%%%%%%%%%%%%%%%%%%%%%%%%%%%%%%%%
\newcommand{\bibliotitlestyle}[1]{\textbf{#1}\par}

\newif\ifinbiblio
\newcounter{bibkey}
\newenvironment{biblio}[2][long]{%
	%\setbeamertemplate{bibliography item}{\insertbiblabel}
	\setbeamertemplate{bibliography item}{}% without numbers
	\setbeamerfont{bibliography item}{size=\footnotesize}
	\setbeamerfont{bibliography entry author}{size=\footnotesize}
	\setbeamerfont{bibliography entry title}{size=\footnotesize}
	\setbeamerfont{bibliography entry location}{size=\footnotesize}
	\setbeamerfont{bibliography entry note}{size=\footnotesize}
	\ifx!#2!\else%
	\bibliotitlestyle{#2}%
	\fi%
	\begin{thebibliography}{}%
		\inbibliotrue%
		\setbeamertemplate{bibliography entry title}[#1]%
	}{%
		\inbibliofalse%
		\setbeamertemplate{bibliography item}{}%
	\end{thebibliography}%
}

\newcommand{\biblioref}[5][short]{
	\setbeamertemplate{bibliography entry title}[#1]
	\stepcounter{bibkey}%
	\ifinbiblio%
	\bibitem{\thebibkey}%
	#2
	\newblock #4
	\ifx!#5!\else\newblock {\em #5}, #3 \fi%
	\else%
	\begin{biblio}{}
		\bibitem{\thebibkey}
		#2
		\newblock #4
		\ifx!#5!\else\newblock {\em #5}, #3\fi
	\end{biblio}
	\fi
}
%
%\newbibmacro*{hypercite}{%
%	\renewcommand{\@makefntext}[1]{\noindent\normalfont##1}%
%	\footnotetext{%
%		\blxmkbibnote{foot}{%
%			\printtext[labelnumberwidth]{%
%				\printfield{prefixnumber}%
%				\printfield{labelnumber}}%
%			\addspace
%			\fullcite{\thefield{entrykey}}}}}
%
%\DeclareCiteCommand{\hypercite}%
%{\usebibmacro{cite:init}}
%{\usebibmacro{hypercite}}
%{}
%{\usebibmacro{cite:dump}}
%
%% Redefine the \footfullcite command to use the reference number
%\renewcommand{\footfullcite}[1]{\cite{#1}\hypercite{#1}}
%\usefonttheme[onlymath]{Serif} % It should be uncommented if Fira fonts in 
%%math does not work

%%%%%%%%%%%%%%%%%%%%%%%%%%%%%%%%%%%%%%%%%%%%%%%%%%
% Custom commands
%%%%%%%%%%%%%%%%%%%%%%%%%%%%%%%%%%%%%%%%%%%%%%%%%%
% matrix command 
\newcommand{\matr}[1]{\mathbf{#1}} % bold upright (Elsevier, Springer)
%\newcommand{\matr}[1]{#1} % pure math version
%\newcommand{\matr}[1]{\bm{#1}} % ISO complying version
% vector command 
\newcommand{\vect}[1]{\mathbf{#1}} % bold upright (Elsevier, Springer)
% bold symbol
\newcommand{\bs}[1]{\boldsymbol{#1}}
% derivative upright command
\DeclareRobustCommand*{\drv}{\mathop{}\!\mathrm{d}}
\newcommand{\ud}{\mathrm{d}}
% 
\newcommand{\themename}{\textbf{\textsc{metropolis}}\xspace}

\def\checkmark{\tikz\fill[scale=0.4](0,.35) -- (.25,0) -- (1,.7) -- (.25,.15) -- cycle;} 


%\setbeameroption{show notes on second screen}
%\setbeamertemplate{note page}{\insertnote}
%%%%%%%%%%%%%%%%%%%%%%%%%%%%%%%%%%%%%%%%%%%%%%%%%%
% Title page options
%%%%%%%%%%%%%%%%%%%%%%%%%%%%%%%%%%%%%%%%%%%%%%%%%%
\date{February 10, 2023}
%\date{}
%%%%%%%%%%%%%%%%%%%%%%%%%%%%%%%%%%%%%%%%%%%%%%%%%%
% option 1
%%%%%%%%%%%%%%%%%%%%%%%%%%%%%%%%%%%%%%%%%%%%%%%%%%%
\justifying\title{FEASIBILITY STUDY OF ARTIFICIAL INTELLIGENCE APPROACH FOR DELAMINATION IDENTIFICATION IN COMPOSITE LAMINATES}
%\subtitle{In preparation for a Ph.D. defence}
\author{\textbf{Ph.D. candidate: Eng. Abdalraheem A. Ijjeh } 
	\and \\ 
	\textbf{Supervisor: D.Sc. Ph.D. Eng. Paweł Kudela}} 
% logo align to Institute 
\institute{Institute of Fluid Flow Machinery \\ 
	Polish Academy of Sciences \\ 
	Gdansk, Poland \\
	\vspace{-1.5cm}
	\flushright 
	\includegraphics[width=6cm]{imp_logo.png}}

%%%%%%%%%%%%%%%%%%%%%%%%%%%%%%%%%%%%%%%%%%%%%%%%%%
%\tikzexternalize % activate!
%%%%%%%%%%%%%%%%%%%%%%%%%%%%%%%%%%%%%%%%%%%%%%%%%%
\setbeamertemplate{section in toc}[sections numbered]
\setbeamertemplate{subsection in toc}[subsections numbered]

\begin{document}
	%%%%%%%%%%%%%%%%%%%%%%%%%%%%%%%%%%%%%%%%%%%%%%%%%%
	\maketitle
	%%%%%%%%%%%%%%%%%%%%%%%%%%%%%%%%%%%%%%%%%%%%%%%%%%%%%%%%%%%%%%%%%%%%
	\note{
		My name is Abdalraheem Ijjeh.
		I am a PhD candidate at the Institute of Fluid Flow Machinery, Polish Academy of Sciences.		
		Supervised by Professor Pawe{l} Kudela.		
		The title of my PhD thesis is FEASIBILITY STUDY OF ARTIFICIAL INTELLIGENCE APPROACH FOR DELAMINATION IDENTIFICATION IN COMPOSITE LAMINATES.		
		Thank you all for you attending my PhD defence presentation. 
	}
	%%%%%%%%%%%%%%%%%%%%%%%%%%%%%%%%%%%%%%%%%%%%%%%%%%%%%%%%%%%%%%%%%%%%
	%%%%%%%%%%%%%%%%%%%%%%%%%%%%%%%%%%%%%%%%%%%%%%%%%%%%%%%%%%%%%%%%%%%%
	% SLIDES
	%%%%%%%%%%%%%%%%%%%%%%%%%%%%%%%%%%%%%%%%%%%%%%%%%%%%%%%%%%%%%%%%%%%%
	\begin{frame}[label=frame1]{Outlines}
			\setbeamertemplate{section in toc}[sections numbered]
			\setbeamertemplate{subsection in toc}[subsections numbered]
			\tableofcontents
	\end{frame}	
	%%%%%%%%%%%%%%%%%%%%%%%%%%%%%%%%%%%%%%%%%%%%%%%%%%%%%%%%%%%%%%%%%%%%
	\note{The presentation will be as follows: 
		In the introduction section, I will briefly introduce the motivations, objectives, contributions, and novelty of my work.
		Section two will be an overview of the deep learning approach
		In sections three and three, I will briefly talk about the development environment for deep learning models.
		Section 4 will introduce my work on delamination identification with deep learning approaches.
		In Section 5, I will present the second part of my work, which is on super-resolution image reconstruction for the full wavefield of Lamb waves. 
		The conclusions and future work will be presented in section 6.
	}
	%%%%%%%%%%%%%%%%%%%%%%%%%%%%%%%%%%%%%%%%%%%%%%%%%%%%%%%%%%%%%%%%%%%%
	\section{Introduction}
	\begin{frame}{Motivations}
		\begin{columns}[T]
			\begin{column}[t]{.55\textwidth}
				\begin{figure}[t]
%					\centering
%					\subfloat{\includegraphics[width=.7\textwidth]{Composite_advantages.png}}					
%					\\
					\subfloat{\includegraphics[width=.85\textwidth]{delaminated_plate1.jpg}}
				\end{figure}
				\begin{tcolorbox}
					\justifying\noindent\alert{Delamination detection} in its early stages can significantly help avoiding catastrophic events.
				\end{tcolorbox}				
			\end{column}
			\begin{column}[t]{0.45\textwidth}
				\begin{figure}[t]
					\includegraphics[width=1\textwidth]{Crashes.png}
				\end{figure}
			\tiny {source: https://www.structuresinsider.com/post/the-difference-between-buckling-compression-shear}
			\end{column}
		\end{columns}
	\end{frame}
	%%%%%%%%%%%%%%%%%%%%%%%%%%%%%%%%%%%%%%%%%%%%%%%%%%%%%%%%%%%%%%%%%%%%
	\note
	{
		Composite laminates have a wide range of applications in various industries due to their good characteristics, such as:
		high strength, low density, light weight, and resistance to fatigue and corrosion, among others.
		
		However, composite laminates may encounter some defects, such as cracks, fibre breakage, and debonding.
		
		In particular, laminated composite materials are more sensitive to damage in the form of delamination due to weak transverse tensile and interlaminar shear strengths.
			
		Delaminations can seriously decrease the performance of composite structures.
		Accordingly, delamination detection in its early stages can significantly help to avoid catastrophic structural collapses.
	}
	%%%%%%%%%%%%%%%%%%%%%%%%%%%%%%%%%%%%%%%%%%%%%%%%%%%%%%%%%%%%%%%%%%%%
	\begin{frame}{Objectives}
%		\begin{footnotesize}
			\justifying
			\begin{itemize}
				\item {To develop \textcolor{logoblue}{a novel artificial intelligence (AI) driven diagnostic system} for delamination identification in composite laminates such as carbon fibre reinforced polymers (CFRP).}
				\item {To address the issue of \textcolor{logoblue}{slow data acquisition} by by scanning laser Doppler vibrometer (SLDV) of high-resolution full wavefields of Lamb wave propagation.}
			\end{itemize}
%		\end{footnotesize}		
		\begin{tcolorbox}			
			\begin{alertblock}{Thesis}
				\justifying
				\textbf{It is possible to use an end-to-end approach in which a deep neural network	processes the animation of propagating waves (input) directly into a damage map (output).}
			\end{alertblock}
%			\begin{alertblock}{Thesis}
%				\justifying
%				It is possible to speed up the data acquisition process SLDV by employing an~appropriate deep learning (DL) approach to recover high-resolution (HR) full wavefield scans from a~low-resolution (LR).
%			\end{alertblock}
		\end{tcolorbox}		
	\end{frame}
	%%%%%%%%%%%%%%%%%%%%%%%%%%%%%%%%%%%%%%%%%%%%%%%%%%%%%%%%%%%%%%%%%%%%
	\note
	{
		I have two primary objectives in my work:
		The primary objective of this work is to an artificial intelligence driven diagnostic system for delamination identification in composite laminates.
		
		The second objective is to address the issue of slow data acquisition  by a scanning laser Doppler vibrometer of high-resolution full wavefields of Lamb waves propagation.
		
		Accordingly, my thesis states that:
	}
	%%%%%%%%%%%%%%%%%%%%%%%%%%%%%%%%%%%%%%%%%%%%%%%%%%%%%%%%%%%%%%%%%%%%
	\begin{frame}{Contribution and novelty}
		\begin{columns}[T]
			\begin{column}[c]{0.95\textwidth}
				\begin{figure}
					\centering
					\includegraphics[height=.85\textheight]{full_procedure.png}	
				\end{figure}		
			\end{column}
		\end{columns}		
	\end{frame}
	%%%%%%%%%%%%%%%%%%%%%%%%%%%%%%%%%%%%%%%%%%%%%%%%%%%%%%%%%%%%%%%%%%%%
	\note
	{
		Guided waves are very often utilised in non destructive testing and structural health monitoring.
		In which a scanning laser Doppler vibrometer that is a non-contact technique dedicated to full wavefield measurements of vibration and guided wave propagation. 
				
		The novelty of my work is in the utilisation for the first time the full wavefield scans of guided wave propagation as an input to various deep learning modes that specifically developed for delamination identification. 
		
		Additionally, to demonstrate the feasibility of my models, I have compared them with adaptive wavenumber filtering that is a conventional signal processing technique for damage imaging.
		
		Furthermore, to address the slow data acquisition by the scanning laser Doppler vibrometer I have developed a deep learning model to recover the high-resolution full wavefield measurements from it is low-resolution measurements.
		
		Additionally, I compared my model with a well-known technique, the compressive sensing.		
	}
	%%%%%%%%%%%%%%%%%%%%%%%%%%%%%%%%%%%%%%%%%%%%%%%%%%%%%%%%%%%%%%%%%%%%
	\section*{Experimental measurements}
	%%%%%%%%%%%%%%%%%%%%%%%%%%%%%%%%%%%%%%%%%%%%%%%%%%%%%%%%%%%%%%%%%%%%
	\begin{frame}[t]{SLDV measurements: Setup}
		\begin{columns}[T]
			\column{0.5\textwidth}
			\begin{figure}
				\includegraphics[width=0.8\textwidth]{wibrometr-laserowy-1d_small-description.png}
			\end{figure}
			\column{0.5\textwidth}
			\begin{enumerate}
				\item Signal generator: TTI 1241 
				\item Amplifier: Piezo Systems EPA-104-230 $\pm$200 Vp
				\item Specimen
				\item Scanning head: Polytec PSV-400
				\item DAQ system: Polytec
			\end{enumerate}
		\end{columns}
		{\small
			Measurements were taken on a uniform grid of \textbf{333$\times$333 points}.\\
			Excitation in the form of Hann windowed sine signal of carrier frequency \textbf{50 kHz} was applied to piezoelectric transducer.}
a	\end{frame}
	%%%%%%%%%%%%%%%%%%%%%%%%%%%%%%%%%%%%%%%%%%%%%%%%%%%%%%%%%%%%%%%%%%%%
	\note
	{
		In this slide, the experimental setup for the SLDV measurements is presented.
		
		The specimen was excited using a Piezoelectric transducer placed at the center of the plate p.
		
		The measurements of guided wave propagation were acquired from the bottom surface of the plate by the scanning laser Doppler vibrometer to produce the full wavefields.	
	}
	%%%%%%%%%%%%%%%%%%%%%%%%%%%%%%%%%%%%%%%%%%%%%%%%%%%%%%%%%%%%%%%%%%%%
	\begin{frame}[t]{Composite specimen}
		\begin{columns}[T]
			\begin{column}[c]{0.6\textwidth}
					\begin{itemize}
						\item 16 layers set at the same angle 
						\item carbon: Prepreg GG 205 P (fibres Toray FT 300 - 3K 200 tex), $E=230$ GPa
						\item epoxy resin: IMP503Z-HT by Impregnatex Compositi 						
						\item density: 1522.4~kg/m\textsuperscript{3}
						\item dimensions: 500$\times$500$\times$3.9 mm
					\end{itemize}			
			\end{column}
			\begin{column}[c]{0.4\textwidth}
				\begin{figure}
					\centering
					\includegraphics[width=.6\textwidth]{weave-1.jpg}
					\caption{Plain weave fabric}
				\end{figure}
			\end{column}			
		\end{columns}
	\end{frame}
	%%%%%%%%%%%%%%%%%%%%%%%%%%%%%%%%%%%%%%%%%%%%%%%%%%%%%%%%%%%%%%%%%%%%
	\note{
		The specimen used for experimental evaluation are carbon fibre reinforcement polymer composed of 16 layers of plain weave fabric with the following characteristics:

		The specimen dimension is \(500 \times 500\)~mm with a thickness of 3.9~mm.
	}
	%%%%%%%%%%%%%%%%%%%%%%%%%%%%%%%%%%%%%%%%%%%%%%%%%%%%%%%%%%%%%%%%%%%%
	\begin{frame}[t]{Specimens with defects}
		\vspace{-0.5cm}
		\begin{columns}[T]
			\column{0.5\textwidth}
			\begin{figure}
				\includegraphics[width=0.9\textwidth]{plate_multi_delam_arrangement_large_fonts.png}
			\end{figure}
			\column{0.5\textwidth}
			\begin{figure}
				\includegraphics[width=0.75\textwidth]{plate_single_delam_arrangement_large_fonts.png}
			\end{figure}
		\end{columns}
	\end{frame}
	%%%%%%%%%%%%%%%%%%%%%%%%%%%%%%%%%%%%%%%%%%%%%%%%%%%%%%%%%%%%%%%%%%%%
\note
	{	
		\footnotesize
		The specifications of the specimens used for evaluating the developed deep-learning models are shown in this slide.
		
		Specimens with multiple delaminations are shown on the left.
		In which there are three specimen with multiple delamination.
		The delaminations were located at the same distance from the centre of the plate and placed at different thickness of the plate as show in the figure. 
		
		The figure on the right shows the specimen with a single delamination place at the half thickness, and the total thickness of the specimen is 3.5 mm.
		It is noteworthy to mention that this specimen has less areal density than the specimen on the left. 
		
		Furthermore, all measurements with SLDV were conducted from the bottom surface of the plate.
		
		
	}	
	%%%%%%%%%%%%%%%%%%%%%%%%%%%%%%%%%%%%%%%%%%%%%%%%%%%%%%%%%%%%%%%%%%%%
	\section*{Synthetic dataset generation}
	\setcounter{subfigure}{0}	%%%%%%%%%%%%%%%%%%%%%%%%%%%%%%%%%%%%%%%%%%%%%%%%%%%%%%%%%%%%%%%%%%%%
	\begin{frame}{Synthetic dataset generation}
		\begin{columns}[T]			
			\begin{column}{0.55\textwidth}
				\justifying
				\begin{itemize}
					\item Mindlin-Reissner plate theory.
					\item Spectral element method (SEM).
					\item Splitting elements and nodes at delamination.
					\item GMSH software was used for meshing quads then converted to spectral elements.
				\end{itemize}	
				\begin{figure}
					\subfloat{\includegraphics[width=0.8\textwidth]{shell.png}}	
				\end{figure}
			\end{column}
			\begin{column}{0.45\textwidth}	
				\begin{figure}
					\animategraphics[controls,autoplay,loop,width=0.9\textwidth]{1}{/gif_figs/mesh/m1_rand_single_delam_}{1}{20}
				\end{figure}	
			\end{column}
		\end{columns}	
	\end{frame}
	%%%%%%%%%%%%%%%%%%%%%%%%%%%%%%%%%%%%%%%%%%%%%%%%%%%%%%%%%%%%%%%%%%%%
	\note{
		\tiny
		To train my developed deep learning models, I need a large dataset of specimens with various delamination sizes, locations, and shapes scanned with the scanning laser Doppler vibrometer to obtain the full wave fields of guided wave propagation. 
		
		However, generating such a real dataset with artificially modeled defects is very difficult, costly, and time-consuming.
		
		Accordingly, a large synthetic dataset was generated, resembling the measurements registered by the scanning laser Doppler vibrometer at the bottom surface of the plate as a response to the piezoelectric transducer excitation at the centre of the plate.
		
		The synthetic dataset was generated by applying the Mindlin-Reissner theory for thin plates and the utilisation of the parallel computing of spectral element method for approximating the solution of wave propagation and their interaction with delamination. 
		
		The meshes on the left show the modelled delamination in green, and the piezoelectric transducer in red for each different scenario.
	}
	%%%%%%%%%%%%%%%%%%%%%%%%%%%%%%%%%%%%%%%%%%%%%%%%%%%%%%%%%%%%%%%%%%%%
	\setcounter{subfigure}{0}
	\begin{frame}{Dataset description}
		\begin{columns}[T]
			\begin{column}[t]{0.35\textwidth}
				\justifying
				\begin{itemize}
					\item 475 delamination scenarios.
					\item CFRP is made of 8-layers.
					\item Delamination modelled between the \(3^{rd}\) and \(4^{th}\) layers.
					\item Delamination size min 10~mm, max 40~mm.
					\item \textbf{3-months of computing}.
				\end{itemize}
			\end{column}
			\begin{column}[t]{0.2\textwidth}
				\begin{figure}[t]
					\centering
					\captionsetup{justification=centering}				
					\subfloat[Delamination placement]{\includegraphics[width=0.95\textwidth]{delamination_placement.png}}
				\end{figure}
			\end{column}
			\begin{column}[t]{0.45\textwidth}
				\begin{figure}[t]					
					\centering	
					\captionsetup{justification=centering}				
					\subfloat[Delamination orientation]{\includegraphics[width=0.95\textwidth]{figure1.png}}					
				\end{figure}
			\end{column}
		\end{columns}
	\end{frame}
	%%%%%%%%%%%%%%%%%%%%%%%%%%%%%%%%%%%%%%%%%%%%%%%%%%%%%%%%%%%%%%%%%%%%
	\note{
		The synthetically generated dataset consists of 475 delamination cases. 
		It was assumed that the composite laminate is made of eight layers with a total thickness of 3.9 mm. 
		The delamination was modeled between the third and fourth layers.
		The delamination geometrical size major and minor axes were randomly selected from interval \([10mm,\ 40mm]\). 
		The angle $\aleph$ ranges form \([0^{\circ}-180^{\circ}]\).
		The computation of the dataset took about three months.
	}
	%%%%%%%%%%%%%%%%%%%%%%%%%%%%%%%%%%%%%%%%%%%%%%%%%%%%%%%%%%%%%%%%%%%%
	\setcounter{subfigure}{0}
	\begin{frame}{Training Sample case}
		\begin{columns}[T]
			\begin{column}[c]{.32\textwidth}
				\begin{figure}
					\centering
					\captionsetup{justification=centering}					
					\animategraphics[autoplay,loop,width=0.85 \textwidth]{16}{figures/gif_figs/7_output/flat_shell_Vz_7_500x500bottom-}{1}{512}
					\caption{Full wavefield $s(x,y,t_k)$}
				\end{figure}
			\end{column}
			\begin{column}[c]{.32\textwidth}
				\begin{figure}
					\centering
					\captionsetup{justification=centering}					
					\includegraphics[width=0.85 \textwidth]{RMS_flat_shell_Vz_7_500x500bottom.png}
					\caption{RMS image $\hat{s}(x,y)$}
				\end{figure}
			\end{column}
			\begin{column}[c]{.32\textwidth}
				\begin{figure}
					\centering
					\captionsetup{justification=centering}			
					\includegraphics[width=0.85 \textwidth]{m1_rand_single_delam_7.png}
					\caption{Ground truth (label) in black/white}
				\end{figure}
			\end{column}
		\end{columns}
		\begin{equation*}
			\hat{s}(x,y) = \sqrt{\frac{1}{N}\sum_{k=1}^{N}s(x,y,t_k)^2} 
			\label{eqn:rms} 
		\end{equation*}
		\footnotesize
		\(N\) is the number of sampling points equal to 512, \((x,y)\) are the point coordinates on the plate, and \(t_k\) is the time.
	\end{frame}
	%%%%%%%%%%%%%%%%%%%%%%%%%%%%%%%%%%%%%%%%%%%%%%%%%%%%%%%%%%%%%%%%%%%%
	\note{
		In this slide, I present a single training scenario.
		The animation on the left represents the full wavefield frames of guided wave propagation, where x and y are the point coordinates, and tk is the time moment.
		The output of applying the root mean square formula to the full wavefield is shown in the middle figure.
		The figure on the right is the ground truth label of the delamination in white showing its location, size and shape.
	}
	%%%%%%%%%%%%%%%%%%%%%%%%%%%%%%%%%%%%%%%%%%%%%%%%%%%%%%%%%%%%%%%%%%%%
	%%%%%%%%%%%%%%%%%%%%%%%%%%%%%%%%%%%%%%%%%%%%%%%%%%%%%%%%%%%%%%%%%%%%
	\section{Deep learning approach: An~Overview}		
	%%%%%%%%%%%%%%%%%%%%%%%%%%%%%%%%%%%%%%%%%%%%%%%%%%%%%%%%%%%%%%%%%%%%
	\setcounter{subfigure}{0}
	%%%%%%%%%%%%%%%%%%%%%%%%%%%%%%%%%%%%%%%%%%%%%%%%%%%%%%%%%%%%%%%%%%%%
	\begin{frame}{Convolution Neural Networks (CNNs)}
		\begin{itemize}
			\item \textbf{\alert{Convolutional Neural Network} (CNN)} is a powerful feature extraction tool.
			\item Features are extracted by applying convolution operations \textbf\alert{(sliding dot product)}.
		\end{itemize}
		\begin{columns}[T]
			\begin{column}[t]{0.58\textwidth}
				\begin{figure}[ht!]
					\centering		
					\caption{\textbf{Convolution operation}}			
					\efbox{\animategraphics[autoplay,loop,width =1\textwidth]{2}{figures/gif_figs/files/plot_convolution_process_}{0}{32}}					
				\end{figure}
			\end{column}
		
			\begin{column}[t]{0.38\textwidth}
				\begin{figure}[ht!]
					\centering
					\caption{\textbf{CNN architecture}}
					\efbox{\includegraphics[width=1\textwidth]{cnn.png}}
				\end{figure}
			\end{column}			
		\end{columns}
	\end{frame}
	%%%%%%%%%%%%%%%%%%%%%%%%%%%%%%%%%%%%%%%%%%%%%%%%%%%%%%%%%%%%%%%%%%%%
	\note{
		Convolutional Neural Network (CNN) is a powerful deep learning architecture for image processing because it can extract complex feature patterns from images using convolution operations.
		
		The convolution operation for image processing is essentially a cross-correlation operation, also known as a sliding dot product or sliding inner product.
		
		The kernel slides over an input image of performing a convolution operation.
		The output of the convolution operation is a feature map.
		
		Consequently, kernels learn to detect different types of edges
		(vertical, horizontal, and diagonal edges), colour intensities, etc.	
		
		During the backpropagation process, all kernel weights are updated. 				
	}
	%%%%%%%%%%%%%%%%%%%%%%%%%%%%%%%%%%%%%%%%%%%%%%%%%%%%%%%%%%%%%%%%%%%%
	\setcounter{subfigure}{0}
	%%%%%%%%%%%%%%%%%%%%%%%%%%%%%%%%%%%%%%%%%%%%%%%%%%%%%%%%%%%%%%%%%%%%
	\begin{frame}{Encoder-decoder architecture}
		\begin{columns}[T]
			\begin{column}[t]{.45\textwidth}
				\justifying
				\begin{itemize}
					\item \alert{Encoder}: extracts features.
					\item \alert{Latent space}: condensed feature maps.
					\item \alert{Decoder}: locates the features.
				\end{itemize}	
			\end{column}
			\quad
			\begin{column}[t]{.55\textwidth}
				\begin{figure}
					\centering
					\includegraphics[width=1\textwidth]{nn_encoder_decoder.png}
				\end{figure}	
			\end{column}
		\end{columns}			
	\end{frame}
	%%%%%%%%%%%%%%%%%%%%%%%%%%%%%%%%%%%%%%%%%%%%%%%%%%%%%%%%%%%%%%%%%%%%
	\note{
		The encoder-decoder is a well-known architecture used for tasks such as computer vision.
		The encoder aims to produce compressed feature maps from the input image at various scale levels using cascaded convolutions and downsampling operations. 
		The decoder is responsible for upsampling the condensed feature maps in the latent space to the original input shape.		
	}
	%%%%%%%%%%%%%%%%%%%%%%%%%%%%%%%%%%%%%%%%%%%%%%%%%%%%%%%%%%%%%%%%%%%%
	\setcounter{subfigure}{0}
	%%%%%%%%%%%%%%%%%%%%%%%%%%%%%%%%%%%%%%%%%%%%%%%%%%%%%%%%%%%%%%%%%%%%
	\begin{frame}{Computer vision}
		\begin{columns}[T]
			\begin{column}[c]{0.27\textwidth}
				\justifying
				\alert {\textbf{Computer vision}} is a~field of AI that enables computers and systems to derive meaningful information from digital images, videos and other visual inputs. 
			\end{column}
			\quad
			\begin{column}[c]{0.7\textwidth}
				\only<1>{\begin{figure}
					\centering
					\efbox{\includegraphics[width=1\textwidth]{computer_vision_tasks_1.png}}					
				\end{figure}}
				\only<2>{\begin{figure}
						\centering
						\efbox{\includegraphics[width=1\textwidth]{computer_vision_tasks.png}}					
				\end{figure}}											
			\end{column}
		\end{columns}
	\only<2>{\textbf{In my work, I went with the hardest task \alert{(Pixel-wise image segmentation)}.}}
	\end{frame}
	%%%%%%%%%%%%%%%%%%%%%%%%%%%%%%%%%%%%%%%%%%%%%%%%%%%%%%%%%%%%%%%%%%%%
	\note{
		We now reach an essential concept: computer vision. 
		Computer vision is a subcategory of artificial intelligence that enables computers to obtain meaningful information from visual inputs.
		
		Computer vision has three hierarchical levels: 
		
		The first level performs a classification for the whole input and predicts one output.
		
		The second level classifies and locates the object that we are looking for. 
		
		The ultimate level of computer vision is to perform Pixel-wise segmentation (or semantic segmentation), in which each pixel in the input image is classified into its corresponding class.	
		
		I am using pixel wise image segmentation in my PhD work.
		
	}
	%%%%%%%%%%%%%%%%%%%%%%%%%%%%%%%%%%%%%%%%%%%%%%%%%%%%%%%%%%%%%%%%%%%%
	\setcounter{subfigure}{0}
	%%%%%%%%%%%%%%%%%%%%%%%%%%%%%%%%%%%%%%%%%%%%%%%%%%%%%%%%%%%%%%%%%%%%
	\begin{frame}{Why deep learning?}
		Conventional methods involve two processes:
		\alert{\textbf{Feature extraction and classification.}}
		\begin{figure}
			\centering
			\includegraphics[width=.95\textwidth]{conventional_ML.png}
		\end{figure}	
		Deep learning offers an \alert{\textbf{end-to-end}} approach: \alert{\textbf{Automatic}} feature extraction and classification.
		\begin{figure}
			\includegraphics[width=.95\textwidth]{DL_approach.png}
		\end{figure}
	\end{frame}
	%%%%%%%%%%%%%%%%%%%%%%%%%%%%%%%%%%%%%%%%%%%%%%%%%%%%%%%%%%%%%%%%%%%%
	\note{
		In this slide, I present a comparison between conventional machine learning and deep learning approaches to damage detection.
		
		In the conventional approaches, two processes must be performed: 
		The first is to extract the useful features from the registered data and then use a proper classification technique.
		
		This approach has several drawbacks:
		First, it requires a great amount of human labor and computational effort
		Secondly, it demands a high amount of experience from the practitioner.
		And lastly, when dealing with big data, it tends to be Inefficient as it requires a complex computation of damage features extraction and classification.
		
		On the other hand, the deep learning approach offers the opportunity to develop models that can automatically perform feature extraction and classification by themselves without human intervention in an end-to-end approach. 
	}
	%%%%%%%%%%%%%%%%%%%%%%%%%%%%%%%%%%%%%%%%%%%%%%%%%%%%%%%%%%%%%%%%%%%%
	\section{Development environment}
	\setcounter{subfigure}{0}
	%%%%%%%%%%%%%%%%%%%%%%%%%%%%%%%%%%%%%%%%%%%%%%%%%%%%%%%%%%%%%%%%%%%	
	\begin{frame}{Utilised Tools for DL models}
		All developed DL models were \alert{coded in-house} \\
		\alert{(Tailored specifically to meet tasks requirements)} 
		\begin{itemize}	
			\item \alert{Pycharm} (a dedicated Python Integrated Development Environment (IDE))		
			\item \alert{Python}: versions: 3.7 - 3.9
			\item \alert{TensorFlow} platform: versions: 2.0 - 2.6
			\item \alert{Keras} (deep learning API over TensorFlow): versions: 2.0 - 2.6
			\item \alert{GPUs}: NVIDIA RTX2080 /8GB, NVIDIA Tesla V100 /32 GB
		\end{itemize}
	\end{frame}
	%%%%%%%%%%%%%%%%%%%%%%%%%%%%%%%%%%%%%%%%%%%%%%%%%%%%%%%%%%%%%%%%%%%%
	\note
	{
		It is important to note that I coded all my deep-learning models in-house.
		
		To do so, I have utilised these tools:		
		(Python, TensorFlow platform, Keras API) under Pycharm IDE, which is open-source.
		Training a deep learning model has computational complexity.
		Therefore, I utilised these GPUs: 
	}
	%%%%%%%%%%%%%%%%%%%%%%%%%%%%%%%%%%%%%%%%%%%%%%%%%%%%%%%%%%%%%%%%%%%%
	\begin{frame}{Data preprocessing \& Hyperparameters tuning}	
				\begin{table}[t]
					\centering
					\resizebox{1\textwidth}{!}{%
						\begin{tabular}{lccc}
							\toprule[1.5pt]
							\multirow{2}{*}{\textbf{Applied techniques}}& \multicolumn{2}{c}{\textbf{Delamination identification}} & \multirow{2}{*}{\textbf{Super-resolution image reconstruction}} \\ \cmidrule(lr){2-3}
							& RMS image based & Full wavefield based & \\ \midrule 
							Dataset normalisation & \checkmark & \checkmark & \checkmark \\
							Dataset augmentation & \checkmark & & \\
							Cross validation & \checkmark & & \\ 
							Early stoppling & \checkmark & \checkmark & \checkmark \\ 
							Hyperparameter tuning & \checkmark & \checkmark & \checkmark \\ 
							Regularization & \checkmark & \checkmark & \checkmark \\ 
							\bottomrule[1.5pt]
						\end{tabular}
					}
				\begin{table}[t]
					\centering
					\resizebox{1\textwidth}{!}{%
					\begin{tabular}{l|c|c|c|c|c|c|c}
						\toprule[1.5pt]
						\textbf{Hyperparameters} & Learning rate & Momentum &
						Dropout rate & Batch size & Epochs & 			Optimisation technique & Objective loss function \\						
						\bottomrule[1.5pt]
					\end{tabular}}
				\end{table}
				\end{table}
				\begin{table}
					\centering
					\resizebox{.4\textwidth}{!}{%
					\begin{tabular}{ccc}
						\toprule[1.5pt]
						Training set & Testing set & Validation set \\
						\midrule
						$80\%$ & $20\%$ & $20\%$ out of training set \\
						\bottomrule[1.5pt]
					\end{tabular}}
				\end{table}					
	\end{frame}
	%%%%%%%%%%%%%%%%%%%%%%%%%%%%%%%%%%%%%%%%%%%%%%%%%%%%%%%%%%%%%%%%%%%%
	\note
	{
		Now, in this slide, I present the most applied techniques on the dataset during the training:
		1. Dataset normalization in which all values are have a range between 0 and 1. \\
		2. I have applied data augmentation to the dataset
		(475 RMS images) by flipping the images horizontally, vertically, and diagonally.
		As a result, the dataset size increased four times -1900 images were produced. \\
		3. Additionally, Cross validation was applied with data augmentation. \\
		4. In order to reduce the overfitting problem I have applied some techniques:
		1. Early stopping,
		2. Hyperparameter tuning (such as: )
		3. Regularization (such as dropouts, batch normalization, adding penalties to objective function, ...)\\ 
		Furthermore, I have splitted the dataset into three portions
		1. Training set \(80\%\) of the dataset\\
		2. Testing set \(20\%\) of the dataset\\
		3. validation set \(20\%\) out of the training set.
	}
	%%%%%%%%%%%%%%%%%%%%%%%%%%%%%%%%%%%%%%%%%%%%%%%%%%%%%%%%%%%%%%%%%%%%
	\section{Part I: Delamination identification}
	\begin{frame}{Semantic segmentation}
		\begin{columns}[T]
			\begin{column}[c]{0.47\textwidth}
				\centering
				\textbf{One-to-one \\image-based approach (RMS)} 
				\begin{figure}
					\centering
					\captionsetup{justification=centering}				
					\subfloat[Single input (image)]{\includegraphics[width=.45\textwidth]{RMS_flat_shell_Vz_381_500x500bottom.png}}\quad
					\subfloat[Single output]{\includegraphics[width=.45\textwidth]{GCN_381.png}}
				\end{figure}
			\end{column}
			\hfill
			\begin{column}[c]{0.47\textwidth}
				\centering
				\textbf{Many-to-one \\animation-based approach}
				\begin{figure}
					\centering
					\captionsetup{justification=centering}					
					\subfloat[Multiple frames (animation)]{\animategraphics[autoplay,loop,width=.45\textwidth]{4}{figures/gif_figs/381_output/output_381-}{85}{113}}\quad
					\subfloat[Single output]{\includegraphics[width=.45\textwidth]{GCN_381.png}}
				\end{figure}
			\end{column}	
		\end{columns}		
	\end{frame}	
	%%%%%%%%%%%%%%%%%%%%%%%%%%%%%%%%%%%%%%%%%%%%%%%%%%%%%%%%%%%%%%%%%%%%
	\note{
		In this part I will present deep learning approaches for delamination identification in composite laminates.
		It is important to note how these models perform in an end-to-end fashion.
		Accordingly, I have developed two approaches for delamination identification based on the inputs to the models:
		
		\begin{itemize}
			\item One-to-one approach (takes one input that is the root mean squared of the full wavefield) and produces one output of damage map)
			\item Many-to-one approach (takes animation of Lamb waves propagation as a sequence of frames and produces one output as damage map).
		\end{itemize}	
		}
	%%%%%%%%%%%%%%%%%%%%%%%%%%%%%%%%%%%%%%%%%%%%%%%%%%%%%%%%%%%%%%%%%%%%
%	\subsection{Developed DL models}
	%\begin{frame}{Common deep learning architectures}
	%	
	%	\begin{column}[t]{0.45\textwidth}
		%		\textbf{RMS based}\\
		%		\begin{itemize}
			%			\item Convolutional neural networks (CNN)
			%			\item Fully convolutional network (FCN)
			%		\end{itemize}
		%	\end{column}
	%	\hfill
	%	\begin{column}[t]{0.45\textwidth}
		%		\textbf{Full wavefield frames}\\
		%		\begin{itemize}
			%			\item Recurrent neural network (RNN)
			%			\item Long short-term memory (LSTM)
			%			\item ConvLSTM
			%		\end{itemize}
		%	\end{column}
	%\end{frame}
	
	%%%%%%%%%%%%%%%%%%%%%%%%%%%%%%%%%%%%%%%%%%%%%%%%%%%%%%%%%%%%%%%%%%%%
	\begin{frame}{Developed model for delamination identification}
		\begin{columns}[T]
			\begin{column}[t]{0.45\textwidth}
				\begin{block}{RMS based models}
					\begin{itemize}
						\item VGG 16 encoder-decoder
						\item Res-UNet					
						\item FCN-DenseNet
						\item PSPNet
						\item GCN
					\end{itemize}				
				\end{block}
			\end{column}
			\hfill
			\begin{column}[t]{.45\textwidth}
				\begin{block}{Full wavefield frames based model}					
					\begin{itemize}
						\item Autoencoder ConvLSTM
					\end{itemize}									
				\end{block}
			\end{column}
		\end{columns}
	\end{frame}	
	%%%%%%%%%%%%%%%%%%%%%%%%%%%%%%%%%%%%%%%%%%%%%%%%%%%%%%%%%%%%%%%%%%%%
	\note{
		Deep learning models for delamination identification were developed based on their inputs.
		
		The models based on the RMS images as input are:
		the residual UNet, VGG16 encoder-decoder, fully convolutional dense network (FCN-DenseNet), Pyramid scene parsing network (PSPNet), and The global convolutional neural network (GCN).
		
		And the model based on the animation of full wavefield frames is 
		Autoencoder ConvLSTM.
		
		All developed models were published in these articles.		
	}
	%%%%%%%%%%%%%%%%%%%%%%%%%%%%%%%%%%%%%%%%%%%%%%%%%%%%%%%%%%%%%%%%%%%%
	\section*{Developed RMS based models}	
	%%%%%%%%%%%%%%%%%%%%%%%%%%%%%%%%%%%%%%%%%%%%%%%%%%%%%%%%%%%%%%%%%%%%
	\setcounter{subfigure}{0}
	\begin{frame}{Developed RMS based models}
		\begin{figure}
			\includegraphics[width=0.95\textwidth]{Developed_rms_models.png}
		\end{figure}
	\end{frame}
	%%%%%%%%%%%%%%%%%%%%%%%%%%%%%%%%%%%%%%%%%%%%%%%%%%%%%%%%%%%%%%%%%%%%
	\setcounter{subfigure}{0}	
	%%%%%%%%%%%%%%%%%%%%%%%%%%%%%%%%%%%%%%%%%%%%%%%%%%%%%%%%%%%%%%%%%%%%
	\note{
		Both the Res-UNet and VGG16 encoder-decoders are autoencoders.
		The main difference between them is the additional skip connections that were added to Res-Unet at the encoder and decoder levels. \\
		FCN-DenseNet applies an encoder-decoder scheme with skip connections between the encoder and the decoder paths.
		The main component in FCN-DenseNet is the dense block. 
		The dense block is constructed from a varying number of convolutional layers. 
		The purpose of the dense block is to concatenate feature maps of a layer with its output to emphasize spatial details information. \\		
		The idea of PSPNet is to provide adequate global contextual information for pixel-level scene parsing by concatenating the local and global features together. 
		Hence, a spatial pyramid pooling module was introduced to perform four different pooling levels with four different pool sizes. 
		In this way, the pyramid pooling module can capture contextual features at different scales.\\		
		GCN addresses the importance of having large kernels at the convolution operations for both localization and classification tasks for semantic segmentation.
		
		
	}
%	%%%%%%%%%%%%%%%%%%%%%%%%%%%%%%%%%%%%%%%%%%%%%%%%%%%%%%%%%%%%%%%%%%%%%%%%
%	
%	\begin{frame}{Pyramid Scene Parsing Network}
%		\begin{figure} [h!]
%			\centering
%			\includegraphics[height=.7\textheight]{figure7.png}
%			\caption{PSPNet architecture.} 
%		\end{figure}
%	\end{frame}
%	%%%%%%%%%%%%%%%%%%%%%%%%%%%%%%%%%%%%%%%%%%%%%%%%%%%%%%%%%%%%%%%%%%%%%%%%
%	\note{
%			
%	}
%	%%%%%%%%%%%%%%%%%%%%%%%%%%%%%%%%%%%%%%%%%%%%%%%%%%%%%%%%%%%%%%%%%%%%%%%%
%	\begin{frame}{Global Convolution Network}
%		\begin{columns}[T]
%			\begin{column}[c]{0.55\textwidth}
%				\begin{figure}
%					\centering
%					\includegraphics[width=.9\textwidth]{figure8.png}
%					\caption{GCN architecture.} 
%				\end{figure}	
%			\end{column}
%			\begin{column}[c]{0.45\textwidth}
%				\begin{figure}
%					\centering
%					\includegraphics[width=.9\textwidth]{figure9.png}
%					\caption{(a) Residual block, (b) GCN block, (c) Boundary Refinement.} 
%				\end{figure}	
%			\end{column}
%		\end{columns}
%	\end{frame}	
%	%%%%%%%%%%%%%%%%%%%%%%%%%%%%%%%%%%%%%%%%%%%%%%%%%%%%%%%%%%%%%%%%%%%%%%%%
%	\note{
%	
%	}	
	%%%%%%%%%%%%%%%%%%%%%%%%%%%%%%%%%%%%%%%%%%%%%%%%%%%%%%%%%%%%%%%%%%%%
	%\setcounter{subfigure}{0}
	%\begin{frame}{RMS based models}
	%	\begin{column}[c]{0.55\textwidth}
		%		\begin{figure}
			%			\subfloat[Res-UNet model]{\includegraphics[width=1\textwidth]{figure4.png}}
			%		\end{figure}
		%	\end{column}
	%	\begin{column}[c]{0.35\textwidth}
		%		\begin{figure}
			%			\subfloat[Data flow \& intermediate outputs of layers \label{fig:}]{\animategraphics[autoplay, controls,width=.8\textwidth]{4}{figures/gif_figs/381__inter_pred/intermediate_output-}{0}{103}}
			%\end{figure}
			%	\end{column}
		%
		%\end{frame}
	%%%%%%%%%%%%%%%%%%%%%%%%%%%%%%%%%%%%%%%%%%%%%%%%%%%%%%%%%%%%%%%%%%%%
	\setcounter{subfigure}{0}
	%%%%%%%%%%%%%%%%%%%%%%%%%%%%%%%%%%%%%%%%%%%%%%%%%%%%%%%%%%%%%%%%%%%%
	\section*{Full wavefield frames based model}
	\begin{frame}{Autoencoder ConvLSTM}
		\begin{figure}[ht!]
			\centering
			\includegraphics[width=1\textwidth]{figure3.png}
		\end{figure}
	\end{frame}
	%%%%%%%%%%%%%%%%%%%%%%%%%%%%%%%%%%%%%%%%%%%%%%%%%%%%%%%%%%%%%%%%%%%%
	\note{
		The autoencoder ConvLSTM was trained on a sequence of consecutive full wavefield frames.
		
		Hence, for the synthetic dataset, the delamination location is known, I only utilised 24 frames after the interaction with the damage for training the model.
		
		Those selected frames containing the required features about the delamination shape and location are fed into an encoder-decoder model at once using a time-distributed layer.
		
		Then, the output of the decoder was forwarded into the ConvLSTM layer to learn the long-term spatiotemporal features.
		
		
		Now, for real-life situations where the damage location is not known, the figure on the left illustrates the complete procedure of obtaining intermediate predictions for the testing cases and finally calculating the RMS image.
		
		Accordingly, the window slides over all the input frames with a shift of one frame at a time.
		
		Then, an RMS of all the intermediate predictions is calculated to produce a damage map.		
	}
	%%%%%%%%%%%%%%%%%%%%%%%%%%%%%%%%%%%%%%%%%%%%%%%%%%%%%%%%%%%%%%%%%%%%
	\begin{frame}{Evaluation metrics for delamination identification}
		\begin{columns}[T]
			\begin{column}[c]{0.45\textwidth}
				For evaluating delamination identification
				\begin{itemize}
					\item Intersection over Union (IoU): 
					\begin{equation*}
						\textup{IoU}=\frac{Intersection}{Union}=\frac{\hat{Y} \cap Y}{\hat{Y} \cup Y}
						\label{eqn:iou}
					\end{equation*}
					\item Percentage area error $\epsilon$:
					\begin{equation*}
						\epsilon=\frac{|A-\hat{A}|}{A} \times 100\%
						\label{eqn:mean_size_error}
					\end{equation*}
				\end{itemize}
			\end{column}
			\begin{column}[c]{0.45\textwidth}
				\begin{figure}
					\centering
					\includegraphics[width=1.0\textwidth]{IoU_figure.png}		
				\end{figure}
			\end{column}
		\end{columns}
	\end{frame}
	%%%%%%%%%%%%%%%%%%%%%%%%%%%%%%%%%%%%%%%%%%%%%%%%%%%%%%%%%%%%%%%%%%%%
	\note{
		To evaluate all developed models for delamination identification, two metrics were used:
		
		The first metric is the mean intersection over union (also known as the Jaccard index), which calculates the area of intersection between actual and predicted values divided by the union of them.
		
		The second metric is the percentage area error \(\epsilon\), which calculates the percentage of the difference between actual and predicted areas to the actual area size. 
	}
	%%%%%%%%%%%%%%%%%%%%%%%%%%%%%%%%%%%%%%%%%%%%%%%%%%%%%%%%%%%%%%%%%%%%
	\section*{Evaluation: Numerical cases}
	%%%%%%%%%%%%%%%%%%%%%%%%%%%%%%%%%%%%%%%%%%%%%%%%%%%%%%%%%%%%%%%%
	\begin{frame}{Numerical test cases RMS based models (GCN model)}
		\begin{columns}[T]
			\begin{column}[c]{0.32\textwidth}
				\begin{figure}[c]
					\centering
					\captionsetup{justification=centering}					
					\animategraphics[controls,width=.9\textwidth]{2}{figures/gif_figs/456/intermediate_output-}{0}{82}
					\caption{\(1^{st}\) numerical case, IoU=0.71}
				\end{figure}
			\end{column}
			\hfill
			\begin{column}[c]{0.32\textwidth}
				\begin{figure}[c]
					\centering
					\captionsetup{justification=centering}					
					\animategraphics[controls,width=.9\textwidth]{2}{figures/gif_figs/438/intermediate_output-}{0}{82}
					\caption{\(2^{nd}\) numerical case, IoU=0.72}
				\end{figure}
			\end{column}
			\hfill
			\begin{column}[c]{0.32\textwidth}
				\begin{figure}[c]
					\centering
					\captionsetup{justification=centering}
					
					\animategraphics[controls,width=.9\textwidth]{2}{figures/gif_figs/397/intermediate_output-}{0}{82}
					\caption{\(3^{rd}\) numerical case, IoU=0.86}
				\end{figure}					
			\end{column}
		\end{columns}
	\end{frame}
	%%%%%%%%%%%%%%%%%%%%%%%%%%%%%%%%%%%%%%%%%%%%%%%%%%%%%%%%%%%%%%%%%%%%
	\note{
		In this section, the numerical and experimental evaluation of the developed models for delamination identification will be presented.
		
		This slide shows the three numerical test cases evaluated with the GCN model.
		
		For the first case, the delamination is barely visible by the naked eye, yet the model could identify the delamination with IOU= 0.71.
		
		For the second case, the IOU = 0.72, and for the third case, the IOU = .86.
		
		It's important to notice that GCN can identify the delamination with high accuracy and almost no noise.		
	}
	%%%%%%%%%%%%%%%%%%%%%%%%%%%%%%%%%%%%%%%%%%%%%%%%%%%%%%%%%%%%%%%%%%%%
%	\begin{frame}{RMS based: Analysis of numerical cases}
%		\begin{columns}[T]
%%				\tiny
%%				\begin{column}[c]{0.48\textwidth}
%%					%%%%%%%%%%%%%%%%%%%%%%%%%%%%%%%%%%%%%%%%%%%%%%%%%%%%%%%%%%%%
%%					\begin{table}[ht!]
%%						\centering
%%						\caption{Evaluation metrics of the three numerical cases.}
%%						\label{tab:RMS_num_cases}
%%						\begin{tabular}{cccccc}
%%							\toprule[1.5pt]
%%							\multirow{2}{*}{Model} & \multirow{2}{*}{case number} & \multicolumn{1}{c}{\multirow{2}{*}{A [mm\textsuperscript{2}]}} & \multicolumn{3}{c}{Predicted output} \\ 
%%							\cmidrule(lr){4-6} & & & \multicolumn{1}{c}{IoU} & \multicolumn{1}{c}{\(\hat{A}\) [mm\textsuperscript{2}]} & \(\epsilon\) \\
%%							\midrule
%%							\multirow{3}{*}{Res-UNet} 							
%%							& 1 & 257 & \multicolumn{1}{c}{0.45} & \multicolumn{1}{c}{143} & \(44.36\%\) \\ 
%%							& 2 & 105 & \multicolumn{1}{c}{0.67} & \multicolumn{1}{c}{88} & \(16.19\%\) \\ 
%%							& 3 & 537 & \multicolumn{1}{c}{0.80} & \multicolumn{1}{c}{478} & \(10.99\%\) \\ 
%%							\midrule
%%							\multirow{3}{*}{VGG16 encoder-decoder} 
%%							& 1 & 257 & \multicolumn{1}{c}{0.69} & \multicolumn{1}{c}{203} & \(21.01\%\) \\ 
%%							& 2 & 105 & \multicolumn{1}{c}{0.75} & \multicolumn{1}{c}{117} & \(11.43\%\) \\ 
%%							& 3 & 537 & \multicolumn{1}{c}{0.65} & \multicolumn{1}{c}{385} & \(28.31\%\) \\ 
%%							\midrule
%%							\multirow{3}{*}{FCN-DenseNet} 
%%							& 1 & 257 & \multicolumn{1}{c}{0.52} & \multicolumn{1}{c}{505} & \(96.50\%\) \\ 
%%							& 2 & 105 & \multicolumn{1}{c}{0.66} & \multicolumn{1}{c}{118} & \(12.38\%\) \\ 
%%							& 3 & 537 & \multicolumn{1}{c}{0.72} & \multicolumn{1}{c}{815} & \(51.77\%\) \\ 
%%							\midrule
%%							\multirow{3}{*}{PSPNet} 
%%							& 1 & 257 & \multicolumn{1}{c}{0.00} & \multicolumn{1}{c}{0} & \(-\%\) \\ 
%%							& 2 & 105 & \multicolumn{1}{c}{0.44} & \multicolumn{1}{c}{156} & \(48.57\%\) \\ 
%%							& 3 & 537 & \multicolumn{1}{c}{0.77} & \multicolumn{1}{c}{610} & \(13.59\%\) \\ 
%%							\midrule
%%							\multirow{3}{*}{GCN} 
%%							& 1 & 257 & \multicolumn{1}{c}{0.71} & \multicolumn{1}{c}{215} & \(16.34\%\) \\ 
%%							& 2 & 105 & \multicolumn{1}{c}{0.72} & \multicolumn{1}{c}{177} & \(68.57\%\) \\ 
%%							& 3 & 537 & \multicolumn{1}{c}{0.86} & \multicolumn{1}{c}{523} & \(2.61\%\) \\ 
%%							\bottomrule[1.5pt]
%%						\end{tabular}	
%%					\end{table}
%	 %%%%%%%%%%%%%%%%%%%%%%%%%%%%%%%%%%%%%%%%%%%%%%%%%%%%%%%%%%%%%%%
%% \end{column}
%%		\hfill
%		\begin{column}[c]{0.9\textwidth}
%			\begin{table}[ht!]
%				\centering
%				\caption{Analysis of numerical cases.}
%				\label{tab:table_all_numerical_cases}	
%				\begin{tabular}{lcc}
%					\toprule[1.5pt]
%					Model & mean IoU & max IoU \\ 
%					\midrule 
%					Res-UNet & \(0.66\) & \(0.89\) \\ 
%					VGG16 encoder-decoder & \(0.57\) & \(0.84\) \\ 
%					FCN-DenseNet & \(0.68\) & \(0.92\) \\ 
%					PSPNet & \(0.55\) & \(0.91\) \\ 
%					GCN & \textbf{\(0.76\)} & \textbf{\(0.93\)} \\ 
%					\bottomrule[1.5pt]
%				\end{tabular}
%			\end{table}
%		\end{column}
%		\end{columns}
%	\end{frame}
	%%%%%%%%%%%%%%%%%%%%%%%%%%%%%%%%%%%%%%%%%%%%%%%%%%%%%%%%%%%%%%%%%%%%
	\note{
		The table presents the mean and maximum values calculated for the previously unseen numerical test set for all RMS-based models. 
		It also shows that all models have a relatively high value, indicating their ability to detect and localize the delamination.
		However, the best performance was achieved by the GCN model.
	}
	%%%%%%%%%%%%%%%%%%%%%%%%%%%%%%%%%%%%%%%%%%%%%%%%%%%%%%%%%%%%%%%%%%%%
	\begin{frame}{Numerical test cases animation of Lamb waves}
		\setcounter{subfigure}{0}
		\only<1>{
			\begin{alertblock}{First test case}
				\begin{figure}
					\centering
					\captionsetup{justification=centering}
					\subfloat[Full wavefield (512 frames)]{\animategraphics[autoplay,loop,height=4cm,keepaspectratio]{32}{figures/gif_figs/381_output/output_381-}{1}{512}}\quad
					\subfloat[RMS of all intermediate predictions]{\includegraphics[height=4.1cm,keepaspectratio]{figures/RMS_Ijjeh_num_case_381.png}}\quad
					\subfloat[Binary RMS, IoU= 0.88]{\includegraphics[height=4cm,keepaspectratio]{figures/Binary_RMS_Ijjeh_num_case381_.png}}\quad
				\end{figure}
			\end{alertblock}}
		\setcounter{subfigure}{0}
		\only<2>{
			\begin{alertblock}{Second test case}
				\begin{figure}
					\centering
					\captionsetup{justification=centering}
					\subfloat[Full wavefield (512 frames)]{\animategraphics[autoplay,loop,height=4cm,keepaspectratio]{32}{figures/gif_figs/385_output/output_385-}{1}{512}}\quad
					\subfloat[RMS of all intermediate predictions]{\includegraphics[height=4.1cm,keepaspectratio]{figures/RMS_Ijjeh_num_case_385.png}}\quad
					\subfloat[Binary RMS, IoU= 0.58]{\includegraphics[height=4cm,keepaspectratio]{figures/Binary_RMS_Ijjeh_num_case385_.png}}
				\end{figure}
			\end{alertblock}}
		\setcounter{subfigure}{0}
		\only<3>{
			\begin{alertblock}{Third test case}
				\begin{figure}
					\centering
					\captionsetup{justification=centering}
					\subfloat[Full wavefield (512 frames)]{\animategraphics[autoplay,loop,height=4cm,keepaspectratio]{32}{figures/gif_figs/394_output/output_394-}{1}{512}}\quad
					\subfloat[RMS of all intermediate predictions]{\includegraphics[height=4.1cm,keepaspectratio]{figures/RMS_Ijjeh_num_case_394.png}}\quad
					\subfloat[Binary RMS, IoU= 0.8]{\includegraphics[height=4cm,keepaspectratio]{figures/Binary_RMS_Ijjeh_num_case394_.png}}
				\end{figure}
			\end{alertblock}}
	\end{frame}
	%%%%%%%%%%%%%%%%%%%%%%%%%%%%%%%%%%%%%%%%%%%%%%%%%%%%%%%%%%%%%%%%%%%%
	\note{
		In the following three slides, I will present three numerical test cases evaluated with the autoencoder ConvLSTM model, which takes animations of full wavefields.
		
		animation (a) shows the full wavefield of 512 frames as an input to the model.
		Figure (b) shows the RMS damage map for all intermediate predictions.
		Figure (c) shows the binary RMS with IoU= 0.88.
		
		The second case is more complex as the delamination is near the corner of the plate.
		In this case, the IoU=0.58.
		
		The third case is also complex, as the delamination is located near the plate edge.
		In this case, the IoU=0.8.
		
		Also, we can notice see how clean all the predictions are.		
	}
	%%%%%%%%%%%%%%%%%%%%%%%%%%%%%%%%%%%%%%%%%%%%%%%%%%%%%%%%%%%%%%%%%%%%
%	\begin{frame}{Animation based: Analysis of numerical cases}
%		%%%%%%%%%%%%%%%%%%%%%%%%%%%%%%%%%%%%%%%%%%%%%%%%%%%%%%%%%%%%%%%%%%%%
%		\begin{table}[!h]
%			\centering
%			\caption{Evaluation metrics of the three numerical cases.}
%			\begin{tabular}{ccccc}
%				\toprule[1.5pt]
%				\multirow{2}{*}{case number} & \multicolumn{1}{c}{\multirow{2}{*}{A [mm\textsuperscript{2}]}} & \multicolumn{3}{c}{Predicted output} \\ 
%				\cmidrule(lr){3-5} & & \multicolumn{1}{c}{IoU} & \multicolumn{1}{c}{\(\hat{A}\) [mm\textsuperscript{2}]} & \(\epsilon\) \\
%				\midrule
%				1 & 763 & \multicolumn{1}{c}{0.88} & \multicolumn{1}{c}{735} & \(3.67\%\) \\ 
%				2 & 388 & \multicolumn{1}{c}{0.58} & \multicolumn{1}{c}{248} & \(36.08\%\) \\ 
%				3 & 297 & \multicolumn{1}{c}{0.80} & \multicolumn{1}{c}{280} & \(5.72\%\) \\			 					
%				\bottomrule[1.5pt]
%			\end{tabular}	
%			\label{tab:num_cases}
%		\end{table}			
%	\end{frame}
%	%%%%%%%%%%%%%%%%%%%%%%%%%%%%%%%%%%%%%%%%%%%%%%%%%%%%%%%%%%%%%%%%%%%%%%%%%%%%
%	\note{
%		 The shown table presents the evaluation metrics for the autoencoder ConvLSTM model regarding the three numerical cases shown in the previous slide.
%		 
%		 The table gathers the actual delamination area, predicted delamination area, intersection over union IoU, and percentage area error \(\epsilon\) to each case.
%	}
	%%%%%%%%%%%%%%%%%%%%%%%%%%%%%%%%%%%%%%%%%%%%%%%%%%%%%%%%%%%%%%%%%%%%
	\section*{Evaluation: Experimental cases}	
	\setcounter{subfigure}{0}		%%%%%%%%%%%%%%%%%%%%%%%%%%%%%%%%%%%%%%%%%%%%%%%%%%%%%%%%%%%%%%%%%%%%
	\begin{frame}{Experimental results: RMS image-based (Single delamination)}
		\begin{columns}[T]
			\begin{column}[t]{.25\textwidth}
				\begin{figure}[ht!]
					\centering
					\captionsetup{justification=centering}
					\includegraphics[height=.35\textheight]{ERMS_with_label.png}
					\caption{ERMS \& label}
				\end{figure}
				\justifying
				\tiny
				Kudela, P., Radzienski, M. and Ostachowicz, W., 2018. \textbf{Impact induced damage assessment by means of Lamb wave image processing}. \textit{Mechanical Systems and Signal Processing}, 102, pp.23-36.
			\end{column}
			\begin{column}[t]{0.5\textwidth}
				\begin{block}{Adaptive wavenumber filtering}
					\centering
					\footnotesize
					IoU=$0.401$
					\begin{figure}[ht!]
						\centering
						\captionsetup{justification=centering}
						\subfloat{\includegraphics[height=.35\textheight]{ERMSF_CFRP_teflon_3o_375_375p_50kHz_5HC_x12_15Vpp.png}}
						\quad
						\subfloat{\includegraphics[height=.35\textheight]{Binary_ERMSF_CFRP_teflon_3o_375_375p_50kHz_5HC_x12_15Vpp.png}}
					\end{figure}
				\end{block}					
			\end{column}		
			\begin{column}[t]{0.25\textwidth}
				\begin{alertblock}{DL approach: GCN}
					\centering
					\footnotesize
					IoU\(=0.723\)
					\begin{figure}[ht!]	
						\centering				
						\subfloat{\includegraphics[height=.35\textheight]{Fig_GCN_7.png}}
					\end{figure} 	
				\end{alertblock}				
			\end{column}
		\end{columns}	
	\end{frame}
	%%%%%%%%%%%%%%%%%%%%%%%%%%%%%%%%%%%%%%%%%%%%%%%%%%%%%%%%%%%%%%%%%%%%
	\note{
		This slide shows the experimental results of a specimen with a single delamination regarding the RMS-based models, GCN, and FCN-DenseNet.
		
		Additionally, the deep learning models were compared with the adaptive wavenumber filtering technique for damage imaging, which is a conventional signal processing technique developed in our department.
		
		Figure (a) shows the ERMS with a label depicting the delamination. 
		
		Figure (b) shows the result of applying the adaptive wavenumber filtering method, and figure (c) shows its binary output with IoU=0.401.
		
		Figure (d) shows the predicted damage map by the GCN model with IoU=0.723.
		Figure (e) shows the predicted damage map by the FCN-DenseNet model with IoU=0.54
		
		As shown, the deep learning models surpasses the conventional signal processing approach.
	}
	%%%%%%%%%%%%%%%%%%%%%%%%%%%%%%%%%%%%%%%%%%%%%%%%%%%%%%%%%%%%%%%%%%%%
%	\setcounter{subfigure}{0}
%	\begin{frame}{RMS based: Analysis of experimental case}
%		\begin{table}[!ht]
%			\centering
%			\caption{Evaluation metrics of the experimental case.}
%			\label{tab:rms_exp_case}
%			\begin{tabular}{lc}
%				\toprule[1.5pt]
%				Model & IoU	\\			
%				\midrule
%				Res-UNet & 0.58 \\ 
%				VGG16 encoder-decoder & 0.62 \\ 
%				FCN-DenseNet & 0.54 \\ 
%				PSPNet & 0.49 \\ 
%				GCN & 0.72\\ 
%				\bottomrule[1.5pt]
%			\end{tabular}		
%		\end{table}
%%			\begin{table}[!ht]
%%				\centering
%%				\caption{Evaluation metrics of the experimental case.}
%%				\label{tab:rms_exp_case}
%%				\begin{tabular}{l@{\ }cccc}
%%					\toprule
%%					\multicolumn{1}{l}{Model} & \multicolumn{1}{c}{A [mm\textsuperscript{2}]} & \multicolumn{3}{c}{Predicted output} \\ 
%%					\cmidrule(lr){3-5} & & \multicolumn{1}{c}{IoU} & \multicolumn{1}{c}{\(\hat{A}\) [mm\textsuperscript{2}]} & \(\epsilon\) \\ \midrule
%%					Res-UNet & \multicolumn{1}{c}{\multirow{5}{*}{210}} & \multicolumn{1}{c}{0.58} & \multicolumn{1}{c}{323} & \(53.8\%\) \\ 
%%					VGG16 encoder-decoder & & \multicolumn{1}{c}{0.62} & \multicolumn{1}{c}{320} & \(52.4\%\) 
%%					\\ 
%%					FCN-DenseNet & & \multicolumn{1}{c}{0.54} & \multicolumn{1}{c}{386} & \(83.8\%\) \\ 
%%					PSPNet & & \multicolumn{1}{c}{0.49} & \multicolumn{1}{c}{580} & \(176.2\%\) 
%%					\\ 
%%					GCN & & \multicolumn{1}{c}{0.72} & \multicolumn{1}{c}{309} & \(47.1\%\) 
%%					\\ 
%%					\bottomrule
%%				\end{tabular}		
%%			\end{table}
%	\end{frame}
	%%%%%%%%%%%%%%%%%%%%%%%%%%%%%%%%%%%%%%%%%%%%%%%%%%%%%%%%%%%%%%%%%%%%
	\note{
		The table shows the IoU values for all developed RMS based models for the single delamination specimen.
		
		Similarly to the numerical dataset, the best accuracy was achieved by using GCN.
		
%		As shown, the models are capable of precise detection and localisation of the delamination. 
%		We can see that the models can identify the delamination with almost free noise indicating the models are capable of generalising and detecting the delamination on previously unseen data. 
	}
	%%%%%%%%%%%%%%%%%%%%%%%%%%%%%%%%%%%%%%%%%%%%%%%%%%%%%%%%%%%%%%%%%%%%
	\setcounter{subfigure}{0}
	\begin{frame}{Experimental results: Full wavefield based (Single delamination)}		
		\begin{alertblock}{DL approach}
			IoU= $0.41$% and $\epsilon=71.56\%$ 
			\begin{figure}[ht!]
				\centering
				\subfloat[Input]{\animategraphics[autoplay,loop,height=3cm]{32}{figures/gif_figs/CFRP_teflon_3o_375_375p_50kHz_5HC_x12_15Vpp/CFRP_teflon_30-}{1}{256}}\quad
				\subfloat[Intermidate ouputs]{\animategraphics[autoplay,loop,height=3cm]{24}{figures/gif_figs/CFRP_ijjeh_single_delamination/intermediate_output-}{0}{231}}\quad
				\subfloat[RMS]{\includegraphics[height=3cm,keepaspectratio]{figures/RMS_CFRP_teflon_3o_375_375p_50kHz_5HC_x12_15Vpp_Ijjeh_updated_results_.png}}\quad
				\subfloat[Binary RMS]{\includegraphics[height=3cm,keepaspectratio]{figures/Binary_RMS_CFRP_teflon_3o__375_375p_50kHz_5HC_x12_15Vpp_Ijjeh_.png}}
			\end{figure}			
		\end{alertblock}
		
	\end{frame}
	%%%%%%%%%%%%%%%%%%%%%%%%%%%%%%%%%%%%%%%%%%%%%%%%%%%%%%%%%%%%%%%%%%%%
	\note{
		In this slide, I present the predicted results using the autoencoder ConvLSTM model regarding the single delamination case.
		
		Animation (a) shows the full wavefield measured by SLDV with 256 frames.
		
		Animation (b) shows the intermediate predictions of the model. 
		
		Figure (c) shows the RMS of all intermediate predictions.
		
		And finally, Figure (d) shows the binary RMS with IoU= 0.41		
	}
	%%%%%%%%%%%%%%%%%%%%%%%%%%%%%%%%%%%%%%%%%%%%%%%%%%%%%%%%%%%%%%%%%%%%
	\setcounter{subfigure}{0}	
	\begin{frame}{Experimental results: Full wavefield based (Multiple delaminations)}
		\begin{columns}[T]
			\begin{column}[t]{0.20\textwidth}
				\begin{block}{Input}
					\footnotesize Full wavefield					
					\begin{figure}[ht!]	
						\centering						
						\subfloat{\animategraphics[autoplay,loop,height=0.32\textheight]{32}{figures/gif_figs/input_specimen_3/specimen_3-}{1}{512}}
					\end{figure}
				\end{block}				
			\end{column}
			\begin{column}[t]{0.40\textwidth}				
				\begin{block}{Adaptive wavenumber filtering}
					\footnotesize IoU$=0.04$					
					\begin{figure}[ht!]	
						\centering
						\subfloat{\includegraphics[height=0.32\textheight]{figures/mul/figure17a.png}}
						\quad
						\centering
						\subfloat{\includegraphics[height=0.32\textheight]{figures/mul/figure17b.png}}								
					\end{figure}
				\end{block}	
			\end{column}
			\begin{column}[t]{0.40\textwidth}				
				\begin{alertblock}{DL approach}	
					\footnotesize IoU= $0.64$						
					\begin{figure}[ht!]	
						\centering
						\subfloat{\includegraphics[height=0.32\textheight]{figures/RMS_L3_S3_B_333x333p_50kHz_5HC_18Vpp_x10_pzt_Ijjeh_updated_results_.png}}
						\quad
						\subfloat{\includegraphics[height=0.32\textheight]{figures/Binary_RMS_L3_S3_B__333x333p_50kHz_5HC_18Vpp_x10_pzt_Ijjeh_.png}}						
					\end{figure}				
				\end{alertblock}				
			\end{column}				
		\end{columns}	
	\end{frame}
	%%%%%%%%%%%%%%%%%%%%%%%%%%%%%%%%%%%%%%%%%%%%%%%%%%%%%%%%%%%%%%%%%%%%
	\note{
		This slide presents the predicted results using the autoencoder ConvLSTM model compared to conventional signal processing adaptive wavenumber filtering regarding the multiple delamination case.
		
		Animation (a) shows the full wavefield measured by SLDV of 512 frames
		
		Figure (b) shows the damage map resulting from applying adaptive wavenumber filtering and figure (c) shows its binary output with IoU = 0.04
		
		Animation (d) shows all intermediate predictions with the deep learning model
		
		Figure (e) shows the RMS of all intermediate predictions and figure (f) shows the binary RMS with IoU = 0.64
		
		It can be concluded that utilising animations of Lamb waves propagation has better outcomes for delamination identification than the processing of a single image representing signal energy or RMS.		
	}
	%%%%%%%%%%%%%%%%%%%%%%%%%%%%%%%%%%%%%%%%%%%%%%%%%%%%%%%%%%%%%%%%%%%%
	\setcounter{subfigure}{0}
	%%%%%%%%%%%%%%%%%%%%%%%%%%%%%%%%%%%%%%%%%%%%%%%%%%%%%%%%%%%%%%%%%%%%
	\section{Part II: Super-resolution image reconstruction}
	\begin{frame}{Super-resolution image reconstruction}
		\begin{columns}[T]
			\begin{column}[t]{0.6\textwidth}
				\begin{alertblock}{Deep learning super-resolution model (DLSR)}					
					\begin{footnotesize}
						\justifying
						\settowidth{\leftmargini}{\usebeamertemplate{itemize item}}
						\addtolength{\leftmargini}{\labelsep}
						\begin{itemize}
							\item Registering HR full wavefield with an SLDV is a~time-consuming process.
							\item DLSR model aims to recover HR full wavefield scans from a~LR measurements (below the Nyquist-Shannon sampling rate).
						\end{itemize} 
					\end{footnotesize}					
				\end{alertblock}						
				\begin{exampleblock}{Compressive sensing (CS) theory}
					\footnotesize
					\justifying
					Any natural signal (\(x\)), e.g. (sounds, images) can be recovered using considerably fewer measurements (\(y\)) than standard methods.
					\vfill
					\begin{figure}[ht!]
					\centering
					\includegraphics[width=.45\textwidth]{matrix_mask.png}
					\end{figure}
				\end{exampleblock}							
			\end{column}
			\begin{column}[t]{0.4\textwidth}
				\begin{figure}[ht!]
					\centering
					\includegraphics[width=1\textwidth]{superresolution_flowchart.png}
				\end{figure}
			\end{column}
		\end{columns}		
	\end{frame}
	%%%%%%%%%%%%%%%%%%%%%%%%%%%%%%%%%%%%%%%%%%%%%%%%%%%%%%%%%%%%%%%%%%%%
	\note{
		Scanning laser Doppler vibrometer is a popular tool for the acquisition of the full wavefield of propagating guided waves, in particular Lamb waves.
		
		However, the process of acquiring the full wavefield of guided waves is time-consuming. 
		
		One possible solution to tackle this problem is to acquire the Lamb waves in a low-resolution form and then apply a deep learning-based super-resolution approach to that low-resolution frame of full wavefield data to retrieve the high-resolution frame.
	}

	%%%%%%%%%%%%%%%%%%%%%%%%%%%%%%%%%%%%%%%%%%%%%%%%%%%%%%%%%%%%%%%%%%%%
	\setcounter{subfigure}{0}
	\begin{frame}{Evaluation metrics for DLSR model}
		For evaluating the reconstructed HR full wavefield:
		\begin{itemize}
			\item Peak signal-to-noise ratio (PSNR):
			\begin{equation*}
				PSNR=20\log_{10}\left(\frac{R}{\sqrt{MSE}}\right)
				\label{PSNR_}
			\end{equation*}			
			\item Pearson correlation coefficient (also known as Pearson's r):
			\begin{equation*}
				r_{xy} = \frac{\sum_{i=1}^{n}(x_i - \bar{x})(y_i-\bar{y})}{\sqrt{\sum_{i=1}^{n}(x_i - \bar{x})^2}\sqrt{\sum_{i=1}^{n}(y_i - \bar{y})^2}}
				\label{Pearson_}
			\end{equation*}
		\end{itemize}
	\end{frame}
	%%%%%%%%%%%%%%%%%%%%%%%%%%%%%%%%%%%%%%%%%%%%%%%%%%%%%%%%%%%%%%%%%%%%
	\note
	{
		To evaluate the developed model, I used two metrics:
		\begin{itemize}
			\item The first one is the peak signal-to-noise ratio (PSNR), which refers to the maximum possible power of a signal and the power of the distorting noise that affects the quality of its representation.
%			This equation depicts the mathematical representation of PSNR, where R is the maximum fluctuation value in the input image, and MSE is the mean squared error between the actual and predicted output.
%			
			\item The second metric is the Pearson correlation coefficient (Pearson CC), which measures the linear relationship between two
			variable sets \(X\) (represents the ground truth values) and \(Y\) (represents the predicted values) as shown in the below equation.
			
%			Where \(n\) is the number of sample points,\(x_i, y_i\) are the individual value points representing the ground truth and predicted values, respectively, and \(\bar{x}\) and \(\bar{y}\) are the mean values of the sample and the prediction.
		\end{itemize}
	}
	%%%%%%%%%%%%%%%%%%%%%%%%%%%%%%%%%%%%%%%%%%%%%%%%%%%%%%%%%%%%%%%%%%%%
	\setcounter{subfigure}{0}
	\begin{frame}{Numerical test cases}
		\begin{columns}[T]				
			\begin{column}[c]{0.5\textwidth}				
				\begin{figure}
					\centering
					%%%%%%%%%%%%%%%%%%%%%%%%%%%%%%%%%%%%%%%%%%%%%%%%%%%%
%					\only<1>{
%						\begin{alertblock}{First test case}
%							\begin{figure}
%								\includegraphics[height=.35\textheight]{LR_456_frame_159_input.png}
%								\caption{LR input, $f_n=159$}
%							\end{figure}							
%							%%%%%%%%%%%%%%%%%%%%%%%%%%%%%%%%%%%%%%%%%%%%
%							\begin{table}[!h]
%								\centering 
%								\footnotesize
%								\begin{tabular}{cccc}
%									\toprule
%									\multicolumn{2}{c}{plate} & \multicolumn{2}{c}{delamination} \\
%									\cmidrule(lr){1-2} \cmidrule(lr){3-4}
%									PSNR & PEARSON CC & PSNR & PEARSON CC \\ 
%									\midrule
%									42.95 & 0.999 & 33.02 & 0.993 \\					
%									\bottomrule
%								\end{tabular}
%							\end{table}
%						\end{alertblock}}
						%%%%%%%%%%%%%%%%%%%%%%%%%%%%%%%%%%%%%%%%%%%%%%%%
					\only<1>{
						\begin{alertblock}{First test case}
							\begin{figure}
								\includegraphics[height=.35\textheight]{LR_438_frame_154_input.png}
								\caption{LR input, $f_n=154$}
							\end{figure}							
							%%%%%%%%%%%%%%%%%%%%%%%%%%%%%%%%%%%%%%%%%%%%
							\begin{table}[!h]
								\centering 
								\footnotesize
								\begin{tabular}{cccc}
									\toprule
									\multicolumn{2}{c}{plate} & \multicolumn{2}{c}{delamination} \\
									\cmidrule(lr){1-2} \cmidrule(lr){3-4}
									PSNR & PEARSON CC & PSNR & PEARSON CC \\ 
									\midrule
									47.00 & 0.998 & 38.52 & 0.995 \\					
									\bottomrule
								\end{tabular}
								\label{tab:num_DLSR_results_2_}
							\end{table}	
							%%%%%%%%%%%%%%%%%%%%%%%%%%%%%%%%%%%%%%%%%%%%
						\end{alertblock}}
					%%%%%%%%%%%%%%%%%%%%%%%%%%%%%%%%%%%%%%%%%%%%%%%%%%%%
					\only<2>{
						\begin{alertblock}{Second test case}
							\begin{figure}
								\includegraphics[height=.35\textheight]{LR_397_frame_127_input.png}
								\caption{LR input, $f_n=127$}
							\end{figure}						
							%%%%%%%%%%%%%%%%%%%%%%%%%%%%%%%%%%%%%%%%%%%%
							\begin{table}[!h]
								\centering 
								\footnotesize	
								\begin{tabular}{cccc}
									\toprule
									\multicolumn{2}{c}{plate} & \multicolumn{2}{c}{delamination} \\
									\cmidrule(lr){1-2} \cmidrule(lr){3-4}
									PSNR & PEARSON CC & PSNR & PEARSON CC \\ 
									\midrule
									48.60 & 0.998 & 46.67 & 0.998 \\					
									\bottomrule
								\end{tabular}
							\end{table}
						%%%%%%%%%%%%%%%%%%%%%%%%%%%%%%%%%%%%%%%%%%%%%%%%
						\end{alertblock}}
					%%%%%%%%%%%%%%%%%%%%%%%%%%%%%%%%%%%%%%%%%%%%%%%%%%%%
				\end{figure}
			\end{column}
			%%%%%%%%%%%%%%%%%%%%%%%%%%%%%%%%%%%%%%%%%%%%%%%%%%%%%%%%%%%%
			\begin{column}[c]{0.5\textwidth}
%				\only<1>{	
%					\setcounter{subfigure}{0}
%					\begin{figure}
%						\subfloat[listentry][HR ref]{\includegraphics[height=.35\textheight]{output_456_frame_159_full_frame_GT.png}}\quad
%						\subfloat[listentry][DLSR]{\includegraphics[height=.35\textheight]{output_456_frame_159_full_frame_pred.png}}
%						\\
%						\subfloat[listentry][Ref]{\includegraphics[height=.35\textheight]{output_456_frame_159_delamination_GT.png}}\quad
%						\subfloat[listentry][DLSR]{\includegraphics[height=.35\textheight]{output_456_frame_159_delamination_pred.png}}
%					\end{figure}}			
				\only<1>{
					\setcounter{subfigure}{0}					
					%%%%%%%%%%%%%%%%%%%%%%%%%%%%%%%%%%%%%%%%%%%%%%%%%%%%
					\begin{figure}
						\subfloat[listentry][HR ref]{\includegraphics[height=.35\textheight]{output_438_frame_154_full_frame_GT.png}}\quad
						\subfloat[listentry][DLSR]{\includegraphics[height=.35\textheight]{output_438_frame_154_full_frame_pred.png}}
						\\
						\subfloat[listentry][Ref]{\includegraphics[height=.35\textheight]{output_438_frame_154_delamination_GT.png}}\quad	
						\subfloat[listentry][DLSR]{\includegraphics[height=.35\textheight]{output_438_frame_154_delamination_pred.png}}
					\end{figure}}
				\only<2>{	
					\setcounter{subfigure}{0}
					\begin{figure}
						\centering
						\subfloat[listentry][HR ref]{\includegraphics[height=.35\textheight]{output_397_frame_127_full_frame_GT.png}}\quad
						\subfloat[listentry][DLSR]{\includegraphics[height=.35\textheight]{output_397_frame_127_full_frame_pred.png}}
						\\
						\subfloat[listentry][Ref]{\includegraphics[height=.35\textheight]{output_397_frame_127_delamination_GT.png}}\quad
						\subfloat[listentry][DLSR]{\includegraphics[height=.35\textheight]{output_397_frame_127_delamination_pred.png}}
					\end{figure}}
			\end{column}
		\end{columns}
	\end{frame}
	%%%%%%%%%%%%%%%%%%%%%%%%%%%%%%%%%%%%%%%%%%%%%%%%%%%%%%%%%%%%%%%%%%%%
	\note
	{
		\tiny
		In the following, the results of the reconstruction of HR frames for three numerical test cases will be presented.
		
		In all test cases, I used the first frame, which shows the initial interaction with delamination.
		
		In the first test case, the figure shows the low-resolution measurements at frame number 159.
		Figure a shows the actual HR frame, and figure b shows the predicted SR frame. 
		The PSNR value is 48.6
		
		Figure c shows the HR sub-frame at the delamination region, figure d shows the SR prediction at the delamination region, and the PSNR is 46.67.					
		
		In the second test case, the figure shows the low-resolution measurements at frame number 154.		
		Figure a shows the actual HR frame, and figure b shows the predicted SR frame. 
		The PSNR value is 47
		
		Figure c shows the HR sub-frame at the delamination region, figure d shows the SR prediction at the delamination region, and the PSNR is 38.52.
		
		In the third test case, the figure shows the low-resolution measurements at frame number 127.
		Figure a shows the actual HR frame, and figure b shows the predicted SR frame. 
		The PSNR value is 42.95
		
		Figure c shows the HR sub-frame at the delamination region, figure d shows the SR prediction at the delamination region, and the PSNR is 33.02.
		
			
	}
	%%%%%%%%%%%%%%%%%%%%%%%%%%%%%%%%%%%%%%%%%%%%%%%%%%%%%%%%%%%%%%%%%%%%
	\setcounter{subfigure}{0}
	\begin{frame}{Experimental test case}		
		\begin{columns}[T]
			%%%%%%%%%%%%%%%%%%%%%%%%%%%%%%%%%%%%%%%%%%%%%%%%%%%%%%%%%%%%
			\begin{column}[t]{0.25\textwidth}				
				\begin{figure}	
					\centering					
					\includegraphics[width=1\textwidth]{frame110_32x32.png}
%					\caption{LR input \((N_f = 110)\)}
				\end{figure}
				\footnotesize
				LR measurements (Input): \(32\times32=1024\)p. \\
				HR (Output): \(512\times512=262144\)p.
			\end{column}
			%%%%%%%%%%%%%%%%%%%%%%%%%%%%%%%%%%%%%%%%%%%%%%%%%%%%%%%%%%%%
			\begin{column}[t]{.25\textwidth}
				\begin{block}{HR label}
				 \begin{figure}
				 	\centering
				 	\subfloat{\includegraphics[width=0.75\textwidth]{figure10a.png}}
				 	\vfill
				 	\subfloat{\includegraphics[width=0.75\textwidth]{figure11a.png}}
				 \end{figure}
				\end{block}
			\end{column}
			%%%%%%%%%%%%%%%%%%%%%%%%%%%%%%%%%%%%%%%%%%%%%%%%%%%%%%%%%%%%
			\begin{column}[t]{.25\textwidth}
				\begin{block}{CS: 1024p}
					\begin{figure}
						\centering
						\subfloat{\includegraphics[width=0.75\textwidth]{figure10b.png}}
						\vfill						
						\subfloat{\includegraphics[width=0.75\textwidth]{figure11b.png}}
					\end{figure}
				\end{block}				
			\end{column}
			%%%%%%%%%%%%%%%%%%%%%%%%%%%%%%%%%%%%%%%%%%%%%%%%%%%%%%%%%%%%
			\begin{column}[t]{.25\textwidth}
				\begin{alertblock}{DLSR}
					\begin{figure}
						\centering
						\subfloat{\includegraphics[width=0.75\textwidth]{figure10e.png}}
						\vfill			
						\subfloat{\includegraphics[width=0.75\textwidth]{figure11e.png}}\quad
					\end{figure}
				\end{alertblock}				
			%%%%%%%%%%%%%%%%%%%%%%%%%%%%%%%%%%%%%%%%%%%%%%%%%%%%%%%%%%%%
			\end{column}				
		\end{columns}
	\end{frame}
	%%%%%%%%%%%%%%%%%%%%%%%%%%%%%%%%%%%%%%%%%%%%%%%%%%%%%%%%%%%%%%%%%%%%
	\note{
		\tiny
		In this slide, I present an experimental test case,
%		, in which the Low-resolution input data was registered by an SLDV at a uniform gird of \(32\times 32\) points.
%		The LR input frame corresponds to the initial interaction with the delamination at frame number 110.
		
		Additionally, the conventional compressive sensing technique was applied as a reference with a various number of points used for the reconstruction of the wavefield. 
		
%		Moreover, two types of masks (sub-sampling schemes) for the construction of measurement matrix were tested, namely random mask and jitter mask.
		
		Figure a shows the actual HR reference frame, and figures c, d, and e show reconstructed HR frames with the CS method.
%		at \(N_p = 1024\). 
		
		As expected, when decreasing the \(N_p\) measuring points, the CS method is not able to recover HR frames accurately. 
		
		Figure f presents the reconstructed HR frame with the DLSR model. 
		
		Furthermore, figure g presents the reference region of interest we are attempting to recover, which shows reflected waves from damage.
		
%		Figures  show a poor reconstruction of the reflected waves at \(N_p = 1024\) using the CS method.
		
		Figure k shows the recovery of the reflected waves using the DLSR model.
		The recovered reflected waves from damage is visible, and can be easily recognised, which indicates that the DLSR approach can be more efficient than the conventional CS method.		
	}
	%%%%%%%%%%%%%%%%%%%%%%%%%%%%%%%%%%%%%%%%%%%%%%%%%%%%%%%%%%%%%%%%%%%%
	\begin{frame}{Analysis of experimental case}
		\begin{table}[!ht]
			\renewcommand{\arraystretch}{1.3}
			\centering \footnotesize
			\caption{Quality metrics for tested methods.}	
			\begin{tabular}{lrrrcrc} 
				\toprule[1.5pt]
				& & & \multicolumn{2}{c}{plate} & \multicolumn{2}{c}{delamination} \\
				\cmidrule(lr){4-5} \cmidrule(lr){6-7}
				Method & $N_p$ & CR [\%] & PSNR & PEARSON CC& PSNR & PEARSON CC \\
				\midrule
				\csvreader
				[table head=\toprule,
				late after line=\\ 
				]{table_metrics.csv}{
					1=\one, 2=\two, 3=\three, 4=\four, 5=\five, 6=\six, 7=\seven
				}%
				{\one & \two & \three & \four & \five & \six & \seven }%	
				\bottomrule[1.5pt]
			\end{tabular}	
			\label{tab:csv_results_}
		\end{table}
	\end{frame}
	%%%%%%%%%%%%%%%%%%%%%%%%%%%%%%%%%%%%%%%%%%%%%%%%%%%%%%%%%%%%%%%%%%%%
	\note{
		The following table presents a detailed comparison of the quality metrics for CS methods with applied jitter and random masks and the DLSR model for various numbers of points and the corresponding compression ratios CR. 
		Metrics were calculated for frame number 110 on the whole plate and delamination region, respectively. 
		It should be underlined, that for the number of points = 1024 points, the CS recovery algorithm behaves poorly at the delamination region, whereas the DLSR model is still able to achieve high PSNR and Pearson CC values.	
		
	}
	%%%%%%%%%%%%%%%%%%%%%%%%%%%%%%%%%%%%%%%%%%%%%%%%%%%%%%%%%%%%%%%%
	\section{Conclusions and future works}
	%%%%%%%%%%%%%%%%%%%%%%%%%%%%%%%%%%%%%%%%%%%%%%%%%%%%%%%%%%%%%%%%
	\begin{frame}{Conclusions}		
		\begin{footnotesize}
			\begin{justify}
				\settowidth{\leftmargini}{\usebeamertemplate{itemize item}}
				\addtolength{\leftmargini}{\labelsep}
				\begin{itemize}
					\item{Deep learning models trained on synthetic dataset generalise well and can be applied directly to experimental wavefields.}				
					\item{Animation-based deep learning models perform better than image-based models but are more complex and require longer time for training.}
					\item{Deep learning approaches surpass conventional approaches for delamination identification.}
					\item{Deep learning models can be generalised to other types of defects and materials.}
					\item{The DLSR model can recover the HR measurements from LR measurements with high accuracy.}
				\end{itemize}
			\end{justify}				
		\begin{tcolorbox}
			\begin{justify}
				\textbf{According to my research study, investigations and results, I can confirm my original thesis:}
				\begin{alertblock}{Thesis}
					It is possible to use an end-to-end approach in which DNN 
					processes the animation of propagating waves (input) directly into a damage map (output).
				\end{alertblock}
%				\begin{alertblock}{Thesis: 2}
%					It is possible to speed up the data acquisition process by SLDV by employing an appropriate DL approach to recover high-resolution (HR) full wavefield scans from a low-resolution (LR).
%				\end{alertblock}
			\end{justify}			
		\end{tcolorbox}				
		\end{footnotesize}			
	\end{frame}		
	%%%%%%%%%%%%%%%%%%%%%%%%%%%%%%%%%%%%%%%%%%%%%%%%%%%%%%%%%%%%%%%%
	\begin{frame}{Future works}
		\begin{columns}[T]
			\begin{column}[t]{0.49\textwidth}
				\centering
				\textbf{\underline{Metamaterials: Dispersion diagrams}}
				\begin{figure}
					\centering
					\efbox{\includegraphics[width=1\textwidth]{Surrogate_DL_model_for_PC_Abdalraheem.png}}
				\end{figure}
			\end{column}
			\begin{column}[t]{0.49\textwidth}
				\centering
				\textbf{\underline{COMSOL vs. surrogate DL model predictions}}
				\begin{figure}
					\centering
					\efbox{\includegraphics[width=1\textwidth]{plot_1029_4672_10425_KF_DL_FEM_BG_triple_tile.png}}							
				\end{figure}			
			\end{column}		
		\end{columns}				
	\end{frame}		
	%%%%%%%%%%%%%%%%%%%%%%%%%%%%%%%%%%%%%%%%%%%%%%%%%%%%%%%%%%%%%%%%
	\begin{frame}{Publications}
		\vspace{5pt}
		\settowidth{\leftmargini}{\usebeamertemplate{itemize item}}
		\addtolength{\leftmargini}{\labelsep}
		\begin{tiny}					
			\begin{columns}[T]
				\begin{column}[t]{0.48\textwidth}
					\underline{\textbf{Journal articles}}
					\begin{enumerate}
						\justifying
						%%%%%%%%%%%%%%%%%%%%%%%%%%%%%%%%%%%%%%%%%%%%%%%%
						\item Ijjeh, A., Ullah, S., Radzienski, M. and Kudela, P., 2023. Deep learning super-resolution for the reconstruction of full wavefield of Lamb waves. \textbf{\textit{Mechanical Systems and Signal Processing}}, 186, p.109878.						
						\textbf{[200~points]/[IF:8.934]}
						%%%%%%%%%%%%%%%%%%%%%%%%%%%%%%%%%%%%%%%%%%%%%%%%
						\item Ullah, S., Ijjeh, A.A. and Kudela, P., 2023. Deep learning approach for delamination identification using animation of Lamb waves. 						
						\textbf{\textit{Engineering Applications of Artificial Intelligence}}, 117, p.105520.		
						\textbf{[140~points]/[IF:7.802]}
						%%%%%%%%%%%%%%%%%%%%%%%%%%%%%%%%%%%%%%%%%%%%%%%%
						\item Ijjeh, A.A. and Kudela, P., 2022. Deep learning based segmentation using full wavefield processing for delamination identification: A comparative study. \textbf{\textit{Mechanical Systems and Signal Processing}}, 168, p.108671. \textbf{[200~points]/[IF:8.934]}
						%%%%%%%%%%%%%%%%%%%%%%%%%%%%%%%%%%%%%%%%%%%%%%%%
						\item Ijjeh, A.A., Ullah, S. and Kudela, P., 2021. Full wavefield processing by using FCN for delamination detection. \textbf{\textit{Mechanical Systems and Signal Processing}}, 153, p.107537.		
						\textbf{[200~points]/[IF:8.934]}	
						%%%%%%%%%%%%%%%%%%%%%%%%%%%%%%%%%%%%%%%%%%%%%%%%
					\end{enumerate}					
				\end{column}
				\begin{column}[t]{0.48\textwidth}
					\underline{\textbf{Conference papers}}
					\begin{enumerate}
						\justifying
						\item {Ijjeh, A.}, Kudela, P. Convolutional LSTM for delamination imaging in composite laminates. 
						The 4th International Conference on Machine Learning and Intelligent Systems (MLIS 2022), November \(8^{th}\) - \(11^{th}\), 2022, Seoul, Republic of Korea.
						\item Ijjeh, A. and Kudela, P., 2022, June. Delamination Identification Using Global Convolution Networks. 
						In European Workshop on Structural Health Monitoring: EWSHM 2022-Volume 3 (pp. 521-529). Cham: Springer International Publishing.		
						\item {Ijjeh, A.}, Kudela, P. Feasibility Study of Full Wavefield Processing by Using CNN for Delamination Detection. 
						Proceedings of the International Conference on Structural Health Monitoring of Intelligent
						Infrastructure, June \(30^{th}\) - July \(2^{nd}\), 2021, Porto, Portugal, ISSN 2564-3738, pages 709-713.
					\end{enumerate}		
					\underline{\textbf{Chapters}}					
					\begin{enumerate}
						\justifying
						\item {Abdalraheem Ijjeh}, Deep Learning based Damage Imaging techniques, chapter in: Wybrane zagadnienia
						inżynierii mechanicznej, Praca zbiorowa pod redakcja M.~Mieloszyk, T. Ochrymiuka, Wydawnictwo Instytutu
						Maszyn Przepływowych PAN, Gdańsk, 2022, ISBN: 978-83-66928-09-1.
						\item {Abdalraheem Ijjeh}, Data-driven based approach for damage detection, chapter in: Wybrane zagadnienia
						inżynierii mechanicznej, Praca zbiorowa pod redakcja M.~Mieloszyk, T. Ochrymiuka, Wydawnictwo Instytutu
						Maszyn Przepływowych PAN, Gdańsk, 2021, ISBN: 978-83-66928-00-8.				
						\item {Abdalraheem Ijjeh}, Machine Learning for SHM: Literature Review, chapter in: Wybrane zagadnienia
						inżynierii mechanicznej, Praca zbiorowa pod redakcja M.~Mieloszyk, T. Ochrymiuka, Wydawnictwo Instytutu
						Maszyn Przepływowych PAN, Gdańsk, 2020, ISBN: 978-83-88237-97-3.
					\end{enumerate}
				\end{column}		
			\end{columns}
		\end{tiny}
	\end{frame}	
	%%%%%%%%%%%%%%%%%%%%%%%%%%%%%%%%%%%%%%%%%%%%%%%%%%%%%%%%%%%%%%%%%%%%
	\setcounter{subfigure}{0}
	%%%%%%%%%%%%%%%%%%%%%%%%%%%%%%%%%%%%%%%%%%%%%%%%%%%%%%%%%%%%%%%%%%%%
	{
		\setbeamercolor{palette primary}{fg=blue, bg=white}
		\begin{frame}[standout]
			Thank you for your listening!\\ \vspace{12pt}
			Questions?\\ \vspace{12pt}
			\url{aijjeh@imp.gda.pl}
			\par\medskip
			\par\medskip
			\footnotesize
			The research work was funded by the Polish National Science Center under grant agreement no. 2018/31/B/ST8/00454.
		\end{frame}
	}
%	%%%%%%%%%%%%%%%%%%%%%%%%%%%%%%%%%%%%%%%%%%%%%%%%%
	\end{document}