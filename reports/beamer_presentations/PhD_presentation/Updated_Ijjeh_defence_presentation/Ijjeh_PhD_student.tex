\documentclass[12pt,a4paper]{article}
\usepackage[dvipsnames]{xcolor}
%\usepackage{dirtytalk}
\usepackage{graphicx}
\usepackage{multirow}
\usepackage{amsmath,amssymb,bm}
%\usepackage[dvips,colorlinks=true,citecolor=green]{hyperref}
\usepackage[colorlinks=true,citecolor=green]{hyperref}
%% my added packages
\usepackage{float}
\usepackage{csquotes}
\usepackage{verbatim}
\usepackage{caption}
\usepackage{subcaption}
\usepackage{booktabs} % for nice tables
\usepackage{csvsimple} % for csv read
\usepackage{graphicx}
\usepackage{geometry}
%\usepackage{showframe} %This line can be used to clearly show the new margins
\usepackage[document]{ragged2e}
\newgeometry{vmargin={20mm}, hmargin={23mm,25mm}}
%\usepackage[outdir=//odroid-sensors/sensors/aidd/reports/journal_papers/MSSP_Paper/Figures/]{epstopdf}
%\usepackage{breqn}
\usepackage{multirow}
%\setbeameroption{show notes on second screen}
%\setbeamertemplate{note page}{\insertnote}
%%%%%%%%%%%%%%%%%%%%%%%%%%%%%%%%%%%%%%%%%%%%%%%%%%
% Title page options
%%%%%%%%%%%%%%%%%%%%%%%%%%%%%%%%%%%%%%%%%%%%%%%%%%
\date{February 10, 2023}
%\date{}
%%%%%%%%%%%%%%%%%%%%%%%%%%%%%%%%%%%%%%%%%%%%%%%%%%
% option 1
%%%%%%%%%%%%%%%%%%%%%%%%%%%%%%%%%%%%%%%%%%%%%%%%%%%%
\title{FEASIBILITY STUDY OF ARTIFICIAL INTELLIGENCE APPROACH FOR DELAMINATION IDENTIFICATION IN COMPOSITE LAMINATES}
%%\subtitle{In preparation for a Ph.D. defence}
\author{Ph.D. candidate: Eng. Abdalraheem A. Ijjeh}
%	\and \\ 
%	\textbf{Supervisor: D.Sc. Ph.D. Eng. Paweł Kudela}} 
%% logo align to Institute 
%\institute{Institute of Fluid Flow Machinery \\ 
%	Polish Academy of Sciences \\ 
%	Gdansk, Poland \\
%	\vspace{-1.5cm}
%	\flushright 
%	\includegraphics[width=6cm]{imp_logo.png}}
%%%%%%%%%%%%%%%%%%%%%%%%%%%%%%%%%%%%%%%%%%%%%%%%%%
%\tikzexternalize % activate!
%%%%%%%%%%%%%%%%%%%%%%%%%%%%%%%%%%%%%%%%%%%%%%%%%%%
%\setbeamertemplate{section in toc}[sections numbered]
%\setbeamertemplate{subsection in toc}[subsections numbered]
\begin{document}
%	\maketitle
	\justifying
	\noindent
	\footnotesize
	\paragraph{Abdalraheem Ijjeh (pronunciation Idżej)}was born in Jordan in 1988, received his bachelor's degree in engineering there in 2010, and then graduated with a very good master's grade from the Jordan University of Science and Technology in Irbid in 2014.
	He graduated with a master's degree in computer engineering. 
	He then worked for five years in Saudi Arabia as a lecturer at a national institute for training technicians in various technological fields.
	During this period, he published two scientific papers (Journal of Communications and International Journal of Computer Networks and Communications) as a co-author.
	
	\noindent
	\paragraph{He started his scientific work} at the institute in December 2019 as a fellow in an Opus project funded by the National Science Center entitled Feasibility studies of diagnostics based on artificial intelligence. 
	The project manager is Dr. Paweł Kudela, professor of IMP PAN, who at the same time became the doctoral student's supervisor due to the doctoral student's admission to the Tri-City Doctoral School (he is one of the first doctoral students in the new path outlined by Law 2.0).
	Abdalraheem Ijjeh has co-authored three articles published in Mechanical Systems and Signal Processing, which is a leading journal in the area of structural health monitoring (200 points).
	\\ \\
	\textbf{He} is listed as first author in these articles:
	\settowidth{\leftmargini}{{itemize item}}
	\addtolength{\leftmargini}{\labelsep}
	\begin{enumerate}
		\justifying
		%%%%%%%%%%%%%%%%%%%%%%%%%%%%%%%%%%%%%%%%%%%%%%%%
		\item Ijjeh, A., Ullah, S., Radzienski, M. and Kudela, P., 2023. Deep learning super-resolution for the reconstruction of full wavefield of Lamb waves. \textbf{\textit{Mechanical Systems and Signal Processing}}, 186, p.109878.						
		\textbf{[200~points]/[IF:8.934]}		
		%%%%%%%%%%%%%%%%%%%%%%%%%%%%%%%%%%%%%%%%%%%%%%%%
		\item Ijjeh, A.A. and Kudela, P., 2022. Deep learning based segmentation using full wavefield processing for delamination identification: A comparative study. \textbf{\textit{Mechanical Systems and Signal Processing}}, 168, p.108671. \textbf{[200~points]/[IF:8.934]}
		%%%%%%%%%%%%%%%%%%%%%%%%%%%%%%%%%%%%%%%%%%%%%%%%
		\item Ijjeh, A.A., Ullah, S. and Kudela, P., 2021. Full wavefield processing by using FCN for delamination detection. \textbf{\textit{Mechanical Systems and Signal Processing}}, 153, p.107537.		
		\textbf{[200~points]/[IF:8.934]}	
		%%%%%%%%%%%%%%%%%%%%%%%%%%%%%%%%%%%%%%%%%%%%%%%%
	\end{enumerate}		
	\textbf{Additionally,} he co-authored one article in journal of Engineering Applications of Artificial Intelligence. 
	\begin{enumerate}
		%%%%%%%%%%%%%%%%%%%%%%%%%%%%%%%%%%%%%%%%%%%%%%%%
		\item Ullah, S., Ijjeh, A.A. and Kudela, P., 2023. Deep learning approach for delamination identification using animation of Lamb waves. 						
		\textbf{\textit{Engineering Applications of Artificial Intelligence}}, 117, p.105520.		
		\textbf{[140~points]/[IF:7.802]}
	\end{enumerate}
		
	\noindent	
	He presented his results at three international conferences \textbf{(SHMII, EWSHM, MLIS)}.
	
	
%	\begin{enumerate}
%		\justifying
%		\item {Ijjeh, A.}, Kudela, P. Convolutional LSTM for delamination imaging in composite laminates. 
%		The 4th International Conference on Machine Learning and Intelligent Systems (MLIS 2022), November \(8^{th}\) - \(11^{th}\), 2022, Seoul, Republic of Korea.
%		\item Ijjeh, A. and Kudela, P., 2022, June. Delamination Identification Using Global Convolution Networks. 
%		In European Workshop on Structural Health Monitoring: EWSHM 2022-Volume 3 (pp. 521-529). Cham: Springer International Publishing.		
%		\item {Ijjeh, A.}, Kudela, P. Feasibility Study of Full Wavefield Processing by Using CNN for Delamination Detection. 
%		Proceedings of the International Conference on Structural Health Monitoring of Intelligent
%		Infrastructure, June \(30^{th}\) - July \(2^{nd}\), 2021, Porto, Portugal, ISSN 2564-3738, pages 709-713.
%	\end{enumerate}		
	\noindent
	In addition, he is the author of three chapters in collective works published by the Institute of Flow Machinery of the Polish Academy of Sciences, which are related to the Tri-City Doctoral School.
%	\begin{enumerate}
%		\justifying
%		\item {Abdalraheem Ijjeh}, Deep Learning based Damage Imaging techniques, chapter in: Wybrane zagadnienia
%		inżynierii mechanicznej, Praca zbiorowa pod redakcja M.~Mieloszyk, T. Ochrymiuka, Wydawnictwo Instytutu
%		Maszyn Przepływowych PAN, Gdańsk, 2022, ISBN: 978-83-66928-09-1.
%		\item {Abdalraheem Ijjeh}, Data-driven based approach for damage detection, chapter in: Wybrane zagadnienia
%		inżynierii mechanicznej, Praca zbiorowa pod redakcja M.~Mieloszyk, T. Ochrymiuka, Wydawnictwo Instytutu
%		Maszyn Przepływowych PAN, Gdańsk, 2021, ISBN: 978-83-66928-00-8.				
%		\item {Abdalraheem Ijjeh}, Machine Learning for SHM: Literature Review, chapter in: Wybrane zagadnienia
%		inżynierii mechanicznej, Praca zbiorowa pod redakcja M.~Mieloszyk, T. Ochrymiuka, Wydawnictwo Instytutu
%		Maszyn Przepływowych PAN, Gdańsk, 2020, ISBN: 978-83-88237-97-3.
%	\end{enumerate}
	
	\noindent
	\paragraph{In June of this year,}the doctoral student completed a short, one-week research internship at the CNRS, University of Lille in France under the Erasmus+ program.
	\noindent
	\paragraph{Additionally}, he was awarded third place in the \textbf{Image Classification Challenge} at the International Summer School on Deep Learning, Gdansk University of Technology, on July 14, 2021.

	\noindent	
	\paragraph{On July 12,}with the positive opinion of the supervisor, the doctoral student submitted the dissertation with complete documents. 
	The topic of the dissertation is related to the development of non-destructive methods for the identification of damage, particularly delamination, in composite laminates. 
	The doctoral student posed a thesis that \textbf{it is possible to use deep neural networks in the so-called end-to-end approach, in which input data in the form of animations of propagating Lamb waves are processed directly into a damage map.}
	In addition, the doctoral student set himself the goal of reducing the measurement time of the full wavefield with a scanning Doppler laser vibrometer by using deep learning and super resolution methods.	
\end{document}