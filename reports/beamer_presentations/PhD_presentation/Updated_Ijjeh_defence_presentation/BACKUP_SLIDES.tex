%\PassOptionsToPackage{draft}{graphicx}
\documentclass[10pt,aspectratio=169,dvipsnames]{beamer} % aspect ratio 16:9
%\graphicspath{{../../figures/}}

%\includeonlyframes{frame1,frame2,frame3}

%%%%%%%%%%%%%%%%%%%%%%%%%%%%%%%%%%%%%%%%%%%%%%%%%%
% Packages
%%%%%%%%%%%%%%%%%%%%%%%%%%%%%%%%%%%%%%%%%%%%%%%%%%
\usepackage{appendixnumberbeamer}
\usepackage{booktabs}
\usepackage{csvsimple} % for csv read
\usepackage[scale=2]{ccicons}
\usepackage{pgfplots}
\usepackage{xspace}
%\usepackage{amsmath}
\usepackage{totcount}
\usepackage{tikz}
\usepackage{bm}
\usepackage{float}
\usepackage{eso-pic} 
\usepackage{wrapfig}
\usepackage{animate,media9}
\usepackage{subfig}
\usepackage{fancybox}
%\usepackage{multimedia}
\usepackage{dashbox}
\usepackage{tcolorbox}
\usepackage{multicol}
\usepackage{multirow}
\usepackage{xcolor}
\usepackage[document]{ragged2e}
\usepackage{caption}
\usepackage{comment}
\usepackage{mathtools}% Loads amsmath

%\usepackage[export]{adjustbox}
%\usepackage{background}
%\backgroundsetup{contents=preliminary,placement=bottom,color=blue}
%\usepackage{FiraSans}

%\usepackage{comment}
%\usetikzlibrary{external} % speedup compilation
%\tikzexternalize % activate!
%\usetikzlibrary{shapes,arrows} 

%\usepackage{bibentry}
%\nobibliography*
\usepackage{ifthen}
\newcounter{angle}
\setcounter{angle}{0}
%\usepackage{bibentry}
%\nobibliography*
\usepackage{caption}%

\graphicspath{{figures/}}

\captionsetup[figure]{labelformat=empty}%
\usefonttheme{structurebold}
%%%%%%%%%%%%%%%%%%%%%%%%%%%%%%%%%%%%%%%%%%%%%%%%%%
% Metropolis theme custom modification file
%%%%%%%%%%%%%%%%%%%%%%%%%%%%%%%%%%%%%%%%%%%%%%%%%%
% Metropolis theme custom modification file
%%%%%%%%%%%%%%%%%%%%%%%%%%%%%%%%%%%%%%%%%%%%%%%%%%
% Metropolis theme custom colors
%%%%%%%%%%%%%%%%%%%%%%%%%%%%%%%%%%%%%%%%%%%%%%%%%%
\usetheme[progressbar=foot]{metropolis}
\useoutertheme{metropolis}
\useinnertheme{metropolis}
\usefonttheme{metropolis}
\setbeamercolor{background canvas}{bg=white}

%\usecolortheme{spruce}

\definecolor{myblue}{rgb}{0.19,0.55,0.91}
\definecolor{mediumblue}{rgb}{0,0,205}
\definecolor{darkblue}{rgb}{0,0,139}
\definecolor{Dodgerblue}{HTML}{1E90FF}
\definecolor{Navy}{HTML}{000080} % {rgb}{0,0,128}
\definecolor{Aliceblue}{HTML}{F0F8FF}
\definecolor{Lightskyblue}{HTML}{87CEFA}
\definecolor{logoblue}{RGB}{1,67,140}
\definecolor{Purple}{HTML}{911146}
\definecolor{Orange}{HTML}{CF4A30}

\setbeamercolor{progress bar}{bg=Lightskyblue}
\setbeamercolor{progress bar}{ fg=logoblue} 
\setbeamercolor{frametitle}{bg=logoblue}
\setbeamercolor{title separator}{fg=logoblue}
\setbeamercolor{block title}{bg=Lightskyblue!30,fg=black}
\setbeamercolor{block body}{bg=Lightskyblue!15,fg=black}
\setbeamercolor{alerted text}{fg=Purple}
% notes colors
\setbeamercolor{note page}{bg=white}
\setbeamercolor{note title}{bg=Lightskyblue}
%%%%%%%%%%%%%%%%%%%%%%%%%%%%%%%%%%%%%%%%%%%%%%%%%%
%  Theme modifications
%%%%%%%%%%%%%%%%%%%%%%%%%%%%%%%%%%%%%%%%%%%%%%%%%%
% modify progress bar linewidth
\makeatletter
\setlength{\metropolis@progressinheadfoot@linewidth}{2pt} 
\setlength{\metropolis@titleseparator@linewidth}{1pt}
\setlength{\metropolis@progressonsectionpage@linewidth}{1pt}

\setbeamertemplate{progress bar in section page}{
	\setlength{\metropolis@progressonsectionpage}{%
		\textwidth * \ratio{\thesection pt}{\totvalue{totalsection} pt}%
	}%
	\begin{tikzpicture}
		\fill[bg] (0,0) rectangle (\textwidth, 
		\metropolis@progressonsectionpage@linewidth);
		\fill[fg] (0,0) rectangle (\metropolis@progressonsectionpage, 
		\metropolis@progressonsectionpage@linewidth);
	\end{tikzpicture}%
}
\makeatother
\newcounter{totalsection}
\regtotcounter{totalsection}

\AtBeginDocument{%
	\pretocmd{\section}{\refstepcounter{totalsection}}{\typeout{Yes, prepending 
	was successful}}{\typeout{No, prepending was not successful}}%
}%
%%%%%%%%%%%%%%%%%%%%%%%%%%%%%%%%%%%%%%%%%%%%%%%%%%
%  Bibliography mods
%%%%%%%%%%%%%%%%%%%%%%%%%%%%%%%%%%%%%%%%%%%%%%%%%%
\setbeamertemplate{bibliography item}{\insertbiblabel} %% Remove book symbol 
%%from references and add number in square brackets
% kill the abominable icon (without number)
%\setbeamertemplate{bibliography item}{}
%\makeatletter
%\renewcommand\@biblabel[1]{#1.} % number only
%\makeatother
% remove line breaks in bibliography
\setbeamertemplate{bibliography entry title}{}
\setbeamertemplate{bibliography entry location}{}
%%%%%%%%%%%%%%%%%%%%%%%%%%%%%%%%%%%%%%%%%%%%%%%%%%
%  Bibliography custom commands
%%%%%%%%%%%%%%%%%%%%%%%%%%%%%%%%%%%%%%%%%%%%%%%%%%
\newcommand{\bibliotitlestyle}[1]{\textbf{#1}\par}

\newif\ifinbiblio
\newcounter{bibkey}
\newenvironment{biblio}[2][long]{%
	%\setbeamertemplate{bibliography item}{\insertbiblabel}
	\setbeamertemplate{bibliography item}{}% without numbers
	\setbeamerfont{bibliography item}{size=\footnotesize}
	\setbeamerfont{bibliography entry author}{size=\footnotesize}
	\setbeamerfont{bibliography entry title}{size=\footnotesize}
	\setbeamerfont{bibliography entry location}{size=\footnotesize}
	\setbeamerfont{bibliography entry note}{size=\footnotesize}
	\ifx!#2!\else%
	\bibliotitlestyle{#2}%
	\fi%
	\begin{thebibliography}{}%
		\inbibliotrue%
		\setbeamertemplate{bibliography entry title}[#1]%
	}{%
		\inbibliofalse%
		\setbeamertemplate{bibliography item}{}%
	\end{thebibliography}%
}

\newcommand{\biblioref}[5][short]{
	\setbeamertemplate{bibliography entry title}[#1]
	\stepcounter{bibkey}%
	\ifinbiblio%
	\bibitem{\thebibkey}%
	#2
	\newblock #4
	\ifx!#5!\else\newblock {\em #5}, #3 \fi%
	\else%
	\begin{biblio}{}
		\bibitem{\thebibkey}
		#2
		\newblock #4
		\ifx!#5!\else\newblock {\em #5}, #3\fi
	\end{biblio}
	\fi
}
%
%\newbibmacro*{hypercite}{%
%	\renewcommand{\@makefntext}[1]{\noindent\normalfont##1}%
%	\footnotetext{%
%		\blxmkbibnote{foot}{%
%			\printtext[labelnumberwidth]{%
%				\printfield{prefixnumber}%
%				\printfield{labelnumber}}%
%			\addspace
%			\fullcite{\thefield{entrykey}}}}}
%
%\DeclareCiteCommand{\hypercite}%
%{\usebibmacro{cite:init}}
%{\usebibmacro{hypercite}}
%{}
%{\usebibmacro{cite:dump}}
%
%% Redefine the \footfullcite command to use the reference number
%\renewcommand{\footfullcite}[1]{\cite{#1}\hypercite{#1}}
%\usefonttheme[onlymath]{Serif} % It should be uncommented if Fira fonts in 
%%math does not work

%%%%%%%%%%%%%%%%%%%%%%%%%%%%%%%%%%%%%%%%%%%%%%%%%%
% Custom commands
%%%%%%%%%%%%%%%%%%%%%%%%%%%%%%%%%%%%%%%%%%%%%%%%%%
% matrix command 
\newcommand{\matr}[1]{\mathbf{#1}} % bold upright (Elsevier, Springer)
%\newcommand{\matr}[1]{#1}   % pure math version
%\newcommand{\matr}[1]{\bm{#1}}  % ISO complying version
% vector command 
\newcommand{\vect}[1]{\mathbf{#1}} % bold upright (Elsevier, Springer)
% bold symbol
\newcommand{\bs}[1]{\boldsymbol{#1}}
% derivative upright command
\DeclareRobustCommand*{\drv}{\mathop{}\!\mathrm{d}}
\newcommand{\ud}{\mathrm{d}}
% 
\newcommand{\themename}{\textbf{\textsc{metropolis}}\xspace}

%\usepackage{pgfpages}
%\setbeameroption{show notes}
%\setbeameroption{show notes on second screen=left}
%\setbeamertemplate{note page}{\insertnote}
%%%%%%%%%%%%%%%%%%%%%%%%%%%%%%%%%%%%%%%%%%%%%%%%%%
% Title page options
%%%%%%%%%%%%%%%%%%%%%%%%%%%%%%%%%%%%%%%%%%%%%%%%%%
% \date{\today}
\date{}
%%%%%%%%%%%%%%%%%%%%%%%%%%%%%%%%%%%%%%%%%%%%%%%%%%
% option 1
%%%%%%%%%%%%%%%%%%%%%%%%%%%%%%%%%%%%%%%%%%%%%%%%%%%
\title{FEASIBILITY STUDY OF ARTIFICIAL INTELLIGENCE APPROACH FOR DELAMINATION IDENTIFICATION IN COMPOSITE LAMINATES}
%\subtitle{In preparation for a Ph.D. defence}
\author{\textbf{Ph.D. candidate, Eng. Abdalraheem A. Ijjeh } 
	\and \\ 
	\textbf{Supervisor: D.Sc. Ph.D. Eng. Paweł Kudela}
} 
% logo align to Institute 
\institute{Institute of Fluid Flow Machinery \\ 
	Polish Academy of Sciences \\ 
	\vspace{-1.5cm}
	\flushright 
	\includegraphics[width=6cm]{imp_logo.png}}

%%%%%%%%%%%%%%%%%%%%%%%%%%%%%%%%%%%%%%%%%%%%%%%%%%
%\tikzexternalize % activate!
%%%%%%%%%%%%%%%%%%%%%%%%%%%%%%%%%%%%%%%%%%%%%%%%%%
\setbeamertemplate{section in toc}[sections numbered]
\setbeamertemplate{subsection in toc}[subsections numbered]

\begin{document}
	%%%%%%%%%%%%%%%%%%%%%%%%%%%%%%%%%%%%%%%%%%%%%%%%%%
	\maketitle
	%%%%%%%%%%%%%%%%%%%%%%%%%%%%%%%%%%%%%%%%%%%%%%%%%%%%%%%%%%%%%%%%%%%%%%%%%%%%
	\note{
		My name is Abdalraheem Ijjeh.
		I am a PhD candidate at the Institute of Fluid Flow Machinery, Polish Academy of Sciences.		
		Supervised by Professor Pawe{l} Kudela.		
		The title of my PhD thesis is FEASIBILITY STUDY OF ARTIFICIAL INTELLIGENCE APPROACH FOR DELAMINATION IDENTIFICATION IN COMPOSITE LAMINATES.		
		Thank you all for you attending my PhD defence presentation. 
	}
	%%%%%%%%%%%%%%%%%%%%%%%%%%%%%%%%%%%%%%%%%%%%%%%%%%%%%%%%%%%%%%%%%%%%%%%%%%%%
	%%%%%%%%%%%%%%%%%%%%%%%%%%%%%%%%%%%%%%%%%%%%%%%%%%
	% SLIDES
	%%%%%%%%%%%%%%%%%%%%%%%%%%%%%%%%%%%%%%%%%%%%%%%%%%
	\begin{frame}[label=frame1]{Outlines}
		\begin{multicols}{2}
			%		\fontsize{6pt}{8pt}\selectfont
			\setbeamertemplate{section in toc}[sections numbered]
			\setbeamertemplate{subsection in toc}[subsections numbered]
			\tableofcontents
		\end{multicols}
	\end{frame}	
	%%%%%%%%%%%%%%%%%%%%%%%%%%%%%%%%%%%%%%%%%%%%%%%%%%%%%%%%%%%%%%%%%%%%%%%%%%%%
	\note{The presentation will be as follows: 
		In the first section, I will briefly talk about the motivations behind this research.	
		Then I will state my objectives in the second section. 
		In sections three and four, I will talk briefly about nondestructive testing methods, and guided waves. 
		In section five, I will present the conventional approach v.s. the deep learning approach for damage detection. 
		Section six will be an introduction to the supervised deep learning approach and how it works in general. 
		The generation of the synthetic dataset used for training will be presented in section seven. 
		In section 8, I will present the developed deep-learning models for damage identification. 
		The evaluation of the developed models will be presented in sections nine and ten.
		The super-resolution deep learning model developed for fast data acquisition will be presented in section 11. 
		Finally, the conclusions will be presented in section 12.
	}
	%%%%%%%%%%%%%%%%%%%%%%%%%%%%%%%%%%%%%%%%%%%%%%%%%%%%%%%%%%%%%%%%%%%%%%%%%%%%
	\section{Work novelty}
	\begin{frame}{Motivations}
		\begin{columns}[T]
			\begin{column}[t]{.5\textwidth}
				\begin{figure}[c]
					\centering
					\includegraphics[width=0.85\textwidth]{Composite_advantages.png}					
				\end{figure}			
				\begin{figure}[c]
					\includegraphics[width=0.7\textwidth]{delaminated_plate1.jpg}
				\end{figure}
				\begin{small}
					\noindent \alert{Delamination detection} in its early stages can significantly help avoiding catastrophic structural collapses.
				\end{small}				
			\end{column}
			\begin{column}[t]{0.5\textwidth}
				\begin{figure}[c]
					\includegraphics[width=0.85\textwidth]{Crashes.png}
					%					\caption{Catastrophic failures}
				\end{figure}
			\end{column}
		\end{columns}
	\end{frame}
	%%%%%%%%%%%%%%%%%%%%%%%%%%%%%%%%%%%%%%%%%%%%%%%%%%%%%%%%%%%%%%%%%%%%%%%%%%%%
	\note{
		Composite laminates have a wide range of applications in various industries due to their characteristics, such as high strength, low density, and resistance to fatigue and corrosion, among others.
		
		However, defects such as cracks, fibre breakage, and debonding can occur in composite laminates. 
		
		In particular, laminated composite materials are more sensitive to damage in the form of delamination due to weak transverse tensile and interlaminar shear strengths.
		
		Delaminations can seriously decrease the performance of composite structures.
		Accordingly, delamination detection in its early stages can significantly help to avoid catastrophic structural collapses.		
	}
	%%%%%%%%%%%%%%%%%%%%%%%%%%%%%%%%%%%%%%%%%%%%%%%%%%%%%%%%%%%%%%%%%%%%%%%%%%%%
	%	\subsection{Objectives (1)}
	\begin{frame}{Objectives (1)}
		\textbf{To develop \textcolor{logoblue}{a novel AI-driven diagnostic system} for delamination identification in composite laminates such as carbon fibre reinforced polymers (CFRP).}
		\vfil
		\textbf{To address the issue of \textcolor{logoblue}{slow data acquisition} by SLDV of high-resolution full wavefields of Lamb wave propagation.}
		\begin{alertblock}{Thesis}
			It is possible to use an end-to-end approach in which DNN 
			processes the animation of propagating waves (input) directly into a damage map (output).
		\end{alertblock}
	\end{frame}
	%%%%%%%%%%%%%%%%%%%%%%%%%%%%%%%%%%%%%%%%%%%%%%%%%%%%%%%%%%%%%%%%%%%%%%%%%%%%
	\note{
		The main objective of this work is to develop an artificial intelligence-based diagnostic system for the detection of delaminations in composite laminates.
		
		Therefore, I investigated the possibility of embracing the end-to-end approach by using animations of Lamb wave propagation in a plate interacting with discontinuities such as damage and edges with an artificial intelligence-based approach to detect and identify delaminations.
		
		My second objective is to address the issue of slow data acquisition of high-resolution full wavefields of Lamb wave propagation by SLDV. 		
	}
	
	%%%%%%%%%%%%%%%%%%%%%%%%%%%%%%%%%%%%%%%%%%%%%%%%%%%%%%%%%%%%%%%%%%%%%%%%%%%%
	%	\subsection{Objective (2)}
	\begin{frame}{Objective (2)}
		\begin{columns}[T]
			\begin{column}[c]{0.95\textwidth}
				\begin{figure}
					\centering
					\includegraphics[height=.85\textheight]{full_procedure.png}	
				\end{figure}		
			\end{column}
		\end{columns}		
	\end{frame}
		%%%%%%%%%%%%%%%%%%%%%%%%%%%%%%%%%%%%%%%%%%%%%%%%%%%%%%%%%%%%%%%%%%%%%%%%%%%%
		\section{Non Destructive Testing (NDTs)}
		%%%%%%%%%%%%%%%%%%%%%%%%%%%%%%%%%%%%%%%%%%%%%%%%%%%%%%%%%%%%%%%%%%%%%%%%%%%%
		\begin{frame}{Structural Health Monitoring (SHM)}
			\begin{figure}
				\includegraphics[height=.8\textheight]{SHM_system.png}
			\end{figure}
		\end{frame}
		%%%%%%%%%%%%%%%%%%%%%%%%%%%%%%%%%%%%%%%%%%%%%%%%%%%%%%%%%%%%%%%%%%%%%%%%%%%%
		\note{
			Structural health monitoring is about monitoring a structure periodically or continuously to evaluate its technical conditions.
			
			As a result, the SHM aims to place integrated sensors on the structure to continuously monitor its condition.
			
			In this way, structural health monitoring mimics the human nervous system, which can provide us with real-time readings and detect any abnormal events.
			
			Accordingly, this is the ultimate aim of SHM: to monitor in real-time in a way that can detect damage early so we can perform preventive measures to protect the structure.
		}
		%%%%%%%%%%%%%%%%%%%%%%%%%%%%%%%%%%%%%%%%%%%%%%%%%%%%%%%%%%%%%%%%%%%%%%%%%%%%
		\begin{frame}{Ultrasonic testing (UT)} 
			\begin{columns}[T]
				\begin{column}[c]{0.38\textwidth}
					\textbf{Ultrasonic waves}	
					\begin{itemize}
						\item Frequency range: 2 MHz - 200 MHz
						\item Wavelength \(\lambda << h\) thickness 
						\item shorter wavelengths
					\end{itemize}
					
					\textbf{Guided waves}	
					\begin{itemize}
						\item Typical frequency range: \\ 10 kHz - 1 MHz
						\item Wavelength \(\lambda > h\) thickness 
						\item longer wavelengths
					\end{itemize}
					
				\end{column}
				\begin{column}[c]{0.6\textwidth}				
					\textbf{Ultrasonic testing} \hspace{50pt} \textbf{Guided wave testing}
					\begin{figure}
						\includegraphics[width=0.95\textwidth]{local_ultrasonic.png}
					\end{figure}
				\end{column}		
			\end{columns}	
		\end{frame}
		\note{
			\tiny
			Many conventional methods have been used for a long time in the industry for damage inspection, such as visual inspection, Eddy current, dye penetration, acoustic emission, and local ultrasonic testing.
			For the local ultrasonic testing, the frequency is in the range of 2 MHz–200 MHz, and the wavelength is way smaller than the thickness of the structure.
			There are different methods for ultrasonic testing, such as:
			pulse echo, in which we have a probe that generates ultrasonic waves that propagate through the thickness of a structure, and then the reflected waves from the bottom surface are captured by the same probe.
			The other ultrasonic method is called the "through-transmission method."
			There are two probes: one for transmitting on one side and the other for receiving on the other side.
			As a result, these methods are local, but they have some drawbacks:
			Labor-intensive, time-consuming, professional, and experienced personnel are required; complex structures may need to be disassembled.
			On the other hand, guided wave testing is a promising global NDE solution for SHM.
			The typical frequency range of guided waves is 10 kHz to 1 MHz, and their wavelength is larger than the thickness of the investigated structure.
			Guided waves can be generated using a piezoelectric actuator and registered by an array of piezoelectric sensors. 
			Alternatively, the propagation of guided Lamb waves can be registered using a scanning laser Doppler vibrometer to produce a full wavefield scan.		 
	 	}
		
	%	\begin{frame}{Ultrasonic testing vs guided wave testing}
	%%		\alert{Bulk waves} exist in infinite homogeneous bodies and propagate indefinitely without being interrupted by boundaries or interfaces. 
	%%		These waves can be decomposed into infinite plane waves propagating along arbitrary direction within the solid.
	%%		
	%%		\alert{Guided waves} are those waves that require a boundary for their existence, such as surface waves, Lamb waves, and interface waves.
	%%		\vspace{5mm}
	%		\begin{columns}[T]
	%			\begin{column}{0.5\textwidth}
	%				
	%			\end{column}
	%			\begin{column}{0.5\textwidth}
	%				
	%			\end{column}
	%		\end{columns}			
	%	\end{frame}
	%	%%%%%%%%%%%%%%%%%%%%%%%%%%%%%%%%%%%%%%%%%%%%%%%%%%%%%%%%%%%%%%%%%%%%%%%%%%%%
	%	\note{
	%		Guided waves are those waves that require a boundary for their existence, such as surface waves, Lamb waves, and interface waves.
	%		
	%	}
	%%%%%%%%%%%%%%%%%%%%%%%%%%%%%%%%%%%%%%%%%%%%%%%%%%%%%%%%%%%%%%%%%%%%%%%%%%	
	\begin{frame}{VGG16 encoder-decoder}
		\begin{figure}
			\centering
			\includegraphics[width=.6\textwidth]{figure5.png}
		\end{figure}
	\end{frame}
	%%%%%%%%%%%%%%%%%%%%%%%%%%%%%%%%%%%%%%%%%%%%%%%%%%%%%%%%%%%%%%%%%%%%%%%%%%%
	\note{
		For the VGG16 encoder-decoder model, the residual connections (skip connections) were added between the encoder and decoder levels to maintain the spatial and contextual information.
	}
	%%%%%%%%%%%%%%%%%%%%%%%%%%%%%%%%%%%%%%%%%%%%%%%%%%%%%%%%%%%%%%%%%%%%%%%%%%%
	\begin{frame}{Residual UNet}
		\begin{figure}
			\centering
			\includegraphics[width=.6\textwidth]{figure4.png}
		\end{figure}
	\end{frame}
	%%%%%%%%%%%%%%%%%%%%%%%%%%%%%%%%%%%%%%%%%%%%%%%%%%%%%%%%%%%%%%%%%%%%%%%%%%%
	\note{
		Residual UNet, also applies encoder-decoder approach,
		But the residual connections (skip connections) were added at two levels:
		\begin{itemize}
			\item at each step of the encoder and the decoder
			\item between the encoder and decoder levels.
		\end{itemize}
	}
	%%%%%%%%%%%%%%%%%%%%%%%%%%%%%%%%%%%%%%%%%%%%%%%%%%%%%%%%%%%%%%%%%%%%%%%%%%%
	\begin{frame}{FCN-DenseNet}
		\begin{columns}[T]
			\begin{column}[c]{0.48\textwidth}
				\begin{figure}[h!]
					\includegraphics[height=.8\textheight]{FCN_dense_net.png}
					\caption{FCN-DenseNet architecture.} 
					\label{fcn}
				\end{figure}
			\end{column}
			\hfill
			\begin{column}[c]{0.48\textwidth}
				\begin{figure}[h!]
					\centering
					\includegraphics[width=0.5\textwidth,angle=-90]{figure6.png}
					\caption{Dense block architecture.} 
				\end{figure}
			\end{column}
		\end{columns}
	\end{frame}
	%%%%%%%%%%%%%%%%%%%%%%%%%%%%%%%%%%%%%%%%%%%%%%%%%%%%%%%%%%%%%%%%%%%%%%%%%%%
	\section{Guided waves}
	%%%%%%%%%%%%%%%%%%%%%%%%%%%%%%%%%%%%%%%%%%%%%%%%%%%%%%%%%%%%%%%%%%%%
	%%%%%%%%%%%%%%%%%%%%%%%%%%%%%%%%%%%%%%%%%%%%%%
		\begin{frame}{Waves used in non-destructive testing}
			%%%%%%%%%%%%%%%%%%%%%%%%%%%%%%%%%%%%%%%%%%%%%%
			Elastic wave propagation types depending on particle motion:
			\begin{itemize}
					\item  \alert{The longitudinal wave} is a compressional wave in which the particle motion is in the same direction as the propagation of the wave
					\item \alert{The shear wave} is a wave motion in which the particle motion is perpendicular to the direction of the propagation
					\item \alert{Surface (Rayleigh) waves} have an elliptical particle motion and travel across the surface of a material. Their velocity is approximately 90\% of the shear wave velocity of the material and their depth of penetration is approximately equal to one
					wavelength
					\item \alert{Plate (Lamb) waves} have a complex vibration occurring in materials where thickness is less than the wavelength of elastic wave introduced into it.
				\end{itemize}
		\end{frame}
		%%%%%%%%%%%%%%%%%%%%%%%%%%%%%%%%%%%%%%%%%%%%%%%%%%
		\setcounter{subfigure}{0}
		\begin{frame}{Waves used in non-destructive testing}
			%%%%%%%%%%%%%%%%%%%%%%%%%%%%%%%%%%%%%%%%%%%%%%%%%%
			\begin{figure}
					\subfloat{\animategraphics[autoplay,loop, controls,width=0.5\textwidth]{10}{figures/gif_figs/Longitudinal_wave/Longitudinal_wave-}{0}{35}}
					\caption{\alert{Longitudinal wave} - plane pressure pulse wave}
				\end{figure}
		\tiny 
		(source: https://nojigon.webs.upv.es/index.php)
		\end{frame}
		%%%%%%%%%%%%%%%%%%%%%%%%%%%%%%%%%%%%%%%%%%%%%%%%%%%%%%%%%%%%%%%%%%%%%%%%%%%%%%%%
		%%%%%%%%%%%%%%%%%%%%%%%%%%%%%%%%%%%%%%%%%%%%%%%%%%
		\setcounter{subfigure}{0}
		\begin{frame}{Waves used in non-destructive testing}
			%%%%%%%%%%%%%%%%%%%%%%%%%%%%%%%%%%%%%%%%%%%%%%%%%%
			\begin{columns}[T]
					\begin{column}{0.5\textwidth}
							\centering
							\begin{figure}
									\subfloat{\animategraphics[autoplay,loop, controls,width=0.95\textwidth]{10}{figures/gif_figs/SH_shear_wave/SH_shear-}{0}{39}}
									\caption{\alert{Shear horizontal wave}}
								\end{figure}			
						\end{column}
					\begin{column}{0.5\textwidth}
							\centering
							\begin{figure}
									\subfloat{\animategraphics[autoplay,loop, controls,width=0.95\textwidth]{10}{figures/gif_figs/SV_shear_wave/SV_shear-}{0}{39}}
									\caption{\alert{Shear vertical wave}}
								\end{figure}			
						\end{column}	
				\end{columns}
		\tiny 
		(source: https://nojigon.webs.upv.es/index.php)
		\end{frame}
		%%%%%%%%%%%%%%%%%%%%%%%%%%%%%%%%%%%%%%%%%%%%%%%%%%
		\setcounter{subfigure}{0}
		\begin{frame}{Waves used in non-destructive testing}
			%%%%%%%%%%%%%%%%%%%%%%%%%%%%%%%%%%%%%%%%%%%%%%%%%%
			\begin{figure}
					\centering
					\subfloat{\animategraphics[autoplay,loop, controls,height=0.65\textheight]{15}{figures/gif_figs/Raileigh_wave/Raileigh_wave-}{0}{67}}
					\caption{\alert{Rayleigh waves}}		
				\end{figure}			
		\tiny 
		(source: https://nojigon.webs.upv.es/index.php)
		\end{frame}
		%%%%%%%%%%%%%%%%%%%%%%%%%%%%%%%%%%%%%%%%%%%%%%
		\setcounter{subfigure}{0}
		\begin{frame}{Waves used in non-destructive testing}
			%%%%%%%%%%%%%%%%%%%%%%%%%%%%%%%%%%%%%%%%%%%%%%
			\begin{figure}			
					\centering
					\subfloat{\animategraphics[autoplay,loop, controls,height=0.65\textheight]{20}{figures/gif_figs/love_wave/love_wave-}{0}{107}}
					\caption{\alert{Love waves} (surface seismic waves) named after Augustus Edward Hough Love}		
				\end{figure}			
		\tiny 
		(source: https://nojigon.webs.upv.es/index.php)
		\end{frame}
		%%%%%%%%%%%%%%%%%%%%%%%%%%%%%%%%%%%%%%%%%%%%%
		\setcounter{subfigure}{0}
		\begin{frame}{Lamb waves}
			%%%%%%%%%%%%%%%%%%%%%%%%%%%%%%%%%%%%%%%%%%%%%%
			\begin{alertblock}{Lamb waves}	
				Lamb waves are plane waves propagating in thin plates.\\
				Shear vertical waves in conjunction with longitudinal P waves interacts with plate surfaces resulting in complex wave mechanism which leads to creation of Lamb waves.
			\end{alertblock}
	%		Horace Lamb discovered these type of waves in 1917.
	%		He derived theory and dispersion relations.
			\begin{columns}[T]
				\begin{column}{0.5\textwidth}
					\centering
					symmetric modes
					\begin{equation*}
						\frac{\tan(q h)}{\tan(p h)} = -\frac{4 k^2 p q}{\left(q^2 - k^2\right)^2}
					\end{equation*}
				\end{column}
				\begin{column}{0.5\textwidth}
					\centering
					antisymmetric modes
					\begin{equation*}
						\frac{\tan(q h)}{\tan(p h)} = -\frac{\left(q^2 - k^2\right)^2}{4 k^2 p q}
					\end{equation*}
				\end{column}	
			\end{columns}	
			\centering
	%		\(q=q(\omega,k), \quad p=p(\omega,k) \)
			\begin{gather*}
				\centering
				p^2 = \frac{\omega^2}{c_{L}^2}-k^2,\ q^2 = \frac{\omega^2}{c_{S}^2}-k^2,\ k = \frac{2\pi}{\lambda},\ f=\frac{\omega}{2\pi}
			\end{gather*}
			\newline
			\begin{gather*}
				\centering
				c_L=\sqrt{\frac{2\mu (1-\nu)}{\rho(1-2\nu)}},\ c_S=\sqrt{\frac{\mu}{\rho}}
			\end{gather*}
		\end{frame}
		%%%%%%%%%%%%%%%%%%%%%%%%%%%%%%%%%%%%%%%%%%%%%%%%%%%%%%%%%%%%%%%%%%%%%%%%%%%%
		\note{
			Lamb waves are a subset of guided waves that propagate between two parallel surfaces, such as in shell and plate structures discovered by Horace Lamb in 1917.
			
			Lamb waves can be separated into symmetric modes and antisymmetric modes in terms of the surface particle motion to the mid-plane, which are described in characteristic Lamb wave equations:
			
			
			where $h$ is the half-thickness of the plate, $k$ is the wavenumber, $\omega$ is the angular frequency, and $\lambda$ is the wavelength. 
			\(cL\) and \(cS\) are the velocities of longitudinal and transverse
			waves, respectively.
			
			where \(\rho\) is the density, \(\mu\) is the shear modulus, and \(\nu\) is Poisson's ratio of the medium.
		}
		%%%%%%%%%%%%%%%%%%%%%%%%%%%%%%%%%%%%%%%%%%%%%%%%%%%%%%%%%%%%%%%%%%%%%%%%%%%%
		\setcounter{subfigure}{0}
		\begin{frame}{Lamb waves modes}
			%%%%%%%%%%%%%%%%%%%%%%%%%%%%%%%%%%%%%%%%%%%%%%
			\begin{columns}[T]
				\begin{column}{0.3\textwidth}
					\centering
					\begin{figure}
						\animategraphics[autoplay,loop,width=1\textwidth]{10}{figures/gif_figs/S0_mode/S0_mode-}{0}{67}
						\caption{Fundamental symmetric, S0, \alert{Lamb wave} mode (in-plane motion)}
					\end{figure}
				\end{column}
				\begin{column}{0.3\textwidth}
					\centering
					\begin{figure}
						\animategraphics[autoplay,loop,width=1\textwidth]{10}{figures/gif_figs/A0_mode/A0_mode-}{0}{67}
						\caption{Fundamental antisymmetric, A0, \alert{Lamb wave} mode (out-of-plane motion)}
					\end{figure}
				\end{column}
				\hfill
				\begin{column}{0.37\textwidth}
						\begin{itemize}
							\item \textcolor{blue}{Travel within guides for long distances}
							\item \textcolor{blue}{Can propagate in complex structures}
	%						\item \textcolor{blue}{High speeds in metals and composites}
	%						\item \textcolor{blue}{Can be automated using software}
						\end{itemize}
						\textbf{Lamb waves are a promising global NDE solution for SHM}
				\end{column}
			\end{columns}	
			\tiny 
			(source: https://nojigon.webs.upv.es/index.php)
		\end{frame}
		%%%%%%%%%%%%%%%%%%%%%%%%%%%%%%%%%%%%%%%%%%%%%%%%%%%%%%%%%%%%%%%%%%%%%%%%%%%%
		\note{
			The animation on the left shows the fundamental symmetric \(S_0\) mode of the Lamb wave.
			
			We can see the in-plane motion of the particles.
			Whereas the second animation shows the fundamental antisymmetric \(A_0\) mode of the Lamb wave.
			We can see the out-of-plane motion of the particles.
			
			Furthermore, Lamb wave has some good characteristics that make it a promising solution for SHM as it:\\
			1. Can travel within guides for long distances. \\
			2. Can propagate in complex structures. \\
			3. Has high speed in metals and composites.\\ 	
		}
	%%%%%%%%%%%%%%%%%%%%%%%%%%%%%%%%%%%%%%%%%%%%%%%%%%%%%%%%%%%%%%%%%%%%%%%%%%%
	
		\begin{frame}{Dispersion curves of Lamb waves}
			%%%%%%%%%%%%%%%%%%%%%%%%%%%%%%%%%%%%%%%%%%%%%%
			\begin{figure}
				\only<1>{
					\includegraphics[width=0.8\textwidth]{/figs/Fig_1_12.png}	
				}
				\only<2>{
					\includegraphics[width=0.8\textwidth]{/figs/Fig_1_13.png}	
				}
			\end{figure}
		\end{frame}
	%%%%%%%%%%%%%%%%%%%%%%%%%%%%%%%%%%%%%%%%%%%%%%%%%%%%%%%%%%%%%%%%%%%%%%%%%%%
	\setcounter{subfigure}{0}
	\section{Artificial intelligence, machine learning, and deep learning}
	%%%%%%%%%%%%%%%%%%%%%%%%%%%%%%%%%%%%%%%%%%%%%%%%%%
	%%%%%%%%%%%%%%%%%%%%%%%%%%%%%%%%%%%%%%%%%%%%%%%%%%
	\begin{frame}{What is deep learning?}
		\begin{figure}
			\centering
			\includegraphics[width=0.85\textwidth]{AI_vs_ML_vs_Deep_Learning.png}
		\end{figure}
		\tiny
		(source: https://www.ingeniovirtual.com/)
	\end{frame}	
	%%%%%%%%%%%%%%%%%%%%%%%%%%%%%%%%%%%%%%%%%%%%%%%%%%%%%%%%%%%%%%%%%%%%%%%%%%%%%%%%
%	\setcounter{subfigure}{0}
	%%%%%%%%%%%%%%%%%%%%%%%%%%%%%%%%%%%%%%%%%%%%%%%%%%
%	\begin{frame}{Deep learning, why now?}
%		\begin{column}[c]{0.4\textwidth}
%			AI technologies are in accelerating growth due to:
%			\begin{itemize}
%				\item Exponential development in computer hardware industries
%				 (e.g. CPUs, GPUs, FPGAs, TPUs and ASICs)
%				\item Era of Big data.
%			\end{itemize}
%		\end{column}
%		\begin{column}[c]{0.55\textwidth}
%			\begin{figure}
%				\centering
%				\subfloat{\animategraphics[autoplay,loop,width=.9\textwidth]{10}{gif_figs/gpu/gpu_-}{0}{34}}
%			\end{figure}
%		\tiny
%		(source: https://www.techbooky.com/)
%		\end{column}
%		
%	\end{frame}
	%%%%%%%%%%%%%%%%%%%%%%%%%%%%%%%%%%%%%%%%%%%%%%%%%%
	\setcounter{subfigure}{0}
	%%%%%%%%%%%%%%%%%%%%%%%%%%%%%%%%%%%%%%%%%%%%%%%%%%
	\begin{frame}{Common learning strategies}
		\centering
		\begin{figure}
			\includegraphics[width=0.9\textwidth]{learning.png}
		\end{figure}
		\tiny
		(source: https://www.aitude.com/supervised-vs-unsupervised-vs-reinforcement/)
	\end{frame}
	%%%%%%%%%%%%%%%%%%%%%%%%%%%%%%%%%%%%%%%%%%%%%%%%%%%%%%%%%%%%%%%%%%%%%%%%%%%%%%%%%
	
	%%%%%%%%%%%%%%%%%%%%%%%%%%%%%%%%%%%%%%%%%%%%%%%%%%%%%
	\section{Introduction to deep learning approach}		
	%%%%%%%%%%%%%%%%%%%%%%%%%%%%%%%%%%%%%%%%%%%%%%%%%%%%%%%%%%%%%%%%%%%%%%%%%%%
	\setcounter{subfigure}{0}
	%%%%%%%%%%%%%%%%%%%%%%%%%%%%%%%%%%%%%%%%%%%%%%%%%%%%%%%%%%%%%%%%%%%%%%%%%%%
	\begin{frame}{Convolution Neural Networks (CNNs)}
		\begin{itemize}
			\item \alert{Convolutional Neural Network} (CNN) is a powerful feature extracting tool.
			\item Features are extracted by applying convolution operations \alert{(sliding dot product)}.
		\end{itemize}
		\begin{columns}[T]
			\begin{column}[t]{0.48\textwidth}
				\begin{figure}[t]
					\centering
					\animategraphics[autoplay,loop,width =0.95\textwidth]{2}{figures/gif_figs/files/plot_convolution_process_}{0}{32}
				\end{figure}
			\end{column}
			\begin{column}[t]{0.48\textwidth}
				\begin{figure}
					\centering
					\includegraphics[width=0.9\textwidth]{cnn.png}
				\end{figure}
			\end{column}			
		\end{columns}
		
		The kernel weights are updated during training phase.
	\end{frame}
	%%%%%%%%%%%%%%%%%%%%%%%%%%%%%%%%%%%%%%%%%%%%%%%%%%%%%%%%%%%%%%%%%%%%%%%%%%%
	\note{
		Convolutional Neural Network (CNN) is a powerful deep learning architecture for image processing because it can extract complex feature patterns from images using convolution operations.
		
		The convolution operation for image processing is essentially a cross-correlation operation, also known as a sliding dot product or sliding inner product.
		
		The kernel slides over an input image of performing a convolution operation.
		The output of the convolution operation is a feature map.
		
		Consequently, kernels learn to detect different types of edges
		(vertical, horizontal, and diagonal edges), colour intensities, etc.	
		
		During the backpropagation process, all kernel weights are updated. 				
	}
	%%%%%%%%%%%%%%%%%%%%%%%%%%%%%%%%%%%%%%%%%%%%%%%%%%%%%%%%%%%%%%%%%%%%%%%%%%%
	\setcounter{subfigure}{0}
	%%%%%%%%%%%%%%%%%%%%%%%%%%%%%%%%%%%%%%%%%%%%%%%%%%%%%%%%%%%%%%%%%%%%%%%%%%%
	\begin{frame}{Encoder-decoder (Autoencoder)}
		\begin{columns}[T]
			\begin{column}[t]{.35\textwidth}
				\begin{itemize}
					\item \alert{Encoder} learns to extract features.
					\item \alert{Latent space} has condensed feature maps.
					\item \alert{Decoder} learns to locate the features learned by the encoder.
				\end{itemize}	
			\end{column}
			\hfill
			\begin{column}[t]{.6\textwidth}
				\begin{figure}
					\centering
					\includegraphics[width=1\textwidth]{nn_encoder_decoder.png}
				\end{figure}	
			\end{column}
		\end{columns}			
	\end{frame}
	%%%%%%%%%%%%%%%%%%%%%%%%%%%%%%%%%%%%%%%%%%%%%%%%%%%%%%%%%%%%%%%%%%%%%%%%%%%
	\note{
		The encoder-decoder is a well-known architecture used for tasks such as computer vision.
		The encoder aims to produce compressed feature maps from the input image at various scale levels using cascaded convolutions and downsampling operations. 
		The decoder is responsible for upsampling the condensed feature maps in the latent space to the original input shape.		
	}
	%%%%%%%%%%%%%%%%%%%%%%%%%%%%%%%%%%%%%%%%%%%%%%%%%%%%%%%%%%%%%%%%%%%%%%%%%%%
	\setcounter{subfigure}{0}
	%%%%%%%%%%%%%%%%%%%%%%%%%%%%%%%%%%%%%%%%%%%%%%%%%%%%%%%%%%%%%%%%%%%%%%%%%%%
	\begin{frame}{Computer vision}
		\begin{columns}[T]
			\begin{column}[c]{0.27\textwidth}
				\justifying
				\alert {\textbf{Computer vision}} is a field of AI that enables computers and systems to derive meaningful information from digital images, videos and other visual inputs. 
			\end{column}
			\quad
			\begin{column}[c]{0.7\textwidth}
				\begin{figure}
					\centering
					\includegraphics[width=1\textwidth]{computer_vision_tasks.png}
				\end{figure}
			\end{column}
		\end{columns}
	\end{frame}
	%%%%%%%%%%%%%%%%%%%%%%%%%%%%%%%%%%%%%%%%%%%%%%%%%%%%%%%%%%%%%%%%%%%%%%%%%%%
	\note{
		We now reach an essential concept: computer vision. 
		Computer vision is a subcategory of artificial intelligence that enables computers to obtain meaningful information from visual inputs.
		
		Computer vision has three hierarchical levels: 
		
		The first level performs a classification for the whole input and predicts one output.
		
		The second level classifies and locates the object that we are looking for. 
		
		The ultimate level of computer vision is to perform Pixel-wise segmentation (or semantic segmentation), in which each pixel in the input image is classified into its corresponding class.	
		
		I am using pixel wise image segmentation in my PhD work.
		
	}
	%%%%%%%%%%%%%%%%%%%%%%%%%%%%%%%%%%%%%%%%%%%%%%%%%%%%%%%%%%%%%%%%%%%%%%%%%%%
	\setcounter{subfigure}{0}
	%%%%%%%%%%%%%%%%%%%%%%%%%%%%%%%%%%%%%%%%%%%%%%%%%%%%%%%%%%%%%%%%%%%%%%%%%%%
	\begin{frame}{Conventional machine learning v.s. deep learning approaches}
		Conventional methods involve two processes:
		\alert{\textbf{Feature extraction and classification}}
		\begin{figure}
			\centering
			\includegraphics[width=.95\textwidth]{conventional_ML.png}
		\end{figure}	
		Deep learning offers an \alert{\textbf{end-to-end}} approach: \alert{\textbf{Automatic}} feature extraction and classification.
		\begin{figure}
			\includegraphics[width=.95\textwidth]{DL_approach.png}
		\end{figure}
	\end{frame}
	%%%%%%%%%%%%%%%%%%%%%%%%%%%%%%%%%%%%%%%%%%%%%%%%%%%%%%%%%%%%%%%%%%%%%%%%%%%
	\note{
		In this slide, I present a comparison between conventional machine learning and deep learning approaches to damage detection.
		
		In the conventional approaches, two processes must be performed: 
		The first is to extract the useful features from the registered data and then use a proper classification technique.
		
		This approach has several drawbacks:
		First, it requires a great amount of human labor and computational effort
		Secondly, it demands a high amount of experience from the practitioner.
		And lastly, when dealing with big data, it tends to be Inefficient as it requires a complex computation of damage features extraction and classification.
		
		On the other hand, the deep learning approach offers the opportunity to develop models that can automatically perform feature extraction and classification by themselves without human intervention in an end-to-end approach. 
	}
	
	\section{Synthetic dataset generation}		
	\setcounter{subfigure}{0}
	%%%%%%%%%%%%%%%%%%%%%%%%%%%%%%%%%%%%%%%%%%%%%%%%%%%%%%%%%%%%%%%%%%%%%%%%%%%
	\begin{frame}{The time domain spectral element method}
		\begin{columns}[T]
			\begin{column}{0.47\textwidth}
				\begin{itemize}
					\item Mindlin-Reissner plate theory
					\item Splitting elements and nodes at delamination
					\item GMSH software was used for meshing quads then converted to spectral elements
				\end{itemize}	
				\begin{figure}
					\subfloat{\includegraphics[width=0.9\textwidth]{shell.png}}	
				\end{figure}
			\end{column}
			\begin{column}{0.47\textwidth}	
				\begin{figure}
					\animategraphics[controls,autoplay,loop,width=0.9\textwidth]{1}{/gif_figs/mesh/m1_rand_single_delam_}{1}{20}
				\end{figure}	
			\end{column}
		\end{columns}	
	\end{frame}
	%%%%%%%%%%%%%%%%%%%%%%%%%%%%%%%%%%%%%%%%%%%%%%%%%%%%%%%%%%%%%%%%%%%%%%%%%%%
	\note{
		To train the supervised deep learning models, a synthetic dataset of propagating waves in carbon fibre-reinforced composite plates was computed by applying the Mindlin-Reissner plate theory and using the parallel implementation of the time domain spectral element method.
		
		For each case, single delamination was modeled by using the method of splitting nodes between appropriate spectral elements.
		
		Essentially, the dataset resembles the particle velocity measurements at the bottom surface of the plate acquired by the SLDV in the transverse direction as a response to the piezoelectric (PZT) excitation at the centre of the plate.		 
	}
	%%%%%%%%%%%%%%%%%%%%%%%%%%%%%%%%%%%%%%%%%%%%%%%%%%%%%%%%%%%%%%%%%%%%%%%%%%%
	\setcounter{subfigure}{0}
	\begin{frame}{Dataset description}
		\begin{columns}[T]
			\begin{column}[t]{0.35\textwidth}
				\begin{itemize}
					\item 475 delamination scenarios
					\item CFRP is made of 8-layers
					\item Delamination modelled between the 3rd and 4th layer
					\item Delamination size min 10 mm, max  40 mm
					\item \textbf{3-months of computing}
				\end{itemize}
			\end{column}
			\begin{column}[t]{0.2\textwidth}
				\begin{figure}[t]
					\centering
					\subfloat[Delamination placement]{\includegraphics[width=0.95\textwidth]{delamination_placement.png}}
				\end{figure}
			\end{column}
			\begin{column}[t]{0.45\textwidth}
				\begin{figure}[t]					
					\centering						
					\subfloat[Delamination orientation]{\includegraphics[width=0.95\textwidth]{figure1.png}}					
				\end{figure}
			\end{column}
		\end{columns}
	\end{frame}
	%%%%%%%%%%%%%%%%%%%%%%%%%%%%%%%%%%%%%%%%%%%%%%%%%%%%%%%%%%%%%%%%%%%%%%%%%%%
	\note{
		The synthetically generated dataset has 475 delamination cases.
		It was assumed that the composite laminate is made of eight layers with a total thickness of 3.9 mm. 
		The delamination was modeled between the third and fourth layers.
		The delamination geometrical size major and minor axes were randomly selected from interval \([10mm,\ 40mm]\). 
		The computation of the dataset took about three months.
	}
	%%%%%%%%%%%%%%%%%%%%%%%%%%%%%%%%%%%%%%%%%%%%%%%%%%%%%%%%%%%%%%%%%%%%%%%%%%%
	\setcounter{subfigure}{0}
	\begin{frame}{Training Sample case}
		\begin{columns}[T]
			\begin{column}[c]{.32\textwidth}
				\begin{figure}
					\centering
					\animategraphics[autoplay,loop,width=0.85 \textwidth]{16}{figures/gif_figs/7_output/flat_shell_Vz_7_500x500bottom-}{1}{512}
					\caption{Full wavefield $s(x,y,t_k)$}
				\end{figure}
			\end{column}
			\begin{column}[c]{.32\textwidth}
				\begin{figure}
					\centering
					\includegraphics[width=0.85 \textwidth]{RMS_flat_shell_Vz_7_500x500bottom.png}
					\caption{RMS image $\hat{s}(x,y)$}
				\end{figure}
			\end{column}
			\begin{column}[c]{.32\textwidth}
				\begin{figure}
					\centering
					\includegraphics[width=0.85 \textwidth]{m1_rand_single_delam_7.png}
					\caption{Ground truth (label)}
				\end{figure}
			\end{column}
		\end{columns}
		\begin{equation*}
			\hat{s}(x,y) = \sqrt{\frac{1}{N}\sum_{k=1}^{N}s(x,y,t_k)^2} 
			\label{eqn:rms} 
		\end{equation*}
		\(N\) is the number of sampling points equal to 512, \((x,y)\) are the point coordinates on the plate, and \(t_k\) is the time step.
	\end{frame}
	%%%%%%%%%%%%%%%%%%%%%%%%%%%%%%%%%%%%%%%%%%%%%%%%%%%%%%%%%%%%%%%%%%%%%%%%%%%
	\note{
		In this slide, I present a training scenario.
		The animation on the left represents the full wavefield frames, where x and y are the point coordinates, and tk is the time step.
		
		The output of applying the root mean square formula to the full wavefield is shown in the middle figure.
		
		The ground truth label that represents the delamination location and shape is shown on the right.
	}
	%%%%%%%%%%%%%%%%%%%%%%%%%%%%%%%%%%%%%%%%%%%%%%%%%%%%%%%%%%%%%%%%%%%%%%%%%%%
	\setcounter{subfigure}{0}
	\section{Semantic segmentation}
	\begin{frame}{Semantic segmentation}
		\begin{columns}[T]
			\begin{column}[c]{0.47\textwidth}
				\centering
				\textbf{One-to-one \\image-based approach (RMS)} 
				\begin{figure}
					\centering
					\subfloat[Single input (image)]{\includegraphics[width=.45\textwidth]{RMS_flat_shell_Vz_381_500x500bottom.png}}\quad
					\subfloat[Single output]{\includegraphics[width=.45\textwidth]{GCN_381.png}}
				\end{figure}
			\end{column}
			\hfill
			\begin{column}[c]{0.47\textwidth}
				\centering
				\textbf{Many-to-one \\animation-based approach}
				\begin{figure}
					\centering
					\subfloat[Multiple frames animation]{\animategraphics[autoplay,loop,width=.45\textwidth]{4}{figures/gif_figs/381_output/output_381-}{85}{113}}\quad
					\subfloat[Single output]{\includegraphics[width=.45\textwidth]{GCN_381.png}}
				\end{figure}
			\end{column}	
		\end{columns}		
	\end{frame}	
	%%%%%%%%%%%%%%%%%%%%%%%%%%%%%%%%%%%%%%%%%%%%%%%%%%%%%%%%%%%%%%%%%%%%%%%%%%%
	\note{
		The novelty of this work consists of applying for the first time a full wavefield dataset of elastic wave propagation as an input to various supervised deep-learning models that are capable of identifying damage by classifying each output pixel as healthy or damaged. 
		
		Accordingly, two main approaches were adopted:	
		\begin{itemize}
			\item One-to-one approach (takes one input (RMS image) and produces one output of damage map)
			\item Many-to-one approach (takes animation of Lamb waves propagation as a sequence of frames and produces one output as damage map).
		\end{itemize}	
	}
	%%%%%%%%%%%%%%%%%%%%%%%%%%%%%%%%%%%%%%%%%%%%%%%%%%%%%%%%%%%%%%%%%%%%%%%%%%%
		\subsection{Developed DL models}
%	\begin{frame}{Common deep learning architectures}		
%		\begin{column}[t]{0.45\textwidth}
%			\textbf{RMS based}\\
%			\begin{itemize}
%				\item Convolutional neural networks (CNN)
%				\item Fully convolutional network (FCN)
%			\end{itemize}
%		\end{column}
%		\hfill
%		\begin{column}[t]{0.45\textwidth}
%			\textbf{Full wavefield frames}\\
%			\begin{itemize}
%				\item Recurrent neural network (RNN)
%				\item Long short-term memory (LSTM)
%				\item ConvLSTM
%			\end{itemize}
%		\end{column}
%	\end{frame}
	%%%%%%%%%%%%%%%%%%%%%%%%%%%%%%%%%%%%%%%%%%%%%%%%%
	\begin{frame}{Developed model for delamination identification}
		\begin{columns}[T]
			\begin{column}[t]{0.45\textwidth}
				\textbf{RMS based models: }
				\medskip
				\begin{itemize}
					\item VGG 16 encoder-decoder
					\item Res-UNet					
					\item FCN-DenseNet
					\item PSPNet
					\item GCN
				\end{itemize}				
			\end{column}
			\hfill
			\begin{column}[t]{.45\textwidth}
				\textbf{Full wavefield frames based model:}
				\begin{itemize}
					\item Autoencoder ConvLSTM
				\end{itemize}				
			\end{column}
		\end{columns}
	\end{frame}	
	%%%%%%%%%%%%%%%%%%%%%%%%%%%%%%%%%%%%%%%%%%%%%%%%%%%%%%%%%%%%%%%%%%%%%%%%%%%
	\note{
		Deep learning models for delamination identification were developed based on their inputs.
		
		The models based on the RMS images as input are:
		the residual UNet, VGG16 encoder-decoder, fully convolutional dense network (FCN-DenseNet), Pyramid scene parsing network (PSPNet), and The global convolutional neural network (GCN).
		
		And the model based on the animation of full wavefield frames is 
		Autoencoder ConvLSTM.
		
		All developed models were published in these articles.		
	}
	%%%%%%%%%%%%%%%%%%%%%%%%%%%%%%%%%%%%%%%%%%%%%%%%%%%%%%%%%%%%%%%%%%%%%%%%%%%
	\setcounter{subfigure}{0}
	\subsection{Developed RMS based models}	
	%%%%%%%%%%%%%%%%%%%%%%%%%%%%%%%%%%%%%%%%%%%%%%%%%%%%%%%%%%%%%%%%%%%%%%%%%%%
	\begin{frame}{VGG16 encoder-decoder}
		\begin{figure}
			\centering
			\includegraphics[width=.6\textwidth]{figure5.png}
		\end{figure}
	\end{frame}
	%%%%%%%%%%%%%%%%%%%%%%%%%%%%%%%%%%%%%%%%%%%%%%%%%%%%%%%%%%%%%%%%%%%%%%%%%%%
	\note{
		For the VGG16 encoder-decoder model, the residual connections (skip connections) were added between the encoder and decoder levels to maintain the spatial and contextual information.
	}
	%%%%%%%%%%%%%%%%%%%%%%%%%%%%%%%%%%%%%%%%%%%%%%%%%%%%%%%%%%%%%%%%%%%%%%%%%%%
	\begin{frame}{Residual UNet}
		\begin{figure}
			\centering
			\includegraphics[width=.6\textwidth]{figure4.png}
		\end{figure}
	\end{frame}
	%%%%%%%%%%%%%%%%%%%%%%%%%%%%%%%%%%%%%%%%%%%%%%%%%%%%%%%%%%%%%%%%%%%%%%%%%%%
	\note{
		Residual UNet, also applies encoder-decoder approach,
		But the residual connections (skip connections) were added at two levels:
		\begin{itemize}
			\item at each step of the encoder and the decoder
			\item between the encoder and decoder levels.
		\end{itemize}
	}
	%%%%%%%%%%%%%%%%%%%%%%%%%%%%%%%%%%%%%%%%%%%%%%%%%%%%%%%%%%%%%%%%%%%%%%%%%%%
	\begin{frame}{FCN-DenseNet}
		\begin{columns}[T]
			\begin{column}[c]{0.48\textwidth}
				\begin{figure}[h!]
					\includegraphics[height=.8\textheight]{FCN_dense_net.png}
					\caption{FCN-DenseNet architecture.} 
					\label{fcn}
				\end{figure}
			\end{column}
			\hfill
			\begin{column}[c]{0.48\textwidth}
				\begin{figure}[h!]
					\centering
					\includegraphics[width=0.5\textwidth,angle=-90]{figure6.png}
					\caption{Dense block architecture.} 
				\end{figure}
			\end{column}
		\end{columns}
	\end{frame}
	%%%%%%%%%%%%%%%%%%%%%%%%%%%%%%%%%%%%%%%%%%%%%%%%%%%%%%%%%%%%%%%%%%%%%%%%%%%
	\note{
		FCN-DenseNet applies an encoder-decoder scheme with skip connections between the encoder and the decoder paths.
		The main component in FCN-DenseNet is the dense block. 
		The dense block is constructed from a varying number of convolutional layers. 
		The purpose of the dense block is to concatenate feature maps of a layer with its output to emphasise spatial details information.
	}
	%%%%%%%%%%%%%%%%%%%%%%%%%%%%%%%%%%%%%%%%%%%%%%%%%%%%%%%%%%%%%%%%%%%%%%%%%%%
	
	\begin{frame}{Pyramid Scene Parsing Network}
		\begin{figure} [h!]
			\centering
			\includegraphics[height=.7\textheight]{figure7.png}
			\caption{PSPNet architecture.} 
		\end{figure}
	\end{frame}
	%%%%%%%%%%%%%%%%%%%%%%%%%%%%%%%%%%%%%%%%%%%%%%%%%%%%%%%%%%%%%%%%%%%%%%%%%%%
	\note{
		The idea of PSPNet is to provide adequate global contextual information for pixel-level scene parsing by concatenating the local and global features together. 
		Hence, a spatial pyramid pooling module was introduced to perform four different pooling levels with four different pool sizes. 
		In this way, the pyramid pooling module can capture contextual features at different scales.	
	}
	%%%%%%%%%%%%%%%%%%%%%%%%%%%%%%%%%%%%%%%%%%%%%%%%%%%%%%%%%%%%%%%%%%%%%%%%%%%
	
	\begin{frame}{Global Convolution Network}
		\begin{columns}[T]
			\begin{column}[c]{0.55\textwidth}
				\begin{figure}
					\centering
					\includegraphics[width=.9\textwidth]{figure8.png}
					\caption{GCN architecture.} 
				\end{figure}	
			\end{column}
			\begin{column}[c]{0.45\textwidth}
				\begin{figure}
					\centering
					\includegraphics[width=.9\textwidth]{figure9.png}
					\caption{(a) Residual block, (b) GCN block, (c) Boundary Refinement.} 
				\end{figure}	
			\end{column}
		\end{columns}
	\end{frame}	
	%%%%%%%%%%%%%%%%%%%%%%%%%%%%%%%%%%%%%%%%%%%%%%%%%%%%%%%%%%%%%%%%%%%%%%%%%%%
	\note{
		GCN addresses the importance of having large kernels at the convolution operations for both localization and classification tasks for semantic segmentation.
	}
	%%%%%%%%%%%%%%%%%%%%%%%%%%%%%%%%%%%%%%%%%%%%%%%%%%%%%%%%%%%%%%%%%%%%%%%%%%%
	
	%%%%%%%%%%%%%%%%%%%%%%%%%%%%%%%%%%%%%%%%%%%%%%%%%%%%%%%%%%%%%%%%%%%
	\setcounter{subfigure}{0}
	\begin{frame}{RMS based models}
		\begin{column}[c]{0.55\textwidth}
			\begin{figure}
				\subfloat[Res-UNet model]{\includegraphics[width=1\textwidth]{figure4.png}}
			\end{figure}
		\end{column}
		\begin{column}[c]{0.35\textwidth}
			\begin{figure}
				\subfloat[Data flow \& intermediate outputs of layers \label{fig:}]{\animategraphics[autoplay, controls,width=.8\textwidth]{4}{figures/gif_figs/381__inter_pred/intermediate_output-}{0}{103}}
	\end{figure}
	\end{column}
	
	\end{frame}
	
	\setcounter{subfigure}{0}
	%%%%%%%%%%%%%%%%%%%%%%%%%%%%%%%%%%%%%%%%%%%%%%%%%%%%%%%%%%%%%%%%%%	
	\subsection{Full wavefield frames based model}
	\begin{frame}{Autoencoder ConvLSTM}
		\begin{columns}[T]		
			\begin{column}[c]{0.5\textwidth}
				\only<1>{
					\begin{figure}[c]
						\centering
						\subfloat{\includegraphics[width=.8\textwidth]{figure2.png}}
						\caption{Sample frames of full wave propagation.}
				\end{figure}}
				\only<2->{
					\begin{figure}[c]
						\centering
						\subfloat{\includegraphics[height=.6\textheight]{figure3.png}}
						\caption{The procedure of calculating the RMS prediction image (damage map).}
				\end{figure}}
			\end{column}
			\begin{column}[c]{0.45\textwidth}
				\begin{figure}
					\centering
					\subfloat{\includegraphics[width=.8\textheight]{figure5b.png}}
					\caption{Autoencoder ConvLSTM model}
				\end{figure}
			\end{column}
		\end{columns}
	\end{frame}
	%%%%%%%%%%%%%%%%%%%%%%%%%%%%%%%%%%%%%%%%%%%%%%%%%%%%%%%%%%%%%%%%%%%%%%%%%%%
	\note{
		The autoencoder ConvLSTM was trained on a sequence of consecutive full wavefield frames.
		
		Hence, for the synthetic dataset, the delamination location is known, I only utilised 24 frames after the interaction with the damage for training the model.
		
		Those selected frames containing the required features about the delamination shape and location are fed into an encoder-decoder model at once using a time-distributed layer.
		
		Then, the output of the decoder was forwarded into the ConvLSTM layer to learn the long-term spatiotemporal features.
		
		
		Now, for real-life situations where the damage location is not known, the figure on the left illustrates the complete procedure of obtaining intermediate predictions for the testing cases and finally calculating the RMS image.
		
		Accordingly, the window slides over all the input frames with a shift of one frame at a time.
		
		Then, an RMS of all the intermediate predictions is calculated to produce a damage map.		
	}
	%%%%%%%%%%%%%%%%%%%%%%%%%%%%%%%%%%%%%%%%%%%%%%%%%%%%%%%%%%%%%%%%%%%%%%%%%%%
	\begin{frame}{Evaluation metrics for delamination identification}
		\begin{columns}[T]
			\begin{column}[c]{0.45\textwidth}
				For evaluating delamination identification
				\begin{itemize}
					\item Intersection over Union (IoU): 
					\begin{equation*}
						\textup{IoU}=\frac{Intersection}{Union}=\frac{\hat{Y} \cap Y}{\hat{Y} \cup Y}
						\label{eqn:iou}
					\end{equation*}
					\item Percentage area error $\epsilon$:
					\begin{equation*}
						\epsilon=\frac{|A-\hat{A}|}{A} \times 100\%
						\label{eqn:mean_size_error}
					\end{equation*}
				\end{itemize}
			\end{column}
			\begin{column}[c]{0.45\textwidth}
				\begin{figure}
					\centering
					\includegraphics[width=1.0\textwidth]{IoU_figure.png}		
				\end{figure}
			\end{column}
		\end{columns}
	\end{frame}
	%%%%%%%%%%%%%%%%%%%%%%%%%%%%%%%%%%%%%%%%%%%%%%%%%%%%%%%%%%%%%%%%%%%%%%%%%%%
	\note{
		To evaluate all developed models for delamination identification, two metrics were used:
		
		The first metric is the mean intersection over union (also known as the Jaccard index), which calculates the area of intersection between actual and predicted values divided by the union of them.
		
		The second metric is the percentage area error \(\epsilon\), which calculates the percentage of the difference between actual and predicted areas to the actual area size. 
	}
	%%%%%%%%%%%%%%%%%%%%%%%%%%%%%%%%%%%%%%%%%%%%%%%%%%%%%%%%%%%%%%%%%%%%%%%%%%%
		\section{Evaluation: Numerical cases}
		\subsection{Numerical test cases}
	%%%%%%%%%%%%%%%%%%%%%%%%%%%%%%%%%%%%%%%%%%%%%%%%%%%%%%%%%%%%%%%
	\begin{frame}{Numerical test cases RMS based models (GCN model)}
		\begin{columns}[T]
			\begin{column}[c]{0.32\textwidth}
				\begin{figure}[c]
					\centering
					\animategraphics[controls,width=.9\textwidth]{2}{figures/gif_figs/456/intermediate_output-}{0}{82}
					\caption{\(1^{st}\) numerical case, IoU=0.71}
				\end{figure}
			\end{column}
			\hfill
			\begin{column}[c]{0.32\textwidth}
				\begin{figure}[c]
					\centering
					\animategraphics[controls,width=.9\textwidth]{2}{figures/gif_figs/438/intermediate_output-}{0}{82}
					\caption{\(2^{nd}\) numerical case, IoU=0.72}
				\end{figure}
			\end{column}
			\hfill
			\begin{column}[c]{0.32\textwidth}
				\begin{figure}[c]
					\centering
					\animategraphics[controls,width=.9\textwidth]{2}{figures/gif_figs/397/intermediate_output-}{0}{82}
					\caption{\(3^{rd}\) numerical case, IoU=0.86}
				\end{figure}					
			\end{column}
		\end{columns}
	\end{frame}
	%%%%%%%%%%%%%%%%%%%%%%%%%%%%%%%%%%%%%%%%%%%%%%%%%%%%%%%%%%%%%%%%%%%%%%%%%%%
	\note{
		In this section, the numerical and experimental evaluation of the developed models for delamination identification will be presented.
		
		This slide shows the three numerical test cases evaluated with the GCN model.
		
		For the first case, the delamination is barely visible by the naked eye, yet the model could identify the delamination with IOU= 0.71.
		
		For the second case, the IOU = 0.72, and for the third case, the IOU = .86.
		
		It's important to notice that GCN can identify the delamination with high accuracy and almost no noise.		
	}
	%%%%%%%%%%%%%%%%%%%%%%%%%%%%%%%%%%%%%%%%%%%%%%%%%%%%%%%%%%%%%%%%%%%%%%%%%%%
	\begin{frame}{RMS based: Analysis of numerical cases}
		\begin{columns}[T]
							\tiny
							\begin{column}[c]{0.48\textwidth}
								%%%%%%%%%%%%%%%%%%%%%%%%%%%%%%%%%%%%%%%%%%%%%%%%%%%%%%%%%%%%
								\begin{table}[ht!]
									\centering
									\caption{Evaluation metrics of the three numerical cases.}
									\label{tab:RMS_num_cases}
									\begin{tabular}{cccccc}
										\toprule[1.5pt]
										\multirow{2}{*}{Model} & \multirow{2}{*}{case number} & \multicolumn{1}{c}{\multirow{2}{*}{A [mm\textsuperscript{2}]}} & \multicolumn{3}{c}{Predicted output} \\ 
										\cmidrule(lr){4-6} & & & \multicolumn{1}{c}{IoU} & \multicolumn{1}{c}{\(\hat{A}\) [mm\textsuperscript{2}]} & \(\epsilon\) \\
										\midrule
										\multirow{3}{*}{Res-UNet} 							
										& 1 & 257 & \multicolumn{1}{c}{0.45} & \multicolumn{1}{c}{143} & \(44.36\%\) \\ 
										& 2 & 105 & \multicolumn{1}{c}{0.67} & \multicolumn{1}{c}{88} & \(16.19\%\) \\ 
										& 3 & 537 & \multicolumn{1}{c}{0.80} & \multicolumn{1}{c}{478} & \(10.99\%\) \\ 
										\midrule
										\multirow{3}{*}{VGG16 encoder-decoder} 
										& 1 & 257 & \multicolumn{1}{c}{0.69} & \multicolumn{1}{c}{203} & \(21.01\%\) \\ 
										& 2 & 105 & \multicolumn{1}{c}{0.75} & \multicolumn{1}{c}{117} & \(11.43\%\) \\ 
										& 3 & 537 & \multicolumn{1}{c}{0.65} & \multicolumn{1}{c}{385} & \(28.31\%\) \\ 
										\midrule
										\multirow{3}{*}{FCN-DenseNet} 
										& 1 & 257 & \multicolumn{1}{c}{0.52} & \multicolumn{1}{c}{505} & \(96.50\%\) \\ 
										& 2 & 105 & \multicolumn{1}{c}{0.66} & \multicolumn{1}{c}{118} & \(12.38\%\) \\ 
										& 3 & 537 & \multicolumn{1}{c}{0.72} & \multicolumn{1}{c}{815} & \(51.77\%\) \\ 
										\midrule
										\multirow{3}{*}{PSPNet} 
										& 1 & 257 & \multicolumn{1}{c}{0.00} & \multicolumn{1}{c}{0} & \(-\%\) \\ 
										& 2 & 105 & \multicolumn{1}{c}{0.44} & \multicolumn{1}{c}{156} & \(48.57\%\) \\ 
										& 3 & 537 & \multicolumn{1}{c}{0.77} & \multicolumn{1}{c}{610} & \(13.59\%\) \\ 
										\midrule
										\multirow{3}{*}{GCN} 
										& 1 & 257 & \multicolumn{1}{c}{0.71} & \multicolumn{1}{c}{215} & \(16.34\%\) \\ 
										& 2 & 105 & \multicolumn{1}{c}{0.72} & \multicolumn{1}{c}{177} & \(68.57\%\) \\ 
										& 3 & 537 & \multicolumn{1}{c}{0.86} & \multicolumn{1}{c}{523} & \(2.61\%\) \\ 
										\bottomrule[1.5pt]
									\end{tabular}	
								\end{table}
			%%%%%%%%%%%%%%%%%%%%%%%%%%%%%%%%%%%%%%%%%%%%%%%%%%%%%%%%%%	
							\end{column}
						\hfill
			\begin{column}[c]{0.9\textwidth}
				\begin{table}[ht!]
					\centering
					\caption{Analysis of numerical cases.}
					\label{tab:table_all_numerical_cases}	
					\begin{tabular}{lcc}
						\toprule[1.5pt]
						Model & mean IoU & max IoU \\ 
						\midrule 
						Res-UNet & \(0.66\) & \(0.89\) \\ 
						VGG16 encoder-decoder & \(0.57\) & \(0.84\) \\ 
						FCN-DenseNet & \(0.68\) & \(0.92\) \\ 
						PSPNet & \(0.55\) & \(0.91\) \\ 
						GCN & \(0.76\) & \(0.93\) \\ 
						\bottomrule[1.5pt]
					\end{tabular}
				\end{table}
			\end{column}
		\end{columns}
	\end{frame}
	%%%%%%%%%%%%%%%%%%%%%%%%%%%%%%%%%%%%%%%%%%%%%%%%%%%%%%%%%%%%%%%%%%%%%%%%%%%
	\note{
		The table presents the mean and maximum values calculated for the previously unseen numerical test set for all RMS-based models. 
		It also shows that all models have a relatively high value, indicating their ability to detect and localize the delamination.
		However, the best performance was achieved by the GCN model.
	}
	%%%%%%%%%%%%%%%%%%%%%%%%%%%%%%%%%%%%%%%%%%%%%%%%%%%%%%%%%%%%%%%%%%%%%%%%%%%
	\begin{frame}{Numerical test cases animation of Lamb waves}
		\setcounter{subfigure}{0}
		\only<1>{
			\textbf{First test case}
			\begin{figure}
				\centering
				\subfloat[Full wavefield (512 frames)]{\animategraphics[autoplay,loop,height=4cm,keepaspectratio]{32}{figures/gif_figs/381_output/output_381-}{1}{512}}\quad
				\subfloat[RMS of all intermediate predictions]{\includegraphics[height=4.1cm,keepaspectratio]{figures/RMS_Ijjeh_num_case_381.png}}\quad
				\subfloat[Binary RMS, IoU= 0.88]{\includegraphics[height=4cm,keepaspectratio]{figures/Binary_RMS_Ijjeh_num_case381_.png}}\quad
		\end{figure}}
		\setcounter{subfigure}{0}
		\only<2>{
			\textbf{Second test case}
			\begin{figure}
				\centering
				\subfloat[Full wavefield (512 frames)]{\animategraphics[autoplay,loop,height=4cm,keepaspectratio]{32}{figures/gif_figs/385_output/output_385-}{1}{512}}\quad
				\subfloat[RMS of all intermediate predictions]{\includegraphics[height=4.1cm,keepaspectratio]{figures/RMS_Ijjeh_num_case_385.png}}\quad
				\subfloat[Binary RMS, IoU= 0.58]{\includegraphics[height=4cm,keepaspectratio]{figures/Binary_RMS_Ijjeh_num_case385_.png}}
		\end{figure}}
		\setcounter{subfigure}{0}
		\only<3>{
			\textbf{Third test case}
			\begin{figure}
				\centering
				\subfloat[Full wavefield (512 frames)]{\animategraphics[autoplay,loop,height=4cm,keepaspectratio]{32}{figures/gif_figs/394_output/output_394-}{1}{512}}\quad
				\subfloat[RMS of all intermediate predictions]{\includegraphics[height=4.1cm,keepaspectratio]{figures/RMS_Ijjeh_num_case_394.png}}\quad
				\subfloat[Binary RMS, IoU= 0.8]{\includegraphics[height=4cm,keepaspectratio]{figures/Binary_RMS_Ijjeh_num_case394_.png}}
		\end{figure}}
	\end{frame}
	%%%%%%%%%%%%%%%%%%%%%%%%%%%%%%%%%%%%%%%%%%%%%%%%%%%%%%%%%%%%%%%%%%%%%%%%%%%
	\note{
		In the following three slides, I will present three numerical test cases evaluated with the autoencoder ConvLSTM model, which takes animations of full wavefields.
		
		animation (a) shows the full wavefield of 512 frames as an input to the model.
		Figure (b) shows the RMS damage map for all intermediate predictions.
		Figure (c) shows the binary RMS with IoU= 0.88.
		
		The second case is more complex as the delamination is near the corner of the plate.
		In this case, the IoU=0.58.
		
		The third case is also complex, as the delamination is located near the plate edge.
		In this case, the IoU=0.8.
		
		Also, we can notice see how clean all the predictions are.		
	}
	%%%%%%%%%%%%%%%%%%%%%%%%%%%%%%%%%%%%%%%%%%%%%%%%%%%%%%%%%%%%%%%%%%%%%%%%%%%
		\begin{frame}{Animation based: Analysis of numerical cases}
			%%%%%%%%%%%%%%%%%%%%%%%%%%%%%%%%%%%%%%%%%%%%%%%%%%%%%%%%%%%%%%%%%%%%
			\begin{table}[!h]
				\centering
				\caption{Evaluation metrics of the three numerical cases.}
				\begin{tabular}{ccccc}
					\toprule[1.5pt]
					\multirow{2}{*}{case number} & \multicolumn{1}{c}{\multirow{2}{*}{A [mm\textsuperscript{2}]}} & \multicolumn{3}{c}{Predicted output} \\ 
					\cmidrule(lr){3-5} & & \multicolumn{1}{c}{IoU} & \multicolumn{1}{c}{\(\hat{A}\) [mm\textsuperscript{2}]} & \(\epsilon\) \\
					\midrule
					1 & 763 & \multicolumn{1}{c}{0.88} & \multicolumn{1}{c}{735} & \(3.67\%\) \\ 
					2 & 388 & \multicolumn{1}{c}{0.58} & \multicolumn{1}{c}{248} & \(36.08\%\) \\ 
					3 & 297 & \multicolumn{1}{c}{0.80} & \multicolumn{1}{c}{280} & \(5.72\%\) \\			 					
					\bottomrule[1.5pt]
				\end{tabular}	
				\label{tab:num_cases}
			\end{table}			
		\end{frame}
		%%%%%%%%%%%%%%%%%%%%%%%%%%%%%%%%%%%%%%%%%%%%%%%%%%%%%%%%%%%%%%%%%%%%%%%%%%%%
		\note{
			 The shown table presents the evaluation metrics for the autoencoder ConvLSTM model regarding the three numerical cases shown in the previous slide.
			 
			 The table gathers the actual delamination area, predicted delamination area, intersection over union IoU, and percentage area error \(\epsilon\) to each case.
		}
	%%%%%%%%%%%%%%%%%%%%%%%%%%%%%%%%%%%%%%%%%%%%%%%%%%%%%%%%%%%%%%%%%%%%%%%%%%%
		\section{Evaluation: Experimental cases}
	\begin{frame}[t]{Composite specimen}
		\begin{columns}[T]
			\column{0.7\textwidth}
			{\small
				\begin{itemize}
					\item 16 layers set at the same angle \\
					\item carbon: Prepreg GG 205  P (fibres Toray FT 300 - 3K 200 tex), $E=230$ GPa
					\item epoxy resin: IMP503Z-HT by Impregnatex Compositi 
					\item dimensions: 500$\times$500$\times$3.9 mm\\
					\item density: 1522.4~kg/m\textsuperscript{3}
				\end{itemize}
			}
			\column{0.3\textwidth}
			\begin{figure}
				\includegraphics[width=0.6\textwidth]{weave-1.jpg}
				\caption{Plain weave fabric}
			\end{figure}
		\end{columns}
		\begin{table}[h]
			\renewcommand{\arraystretch}{1.1}
			\centering \footnotesize
			\caption{Geometry of a plain weave fabric reinforced composite [mm]}
			\begin{tabular}{cccccc} 
				\hline
				\toprule[1.5pt]
				\multicolumn{4}{c}{\textbf{width} }	& \multicolumn{2}{c}{\textbf{thickness} }  \\ 
					\hline \hline
				\cmidrule(lr){1-4} \cmidrule(lr){5-6} 
				fill & warp & fill gap& warp gap& fill & warp\\
				\hline
				$a_f$ &$a_w$& $g_f$  & $g_w$  & $h_f$& $h_w$ \\ 
				\hline
				\midrule
				\cmidrule(lr){1-2} \cmidrule(lr){3-4} \cmidrule(lr){5-6}
				1.92 &2.0& 0.05& 0.05 & 0.121875 & 0.121875 \\
				\hline 
				\bottomrule[1.5pt] 
			\end{tabular} 
			\label{tab:weave_geo}
		\end{table}
	\end{frame}
	%%%%%%%%%%%%%%%%%%%%%%%%%%%%%%%%%%%%%%%%%%%%%%%%%%%%%%%%%%%%%%%%%%%%%%%%%%%
	\note{
		The composite laminates used for experimental evaluation were composed of 16 layers set at the same angle of plain weave fabric with the following characteristics:
		
		Table 3 shows the fill and warp strands for the plain weave fabric reinforced composite in [mm].
	}
	%%%%%%%%%%%%%%%%%%%%%%%%%%%%%%%%%%%%%%%%%%%%%%%%%%%%%%%%%%%%%%%%%%%%%%%%%%%
	\begin{frame}[t]{Specimens with defects}
		\vspace{-0.5cm}
		\begin{columns}[T]
			\column{0.5\textwidth}
			\begin{figure}
				\includegraphics[scale=0.36]{plate_multi_delam_arrangement_large_fonts.png}
			\end{figure}
			\column{0.5\textwidth}
			\begin{figure}
				\includegraphics[scale=0.36]{plate_single_delam_arrangement_large_fonts.png}
			\end{figure}
		\end{columns}
	\end{frame}
	%%%%%%%%%%%%%%%%%%%%%%%%%%%%%%%%%%%%%%%%%%%%%%%%%%%%%%%%%%%%%%%%%%%%%%%%%%%
	\note{	
		\footnotesize
		In this slide, I present the specifications of the specimens used for evaluating the developed deep-learning models.
		
		Specimens with multiple delaminations are shown on the left.
		
		In Specimen II, three large artificial delaminations of elliptic shape were inserted in the upper thickness quarter of the plate between the 4th and the 5th layer. 
		The delaminations were located at the same distance, equal to 150 mm from the centre of the plate. 
		
		In specimen III, delaminations were inserted in the middle of the cross-section of the plate between the 8th layer and the 9th layer.
		For Specimen IV, three small delaminations were inserted in the upper quarter of the cross-section of the plate, and three large delaminations were inserted at the lower quarter of the cross-section of the plate between the 12th layer and 13th layer. 
		
		The fifth specimen on the right has a single delamination. 
		The delamination between layers of the fabric was created artificially by a Teflon insert of a thickness 250 \(\mu\)m. 
		The Teflon of a square shape was inserted during specimen manufacturing, so its shape and location are known.
		
		Furthermore, all SLDV measurements were conducted from the bottom surface of the plate.
		
		I would like to mention that the specimens on the left were brought specifically to my PhD project.
	}
	%%%%%%%%%%%%%%%%%%%%%%%%%%%%%%%%%%%%%%%%%%%%%%%%%%%%%%%%%%%%%%%%%%%%%%%%%%%
	\begin{frame}[t]{SLDV measurements: laboratory}
		\begin{columns}[T]
			\column{0.5\textwidth}
			\begin{figure}
				\includegraphics[width=0.8\textwidth]{wibrometr-laserowy-1d_small-description.png}
			\end{figure}
			\column{0.5\textwidth}
			\begin{enumerate}
				\item Signal generator: TTI 1241 
				\item Amplifier: Piezo Systems EPA-104-230 $\pm$200 Vp
				\item Specimen
				\item Scanning head: Polytec PSV-400
				\item DAQ system: Polytec
			\end{enumerate}
		\end{columns}
		{\small
			Measurements were taken on a uniform grid of \textbf{333$\times$333 points}.\\
			Excitation in the form of Hann windowed sine signal of carrier frequency \textbf{50 kHz} was applied to piezoelectric transducer.}
	\end{frame}
	%%%%%%%%%%%%%%%%%%%%%%%%%%%%%%%%%%%%%%%%%%%%%%%%%%%%%%%%%%%%%%%%%%%%%%%%%%%
	\note{
		This slide shows the experimental setup at the laboratory for the SLDV measurements.
		
		The Piezoelectric transducer is placed on the other side of the plate towards the wall.
		
		The measurements were taken on a uniform grid of \(333\times 333\) points from the bottom surface of the plate.
		
		The excitation in the form of a Hann windowed sine signal of carrier frequency 50 kHz was applied to the PZT. 
	}
	%%%%%%%%%%%%%%%%%%%%%%%%%%%%%%%%%%%%%%%%%%%%%%%%%%%%%%%%%%%%%%%%%%%%%%%%%%%
	
			\begin{frame}{Experimental setup}
				\begin{columns}[T]
					\begin{column}[t]{0.55\textwidth}
						\begin{figure}
							\centering
							\includegraphics[width=.9\textwidth]{wibrometr-laserowy-1d_small-description.png}
						\end{figure}
					\end{column}
					\begin{column}[t]{0.4\textwidth}
						\begin{enumerate}
							\item Waveform generator
							\item Power amplifier	
							\item Specimen
							\item SLDV head
							\item DAQ
						\end{enumerate}
					\end{column}
				\end{columns}
			\end{frame}
	
			\begin{frame}{Single delamination arrangement}
				\begin{minipage}[c]{0.4\textwidth}
					\begin{itemize}[<alert@+>]
						\item 
						\item 
						\item 
					\end{itemize}
				\end{minipage}
				\begin{minipage}[c]{0.55\textwidth}
					\centering
					\includegraphics[width=.7\textwidth]{plate_single_delam_arrangement_large_fonts.jpg}
				\end{minipage}
			\end{frame}
	%%%%%%%%%%%%%%%%%%%%%%%%%%%%%%%%%%%%%%%%%%%%%%%%%%%%%%%%%%%%%%%
	\setcounter{subfigure}{0}
	\begin{frame}{Experimental results RMS based (Single delamination)}
		\begin{columns}[T]
			\begin{column}[c]{0.28\textwidth}
				\small
				Adaptive wavenumber filtering (AWF) \alert{(Conventional signal processing)}.
				\begin{small}
					\biblioref{Kudela P, Radzienski M, Ostachowicz W.}{2018}{Impact induced damage assessment by means of Lamb wave image processing}{Mechanical Systems and Signal Processing,102:23-36}						
										\biblioref{Kudela, P., Radzieński, M. and Ostachowicz, W.}{2015}{ Identification of cracks in thin-walled structures by means of wavenumber filtering}{Mechanical Systems and Signal Processing, 50, pp.456-466}					.
				\end{small}
			\end{column}
			\hfill
			\begin{column}[c]{0.7\textwidth}
				\centering
				\begin{figure}[ht!]
					\subfloat[ERMS \& label]{\includegraphics[width=.27\textwidth]{ERMS_with_label.png}}\quad
					\subfloat[AWF]{\includegraphics[width=.27\textwidth]{ERMSF_CFRP_teflon_3o_375_375p_50kHz_5HC_x12_15Vpp.png}}\quad
					\subfloat[Binary AWF: IoU=$0.401$]{\includegraphics[width=.27\textwidth]{Binary_ERMSF_CFRP_teflon_3o_375_375p_50kHz_5HC_x12_15Vpp.png}}\quad
					\subfloat[GCN: IoU\(=0.723\)]{\includegraphics[width=.27\textwidth]{Fig_GCN_7.png}}\quad
					\subfloat[FCN-DenseNet: IoU=$0.54$]{\includegraphics[width=.27\textwidth]{Fig_FCN_DenseNet_7.png}}\qquad
				\end{figure} 
			\end{column}
		\end{columns}	
	\end{frame}
	%%%%%%%%%%%%%%%%%%%%%%%%%%%%%%%%%%%%%%%%%%%%%%%%%%%%%%%%%%%%%%%%%%%%%%%%%%%
	\note{
		This slide shows the experimental results of a specimen with a single delamination regarding the RMS-based models, GCN, and FCN-DenseNet.
		
		Additionally, the deep learning models were compared with the adaptive wavenumber filtering technique for damage imaging, which is a conventional signal processing technique developed in our department.
		
		Figure (a) shows the ERMS with a label depicting the delamination. 
		
		Figure (b) shows the result of applying the adaptive wavenumber filtering method, and figure (c) shows its binary output with IoU=0.401.
		
		Figure (d) shows the predicted damage map by the GCN model with IoU=0.723.
		Figure (e) shows the predicted damage map by the FCN-DenseNet model with IoU=0.54
		
		As shown, the deep learning models surpasses the conventional signal processing approach.
	}
	%%%%%%%%%%%%%%%%%%%%%%%%%%%%%%%%%%%%%%%%%%%%%%%%%%%%%%%%%%%%%%%%%%%%%%%%%%%
	\setcounter{subfigure}{0}
	\begin{frame}{RMS based: Analysis of experimental case}
		\begin{table}[!ht]
			\centering
			\caption{Evaluation metrics of the experimental case.}
			\label{tab:rms_exp_case}
			\begin{tabular}{lc}
				\toprule[1.5pt]
				Model & IoU  	\\			
				\midrule
				Res-UNet & 0.58 \\ 
				VGG16 encoder-decoder & 0.62 \\ 
				FCN-DenseNet & 0.54 \\ 
				PSPNet & 0.49 \\ 
				GCN & 0.72\\ 
				\bottomrule[1.5pt]
			\end{tabular}		
		\end{table}
					\begin{table}[!ht]
						\centering
						\caption{Evaluation metrics of the experimental case.}
						\label{tab:rms_exp_case}
						\begin{tabular}{l@{\ }cccc}
							\toprule
							\multicolumn{1}{l}{Model} & \multicolumn{1}{c}{A [mm\textsuperscript{2}]} & \multicolumn{3}{c}{Predicted output} \\ 
							\cmidrule(lr){3-5} & & \multicolumn{1}{c}{IoU} & \multicolumn{1}{c}{\(\hat{A}\) [mm\textsuperscript{2}]} & \(\epsilon\) \\ \midrule
							Res-UNet & \multicolumn{1}{c}{\multirow{5}{*}{210}} & \multicolumn{1}{c}{0.58} & \multicolumn{1}{c}{323}  & \(53.8\%\) \\ 
							VGG16 encoder-decoder &  & \multicolumn{1}{c}{0.62} & \multicolumn{1}{c}{320} & \(52.4\%\) 
							\\ 
							FCN-DenseNet &  & \multicolumn{1}{c}{0.54} & \multicolumn{1}{c}{386} & \(83.8\%\) \\ 
							PSPNet &  & \multicolumn{1}{c}{0.49} & \multicolumn{1}{c}{580} & \(176.2\%\) 
							\\ 
							GCN &  & \multicolumn{1}{c}{0.72} & \multicolumn{1}{c}{309} & \(47.1\%\) 
							\\ 
							\bottomrule
						\end{tabular}		
					\end{table}
	\end{frame}
	%%%%%%%%%%%%%%%%%%%%%%%%%%%%%%%%%%%%%%%%%%%%%%%%%%%%%%%%%%%%%%%%%%%%%%%%%%%
	\note{
		The table shows the IoU values for all developed RMS based models for the single delamination specimen.
		
		Similarly to the numerical dataset, the best accuracy was achieved by using GCN.
		
				As shown, the models are capable of precise detection and localisation of the delamination. 
				We can see that the models can identify the delamination with almost free noise indicating the models are capable of generalising and detecting the delamination on previously unseen data. 
	}
	%%%%%%%%%%%%%%%%%%%%%%%%%%%%%%%%%%%%%%%%%%%%%%%%%%%%%%%%%%%%%%%%%%%%%%%%%%%
	\setcounter{subfigure}{0}
	\begin{frame}{Experimental results full wavefield based (Single delamination)}
		\begin{figure}[ht!]
			\centering
			\subfloat[Full wavefield (256 frames)]{\animategraphics[autoplay,loop,height=3cm]{32}{figures/gif_figs/CFRP_teflon_3o_375_375p_50kHz_5HC_x12_15Vpp/CFRP_teflon_30-}{1}{256}}\quad
			\subfloat[Intermidate ouputs]{\animategraphics[autoplay,loop,height=3cm]{24}{figures/gif_figs/CFRP_ijjeh_single_delamination/intermediate_output-}{0}{231}}\quad
			\subfloat[RMS]{\includegraphics[height=3cm,keepaspectratio]{figures/RMS_CFRP_teflon_3o_375_375p_50kHz_5HC_x12_15Vpp_Ijjeh_updated_results_.png}}\quad
			\subfloat[Binary RMS]{\includegraphics[height=3cm,keepaspectratio]{figures/Binary_RMS_CFRP_teflon_3o__375_375p_50kHz_5HC_x12_15Vpp_Ijjeh_.png}}
		\end{figure}
		IoU= $0.41$ for the thresholded damage map.% and $\epsilon=71.56\%$  
	\end{frame}
	%%%%%%%%%%%%%%%%%%%%%%%%%%%%%%%%%%%%%%%%%%%%%%%%%%%%%%%%%%%%%%%%%%%%%%%%%%%
	\note{
		In this slide, I present the predicted results using the autoencoder ConvLSTM model regarding the single delamination case.
		
		Animation (a) shows the full wavefield measured by SLDV with 256 frames.
		
		Animation (b) shows the intermediate predictions of the model. 
		
		Figure (c) shows the RMS of all intermediate predictions.
		
		And finally, Figure (d) shows the binary RMS with IoU= 0.41		
	}
	%%%%%%%%%%%%%%%%%%%%%%%%%%%%%%%%%%%%%%%%%%%%%%%%%%%%%%%%%%%%%%%%%%%%%%%%%%%
	\setcounter{subfigure}{0}	
	\begin{frame}{Experimental results full wavefield based (Multiple delaminations)}
		\begin{columns}[T]
			\begin{column}[t]{0.24\textwidth}
				\begin{figure}[ht!]
					\centering
					\subfloat[Full wavefield (512 frames)]{\animategraphics[autoplay,loop,height=0.25\textheight]{32}{figures/gif_figs/input_specimen_3/specimen_3-}{1}{512}}
				\end{figure}
			\end{column}	
						\quad				
			\begin{column}[t]{0.24\textwidth}
				\begin{figure}
					\centering
					\subfloat[AWF]{\includegraphics[height=0.25\textheight]{figures/mul/figure17a.png}}
					\\
					\centering
					\subfloat[Binary AWF, IoU$=0.04$]{\includegraphics[height=0.25\textheight]{figures/mul/figure17b.png}}								
				\end{figure}
			\end{column}
			\begin{column}[t]{0.24\textwidth}
				\begin{figure}
					\centering
					\subfloat[Intermidate ouputs]{\animategraphics[autoplay,loop,height=0.25\textheight]{24}{figures/gif_figs/Intermediate_specimen_3/Intermediate_specimen_3-}{0}{487}}
				\end{figure}
			\end{column}
						\quad
			\begin{column}[t]{0.24\textwidth}
				\begin{figure}
										\hspace{5pt}
					\centering
					\subfloat[RMS]{\includegraphics[height=0.25\textheight,keepaspectratio]{figures/RMS_L3_S3_B_333x333p_50kHz_5HC_18Vpp_x10_pzt_Ijjeh_updated_results_.png}}
					\\
					\centering
					\subfloat[Binary RMS]{\includegraphics[height=0.25\textheight,keepaspectratio]{figures/Binary_RMS_L3_S3_B__333x333p_50kHz_5HC_18Vpp_x10_pzt_Ijjeh_.png}}
				\end{figure}
			\end{column}				
		\end{columns}	
		IoU= $0.64$ and $\epsilon=1.69\%$  for the thresholded damage map (Binary RMS).		
	\end{frame}
	%%%%%%%%%%%%%%%%%%%%%%%%%%%%%%%%%%%%%%%%%%%%%%%%%%%%%%%%%%%%%%%%%%%%%%%%%%%
	\note{
		This slide presents the predicted results using the autoencoder ConvLSTM model compared to conventional signal processing adaptive wavenumber filtering regarding the multiple delamination case.
		
		Animation (a)  shows the full wavefield measured by SLDV of 512 frames
		
		Figure (b) shows the damage map resulting from applying adaptive wavenumber filtering and figure (c) shows its binary output with IoU = 0.04
		
		Animation (d)  shows all intermediate predictions with the deep learning model
		
		Figure (e) shows the RMS of all intermediate predictions and figure (f) shows the binary RMS with IoU = 0.64
		
		It can be concluded that utilising animations of Lamb waves propagation has better outcomes for delamination identification than the processing of a single image representing signal energy or RMS.		
	}
	%%%%%%%%%%%%%%%%%%%%%%%%%%%%%%%%%%%%%%%%%%%%%%%%%%%%%%%%%%%%%%%%%%%%%%%%%%%
	\setcounter{subfigure}{0}
	\section{Super-resolution image reconstruction}
	\begin{frame}{Super-resolution image reconstruction}
		\begin{columns}[T]
			\begin{column}[c]{0.35\textwidth}
				\justifying	
				\small	
				Deep learning model for superresulution image reconstruction aims to speed up the data acquisition process for SLDV by recovering HR full wavefield frames with satisfying accuracy from the LR measurement that is below Nyquist sampling rate.				
			\end{column}
			\begin{column}[c]{0.6\textwidth}
				\begin{figure}
					\subfloat{\includegraphics[width=0.95\textwidth]{superresolution_flowchart.jpeg}}
				\end{figure}
			\end{column}
		\end{columns}
	\end{frame}
	%%%%%%%%%%%%%%%%%%%%%%%%%%%%%%%%%%%%%%%%%%%%%%%%%%%%%%%%%%%%%%%%%%%%%%%%%%%
	\note{
		Scanning laser Doppler vibrometer is a popular tool for the acquisition of the full wavefield of propagating guided waves, in particular Lamb waves.
		
		However, the process of acquiring the full wavefield of guided waves is time-consuming. 
		
		One possible solution to tackle this problem is to acquire the Lamb waves in a low-resolution form and then apply a deep learning-based super-resolution approach to that low-resolution frame of full wavefield data to retrieve the high-resolution frame.
	}
	%%%%%%%%%%%%%%%%%%%%%%%%%%%%%%%%%%%%%%%%%%%%%%%%%%%%%%%%%%%%%%%%%%%%%%%%%%%
	\setcounter{subfigure}{0}
	\begin{frame}{Compressive sensing theory}
		\tiny
		\begin{columns}[T]
			\begin{column}[t]{0.48\textwidth}
				\justifying
				Compressed sensing (CS) theory $\rightarrow$ any natural signal (\(x\)), e.g. (sounds, images, $\dots$) \alert{can be recovered using considerably fewer measurements than standard methods}.
				\\
				CS relies on two principles:
				\begin{itemize}
					\item \alert{Sparsity}: which relates to the signal of interest.
					\begin{equation*}
						x=\Psi s,
					\end{equation*}
					where \(\Psi\) is the universal basis (in this work, Fourier domain was applied), \(s\) is a sparse vector of coefficients  (\alert{most of the coefficients are equal or close to zero}).
					\item \alert{Incoherence}: which relates to the sensing modality.
					\begin{gather*}
						y=Cx, \\
						y=C\Psi s,
					\end{gather*}
					where \(y\) is the measurements in \alert{Low-Resolution (LR)} (below the Nyquist sampling rate),
					\(C\) is the mask matrix applied to the \(x\).\\
					This system of equations is \alert{underdetermined}.
					\\
					$C$ and \(\Psi\) matrices must be incoherent \alert{(smallest correlation between any two elements in $C$ and $\Psi$)}
				\end{itemize}		
			\end{column}			
			\begin{column}[t]{0.48\textwidth}
				\justifying
				\alert{The goal is to find sparsest \(s\) vector} that solve the underdetermined system of equation.
				\\ This is an optimization problem:				
				\begin{equation}
					\min{\lVert {\bs{s}} \rVert}_1 \quad \textrm{subject to} \quad {\lVert \bs{C} \bs{\Psi} \bs{s} -y \rVert}_2 \leq \sigma ,
				\end{equation}
				where $\sigma$ is related to the noise level in the data.
				\begin{figure}[ht!]
					\subfloat[\centering \small Random mask (3000 points)]{\includegraphics[width=0.40\textwidth]{random_mask_3000.png}}\quad
					\subfloat[\centering \small Jitter mask (3000 points)]{\includegraphics[width=0.40\textwidth]{jitter_mask_3000.png}}					
				\end{figure}										
			\end{column}		
		\end{columns}						
	\end{frame}
	\note
	{	
		\tiny
		The theory of compressed sensing (CS) states that natural signals (such as sounds and images) can be recovered using considerably fewer samples or measurements (below the Nyquist sampling rate) than standard methods.
		
		Now, CS relies on two essential conditions: 	
		\begin{itemize}
			\item The first one is Sparsity: which means that(\alert{most of the coefficients in vector \(s\) are equal or close to zero}).
			$\Psi$ represent a universal basis such as Fourier, cosine or wavelet domains.
			In this work the Fourier domain was applied. 
			\item The second one is that the measuring mask C matrix must be incoherent with \(\Psi\) matrix, which means the correlation between any two elements in C and $\Psi$ matrices is small.
		\end{itemize}		
		
		regarding to this equation: 
		\begin{equation}
			y = \bs{C}\bs{\Psi}\bs{s}
		\end{equation}
		There are an infinite number of solutions for the sparse vector (s) that can solve the y vector.
		
		However, we need to find the sparsest s vector.
		
		Therefore, it is an optimization problem that can be solved by minimizing the \(L_1\) norm of s such that \(L_2\) norm \(\bs{C}\bs{\Psi}\bs{s}\)-\(\bs{y}\) equals zero, or some noise $\sigma$.
		
		Now that we have this sparsest vector (s), we can use the inverse Fourier transform to recover the high-resolution signal.		
		
	}
	%%%%%%%%%%%%%%%%%%%%%%%%%%%%%%%%%%%%%%%%%%%%%%%%%%%%%%%%%%%%%%%%%%%%%%%%%%%
	\setcounter{subfigure}{0}
	\begin{frame}{Evaluation metrics for SR model}
		For evaluating the reconstructed full wavefield
		\begin{itemize}
			\item Peak signal-to-noise ratio (PSNR):
			\begin{equation*}
				PSNR=20\log_{10}\left(\frac{R}{\sqrt{MSE}}\right)
				\label{PSNR}
			\end{equation*}
			\item Pearson correlation coefficient:
			\begin{equation*}
				r_{xy} = \frac{\sum_{i=1}^{n}(x_i - \bar{x})(y_i-\bar{y})}{\sqrt{\sum_{i=1}^{n}(x_i - \bar{x})^2}\sqrt{\sum_{i=1}^{n}(y_i - \bar{y})^2}}
				\label{Pearson}
			\end{equation*}
		\end{itemize}
	\end{frame}
	%%%%%%%%%%%%%%%%%%%%%%%%%%%%%%%%%%%%%%%%%%%%%%%%%%%%%%%%%%%%%%%%%%%%%%%%%%%
	\note{
		To evaluate the developed model, I used two metrics:
		\begin{itemize}
			\item The first one is the peak signal-to-noise ratio (PSNR), which refers to the maximum possible power of a signal and the power of the distorting noise that affects the quality of its representation.
						This equation depicts the mathematical representation of PSNR, where R is the maximum fluctuation value in the input image, and MSE is the mean squared error between the actual and predicted output.
						
			\item The second metric is the Pearson correlation coefficient (Pearson CC), which measures the linear relationship between two
			variable sets \(X\) (represents the ground truth values) and \(Y\) (represents the predicted values) as shown in the below equation.
			
						Where \(n\) is the number of sample points,\(x_i, y_i\) are the individual value points representing the ground truth and predicted values, respectively, and \(\bar{x}\) and \(\bar{y}\) are the mean values of the sample and the prediction.
		\end{itemize}
	}
	%%%%%%%%%%%%%%%%%%%%%%%%%%%%%%%%%%%%%%%%%%%%%%%%%%%%%%%%%%%%%%%%%%%%%%%%%%%
	\setcounter{subfigure}{0}
	\begin{frame}{Numerical test cases at certain frames}
		\begin{columns}[T]				
			\begin{column}[c]{0.5\textwidth}
								\centering
				\only<1>{\textbf{First test case}}
				\only<2>{\textbf{Second test case}}
				\only<3>{\textbf{Third test case}}
				\begin{figure}
					\centering
					%%%%%%%%%%%%%%%%%%%%%%%%%%%%%%%%%%%%%%%%%%%%%%%%%%%%%%%%%%%
					\only<1>{
						\begin{figure}
							\includegraphics[height=.45\textheight]{LR_397_frame_127_input.png}
							\caption{LR input, $f_n=127$}
						\end{figure}						
						%%%%%%%%%%%%%%%%%%%%%%%%%%%%%%%%%%%%%%%%%%%%%%%%%%%%%%%
						\begin{table}[!h]
							\centering 
							\footnotesize
							\begin{tabular}{cccc}
								\toprule
								\multicolumn{2}{c}{plate} & \multicolumn{2}{c}{delamination} \\
								\cmidrule(lr){1-2} \cmidrule(lr){3-4}
								PSNR & PEARSON CC & PSNR & PEARSON CC \\ 
								\midrule
								42.95 & 0.999 & 33.02 & 0.993  \\					
								\bottomrule
							\end{tabular}
						\end{table}
						%%%%%%%%%%%%%%%%%%%%%%%%%%%%%%%%%%%%%%%%%%%%%%%%%%%%%%%
					}
					%%%%%%%%%%%%%%%%%%%%%%%%%%%%%%%%%%%%%%%%%%%%%%%%%%%%%%%%%%%
					\only<2>{
						\begin{figure}
							\includegraphics[height=.45\textheight]{LR_438_frame_154_input.png}
							\caption{LR input, $f_n=154$}
						\end{figure}
						
						%%%%%%%%%%%%%%%%%%%%%%%%%%%%%%%%%%%%%%%%%%%%%%%%%%%%%%%
						\begin{table}[!h]
							\centering 
							\footnotesize
							\begin{tabular}{cccc}
								\toprule
								\multicolumn{2}{c}{plate} & \multicolumn{2}{c}{delamination} \\
								\cmidrule(lr){1-2} \cmidrule(lr){3-4}
								PSNR & PEARSON CC & PSNR & PEARSON CC \\ 
								\midrule
								47.00 & 0.998 & 38.52 & 0.995 \\					
								\bottomrule
							\end{tabular}
							\label{tab:num_DLSR_results_2}
						\end{table}	
						%%%%%%%%%%%%%%%%%%%%%%%%%%%%%%%%%%%%%%%%%%%%%%%%%%%%%%%
					}
					%%%%%%%%%%%%%%%%%%%%%%%%%%%%%%%%%%%%%%%%%%%%%%%%%%%%%%%%%%%
					\only<3>{
						\begin{figure}
							\includegraphics[height=.45\textheight]{LR_456_frame_159_input.png}
							\caption{LR input, $f_n=159$}
						\end{figure}
						
						%%%%%%%%%%%%%%%%%%%%%%%%%%%%%%%%%%%%%%%%%%%%%%%%%%%%%%%
						\begin{table}[!h]
							\centering 
							\footnotesize	
							\begin{tabular}{cccc}
								\toprule
								\multicolumn{2}{c}{plate} & \multicolumn{2}{c}{delamination} \\
								\cmidrule(lr){1-2} \cmidrule(lr){3-4}
								PSNR & PEARSON CC & PSNR & PEARSON CC \\ 
								\midrule
								48.60 & 0.998 & 46.67 & 0.998 \\					
								\bottomrule
							\end{tabular}
						\end{table}
						%%%%%%%%%%%%%%%%%%%%%%%%%%%%%%%%%%%%%%%%%%%%%%%%%%%%%%%
					}
					%%%%%%%%%%%%%%%%%%%%%%%%%%%%%%%%%%%%%%%%%%%%%%%%%%%%%%%%%%%
				\end{figure}
			\end{column}
			%%%%%%%%%%%%%%%%%%%%%%%%%%%%%%%%%%%%%%%%%%%%%%%%%%%%%%%%%%%%%%%%%%%
			\begin{column}[c]{0.5\textwidth}
				\only<1>{	
					\setcounter{subfigure}{0}				
					%%%%%%%%%%%%%%%%%%%%%%%%%%%%%%%%%%%%%%%%%%%%%%%%%%%%%%%%%%%
					\begin{figure}
						\centering
						\subfloat[listentry][HR ref]{\includegraphics[height=.35\textheight]{output_397_frame_127_full_frame_GT.png}}\quad
						\subfloat[listentry][SR $f_n$]{\includegraphics[height=.35\textheight]{output_397_frame_127_full_frame_pred.png}}
						\\
						\subfloat[listentry][Ref]{\includegraphics[height=.35\textheight]{output_397_frame_127_delamination_GT.png}}\quad
						\subfloat[listentry][Pred]{\includegraphics[height=.35\textheight]{output_397_frame_127_delamination_pred.png}}
					\end{figure}
					%%%%%%%%%%%%%%%%%%%%%%%%%%%%%%%%%%%%%%%%%%%%%%%%%%%%%%%%%%%
				}
				
				\only<2>{
					\setcounter{subfigure}{0}					
					%%%%%%%%%%%%%%%%%%%%%%%%%%%%%%%%%%%%%%%%%%%%%%%%%%%%%%%%%%%
					\begin{figure}
						\subfloat[listentry][HR ref]{\includegraphics[height=.35\textheight]{output_438_frame_154_full_frame_GT.png}}\quad
						\subfloat[listentry][SR $f_n$]{\includegraphics[height=.35\textheight]{output_438_frame_154_full_frame_pred.png}}
						\\
						\subfloat[listentry][Ref]{\includegraphics[height=.35\textheight]{output_438_frame_154_delamination_GT.png}}\quad	
						\subfloat[listentry][Pred]{\includegraphics[height=.35\textheight]{output_438_frame_154_delamination_pred.png}}
					\end{figure}
					%%%%%%%%%%%%%%%%%%%%%%%%%%%%%%%%%%%%%%%%%%%%%%%%%%%%%%%%%%%
				}
				\only<3>{	
					\setcounter{subfigure}{0}				
					\begin{figure}
						\subfloat[listentry][HR ref]{\includegraphics[height=.35\textheight]{output_456_frame_159_full_frame_GT.png}}\quad
						\subfloat[listentry][SR $f_n$]{\includegraphics[height=.35\textheight]{output_456_frame_159_full_frame_pred.png}}
						\\
						\subfloat[listentry][Ref]{\includegraphics[height=.35\textheight]{output_456_frame_159_delamination_GT.png}}\quad
						\subfloat[listentry][Pred]{\includegraphics[height=.35\textheight]{output_456_frame_159_delamination_pred.png}}
					\end{figure}
					%%%%%%%%%%%%%%%%%%%%%%%%%%%%%%%%%%%%%%%%%%%%%%%%%%%%%%%%%%%
				}
			\end{column}
		\end{columns}
	\end{frame}
	%%%%%%%%%%%%%%%%%%%%%%%%%%%%%%%%%%%%%%%%%%%%%%%%%%%%%%%%%%%%%%%%%%%%%%%%%%%
	\note{
		\tiny
		In the following, the results of the reconstruction of HR frames for three numerical test cases will be presented.
		
		In all test cases, I used the first frame, which shows the initial interaction with delamination.
		
		In the first test case, the figure shows the low-resolution measurements at frame number 127.
		Figure a shows the actual HR frame, and figure b shows the predicted SR frame. 
		The PSNR value is 42.95
		
		Figure c shows the HR sub-frame at the delamination region, figure d shows the SR prediction at the delamination region, and the PSNR is 33.02.
		
		In the second test case, the figure shows the low-resolution measurements at frame number 154.		
		Figure a shows the actual HR frame, and figure b shows the predicted SR frame. 
		The PSNR value is 47
		
		Figure c shows the HR sub-frame at the delamination region, figure d shows the SR prediction at the delamination region, and the PSNR is 38.52.
		
		In the third test case, the figure shows the low-resolution measurements at frame number 159.
		Figure a shows the actual HR frame, and figure b shows the predicted SR frame. 
		The PSNR value is 48.6
		
		Figure c shows the HR sub-frame at the delamination region,  figure d shows the SR prediction at the delamination region, and the PSNR is 46.67.		
	}
		%%%%%%%%%%%%%%%%%%%%%%%%%%%%%%%%%%%%%%%%%%%%%%%%%%%%%%%%%%%%%%%%%%%%%%%%%%%%
		\setcounter{subfigure}{0}
		\begin{frame}{Analysis of numerical cases}
			\begin{table}[!h]
				\centering \footnotesize
				\caption{Quality metrics for the DLSR model for three numerical test cases calculated at frames $N_f$ per case.}	
				\begin{tabular}{lccccc}
					\toprule[1.5pt]
					& & \multicolumn{2}{c}{plate} & \multicolumn{2}{c}{delamination} \\
					\cmidrule(lr){3-4} \cmidrule(lr){5-6}
					Case & $N_f$ & PSNR & PEARSON CC & PSNR & PEARSON CC \\ 
					\midrule
					1  & 127  & 42.95 & 0.999 & 33.02 & 0.993 \\					
					\midrule
					2  & 154 & 47.00 & 0.998 & 38.52 & 0.995 
					\\
					\midrule					
					3  & 159 & 48.60 & 0.998 & 46.67 & 0.998 \\					
					\bottomrule[1.5pt]
				\end{tabular}
				\label{tab:num_DLSR_results}
			\end{table}			
		\end{frame}
		%%%%%%%%%%%%%%%%%%%%%%%%%%%%%%%%%%%%%%%%%%%%%%%%%%%%%%%%%%%%%%%%%%%%%%%%%%%%
		\note{note text}
		%%%%%%%%%%%%%%%%%%%%%%%%%%%%%%%%%%%%%%%%%%%%%%%%%%%%%%%%%%%%%%%%%%%%%%%%%%%%
			\begin{frame}{Experimental setup}
				\begin{figure}
					\centering
					\includegraphics[height=0.8\textheight]{Set_up_CS.png}
				\end{figure}
			\end{frame}
		%%%%%%%%%%%%%%%%%%%%%%%%%%%%%%%%%%%%%%%%%%%%%%%%%%%%%%%%%%%%%%%%%%%%%%%%%%%%
		\note{This }
	%%%%%%%%%%%%%%%%%%%%%%%%%%%%%%%%%%%%%%%%%%%%%%%%%%%%%%%%%%%%%%%%%%%%%%%%%%%
	\setcounter{subfigure}{0}
	\begin{frame}{Experimental test case}
		Low-resolution measurements (Input): \(32\times32=1024\) points. (Nf = 110.)\\
		High-resolution (Output): \(512\times512=262144\) points (Nf = 110.).
		\begin{columns}[T]
			\begin{column}[c]{0.19\textwidth}
				\begin{figure}						
					\includegraphics[width=1\textwidth]{frame110_32x32.png}
					\caption{LR input}
				\end{figure}
			\end{column}
			\begin{column}[c]{0.8\textwidth}
				\begin{figure}[ht!]
					\subfloat[HR ref]{\includegraphics[height=.22\textheight]{figure10a.png}}\quad
					\subfloat[CS: 1024 points]{\includegraphics[height=.22\textheight]{figure10b.png}}\quad
					\subfloat[CS: 3000 points]{\includegraphics[height=.22\textheight]{figure10c.png}}\quad
					\subfloat[CS: 4000 points]{\includegraphics[height=.22\textheight]{figure10d.png}}\quad
					\subfloat[DLSR]{\includegraphics[height=.22\textheight]{figure10e.png}}\quad
					
					\subfloat[Ref]{\includegraphics[height=.22\textheight]{figure11a.png}}\quad
					\subfloat[CS: 1024 points]{\includegraphics[height=.22\textheight]{figure11b.png}}\quad
					\subfloat[CS: 3000 points]{\includegraphics[height=.22\textheight]{figure11c.png}}\quad
					\subfloat[CS: 4000 points]{\includegraphics[height=.22\textheight]{figure11d.png}}\quad
					\subfloat[DLSR]{\includegraphics[height=.22\textheight]{figure11e.png}}\quad
				\end{figure}
			\end{column}				
		\end{columns}
	\end{frame}
	%%%%%%%%%%%%%%%%%%%%%%%%%%%%%%%%%%%%%%%%%%%%%%%%%%%%%%%%%%%%%%%%%%%%%%%%%%%
	\note{
		\tiny
		In this slide, I present an experimental test case, in which the Low-resolution input data was registered by an SLDV at a uniform gird of  \(32\times 32\) points.
		The LR input frame corresponds to the initial interaction with the delamination at frame number 110.
		
		Additionally, the conventional compressive sensing technique was applied as a reference with a various number of points used for the reconstruction of the wavefield. 
		
		Moreover, two types of masks (sub-sampling schemes) for the construction of measurement matrix were tested, namely random mask and jitter mask.
		
		Figure a shows the actual HR reference frame, and figures c, d, and e  show reconstructed HR frames with the CS method at \(N_p = 1024\), \(N_p = 3000\) and \(N_p = 4000\) points, respectively. 
		
		As expected, when decreasing the \(N_p\) measuring points, the CS method is not able to recover HR frames accurately. 
		
		Figure f presents the reconstructed HR frame with the DLSR model. 
		
		Furthermore, figure g presents the reference region of interest we are attempting to recover, which shows reflected waves from damage.
		
		Figures h, i, and j show a poor reconstruction of the reflected waves at \(N_p = 1024\) and \(N_ = 3000\), respectively, using the CS method.
		
		Figure j shows the recovery of the reflected frames at \(N_p = 4000\) using the CS method, in which the reflected waves are recognisable.
		
		Figure k shows the recovery of the reflected waves using the DLSR model.
		The recovered reflected waves from damage is visible, and can be easily recognised, which indicates that the DLSR approach can be more efficient than the conventional CS method.		
	}
	%%%%%%%%%%%%%%%%%%%%%%%%%%%%%%%%%%%%%%%%%%%%%%%%%%%%%%%%%%%%%%%%%%%%%%%%%%%
	\begin{frame}{Analysis of experimental case}
		\begin{table}[!ht]
			\renewcommand{\arraystretch}{1.3}
			\centering \footnotesize
			\caption{Quality metrics for tested methods for various number of points $N_p$ and corresponding compression ratios CR calculated for the frame no $N_f=110$.}	
			\begin{tabular}{lrrrcrc} 
				\toprule[1.5pt]
				& & & \multicolumn{2}{c}{plate} & \multicolumn{2}{c}{delamination} \\
				\cmidrule(lr){4-5} \cmidrule(lr){6-7}
				Method & $N_p$ & CR [\%] & PSNR & PEARSON CC& PSNR & PEARSON CC \\
				\midrule
				\csvreader
				[table head=\toprule,
				late after line=\\ 
				]{table_metrics.csv}{
					1=\one, 2=\two, 3=\three, 4=\four, 5=\five, 6=\six, 7=\seven
				}%
				{\one & \two & \three & \four & \five & \six & \seven }%	
				\bottomrule[1.5pt]
			\end{tabular}	
			\label{tab:csv_results}
		\end{table}
	\end{frame}
	%%%%%%%%%%%%%%%%%%%%%%%%%%%%%%%%%%%%%%%%%%%%%%%%%%%%%%%%%%%%%%%%%%%%%%%%%%%
	\note{
		The following table presents a detailed comparison of the quality metrics for CS methods with applied jitter and random masks and the DLSR model for various numbers of points and the corresponding compression ratios CR. 
		Metrics were calculated for frame number 110 on the whole plate and delamination region, respectively.  
		It should be underlined, that for the number of points = 1024 points, the CS recovery algorithm behaves poorly at the delamination region, whereas the DLSR model is still able to achieve high PSNR and Pearson CC values.	
		
	}
	%%%%%%%%%%%%%%%%%%%%%%%%%%%%%%%%%%%%%%%%%%%%%%%%%%%%%%%%%%%%%%%%%%%%%%%%%%%
	%%%%%%%%%%%%%%%%%%%%%%%%%%%%%%%%%%%%%%%%%%%%%%%%%%%%%%%%%%%%%%%
	\section{Conclusions}
	%%%%%%%%%%%%%%%%%%%%%%%%%%%%%%%%%%%%%%%%%%%%%%%%%%%%%%%%%%%%%%%
	\begin{frame}{Conclusions}
		\footnotesize
		\begin{itemize}
			\item Full wavefields contain rich damage-related information
			\item Full wavefields can be utilised to train deep learning models to perform damage identification in an end-to-end approach
			\item Deep learning models trained on synthetic dataset generalise well and can be applied directly to experimental wavefields
			\item Animation-based deep learning models perform better than image-based models but are more complex and require longer time for training
			\item Deep learning approaches surpass the conventional signal processing techniques
			\item The DLSR model can speed up the process of data acquisition by SLDV (from LR measurements to HR measurements).
		\end{itemize}
	\end{frame}		
	%%%%%%%%%%%%%%%%%%%%%%%%%%%%%%%%%%%%%%%%%%%%%%%%%%%%%%%%%%%%%%%
	\begin{frame}{Publications}
		\vspace{5pt}
		\begin{tiny}					
			\begin{columns}[T]
				\begin{column}[t]{0.48\textwidth}
					\underline{\textbf{Journals}}
					\begin{itemize}
						\justifying
						\item Ijjeh, A., Ullah, S., Radzienski, M. and Kudela, P., 2023. Deep learning super-resolution for the reconstruction of full wavefield of Lamb waves. \textbf{\textit{Mechanical Systems and Signal Processing}}, 186, p.109878.
						\textbf{[200 points]/[IF:8.934]}
						\item Ullah, S., Ijjeh, A.A. and Kudela, P., 2023. Deep learning approach for delamination identification using animation of Lamb waves. 
						\textbf{\textit{Engineering Applications of Artificial Intelligence}}, 117, p.105520.		
						\textbf{[140 points]/[IF:7.802]}
						\item Ijjeh, A.A. and Kudela, P., 2022. Deep learning based segmentation using full wavefield processing for delamination identification: A comparative study. \textbf{\textit{Mechanical Systems and Signal Processing}}, 168, p.108671.
						\textbf{[200 points]/[IF:8.934]}
						\item Ijjeh, A.A., Ullah, S. and Kudela, P., 2021. Full wavefield processing by using FCN for delamination detection. \textit{Mechanical Systems and Signal Processing, 153,} p.107537.		
						\textbf{[200 points]/[IF:8.934]}			
					\end{itemize}					
				\end{column}
				\begin{column}[t]{0.48\textwidth}
					\underline{\textbf{Conference papers}}
					\begin{itemize}
						\justifying
						\item {Ijjeh, A.}, Kudela, P. Convolutional LSTM for delamination imaging in composite laminates. 
						The 4th International Conference on Machine Learning and Intelligent Systems (MLIS 2022), November 8th-11th, 2022, Seoul, Republic of Korea.
						\item {Ijjeh, A.}, Kudela, P. (2023). Delamination Identification Using Global Convolution Networks. In: Rizzo,
						P., Milazzo, A. (eds) European Workshop on Structural Health Monitoring. EWSHM 2022. Lecture Notes
						in Civil Engineering, vol 270. Springer, Cham. https://doi.org/10.1007/978-3-031-07322-9 5.		
						\item {Ijjeh, A.}, Kudela, P. Feasibility Study of Full Wavefield Processing by Using CNN for Delamination
						Detection. 
						Proceedings of the International Conference on Structural Health Monitoring of Intelligent
						Infrastructure, Porto, Portugal, 30 June -2 July 2021, ISSN 2564-3738, pages 709-713.
					\end{itemize}		
					\underline{\textbf{Chapters}}					
					\begin{itemize}
						\justifying
						\item {Abdalraheem Ijjeh}, Deep Learning based Damage Imaging techniques, chapter in: Wybrane zagadnienia
						inżynierii mechanicznej, Praca zbiorowa pod redakcja M. Mieloszyk, T. Ochrymiuka, Wydawnictwo Instytutu
						Maszyn Przepływowych PAN, Gdańsk, 2022, ISBN: 978-83-66928-09-1.
						\item {Abdalraheem Ijjeh}, Data-driven based approach for damage detection, chapter in: Wybrane zagadnienia
						inżynierii mechanicznej, Praca zbiorowa pod redakcja M. Mieloszyk, T. Ochrymiuka, Wydawnictwo Instytutu
						Maszyn Przepływowych PAN, Gdańsk, 2021, ISBN: 978-83-66928-00-8.				
						\item {Abdalraheem Ijjeh}, Machine Learning for SHM: Literature Review, chapter in: Wybrane zagadnienia
						inżynierii mechanicznej, Praca zbiorowa pod redakcja M. Mieloszyk, T. Ochrymiuka, Wydawnictwo Instytutu
						Maszyn Przepływowych PAN, Gdańsk, 2020, ISBN: 978-83-88237-97-3.
					\end{itemize}
				\end{column}		
			\end{columns}
		\end{tiny}
	\end{frame}	
	
	%%%%%%%%%%%%%%%%%%%%%%%%%%%%%%%%%%%%%%%%%%%%%%%%%
	{
		\setbeamercolor{palette primary}{fg=blue, bg=white}
		\begin{frame}[standout]
			Thank you for your listening!\\ \vspace{12pt}
			Questions?\\ \vspace{12pt}
						\url{pk@imp.gda.pl} 
						\par\medskip
			\url{aijjeh@imp.gda.pl}
			\par\medskip
			\par\medskip
			\footnotesize
			The research work was funded by the Polish National Science Center under grant agreement no. 2018/31/B/ST8/00454.
		\end{frame}
	}
	%%%%%%%%%%%%%%%%%%%%%%%%%%%%%%%%%%%%%%%%%%%%%%%%%%%%%%%%%%%%%%%%%%%%%%%%%%%
	\section*{Backup slides}
	\addtocounter{section}{1}
	%%%%%%%%%%%%%%%%%%%%%%%%%%%%%%%%%%%%%%%%%%%%%%%%%%%%%%%%%%%%%%%%%%%%%%%%%%%
	\setcounter{subfigure}{0}
	\begin{frame}{Optimization and Deep Learning}
		\begin{columns}[T]
			\begin{column}[t]{.4\textwidth}
				\begin{itemize}
					\item \alert{Labeled data} $\rightarrow$ (input data has labels/ground truths)
					\item \alert{Loss function} $\rightarrow$ measures error between predicted and the ground truth values
					\item \alert{Learnable parameters} $\rightarrow$ updated during the backpropagation step (optimization e.g. Gradient descent )					
				\end{itemize}
			\end{column}
			\begin{column}[t]{.55\textwidth}
				\begin{figure}[t]
					\centering
					\animategraphics[autoplay,loop,width =1.0\textwidth]{1}{figures/gif_figs/BP/png/BP_technique_}{0}{14}
				\end{figure}
			\end{column}
		\end{columns}		
	\end{frame}
	%%%%%%%%%%%%%%%%%%%%%%%%%%%%%%%%%%%%%%%%%%%%%%%%%%%%%%%%%%%%%%%%%%%%%%%%%%%
	\note{			
		In this work, I used the supervised approach, which means that the input data is labelled.
		
		Accordingly, the idea of supervised learning is to learn a model of how to map the inputs to the outputs.
		
		Initially, when we start training the deep learning model, all learnable parameters, such as weights and biases, start with random values.
		These learnable parameters are updated during the training phase.
		Now, in the forward pass, we feed the data into the model, and it flows through the layers until we get the predicted output.
		Here, we need an objective loss function to estimate the loss (error) between the predicted output and the ground truth.
		
		A well-known optimization algorithm is a gradient descent, which aims to reduce the loss at each step to reach the global minimum value by calculating the gradient and then pushing back the calculated gradients across all the neurons in a technique called backpropagation.
		
		Accordingly, all learnable parameters are updated, which leads to minimizing the loss value.
	}
	%%%%%%%%%%%%%%%%%%%%%%%%%%%%%%%%%%%%%%%%%%%%%%%%%%%%%%%%%%%%%%%%%%%%%%%%%%%
	\begin{frame}{Dataset}
		\footnotesize	
		\begin{itemize}
			\item The input signal was a five-cycle Hann window modulated sinusoidal tone burst. 
			\item The carrier frequency was assumed to be 50 kHz. 
			\item The total wave propagation time was set to 0.75 ms so that the guided wave could propagate to the plate edges and back to the actuator twice. 
			\item The number of time integration steps was 150000, which was selected for the stability of the central difference scheme.
			\item The material was a typical cross-ply CFRP laminate. 
			\item The stacking sequence \([0/90]_4\) was used in the model. 
			\item The properties of a single ply were as follows [GPa]: \(C_{11}=52.55,\ C_{12}=6.51,\ C_{22}=51.83,\ C_{44}=2.93,\ C_{55}=2.92,\ C_{66}=3.81\). 
			\item The assumed mass density was 1522.4 kg/\(m^3\). 
			\item These properties were selected so that wave front patterns and wavelengths simulated numerically are similar to the wavefields measured by the SLDV on CFRP specimens used later on for testing the developed methods for delamination identification. 
			\item The shortest wavelength of the propagating A0 Lamb wave mode was 21.2 mm for numerical simulations and 19.5 mm for experimental measurements, respectively.
		\end{itemize}		
	\end{frame}
	%%%%%%%%%%%%%%%%%%%%%%%%%%%%%%%%%%%%%%%%%%%%%%%%%%%%%%%%%%%%%%%%%%%%%%%%%%%
	\begin{frame}{Rule of mixture and homogenization}	
		\footnotesize		
		\noindent Composite materials have their micro-structure designed in terms of their macroscopic constituents, e.g., fibers in a homogeneous matrix material. 
		By controlling the choice of fibres, their volume fraction, and alignment, the mechanical properties may be tailored to meet specific design requirements.
		\begin{columns}[T]				
			\begin{column}{0.45\textwidth}
				Diagram (a) shows a 'uniaxial fibre-reinforced composite material," and (b) shows how the stress on the composite is carried by the fibres and the matrix. 
				In normal situations, the fibre has a larger Young's modulus than the matrix, and for the continuous fibres shown, where the strain is the same in the matrix and the fibre, the fibre stress is higher than the matrix stress.
			\end{column}
			\begin{column}{0.45\textwidth}
				\begin{figure}
					\includegraphics[width=0.8\textwidth]{rule_of_mixture.png}
					\caption{From: McMahon and Graham, :"The Bicycle and the Walkman," Merion (1992)}
				\end{figure}
			\end{column}
		\end{columns}
		The Young's modulus of the composite is given by the 'rule of mixtures' i.e. \(E_C = E_F V_F + E_MV_M\), also \((V_M + V_F) = 1\) or \(V_M = (1 - V_F )\). 
		The elastic modulus along the fibre direction can be controlled by selecting the volume fraction of the fibres.		
	\end{frame}
	%%%%%%%%%%%%%%%%%%%%%%%%%%%%%%%%%%%%%%%%%%%%%%%%%%%%%%%%%%%%%%%%%%%%%%%%%%%
	%%%%%%%%%%%%%%%%%%%%%%%%%%%%%%%%%%%%%%%%%%%%%%%%%%%%%%%%%%%%%%%%%%%%%%%%%%%
	\begin{frame}{RMS based: Analysis of numerical cases}
		\begin{columns}[T]
			\tiny
			\begin{column}[t]{0.48\textwidth}
				%%%%%%%%%%%%%%%%%%%%%%%%%%%%%%%%%%%%%%%%%%%%%%%%%%%%%%%%%%%
				\begin{table}
					\caption{Evaluation metrics of the three numerical cases.}
					\label{tab:RMS_num_cases}
					\begin{tabular}{cccccc}
						\toprule[1.5pt]
						\multirow{2}{*}{Model} & \multirow{2}{*}{case number} & \multicolumn{1}{c}{\multirow{2}{*}{A [mm\textsuperscript{2}]}} & \multicolumn{3}{c}{Predicted output} \\ 
						\cmidrule(lr){4-6} & & & \multicolumn{1}{c}{IoU} & \multicolumn{1}{c}{\(\hat{A}\) [mm\textsuperscript{2}]} & \(\epsilon\) \\
						\midrule
						\multirow{3}{*}{Res-UNet} 							
						& 1 & 257 & \multicolumn{1}{c}{0.45} & \multicolumn{1}{c}{143} & \(44.36\%\) \\ 
						& 2 & 105 & \multicolumn{1}{c}{0.67} & \multicolumn{1}{c}{88} & \(16.19\%\) \\ 
						& 3 & 537 & \multicolumn{1}{c}{0.80} & \multicolumn{1}{c}{478} & \(10.99\%\) \\ 
						\midrule
						\multirow{3}{*}{VGG16 encoder-decoder} 
						& 1 & 257 & \multicolumn{1}{c}{0.69} & \multicolumn{1}{c}{203} & \(21.01\%\) \\ 
						& 2 & 105 & \multicolumn{1}{c}{0.75} & \multicolumn{1}{c}{117} & \(11.43\%\) \\ 
						& 3 & 537 & \multicolumn{1}{c}{0.65} & \multicolumn{1}{c}{385} & \(28.31\%\) \\ 
						\midrule
						\multirow{3}{*}{FCN-DenseNet} 
						& 1 & 257 & \multicolumn{1}{c}{0.52} & \multicolumn{1}{c}{505} & \(96.50\%\) \\ 
						& 2 & 105 & \multicolumn{1}{c}{0.66} & \multicolumn{1}{c}{118} & \(12.38\%\) \\ 
						& 3 & 537 & \multicolumn{1}{c}{0.72} & \multicolumn{1}{c}{815} & \(51.77\%\) \\ 
						\midrule
						\multirow{3}{*}{PSPNet} 
						& 1 & 257 & \multicolumn{1}{c}{0.00} & \multicolumn{1}{c}{0} & \(-\%\) \\ 
						& 2 & 105 & \multicolumn{1}{c}{0.44} & \multicolumn{1}{c}{156} & \(48.57\%\) \\ 
						& 3 & 537 & \multicolumn{1}{c}{0.77} & \multicolumn{1}{c}{610} & \(13.59\%\) \\ 
						\midrule
						\multirow{3}{*}{GCN} 
						& 1 & 257 & \multicolumn{1}{c}{0.71} & \multicolumn{1}{c}{215} & \(16.34\%\) \\ 
						& 2 & 105 & \multicolumn{1}{c}{0.72} & \multicolumn{1}{c}{177} & \(68.57\%\) \\ 
						& 3 & 537 & \multicolumn{1}{c}{0.86} & \multicolumn{1}{c}{523} & \(2.61\%\) \\ 
						\bottomrule[1.5pt]
					\end{tabular}	
				\end{table}
				%%%%%%%%%%%%%%%%%%%%%%%%%%%%%%%%%%%%%%%%%%%%%%%%%%%%%%%%%	
			\end{column}
			\begin{column}[t]{0.48\textwidth}
				\begin{table}
					\caption{Analysis of numerical cases.}
					\label{tab:table_all_numerical_cases_backup}	
					\begin{tabular}{lcc}
						\toprule[1.5pt]
						Model & mean IoU & max IoU \\ 
						\midrule 
						Res-UNet & \(0.66\) & \(0.89\) \\ 
						VGG16 encoder-decoder & \(0.57\) & \(0.84\) \\ 
						FCN-DenseNet & \(0.68\) & \(0.92\) \\ 
						PSPNet & \(0.55\) & \(0.91\) \\ 
						GCN & \(0.76\) & \(0.93\) \\ 
						\bottomrule[1.5pt]
					\end{tabular}
				\end{table}
			\end{column}
		\end{columns}
	\end{frame}
	%%%%%%%%%%%%%%%%%%%%%%%%%%%%%%%%%%%%%%%%%%%%%%%%%%%%%%%%%%%%%%%%%%%%%%%%%%%
	\begin{frame}{Initial frame interaction}
		\begin{itemize}
			\item The total wave propagation time was set to \(0.75\) ms so that the guided wave could propagate to the plate edges and back to the actuator twice.
			\item The total time of propagation was converted into \(512\) frames of animated Lamb waves.
			\item The calculated group velocity of \(A_0\) mode is about \(1100\ m/s\).
			\item The \((x, y)\) coordinates of the center of the delaminations are known for the numerically generated dataset. 
			Therefore, we can calculate the distance between the center of the plate and the center of the delamination.
			\item As the group velocity of \(A_0\) mode is known, and the distance is known, we can calculate the required time for the propagating wave to reach the center of the delamination. 
		\end{itemize}
		
		When the time of interaction \(t_i\) with the delamination is know, we can approximately convert it to the frame number \(f_n\) as depicted in the equations given below:		
		\begin{gather*}
			t_i = \frac{\sqrt{(x-0.25)^2 +(y-0.25)^2} \ m}{1100\ m/s}
			\\
			f_n = \frac{t_i}{0.75ms} \times 512
		\end{gather*}			
	\end{frame}
	%%%%%%%%%%%%%%%%%%%%%%%%%%%%%%%%%%%%%%%%%%%%%%%%%%%%%%%%%%%%%%%%%%%%%%%%%%%
	\begin{frame}{Animation based: Analysis of numerical cases}
		The mean IoU for all 95 test cases is 0.80, and the \(mean\ \epsilon\) is \(5.6\%\)
		\begin{table}[!h]
			\centering
			\caption{Evaluation metrics of the three numerical cases.}
			\begin{tabular}{ccccc}
				\toprule[1.5pt]
				\multirow{2}{*}{case number} & \multicolumn{1}{c}{\multirow{2}{*}{A [mm\textsuperscript{2}]}} & \multicolumn{3}{c}{Predicted output} \\ 
				\cmidrule(lr){3-5} & & \multicolumn{1}{c}{IoU} & \multicolumn{1}{c}{\(\hat{A}\) [mm\textsuperscript{2}]} & \(\epsilon\) \\
				\midrule
				1 & 763 & \multicolumn{1}{c}{0.88} & \multicolumn{1}{c}{735} & \(3.67\%\) \\ 
				2 & 388 & \multicolumn{1}{c}{0.58} & \multicolumn{1}{c}{248} & \(36.08\%\) \\ 
				3 & 297 & \multicolumn{1}{c}{0.80} & \multicolumn{1}{c}{280} & \(5.72\%\) \\			 					
				\bottomrule[1.5pt]
			\end{tabular}	
			\label{tab:num_cases_}
		\end{table}			
	\end{frame}
	%%%%%%%%%%%%%%%%%%%%%%%%%%%%%%%%%%%%%%%%%%%%%%%%%%%%%%%%%%%%%%%%%%%%%%%%%%
	\begin{frame}{SLDV measurements: setup}
		\begin{figure}
			\includegraphics[width=0.7\textwidth]{sensors_fig4_setup.png}
		\end{figure}
	\end{frame}
	%%%%%%%%%%%%%%%%%%%%%%%%%%%%%%%%%%%%%%%%%%%%%%%%%%%%%%%%%%%%%%%%%%%%%%%%%%%
	\setcounter{subfigure}{0}
	%%%%%%%%%%%%%%%%%%%%%%%%%%%%%%%%%%%%%%%%%%%%%%%%%%%%%%%%%%%%%%%%%%%%%%%%%%%
	\begin{frame}[t]{Conventional signal processing and wavefield imaging}
		\begin{figure}
			\includegraphics[width=0.7\textwidth]{figs2/sensors_fig1_algorithm.png}
		\end{figure}
		\biblioref{M. Radzienski, P. Kudela, A. Marzani, L. de Marchi, W. Ostachowicz}{2019}{ Damage Identification in Various Types of Composite Plates Using Guided Waves Excited by a Piezoelectric Transducer and Measured by a Laser Vibrometer}{Sensors, 19, 1958}
	\end{frame}
	%%%%%%%%%%%%%%%%%%%%%%%%%%%%%%%%%%%%%%%%%%%%%%%%%%%%%%%%%%%%%%%%%%%%%%%%%%%
	\setcounter{subfigure}{0}
	\begin{frame}{RMS based: Analysis of experimental case}
				\begin{table}[!ht]
					\centering
					\caption{Evaluation metrics of the experimental case.}
					\label{tab:rms_exp_case__}
					\begin{tabular}{lc}
						\toprule[1.5pt]
						Model & IoU  	\\			
						\midrule
						Res-UNet & 0.58 \\ 
						VGG16 encoder-decoder & 0.62 \\ 
						FCN-DenseNet & 0.54 \\ 
						PSPNet & 0.49 \\ 
						GCN & 0.72\\ 
						\bottomrule[1.5pt]
					\end{tabular}		
				\end{table}
		\begin{table}[!ht]
			\centering
			\caption{Evaluation metrics of the experimental case.}
			\label{tab:rms_exp_case_}
			\begin{tabular}{l@{\ }cccc}
				\toprule
				\multicolumn{1}{l}{Model} & \multicolumn{1}{c}{A [mm\textsuperscript{2}]} & \multicolumn{3}{c}{Predicted output} \\ 
				\cmidrule(lr){3-5} & & \multicolumn{1}{c}{IoU} & \multicolumn{1}{c}{\(\hat{A}\) [mm\textsuperscript{2}]} & \(\epsilon\) \\ \midrule
				Res-UNet & \multicolumn{1}{c}{\multirow{5}{*}{210}} & \multicolumn{1}{c}{0.58} & \multicolumn{1}{c}{323}  & \(53.8\%\) \\ 
				VGG16 encoder-decoder &  & \multicolumn{1}{c}{0.62} & \multicolumn{1}{c}{320} & \(52.4\%\) 
				\\ 
				FCN-DenseNet &  & \multicolumn{1}{c}{0.54} & \multicolumn{1}{c}{386} & \(83.8\%\) \\ 
				PSPNet &  & \multicolumn{1}{c}{0.49} & \multicolumn{1}{c}{580} & \(176.2\%\) 
				\\ 
				GCN &  & \multicolumn{1}{c}{0.72} & \multicolumn{1}{c}{309} & \(47.1\%\) 
				\\ 
				\bottomrule
			\end{tabular}		
		\end{table}
	\end{frame}
	%%%%%%%%%%%%%%%%%%%%%%%%%%%%%%%%%%%%%%%%%%%%%%%%%%%%%%%%%%%%%%%%%%%%%%%%%%%
	\setcounter{subfigure}{0}
	\begin{frame}{Experimental results full wavefield based (Single delamination)}
		\begin{figure}[ht!]
			\centering
			\subfloat[Full wavefield (512 frames)]{\animategraphics[autoplay,loop,height=3cm]{32}{figures/gif_figs/CFRP_teflon_3o_375_375p_50kHz_5HC_x12_15Vpp/CFRP_teflon_30-}{1}{256}}\quad
			\subfloat[Intermidate ouputs]{\animategraphics[autoplay,loop,height=3cm]{24}{figures/gif_figs/CFRP_ijjeh_single_delamination/intermediate_output-}{0}{231}}\quad
			\subfloat[RMS]{\includegraphics[height=3cm,keepaspectratio]{figures/RMS_CFRP_teflon_3o_375_375p_50kHz_5HC_x12_15Vpp_Ijjeh_updated_results_.png}}\quad
			\subfloat[Binary RMS]{\includegraphics[height=3cm,keepaspectratio]{figures/Binary_RMS_CFRP_teflon_3o__375_375p_50kHz_5HC_x12_15Vpp_Ijjeh_.png}}
		\end{figure}
		IoU= $0.41$ for the thresholded damage map and $\epsilon=71.56\%$  
	\end{frame}
	%%%%%%%%%%%%%%%%%%%%%%%%%%%%%%%%%%%%%%%%%%%%%%%%%%%%%%%%%%%%%%%%%%%%%%%%%%%
		%%%%%%%%%%%%%%%%%%%%%%%%%%%%%%%%%%%%%%%%%%%%%%%%%%%%%%%%%%%%%%%%%%%%%%%%%%%%
	\setcounter{subfigure}{0}
	\begin{frame}{Compressive sensing theory}
		\tiny
		\begin{columns}[T]
			\begin{column}[t]{0.48\textwidth}
				\justifying
				Compressed sensing (CS) theory $\rightarrow$ any natural signal (\(x\)), e.g. (sounds, images, $\dots$) \alert{can be recovered using considerably fewer measurements than standard methods}.
				\\
				CS relies on two principles:
				\begin{itemize}
					\item \alert{Sparsity}: which relates to the signal of interest.
					\begin{equation*}
						x=\Psi s,
					\end{equation*}
					where \(\Psi\) is the universal basis (in this work, Fourier domain was applied), \(s\) is a sparse vector of coefficients (\alert{most of the coefficients are equal or close to zero}).
					\item \alert{Incoherence}: which relates to the sensing modality.
					\begin{gather*}
						y=Cx, \\
						y=C\Psi s,
					\end{gather*}
					where \(y\) is the measurements in \alert{Low-Resolution (LR)} (below the Nyquist sampling rate),
					\(C\) is the mask matrix applied to the \(x\).\\
					This system of equations is \alert{underdetermined}.
					\\
					$C$ and \(\Psi\) matrices must be incoherent \alert{(smallest correlation between any two elements in $C$ and $\Psi$)}
				\end{itemize}		
			\end{column}			
			\begin{column}[t]{0.48\textwidth}
				\justifying
				\alert{The goal is to find sparsest \(s\) vector} that solve the underdetermined system of equation.
				\\ This is an optimization problem:				
				\begin{equation}
					\min{\lVert {\bs{s}} \rVert}_1 \quad \textrm{subject to} \quad {\lVert \bs{C} \bs{\Psi} \bs{s} -y \rVert}_2 \leq \sigma ,
				\end{equation}
				where $\sigma$ is related to the noise level in the data.
				\begin{figure}[ht!]
					\subfloat[\centering \small Random mask (3000 points)]{\includegraphics[width=0.40\textwidth]{random_mask_3000.png}}\quad
					\subfloat[\centering \small Jitter mask (3000 points)]{\includegraphics[width=0.40\textwidth]{jitter_mask_3000.png}}					
				\end{figure}										
			\end{column}		
		\end{columns}						
	\end{frame}
	\note
	{	
		\tiny
		The theory of compressed sensing (CS) states that natural signals (such as sounds and images) can be recovered using considerably fewer samples or measurements (below the Nyquist sampling rate) than standard methods.
		
		Now, CS relies on two essential conditions: 	
		\begin{itemize}
			\item The first one is Sparsity: which means that(\alert{most of the coefficients in vector \(s\) are equal or close to zero}).
			$\Psi$ represent a universal basis such as Fourier, cosine or wavelet domains.
			In this work the Fourier domain was applied. 
			\item The second one is that the measuring mask C matrix must be incoherent with \(\Psi\) matrix, which means the correlation between any two elements in C and $\Psi$ matrices is small.
		\end{itemize}		
		
		regarding to this equation: 
		\begin{equation}
			y = \bs{C}\bs{\Psi}\bs{s}
		\end{equation}
		There are an infinite number of solutions for the sparse vector (s) that can solve the y vector.
		
		However, we need to find the sparsest s vector.
		
		Therefore, it is an optimization problem that can be solved by minimizing the \(L_1\) norm of s such that \(L_2\) norm \(\bs{C}\bs{\Psi}\bs{s}\)-\(\bs{y}\) equals zero, or some noise $\sigma$.
		
		Now that we have this sparsest vector (s), we can use the inverse Fourier transform to recover the high-resolution signal.		
		
	}
	\begin{frame}{Compressive sensing theory}
		\tiny
		A signal $\vec{s}\in \mathbb{R}^n$ can be reconstructed from a linear combination of random measurements $\vec{y} \in \mathbb{R}^m$.
		The general under-sampling problem can be written as:
		\begin{equation}
			\vec{y} = \bs{\Phi} \vec{s},
		\end{equation}
		where $\bs{\Phi} \in \mathbb{R}^{m\times n}$ is the measurement matrix and $m$ is the number of measurements which can be much smaller than the number of samples in the signal ($m<<n$).
		
		The signal $\vec{s}$ can be recovered from the compressed measurements $\vec{y}$, if it has sparse representation in some model basis $\bs{\Psi} \in \mathbb{R}^{n\times n}$, which can be written as:
		\begin{equation}
			\vec{s} = \bs{\Psi} \bs{\alpha},
		\end{equation}
		where $\bs{\alpha}$ has $K$ nonzero elements ($K<m<n$) and signal $\vec{s}$ is called $K$-sparse. 
		Moreover, the measurement matrix $\bs{\Phi}$ has to be incoherent with the model basis $\bs{\Psi}$.
		In practical applications, signals contain noise, hence:
		\begin{equation}
			\vec{y} = \bs{\Phi} \bs{\Psi} \bs{\alpha} + \vec{z},
			\label{eq:cs_with_noise}
		\end{equation}
		where $\vec{z}$ represents noise.
		The recovery problem can be solved by the basis pursuit denoising algorithm by relaxing Eq.~(\ref{eq:cs_with_noise}) and using sparsity promoting $L_1$ norm which can be written as:
		\begin{equation}
			\min{\lVert \tilde{\bs{\alpha}} \rVert}_1 \quad \textrm{subject to} \quad {\lVert \bs{\Phi} \bs{\Psi} \tilde{\bs{\alpha}} -\vec{y} \rVert}_2 \leq \sigma ,
		\end{equation}
		where $\sigma$ is related to the noise level in the data.
		
		In particular, in this paper we employed SPGL1 and Fourier basis.
		
		It should be added that the subsampling is performed only in the spatial domain, because in practice during SLDV measurements the time domain sampling frequency is fixed and cannot be altered in the acquisition software. 	
	\end{frame}
	
	\begin{frame}{Compression rate}
		\begin{columns}[T]
			\begin{column}[t]{0.48\textwidth}
				\tiny
				The maximum permissible distance between grid points according to Nyquist theorem is calculated as in Eqn.~(\ref{eq:dx}):
				\begin{equation}
					d_{max}= \frac{1}{2*k_{max}} = \frac{1}{2*51.28\ [\textup{m}]} = \frac{\lambda}{2} = \frac{19.5}{2}\ \textup{[mm]}.
					\label{eq:dx}	
				\end{equation} 
				where $k_{max}$ is the maximum wavenumber, and $\lambda$ is the shortest wavelength.
				
				On the other hand, the longest distance between grid points on uniform square grid in 2D space is along the diagonal as shown in Figure below.
				Therefore, the number of Nyquist sampling points along edges of the plate is defined as:
				\begin{align}
					\begin{split}
						N_x= \frac{L}{d_{max}/\sqrt{2}}, \\
						N_y=  \frac{W}{d_{max}/\sqrt{2}},
					\end{split}
					\label{eq:Nyq}
				\end{align}
				where $L$ is the plate length, and $W$ is the plate width.
				
				In our particular case, $L=W=500$~[mm], and number of Nyquist points $N_x= N_y= N_{Nyq} =73$.				
			\end{column}
			\begin{column}[t]{0.48\textwidth}
				\tiny
				\begin{figure} [!h]
					\centering
					\includegraphics[width=0.75\textwidth]{Nyquist_wavelength.png}
					\caption{Longest distance between grid points.}
				\end{figure}
				
				In this work, we have generated a low-resolution training set with a frame size \((32\times32)\) pixels, which is below the Nyquist sampling rate of a 2D frame.
				Hence, we have performed image subsampling with bi-cubic interpolation and a uniform mesh of size \((32\times32)\) pixels with a compression rate (CR) of \(19.2\%\) from the Nyquist sampling rate as depicted in Eqn.~\ref{CR}:
				\begin{equation}
					CR = \frac{(Low-resolution\ dimension)^2}{(Nyquist\ sampling\ rate)^2} = \frac{(32\times32)}{(73\times73)}=19.2\%
					\label{CR}
				\end{equation}			
			\end{column}
		\end{columns}	
	\end{frame}
	%%%%%%%%%%%%%%%%%%%%%%%%%%%%%%%%%%%%%%%%%%%%%%%%
		 END OF SLIDES
	%%%%%%%%%%%%%%%%%%%%%%%%%%%%%%%%%%%%%%%%%%%%%%%%
\end{document}