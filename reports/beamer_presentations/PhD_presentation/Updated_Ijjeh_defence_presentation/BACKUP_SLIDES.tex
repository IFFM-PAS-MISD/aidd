%\PassOptionsToPackage{draft}{graphicx}
\documentclass[10pt,aspectratio=169,dvipsnames]{beamer} % aspect ratio 16:9
%\graphicspath{{../../figures/}}

%\includeonlyframes{frame1,frame2,frame3}

%%%%%%%%%%%%%%%%%%%%%%%%%%%%%%%%%%%%%%%%%%%%%%%%%%
% Packages
%%%%%%%%%%%%%%%%%%%%%%%%%%%%%%%%%%%%%%%%%%%%%%%%%%
\usepackage{appendixnumberbeamer}
\usepackage{booktabs}
\usepackage{csvsimple} % for csv read
\usepackage[scale=2]{ccicons}
\usepackage{pgfplots}
\usepackage{xspace}
%\usepackage{amsmath}
\usepackage{totcount}
\usepackage{tikz}
\usepackage{bm}
\usepackage{float}
\usepackage{eso-pic} 
\usepackage{wrapfig}
\usepackage{animate,media9}
\usepackage{subfig}
\usepackage{fancybox}
%\usepackage{multimedia}
\usepackage{dashbox}
\usepackage{tcolorbox}
\usepackage{multicol}
\usepackage{multirow}
\usepackage{xcolor}
\usepackage[document]{ragged2e}
\usepackage{caption}
\usepackage{comment}
\usepackage{mathtools}% Loads amsmath

%\usepackage[export]{adjustbox}
%\usepackage{background}
%\backgroundsetup{contents=preliminary,placement=bottom,color=blue}
%\usepackage{FiraSans}

%\usepackage{comment}
%\usetikzlibrary{external} % speedup compilation
%\tikzexternalize % activate!
%\usetikzlibrary{shapes,arrows} 

%\usepackage{bibentry}
%\nobibliography*
\usepackage{ifthen}
\newcounter{angle}
\setcounter{angle}{0}
%\usepackage{bibentry}
%\nobibliography*
\usepackage{caption}%

\graphicspath{{figures/}}

\captionsetup[figure]{labelformat=empty}%
\usefonttheme{structurebold}
%%%%%%%%%%%%%%%%%%%%%%%%%%%%%%%%%%%%%%%%%%%%%%%%%%
% Metropolis theme custom modification file
%%%%%%%%%%%%%%%%%%%%%%%%%%%%%%%%%%%%%%%%%%%%%%%%%%
% Metropolis theme custom modification file
%%%%%%%%%%%%%%%%%%%%%%%%%%%%%%%%%%%%%%%%%%%%%%%%%%
% Metropolis theme custom colors
%%%%%%%%%%%%%%%%%%%%%%%%%%%%%%%%%%%%%%%%%%%%%%%%%%
\usetheme[progressbar=foot]{metropolis}
\useoutertheme{metropolis}
\useinnertheme{metropolis}
\usefonttheme{metropolis}
\setbeamercolor{background canvas}{bg=white}

%\usecolortheme{spruce}

\definecolor{myblue}{rgb}{0.19,0.55,0.91}
\definecolor{mediumblue}{rgb}{0,0,205}
\definecolor{darkblue}{rgb}{0,0,139}
\definecolor{Dodgerblue}{HTML}{1E90FF}
\definecolor{Navy}{HTML}{000080} % {rgb}{0,0,128}
\definecolor{Aliceblue}{HTML}{F0F8FF}
\definecolor{Lightskyblue}{HTML}{87CEFA}
\definecolor{logoblue}{RGB}{1,67,140}
\definecolor{Purple}{HTML}{911146}
\definecolor{Orange}{HTML}{CF4A30}

\setbeamercolor{progress bar}{bg=Lightskyblue}
\setbeamercolor{progress bar}{ fg=logoblue} 
\setbeamercolor{frametitle}{bg=logoblue}
\setbeamercolor{title separator}{fg=logoblue}
\setbeamercolor{block title}{bg=Lightskyblue!30,fg=black}
\setbeamercolor{block body}{bg=Lightskyblue!15,fg=black}
\setbeamercolor{alerted text}{fg=Purple}
% notes colors
\setbeamercolor{note page}{bg=white}
\setbeamercolor{note title}{bg=Lightskyblue}
%%%%%%%%%%%%%%%%%%%%%%%%%%%%%%%%%%%%%%%%%%%%%%%%%%
%  Theme modifications
%%%%%%%%%%%%%%%%%%%%%%%%%%%%%%%%%%%%%%%%%%%%%%%%%%
% modify progress bar linewidth
\makeatletter
\setlength{\metropolis@progressinheadfoot@linewidth}{2pt} 
\setlength{\metropolis@titleseparator@linewidth}{1pt}
\setlength{\metropolis@progressonsectionpage@linewidth}{1pt}

\setbeamertemplate{progress bar in section page}{
	\setlength{\metropolis@progressonsectionpage}{%
		\textwidth * \ratio{\thesection pt}{\totvalue{totalsection} pt}%
	}%
	\begin{tikzpicture}
		\fill[bg] (0,0) rectangle (\textwidth, 
		\metropolis@progressonsectionpage@linewidth);
		\fill[fg] (0,0) rectangle (\metropolis@progressonsectionpage, 
		\metropolis@progressonsectionpage@linewidth);
	\end{tikzpicture}%
}
\makeatother
\newcounter{totalsection}
\regtotcounter{totalsection}

\AtBeginDocument{%
	\pretocmd{\section}{\refstepcounter{totalsection}}{\typeout{Yes, prepending 
	was successful}}{\typeout{No, prepending was not successful}}%
}%
%%%%%%%%%%%%%%%%%%%%%%%%%%%%%%%%%%%%%%%%%%%%%%%%%%
%  Bibliography mods
%%%%%%%%%%%%%%%%%%%%%%%%%%%%%%%%%%%%%%%%%%%%%%%%%%
\setbeamertemplate{bibliography item}{\insertbiblabel} %% Remove book symbol 
%%from references and add number in square brackets
% kill the abominable icon (without number)
%\setbeamertemplate{bibliography item}{}
%\makeatletter
%\renewcommand\@biblabel[1]{#1.} % number only
%\makeatother
% remove line breaks in bibliography
\setbeamertemplate{bibliography entry title}{}
\setbeamertemplate{bibliography entry location}{}
%%%%%%%%%%%%%%%%%%%%%%%%%%%%%%%%%%%%%%%%%%%%%%%%%%
%  Bibliography custom commands
%%%%%%%%%%%%%%%%%%%%%%%%%%%%%%%%%%%%%%%%%%%%%%%%%%
\newcommand{\bibliotitlestyle}[1]{\textbf{#1}\par}

\newif\ifinbiblio
\newcounter{bibkey}
\newenvironment{biblio}[2][long]{%
	%\setbeamertemplate{bibliography item}{\insertbiblabel}
	\setbeamertemplate{bibliography item}{}% without numbers
	\setbeamerfont{bibliography item}{size=\footnotesize}
	\setbeamerfont{bibliography entry author}{size=\footnotesize}
	\setbeamerfont{bibliography entry title}{size=\footnotesize}
	\setbeamerfont{bibliography entry location}{size=\footnotesize}
	\setbeamerfont{bibliography entry note}{size=\footnotesize}
	\ifx!#2!\else%
	\bibliotitlestyle{#2}%
	\fi%
	\begin{thebibliography}{}%
		\inbibliotrue%
		\setbeamertemplate{bibliography entry title}[#1]%
	}{%
		\inbibliofalse%
		\setbeamertemplate{bibliography item}{}%
	\end{thebibliography}%
}

\newcommand{\biblioref}[5][short]{
	\setbeamertemplate{bibliography entry title}[#1]
	\stepcounter{bibkey}%
	\ifinbiblio%
	\bibitem{\thebibkey}%
	#2
	\newblock #4
	\ifx!#5!\else\newblock {\em #5}, #3 \fi%
	\else%
	\begin{biblio}{}
		\bibitem{\thebibkey}
		#2
		\newblock #4
		\ifx!#5!\else\newblock {\em #5}, #3\fi
	\end{biblio}
	\fi
}
%
%\newbibmacro*{hypercite}{%
%	\renewcommand{\@makefntext}[1]{\noindent\normalfont##1}%
%	\footnotetext{%
%		\blxmkbibnote{foot}{%
%			\printtext[labelnumberwidth]{%
%				\printfield{prefixnumber}%
%				\printfield{labelnumber}}%
%			\addspace
%			\fullcite{\thefield{entrykey}}}}}
%
%\DeclareCiteCommand{\hypercite}%
%{\usebibmacro{cite:init}}
%{\usebibmacro{hypercite}}
%{}
%{\usebibmacro{cite:dump}}
%
%% Redefine the \footfullcite command to use the reference number
%\renewcommand{\footfullcite}[1]{\cite{#1}\hypercite{#1}}
%\usefonttheme[onlymath]{Serif} % It should be uncommented if Fira fonts in 
%%math does not work

%%%%%%%%%%%%%%%%%%%%%%%%%%%%%%%%%%%%%%%%%%%%%%%%%%
% Custom commands
%%%%%%%%%%%%%%%%%%%%%%%%%%%%%%%%%%%%%%%%%%%%%%%%%%
% matrix command 
\newcommand{\matr}[1]{\mathbf{#1}} % bold upright (Elsevier, Springer)
%\newcommand{\matr}[1]{#1}   % pure math version
%\newcommand{\matr}[1]{\bm{#1}}  % ISO complying version
% vector command 
\newcommand{\vect}[1]{\mathbf{#1}} % bold upright (Elsevier, Springer)
% bold symbol
\newcommand{\bs}[1]{\boldsymbol{#1}}
% derivative upright command
\DeclareRobustCommand*{\drv}{\mathop{}\!\mathrm{d}}
\newcommand{\ud}{\mathrm{d}}
% 
\newcommand{\themename}{\textbf{\textsc{metropolis}}\xspace}

%\usepackage{pgfpages}
%\setbeameroption{show notes}
%\setbeameroption{show notes on second screen=left}
%\setbeamertemplate{note page}{\insertnote}
%%%%%%%%%%%%%%%%%%%%%%%%%%%%%%%%%%%%%%%%%%%%%%%%%%
% Title page options
%%%%%%%%%%%%%%%%%%%%%%%%%%%%%%%%%%%%%%%%%%%%%%%%%%
% \date{\today}
\date{}
%%%%%%%%%%%%%%%%%%%%%%%%%%%%%%%%%%%%%%%%%%%%%%%%%%
% option 1
%%%%%%%%%%%%%%%%%%%%%%%%%%%%%%%%%%%%%%%%%%%%%%%%%%%
\title{Backup slides for the defence}
%\subtitle{In preparation for a Ph.D. defence}
\author{\textbf{Ph.D. candidate, Eng. Abdalraheem A. Ijjeh } 
	\and \\ 
	\textbf{Supervisor: D.Sc. Ph.D. Eng. Paweł Kudela}
} 
% logo align to Institute 
\institute{Institute of Fluid Flow Machinery \\ 
	Polish Academy of Sciences \\ 
	\vspace{-1.5cm}
	\flushright 
	\includegraphics[width=6cm]{imp_logo.png}}

%%%%%%%%%%%%%%%%%%%%%%%%%%%%%%%%%%%%%%%%%%%%%%%%%%
%\tikzexternalize % activate!
%%%%%%%%%%%%%%%%%%%%%%%%%%%%%%%%%%%%%%%%%%%%%%%%%%
\setbeamertemplate{section in toc}[sections numbered]
\setbeamertemplate{subsection in toc}[subsections numbered]

\begin{document}
	%%%%%%%%%%%%%%%%%%%%%%%%%%%%%%%%%%%%%%%%%%%%%%%%%%
	\maketitle
	%%%%%%%%%%%%%%%%%%%%%%%%%%%%%%%%%%%%%%%%%%%%%%%%%%%%%%%%%%%%%%%%%%%%%%%%%%%%
	%%%%%%%%%%%%%%%%%%%%%%%%%%%%%%%%%%%%%%%%%%%%%%%%%%%%%%%%%%%%%%%%%%%%%%%%%%%%
	%%%%%%%%%%%%%%%%%%%%%%%%%%%%%%%%%%%%%%%%%%%%%%%%%%
	% SLIDES
	%%%%%%%%%%%%%%%%%%%%%%%%%%%%%%%%%%%%%%%%%%%%%%%%%%	
	\section{Backup slides}
	\addtocounter{section}{1}
	%%%%%%%%%%%%%%%%%%%%%%%%%%%%%%%%%%%%%%%%%%%%%%%%%%%%%%%%%%%%%%%%%%%%%%%%%%%
	\setcounter{subfigure}{0}
	\begin{frame}{Optimization and Deep Learning}
		\begin{columns}[T]
			\begin{column}[t]{.4\textwidth}
				\begin{itemize}
					\item \alert{Labeled data} $\rightarrow$ (input data has labels/ground truths)
					\item \alert{Loss function} $\rightarrow$ measures error between predicted and the ground truth values
					\item \alert{Learnable parameters} $\rightarrow$ updated during the backpropagation step (optimization e.g. Gradient descent )					
				\end{itemize}
			\end{column}
			\begin{column}[t]{.55\textwidth}
				\begin{figure}[t]
					\centering
					\animategraphics[autoplay,loop,width =1.0\textwidth]{1}{figures/gif_figs/BP/png/BP_technique_}{0}{14}
				\end{figure}
			\end{column}
		\end{columns}		
	\end{frame}
	%%%%%%%%%%%%%%%%%%%%%%%%%%%%%%%%%%%%%%%%%%%%%%%%%%%%%%%%%%%%%%%%%%%%%%%%%%%
	\note{			
		In this work, I used the supervised approach, which means that the input data is labelled.
		
		Accordingly, the idea of supervised learning is to learn a model of how to map the inputs to the outputs.
		
		Initially, when we start training the deep learning model, all learnable parameters, such as weights and biases, start with random values.
		These learnable parameters are updated during the training phase.
		Now, in the forward pass, we feed the data into the model, and it flows through the layers until we get the predicted output.
		Here, we need an objective loss function to estimate the loss (error) between the predicted output and the ground truth.
		
		A well-known optimization algorithm is a gradient descent, which aims to reduce the loss at each step to reach the global minimum value by calculating the gradient and then pushing back the calculated gradients across all the neurons in a technique called backpropagation.
		
		Accordingly, all learnable parameters are updated, which leads to minimizing the loss value.
	}
	%%%%%%%%%%%%%%%%%%%%%%%%%%%%%%%%%%%%%%%%%%%%%%%%%%%%%%%%%%%%%%%%%%%%%%%%%%%
	\begin{frame}{Dataset}
		\footnotesize	
		\begin{itemize}
			\item The input signal was a five-cycle Hann window modulated sinusoidal tone burst. 
			\item The carrier frequency was assumed to be 50 kHz. 
			\item The total wave propagation time was set to 0.75 ms so that the guided wave could propagate to the plate edges and back to the actuator twice. 
			\item The number of time integration steps was 150000, which was selected for the stability of the central difference scheme.
			\item The material was a typical cross-ply CFRP laminate. 
			\item The stacking sequence \([0/90]_4\) was used in the model. 
			\item The properties of a single ply were as follows [GPa] (elasticity matrix): \(C_{11}=52.55,\ C_{12}=6.51,\ C_{22}=51.83,\ C_{44}=2.93,\ C_{55}=2.92,\ C_{66}=3.81\). 
			\item The assumed mass density was 1522.4 kg/\(m^3\). 
			\item These properties were selected so that wave front patterns and wavelengths simulated numerically are similar to the wavefields measured by the SLDV on CFRP specimens used later on for testing the developed methods for delamination identification. 
			\item The shortest wavelength of the propagating A0 Lamb wave mode was 21.2 mm for numerical simulations and 19.5 mm for experimental measurements, respectively.
		\end{itemize}		
	\end{frame}
	%%%%%%%%%%%%%%%%%%%%%%%%%%%%%%%%%%%%%%%%%%%%%%%%%%%%%%%%%%%%%%%%%%%%%%%%%%%
	\begin{frame}{Rule of mixture and homogenization}	
		\footnotesize		
		\noindent Composite materials have their micro-structure designed in terms of their macroscopic constituents, e.g., fibers in a homogeneous matrix material. 
		By controlling the choice of fibres, their volume fraction, and alignment, the mechanical properties may be tailored to meet specific design requirements.
		\begin{columns}[T]				
			\begin{column}{0.45\textwidth}
				Diagram (a) shows a 'uniaxial fibre-reinforced composite material," and (b) shows how the stress on the composite is carried by the fibres and the matrix. 
				In normal situations, the fibre has a larger Young's modulus than the matrix, and for the continuous fibres shown, where the strain is the same in the matrix and the fibre, the fibre stress is higher than the matrix stress.
			\end{column}
			\begin{column}{0.45\textwidth}
				\begin{figure}
					\includegraphics[width=0.8\textwidth]{rule_of_mixture.png}
					\caption{From: McMahon and Graham, :"The Bicycle and the Walkman," Merion (1992)}
				\end{figure}
			\end{column}
		\end{columns}
		The Young's modulus of the composite is given by the 'rule of mixtures' i.e. \(E_C = E_F V_F + E_MV_M\), also \((V_M + V_F) = 1\) or \(V_M = (1 - V_F )\). 
		The elastic modulus along the fibre direction can be controlled by selecting the volume fraction of the fibres.		
	\end{frame}
	%%%%%%%%%%%%%%%%%%%%%%%%%%%%%%%%%%%%%%%%%%%%%%%%%%%%%%%%%%%%%%%%%%%%%%%%%%%
	%%%%%%%%%%%%%%%%%%%%%%%%%%%%%%%%%%%%%%%%%%%%%%%%%%%%%%%%%%%%%%%%%%%%%%%%%%%
	\begin{frame}{RMS based: Analysis of numerical cases}
		\begin{columns}[T]
			\tiny
			\begin{column}[t]{0.48\textwidth}
				%%%%%%%%%%%%%%%%%%%%%%%%%%%%%%%%%%%%%%%%%%%%%%%%%%%%%%%%%%%
				\begin{table}
					\caption{Evaluation metrics of the three numerical cases.}
					\label{tab:RMS_num_cases}
					\begin{tabular}{cccccc}
						\toprule[1.5pt]
						\multirow{2}{*}{Model} & \multirow{2}{*}{case number} & \multicolumn{1}{c}{\multirow{2}{*}{A [mm\textsuperscript{2}]}} & \multicolumn{3}{c}{Predicted output} \\ 
						\cmidrule(lr){4-6} & & & \multicolumn{1}{c}{IoU} & \multicolumn{1}{c}{\(\hat{A}\) [mm\textsuperscript{2}]} & \(\epsilon\) \\
						\midrule
						\multirow{3}{*}{Res-UNet} 							
						& 1 & 257 & \multicolumn{1}{c}{0.45} & \multicolumn{1}{c}{143} & \(44.36\%\) \\ 
						& 2 & 105 & \multicolumn{1}{c}{0.67} & \multicolumn{1}{c}{88} & \(16.19\%\) \\ 
						& 3 & 537 & \multicolumn{1}{c}{0.80} & \multicolumn{1}{c}{478} & \(10.99\%\) \\ 
						\midrule
						\multirow{3}{*}{VGG16 encoder-decoder} 
						& 1 & 257 & \multicolumn{1}{c}{0.69} & \multicolumn{1}{c}{203} & \(21.01\%\) \\ 
						& 2 & 105 & \multicolumn{1}{c}{0.75} & \multicolumn{1}{c}{117} & \(11.43\%\) \\ 
						& 3 & 537 & \multicolumn{1}{c}{0.65} & \multicolumn{1}{c}{385} & \(28.31\%\) \\ 
						\midrule
						\multirow{3}{*}{FCN-DenseNet} 
						& 1 & 257 & \multicolumn{1}{c}{0.52} & \multicolumn{1}{c}{505} & \(96.50\%\) \\ 
						& 2 & 105 & \multicolumn{1}{c}{0.66} & \multicolumn{1}{c}{118} & \(12.38\%\) \\ 
						& 3 & 537 & \multicolumn{1}{c}{0.72} & \multicolumn{1}{c}{815} & \(51.77\%\) \\ 
						\midrule
						\multirow{3}{*}{PSPNet} 
						& 1 & 257 & \multicolumn{1}{c}{0.00} & \multicolumn{1}{c}{0} & \(-\%\) \\ 
						& 2 & 105 & \multicolumn{1}{c}{0.44} & \multicolumn{1}{c}{156} & \(48.57\%\) \\ 
						& 3 & 537 & \multicolumn{1}{c}{0.77} & \multicolumn{1}{c}{610} & \(13.59\%\) \\ 
						\midrule
						\multirow{3}{*}{GCN} 
						& 1 & 257 & \multicolumn{1}{c}{0.71} & \multicolumn{1}{c}{215} & \(16.34\%\) \\ 
						& 2 & 105 & \multicolumn{1}{c}{0.72} & \multicolumn{1}{c}{177} & \(68.57\%\) \\ 
						& 3 & 537 & \multicolumn{1}{c}{0.86} & \multicolumn{1}{c}{523} & \(2.61\%\) \\ 
						\bottomrule[1.5pt]
					\end{tabular}	
				\end{table}
				%%%%%%%%%%%%%%%%%%%%%%%%%%%%%%%%%%%%%%%%%%%%%%%%%%%%%%%%%	
			\end{column}
			\begin{column}[t]{0.48\textwidth}
				\begin{table}
					\caption{Analysis of numerical cases.}
					\label{tab:table_all_numerical_cases_backup}	
					\begin{tabular}{lcc}
						\toprule[1.5pt]
						Model & mean IoU & max IoU \\ 
						\midrule 
						Res-UNet & \(0.66\) & \(0.89\) \\ 
						VGG16 encoder-decoder & \(0.57\) & \(0.84\) \\ 
						FCN-DenseNet & \(0.68\) & \(0.92\) \\ 
						PSPNet & \(0.55\) & \(0.91\) \\ 
						GCN & \(0.76\) & \(0.93\) \\ 
						\bottomrule[1.5pt]
					\end{tabular}
				\end{table}
			\end{column}
		\end{columns}
	\end{frame}
	%%%%%%%%%%%%%%%%%%%%%%%%%%%%%%%%%%%%%%%%%%%%%%%%%%%%%%%%%%%%%%%%%%%%%%%%%%%
	\begin{frame}{Initial frame interaction}
		\begin{itemize}
			\item The total wave propagation time was set to \(0.75\) ms so that the guided wave could propagate to the plate edges and back to the actuator twice.
			\item The total time of propagation was converted into \(512\) frames of animated Lamb waves.
			\item The calculated group velocity of \(A_0\) mode is about \(1100\ m/s\).
			\item The \((x, y)\) coordinates of the center of the delaminations are known for the numerically generated dataset. 
			Therefore, we can calculate the distance between the center of the plate and the center of the delamination.
			\item As the group velocity of \(A_0\) mode is known, and the distance is known, we can calculate the required time for the propagating wave to reach the center of the delamination. 
		\end{itemize}
		
		When the time of interaction \(t_i\) with the delamination is know, we can approximately convert it to the frame number \(f_n\) as depicted in the equations given below:		
		\begin{gather*}
			t_i = \frac{\sqrt{(x-0.25)^2 +(y-0.25)^2} \ m}{1100\ m/s}
			\\
			f_n = \frac{t_i}{0.75ms} \times 512
		\end{gather*}			
	\end{frame}
	%%%%%%%%%%%%%%%%%%%%%%%%%%%%%%%%%%%%%%%%%%%%%%%%%%%%%%%%%%%%%%%%%%%%%%%%%%%
	\begin{frame}{Animation based: Analysis of numerical cases}
		The mean IoU for all 95 test cases is 0.80, and the \(mean\ \epsilon\) is \(5.6\%\)
		\begin{table}[!h]
			\centering
			\caption{Evaluation metrics of the three numerical cases.}
			\begin{tabular}{ccccc}
				\toprule[1.5pt]
				\multirow{2}{*}{case number} & \multicolumn{1}{c}{\multirow{2}{*}{A [mm\textsuperscript{2}]}} & \multicolumn{3}{c}{Predicted output} \\ 
				\cmidrule(lr){3-5} & & \multicolumn{1}{c}{IoU} & \multicolumn{1}{c}{\(\hat{A}\) [mm\textsuperscript{2}]} & \(\epsilon\) \\
				\midrule
				1 & 763 & \multicolumn{1}{c}{0.88} & \multicolumn{1}{c}{735} & \(3.67\%\) \\ 
				2 & 388 & \multicolumn{1}{c}{0.58} & \multicolumn{1}{c}{248} & \(36.08\%\) \\ 
				3 & 297 & \multicolumn{1}{c}{0.80} & \multicolumn{1}{c}{280} & \(5.72\%\) \\			 					
				\bottomrule[1.5pt]
			\end{tabular}	
			\label{tab:num_cases_}
		\end{table}			
	\end{frame}
	%%%%%%%%%%%%%%%%%%%%%%%%%%%%%%%%%%%%%%%%%%%%%%%%%%%%%%%%%%%%%%%%%%%%%%%%%%
	\begin{frame}{SLDV measurements: setup}
		\begin{figure}
			\includegraphics[width=0.7\textwidth]{sensors_fig4_setup.png}
		\end{figure}
	\end{frame}
	%%%%%%%%%%%%%%%%%%%%%%%%%%%%%%%%%%%%%%%%%%%%%%%%%%%%%%%%%%%%%%%%%%%%%%%%%%%
	\setcounter{subfigure}{0}
	%%%%%%%%%%%%%%%%%%%%%%%%%%%%%%%%%%%%%%%%%%%%%%%%%%%%%%%%%%%%%%%%%%%%%%%%%%%
	\begin{frame}[t]{Conventional signal processing and wavefield imaging}
		\begin{figure}
			\includegraphics[width=0.7\textwidth]{figs2/sensors_fig1_algorithm.png}
		\end{figure}
		\biblioref{M. Radzienski, P. Kudela, A. Marzani, L. de Marchi, W. Ostachowicz}{2019}{ Damage Identification in Various Types of Composite Plates Using Guided Waves Excited by a Piezoelectric Transducer and Measured by a Laser Vibrometer}{Sensors, 19, 1958}
	\end{frame}
	%%%%%%%%%%%%%%%%%%%%%%%%%%%%%%%%%%%%%%%%%%%%%%%%%%%%%%%%%%%%%%%%%%%%%%%%%%%
	\setcounter{subfigure}{0}
	\begin{frame}{RMS based: Analysis of experimental case}
				\begin{table}[!ht]
					\centering
					\caption{Evaluation metrics of the experimental case.}
					\label{tab:rms_exp_case__}
					\begin{tabular}{lc}
						\toprule[1.5pt]
						Model & IoU  	\\			
						\midrule
						Res-UNet & 0.58 \\ 
						VGG16 encoder-decoder & 0.62 \\ 
						FCN-DenseNet & 0.54 \\ 
						PSPNet & 0.49 \\ 
						GCN & 0.72\\ 
						\bottomrule[1.5pt]
					\end{tabular}		
				\end{table}
		\begin{table}[!ht]
			\centering
			\caption{Evaluation metrics of the experimental case.}
			\label{tab:rms_exp_case_}
			\begin{tabular}{l@{\ }cccc}
				\toprule
				\multicolumn{1}{l}{Model} & \multicolumn{1}{c}{A [mm\textsuperscript{2}]} & \multicolumn{3}{c}{Predicted output} \\ 
				\cmidrule(lr){3-5} & & \multicolumn{1}{c}{IoU} & \multicolumn{1}{c}{\(\hat{A}\) [mm\textsuperscript{2}]} & \(\epsilon\) \\ \midrule
				Res-UNet & \multicolumn{1}{c}{\multirow{5}{*}{210}} & \multicolumn{1}{c}{0.58} & \multicolumn{1}{c}{323}  & \(53.8\%\) \\ 
				VGG16 encoder-decoder &  & \multicolumn{1}{c}{0.62} & \multicolumn{1}{c}{320} & \(52.4\%\) 
				\\ 
				FCN-DenseNet &  & \multicolumn{1}{c}{0.54} & \multicolumn{1}{c}{386} & \(83.8\%\) \\ 
				PSPNet &  & \multicolumn{1}{c}{0.49} & \multicolumn{1}{c}{580} & \(176.2\%\) 
				\\ 
				GCN &  & \multicolumn{1}{c}{0.72} & \multicolumn{1}{c}{309} & \(47.1\%\) 
				\\ 
				\bottomrule
			\end{tabular}		
		\end{table}
	\end{frame}
	%%%%%%%%%%%%%%%%%%%%%%%%%%%%%%%%%%%%%%%%%%%%%%%%%%%%%%%%%%%%%%%%%%%%%%%%%%%
	\setcounter{subfigure}{0}
	\begin{frame}{Experimental results full wavefield based (Single delamination)}
		\begin{figure}[ht!]
			\centering
%			\subfloat[Full wavefield (512 frames)]{\animategraphics[autoplay,loop,height=3cm]{32}{figures/gif_figs/CFRP_teflon_3o_375_375p_50kHz_5HC_x12_15Vpp/CFRP_teflon_30-}{1}{256}}\quad
%			\subfloat[Intermidate ouputs]{\animategraphics[autoplay,loop,height=3cm]{24}{figures/gif_figs/CFRP_ijjeh_single_delamination/intermediate_output-}{0}{231}}\quad
			\subfloat[RMS]{\includegraphics[height=3cm,keepaspectratio]{figures/RMS_CFRP_teflon_3o_375_375p_50kHz_5HC_x12_15Vpp_Ijjeh_updated_results_.png}}\quad
			\subfloat[Binary RMS]{\includegraphics[height=3cm,keepaspectratio]{figures/Binary_RMS_CFRP_teflon_3o__375_375p_50kHz_5HC_x12_15Vpp_Ijjeh_.png}}
		\end{figure}
		IoU= $0.41$ for the thresholded damage map and $\epsilon=71.56\%$  
	\end{frame}
	%%%%%%%%%%%%%%%%%%%%%%%%%%%%%%%%%%%%%%%%%%%%%%%%%%%%%%%%%%%%%%%%%%%%%%%%%%%
		%%%%%%%%%%%%%%%%%%%%%%%%%%%%%%%%%%%%%%%%%%%%%%%%%%%%%%%%%%%%%%%%%%%%%%%%%%%%
	\setcounter{subfigure}{0}
	\begin{frame}{Compressive sensing theory}
		\tiny
		\begin{columns}[T]
			\begin{column}[t]{0.48\textwidth}
				\justifying
				Compressed sensing (CS) theory $\rightarrow$ any natural signal (\(x\)), e.g. (sounds, images, $\dots$) \alert{can be recovered using considerably fewer measurements than standard methods}.
				\\
				CS relies on two principles:
				\begin{itemize}
					\item \alert{Sparsity}: which relates to the signal of interest.
					\begin{equation*}
						x=\Psi s,
					\end{equation*}
					where \(\Psi\) is the universal basis (in this work, Fourier domain was applied), \(s\) is a sparse vector of coefficients (\alert{most of the coefficients are equal or close to zero}).
					\item \alert{Incoherence}: which relates to the sensing modality.
					\begin{gather*}
						y=Cx, \\
						y=C\Psi s,
					\end{gather*}
					where \(y\) is the measurements in \alert{Low-Resolution (LR)} (below the Nyquist sampling rate),
					\(C\) is the mask matrix applied to the \(x\).\\
					This system of equations is \alert{underdetermined}.
					\\
					$C$ and \(\Psi\) matrices must be incoherent \alert{(smallest correlation between any two elements in $C$ and $\Psi$)}
				\end{itemize}		
			\end{column}			
			\begin{column}[t]{0.48\textwidth}
				\justifying
				\alert{The goal is to find sparsest \(s\) vector} that solve the underdetermined system of equation.
				\\ This is an optimization problem:				
				\begin{equation}
					\min{\lVert {\bs{s}} \rVert}_1 \quad \textrm{subject to} \quad {\lVert \bs{C} \bs{\Psi} \bs{s} -y \rVert}_2 \leq \sigma ,
				\end{equation}
				where $\sigma$ is related to the noise level in the data.
				\begin{figure}[ht!]
					\subfloat[\centering \small Random mask (3000 points)]{\includegraphics[width=0.40\textwidth]{random_mask_3000.png}}\quad
					\subfloat[\centering \small Jitter mask (3000 points)]{\includegraphics[width=0.40\textwidth]{jitter_mask_3000.png}}					
				\end{figure}										
			\end{column}		
		\end{columns}						
	\end{frame}
	\note
	{	
		\tiny
		The theory of compressed sensing (CS) states that natural signals (such as sounds and images) can be recovered using considerably fewer samples or measurements (below the Nyquist sampling rate) than standard methods.
		
		Now, CS relies on two essential conditions: 	
		\begin{itemize}
			\item The first one is Sparsity: which means that(\alert{most of the coefficients in vector \(s\) are equal or close to zero}).
			$\Psi$ represent a universal basis such as Fourier, cosine or wavelet domains.
			In this work the Fourier domain was applied. 
			\item The second one is that the measuring mask C matrix must be incoherent with \(\Psi\) matrix, which means the correlation between any two elements in C and $\Psi$ matrices is small.
		\end{itemize}		
		
		regarding to this equation: 
		\begin{equation}
			y = \bs{C}\bs{\Psi}\bs{s}
		\end{equation}
		There are an infinite number of solutions for the sparse vector (s) that can solve the y vector.
		
		However, we need to find the sparsest s vector.
		
		Therefore, it is an optimization problem that can be solved by minimizing the \(L_1\) norm of s such that \(L_2\) norm \(\bs{C}\bs{\Psi}\bs{s}\)-\(\bs{y}\) equals zero, or some noise $\sigma$.
		
		Now that we have this sparsest vector (s), we can use the inverse Fourier transform to recover the high-resolution signal.		
		
	}
	\begin{frame}{Compression rate}
		\begin{columns}[T]
			\begin{column}[t]{0.48\textwidth}
				\tiny
				The maximum permissible distance between grid points according to Nyquist theorem is calculated as in Eqn.~(\ref{eq:dx}):
				\begin{equation}
					d_{max}= \frac{1}{2*k_{max}} = \frac{1}{2*51.28\ [\textup{m}]} = \frac{\lambda}{2} = \frac{19.5}{2}\ \textup{[mm]}.
					\label{eq:dx}	
				\end{equation} 
				where $k_{max}$ is the maximum wavenumber, and $\lambda$ is the shortest wavelength.
				
				On the other hand, the longest distance between grid points on uniform square grid in 2D space is along the diagonal as shown in Figure below.
				Therefore, the number of Nyquist sampling points along edges of the plate is defined as:
				\begin{align}
					\begin{split}
						N_x= \frac{L}{d_{max}/\sqrt{2}}, \\
						N_y=  \frac{W}{d_{max}/\sqrt{2}},
					\end{split}
					\label{eq:Nyq}
				\end{align}
				where $L$ is the plate length, and $W$ is the plate width.
				
				In our particular case, $L=W=500$~[mm], and number of Nyquist points $N_x= N_y= N_{Nyq} =73$.				
			\end{column}
			\begin{column}[t]{0.48\textwidth}
				\tiny
				\begin{figure} [!h]
					\centering
					\includegraphics[width=0.75\textwidth]{Nyquist_wavelength.png}
					\caption{Longest distance between grid points.}
				\end{figure}
				
				In this work, we have generated a low-resolution training set with a frame size \((32\times32)\) pixels, which is below the Nyquist sampling rate of a 2D frame.
				Hence, we have performed image subsampling with bi-cubic interpolation and a uniform mesh of size \((32\times32)\) pixels with a compression rate (CR) of \(19.2\%\) from the Nyquist sampling rate as depicted in Eqn.~\ref{CR}:
				\begin{equation}
					CR = \frac{(Low-resolution\ dimension)^2}{(Nyquist\ sampling\ rate)^2} = \frac{(32\times32)}{(73\times73)}=19.2\%
					\label{CR}
				\end{equation}			
			\end{column}
		\end{columns}	
	\end{frame}
	%%%%%%%%%%%%%%%%%%%%%%%%%%%%%%%%%%%%%%%%%%%%%%%%
%		 END OF SLIDES
	%%%%%%%%%%%%%%%%%%%%%%%%%%%%%%%%%%%%%%%%%%%%%%%%
\end{document}