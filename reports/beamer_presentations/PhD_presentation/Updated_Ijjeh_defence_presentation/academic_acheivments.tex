\documentclass[11pt, a4paper]{article}
\usepackage[a4paper, total={6.5in, 8.5in}]{geometry}
\usepackage{mathptmx}
\usepackage{newtxmath}
\title{ACADEMIC ACHIEVEMENTS}
\author{Abdalraheem Ijjeh}

\begin{document}
	\maketitle
	\begin{enumerate}
		\item \textbf{Journal papers:}
		\begin{enumerate}
			%%%%%%%%%%%%%%%%%%%%%%%%%%%%%%%%%%%%%%%%%%%%%%%%
			\item Ijjeh, A., Ullah, S., Radzienski, M. and Kudela, P., 2023. Deep learning super-resolution for the reconstruction of full wavefield of Lamb waves. \textbf{\textit{Mechanical Systems and Signal Processing}}, 186, p.109878.						
			\textbf{[200~points]/[IF:8.934]}
			%%%%%%%%%%%%%%%%%%%%%%%%%%%%%%%%%%%%%%%%%%%%%%%%
			\item Ullah, S., Ijjeh, A.A. and Kudela, P., 2023. Deep learning approach for delamination identification using animation of Lamb waves. 						
			\textbf{\textit{Engineering Applications of Artificial Intelligence}}, 117, p.105520.		
			\textbf{[140~points]/[IF:7.802]}
			%%%%%%%%%%%%%%%%%%%%%%%%%%%%%%%%%%%%%%%%%%%%%%%%
			\item Ijjeh, A.A. and Kudela, P., 2022. Deep learning based segmentation using full wavefield processing for delamination identification: A comparative study. \textbf{\textit{Mechanical Systems and Signal Processing}}, 168, p.108671. \textbf{[200~points]/[IF:8.934]}
			%%%%%%%%%%%%%%%%%%%%%%%%%%%%%%%%%%%%%%%%%%%%%%%%
			\item Ijjeh, A.A., Ullah, S. and Kudela, P., 2021. Full wavefield processing by using FCN for delamination detection. \textbf{\textit{Mechanical Systems and Signal Processing}}, 153, p.107537.		
			\textbf{[200~points]/[IF:8.934]}	
			%%%%%%%%%%%%%%%%%%%%%%%%%%%%%%%%%%%%%%%%%%%%%%%%
		\end{enumerate}		
		\item \textbf{Conference papers:}
		\begin{enumerate}
			\item {Ijjeh, A.}, Kudela, P. Convolutional LSTM for delamination imaging in composite laminates. 
			The 4th International Conference on Machine Learning and Intelligent Systems (MLIS 2022), November \(8^{th}\) - \(11^{th}\), 2022, Seoul, Republic of Korea.
			\item Ijjeh, A. and Kudela, P., 2022, June. Delamination Identification Using Global Convolution Networks. 
			In European Workshop on Structural Health Monitoring: EWSHM 2022-Volume 3 (pp. 521-529). Cham: Springer International Publishing.		
			\item {Ijjeh, A.}, Kudela, P. Feasibility Study of Full Wavefield Processing by Using CNN for Delamination Detection. 
			Proceedings of the International Conference on Structural Health Monitoring of Intelligent
			Infrastructure, June \(30^{th}\) - July \(2^{nd}\), 2021, Porto, Portugal, ISSN 2564-3738, pages 709-713.
		\end{enumerate}	
		\item \textbf{Chapters:}
		\begin{enumerate}
			\item {Abdalraheem Ijjeh}, Deep Learning based Damage Imaging techniques, chapter in: Wybrane zagadnienia
			inżynierii mechanicznej, Praca zbiorowa pod redakcja M.~Mieloszyk, T. Ochrymiuka, Wydawnictwo Instytutu
			Maszyn Przepływowych PAN, Gdańsk, 2022, ISBN: 978-83-66928-09-1.
			\item {Abdalraheem Ijjeh}, Data-driven based approach for damage detection, chapter in: Wybrane zagadnienia
			inżynierii mechanicznej, Praca zbiorowa pod redakcja M.~Mieloszyk, T. Ochrymiuka, Wydawnictwo Instytutu
			Maszyn Przepływowych PAN, Gdańsk, 2021, ISBN: 978-83-66928-00-8.				
			\item {Abdalraheem Ijjeh}, Machine Learning for SHM: Literature Review, chapter in: Wybrane zagadnienia
			inżynierii mechanicznej, Praca zbiorowa pod redakcja M.~Mieloszyk, T. Ochrymiuka, Wydawnictwo Instytutu
			Maszyn Przepływowych PAN, Gdańsk, 2020, ISBN: 978-83-88237-97-3.
		\end{enumerate}
		\item \textbf{Internships:}
		\begin{itemize}
			\item Erasmus+, Wave propagation in phononic crystals, 05/06/2022 – 10/06/2022, ISEN-JUNIA, CNRSIEMN,
			Lille, France, under supervision of Dr. Marco Miniaci.
		\end{itemize}			
		\item \textbf{Involvement of research projects carried out independently by PAS research institute}:
		\begin{itemize}
			\item NCN OPUS 16, Feasibility studies of artificial intelligence-driven diagnostics, December 2019 – now, leading researcher.
			\item NCN ALPHORN, Fusion of Models and Data for Enriched Evaluation of Structural Health, Institute of Fluid Flow Machinery PAS in collaboration with ETH Zürich, June 2021 – now, researcher.
			\item EU-funded BOHEME project, May 2022 - now, researcher.
		\end{itemize} 
		\item \textbf{Prizes awarded in international competitions:}
		\begin{itemize}
			\item Third-place winner, 14th July, 2021, Gdańsk, Image Classification Challenge – International Summer School on Deep Learning, Gdańsk University of Technology. \\
			the prize type (individual/team): Individual \\
			the number of participants in the competition: about 200 \\
			the number of countries represented in the competition: more than 40 countries.  \\ \\
			\textbf{Organizers} \\
			General Chair, Jacek Ruminski, prof. GUT, jacek.ruminski@pg.edu.pl \\
			Training Program Chair, Alicja Kwasniewska, SiMa.ai, alicja.kwasniewska@sima.ai \\
			Publicity Chair, Maciej Szankin, Intel, maciej.szankin@intel.com
		\end{itemize}
	\end{enumerate}		
\end{document}